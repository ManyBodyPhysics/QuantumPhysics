
\documentstyle[transpar,psfig,layout3]{article}
%
\setlength{\leftmargin}{4cm}
\newcommand{\bra}[1]{\left\langle #1 \right|}
\newcommand{\ket}[1]{\left| #1 \right\rangle}
\newcommand{\braket}[2]{\left\langle #1 \right| #2 \right\rangle}
\newcommand{\OP}[1]{{\bf\widehat{#1}}}
\newcommand{\matr}[1]{{\bf \cal{#1}}}
\pagestyle{empty}
\begin{document}
\tran{\vspace*{-1.5cm} \small \bf "Kvantedatamaskinen, den neste teknologiske
revolusjonen?"}

\Large
\begin{center}
   \begin{minipage}{0.8\textwidth}
       \vspace*{0.5cm}
       \begin{center}
            { \LARGE \bf Innhold}
       \end{center}
%       \begin{center}
\section*{Hvorfor tenke nytt?}
Jo lengre en kan se bakover i tida desto lengre kan en se 
framover i tida (fritt etter W.~Churcill), eller om 
datamaskiner fra sin spede begynnelse til 2015....eller enda lengre
\section*{Fra klassiske bits til {\bf Qu}(antum)bits}
Kvantefysikk  gir oss en 'naturlig' m\aa te \aa\ representere bits p\aa\
vha.~tilstander i mikrosystemer.  
\section*{Kvanteinformasjonsteori og kvantedatamaskiner}
Hva kan en kvantedatamaskin gj\o re som ikke en klassisk datamaskin
kan?

\section*{Vil vi noensinne se en slik datamaskin?}
Litt om framtidsperspektiver. 
%\end{center}
\vspace{2cm}
\begin{center}
{\centering\mbox{\psfig {figure=dilbert.ps,height=5cm,width=14cm}
}}
\end{center}      

\end{minipage}
\end{center}

\tran{\vspace*{-1.5cm} \small \bf "Kvantedatamaskinen, den neste teknologiske
revolusjonen?"}

\Large
\begin{center}
   \begin{minipage}{0.8\textwidth}
       \vspace*{0.5cm}
       \begin{center}
            { \LARGE \bf Hvorfor er dette spennende?}
       \end{center}
        Kvantefysikken inneholder en del postulater med konsekvenser
        for v\aa r forst\aa else av naturen som er h\o yst ikke-trivielle.
        Slik kvantemekanikken framst\aa r idag, utgj\o r den v\aa r beste
        forst\aa else av naturen. Schr\"odingers likning, hvis tilh\o rende l\o sning forteller om egenskaper til et mikrosystem,  er v\aa r naturlov. 
        
       \begin{center}
           \begin{itemize}
               \item Ny teknologi
               \item Nye fagfelt samt overlapp med flere eksisterende
                     felt, fysikk, matematikk, informatikk, kjemi m.m.
               \item Eksisterende teknologi gj\o r at vi kan studere
                     og kanskje utnytte sider av kvantemekanikken
                     som har v\ae rt (og er) betrakta som mindre
                     trivielle. Eksempler er Schr\"odingers katt 
                     paradokset og 'Entanglement' (mere om dette seinere).  
               \item Kanskje vi utvikler ogs\aa\ en bedre forst\aa else av
                     naturen, {\bf en ny og bedre teori?}
               \item Parallell til begynnelsen av forrige \aa rhundre:
                     mange eksperiment (fotoelektrisk effekt, svart legeme str\aa ling m.m.) kunne ikke forklares vha.~klassisk fysikk. Leda til utviklingen
av kvantemekanikken rundt 1925.  Med dagens teknologi kan vi f.eks.~fange inn enkeltatomer og elektroner i sm\aa\ omr\aa der (noen f\aa\ nanometre) 
og studere tilh\o rende kvantemekaniske egenskaper                 
           \end{itemize}
       \end{center}

\end{minipage}
\end{center}

\tran{\vspace*{-1.5cm} \small \bf "Kvantedatamaskinen, den neste teknologiske
revolusjonen?"}

\Large
\begin{center}
   \begin{minipage}{0.8\textwidth}
       \vspace*{0.5cm}
       \begin{center}
            { \Huge \bf ENIAC 1946}
       \end{center}
\LARGE  ENIAC (Electronic Numerical Integrator And Computer) kunne addere
5000 tall per sekund. Dette svarer til noen tusen flytende talls 
operasjoner 
per sekund (FLOPS). Idag har vi maskiner som kan utf\o re trillioner av FLOPS.
 
ENIAC besto av 
\begin{enumerate}
\item ca.~19000 vakuumr\o r
\item veide ca.~30 tonn
\item brukte ca 174 kW, eller 233 hk
\item trengte ca.~150 m$^2$ med plass (30 ft $\times$ 50 ft) 
\end{enumerate}
\begin{center}
{\centering\mbox{\psfig {figure=eniac2.ps,height=8cm,width=10cm}
}}
\end{center}      
Et ekspertpanel i 1949 uttrykte forh\aa pningsfullt f\o lgende\newline\newline
``.... {\em en eller annen dag kan vi utvikle en like kraftig datamaskin med
bare 1500 vakkumr\o r, med vekt p\aa\ kanskje 1500 kg og et forbruk p\aa\
ca.~10 kW......}\newline
Resten er vel historie...?


 
\end{minipage}
\end{center}



\tran{\vspace*{-1.5cm} \small \bf "Kvantedatamaskinen, den neste teknologiske
revolusjonen?"}

\Large
\begin{center}
   \begin{minipage}{0.8\textwidth}
       \vspace*{0.5cm}
       \begin{center}
            { \LARGE \bf Fire postulater om naturen}
       \end{center}

Det er fire viktige postulater som danner grunnlaget for kvantemekanikkens
beskrivelse av naturen og Schr\"odingers likning som bevegelseslov.

Disse fire postulatene har ingen klassisk analog, og utgj\o r et sett
med  p\aa stander om naturen. 

\begin{itemize} 
\item Einsteins og Plancks postulat om energiens kvantisering 
     
      \[
          E=nh\nu,
      \]
       hvor $\nu$ er frekvensen og $h$ er Plancks konstant. Tallet 
       $n$ er et heltall og kalles for et kvantetall. 
\item De Broglie sitt postulat om materiens b\o lge og partikkel
      egenskaper. Det uttrykkes vha.~relasjonene
\[
     \lambda =\frac{h}{p}  \hspace{1cm} \nu=\frac{E}{h}    
\]
hvor $\lambda$ er b\o lgelengden og $p$ bevegelsesmengden.
Partikkelegenskapene uttrykkes via energien og bevegelsesmengden,
mens b\o lgelengden og frekvensen uttrykker b\o lgeegenskapene.
\item Heisenbergs uskarphetsrelasjon. 
\[
    \Delta {\bf p}\Delta {\bf x} \geq \frac{\hbar}{2},
\]
Klassisk er det slik at ${\bf x}$ og ${\bf p}$ er uavhengige st\o rrelser.
Kvantemekanisk derimot, siden $h\ne 0$, s\aa\ er avhengighets forholdet gitt ved
Heisenbergs uskarphets relasjon. Dette impliserer igjen at vi ikke kan 
lokalisere en partikkel og samtidig bestemme dens bevegelsesmengde skarpt. 
En ytterligere konsekvens er at vi ikke kan i et bestemt eksperiment observere
b\aa de partikkel og b\o lgeegenskaper. 

\item Paulis eksklusjonsprinsipp: den totale b\o lgefunksjonen for et 
      system som best\aa r av identiske partikler med halvtallig
      spinn m\aa\ v\ae re antisymmetrisk.  
\end{itemize}

 
\end{minipage}
\end{center}


\tran{\vspace*{-1.5cm} \small \bf "Kvantedatamaskinen, den neste teknologiske
revolusjonen?"}

\Large
\begin{center}
   \begin{minipage}{0.8\textwidth}
       \vspace*{0.5cm}
       \begin{center}
            { \LARGE \bf Kvante bits = QUBITS}
       \end{center}
\LARGE
\begin{itemize}
\item Idag representeres bit 0 og 1 vha.~spenningsforksjeller
\item Med dagens teknologi brukes ca.~100000 elektroner for \aa\
lagre en bit med informasjon
\item Kvantemekanikk tilbyr en enkel og naturlig representasjon av bits:
      vi kan f.eks.~tenke p\aa\ tilstander i et atom, hvor
bit 0 er gitt ved normaltilstanden (grunntilstanden) mens bit 1 er gitt
ved en elller annen eksitert tilstand
\item En enklere m\aa te er \aa\ se p\aa\ enkeltelektroner. Kan vi isolere
et enkelt elektron, kan vi vha.~et ytre p\aa satt magnetfelt ha spinn
egenverdier $+1/2$ eller $-1/2$. Den f\o rste kan da tilsvare bit 0 mens
den andre spinnegenveriden svarer til bit 1.
\item Nylig eksperiment (Physical Review Letters 84 (2000) 2223) isolerte
enkeltelektroner (kunstig hydrogenatom) i et omr\aa de av st\o rrelse
p\aa\ ca.~50 nanometer.
\end{itemize} 

En slik kvantemekanisk representasjon av en bit kalles {\bf QUBIT}.
\end{minipage}
\end{center}



\tran{\vspace*{-1.5cm} \small \bf "Kvantedatamaskinen, den neste teknologiske
revolusjonen?"}

\Large
\begin{center}
   \begin{minipage}{0.8\textwidth}
       \vspace*{0.5cm}
       \begin{center}
            { \LARGE \bf Heisenbergs uskarphetsrelasjon}
       \end{center}
\LARGE
\[
    \Delta {\bf p}\Delta {\bf x} \geq \frac{\hbar}{2}
\]

\end{minipage}
\end{center}

\tran{\vspace*{-1.5cm} \small \bf "Kvantedatamaskinen, den neste teknologiske
revolusjonen?"}

\begin{center}
   \begin{minipage}{0.8\textwidth}
       \vspace*{0.5cm}
       \begin{center}
            { \LARGE \bf Materiens b\o lge og partikkelnatur}
       \end{center}

\end{minipage}
\end{center}

\tran{\vspace*{-1.5cm} \small \bf "Kvantedatamaskinen, den neste teknologiske
revolusjonen?"}

\Large
\begin{center}
   \begin{minipage}{0.8\textwidth}
       \vspace*{0.5cm}
       \begin{center}
            { \LARGE \bf Schr\"odingers katt paradokset}
       \end{center}
\LARGE
Schr\"odingers likning (v\aa r naturlov)
\[
   -\hbar^2\frac{\partial^2\Psi(x,t)}{\partial x^2}+V(x,t)\Psi(x,t)=
    i\hbar\frac{\partial\Psi(x,t)}{\partial t}
\]
hvor $\Psi(x,t)$ er l\o sningen og kalles for b\o lgefunksjonen. Den er kompleks, og kvadratet tolkes som en sannsynlighet for \aa\ finne systemet i en bestemt
tilstand ved en gitt tid $t$. Ulike l\o sninger kan legges sammen, noe som
betyr at v\aa r katt nedenfor kan  v\ae re b\aa de d\o d og levende
samtidig!
\[
   \Psi(x,t)=\Psi(x,t)_{\mathrm{d\o d}}+\Psi(x,t)_{\mathrm{levende}}
\]
\end{minipage}
\end{center}



\tran{\vspace*{-1.5cm} \small \bf "Kvantedatamaskinen, den neste teknologiske
revolusjonen?"}

\Large
\begin{center}
   \begin{minipage}{0.8\textwidth}
       \vspace*{0.5cm}
       \begin{center}
            { \LARGE \bf Koherens og dekoherens}
       \end{center}
\LARGE
Siden ulike l\o sninger kan legges sammen, kan katten i prinsippet 
v\ae re b\aa de d\o d og levende
\[
   \Psi(x,t)=\Psi(x,t)_{\mathrm{d\o d}}+\Psi(x,t)_{\mathrm{levende}}
\]
En slik {\bf superposisjon} av kvantemekaniske tilstander kalles for koherente
tilstander (kvante koherens) og f\o lger fra postulatet om materiens
b\o lge og partikkel natur.

Ved m\aa ling, 'kollapser' l\o sningen til enten i live eller d\o d.

Kvantemekanisk sier vi at sannsynligheten for at katten er i live er
gitt ved 
\[
   |\Psi(x,t)_{\mathrm{levende}}|^2
\]
og for at den er d\o d ved
\[
   |\Psi(x,t)_{\mathrm{d\o d}}|^2
\]

N\aa r l\o sningen klapper sammen til enten den ene eller andre, kalles
det for dekoherens. 
\end{minipage}
\end{center}


\tran{\vspace*{-1.5cm} \small \bf "Kvantedatamaskinen, den neste teknologiske
revolusjonen?"}

\Large
\begin{center}
   \begin{minipage}{0.8\textwidth}
       \vspace*{0.5cm}
       \begin{center}
            { \LARGE \bf Entanglement}
       \end{center}
Den s\aa kalte Bell tilstanden , eller EPR par (etter Einstein, Podolsky
og Rosen) 
som er gitt vha.~to qubits
\[
   \frac{|00\rangle + |11\rangle}{\sqrt{2}}
\]
er en viktig byggestein i kvanteinformasjonsteori, teleportasjon og
superdense koding.

Den har den egenskapen at ved m\aa ling p\aa\ den f\o rste
qubiten har vi to utkommer: 0 med sannsynlighet 1/2 og sluttilstanden
$|00\rangle$ og 1 med
sannsynlighet 1/2 og sluttilstand $|11\rangle$.

Det interessante her er at m\aa ling p\aa\ den andre qubiten gir samme
resultat som m\aa ling p\aa\ den f\o rste.

Dvs.~at utkommet av m\aa lingene er korrelerte.

Slike korrelasjoner har v\ae rt heftig debattert siden artikkelen til  
Einstein, Podolsky
og Rosen i 1935. 

John Bell viste at m\aa lingskorrelasjonene mellom slike kvantemekaniske
tilstander er mye sterkere enn de som finnes blant klassiske
systemer. 

Kvantemekanikk tillatter dermed informasjonsbehandling langt utover det klassiske systemer gj\o r. 

Med $n=500$ qubits har vi $2^{500}$ mulige tilstander, som er mye st\o rre
enn det estimerte antallet atomer i verdensrommet.

Med dagens teknologi, kan vi lagre informasjon som svarer til
ca.~$\sim 2^{30}-2^{35}$ i RAM p\aa\ de beste datamaskinene vi har
tilgjengelig. 

\end{minipage}
\end{center}


\tran{\vspace*{-1.5cm} \small \bf "Kvantedatamaskinen, den neste teknologiske
revolusjonen?"}

\Large
\begin{center}
   \begin{minipage}{0.8\textwidth}
       \vspace*{0.5cm}
       \begin{center}
            { \LARGE \bf Hva kan en kvantedatamaskin gj\o re i tillegg til en vanlig datamaskin?}
       \end{center}
\begin{itemize}
 \item Kvanteinformasjonsteori, ny m\aa te \aa\ tenke informasjonsteori p\aa\
 \item Kvantealgoritmer til kryptering og s\o king i store databaser
 \item Simulering av fysiske systemer i langt st\o rre skala enn dagens maskiner.  
 \item Raskere s\o king i store databaser
 \item Bra utgangspunkt for parallellisering pga.~superponering av kvantemekaniske tilstander 
  \item Operere med mye st\o rre datamengder
 \item Generere virkelige vilk\aa rlige tall
  \item Teleportasjon?
\item ...
\end{itemize}

\end{minipage}
\end{center}



\tran{\vspace*{-1.5cm} \small \bf "Kvantedatamaskinen, den neste teknologiske
revolusjonen?"}

\Large
\begin{center}
   \begin{minipage}{0.8\textwidth}
       \vspace*{0.5cm}
       \begin{center}
            { \LARGE \bf Hvordan lage en kvantedatamaskin?}
       \end{center}
\begin{itemize}
 \item Ionefelle baserte kretser, NOT og CNOT
 \item Quantum dots, et eller flere elektroner som er fanga inn i sm\aa\
       omr\aa der mellom halvlederlag
 \item Hulroms kvanteeletrodynamikk
 \item Kjernemagnetisk resonans
 \item Supraledende Qubits
\end{itemize}

Hittil har en v\ae rt i stand til \aa\ lage CNOT kretser vha.~ionefelle
baserte kretser og kjernemagnetisk resonans. I tillegg har kryptering
av tall opp til ca.~100 blitt gjort. 

Stort potensiale i dagens nanoteknologi med tanke p\aa\ b\aa de
quantm dots og supraledende Qubits baserte kretser. Bare i EU finnes
det per dags dato 17 rammeprogrammer som utforsker ulike sider
av kvantedatamaskiner, fra reint teoretiske studier til konkret bygging
av kretser.

Kvanteinformasjonsteori representerer kanskje ogs\aa\ en heilt ny m\aa te
\aa\ tenke informasjonsteori p\aa\ . 
\end{minipage}
\end{center}



\end{document}







