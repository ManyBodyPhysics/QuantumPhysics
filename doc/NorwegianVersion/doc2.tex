\documentclass{article}
\usepackage{maple2e}
\usepackage{a4wide}
\DefineParaStyle{Maple Output}
\DefineParaStyle{Maple Plot}
\DefineCharStyle{2D Math}
\DefineCharStyle{2D Output}
\newcommand{\be}{\begin{equation}}
\newcommand{\ee}{\end{equation}}
\newcommand{\ba}{\begin{eqnarray}}
\newcommand{\ea}{\end{eqnarray}} 

\begin{document}

\section*{Et lite notat om  oppgavene 2.2 og 2.4}

\input{opp1}
\begin{maplegroup}
\begin{mapleinput}
\mapleinline{active}{1d}{beta:=2*938/197;}{%
}
\end{mapleinput}

\mapleresult
\begin{maplelatex}
\[
\beta  := {\displaystyle \frac {1876}{197}} 
\]
\end{maplelatex}

\end{maplegroup}
\begin{maplegroup}
\begin{mapleinput}
\mapleinline{active}{1d}{a:=2;}{%
}
\end{mapleinput}

\mapleresult
\begin{maplelatex}
\[
a := 2
\]
\end{maplelatex}

\end{maplegroup}
\begin{maplegroup}
\begin{mapleinput}
\mapleinline{active}{1d}{\xi_0:=100/2/938;}{%
}
\end{mapleinput}

\mapleresult
\begin{maplelatex}
\[
\mathit{\xi_0} := {\displaystyle \frac {25}{469}} 
\]
\end{maplelatex}

\end{maplegroup}
\begin{maplegroup}
\begin{mapleinput}
\mapleinline{active}{1d}{plot(\{tan(beta*a*sqrt(xi_0-xi)),-sqrt((xi_0-xi)/xi)
\},xi=0.001..0.045);}{%
}
\end{mapleinput}

\mapleresult
\begin{center}
\mapleplot{doc201.eps}
\end{center}

\end{maplegroup}
Plottet forteller oss at l\o sningen ligger n\ae r $\xi=0.03$.
Den grafiske l\o sningen gir oss kun en pekepinn p\aa\ st\o rrelsen
til $\xi$. \O nsker vi et mere presist svar m\aa\ vi ty til numerisk
innsats. Det finnes et vell av metoder for \aa\ finne nullpunktet
av 
\[
   tan(a\beta\sqrt{\xi_0-\xi})\pm\sqrt{\frac{\xi_0-\xi}{\xi}}=0.
\]
Vi velger Newton sin metode og lager en prosedyre i Maple.
Newtons metode kan beskrives som f\o lger
\begin{itemize}
\item Vi finner f\o rst et punkt p\aa\ x-aksen som vi antar 
er n\ae r l\o sningen, jfr.\ den grafiske l\o sningen ovenfor.
\item Vi finner krumningen til kurven i dette punktet.
\item Vi trekker tangenten ved dette punktet og ser hvr 
tangenten treffer x-aksen. Det sistnevnte gir oss en ny x verdi som
er n\ae rmere l\o sningen. Det nye punktet $x_1$ kan vi s\aa\ bruke
til \aa\ bestemme et nytt punkt.
\item Slik kan en fortsette til en finner den numeriske verdien
for $x$ hvor den ovennevnte likning blir null.
\item Matematisk kan vi uttrykke denne iterasjons prosessen som
     \[
        x_{i+1}=x_i-\frac{f(x_i)}{f'(x_i)},
     \]
hvor $f$ er selve funksjonen, i v\aa rt tilfelle
\[
   tan(a\beta\sqrt{\xi_0-\xi})\pm\sqrt{\frac{\xi_0-\xi}{\xi}}
\]
\end{itemize}
I Maple kan vi skrive dette som 
\begin{maplegroup}
\begin{mapleinput}
\mapleinline{active}{1d}{MakeIteration:= proc(expr::algebraic,
x::name)}{%
}
\end{mapleinput}

\begin{mapleinput}
\mapleinline{active}{1d}{    local iteration;}{%
}
\end{mapleinput}

\begin{mapleinput}
\mapleinline{active}{1d}{    iteration:=x-expr/diff(expr,x);}{%
}
\end{mapleinput}

\begin{mapleinput}
\mapleinline{active}{1d}{    unapply(iteration, x);}{%
}
\end{mapleinput}

\end{maplegroup}
Uttrykket {\it expr} er selve funksjonen v\aa r 
\begin{maplegroup}
\begin{mapleinput}
\mapleinline{active}{1d}{expr:=tan(beta*a*sqrt(xi_0-xi))+sqrt((xi_0-xi)/xi);}
{%
}
\end{mapleinput}

\end{maplegroup}
\begin{maplegroup}
\begin{mapleinput}
\mapleinline{active}{1d}{Newton:=MakeIteration(expr, xi);}{%
}
\end{mapleinput}
\end{maplegroup}
Fra grafen finner vi en startverdi for iterasjonen {\it guess}
\begin{maplegroup}
\begin{mapleinput}
\mapleinline{active}{1d}{guess:=0.03;}{%
}
\end{mapleinput}

\mapleresult
\begin{maplelatex}
\[
\mathit{guess} := .03
\]
\end{maplelatex}

\end{maplegroup}
\begin{maplegroup}
\begin{mapleinput}
\mapleinline{active}{1d}{to 1. do guess:=Newton(guess); od;}{%
}
\end{mapleinput}

\mapleresult
\begin{maplelatex}
\[
\mathit{guess} := .03645979678
\]
\end{maplelatex}

\end{maplegroup}
\begin{maplegroup}
\begin{mapleinput}
\mapleinline{active}{1d}{answer:=guess*2*938;}{%
}
\end{mapleinput}

\mapleresult
\begin{maplelatex}
\[
\mathit{answer} := 68.39857876
\]
\end{maplelatex}

\end{maplegroup}
\begin{maplegroup}
\begin{mapleinput}
\end{mapleinput}

\end{maplegroup}
Vi f\aa r alts\aa\ en bindingsenergi p\aa\ $68$ MeV, som er mye
mere enn den eksperimentelle til deuterium.
Velger vi $V_0=30$ MeV f\aa r vi $6$ MeV, vi n\ae rmer oss!
Selv om vi har valgt et enkelt kasse potensial, kan vi vha.\
av kun massen til protonet og en antagelse om rekkevidden
finne realistiske verdier for bindingsnergien.
Rekkevidden til vekselvirkningen mellom proton og n\o ytroner 
er typisk noen f\aa\ fm, og verdien p\aa\ $V_0 \approx 50 $ MeV
er heller ikke s\aa\ urealistisk.

Vi legger ogs\aa\ merke til at n\aa r styrken til potensialet blir 
mindre, s\aa\ blir bindingsenergien ogs\aa\ mindre.
Hvorfor er ikke bindingsenergien lik styrken p\aa\ potensialet?

S\aa\ en ny utfordring. Vi fant ovenfor b\o lgefunksjonen til systemet
v\aa rt. Med den gitte bindingsenergien for $V_0=100$ og $V_0=30$ MeV
pr\o v \aa\ plotte i Maple selve b\o lgefunksjonen n\aa r $E<0$.
\input{opp2}
\input{opp3}
\end{document}
%% End of Maple V Output
