
%  Tex-fil for eksamensoppgaver i FYS 113
%
\documentclass[11pt]{article}
%
\usepackage[T1]{fontenc}       % DC-fonter
\usepackage{epsfig}
\usepackage{a4}
\usepackage{norsk}             % Norske ord
\renewcommand{\rmdefault}{ptm} % Times
%
%%
\addtolength\textwidth{1cm}
%	\addtolength\evensidemargin{-1cm}
%	\addtolength\oddsidemargin{1cm}

\listfiles

% Spesielle kommandoer
\newcommand{\OP}[1]{{\bf\widehat{#1}}}
%
\newcommand{\BE}{\begin{eqnarray}}
%
\newcommand{\EE}{\end{eqnarray}}
%
\newcommand{\BES}{\begin{eqnarray*}}
%
\newcommand{\EES}{\end{eqnarray*}}
%



%
\pagestyle{myheadings}
%

\begin{document}

\newpage

%
\begin{center}
\LARGE \bf UNIVERSITETET I OSLO
\end{center}
%
{\Large Det matematisk-naturvitenskapelige fakultet}\\[2ex]
%
Eksamen i FYS-113: Kvantefysikk\\
Eksamensdag: Tirsdag 19 juni 2001\\
Tid for eksamen: 0900 - 1500\\
Eksamenssettet  best\aa r av 4 sider, kontroll\'{e}r at du har riktig antall.\\
\begin{minipage}[t]{0.3\textwidth}
Tillatte hjelpemidler:
\end{minipage}
\hspace*{0.1\textwidth}
\begin{minipage}[t]{0.6\textwidth}
{\O}grim: St{\o}rrelser og enheter i fysikken\\
Rottmann: Matematische Formelsamlung\\
To A4 ark med egne notater\\
Godkjent numerisk elektronisk kalkulator
\end{minipage}



\noindent {\Large \bf Oppgave 1.}\\[1ex]
%
\noindent

Vi skal studere r\o ntgenstr\aa ling ved hjelp av et r\o ntgenr\o r, 
der vi lar
en elektronstr\aa le gjennoml\o pe et spenningsfall $V$ f\o r den treffer
en anode. Et eksempel p\aa\ den resulterende str\aa lingen for en slik
prosess er vist i figur 1. 
\begin{itemize}
%
\item[a)] Tegn en skjematisk skisse av et r\o ntgenr\o r og forklar kort
hvordan r\o ntgenstr\aa ling dannes. 
Forklar den jevne kontinuerlige delen av
intensitetsfordelingen.
\item [b)] Skriv ned og begrunn sammenhengen mellom spenningen
$V$ og den minste b\o lgelengden $\lambda_{\mathrm{min}}$ for
r\o ntgenstr\aa lning. 
Gj\o r det samme for $V$ og den maksimale frekvensen
$\nu_{\mathrm{maks}}$. Bruk figur 1 til \aa\ finne spenningen $V$,
$\nu_{\mathrm{maks}}$ og den maksimale energien for r\o ntgenstr\aa lning.
\item [d)] Forklar \aa rsaken til de skarpe toppene (s\aa kalte
karakteristiske linjer) i figur 1. I figuren ser vi ogs\aa\ to 
topper $\alpha_1$ og $\alpha_2$ ved n\ae r samme b\o lgelengde.
Forklar den fysiske prosessen som gir opphav til en slik oppsplitting 
i to topper $\alpha_1$ og $\alpha_2$. 

\end{itemize}
\begin{figure}[h]
\vspace{6cm}
\caption{ }
\end{figure}


%







\noindent
\noindent{\Large \bf Oppgave 2.}\\[1ex]
\noindent
I denne oppgaver skal vi studere heliumatomet som best�r av en atomkjerne
og  to elektroner. 
% 
\begin{itemize}
% 
\item[a)] Lag en skisse som beskriver heliumatomet som fysisk system
og defin\'{e}r koordinatene som trengs for � beskrive systemet.
Vi ser i denne oppgaver bort
fra ledd som beskriver spinn-bane effekten.

Systemets hamiltonoperator kan da skrives 
\[
   \OP{H}_{\mathrm{He}}=-\frac{\hbar^2\nabla_1^2}{2m_e}-\frac{\hbar^2\nabla_2^2}{2m_e}
          -\frac{2ke^2}{r_1}-\frac{2ke^2}{r_2}+
               \frac{ke^2}{r_{12}},
\]
hvor $k=\frac{1}{4\pi\epsilon_0}$ og $m_e$ elektronets masse. 
Forklar kort hva de ulike leddene st\aa r for.
% 
\item[b)] Ta utgangspunkt i den kvantemekaniske l�sningen av
hydrogenatomet 
%
\begin{equation}
\OP{H}_{\mathrm{H}} |\psi_{nlm_lm_s} \rangle
      = - E_0\frac{1}{n^2}|\psi_{nlm_lm_s} \rangle
\end{equation}
%
$E_0 =\frac{1}{2}\frac{e^4m_e}{(4\pi\epsilon_0)^2\hbar^2}=13.6$ eV.

 Finn en tiln�rmet l�sning for heliumatomet
hvor vi utnytter tilstanden i likning  (1)  over 
og neglisjerer vekselvirkningen
mellom elektronene.
Bestem den tiln�rmede verdien av energien for grunntilstanden. 
Angi med begrunnelse elektronkonfigurasjonen og  kvantetallene $L$, $S$ og $J$.

%
\end{itemize}
%
I det f�lgende skal vi bruke sentralfeltmodellen med et energispekter 
gitt i figur~2 
og studere  eksiterte tilstander for heliumatomet.
\begin{figure}
\vspace{6cm}
\caption{ }
\end{figure}
% 
\begin{itemize}
%
\item[c)] Sett opp alle eksiterte tilstander for heliumatomet hvor
ett elektron er i $1s$ mens det andre kan v�re i $2s$ eller $2p$. Bestem
eksitasjonesenergien og elektron konfigurasjonen. Angi med begrunnelse
kvantetallene $L$, $S$ og $J$ for alle tilstandene.
%
\end{itemize}
% 
Noen av tilstandene under punkt c) har $S = 0$ (singlet) mens andre har
$S = 1$ (triplet). Eksperimentelt har slike tilstander som  kun
adskiller seg p� kvantetallet $S$, forskjellig energi.
% 
\begin{itemize}
% 
\item[d)] Forklar  hvilken mekanisme som ligger til grunn for denne
forskjellen og angi hvilken kvalitativ effekt den har.
%
\item[e)] Hvis heliumatomet er i en eksitert tilstand kan den i visse
tilfeller henfalle til en annen tilstand med lavere energi gjennom
dipol str�ling. Sett opp utvalgsreglene for slik str�ling og angi de
mulige dipol overgangene i det eksiterte spektret som ble funnet ovenfor. 
%
\end{itemize}
% 
Vi skal i den resterende del av oppgaven returnere til heliumatomets
grunntilstand diskutert under punkt a) og b) og pr�ve � finne en bedre
l�sning for grunntilstandsenergien.

Vi antar en grunntilstands konfigurasjon $1s^2$ og hvert av
elektronene har en b�lgefunksjon av formen 
%
\[
g(r) = \frac{\beta^{3/2}}{\sqrt{\pi}} \exp( -\beta r)
\]
% 
hvor $\beta$ er en konstant som vi skal variere for � oppn� en best
mulig verdien p� grunntilstands energien.
Til en slik beregning trenger vi endel tredimensjonale rom--integraler
%
\[
\int \left \{ g(r)\right \}^2 d\vec{r} = 1; \;\;\;
\int \left \{ g(r)\right \}^2 \frac{1}{r} d\vec{r} = \beta; \;\;\;
\int g(r) \nabla^2 g(r) d\vec{r} = -\beta^2,
\]
%
og det dobbelte rom--integralet
%
\[
\int \int \left \{ g(r_1) \right \}^2 \left \{ g(r_2) \right \}^2
				\frac{1}{r_{12}}  d\vec{r}_1 d\vec{r}_2 = \frac{5}{8} \beta
					 \;\;\; hvor \;\;\; r_{12} = | \vec{r_1} - \vec{r_2} |.
\]
%
\begin{itemize}
% 
\item[f)]  Vis at forventningsverdien (middelverdien) 
$\langle \OP{H}_{\mathrm{He}}\rangle$ fra punkt a)
i den gitte konfigurasjonen gir f�lgende uttrykk
%
\[
\langle\OP{H}_{\mathrm{He}} \rangle = \frac{\hbar^2}{m_e} \beta^2
		  - \frac{27}{32} \frac{e^2}{\pi \epsilon_0} \beta.
\]
\end{itemize}
%
%
Vi kan n{\aa} finne energien i grunntilstanden ganske n{\o}yaktig ved
et variasjonsprinsipp. Det sier at b{\o}lgefunksjoner som ikke er
lik den eksakte energi egenfunksjonen for
grunntilstanden, alltid gir en forventningsverdi for energien  som er
st{\o}rre enn den virkelige egenverdien.
Metoden g{\aa}r ut p{\aa} {\aa} finne den b{\o}lgefunksjonen
(av typen $g(r)$ ), som gir den minste
forventningsverdien for energien.
%
\begin{itemize}
%
\item[g)] Hvilken (tiln{\ae}rmet) verdi for energien i grunntilstanden
finner vi etter den skisserte metoden?\\
Til kontroll: Eksperimentelt er energien $-78,98 $~eV.

%
\end{itemize}



\noindent
\noindent{\Large \bf Oppgave 3.}\\[1ex]
\noindent
\begin{itemize}
\item[a)]
Forklar kort hvilken fysisk betydning vi tillegger en b\o lgefunksjon 
$\Psi(\vec{r},t)$ i kvantemekanikken, hvor $\vec{r}$ er romkoordinaten
og $t$ tiden. Hvilke matematiske krav m\aa\
vi stille til en slik b\o lgefunksjon?
Forklar kort hva vi mener med en kvantemekanisk 
operator, dens forventningsverdi  og en egenfunksjon
for en operator.
\item[b)] Skriv ned den tidsavhengige Schr\"odingerlikningen for en partikkel
med potensiell energi $V(\vec{r})$ og vis at l\o sningene kan skrives 
p\aa\ formen
\[
   \Psi(\vec{r},t)= \psi(\vec{r})\exp{(-iEt/\hbar)},
\]
og finn likningen som $\psi(\vec{r})$ tilfredsstiller. 

\end{itemize}


Vi skal i resten av denne oppgaven studere det 
kvantemekaniske problemet for et
elektron som er fanget inn i et en-dimensjonalt harmonisk oscillator potensial
$V(x)=\frac{1}{2}kx^2$. 
Vi skal bruke enhet $eV$ for energi og $nm$ for lengde. 
Massen til elektronet er $m_e$ og $k$ er en reell konstant.
Hamiltonoperatoren 
$\hat{H}$ kan skrives som 
\begin{displaymath}
\hat{H} = -\frac{\hbar^2}{2m_e}\frac{d^2}{dx^2} +\frac{1}{2}kx^2.
\end{displaymath}
Tilstandsfunksjonen for de to laveste tilstandene ved tiden $t=0$ er gitt ved
\begin{displaymath}
 \psi_0(x)=\left(\frac{\alpha}{\sqrt{\pi}}\right)^{\frac{1}{2}}
                 \exp{(-\frac{1}{2}\alpha^2x^2)}, 
\end{displaymath}
og 
\begin{displaymath}
 \psi_1(x)= \left(\frac{2\alpha^3}{\sqrt{\pi}}\right)^{\frac{1}{2}}
                 x\exp{(-\frac{1}{2}\alpha^2x^2)}, 
\end{displaymath}

\begin{itemize}
\item[c)] Vis at begge tilstandsfunksjonene er energi egentilstander for
den harmoniske oscillatoren og bestem konstanten $\alpha$ (funksjonene er normerte). Finn 
de tilh\o rende energi egenverdiene. 
Uttrykk energien ved hjelp av vinkelhastigheten $\omega =\sqrt{k/m_e}$.
Skriv ned b\o lgefunksjonene
$\Psi_0(x, t)$ og $\Psi_1(x, t)$ ved tiden $t$. 
\end{itemize}
Det elektriske dipolmomentet er gitt ved $D=-ex$. Den kvantemekaniske 
forventningsverdien for dipolmomentet ved tiden $t=0$ er gitt ved
\begin{displaymath}
 -e\langle x\rangle =-e\int_{-\infty}^{\infty}
                      \Psi_i(x,t)^*x\Psi_j(x,t)dx, 
\end{displaymath}
hvor $i$ og $j$ kan svare til tilstandene 
$\Psi_0(x,t)$ og $\Psi_1(x, t)$.

\begin{itemize}
\item[d)] 
Hvilke av kombinasjonene $i=0,1$ og $j=0,1$ er forskjellige fra null?
Hva forteller det om elektrisk dipoloverganger? 
\end{itemize}

Systemet er n\aa\ i en blandet tilstand gitt ved
\begin{displaymath}
    \Psi(x,t)=\frac{1}{\sqrt{2}}\left(\Psi_0(x, t)+\Psi_1(x, t)\right).
\end{displaymath}
\begin{itemize}
\item[e)] Bruk de tidsavhengige l\o sningene fra punkt c) til \aa\
rekne ut dipolmomentet ved $t\neq 0$ 
\begin{displaymath}
 -e\langle x(t)\rangle =-e\int_{-\infty}^{\infty}
                      \Psi(x,t)^*x\Psi(x,t)dx. 
\end{displaymath}
som funksjon av vinkelhastigheten
$\omega$. 

\end{itemize}






\end{document}

