

\chapter{DET PERIODISKE SYSTEMET}



\section{Introduksjon}

Med utgangspunktet\footnote{Notatene i denne versjonen
er litt mangelfulle. Emnet dekkes bedre av
boka, se avsnittene 9-1, 9-2, 9-3, og fra og med 9-5
til og med 9-9, som er pensum.}  
i resultatene fra hydrogenatomet, skal
vi n\aa\ pr\o ve oss p\aa\ forst\aa\ det periodiske 
systemet utifra kvantemekanikk.

Et av kjennetegnene er at vi har grunnstoffer med liknende
kjemiske egenskaper men h\o yst ulik verdi p\aa\ kjerneladningen
$Z$. Eksempler er atomene i edelgasserien, helium ($Z=2$), 
neon ($Z=10$), argon ($Z=18$),
krypton ($Z=38$) osv., eller alkalimetallene, 
med atomer som f.eks.~litium ($Z=3$),
natrium ($Z=11$), kalium ($Z=19$)osv. Felles for 
edelgassene er at ionisasjonsenergien, dvs.~energien som trengs
for \aa\ l\o srive et elektron er stor i forhold
til f.eks.~alkalimetallene. Figur 5.1 viser ionisasjonsenergien
som funksjon av  kjerneladningen $Z$ opp til $Z=50$. 
Det er bla.~denne trenden kvantemekanikken med Schr\"odingers
likning (SL) var i stand til \aa\ forklare. 
For \aa\ forst\aa\ det periodiske systemet og dets oppbygging
skal vi f\o rst g\aa\ til det nest enkleste atomet, helium
med to elektroner i stedet for ett. 
I de etterf\o lgende avsnitt skal vi diskutere egenskapene
til andre elementer.
\begin{figure}[h]
\setlength{\unitlength}{1mm}
   \begin{picture}(100,100)
   \put(-60,-130){\epsfxsize=25cm \epsfbox{ioni.ps}}
   \end{picture}
%\begin{center}
%% GNUPLOT: LaTeX picture with Postscript
\setlength{\unitlength}{0.1bp}
\special{!
%!PS-Adobe-2.0
%%Creator: gnuplot
%%DocumentFonts: Helvetica
%%BoundingBox: 50 50 770 554
%%Pages: (atend)
%%EndComments
/gnudict 40 dict def
gnudict begin
/Color false def
/Solid false def
/gnulinewidth 5.000 def
/vshift -33 def
/dl {10 mul} def
/hpt 31.5 def
/vpt 31.5 def
/M {moveto} bind def
/L {lineto} bind def
/R {rmoveto} bind def
/V {rlineto} bind def
/vpt2 vpt 2 mul def
/hpt2 hpt 2 mul def
/Lshow { currentpoint stroke M
  0 vshift R show } def
/Rshow { currentpoint stroke M
  dup stringwidth pop neg vshift R show } def
/Cshow { currentpoint stroke M
  dup stringwidth pop -2 div vshift R show } def
/DL { Color {setrgbcolor Solid {pop []} if 0 setdash }
 {pop pop pop Solid {pop []} if 0 setdash} ifelse } def
/BL { stroke gnulinewidth 2 mul setlinewidth } def
/AL { stroke gnulinewidth 2 div setlinewidth } def
/PL { stroke gnulinewidth setlinewidth } def
/LTb { BL [] 0 0 0 DL } def
/LTa { AL [1 dl 2 dl] 0 setdash 0 0 0 setrgbcolor } def
/LT0 { PL [] 0 1 0 DL } def
/LT1 { PL [4 dl 2 dl] 0 0 1 DL } def
/LT2 { PL [2 dl 3 dl] 1 0 0 DL } def
/LT3 { PL [1 dl 1.5 dl] 1 0 1 DL } def
/LT4 { PL [5 dl 2 dl 1 dl 2 dl] 0 1 1 DL } def
/LT5 { PL [4 dl 3 dl 1 dl 3 dl] 1 1 0 DL } def
/LT6 { PL [2 dl 2 dl 2 dl 4 dl] 0 0 0 DL } def
/LT7 { PL [2 dl 2 dl 2 dl 2 dl 2 dl 4 dl] 1 0.3 0 DL } def
/LT8 { PL [2 dl 2 dl 2 dl 2 dl 2 dl 2 dl 2 dl 4 dl] 0.5 0.5 0.5 DL } def
/P { stroke [] 0 setdash
  currentlinewidth 2 div sub M
  0 currentlinewidth V stroke } def
/D { stroke [] 0 setdash 2 copy vpt add M
  hpt neg vpt neg V hpt vpt neg V
  hpt vpt V hpt neg vpt V closepath stroke
  P } def
/A { stroke [] 0 setdash vpt sub M 0 vpt2 V
  currentpoint stroke M
  hpt neg vpt neg R hpt2 0 V stroke
  } def
/B { stroke [] 0 setdash 2 copy exch hpt sub exch vpt add M
  0 vpt2 neg V hpt2 0 V 0 vpt2 V
  hpt2 neg 0 V closepath stroke
  P } def
/C { stroke [] 0 setdash exch hpt sub exch vpt add M
  hpt2 vpt2 neg V currentpoint stroke M
  hpt2 neg 0 R hpt2 vpt2 V stroke } def
/T { stroke [] 0 setdash 2 copy vpt 1.12 mul add M
  hpt neg vpt -1.62 mul V
  hpt 2 mul 0 V
  hpt neg vpt 1.62 mul V closepath stroke
  P  } def
/S { 2 copy A C} def
end
%%EndProlog
}
\begin{picture}(3600,2160)(0,0)
\special{"
%%Page: 1 1
gnudict begin
gsave
50 50 translate
0.100 0.100 scale
0 setgray
/Helvetica findfont 100 scalefont setfont
newpath
-500.000000 -500.000000 translate
LTa
600 251 M
2817 0 V
LTb
600 251 M
63 0 V
2754 0 R
-63 0 V
600 623 M
63 0 V
2754 0 R
-63 0 V
600 994 M
63 0 V
2754 0 R
-63 0 V
600 1366 M
63 0 V
2754 0 R
-63 0 V
600 1737 M
63 0 V
2754 0 R
-63 0 V
600 2109 M
63 0 V
2754 0 R
-63 0 V
1107 251 M
0 63 V
0 1795 R
0 -63 V
1670 251 M
0 63 V
0 1795 R
0 -63 V
2234 251 M
0 63 V
0 1795 R
0 -63 V
2797 251 M
0 63 V
0 1795 R
0 -63 V
3361 251 M
0 63 V
0 1795 R
0 -63 V
600 251 M
2817 0 V
0 1858 V
-2817 0 V
600 251 L
LT0
3114 1946 M
180 0 V
600 1262 M
56 817 V
713 652 L
56 292 V
56 -76 V
57 220 V
56 243 V
56 -68 V
57 283 V
56 307 V
1163 633 L
57 187 V
56 -124 V
56 161 V
57 174 V
56 -10 V
56 194 V
57 207 V
56 -848 V
56 131 V
57 32 V
56 21 V
56 -6 V
57 2 V
56 50 V
57 32 V
56 -1 V
56 -16 V
57 6 V
56 124 V
56 -252 V
57 141 V
56 142 V
56 -4 V
57 153 V
56 162 V
56 -729 V
57 113 V
112 50 V
57 34 V
56 3 V
56 17 V
57 13 V
56 13 V
56 66 V
57 -57 V
56 105 V
56 -238 V
57 116 V
3174 1946 D
600 1262 D
656 2079 D
713 652 D
769 944 D
825 868 D
882 1088 D
938 1331 D
994 1263 D
1051 1546 D
1107 1853 D
1163 633 D
1220 820 D
1276 696 D
1332 857 D
1389 1031 D
1445 1021 D
1501 1215 D
1558 1422 D
1614 574 D
1670 705 D
1727 737 D
1783 758 D
1839 752 D
1896 754 D
1952 804 D
2009 836 D
2065 835 D
2121 819 D
2178 825 D
2234 949 D
2290 697 D
2347 838 D
2403 980 D
2459 976 D
2516 1129 D
2572 1291 D
2628 562 D
2685 675 D
2797 725 D
2854 759 D
2910 762 D
2966 779 D
3023 792 D
3079 805 D
3135 871 D
3192 814 D
3248 919 D
3304 681 D
3361 797 D
stroke
grestore
end
showpage
}
\put(3054,1946){\makebox(0,0)[r]{Ionisasjonsenergi}}
\put(2008,51){\makebox(0,0){$Z$}}
\put(100,1180){%
\special{ps: gsave currentpoint currentpoint translate
270 rotate neg exch neg exch translate}%
\makebox(0,0)[b]{\shortstack{Ionisasjonsenergi (eV)}}%
\special{ps: currentpoint grestore moveto}%
}
\put(3361,151){\makebox(0,0){50}}
\put(2797,151){\makebox(0,0){40}}
\put(2234,151){\makebox(0,0){30}}
\put(1670,151){\makebox(0,0){20}}
\put(1107,151){\makebox(0,0){10}}
\put(540,2109){\makebox(0,0)[r]{25}}
\put(540,1737){\makebox(0,0)[r]{20}}
\put(540,1366){\makebox(0,0)[r]{15}}
\put(540,994){\makebox(0,0)[r]{10}}
\put(540,623){\makebox(0,0)[r]{5}}
\put(540,251){\makebox(0,0)[r]{0}}
\end{picture}

%\end{center}

\caption{Ionisasjonsenergi i eV som funksjon av ladningstallet $Z$
til kjernen.}
\end{figure}

\section{Heliumatomet}

Heliumatomet best\aa r av to elektroner pluss en kjerne.
Kjerneladningen er $Z=2$. 
N\aa r vi skal sette opp den potensielle energien til
systemet v\aa rt basert p\aa\ Coulombpotensialet,
m\aa\ vi ogs\aa\ ta hensyn til frast\o tingen mellom
elektronene.

Vi kaller avstanden mellom elektron 1 og kjernen
(som vi n\aa\ betrakter som et uendelig tungt massesenter)
for $r_1$, og avstanden mellom elektron 2 og kjernen for 
$r_2$ . Bidraget til den potensielle
energien pga.~tiltrekningen fra kjernen p\aa\ de to elektronene
blir dermed
\be
   -\frac{2ke^2}{r_1}-\frac{2ke^2}{r_2},
\ee 
og legger vi til frast\o tingen mellom de to elektronene
som er i en avstand $r_{12}=|{\bf r}_1-{\bf r}_2|$
har vi at den potensielle energien $V(r_1, r_2)$ er gitt
ved
\be
 V(r_1, r_2)=-\frac{2ke^2}{r_1}-\frac{2ke^2}{r_2}+
               \frac{ke^2}{r_{12}},
\ee
slik at den totale hamiltonfunksjonen for elektronene i heliumatomet
kan skrives
\be
   \OP{H}=-\frac{\hbar^2\nabla_1^2}{2m}-\frac{\hbar^2\nabla_2^2}{2m}
          -\frac{2ke^2}{r_1}-\frac{2ke^2}{r_2}+
               \frac{ke^2}{r_{12}},
\ee
og Schr\"odingers likning blir dermed 
\be
   \OP{H}\psi=E\psi.
\ee

Vi ser av uttrykket for den potensielle energien at vi ikke
lenger har et enkelt sentralsymmetrisk potensial.
For flere partikler blir det enda mere komplisert.

Dersom vi neglisjerer frast\o tingen mellom
elektroner, kan vi betrakte systemet som best\aa ende av 
ikke vekselvirkende elektroner, dvs.~elektronene f\o ler
kun tiltrekningen fra kjernen og vi kan addere deres bidrag til
energien. 

For heliumatomet betyr det at den potensielle energien
blir 
\be
    V(r_1, r_2)\approx -\frac{Zke^2}{r_1}-
                      \frac{Zke^2}{r_2}.
\ee
Fordelen med denne tiln\ae rmingen er at hvert elektron 
kan betraktes som uavhengig av det andre, den s\aa kalte
uavhengig-elektron model hvor hvert elektron kun ser
et sentralsymmetrisk potensial, eller sentral felt.

La oss se om det kan gi oss noe meningsfylt.
F\o rst setter vi $Z=2$ og neglisjerer fullstendig
frast\o tingen mellom elektronene.

Vi kan n\aa\ bruke resultatene fra hydrogenatomet.
Elektron 1 har dermed en hamiltonoperator
\be
   \OP{h}_1=-\frac{\hbar^2\nabla_1^2}{2m}
          -\frac{2ke^2}{r_1},
\ee
med tilh\o rende egenfunksjon og egenverdilikning
\be
   \OP{h}_1\psi_a=E_a\psi_a,
\ee
hvor $a=\{ n_al_am_{l_a}\}$, kvantetallene fra hydrogenatomet.
Energien $E_a$ er dermed
\be
   E_a=\frac{Z^2E_0}{n_a^2},
\ee
med $E_0=-13.6$ eV, grunntilstanden i hydrogenatomet.
Helt tilsvarende har vi for elektron 2
\be
   \OP{h}_2=-\frac{\hbar^2\nabla_2^2}{2m}
          -\frac{2ke^2}{r_2},
\ee
med tilh\o rende egenfunksjon og egenverdilikning
\be
   \OP{h}_2\psi_b=E_b\psi_b,
\ee
hvor $b=\{ n_bl_bm_{l_b}\}$, og 
\be
   E_b=\frac{Z^2E_0}{n_b^2}.
\ee

Siden elektronene ikke vekselvirker kan vi anta at
grunntilstanden til heliumatomet er gitt ved produktet
\be
  \psi=\psi_a\psi_b,
\ee
slik at approksimasjonen til $\OP{H}$ gir f\o lgende SL
\be
   \left(\OP{h}_1+\OP{h}_2\right)\psi=
    \left(\OP{h}_1+\OP{h}_2\right)
    \psi_a({\bf r}_1)\psi_b({\bf r}_2)=
    E_{ab}\psi_a({\bf r}_1)\psi_b({\bf r}_2).
\ee
Energien blir dermed
\be
    \left(\OP{h}_1\psi_a({\bf r}_1)\right)\psi_b({\bf r}_2) +
    \left(\OP{h}_2\psi_b({\bf r}_2)\right)\psi_a({\bf r}_1) =
    \left(E_{a}+E_b\right)\psi_a({\bf r}_1)\psi_b({\bf r}_2),
\ee
dvs.
\be
   E_{ab}=Z^2E_0\left(\frac{1}{n_a^2}+\frac{1}{n_b^2}\right).
\ee
Setter vi inn $Z=2$ og antar at grunntilstanden er gitt ved
elektronene i laveste en-elektron tilstand med $n_a=n_b=1$
blir energien
\be
    E_{ab}=8E_0=-108.8\hspace{0.1cm} \mathrm{eV},
\ee
mens den eksperimentelle verdien er $-78.8$ eV.
{\bf Vi ser klart at \aa\ kutte det frast\o tende leddet
pga.~vekselvirkningen mellom elektronene gj\o r at vi
f\aa r for mye binding.}

Ionisasjonsenergien er gitt ved den energien som kreves for \aa\
l\o srive et elektron. Energien for \aa\ l\o srive det ene elektronet
er i den uavhengige-elektron modellen gitt ved
halve bindingsenergien, dvs.~$54.4$ eV, mens den eksperimentelle
verdien er $24.59$ eV.

Selv om vi har neglisjert frast\o tingen mellom elektronene, kommer
fremdeles mesteparten av bidraget til energien fra
tiltrekningen mellom kjernen og elektronene. 
La oss derfor gj\o re f\o lgende tankeeksperiment.

Anta at vi skal bygge det periodiske systemet vha.~den uavhengige
elektronmodellen. Vi antar ogs\aa\ at alle elektronene kan
plasseres i laveste en-elektron orbital gitt ved
$n=1$. 
For et system med $N$ elektroner og $Z$ protoner 
blir dermed bindingsenergien
n\aa r vi plasserer alle elektronene i laveste tilstand
\be
   E_N=NZ^2E_0.
\ee
Selv om den effektive ladningen skulle v\ae re liten, vil
fremdeles bindingsenergien \o ke som funksjon av elektrontallet
og ladningen. Ionisasjonsenergien vil v\ae re gitt ved
\be
   Z^2E_0,
\ee
som  er i strid med de eksperimentelle dataene vist i figur
5.1.

Hvordan komme ut av dette uf\o ret? Svaret er Paulis prinsipp
som vi diskuterer i neste avsnitt. 

\section{System av flere elektroner}
%
\noindent
\begin{minipage}{0.45\textwidth}
%
Vi skal i det f{\o}lgende gi en kort oversikt over
hvordan vi bestemmer elektron konfigurasjonen
og de karakteristiske egenskaper for
grunntilstanden (tilstanden med st{\o}rst bindingsenergi)
i et atom innenfor det som heter {\bf sentralfelt  modellen}.

\hspace*{0.5cm}  Et atom karakteriseres ved et atomnummer Z.
Det angir samtidig antall elektroner i systemet.  I sentralfelt modellen
antar vi at den elektriske tiltrekningen mellom elektronene og
den indre atomkjernen samt  frast{\o}tningen elektronene imellom
gir opphav til et modifisert Coulombfelt. I dette feltet
beveger elektronene seg tiln{\ae}rmet uavhengig av hverandre
p{\aa} tilsvarende m{\aa}te som det ene elektronet i hydrog\'{e}n
atomet. Hvert enkelt elektron er karakterisert ved et sett kvantetall
$n, l,m_l, m_s$ med samme fysiske betydningen
som i hydrog\'{e}n atomet.

%
\end{minipage}
%
\hspace{0.05\textwidth}
%
\begin{minipage}{0.50\textwidth}
%
\begin{center}
\setlength{\unitlength}{1.0mm}
\begin{picture}(50,85)(0,-15)
\thicklines
%1s
\put(0,0){K}
\put(10,0){1s}
\put(20,0){\line(1,0){10}}
\put(35,0){2}
\put(45,0){2}
%2s2p
\put(0,12.5){L}
\put(10,10){2s}
\put(20,10){\line(1,0){10}}
\put(35,10){2}
\put(10,15){2p}
\put(20,15){\line(1,0){10}}
\put(35,15){6}
\put(45,15){10}
%3s3p
\put(0,27.5){M}
\put(10,25){3s}
\put(20,25){\line(1,0){10}}
\put(35,25){2}
\put(10,30){3p}
\put(20,30){\line(1,0){10}}
\put(35,30){6}
\put(45,30){18}

%4s3d4p
\put(0,45){N}
\put(10,40){4s}
\put(20,40){\line(1,0){10}}
\put(35,40){2}
\put(10,45){3d}
\put(20,45){\line(1,0){10}}
\put(35,45){10}
\put(10,50){4p}
\put(20,50){\line(1,0){10}}
\put(35,50){6}
\put(45,50){36}
%5s4d5p
\put(0,65){O}
\put(10,60){5s}
\put(20,60){\line(1,0){10}}
\put(35,60){2}
\put(10,65){3d}
\put(20,65){\line(1,0){10}}
\put(35,65){10}
\put(10,70){4p}
\put(20,70){\line(1,0){10}}
\put(35,70){6}
\put(45,70){54}
\put(0,-10){Skall}
\put(10,-10){Niv\aa\ }
\put(35,-10){$N_{e^-}$}
\put(45,-10){$\sum e^-$}
\end{picture}
\end{center}
%\end{center}
\end{minipage}
%
\vspace{0.1mm}

De mulige elektrontilstandene i et atom ut fra sentralfelt modellen
er vist i figur~5.2. Der vises
spektroskopisk notasjon, maksimalt  antall elektroner
($N_{e^-}$) i et skall og totalt antall elektroner opp til og med 
et gitt skall, ($\sum e^-$).
Vi har i figuren  en vertikal skala som angir energiforskjellen mellom
tilstandene karakterisert med kvantetallene  $(nl) $, og med 
$ (nl) = 1s$ som den laveste -  den sterkest
bundne tilstand.  Rekkef{\o}lgen av elektron tilstandene er noe
forskjellig fra det vi hadde for hydrog\'{e}n atomet
og skyldes at elektron frast{\o}tninger er inkludert.
Det vil v{\ae}re n{\o}dvendig med visse sm{\aa} modifikasjoner
i rekkef{\o}lgen for enkelte atomer, men i alminnelighet vil elektron
tilstandene v{\ae}re som vist p{\aa} figuren.

Legg merke til visse energi gap, omr{\aa}der hvor
avstanden  mellom to nabo tilstander er klart st{\o}rre enn gjennomsnittet.
Dette gir opphav til en skallstruktur
i atomene. Energi gap finner vi mellom $1s$ og $2s$, mellom $2p$
og $3s$ og mellom $3p$ og $4s$ osv. Energi tilstandene  mellom
to energi gap danner et {\sl \bf skall}. Hvert sett av $nl$
tilstander innen et skall kalles et {\bf \sl underskall}.
Antall elektroner som et underskall if{\o}lge
Pauliprinsippet kan inneholde, er vist p{\aa} figuren i kolonnen
$N_{e^-}$. Det er ogs{\aa} summert opp for hele skallet.

He atomet har 2 elektroner og fyller opp det innerste
$1s$ skallet. Den relativt store avstand til de neste
elektron tilstandene forklarer at He  er et spesielt  stabilt
grunnstoff. Samme fenomen opptrer ved de andre energi
gapene - Ne ved $Z = 10$, Ar ved $Z = 18$ osv..

Egenskapene til de forskjellige atomene er bestemt av elektron
konfigurasjonen og de {\bf \sl spektroskopiske kvantetallene}.
Vi skal n{\aa} definere hva det er og sette opp regler til {\aa} bestemme
dem for det enkelte atom.

La oss starte med et system av to elektroner.  Hvert
elektron har et banespinn $\vec{l}$ med tilh{\o}rende kvantetall $l$
og $m_l$.
For to-elektron systemet har vi et total banespinn,
gitt ved
%
\begin{equation}
\vec{L} = \vec{l_1} + \vec{l_2}.
\label{b1}
\end{equation}
%
Det tilh{\o}rende kvantetallet $L$ er bestemt ut fra regelen
%
\begin{equation}
L = |l_1 - l_2|, |l_1 - l_2|+1, |l_1 - l_2| +2,\ldots ,l_1 + l_2,
\label{b2}
\end{equation}
%
og med $M_L = -L,\ldots ,+L$ 
Legg merke til at dette er  samme regel som ble brukt til {\aa}
sette sammen $\vec{L}$ og $\vec{S}$ til $\vec{J}$ i forbindelse
med spinn-bane koblingen  i hydrog\'{e}n atomet.\\[2ex]
%
{\sl Eksempel:} For $l_1 = 1$ og $l_2 = 2$, f{\aa}r vi fra regelen
ovenfor $L = 1, 2, 3$.
Videre har hvert elektron et egenspinn $\vec{s}$ med kvantetall $s = 1/2$.
Dette gir et totalt egenspinn
%
\begin{equation}
\vec{S} = \vec{s}_1 + \vec{s}_2,
\label{b3}
\end{equation}
%
for to-elektron systemet med kvantetall $ S = 0$ eller $1$
etter samme regel som i lign.~(\ref{b2}).
For He med begge elektronene i $(n,l) = (1,0)$ betyr dette
at $l_1 = l_2 = 0$, og lign.~(\ref{b2}) gir $L = 0$.
For egenspinnet har vi derimot to muligheter, $S = 0$ eller $S = 1$.
Hvilken skal vi velge?  Her kommer Pauliprinsippet til hjelp og krever
at den totale b{\o}lgefunksjonen m{\aa} v{\ae}re antisymmetrisk. La oss se
n{\ae}rmere p{\aa} hva dette betyr.

B{\o}lgefunksjonen for ett elektron kan skrives som $\psi_{l,m_l,,m_s}$.
Kvantetallet $m_s$ har verdiene $\pm 1/2$. En alternativ  skrivem{\aa}te for
$m_s = + 1/2$ er $+$ og for $m_s = - 1/2$ $-$. 
Den totale b{\o}lgefunksjonen for to elektroner i He f{\aa}r formen
%
\begin{equation}
\psi(1, 2) = \sqrt{\frac{1}{2}}
	 \left (\psi_{1, 0, 0, +}(1) \psi_{1, 0, 0, -}(2)
			- \psi_{1, 0, 0, +}(2) \psi_{1, 0, 0, -}(1) \right ).
\label{b4}
\end{equation}
%
Romdelen av b{\o}lgefunksjonen for de to elektronene er symmetrisk,
mens egenspinndelen er antisymmetrisk. Det ser vi ved {\aa} omforme lign.~(\ref{b4})
til
%
\begin{equation}
\psi(1, 2) = \psi_{1, 0, 0}(1) \psi_{1, 0, 0}(2)
	         \sqrt{\frac{1}{2}} \left (| +, - > - | -, + > \right ).
\label{b5}
\end{equation}
For egenspinnet bruker vi betegnelsen $ | +, - >$ til {\aa} angi
at elektron nr.~1 har $m_s = + 1 / 2 $ (opp), og elektron nr.~2
har $m_s = - 1 / 2$ (ned). I tilstanden $ | -, + >$ er de byttet om.
Symmetri egenskapen til b{\o}lgefunksjonen $\psi( 1, 2 )$ kan vi
formulere slik
%
\begin{equation}
\psi( 1, 2 ) = \left ( \begin{array}{c}
					  romtilstand\\
					  symmetrisk
					  \end{array}
					  \right )
					  \times
					  \left ( \begin{array}{c}
					  spinntilstand\\
					  antisymmetrisk
					  \end{array}
					  \right ),
\label{b6}
\end{equation}
%
og den totale b{\o}lgefunksjonen er antisymmetrisk og tilfredsstiller
Pauliprinsippet.
%

To-elektron b{\o}lgefunksjonen for He i lign.~(\ref{b5}) tilfredsstiller
Pauliprinsippet og har totalt banespinn $ L = 0$. Neste sp{\o}rsm{\aa}l
blir da hvilket totalt egenspinn $S$  tilstanden har.
Her skal vi oppgi svaret uten {\aa} g{\aa} inn p{\aa} detaljene.
De er ikke vanskelige, men krever litt tid.

For en to-elektron egenspinn funksjon har vi
%
\begin{equation}
\chi_{S = 0}(1,2) = 	\sqrt{\frac{1}{2}} \left (| +, - > - | -, + > \right ).
\label{b7}
\end{equation}
%
Denne tilstanden har $S = 0$ og er antisymmetrisk. Det siste er lett
{\aa} se av formen. En annen mulighet er en egenspinn tilstand med
$S = 1$. Denne kommer i tre varianter avhengig av $M_S$.
%
\begin{equation}
\chi_{S = 1}(1,2) = \left \{
	\begin{array}{rll}
	&|+, + >           &M_S = +1\\
	\sqrt{\frac{1}{2}} & \left (| +, - > + | -, + > \right ).
							&M_S = \;\; 0\\
	&| -, - >          &M_S = -1
	\end{array}
	\right .
\label{b8}
\end{equation}
%
Disse tre tilstandene har $S = 1$ og er alle symmetriske,
hvilket ogs{\aa} tydelig fremg{\aa}r av formen.
Egenspinn tilstanden i lign.~(\ref{b7}) brukes i b{\o}lgefunksjonen i lign.~(\ref{b5}).
Men vi har ogs{\aa} situasjoner hvor romfunksjonen kan v{\ae}re antisymmetrisk.
Da bruker vi egenspinn funksjonen i lign.~(\ref{b8}).

For He atomet betyr dette at grunntilstanden har $L = 0$ og $S = 0$.
Men vi trenger ett kvantetall
til, det totale spinn som  i hydrog\'{e}n ble betegnet med
$J$. Den generelle definisjonen er
%
\begin{equation}
\vec{J} = \vec{L} + \vec{S}.
\label{b9}
\end{equation}
%
I tilfelle He betyr dette $J = 0$.

Et samlet symbol for disse kvantetallene er
%
\begin{equation}
^{2 S + 1}L_J,
\label{b10}
\end{equation}
%
og vi f{\aa}r for He: $^{2 S + 1}L_J$ = $ ^{1}S_0$.

La oss n{\aa} anvende denne teorien til {\aa} lage en eksitert tilstand
i He, dvs. en tilstand som har h{\o}yere energi - mindre
bindingsenergi enn He i grunntilstanden. Dette kan vi gj{\o}re
ved {\aa} flytte ett av elektronene fra $1s$ til
$2s$ tilstanden. B{\o}lgefunksjonen f{\aa} da formen
%
\begin{equation}
\psi(1, 2) = \sqrt{\frac{1}{2}}
\left ( R_{1s}(1) R_{2s}(2) + R_{1s}(2) R_{2s}(1) \right )
\chi_{S = 0}(1,2).
\label{b11}
\end{equation}
%
Romdelen av b{\o}lgefunksjonen er symmetrisk og egenspinndelen
er antisymmetrisk med $ S = 0$, se  lign.~(\ref{b7}). Spektroskopisk
betegnelse blir $^{1}S_0$. En annen mulighet er
 %
\begin{equation}
\psi(1, 2) = \sqrt{\frac{1}{2}}
\left ( R_{1s}(1) R_{2s}(2) - R_{1s}(2) R_{2s}(1) \right )
\chi_{S = 1}(1,2),
\label{b12}
\end{equation}
%
hvor romdelen er antisymmetrisk og egenspinndelen symmetrisk
med $S = 1$. Det gir spektroskopisk betegnelse
$^{3}S_1$. Eksperimentelt har He-tilstanden i lign.~(\ref{b12}) med $S = 1$
st{\o}rre bindingsenergi enn tilstanden i lign.~(\ref{b11}) med $S = 0$.
Dette kan forklares ved at i den antisymmetriske romdel av
b{\o}lgefunksjonen i lign.~(\ref{b12}) har de to elektronene i middel
st{\o}rre avstand enn i tilstand (\ref{b11}). Derved blir elektron
frast{\o}tningen mindre.
Legg merke til at for en symmetrisk romfunksjon kan de to elektronene
v{\ae}re p{\aa} samme sted i rommet, mens for den antisymmetriske
romfunksjonen er sannsynligheten for dette lik null.
%
\begin{figure}
%
\begin{center}
%
\begin{pspicture}(10,6)

%%%%%%%%%%

%%%     linje 1   %%%

\rput(0,4.5){
             \rput(0,0){
                \psframe(0,0)(0.5,0.5)
              }
              \multiput(0,0.5)(0.5,0){4}{ 
                 \psframe(0,0)(0.5,0.5)
              }
              \rput(0.2,1.1){s}
              \rput(1.2,1.1){p}
              \rput(-0.3,.2){K}
              \rput(-0.3,.7){L}
              \rput(1.2,.2){H}

              \psline{->}(0.25,0.05)(0.25,0.45)

} %%* end rput linje 1

%%%  linje 2   %%%

\rput(0,3){
          \multiput(0,0)(2.5,0){2}  {
             \rput(0,0){
                \psframe(0,0)(0.5,0.5)   
                 \psline{->}(0.15,0.05)(0.15,0.45)
                 \psline{<-}(0.35,0.05)(0.35,0.45)

                \rput(0.2,1.1){s}
                \rput(1.2,1.1){p}
             }
             \multiput(0,0.5)(0.5,0){4}{ 
                 \psframe(0,0)(0.5,0.5)
             }
          }
          \rput(-0.3,.2){K}
          \rput(-0.3,.7){L}
          \rput(1.2,.2){He}
          \rput(3.7,.2){Li}
          \psline{->}(2.75,0.55)(2.75,0.95)

}  %* end rput linje 2

%%%   linje 3  %%%%

\rput(0,1.5)  {
             \multiput(0,0)(2.5,0){4}  {
%%%%%%%%%%
                \rput(0,0){
                   \psframe(0,0)(0.5,0.5)
                   \psline{->}(0.15,0.05)(0.15,0.45)
                   \psline{<-}(0.35,0.05)(0.35,0.45) 
                }  
                \multiput(0,0.5)(0.5,0){4}  {
                   \psframe(0,0)(0.5,0.5)
                }
                \rput(0.2,1.1){s}
                \rput(1.2,1.1){p}
                \psline{->}(0.15,0.55)(0.15,0.95)
                \psline{<-}(0.35,0.55)(0.35,0.95)
             }
             \rput(-0.3,0.2){K}  
             \rput(-0.3,0.7){L}
             \rput(1.2,0.2){Be}     
             \rput(3.7,0.2){B}   
             \rput(6.2,0.2){C}
             \rput(8.7,0.2){N}
             \multiput(2.5,0.55)(2.5,0){3} {
               \psline{->}(0.75,0.05)(0.75,0.45)
             }
             \multiput(5,0.55)(2.5,0){2} {
                 \psline{->}(1.25,0.05)(1.25,0.45)
              }
              \psline{->}(9.25,0.55)(9.25,0.95)

}    %% end linje 3 

%%%   linje 4   %%%
\rput(0,0)  {
             \multiput(0,0)(2.5,0){3} {
%%%%%%%%%%
                \rput(0,0){
                   \psframe(0,0)(0.5,0.5)
                   \psline{->}(0.15,0.05)(0.15,0.45)
                   \psline{<-}(0.35,0.05)(0.35,0.45)
                }
                \multiput(0,0.5)(0.5,0){4}  {
                   \psframe(0,0)(0.5,0.5)
                }
                \psline{->}(0.15,0.55)(0.15,0.95)
                \psline{<-}(0.35,0.55)(0.35,0.95)
                \psline{->}(0.65,0.55)(0.65,0.95)
                \psline{<-}(0.85,0.55)(0.85,0.95)

                \rput(0.2,1.1){s}
                \rput(1.2,1.1){p}
             }

             \psline{->}(1.25,0.55)(1.25,0.95)
             \psline{->}(1.75,0.55)(1.75,0.95)

             \multiput(2.5,0.55)(2.5,0){2}  {
                \psline{->}(1.15,0)(1.15,0.45)
                \psline{<-}(1.35,0)(1.35,0.45)
             }
             \psline{->}(4.25,0.55)(4.25,0.95)
             \psline{->}(6.65,0.55)(6.65,0.95)
             \psline{<-}(6.85,0.55)(6.85,0.95)

             \rput(-0.3,0.2){K}
             \rput(-0.3,0.7){L}
             \rput(1.2,0.2){O}
             \rput(3.7,0.2){F}
             \rput(6.2,0.2){Ne}
 }   %% end linje 4 




\end{pspicture}
%
\end{center}
\caption{Elektron konfigurasjonene for de 10 f{\o}rste grunnstoffene.}
\label{elek}
\end{figure}
%

Vi skal n{\aa} bruke teorien p{\aa} systemer med flere elektroner. Med
utgangspunkt i de regler vi hittil har presentert kan vi sette opp
elektron konfigurasjonen til de 10 f{\o}rste grunnstoffene.
Dette er vist i figur~\ref{elek}.
Symbolet $\uparrow \downarrow$ i figuren betyr at de to elektronene
er i en antisymmetrisk spinn tilstand med $S = 0$.
For de fem f{\o}rste grunnstoffen er
de spektroskopiske kvantetallene:
%
\begin{center}
H: $^2S_{1/2}$\hspace{0.5cm} He: $^1S_0$\hspace{0.5cm} Li: $^2S_{1/2}$
\hspace{0.5cm} Be: $^1S_0$\hspace{0.5cm} B: $^2P_J,
J = 1/2 \;\;eller\;\; 3/2$
\end{center}
%
St{\o}rre problemer blir det med de fem siste grunnstoffene.
Her trenger vi noen flere regler:
%
\begin{enumerate}
%
\item Hvis alle tilstandene i et underskall inneholder elektroner,
$(nl)^{2 (2 l + 1)}$, danner  det et {\sl lukket skall}.
En slik konfigurasjon har totalt $L = 0$ og $S = 0$.\\
Eksempel: Underskallet $2p$ i Ne inneholder
$2 \cdot (2 \cdot 1 + 1 ) = 6$ elektroner if{\o}lge figur~\ref{elek}.
Dette er det maksimalt tillatte  antall etter
Pauliprinsippet, og  konfigurasjonen har $L = S = 0$.
Det samme argumentet kan vi bruke p{\aa} skallet $1s$
og underskallet $2s$. Til sammen vil derfor de 10 elektronene i Ne
ha $L = S = J = 0$, dvs. $^1S_0$.
%
\item Grunnstoffet N har if{\o}lge figur~\ref{elek} 3 elektroner i $2p$
og spinnfunksjonen er $|+, +, +>$. Alle spinnvektorene er rettet
samme vei, og vi f{\aa}r $M_S = 1/2 + 1/2 + 1/2 = 3/2$ med $S = 3/2$
siden $ S \geq M_S$. Dette er et eksempel p{\aa}
{\sl \bf Hunds regel}: {\sl Elektronene fyller et underskall slik at
$S$ blir st{\o}rst mulig n{\aa}r vi tar hensyn til at
Pauliprinsippet skal v{\ae}re oppfylt.}
Forklaringen er f{\o}lgende:
Vi vet fra He atomet at vi f{\aa}r st{\o}rst
bindingsenergi  ved {\aa} minimalisere  elektron
frast{\o}tningen. Det oppn{\aa}r vi ved {\aa} antisymmetrisere
romdel av b{\o}lgefunksjonen, og etter Pauliprinsippet
betyr det {\aa} gj{\o}re  spinnfunksjonen mest mulig symmetrisk.
Dette er Hund's regel og vi ser den anvendt i figur~\ref{elek} b{\aa}de
for  C, $Z = 6$, N, $Z = 7$, O, $Z=8$ og F, $Z = 9$.
%
\item En regel til er viktig og kommer til anvendelse i tilfelle
F med $Z = 9$. Fra figur~\ref{elek} ser vi at $2p$ underskallet
inneholder 5 elektroner. Adderer vi til ett elektron, f{\aa}r
vi et lukket $2p$ underskall. Det gir $L = S = 0$.
Vi kan da skrive
%
\begin{equation}
\vec{L} = \vec{L}_1 + \vec{l},
\label{b13}
\end{equation}
%
hvor $L$ er banespinnet for $2p^6$, $L_1$ er banespinnet
for $2p^5$ og $l$ er banespinnet for et elektron i $2p$
skallet. De tilh{\o}rende kvantetallene blir $L = 0$, $l = 1$
og $L_1$. Bruker vi regelen i lign.~(\ref{b2}), f{\aa}r vi som eneste
mulighet $L_1 = 1$. Tilsvarende analyse kan vi gjennomf{\o}re
for egenspinnet med svar $S_1 = 1/2$.\\
Konklusjon: Har vi et underskall $nl$,
vil elektron konfigurasjonen
$(nl)^{2 (2 l + 1) - 1}$ ha samme banespinn og egenspinn som et enkelt
elektron i samme underskall.\\
Denne symmetrien betyr at $F, Z = 9$ og $B, Z = 5$ har samme
spektroskopiske kvantetall - $^2P_J$ med mulighetene
$J = 1/2 \;\;eller\;\; 3/2$.
%
\end{enumerate}
%
For en fullstendig beskrivelse av de 10 f{\o}rste grunnstoffene
i figur~\ref{elek} trenges enn{\aa} litt mer teori. Vi f{\aa}r ikke fastlagt
banespinnet for C, N og O n{\o}yaktig. Men det vil vi overlate til
videre studium i kvantefysikken. Vi kan likevel bruke teorien v{\aa}r
til {\aa} gi en god beskrivelse av mange grunnstoffer
i det periodiske system, ikke bare de som er angitt i figur~\ref{elek}.

\section{Zeeman effekten for atomer}
Se avsnitt 8-11 i boka. Erstatt det  
totale spinn for en tilstand
i hydrogenatomet med det tilsvarende
totale spinn for det aktuelle atomet. Se ogs\aa\ oppgave 3.8 for en anvendelse


\section{Lasere}


