\subsection*{Kursets struktur}

Kurset er delt inn i tre hoveddeler:
\begin{itemize}
   \item \underline{F\o rste del} tar for seg den historiske utviklingen fra
         slutten av det nittende \aa rhundre til begynnelsen av
         forrige \aa rhundre. I denne tidsperioden vokste
         erkjennelsen av at klassisk fysikk (Newtons lover m.m.)
         ikke kunne beskrive resultater fra flere nye eksperimenter, bla.~flerespektroskopiske data. 
          Denne utviklingen f\o rte 
         fram til den nye kvanteteorien i 1925. De nye begrepene
         som ble innf\o rt var bla.~{\bf materieegenskapen til str\aa ling}, 
         {\bf b\o lgegenskapene til materien} og {\bf kvantiseringen av
              fysiske st\o rrelser som f.eks.~energien eller banespinnet}.
    \item \underline{Andre del} tar for seg en f\o rste introduksjon til 
          kvantemekanikk, med hovedvekt p\aa\ b\o lgemekanikk
          og en-partikkel problemer. Denne delen avsluttes
          med en kvantemekanisk beskrivelse av hydrogenatomet.
    \item \underline{Tredje og siste del} Undervises resten av 
          semesteret. Her tar vi for oss ulike andvendelser  
          fra kvantemekanikkens spede begynnelse med
          atomfysikk, til kjernefysikk, moderne partikkelfysikk
          og faste stoffers fysikk. Litt om kvantedatamaskiner.
\end{itemize}  
\subsection*{Kursets innhold}

\begin{itemize}
   \item \underline{F\o rste del}.
        \begin{itemize}
           \item Enheter og st\o rrelser i FYS2140
           \item Fordelingsfunksjoner
           \item Svart legeme str\aa ling og Plancks kvantiseringshypotese
           \item Fotoelektrisk effekt
           \item R\"ontgenstr\aa ling
           \item Comptonspredning
           \item Bohrs atommodell
           \item Materieb\o lger og partikkel-b\o lge dualitet
           \item Heisenbergs uskarphetsprinsipp
           \item Litt b\o lgel\ae re
           \item Schr\"odingers katt paradokset og Entanglement
         \end{itemize}
      \item \underline{Andre del}.
        \begin{itemize}
           \item Introduksjon til kvantemekanikk og enkle
                 kvantemekaniske systemer
           \item Kvantisering av banespinn
           \item Hydrogen atomet
        \end{itemize}
      \item \underline{Tredje del}.
        \begin{itemize}
           \item Atomfysikk, det periodiske systemet
           \item Molekyler
           \item Kvantestatistikk og lasere
           \item Kvantecomputere
           \item Faste stoffers fysikk, Bose-Einstein kondensasjon 
                 og supraledning.
           \item Kjernefysikk, strukturen til kjerner, str\aa ling,
                 dannelsen av elementene, fisjon og fusjon, kjernekrefter
           \item Moderne partikkelfysikk, kvarker og leptoner
         \end{itemize}
      \end{itemize}





Til slutt en takk til alle som har kommet med kommentarer, trykkfeil
m.m.~om de ulike utkastene som har florert p\aa\ kursets hjemmeside.
I tillegg er vi sv\ae rt takknemlige for hjelpen Simen Kvaal ga oss i
utviklingen av flere av de numeriske oppgavene.















