
%  Tex-fil for eksamensoppgaver i FYS 113
%
\documentclass[11pt]{article}
%
\usepackage[T1]{fontenc}       % DC-fonter
\usepackage{epsfig}
\usepackage{a4}
\usepackage{norsk}             % Norske ord
\renewcommand{\rmdefault}{ptm} % Times
%
%%
\addtolength\textwidth{1cm}
%	\addtolength\evensidemargin{-1cm}
%	\addtolength\oddsidemargin{1cm}

\listfiles

% Spesielle kommandoer
\newcommand{\OP}[1]{{\bf\widehat{#1}}}
%
\pagestyle{myheadings}
%

\begin{document}

\newpage

%
\begin{center}
\LARGE \bf UNIVERSITETET I OSLO
\end{center}
%
{\Large Det matematisk-naturvitenskapelige fakultet}\\[2ex]
%
Eksamen i FYS-113: Kvantefysikk\\
Eksamensdag: Tirsdag 20 juni 2000\\
Tid for eksamen: 0900 - 1500\\
Eksamenssettet  best\aa r av 4 sider, kontroll\'{e}r at du har riktig antall.\\
\begin{minipage}[t]{0.3\textwidth}
Tillatte hjelpemidler:
\end{minipage}
\hspace*{0.1\textwidth}
\begin{minipage}[t]{0.6\textwidth}
{\O}grim: St{\o}rrelser og enheter i fysikken\\
Rottmann: Matematische Formelsamlung\\
To A4 ark med egne notater\\
Godkjent numerisk elektronisk kalkulator
\end{minipage}



\noindent {\Large \bf Oppgave 1.}\\[1ex]
%
\begin{itemize}
\item[a)]  Gj\o r kort rede for den fotoelektriske effekten og skiss\'{e}r
en eksperimentell oppstilling som kan observere og m\aa le
denne effekten. 
\item[b)] Et fotoelektrisk eksperiment gj\o res med
kalsium som metall. F\o lgende stoppepotensialer 
$V_s$ m\aa les
\begin{table}[h]
\begin{center} 
\begin{tabular}{|l|llll|} \hline
B\o lgeldengde $\lambda$ [nm] & 253,6 & 313,2 & 365,0 & 404,7 \\ \hline 
Stoppepotensial $V_s$ [V] & 1,95 & 0,98 & 0,50 & 0,14 \\ \hline
\end{tabular}
\end{center}
\end{table}

B\o lgelengden for den elektromagnetiske str\aa ling er gitt
ved $\lambda$. 
Bruk tallene
i tabellen til \aa\ finne Plancks konstant. 
Du kan f\aa\ bruk for f\o lgende konstanter; 
$c=2,997925 \times 10^{8}$ m s$^{-1}$  og 
$h= 4,135669 \times 10^{-15}$eV s.


\item[c)] Vi skal n\aa\ studere Compton spredning. 
I den anledning trenger vi 
det relativistiske uttrykket for energi av en partikkel
%
\[
E = mc^2= \sqrt{E_0^2 + (pc)^2}.
\]
%
Som energi enhet bruk $eV$ og $eV/c$ som enhet for bevegelsesmengde.
%
 Gj{\o}r kort rede for de st{\o}rrelser som inng{\aa}r i relasjonen
ovenfor.
%
%\item[d)] Anvend relasjonen ovenfor p{\aa} et foton og finn fotonets
%energi og bevegelsesmengde n{\aa}r b{\o}lgelengden er
%$500 \; nm = 5,00 \times 10^{-7}\; m$.
%

%
\item[d)] Et foton med energi $E = h \nu_0 = h c / \lambda_0$ spres en vinkel
$\theta$ ved Comptoneffekt
mot et elektron som antas {\aa} ligge i ro f{\o}r spredning.
Tegn en prinsippskisse for et Compton eksperiment
og angi spredningsvinklene for fotonet og elektronet etter
spredningen.

Formul\'{e}r de to prinsippene som anvendes til {\aa} beregne
det spredte fotonets b{\o}lgelengde $\lambda^{'}$ og 
til {\aa} utlede Comptons formel (skal ikke utledes).

Vis deretter at et fritt elektron i ro ikke kan absorbere
et foton. 

\item[e)]  Bruk det siste resultatet til \aa\ gj\o re rede
for forskjellen mellom Compton spredning og fotoelektrisk
effekt.
\end{itemize}
%



\noindent
{\Large \bf Oppgave 2.}\\[1ex]
%

Vi skal i denne oppgaven studere det kvantemekaniske problemet for et
elektron i et en-dimensjonalt potensial
$V(x)$ gitt ved
\[
  V(x)=\left\{\begin{array} {cc} -k/x & 0 \leq x < \infty \\
                            0& \mathrm{ellers}\end{array} \right.
\] 
Vi skal bruke enhet $eV$ for energi og $nm$ for lengde. 
Massen til elektronet er $m_e$. $k$ er en reell konstant.
\begin{enumerate}
\item[a)]
Hamilton operatoren 
$\hat{H}$ kan skrives som 
\begin{displaymath}
\hat{H} = -\frac{\hbar^2}{2m}\frac{d^2}{dx^2} -\frac{k}{x}.
\end{displaymath}
Hvilken fysisk st\o rrelse representerer Hamilton operatoren?
Hvilken dimensjon og enhet m\aa\ konstanten $k$ ha?
Sett opp den tidsuavhengige Schr\"{o}dingerlikningen for
dette systemet med egenfunksjon $\psi_i(x)$.   

\end{enumerate}

Grunntilstanden for systemet har egenfunksjonen
\[
   \psi _0(x) = N xe^{-\alpha x},
\]
der $\alpha$ er en reell konstant og $N$ er en normeringskonstant.
Egenfunksjonen er null for $x < 0$. 
\begin{enumerate}
\item[b)] Beregn normeringskonstanten $N$ uttrykt ved $\alpha$.

F\o lgende integral kan v\ae re nyttig
\[
\int_{0}^{\infty}x^ne^{-x}dx = n! 
\]

 
\item[c)] Vis at $\psi _0(x)$ er en egenfunksjon for $\hat{H}$ 
og finn den tilh\o rende energiegenverdien for grunntilstanden
$E_0$ som funksjon av $\alpha$. Finn ogs\aa\ konstanten 
$k$ som funksjon av $\alpha$.  

\item[d)] Vis at forventningsverdien i grunntilstanden for
$\langle \hat{p}^2 \rangle=-2m_eE_0$, 
hvor $E_0$ er grunntilstandsenergien
du fant i forrige punkt. 

\item[e)] Vis deretter at
\[
\langle \hat{x} \rangle=\frac{3}{2\alpha},
\]
og at
$\langle \hat{p} \rangle=0$.
\end{enumerate}

Uskarpheten for den fysiske st\o rrelsen $A$ er gitt ved
\[
   \Delta A=\sqrt{\langle \hat{A}^2\rangle - \langle \hat{A}\rangle^2}.
\]
\begin{enumerate}
\item[f)] Rekn ut b\aa de $\Delta p$ og $\Delta x$. Formuler Heisenbergs
uskarphetsrelasjon og gi en fysisk forklaring p\aa\ de st\o rrelsene
som inng\aa r. Sjekk at resultatene for   $\Delta p$ og $\Delta x$
er i samsvar med Heisenbergs
uskarphetsrelasjon.   
\end{enumerate}

\noindent
\noindent{\Large \bf Oppgave 3.}\\[1ex]
\noindent
I den enkle skallmodellen for fler-elektron atomer plasseres elektronene i
tilstander som har de samme kvantetallene som energi egentilstandene for
Hydrogenatomet:\\
$1s; 2s, 2p; 3s, 3p, 3d;\ldots ;nl = n0, nl = n1, \ldots, nl = n,n-1$.
\begin{itemize}
%
\item[a)] Hvor mange elektroner kan det plasseres i hver av tilstandene
i hovedskallene 
$1s; 2s, 2p; 3s, 3p, 3d;\ldots ;$ $ nl = n0, \ldots , nl = n,n-1$?
Skriv ned det prinsippet som bestemmer hvor mange elektroner
det kan v\ae re i hver tilstand.
%
\end{itemize}

\begin{minipage}{0.45\textwidth}
%
I H-atomet har tilstandene $nl = n0, nl = n1\ldots , nl = n,n-1$ alle
samme energi. De er degenererte. I et fler-elektron
atom, er dette ikke lenger tilfelle. Figuren 
viser $nl$~tilstandene for sentralfeltmodellen.

\begin{itemize}
% 
\item[b)] Diskuter �rsaken til at tilstandene i
figuren ved siden av 
er forskjellige fra Hydrogenatomets tilstander.
%
\end{itemize}

%

\end{minipage}
%
\hspace{0.05\textwidth}
%
\begin{minipage}{0.50\textwidth}
%
\begin{center}
\setlength{\unitlength}{1.0mm}
\begin{picture}(50,85)(0,-15)
\thicklines
%1s
\put(10,0){1s}
\put(20,0){\line(1,0){10}}
%2s2p
\put(10,10){2s}
\put(20,10){\line(1,0){10}}
\put(10,15){2p}
\put(20,15){\line(1,0){10}}
%3s3p
\put(10,25){3s}
\put(20,25){\line(1,0){10}}
\put(10,30){3p}
\put(20,30){\line(1,0){10}}

%4s3d4p
\put(10,40){4s}
\put(20,40){\line(1,0){10}}
\put(10,45){4p}
\put(20,45){\line(1,0){10}}
\put(10,50){3d}
\put(20,50){\line(1,0){10}}
\put(0,-10){Niv\aa\ i sentralfelt modellen}
\end{picture}
\end{center}
%\end{center}
\end{minipage}
%
\vspace{0.1mm}


\noindent
I denne oppgaven skal vi studere Kaliumatomet som har 19 elektroner.
% 
\begin{itemize}
% 
\item[c)] Bruk figuren til � bestemme elektron
konfigurasjonen i grunntilstanden for Kaliumatomet. Finn verdiene for 
kvantetallet (S) for det totale egenspinn, kvantetallet (L) for
det totale banespinn og kvantetallet (J) for det totale spinn. 
%
\end{itemize}
Energien til den f\o rste eksiterte tilstanden er bestemt eksperimentelt til
1,617~eV. En n\ae rmere analyse viser at denne tilstanden er splittet
i to. Energi oppsplittingen for denne dubletten skyldes spinn-bane
koplingen.

\begin{itemize}
\item[d)]
Bruk figuren til \aa\ bestemme elektron konfigurasjonen for den f\o rste
eksiterte dublett tilstanden. Finn deretter kvantetallene S, L
og J for de to tilstandene i denne dubletten. 
\end{itemize}
% 
 Tilleggsenergien p\aa\ grunn av spinn-bane koplingen  kan skrives 
% 
\[
\OP{H}_{SL} = a \OP{S}\cdot \OP{L}
\]
hvor $a$ er en konstant.
% 
\begin{itemize}
% 
\item[e)] Finn energi differansen mellom tilstandene i dubletten n�r $a$
har en verdi gitt ved $a\hbar^2 = 0,0075$~eV. 
% 
\end{itemize}

%
Dersom Kaliumatomet er i en av de to dublett tilstandene
vil
det g\aa ~over til grunntilstanden og emittere et foton.
%
\begin{itemize}
%
\item[g)] Hvor stor b\o lgelengdeoppl\o sning
m\aa ~en minst ha for \aa ~skille
spektrallinjene fra de to overgangene
 i et spektrometer?
%
\end{itemize}
%

N\aa r vi tar hensyn til egenspinn-banespinn kobling (spinn-banekobling) er
operatoren $\OP{M}$ for det magnetiske dipol
momentet gitt ved

\[
\OP{M} = -\frac{e}{2m_e} g \OP{J},
\]

der $m_e$ er elektronets hvilemasse,
$\OP{J}$ er operatoren for det
totale spinnet og

\[
g = 1 + \frac{j(j+1) + s(s+1) -l(l+1)}{2j(j+1)}.
\]

\begin{itemize}

\item[h)] Bestem verdien av $g$ for grunntilstanden og de to f�rste
eksiterte tilstandene.

\end{itemize}






Dersom Kaliumatomet plasseres i et svakt og homogent ytre magnetfelt
$\vec{B}$ vil
vekselvirkningen mellom atomets magnetiske dipolmoment $\OP{M}$
og magnetfeltet
$\vec{B}$ gi opphav til et ekstra ledd $\OP{H}^{'}$ i
Hamiltonoperatoren
%
\[
\OP{H}^{'} = - \OP{M} \cdot \vec{B}.
\]
%
$\OP{H}^{'}$ vil gi opphav til en splitting av energiniv\aa ene.
%
\begin{itemize}
%
\item[i)] Beregn, uttrykt ved $\mu_{B} = e \hbar / 2 m$ og feltstyrken B,
energi oppsplittingen av grunntilstanden og de to f�rste eksiterte
tilstandene og deres degenerasjonsgrad.
%
\end{itemize}
%
Utvalgsreglene for elektromagnetiske dipoloverganger er $\Delta j = 0, \pm 1$;
$\Delta m_{j} = 0, \pm 1$; \mbox{$\Delta l = \pm 1$}.
%
\begin{itemize}
%
\item[j)] Tegn figurer som viser hvordan overgangene
fra de to f�rste eksiterte tilstanden til grunntilstandene blir 
og hvordan de tilsvarende spektrallinjene blir modifisert av 
magnetfeltet $\vec{B}$.
%
\end{itemize}
%














\end{document}




