\documentclass[graybox,envcountchap,sectrefs]{svmono}

\usepackage{mathptmx}
\usepackage{helvet}
\usepackage{courier}

\usepackage{type1cm}         
\usepackage{exercise}
\usepackage{makeidx}         % allows index generation
\usepackage{graphicx}        % standard LaTeX graphics tool
                             % when including figure files
\usepackage{multicol}        % used for the two-column index
\usepackage[bottom]{footmisc}% places footnotes at page bottom
%\usepackage{graybox}
\usepackage[usenames,dvipsnames,x11names]{xcolor}
\usepackage{tikz}
\usetikzlibrary{arrows,snakes,shapes}

 \usepackage{listings}
 \usepackage{graphicx}
 \usepackage{epic}
 \usepackage{eepic}
 \usepackage{a4wide}
 \usepackage{color}
 \usepackage{amsmath}
 \usepackage{amssymb}
% \usepackage[dvips]{epsfig}
% \usepackage{psfig}
 \usepackage[T1]{fontenc}
 \usepackage{cite} % [2,3,4] --> [2--4]
 \usepackage{shadow}
 \usepackage{hyperref}
 \usepackage{bezier}
 \usepackage{pstricks}
% \usepackage{refcheck}
\setcounter{tocdepth}{2}
%\usepackage{gnuplot-lua-tikz}


\usepackage{textcomp,type1ec,pdfpages}
\usepackage{bera}

\definecolor{dkgreen}{rgb}{0,0.6,0}
\definecolor{gray}{rgb}{0.5,0.5,0.5}
\definecolor{mauve}{rgb}{0.58,0,0.82}

 \lstset{language=c++}
 \lstset{alsolanguage=[90]Fortran}
 \lstset{alsolanguage=python}
% \lstset{basicstyle=\small}
 \lstset{backgroundcolor=\color{white}}
 \lstset{frame=single}
 \lstset{stringstyle=\ttfamily}
 \lstset{keywordstyle=\color{red}\bfseries}
 \lstset{commentstyle=\itshape\color{blue}}
 \lstset{showspaces=false}
 \lstset{showstringspaces=false}
 \lstset{showtabs=false}
 \lstset{breaklines}
 

% Default settings for code listings
% \lstnewenvironment{Python}[1]{
\lstset{%frame=tb,
  language=c++,
  alsolanguage=python,
  %aboveskip=3mm,
 % belowskip=3mm,
  showstringspaces=false,
  columns=flexible,
  basicstyle={\footnotesize\ttfamily},
  numbers=none,
  numberstyle=\tiny\color{gray},
  commentstyle=\color{dkgreen},
  stringstyle=\color{mauve},
  frame=single,  
  breaklines=true,
  %%%% FOR PYTHON 
  otherkeywords={\ , \}, \{},
  keywordstyle=\color{blue},
  emph={void, ||, &&, break, class,continue, delete, else,
  for, if, include, return,try,while},
  emphstyle=\color{black}\bfseries,
  emph={[2]True, False, None, self},
  emphstyle=[2]\color{dkgreen},
  emphstyle=[2]\color{red},
  emph={[3]from, import, as},
  emphstyle=[3]\color{blue},
  upquote=true,
  morecomment=[s]{"""}{"""},
  commentstyle=\color{green}\slshape, %%% cambie gray por green
  emph={[4]1, 2, 3, 4, 5, 6, 7, 8, 9, 0},
  emphstyle=[4]\color{blue},
  breakatwhitespace=true,
  tabsize=2
}

\renewcommand{\lstlistlistingname}{Code Listings}
\renewcommand{\lstlistingname}{Code Listing}
\definecolor{gray}{gray}{0.5}
\definecolor{green}{rgb}{0,0.5,0}

\lstnewenvironment{Python}[1]{
\lstset{
language=python,
basicstyle=\footnotesize\setstretch{1},
stringstyle=\color{red},
showstringspaces=false,
alsoletter={1234567890},
otherkeywords={\ , \}, \{},
keywordstyle=\color{blue},
emph={access,and,break,class,continue,def,del,elif ,else,%
except,exec,finally,for,from,global,if,import,in,is,%
lambda,not,or,pass,print,raise,return,try,while},
emphstyle=\color{black}\bfseries,
emph={[2]True, False, None, self},
emphstyle=[2]\color{red},
emph={[3]from, import, as},
emphstyle=[3]\color{blue},
upquote=true,
morecomment=[s]{"""}{"""},
commentstyle=\color{dkgreen}\slshape, % el color era gray pero lo cambie a verde
emph={[4]1, 2, 3, 4, 5, 6, 7, 8, 9, 0},
emphstyle=[4]\color{blue},
framexleftmargin=1mm, framextopmargin=1mm, rulesepcolor=\color{blue},
breakatwhitespace=true,
tabsize=2
}}{}


\lstnewenvironment{C++}[1]{
\lstset{
language=c++,
% basicstyle=\ttfamily\small\setstretch{1},
basicstyle=\footnotesize\setstretch{1},
stringstyle=\color{red},
showstringspaces=false,
alsoletter={1234567890},
otherkeywords={\ , \}, \{},
keywordstyle=\color{blue},
emph={access,and,break,class,continue,def,del,elif ,else,%
except,exec,finally,for,from,global,if,import,in,is,%
lambda,not,or,pass,print,raise,return,try,while},
emphstyle=\color{black}\bfseries,
emph={[2]True, False, None, self},
emphstyle=[2]\color{red},
emph={[3]from, import, as},
emphstyle=[3]\color{blue},
upquote=true,
morecomment=[s]{"""}{"""},
commentstyle=\color{dkgreen}\slshape, % el color era gray pero lo cambie a verde
emph={[4]1, 2, 3, 4, 5, 6, 7, 8, 9, 0},
emphstyle=[4]\color{blue},
% literate=*{:}{{\textcolor{blue}:}}{1}%
% {=}{{\textcolor{blue}=}}{1}%
% {-}{{\textcolor{blue}-}}{1}%
% {+}{{\textcolor{blue}+}}{1}%
% {*}{{\textcolor{blue}*}}{1}%
% {!}{{\textcolor{blue}!}}{1}%
% {(}{{\textcolor{blue}(}}{1}%
% {)}{{\textcolor{blue})}}{1}%
% {[}{{\textcolor{blue}[}}{1}%
% {]}{{\textcolor{blue}]}}{1}%
% {<}{{\textcolor{blue}<}}{1}%
% {>}{{\textcolor{blue}>}}{1},%
framexleftmargin=1mm, framextopmargin=1mm, rulesepcolor=\color{blue},
breakatwhitespace=true,
tabsize=2
}}{}




\usepackage{tikz}
\usetikzlibrary{shapes,arrows}

% Define block styles
\tikzstyle{decision} = [diamond, draw, fill=blue!20,
    text width=3.5em, text badly centered, node distance=2.5cm, inner sep=0pt]
\tikzstyle{block} = [rectangle, draw, fill=blue!20,
    text width=8em, text centered, rounded corners, minimum height=4em]
\tikzstyle{line} = [draw, very thick, color=black!50, -latex']
\tikzstyle{cloud} = [draw, ellipse,fill=red!20, node distance=2.5cm,
    minimum height=2em]

\def\radius{.7mm} 
\tikzstyle{branch}=[fill,shape=circle,minimum size=3pt,inner sep=0pt]


\newcommand{\bfv}[1]{\boldsymbol{#1}} 
\newcommand{\Div}[1]{\nabla \bullet \vbf{#1}}           % define divergence
\newcommand{\Grad}[1]{\boldsymbol{\nabla}{#1}}
 \newcommand{\OP}[1]{{\bf\widehat{#1}}}
 \newcommand{\be}{\begin{equation}}
 \newcommand{\ee}{\end{equation}}
\newcommand{\beN}{\begin{equation*}}
\newcommand{\bea}{\begin{eqnarray}}
\newcommand{\beaN}{\begin{eqnarray*}}
\newcommand{\eeN}{\end{equation*}}
\newcommand{\eea}{\end{eqnarray}}
\newcommand{\eeaN}{\end{eqnarray*}}
\newcommand{\bdm}{\begin{displaymath}}
\newcommand{\edm}{\end{displaymath}}
\newcommand{\bsubeqs}{\begin{subequations}}
\newcommand{\esubeqs}{\end{subequations}}
\newcommand{\Obs}[1]{\langle{\Op{#1}\rangle}}             % define observable
\newcommand{\PsiT}{\bfv{\Psi_T}(\bfv{R})}                       % symbol for trial wave function
%\newcommand{\braket}[2]{\langle{#1}|\Op{#2}|{#1}\rangle}
\newcommand{\Det}[1]{{|\bfv{#1}|}}
\newcommand{\uvec}[1]{\mbox{\boldmath$\hat{#1}$\unboldmath}}
\newcommand{\Op}[1]{{\bf\widehat{#1}}}    
\newcommand{\eqbrace}[4]{\left\{
\begin{array}{ll}
#1 & #2 \\[0.5cm]
#3 & #4
\end{array}\right.}
\newcommand{\eqbraced}[4]{\left\{
\begin{array}{ll}
#1 & #2 \\[0.5cm]
#3 & #4
\end{array}\right\}}
\newcommand{\eqbracetriple}[6]{\left\{
\begin{array}{ll}
#1 & #2 \\
#3 & #4 \\
#5 & #6
\end{array}\right.}
\newcommand{\eqbracedtriple}[6]{\left\{
\begin{array}{ll}
#1 & #2 \\
#3 & #4 \\
#5 & #6
\end{array}\right\}}

\newcommand{\mybox}[3]{\mbox{\makebox[#1][#2]{$#3$}}}
\newcommand{\myframedbox}[3]{\mbox{\framebox[#1][#2]{$#3$}}}

%% Infinitesimal (and double infinitesimal), useful at end of integrals
%\newcommand{\ud}[1]{\mathrm d#1}
\newcommand{\ud}[1]{d#1}
\newcommand{\udd}[1]{d^2\!#1}

%% Operators, algebraic matrices, algebraic vectors

%% Operator (hat, bold or bold symbol, whichever you like best):
\newcommand{\op}[1]{\widehat{#1}}
%\newcommand{\op}[1]{\mathbf{#1}}
%\newcommand{\op}[1]{\boldsymbol{#1}}

%% Vector:
\renewcommand{\vec}[1]{\boldsymbol{#1}}

%% Matrix symbol:
\newcommand{\matr}[1]{\boldsymbol{#1}}
%\newcommand{\bb}[1]{\mathbb{#1}}

%% Determinant symbol:
\renewcommand{\det}[1]{|#1|}

%% Means (expectation values) of varius sizes
\newcommand{\mean}[1]{\langle #1 \rangle}
\newcommand{\meanb}[1]{\big\langle #1 \big\rangle}
\newcommand{\meanbb}[1]{\Big\langle #1 \Big\rangle}
\newcommand{\meanbbb}[1]{\bigg\langle #1 \bigg\rangle}
\newcommand{\meanbbbb}[1]{\Bigg\langle #1 \Bigg\rangle}


%% Big-O (typically for specifying the speed scaling of an algorithm)
\newcommand{\bigO}{\mathcal{O}}

%% Real value of a complex number
%\newcommand{\real}[1]{\mathrm{Re}\!\left\{#1\right\}}

%% Quantum mechanical state vectors and matrix elements (of different sizes)
\newcommand{\brab}[1]{\big\langle #1 \big|}
\newcommand{\brabb}[1]{\Big\langle #1 \Big|}
\newcommand{\brabbb}[1]{\bigg\langle #1 \bigg|}
\newcommand{\brabbbb}[1]{\Bigg\langle #1 \Bigg|}
\newcommand{\ketb}[1]{\big| #1 \big\rangle}
\newcommand{\ketbb}[1]{\Big| #1 \Big\rangle}
\newcommand{\ketbbb}[1]{\bigg| #1 \bigg\rangle}
\newcommand{\ketbbbb}[1]{\Bigg| #1 \Bigg\rangle}
\newcommand{\overlap}[2]{\langle #1 | #2 \rangle}
\newcommand{\overlapb}[2]{\big\langle #1 \big| #2 \big\rangle}
\newcommand{\overlapbb}[2]{\Big\langle #1 \Big| #2 \Big\rangle}
\newcommand{\overlapbbb}[2]{\bigg\langle #1 \bigg| #2 \bigg\rangle}
\newcommand{\overlapbbbb}[2]{\Bigg\langle #1 \Bigg| #2 \Bigg\rangle}
\newcommand{\bracket}[3]{\langle #1 | #2 | #3 \rangle}
\newcommand{\bracketb}[3]{\big\langle #1 \big| #2 \big| #3 \big\rangle}
\newcommand{\bracketbb}[3]{\Big\langle #1 \Big| #2 \Big| #3 \Big\rangle}
\newcommand{\bracketbbb}[3]{\bigg\langle #1 \bigg| #2 \bigg| #3 \bigg\rangle}
\newcommand{\bracketbbbb}[3]{\Bigg\langle #1 \Bigg| #2 \Bigg| #3 \Bigg\rangle}
\newcommand{\projection}[2]
{| #1 \rangle \langle  #2 |}
\newcommand{\projectionb}[2]
{\big| #1 \big\rangle \big\langle #2 \big|}
\newcommand{\projectionbb}[2]
{ \Big| #1 \Big\rangle \Big\langle #2 \Big|}
\newcommand{\projectionbbb}[2]
{ \bigg| #1 \bigg\rangle \bigg\langle #2 \bigg|}
\newcommand{\projectionbbbb}[2]
{ \Bigg| #1 \Bigg\rangle \Bigg\langle #2 \Bigg|}





\makeindex             % used for the subject index
                       % please use the style svind.ist with
                       % your makeindex program

%%%%%%%%%%%%%%%%%%%%%%%%%%%%%%%%%%%%%%%%%%%%%%%%%%%%%%%%%%%%%%%%%%%%%

\begin{document}

\author{Morten Hjorth-Jensen}
\title{Quantum Physics; with a computational perspective}
 
\subtitle{Michigan State University and University of Oslo}
\date{Januar 2017}
\maketitle



\mainmatter%%%%%%%%%%%%%%%%%%%%%%%%%%%%%%%%%%%%%%%%%%%%%%%%%%%%%%%
\subsection*{Kursets struktur}

Kurset er delt inn i tre hoveddeler:
\begin{itemize}
   \item \underline{F\o rste del} tar for seg den historiske utviklingen fra
         slutten av det nittende \aa rhundre til begynnelsen av
         forrige \aa rhundre. I denne tidsperioden vokste
         erkjennelsen av at klassisk fysikk (Newtons lover m.m.)
         ikke kunne beskrive resultater fra flere nye eksperimenter, bla.~flerespektroskopiske data. 
          Denne utviklingen f\o rte 
         fram til den nye kvanteteorien i 1925. De nye begrepene
         som ble innf\o rt var bla.~{\bf materieegenskapen til str\aa ling}, 
         {\bf b\o lgegenskapene til materien} og {\bf kvantiseringen av
              fysiske st\o rrelser som f.eks.~energien eller banespinnet}.
    \item \underline{Andre del} tar for seg en f\o rste introduksjon til 
          kvantemekanikk, med hovedvekt p\aa\ b\o lgemekanikk
          og en-partikkel problemer. Denne delen avsluttes
          med en kvantemekanisk beskrivelse av hydrogenatomet.
    \item \underline{Tredje og siste del} Undervises resten av 
          semesteret. Her tar vi for oss ulike andvendelser  
          fra kvantemekanikkens spede begynnelse med
          atomfysikk, til kjernefysikk, moderne partikkelfysikk
          og faste stoffers fysikk. Litt om kvantedatamaskiner.
\end{itemize}  
\subsection*{Kursets innhold}

\begin{itemize}
   \item \underline{F\o rste del}.
        \begin{itemize}
           \item Enheter og st\o rrelser i FYS2140
           \item Fordelingsfunksjoner
           \item Svart legeme str\aa ling og Plancks kvantiseringshypotese
           \item Fotoelektrisk effekt
           \item R\"ontgenstr\aa ling
           \item Comptonspredning
           \item Bohrs atommodell
           \item Materieb\o lger og partikkel-b\o lge dualitet
           \item Heisenbergs uskarphetsprinsipp
           \item Litt b\o lgel\ae re
           \item Schr\"odingers katt paradokset og Entanglement
         \end{itemize}
      \item \underline{Andre del}.
        \begin{itemize}
           \item Introduksjon til kvantemekanikk og enkle
                 kvantemekaniske systemer
           \item Kvantisering av banespinn
           \item Hydrogen atomet
        \end{itemize}
      \item \underline{Tredje del}.
        \begin{itemize}
           \item Atomfysikk, det periodiske systemet
           \item Molekyler
           \item Kvantestatistikk og lasere
           \item Kvantecomputere
           \item Faste stoffers fysikk, Bose-Einstein kondensasjon 
                 og supraledning.
           \item Kjernefysikk, strukturen til kjerner, str\aa ling,
                 dannelsen av elementene, fisjon og fusjon, kjernekrefter
           \item Moderne partikkelfysikk, kvarker og leptoner
         \end{itemize}
      \end{itemize}





Til slutt en takk til alle som har kommet med kommentarer, trykkfeil
m.m.~om de ulike utkastene som har florert p\aa\ kursets hjemmeside.
I tillegg er vi sv\ae rt takknemlige for hjelpen Simen Kvaal ga oss i
utviklingen av flere av de numeriske oppgavene.

















\clearemptydoublepage

\tableofcontents

\clearemptydoublepage

\pagenumbering{arabic}
\part{Introducing quantum physics}
	\chapter{Breaking with classical physics}
\begin{quotation}
Quantum mechanics: Real black magic calculus. {\em Albert Einstein}
\end{quotation}
\begin{quotation}
Anybody who is not shocked by quantum theory has not understood it. {\em Niels Bohr}
\end{quotation}

Kursets f\o rste del har  som hensikt \aa\ gi dere 
en viss oversikt over den historiske utviklingen som f\o rte 
til formuleringen av kvantemekanikken i 1925.
Det er p\aa\ ingen m\aa te en fullstendig historisk oversikt. Vi vil kun ta for
oss det vi anser som de viktigste bitene p\aa\ denne vei, deriblant en 
beskrivelse av svart legeme str\aa ling vha.\ Plancks kvantiseringshypotese,
fotoelektrisk effekt, Compton spredning, R\"ontgen str\aa ling, Bohrs
atommodell, materiens partikkel og b\o lgeegenskaper samt om Heisenbergs
uskarphetsrelasjon. I tillegg, vil vi ogs\aa\ ta for oss litt b\o lgel\ae re
som er relevant for dette kurset. Men f\o rst en generell introduksjon.

\section{Introduksjon}

Ved starten av forrige \aa rhundre kan en kanskje beskrive situasjonen 
i Fysikk som Pandoras
boks av eksperimentelle observasjoner som, med basis i veletablert klassisk
fysikk, ikke lot seg forklare.  
V\aa r historie begynner dermed ved slutten og 188-tallet og 
starten av forrige \aa rhundre,
n\ae rmere bestemt 14 desember 1900, da Planck foreleste om 
sin teori for svart legeme str\aa lning til det Tyske Fysiske
selskap. En rekke fysiske fenomen, fotoelektrisk effekt, diskrete
spektra fra ulike gasser, R\"ontgen str\aa lning m.m., kunne ikke
forklares vha.~det vi idag kaller for klassisk fysikk, dvs.~den 
tids bevegelseslover for naturen, utrykt vha.~Newtons og Maxwells likninger.
Teorien resulterte i flere inkonsekvenser i forhold til eksperiment, 
slik som den s\aa kalte 'ultrafiolette 
katastrofe', eller at elektroner kunne klappe sammen med kjernen.  

I f\o rste omgang ble disse problemene l\o st vha.~{\em ad hoc} hypoteser,
slik som Plancks kvantiserings hypotese eller Bohrs atommodell. Den historiske
gangen viser oss ogs\aa\ klart at fysikk er et eksperimentelt fag, {\em teorier
om naturen utvikles h\aa nd i h\aa nd med eksperiment.}

Etterhvert som en fikk st\o rre eksperimentell innsikt om b\aa de atomer og
str\aa ling, viste slike {\em ad hoc} forklaringer seg som utilstrekkelige.
Krisen i klassisk fysikk kom til sin ende i 1925 med formuleringen
av kvantemekanikken.
Her f\o lger noen viktige oppdagelser og teorier som var 
med \aa\ forme den f\o rste tida. \newline\newline
\begin{tabular}{lll}
1898 & Madam Curie & Radioaktiv polonium og radium\\
1900 & Planck & Plancks kvantiserings hypotese og svart legeme str\aa ling\\
1905 & Einstein & Fotoelektrisk effekt \\
1911 & Rutherford & Atommodell\\
1913 & Bohr & Kvanteteori for atomspektra\\
1922 & Compton & Spredning av fotoner p\aa\ elektroner\\
1923 & Goudsmit og Uhlenbeck & Elektronets egenspinn\\
1924 & Pauli & Paulis ekslusjonsprinsipp \\
1925 & De Broglie & Materieb\o lger\\
1926 & Schr\"odinger & B\o lgelikning og ny naturlov\\
1927 & Heisenberg & Uskarphetsrelasjonen \\
1927 & Davisson og Germer & Eksperiment som p\aa viste materiens b\o lgeegenskaper\\
1927 & Born & Tolkningen av b\o lgefunksjonen \\
1928 & Dirac & Relativistisk kvantemekanikk og prediksjon av positronet 
\end{tabular}\newline\newline
Kvantemekanikken har v\ae rt et unnv\ae rlig verkt\o y i v\aa r s\o ken etter
\aa\ beskrive naturen, fra \aa\ forklare hvordan sola skinner, til studier
av atomer og deres struktur og understrukturer, supraledning, kjernefusjon
i stjerner, n\o ytronstjerner, strukturen til DNA molekylet,  
element\ae rpartiklene i naturen og m.m. Og kvantemekanikken ligger til grunn
for store deler av v\aa r n\aa v\ae rende og framtidige teknologiske utvikling.

Fra et mer filosofisk st\aa sted kan vi si at 
dialektikken mellom eksperiment og teori var med \aa\ 
forme en ny naturvitenskapelig
filosofi. Som basis for v\aa r forst\aa else av naturen har vi erstatta
den objektive determinismen gitt ved f.eks.~Newtons lover med en 
subjektiv og sannsynlighetsbestemt determinisme. 

\subsection{Hva er kvantemekanikk?} 

Kvantemekanikk er et matematisk byggverk, et sett av regler for \aa\
lage fysiske teorier om naturen, en matematisk m\aa te 
\aa\ uttrykke naturlover.
Schr\"odingers likning er v\aa r naturlov, b\aa de for mikroskopiske
og makroskopiske systemer. Settet med regler inneholder ogs\aa\ tolkninger av 
teorien samt ulike postulater, f.eks.~hvordan en m\aa ling av en fysisk
st\o rrelsen skal defineres. 
Reglene er enkle men h\o yst ikke-trivielle i sine tolkninger, noe som har leda
og leder til interessante kontroverser om naturens egenskaper og v\aa r 
evne til \aa\ forst\aa\ den. Den kanskje mest kjente kritikeren av 
kvantemekanikk er vel Einstein, som var, sammen med Planck, en av teoriens
'jordm\"odre.

I tillegg, og her er det viktig \aa\ skille mellom kvantemekanikken som teori
og postulatene om naturen som resulterte i  etableringen av kvantemekanikken. 
Kvantemekanikk som teori baserer seg p\aa\ bla.~flere viktige 
p\aa stander om naturen. 
Det er fire viktige postulater som danner grunnlaget for kvantemekanikkens
beskrivelse av naturen og Schr\"odingers likning som bevegelseslov og som
dere kommer til \aa\ f\aa\ presentert i dette kurset.

Disse fire postulatene har ingen klassisk analog, og utgj\o r et sett
med  p\aa stander om naturen. 

\begin{itemize} 
\item Einsteins og Plancks postulat om energiens kvantisering 
     
      \[
          E=nh\nu,
      \]
       hvor $\nu$ er frekvensen og $h$ er Plancks konstant. Tallet 
       $n$ er et heltall og kalles for et kvantetall. 
\item De Broglie sitt postulat om materiens b\o lge og partikkel
      egenskaper. Det uttrykkes vha.~relasjonene
\[
     \lambda =\frac{h}{p}  \hspace{1cm} \nu=\frac{E}{h}    
\]
hvor $\lambda$ er b\o lgelengden og $p$ bevegelsesmengden.
Partikkelegenskapene uttrykkes via energien og bevegelsesmengden,
mens b\o lgelengden og frekvensen uttrykker b\o lgeegenskapene.
\item Heisenbergs uskarphetsrelasjon. 
\[
    \Delta {\bf p}\Delta {\bf x} \geq \frac{\hbar}{2},
\]
Klassisk er det slik at ${\bf x}$ og ${\bf p}$ er uavhengige st\o rrelser.
Kvantemekanisk  har vi en avhengighet gitt ved
Heisenbergs uskarphets relasjon. Dette impliserer igjen at vi ikke kan 
lokalisere en partikkel og samtidig bestemme dens bevegelsesmengde skarpt. 
En ytterligere konsekvens er at vi ikke kan i et bestemt eksperiment observere
b\aa de partikkel og b\o lgeegenskaper samtidig. 

\item Paulis eksklusjonsprinsipp: den totale b\o lgefunksjonen for et 
      system som best\aa r av identiske partikler med halvtallig
      spinn m\aa\ v\ae re antisymmetrisk.  Det har som f\o lge at i en
      sentralfelt modell som anvendes i f.eks.~atomfysikk, s\aa\ kan ikke
      to eller flere elektroner ha samme sett kvantetall. 
      Partikler med halvtallig spinn kalles for fermioner. Eksempler
      er elektroner, protoner, n\o ytroner, kvarker og n\o ytrinoer.
For partikler med heltallig spinn
      m\aa\ b\o lgefunksjonen v\ae re symmetrisk.
      Partikler med heltallig spinn kalles bosoner. Eksempler er
      fotoner, Helium atomer,  ulike mesoner og gluoner. 
\end{itemize}


\subsection{Hvorfor er kvantemekanikk spennende?}

Uten \aa\ ta munnen for full, dere som f\o lger dette kurset 
har en utrolig spennende tid \aa\ se fram til! Hvorfor? 


Kvantefysikken inneholder alts\aa\ en del postulater med konsekvenser
for v\aa r forst\aa else av naturen som er h\o yst ikke-trivielle.
Slik kvantemekanikken framst\aa r idag, utgj\o r den v\aa r beste
forst\aa else av naturen. Schr\"odingers likning, 
hvis tilh\o rende l\o sning forteller om egenskaper til et mikrosystem,  
er v\aa r naturlov. 

Fram til ca.~1970 kan vi si at mye av den eksperimentelle
informasjonen vi hadde om mikrosystemer i all hovedsak dreide seg om
systemer med mange partikler, f.eks.~mange enkeltatomer. 
Grovt sett kan en si at vi ikke hadde tilgang til informasjon om 
kvantemekaniske enkeltsystemer som f.eks.~et atom. Supraledning er et slikt 
eksempel. Her har vi en makroskopisk manifestasjon av en kvantemekanisk 
effekt, men vi kan ikke trekke ut eksperimentell informasjon om 
enkeltelektronene som bidrar.   
Siden 1970 har det blitt utviklet teknikker, f.eks.~det som g\aa r under
navnet ionefeller (Nobel pris i fysikk i 1989), 
hvor vi vha.~f.eks.~elektriske kvadrupolfelt 
kan fange inn enkeltatomer i sm\aa\
omr\aa der som er isolerte fra omgivelsene. Deretter kan vi studere 
ulike kvantemekaniske frihetsgrader til dette enkeltatomet ved f.eks.~\aa\
sende
laserlys med bestemte frekvenser. 
Den nye eksperimentelle og teknologiske situasjonen vi er ved kan v\ae re
med \aa\ legge grunnlaget for   
           \begin{itemize}
               \item Ny teknologi
               \item Nye fagfelt samt overlapp med flere eksisterende
                     felt, fysikk, matematikk, informatikk, kjemi m.m.
               \item Eksisterende teknologi gj\o r at vi kan studere
                     og kanskje utnytte sider av kvantemekanikken
                     som anses   som mindre
                     trivielle. Eksempler er Schr\"odingers katt 
                     paradokset og 'Entanglement' (mere om dette senere).  
               \item Kanskje vi utvikler ogs\aa\ en bedre forst\aa else av
                     naturen, {\bf en ny og bedre teori?}
               \item Det er ogs\aa\ en interessant parallell til 
                     begynnelsen av forrige \aa rhundre. Flere 
                     eksperiment (fotoelektrisk effekt, svart legeme str\aa ling m.m.) kunne ikke forklares vha.~klassisk fysikk. Det ledet igjen  
til utviklingen
av kvantemekanikken rundt 1925.  Med dagens teknologi kan vi f.eks.~fange inn enkeltatomer og elektroner i sm\aa\ omr\aa der (noen f\aa\ nanometre) 
og studere tilh\o rende kvantemekaniske egenskaper.                 
           \end{itemize}



V\aa r m\aa lsetting er \aa\ gi dere en introduksjon til 
kvantemekanikken, hvor vi vektlegger den historiske
gangen fram til Schr\"odingers likning, forst\aa\ enkle
kvantemekaniske systemer og det periodiske systemet. I tillegg,
tar vi med oss anvendelser fra moderne forskningsfelt som kvantedatamaskiner,
litt om molekyler, halvledere og til slutt litt kjerne og partikkelfysikk.

Matematikken i dette kurset er ikke vanskelig, selv om en del 
manipulering med matematiske uttrykk kan virke innfl\o kt 
innledningsvis. Den formelle 
matematiske formalismen som kjennetegner kvantemekanikk vektlegges ikke
i dette kurset. Videreg\aa ende emner som FYS 201 har dette som et viktig
tema.
De vanskeligste matematiske problem vi kan komme i kontakt med er integral
av typen
\[
   \int_a^be^{-\alpha x} x^n dx,
\]
og
\[
   \int_a^be^{-\alpha x^2} x^n dx,
\]
med $\alpha$ en reell positiv konstant og $n$ et positivt tall.
Bokstavene $a$ og $b$ representerer integrasjonsgrensene. 
I tillegg kommer kjennskap til regning med komplekse tall og variable.

For de av dere som er av den  ut\aa lmodige typen og t\o rster lengselsfullt  
etter en mer formell beskrivelse enn det som gis her, 
kan et alternativ v\ae re \aa\ f\o lge FYS 201 (et veldig bra kurs) 
parallellt med dette kurset.    

\section{Enheter i kvantefysikk}

I kvantefysikk er vi opptatt av \aa\ beskrive fysiske fenomen
p\aa\ det vi kan kalle mikroskala. Typiske lengdeskalaer av interesse
g\aa r fra $10^{-8}$ m ned til $=10^{-18}$ m. Enhetene som da benyttes er
{\bf nm}, leses nanometer, som er $10^{-9}$ m og {\bf fm}, leses femtometer,
og er gitt ved  1fm$=10^{-15}$ m.
Nanometer er lengdeskalaen i atomfysikk, 
             faste stoffers fysikk og molekylfysikk. Tilsvarende anvendes
femtometer i kjerne og
             partikkelfysikk. Senere i dette kurset skal vi se at
det er en sammenheng mellom de kreftene som virker (f.eks.~Coulomb 
vekselvirkningen i atomfysikk) og et systems energi og dermed dets lengdeskala. 

For \aa\ gi dere et enkelt eksempel p\aa\ de lengdeskalaer som vi skal
befatte oss med, la oss hente fram Avogadros tall
\[
   N_A=6.023\times 10^{23},
\]
som betyr at det er $N_A$ atomer i $A$ gram av et ethvert element, 
hvor $A$ er det atom\ae re massetall. Det vi si at et 1 g hydrogen,
12 g karbon ($^{12}$C) og 238 g uran ($^{238}$U) har like mange atomer.
La oss s\aa\ anta at vi har et gram av flytende hydrogen
og stiller oss selv sp\o rsm\aa let om hvor stor utbredelse
et hydrogenatom har, dvs. hvor stor er diameteren til hydrogenatomet
som best\aa r av et elektron og et proton. 
Vi har oppgitt at tettheten $\rho$ av flytende hydrogen
er $\rho=71$ kg/m$^3$.
Volumet opptatt av et gram er da
\be
   V=\frac{10^{-3}\hspace{0.1cm}\mathrm{kg}}{\rho},
\ee
og volumet opptatt av et atom er da
\be
   V_{\hspace{0.1cm}\mathrm{atom}}=\frac{V}{N_A}=\frac{10^{-3}\hspace{0.1cm}\mathrm{kg}}{N_A\rho}=
                     \frac{10^{-3}\hspace{0.1cm}\mathrm{kg}}{71\hspace{0.1cm}\mathrm{kg/m}^2
                      6.023\times 10^{23}}=2.3\times 10^{-29} \hspace{0.1cm}\mathrm{m}^3.  
\ee
Deretter antar vi at denne v\ae sken best\aa r av tettpakka kuler av hydrogen
atomer, slik at vi kan sette diameteren $d$ 
\be
   d\sim V_{atom}^{1/3} = 3\times 10^{-10}\hspace{0.1cm}\mathrm{m}=0.3 \hspace{0.1cm}\mathrm{nm}.
\ee
Vi skal senere i kurset ( i kapittel 7 i tektsboka) se at n\aa r vi regner
ut den gjennomsnittlige diameteren for hydrogenatomet vha.\
kvantemekanikk, vil vi finne en liknende st\o rrelsesorden for diameteren.
Dette enkle eksempel er ment som en illustrasjon p\aa\
de lengdeskalaer som er av betydning for det vi skal drive med her.
I forbifarten kan vi nevne at radius til et proton er ca.\ 1 fm,
mens radius til en atomkjerne (uten elektronene, kun protoner og n\o ytroner)
slik som bly er p\aa\ ca 7 fm. Dette forteller ogs\aa\ noe om 
at de sterke kjernekreftene som holder 
en atomkjerne sammen har kort rekkevidde (mere om dette
i kapittel 14 i l\ae reboka).

La oss n\aa\ introdusere den viktige energienheten v\aa r.
Fra FYS-ME1100 og FYS1120 har dere v\ae rt vant med Joule som 
energienhet, J=kgm$^2$/s$^2$. I FYS2140 vil vi operere med energiskalaer
av typen $10^{-19}$ J. Da er det hensiktsmessig \aa\ innf\o re en ny
energienhet, {\bf elektronvolt} med enhet eV.
Fra
 FYS2140 har vi at elektronets
ladning er gitt ved 
\be
    e=1.602\times 10^{-19}\hspace{0.1cm}\mathrm{C},
\ee
og at 1 V =1J/1C. 
I Fys2140 
 definerer vi
1 eV som den mengde kinetisk
energi som et elektron f\aa r n\aa r det akselereres gjennom en potensial
differanse p\aa\ 1 V.
Setter vi n\aa\
\be
1\hspace{0.1cm}\mathrm{J}=1\hspace{0.1cm}\mathrm{V}1\hspace{0.1cm}\mathrm{C}=1\hspace{0.1cm}\mathrm{V}1\hspace{0.1cm}\mathrm{C}\frac{e}{e}=
            1\hspace{0.1cm}\mathrm{V}1\hspace{0.1cm}\mathrm{C}\frac{e}{1.602\times 10^{-19}\hspace{0.1cm}\mathrm{C}}=
\frac{1\hspace{0.1cm}\mathrm{eV}}{1.602\times 10^{-19}},
\ee
har vi at 
\be
   1 \hspace{0.1cm}\mathrm{eV}=1.602\times 10^{-19} \hspace{0.1cm}\mathrm{J}.
\ee
Vi kan n\aa\ omregne hvileenergien til
elektronet $E_0^{\mathrm{elektron}}=m_ec^2$, hvor massen til elektronet er
\be
    m_e=9.11 \times 10^{-31} \hspace{0.1cm}\mathrm{kg},
\ee
i enheter eV ved \aa\ sette
             \be
                m_ec^2= 9.11\times 10^{-31}(3\times 10^8)^2
                                       \hspace{0.1cm}\mathrm{J}=
                (9.11\times 10^{-31}(3\times 10^8)^2/1.602\times 10^{-19})
                                       \hspace{0.1cm}\mathrm{eV},
             \ee
som gir
             \be
                E_0^{\mathrm{elektron}}=m_ec^2=5.11\times 10^5 \hspace{0.1cm}\mathrm{eV}
             \ee
             eller 0.511 MeV, med 1 MeV = 1000000 eV.
{\bf For massen brukes ofte}
             \be
                m_e= E_0^{\mathrm{elektron}}/c^2
             \ee
              dvs at vi skriver
$m_e=0.511$ MeV/$c^2$, som leses MeV-over-c-i-andre, men i bekvemmelighetens
\aa nd forkortes den oftest
til bare MeV.
             For protonet har vi  $m_p=938$ MeV/$c^2$.
       I atomfysikk, faste stoffers fysikk og molekylfysikk
             har vi energier p\aa\ st\o rrelsesorden {\bf med noen eV},
             i all hovedsak er det Coulomb vekselvirkningen som gir 
             vesentlige bidrag til disse systemenes fysikk.
       I kjerne og partikkelfysikk opererer vi med energier
             p\aa\ st\o rrelse med MeV, GeV (=1000 MeV) (massen til kvarker
             og tunge bosoner) eller TeV (massen til den mystiske Higgs
             partikkelen). Maks bindingsenergi til kjerner, n\aa r vi ser bort
             fra hvilenergien til protoner og n\o ytroner, er p\aa\ ca.
             8 MeV ($^{56}$Fe).
       Andre nyttige st\o rrelser er Plancks konstant $h$
             \be
               h=6.626\times 10^{-34} \hspace{0.1cm}\mathrm{Js}
             \ee
men oftest brukes $\hbar$ (leses h-strek) 
              \be
               \hbar=\frac{h}{2\pi}=6.582\times 10^{-16} \hspace{0.1cm}\mathrm{eVs}.
             \ee
I tillegg, forekommer $\hbar$ ofte sammen med lyshastighten $c$,
slik at en ny 'hendig' st\o rrelse er
              \be
               \hbar c=197 \hspace{0.1cm}\mathrm{eVnm} (\hspace{0.1cm}\mathrm{MeVfm})
             \ee
Til slutt kan faktoren i Coulombvekselvirkningen mellom
to f.eks.\ to elektroner
\be
    \frac{e^2}{4\pi\epsilon_0 r}
\ee
hvor $\epsilon_0$ er permittiviteten og $r$ er absolutt verdien av avstanden
mellom de to elektronene, settes lik
             \be
               \frac{e^2}{4\pi\epsilon_0}=1.44 \hspace{0.1cm}\mathrm{eVnm}
             \ee


Som et eksempel p\aa\ st\o rrelsesordener, la oss se p\aa\ forholdet
mellom gravitasjonskrefter og elektrostatiske krefter.
Gravitasjonskraften er gitt ved
\be
    F_G=-\frac{Gm_1m_2}{r^2},
\ee
hvor $G=6.67\times 10^{-11}$ m$^{3}$kg$^{-1}$s$^{-2}$ 
er gravitasjonskonstanten, 
$r$ er avstanden mellom legeme 1 og 2
og $m_1$ og $m_2$ deres respektive masser. La oss anta at $m_1$ er et elektron
og at $m_2$ er et proton. Coulombkraften er gitt ved
\be
    F_C=-\frac{e^2}{4\pi\epsilon_0 r^2}.
\ee
Setter vi inn for elektronets og protonets masse har vi
\be
    F_G/F_C \approx 4 \times 10^{-40}.
\ee
N\aa\ vet vi at bindingsenergien til elektronet i hydrogenatomet er
$-13.6$ eV og at den midlere avstanden mellom elektronet og protonet er
ca.~0.05 nm. Dersom det er gravitasjonskreftene som holder hydrogenatomet
sammen, hvor stor blir da den midlere avstanden?
Her kan du anta at bindingsenergien til elektronet er proporsjonalt
med 
\[
    -\frac{e^2}{4\pi\epsilon_0 r}.
\]
Sammenlikner radien du finner med den estimerte avstanden til den fjerneste
galakse, s\aa\ har vi ca.~$10^{25}$ m.

N\aa r vi f\o rst har tatt steget ut i universet,
la oss avslutte dette avsnittet med en ytterligere digresjon fra verdensrommet.
Solas masse er gitt ved 
\be
M_{\odot}=1.989\times 10^{30} \hspace{0.1cm}\mathrm{kg}.
\ee
og protonets masse er
\be
  M_{p}=1.673\times 10^{-27} \hspace{0.1cm}\mathrm{kg}.
\ee
Siden elektronet har en masse som er ca 2000 ganger mindre en protonet, kan vi anta at det er hovedsaklig protoner og n\o ytroner (som har nesten samme masse som
protonet) som bidrar til solas totale masse. 
Antallet protoner og n\o ytroner $N$ er da gitt ved
\be 
    N=  \frac{M_{\odot}}{m_p}\sim 10^{57}.
\ee
Vi kan s\aa\ pr\o ve \aa\ gjenta eksemplet med hydrogenatomet. Vi antar
at sola best\aa r av en gass med tettpakkede protoner og n\o ytroner
og at volumet $V_p$ opptatt av et proton (n\o ytron) er
\be
    V_p= \frac{4\pi r_p^3}{3},
\ee
hvor $r_p$ er radien til protonet. Ovenfor oppga vi at $r_p\sim 1$ fm
eller $10^{-15}$ m.
Volumet til sola $V_S=4\pi R^3/3$, hvor $R$ er solas radius,  blir da
\be 
   V_S=NV_p = \frac{4\pi (10^{-15})^3 \hspace{0.1cm}\mathrm{m}^3}{3}\times 10^{57}.
\ee
Dette gir oss en radius
\be 
   R\sim 10 \hspace{0.1cm}\mathrm{km}!!
\ee
Solradien er gitt ved 
\be
   R_{\odot}= 7\times 10^5 \hspace{0.1cm}\mathrm{km}.
\ee
Dette forteller oss at den modellen vi antok for \aa\ beskrive den gjennomsnittlige tettheten i sola er feil. Gjennomsnittlig tetthet i sola er estimert til
ca.\ 1.4 g/cm$^3$. Derimot vil radien p\aa\ ca 10 km svare til radien
til ei n\o ytronstjerne, som er et mulig resultat av en supernova eksplosjon,
sluttstadiet for en stjerne som har brukt opp all brennstoffet sitt.
Massen til en n\o ytronstjerne er ca.\ 1.4 solmasser, s\aa\ dere kan tenke
dere sola konsentrert i et omr\aa de med Fysisk institutt som sentrum 
med radius 10 km. Det sier seg selv at tettheten m\aa\ v\ae re enorm.  
Tettheten  i ei n\o ytronstjerne varierer fra
$10^6$  g/cm$^3$ i de ytre lag til $10^{15}$  g/cm$^3$ i det indre av stjernen.

Til slutt, for \aa\ bringe oss over til neste tema, en st\o rrelse som er 
interesse er hvor mye energi sola str\aa ler ut per sekund.
Denne st\o rrelsen kalles luminositeten og er gitt
ved
\be
   L_{\odot}=3.826\times 10^{26} \hspace{0.1cm}\mathrm{J/s},
\ee
hvor J/s=W (watt). 
Ser vi p\aa\ utstr\aa lt energi per sekund, f\aa r vi radiansen,
som vi skal diskutere n\ae rmere i neste avsnitt.


\begin{table}[h]
\caption{Standard metrisk notasjon for tierpotenser}
\begin{tabular}{lll}\hline
{\bf Potens} & {\bf prefiks} & {\bf Symbol}\\ \hline
$10^{18}$ & exa & E\\ 
$10^{15}$ & peta & P\\ 
$10^{12}$ & tera & T\\ 
$10^{9}$ & giga & G\\ 
$10^{6}$ & mega & M\\ 
$10^{3}$ & kilo & k\\ 
$10^{-2}$ & centi & c\\ 
$10^{-3}$ & milli & m\\ 
$10^{-6}$ & micro & $\mu$\\ 
$10^{-9}$ & nano & n\\ 
$10^{-12}$ & pico & p\\ 
$10^{-15}$ & femto & f\\ 
$10^{-18}$ & atto & a\\ \hline
\end{tabular}
\end{table}

Vi kommer til \aa\ bruke det internasjonale enhets systemet, SI, 
hvor dynamiske variable uttrykkes i fem fundamentale enheter,
meter (m), kilogram (kg), sekund (s), ampere (A) og kelvin (K).
I kvantefysikk er det, som vist ovenfor, mer hensiktsmessig \aa\
bruke enheter som eV for energi.
Nyttige omregningsfaktorer er 1 eV=$1.60\times 10^{-19}$ J og
en atom\ae r masseenhet gitt ved 1 u$=1/12$ av massen til
$^{12}$C=$1.6604\times 10^{-27}$ kg $=931.48$ MeV/c$^2$.
Nedenfor finner dere flere konstanter som blir brukt i dette kurset.
\begin{table}[h]
\caption{Nyttige konstanter}
\begin{tabular}{lll}\hline
{\bf Konstant} & {\bf symbol} & {\bf verdi}\\ \hline 
 Lyshastighet&$c$ &$3.00\times 10^{8}$ m/s  \\ 
 Gravitasjonskonstant&$G$ &$6.67\times 10^{-11}$ Nm$^2$/kg$^2$ \\ 
 Coulombkonstant&$k$ &$8.99\times 10^{9}$ Nm$^2$/C$^2$ \\ 
 Boltzmannkonstant&$k_B$ &$1.38\times 10^{-23}$ J/K \\ 
 Element\ae rladning&$e$ &$1.60\times 10^{-19}$ C \\
 Plancks konstant&$h$ &$6.63\times 10^{-34}$ Js \\
                 &$hc$ &1240 eVnm \\
                 &$\hbar=h/2\pi$ &$1.055\times 10^{-34}$ Js \\
                 &$\hbar c$ &197 eVnm \\
  Bohrradius&$a_0=\hbar^2/m_eke^2$ &0.0529 nm  \\
Finstrukturkonstanten&$\alpha$ &1/137.036  \\
  Coulombfaktor&$ke^2$ &1.44 eVnm  \\
Elektronets gyromagnetisk faktor & $g_e$ & 2.002 \\ 
  Grunntilstand hydrogen&$E_0=-ke^2/2a_0$ &-13.606 eV  \\ 
Rydberg&Ry &13.606 eV  \\ \hline
\end{tabular}
\end{table}


\begin{table}[h]
\caption{Masser til viktige partikler}
\begin{tabular}{llll}\hline
{\bf Partikkel} & {\bf i kg } & {\bf i MeV/c$^2$} & {\bf i u}\\\hline 
elektron &$9.109\times 10^{-31}$ kg&0.511 MeV/c$^2$& 0.000549 u\\
proton &$1.672\times 10^{-27}$ kg&938.3 MeV/c$^2$& 1.007277 u\\
n\o ytron &$1.675\times 10^{-27}$ kg&939.6 MeV/c$^2$& 1.008665 u\\
hydrogen &$1.673\times 10^{-27}$ kg&938.8 MeV/c$^2$& 1.007825 u\\ \hline
\end{tabular}
\end{table}

Annen nyttig informasjon er b\o lgelengden til synlig lys
som g\aa r fra 700 nm (m\o rk r\o d ) til 400 nm (m\o rk fiolett).

I mange tekstb\o ker i fysikk brukes ogs\aa\ det Gaussiske enhetssystemet.
De viktigste enhetene der er gram, centimeter og sekund. Ladningsenheten,
som kalles statcoulomb, representerer en ladning, som i en avstand p\aa\
1 cm fra en identisk ladning f\o ler en kraft p\aa\ 1 dyne.
En statcoulomb er da gitt ved
\[
  1\hspace{0.1cm} \mathrm{statcoulomb}=1 \hspace{0.1cm} \mathrm{dyne}^{1/2}
\mathrm{cm}=1 \hspace{0.1cm}\mathrm{g}^{1/2}\mathrm{cm}^{3/2}\mathrm{s}^{-1}.
\]
Tabellen nedenfor gir faktoren som trengs n\aa r vi g\aa r fra SI systemet til
det Gaussiske systemet.
\begin{table}[h]
\caption{Fra SI systemet til Gaussiske enheter}
\begin{tabular}{ll}\hline
{\bf variabel} & {\bf transformasjon} \\ \hline
Elektrisk felt & ${\bf E}\rightarrow \frac{1}{\sqrt{4\pi\epsilon_0}}{\bf E}$ \\
Vektorpotensial & ${\bf A}\rightarrow\sqrt{\frac{\mu_0}{4\pi}}{\bf A}$ \\
B-felt & ${\bf B}\rightarrow\sqrt{\frac{\mu_0}{4\pi}}{\bf B} $\\
Magnetisk moment & ${\bf \mu}\rightarrow\sqrt{\frac{4\pi}{\mu_0}}{\bf \mu}$ \\
Skalart potensial & $\phi\rightarrow \frac{1}{\sqrt{4\pi\epsilon_0}}\phi$ \\\hline
\end{tabular}
\end{table}





\section{Plancks kvantiseringshypotese}


Et av problemene\footnote{Sidehenvisning i l\ae reboka er kap 2-1, sidene 74-80,
kap 2-2 er kun bakgrunnsmateriale, men vi kommer i kap 5 til \aa\ utlede
energien til et system som utviser st\aa ende b\o lger. Kap 2-3, side 86-93
undervises i FYS2160 og vil ikke bli vektlagt som pensum. Kap 2-4, side 93-99
gir den f\o rste virkelige anvendelse av kvantiseringshypotesen og det som
la grunnlaget for kvantemekanikken. Det viktige med disse avsnittene er at
dere har klart for dere hva som skapte bruddet med klassisk fysikk. 
En bedre forklaring vil dere f\aa\ i FYS2160 og eventuelt FYS3130.} en ikke var i stand til \aa\ forklare vha.~klassisk fysikk 
var frekvensfordelingen til elektromagnetisk (e.m.) str\aa ling fra et
legeme ved en gitt temperatur, f.eks.~sola eller ei kokeplate. N\aa r vi setter
p\aa\ ei kokeplate merker vi i begynnelsen ikke noen nevneverdig
fargeforandring, selv om vi registrerer  
at den blir litt varmere. Etter en stund
blir den r\o dgl\o dende og innbyr neppe til \aa\ bli tatt p\aa\ . Men 
f\o r vi definerer problemet noe n\ae rmere, la oss ta for oss noen 
definisjoner. 
    \begin{itemize}
       \item Termisk str\aa ling : den e.m.~str\aa ling som sendes ut fra
             et legeme som resultat av dets temperatur. 
             Alle legemer sender ut (emisjon) og mottar 
             (absorpsjon) e.m.~str\aa ling. 
       \item Ved gitt temperatur $T$ er vi interessert i \aa\ finne 
             fordelingen av emittert str\aa ling som funksjon 
             av den e.m.~str\aa lingen sin frekvens $\nu$ eller 
             b\o lgelengde $\lambda$. Vi har f\o lgende relasjon mellom
             frekvensen $\nu$ og b\o lgelengden $\lambda$ 
             \[
                \nu=\frac{c}{\lambda}
             \]
       \item Frekvensfordelingen 
             \[
                 M_{\nu}(T)d\nu
             \]
             kalles spektralfordelingen eller kanskje bedre fordelingsfunksjonen for frekvensspekteret, eller bare frekvensfordeling.
       Denne fordelingen leses ogs\aa\ som {\bf utstr\aa lt energi
             fra en gjenstand ved temperatur $T$ per areal per tid per frekvensenhet}. Figur \ref{fig:blackbody} viser eksempler p\aa\ frekvensfordelinger
for ulike temperaturer fra et s\aa kalt svart legeme. 
Denne figuren viser ogs\aa\ resultatet fra klassisk
teori. Vi ser at denne fordelingsfunksjonen viser en divergerende oppf\o rsel
(ultrafiolett katastrofe)
ved h\o ye frekvenser (eller sm\aa\ b\o lgelengder), i strid med eksperimentelle resultat.
       \item Integrerer vi over alle frekvenser 
             \[
                M(T)=\int_0^{\infty} M_{\nu}(T) d\nu
             \]
             f\aa r vi totalt utstr\aa lt energi per sekund per areal ved gitt
             temperatur $T$. Dimensjonen til $M(T)$ er da
             [$M(T)$]=J/(m$^2$s)=W/m$^2$. Denne st\o rrelsen kalles 
             radiansen.
       \item {\bf V\aa rt problem er \aa\ finne fram til en fysisk forklaring
             for den eksperimentelle formen til $M_{\nu}(T)$.       }
    \end{itemize}
\begin{figure}
% GNUPLOT: LaTeX picture with Postscript
\begingroup%
  \makeatletter%
  \newcommand{\GNUPLOTspecial}{%
    \@sanitize\catcode`\%=14\relax\special}%
  \setlength{\unitlength}{0.1bp}%
{\GNUPLOTspecial{!
%!PS-Adobe-2.0 EPSF-2.0
%%Title: planck.tex
%%Creator: gnuplot 3.7 patchlevel 1
%%CreationDate: Sun Jan 20 18:55:04 2002
%%DocumentFonts: 
%%BoundingBox: 0 0 360 216
%%Orientation: Landscape
%%EndComments
/gnudict 256 dict def
gnudict begin
/Color false def
/Solid false def
/gnulinewidth 5.000 def
/userlinewidth gnulinewidth def
/vshift -33 def
/dl {10 mul} def
/hpt_ 31.5 def
/vpt_ 31.5 def
/hpt hpt_ def
/vpt vpt_ def
/M {moveto} bind def
/L {lineto} bind def
/R {rmoveto} bind def
/V {rlineto} bind def
/vpt2 vpt 2 mul def
/hpt2 hpt 2 mul def
/Lshow { currentpoint stroke M
  0 vshift R show } def
/Rshow { currentpoint stroke M
  dup stringwidth pop neg vshift R show } def
/Cshow { currentpoint stroke M
  dup stringwidth pop -2 div vshift R show } def
/UP { dup vpt_ mul /vpt exch def hpt_ mul /hpt exch def
  /hpt2 hpt 2 mul def /vpt2 vpt 2 mul def } def
/DL { Color {setrgbcolor Solid {pop []} if 0 setdash }
 {pop pop pop Solid {pop []} if 0 setdash} ifelse } def
/BL { stroke userlinewidth 2 mul setlinewidth } def
/AL { stroke userlinewidth 2 div setlinewidth } def
/UL { dup gnulinewidth mul /userlinewidth exch def
      10 mul /udl exch def } def
/PL { stroke userlinewidth setlinewidth } def
/LTb { BL [] 0 0 0 DL } def
/LTa { AL [1 udl mul 2 udl mul] 0 setdash 0 0 0 setrgbcolor } def
/LT0 { PL [] 1 0 0 DL } def
/LT1 { PL [4 dl 2 dl] 0 1 0 DL } def
/LT2 { PL [2 dl 3 dl] 0 0 1 DL } def
/LT3 { PL [1 dl 1.5 dl] 1 0 1 DL } def
/LT4 { PL [5 dl 2 dl 1 dl 2 dl] 0 1 1 DL } def
/LT5 { PL [4 dl 3 dl 1 dl 3 dl] 1 1 0 DL } def
/LT6 { PL [2 dl 2 dl 2 dl 4 dl] 0 0 0 DL } def
/LT7 { PL [2 dl 2 dl 2 dl 2 dl 2 dl 4 dl] 1 0.3 0 DL } def
/LT8 { PL [2 dl 2 dl 2 dl 2 dl 2 dl 2 dl 2 dl 4 dl] 0.5 0.5 0.5 DL } def
/Pnt { stroke [] 0 setdash
   gsave 1 setlinecap M 0 0 V stroke grestore } def
/Dia { stroke [] 0 setdash 2 copy vpt add M
  hpt neg vpt neg V hpt vpt neg V
  hpt vpt V hpt neg vpt V closepath stroke
  Pnt } def
/Pls { stroke [] 0 setdash vpt sub M 0 vpt2 V
  currentpoint stroke M
  hpt neg vpt neg R hpt2 0 V stroke
  } def
/Box { stroke [] 0 setdash 2 copy exch hpt sub exch vpt add M
  0 vpt2 neg V hpt2 0 V 0 vpt2 V
  hpt2 neg 0 V closepath stroke
  Pnt } def
/Crs { stroke [] 0 setdash exch hpt sub exch vpt add M
  hpt2 vpt2 neg V currentpoint stroke M
  hpt2 neg 0 R hpt2 vpt2 V stroke } def
/TriU { stroke [] 0 setdash 2 copy vpt 1.12 mul add M
  hpt neg vpt -1.62 mul V
  hpt 2 mul 0 V
  hpt neg vpt 1.62 mul V closepath stroke
  Pnt  } def
/Star { 2 copy Pls Crs } def
/BoxF { stroke [] 0 setdash exch hpt sub exch vpt add M
  0 vpt2 neg V  hpt2 0 V  0 vpt2 V
  hpt2 neg 0 V  closepath fill } def
/TriUF { stroke [] 0 setdash vpt 1.12 mul add M
  hpt neg vpt -1.62 mul V
  hpt 2 mul 0 V
  hpt neg vpt 1.62 mul V closepath fill } def
/TriD { stroke [] 0 setdash 2 copy vpt 1.12 mul sub M
  hpt neg vpt 1.62 mul V
  hpt 2 mul 0 V
  hpt neg vpt -1.62 mul V closepath stroke
  Pnt  } def
/TriDF { stroke [] 0 setdash vpt 1.12 mul sub M
  hpt neg vpt 1.62 mul V
  hpt 2 mul 0 V
  hpt neg vpt -1.62 mul V closepath fill} def
/DiaF { stroke [] 0 setdash vpt add M
  hpt neg vpt neg V hpt vpt neg V
  hpt vpt V hpt neg vpt V closepath fill } def
/Pent { stroke [] 0 setdash 2 copy gsave
  translate 0 hpt M 4 {72 rotate 0 hpt L} repeat
  closepath stroke grestore Pnt } def
/PentF { stroke [] 0 setdash gsave
  translate 0 hpt M 4 {72 rotate 0 hpt L} repeat
  closepath fill grestore } def
/Circle { stroke [] 0 setdash 2 copy
  hpt 0 360 arc stroke Pnt } def
/CircleF { stroke [] 0 setdash hpt 0 360 arc fill } def
/C0 { BL [] 0 setdash 2 copy moveto vpt 90 450  arc } bind def
/C1 { BL [] 0 setdash 2 copy        moveto
       2 copy  vpt 0 90 arc closepath fill
               vpt 0 360 arc closepath } bind def
/C2 { BL [] 0 setdash 2 copy moveto
       2 copy  vpt 90 180 arc closepath fill
               vpt 0 360 arc closepath } bind def
/C3 { BL [] 0 setdash 2 copy moveto
       2 copy  vpt 0 180 arc closepath fill
               vpt 0 360 arc closepath } bind def
/C4 { BL [] 0 setdash 2 copy moveto
       2 copy  vpt 180 270 arc closepath fill
               vpt 0 360 arc closepath } bind def
/C5 { BL [] 0 setdash 2 copy moveto
       2 copy  vpt 0 90 arc
       2 copy moveto
       2 copy  vpt 180 270 arc closepath fill
               vpt 0 360 arc } bind def
/C6 { BL [] 0 setdash 2 copy moveto
      2 copy  vpt 90 270 arc closepath fill
              vpt 0 360 arc closepath } bind def
/C7 { BL [] 0 setdash 2 copy moveto
      2 copy  vpt 0 270 arc closepath fill
              vpt 0 360 arc closepath } bind def
/C8 { BL [] 0 setdash 2 copy moveto
      2 copy vpt 270 360 arc closepath fill
              vpt 0 360 arc closepath } bind def
/C9 { BL [] 0 setdash 2 copy moveto
      2 copy  vpt 270 450 arc closepath fill
              vpt 0 360 arc closepath } bind def
/C10 { BL [] 0 setdash 2 copy 2 copy moveto vpt 270 360 arc closepath fill
       2 copy moveto
       2 copy vpt 90 180 arc closepath fill
               vpt 0 360 arc closepath } bind def
/C11 { BL [] 0 setdash 2 copy moveto
       2 copy  vpt 0 180 arc closepath fill
       2 copy moveto
       2 copy  vpt 270 360 arc closepath fill
               vpt 0 360 arc closepath } bind def
/C12 { BL [] 0 setdash 2 copy moveto
       2 copy  vpt 180 360 arc closepath fill
               vpt 0 360 arc closepath } bind def
/C13 { BL [] 0 setdash  2 copy moveto
       2 copy  vpt 0 90 arc closepath fill
       2 copy moveto
       2 copy  vpt 180 360 arc closepath fill
               vpt 0 360 arc closepath } bind def
/C14 { BL [] 0 setdash 2 copy moveto
       2 copy  vpt 90 360 arc closepath fill
               vpt 0 360 arc } bind def
/C15 { BL [] 0 setdash 2 copy vpt 0 360 arc closepath fill
               vpt 0 360 arc closepath } bind def
/Rec   { newpath 4 2 roll moveto 1 index 0 rlineto 0 exch rlineto
       neg 0 rlineto closepath } bind def
/Square { dup Rec } bind def
/Bsquare { vpt sub exch vpt sub exch vpt2 Square } bind def
/S0 { BL [] 0 setdash 2 copy moveto 0 vpt rlineto BL Bsquare } bind def
/S1 { BL [] 0 setdash 2 copy vpt Square fill Bsquare } bind def
/S2 { BL [] 0 setdash 2 copy exch vpt sub exch vpt Square fill Bsquare } bind def
/S3 { BL [] 0 setdash 2 copy exch vpt sub exch vpt2 vpt Rec fill Bsquare } bind def
/S4 { BL [] 0 setdash 2 copy exch vpt sub exch vpt sub vpt Square fill Bsquare } bind def
/S5 { BL [] 0 setdash 2 copy 2 copy vpt Square fill
       exch vpt sub exch vpt sub vpt Square fill Bsquare } bind def
/S6 { BL [] 0 setdash 2 copy exch vpt sub exch vpt sub vpt vpt2 Rec fill Bsquare } bind def
/S7 { BL [] 0 setdash 2 copy exch vpt sub exch vpt sub vpt vpt2 Rec fill
       2 copy vpt Square fill
       Bsquare } bind def
/S8 { BL [] 0 setdash 2 copy vpt sub vpt Square fill Bsquare } bind def
/S9 { BL [] 0 setdash 2 copy vpt sub vpt vpt2 Rec fill Bsquare } bind def
/S10 { BL [] 0 setdash 2 copy vpt sub vpt Square fill 2 copy exch vpt sub exch vpt Square fill
       Bsquare } bind def
/S11 { BL [] 0 setdash 2 copy vpt sub vpt Square fill 2 copy exch vpt sub exch vpt2 vpt Rec fill
       Bsquare } bind def
/S12 { BL [] 0 setdash 2 copy exch vpt sub exch vpt sub vpt2 vpt Rec fill Bsquare } bind def
/S13 { BL [] 0 setdash 2 copy exch vpt sub exch vpt sub vpt2 vpt Rec fill
       2 copy vpt Square fill Bsquare } bind def
/S14 { BL [] 0 setdash 2 copy exch vpt sub exch vpt sub vpt2 vpt Rec fill
       2 copy exch vpt sub exch vpt Square fill Bsquare } bind def
/S15 { BL [] 0 setdash 2 copy Bsquare fill Bsquare } bind def
/D0 { gsave translate 45 rotate 0 0 S0 stroke grestore } bind def
/D1 { gsave translate 45 rotate 0 0 S1 stroke grestore } bind def
/D2 { gsave translate 45 rotate 0 0 S2 stroke grestore } bind def
/D3 { gsave translate 45 rotate 0 0 S3 stroke grestore } bind def
/D4 { gsave translate 45 rotate 0 0 S4 stroke grestore } bind def
/D5 { gsave translate 45 rotate 0 0 S5 stroke grestore } bind def
/D6 { gsave translate 45 rotate 0 0 S6 stroke grestore } bind def
/D7 { gsave translate 45 rotate 0 0 S7 stroke grestore } bind def
/D8 { gsave translate 45 rotate 0 0 S8 stroke grestore } bind def
/D9 { gsave translate 45 rotate 0 0 S9 stroke grestore } bind def
/D10 { gsave translate 45 rotate 0 0 S10 stroke grestore } bind def
/D11 { gsave translate 45 rotate 0 0 S11 stroke grestore } bind def
/D12 { gsave translate 45 rotate 0 0 S12 stroke grestore } bind def
/D13 { gsave translate 45 rotate 0 0 S13 stroke grestore } bind def
/D14 { gsave translate 45 rotate 0 0 S14 stroke grestore } bind def
/D15 { gsave translate 45 rotate 0 0 S15 stroke grestore } bind def
/DiaE { stroke [] 0 setdash vpt add M
  hpt neg vpt neg V hpt vpt neg V
  hpt vpt V hpt neg vpt V closepath stroke } def
/BoxE { stroke [] 0 setdash exch hpt sub exch vpt add M
  0 vpt2 neg V hpt2 0 V 0 vpt2 V
  hpt2 neg 0 V closepath stroke } def
/TriUE { stroke [] 0 setdash vpt 1.12 mul add M
  hpt neg vpt -1.62 mul V
  hpt 2 mul 0 V
  hpt neg vpt 1.62 mul V closepath stroke } def
/TriDE { stroke [] 0 setdash vpt 1.12 mul sub M
  hpt neg vpt 1.62 mul V
  hpt 2 mul 0 V
  hpt neg vpt -1.62 mul V closepath stroke } def
/PentE { stroke [] 0 setdash gsave
  translate 0 hpt M 4 {72 rotate 0 hpt L} repeat
  closepath stroke grestore } def
/CircE { stroke [] 0 setdash 
  hpt 0 360 arc stroke } def
/Opaque { gsave closepath 1 setgray fill grestore 0 setgray closepath } def
/DiaW { stroke [] 0 setdash vpt add M
  hpt neg vpt neg V hpt vpt neg V
  hpt vpt V hpt neg vpt V Opaque stroke } def
/BoxW { stroke [] 0 setdash exch hpt sub exch vpt add M
  0 vpt2 neg V hpt2 0 V 0 vpt2 V
  hpt2 neg 0 V Opaque stroke } def
/TriUW { stroke [] 0 setdash vpt 1.12 mul add M
  hpt neg vpt -1.62 mul V
  hpt 2 mul 0 V
  hpt neg vpt 1.62 mul V Opaque stroke } def
/TriDW { stroke [] 0 setdash vpt 1.12 mul sub M
  hpt neg vpt 1.62 mul V
  hpt 2 mul 0 V
  hpt neg vpt -1.62 mul V Opaque stroke } def
/PentW { stroke [] 0 setdash gsave
  translate 0 hpt M 4 {72 rotate 0 hpt L} repeat
  Opaque stroke grestore } def
/CircW { stroke [] 0 setdash 
  hpt 0 360 arc Opaque stroke } def
/BoxFill { gsave Rec 1 setgray fill grestore } def
end
%%EndProlog
}}%
\begin{picture}(3600,2160)(0,0)%
{\GNUPLOTspecial{"
gnudict begin
gsave
0 0 translate
0.100 0.100 scale
0 setgray
newpath
1.000 UL
LTb
600 300 M
63 0 V
2787 0 R
-63 0 V
600 740 M
63 0 V
2787 0 R
-63 0 V
600 1180 M
63 0 V
2787 0 R
-63 0 V
600 1620 M
63 0 V
2787 0 R
-63 0 V
600 2060 M
63 0 V
2787 0 R
-63 0 V
600 300 M
0 63 V
0 1697 R
0 -63 V
1170 300 M
0 63 V
0 1697 R
0 -63 V
1740 300 M
0 63 V
0 1697 R
0 -63 V
2310 300 M
0 63 V
0 1697 R
0 -63 V
2880 300 M
0 63 V
0 1697 R
0 -63 V
3450 300 M
0 63 V
0 1697 R
0 -63 V
1.000 UL
LTb
600 300 M
2850 0 V
0 1760 V
-2850 0 V
600 300 L
1.000 UL
LT0
3087 1947 M
263 0 V
629 305 M
29 15 V
28 24 V
29 30 V
29 37 V
29 41 V
29 46 V
28 49 V
29 51 V
29 53 V
29 54 V
28 54 V
29 53 V
29 53 V
29 52 V
29 50 V
28 48 V
29 46 V
29 43 V
29 41 V
29 38 V
28 35 V
29 32 V
29 29 V
29 26 V
28 23 V
29 19 V
29 17 V
29 14 V
29 10 V
28 8 V
29 5 V
29 3 V
29 0 V
29 -3 V
28 -4 V
29 -7 V
29 -8 V
29 -11 V
29 -12 V
28 -13 V
29 -16 V
29 -16 V
29 -17 V
28 -19 V
29 -19 V
29 -21 V
29 -21 V
29 -21 V
28 -22 V
29 -23 V
29 -22 V
29 -23 V
29 -23 V
28 -23 V
29 -24 V
29 -23 V
29 -23 V
28 -22 V
29 -23 V
29 -22 V
29 -22 V
29 -22 V
28 -21 V
29 -21 V
29 -20 V
29 -20 V
29 -20 V
28 -19 V
29 -18 V
29 -18 V
29 -18 V
29 -17 V
28 -16 V
29 -16 V
29 -16 V
29 -15 V
28 -14 V
29 -14 V
29 -13 V
29 -13 V
29 -13 V
28 -12 V
29 -12 V
29 -11 V
29 -10 V
29 -11 V
28 -9 V
29 -10 V
29 -9 V
29 -8 V
28 -9 V
29 -8 V
29 -7 V
29 -7 V
29 -7 V
28 -7 V
29 -6 V
29 -6 V
1.000 UL
LT1
3087 1847 M
263 0 V
629 304 M
29 13 V
28 18 V
29 25 V
29 28 V
29 32 V
29 35 V
28 36 V
29 37 V
29 37 V
29 38 V
28 36 V
29 35 V
29 34 V
29 32 V
29 30 V
28 28 V
29 26 V
29 22 V
29 21 V
29 17 V
28 16 V
29 12 V
29 10 V
29 8 V
28 5 V
29 3 V
29 1 V
29 -1 V
29 -3 V
28 -5 V
29 -6 V
29 -8 V
29 -9 V
29 -11 V
28 -11 V
29 -12 V
29 -14 V
29 -14 V
29 -14 V
28 -15 V
29 -16 V
29 -15 V
29 -16 V
28 -16 V
29 -16 V
29 -16 V
29 -16 V
29 -16 V
28 -16 V
29 -15 V
29 -15 V
29 -15 V
29 -15 V
28 -14 V
29 -14 V
29 -13 V
29 -13 V
28 -13 V
29 -12 V
29 -12 V
29 -11 V
29 -11 V
28 -11 V
29 -10 V
29 -9 V
29 -10 V
29 -8 V
28 -9 V
29 -8 V
29 -8 V
29 -7 V
29 -7 V
28 -6 V
29 -7 V
29 -6 V
29 -5 V
28 -6 V
29 -5 V
29 -5 V
29 -4 V
29 -4 V
28 -4 V
29 -4 V
29 -4 V
29 -3 V
29 -3 V
28 -3 V
29 -3 V
29 -3 V
29 -2 V
28 -3 V
29 -2 V
29 -2 V
29 -2 V
29 -2 V
28 -2 V
29 -1 V
29 -2 V
1.000 UL
LT2
3087 1747 M
263 0 V
629 303 M
29 10 V
28 14 V
29 18 V
29 21 V
29 22 V
29 24 V
28 24 V
29 24 V
29 23 V
29 22 V
28 20 V
29 20 V
29 17 V
29 15 V
29 13 V
28 11 V
29 9 V
29 7 V
29 5 V
29 4 V
28 1 V
29 0 V
29 -2 V
29 -3 V
28 -5 V
29 -5 V
29 -7 V
29 -7 V
29 -8 V
28 -9 V
29 -9 V
29 -10 V
29 -10 V
29 -10 V
28 -10 V
29 -10 V
29 -11 V
29 -10 V
29 -10 V
28 -10 V
29 -9 V
29 -10 V
29 -9 V
28 -9 V
29 -8 V
29 -8 V
29 -8 V
29 -8 V
28 -7 V
29 -7 V
29 -6 V
29 -6 V
29 -6 V
28 -6 V
29 -5 V
29 -5 V
29 -4 V
28 -5 V
29 -4 V
29 -3 V
29 -4 V
29 -3 V
28 -3 V
29 -3 V
29 -3 V
29 -2 V
29 -3 V
28 -2 V
29 -2 V
29 -2 V
29 -2 V
29 -1 V
28 -2 V
29 -1 V
29 -1 V
29 -1 V
28 -1 V
29 -1 V
29 -1 V
29 -1 V
29 -1 V
28 -1 V
29 0 V
29 -1 V
29 0 V
29 -1 V
28 0 V
29 -1 V
29 0 V
29 0 V
28 -1 V
29 0 V
29 0 V
29 -1 V
29 0 V
28 0 V
29 0 V
29 0 V
1.000 UL
LT3
3087 1647 M
263 0 V
600 300 M
29 8 V
29 25 V
28 41 V
29 58 V
29 74 V
29 91 V
29 108 V
28 123 V
29 141 V
29 157 V
29 173 V
28 190 V
29 206 V
29 223 V
17 142 V
stroke
grestore
end
showpage
}}%
\put(3037,1647){\makebox(0,0)[r]{Klassisk $k_BT=0.6$ eV}}%
\put(3037,1747){\makebox(0,0)[r]{$k_BT=0.4$ eV}}%
\put(3037,1847){\makebox(0,0)[r]{$k_BT=0.5$ eV}}%
\put(3037,1947){\makebox(0,0)[r]{$k_BT=0.6$ eV}}%
\put(2025,50){\makebox(0,0){Energi $h\nu$ [eV]}}%
\put(100,1180){%
\special{ps: gsave currentpoint currentpoint translate
270 rotate neg exch neg exch translate}%
\makebox(0,0)[b]{\shortstack{$M_{\nu}(h\nu)$ [eV/nm$^2$]}}%
\special{ps: currentpoint grestore moveto}%
}%
\put(3450,200){\makebox(0,0){5}}%
\put(2880,200){\makebox(0,0){4}}%
\put(2310,200){\makebox(0,0){3}}%
\put(1740,200){\makebox(0,0){2}}%
\put(1170,200){\makebox(0,0){1}}%
\put(600,200){\makebox(0,0){0}}%
\put(550,2060){\makebox(0,0)[r]{2e-06}}%
\put(550,1620){\makebox(0,0)[r]{1.5e-06}}%
\put(550,1180){\makebox(0,0)[r]{1e-06}}%
\put(550,740){\makebox(0,0)[r]{5e-07}}%
\put(550,300){\makebox(0,0)[r]{0}}%
\end{picture}%
\endgroup
\endinput

\caption{Figuren viser frekvensfordelingen fra Plancks kvantiseringspostulat
i likning (\ref{eq:plankcfirst}) og den klassiske fordelingsfunksjonen fra likning 
(\ref{eq:klassisk}). Legg merke til at energi er i enhet eV og frekvensfordelingen har enheten eV/nm$^2$. \label{fig:blackbody}. }
\end{figure}

Det klassiske eksempel p\aa\ en slik 
frekvensfordeling  $M_{\nu}(T)$ var gitt ved str\aa ling fra et 
s\aa kalt svart legeme. Et svart legeme er et idealisert objekt som 
ikke reflekterer noe av den innkommende  e.m.~str\aa ling. All innkommende
e.m.~str\aa ling blir absorbert. Grunnen til at det kalles svart legeme
var at ved lave temperaturer (tenk igjen p\aa\ ei kokeplate som nettopp
er satt p\aa\  ) s\aa\ forble legemet m\o rkt, selv om det sendte ut termisk
str\aa ling. Den var bare ikke synlig for oss.
Frekvensfordelingen til et svart legeme  er uavhengig av materiale, slik at
dets frekvensfordeling er en universell funksjon av frekvens $\nu$ og
temperatur $T$. P\aa\ slutten av 1800-tallet hadde en gjennomf\o rt
flere eksperiment ved \aa\ observere utstr\aa lt e.m.~energi fra modeller
som skulle representere et s\aa kalt svart legeme.
Modellen var et hulrom som ble varmet opp til en bestemt temperatur.
Atomene
i materialet til dette hulrommet ble da satt i svingninger og
sendte ut e.m.~str\aa ling (mere om dette i FYS2160). En kan tenke
seg atomene som harmoniske oscillatorer som vibrerer og sender ut str\aa ling.
Ved termisk likevekt var hulrommet fylt av e.m.~str\aa ling.
{\bf Teknisk sett vil dette svare til st\aa ende e.m.~b\o lger} og en kan da
regne ut energien til det e.m.~feltet i et slikt hulrom.  
Hulrommet hadde et hull, hvis st\o rrelse var mye mindre en hulrommets
overflate. E.m.~str\aa ling ble emittert fra dette hullet som ideelt sett
skal representere et svart legeme. 
Fordelen med dette oppsettet var at det lot seg b\aa de gjennomf\o re
eksperimentelt og at en kunne regne ut teoretisk frekvensfordelingen.

En kan da vise (se kap 2-1 og sidene 83-85 i l\ae reboka) at 
\be
   M_{\nu}(T)=\frac{2\pi\nu^2}{c^2}\left\langle E\right\rangle ,
\ee
hvor $\left\langle E\right\rangle $ er den gjennomsnittlige energien per svingemode til det
eletromagnetiske feltet i hulrommet. Dette feltet skal igjen gjenspeile 
svingingene til atomene i materialet til hulrommet.
Den tilsvarende radiansen var da gitt ved
\be
   M(T)=\sigma T^4,
\ee
hvor $\sigma$ er en konstant. Dette uttrykket kalles Stefan-Boltzmanns
lov. Det var Stefan som i 1879 foreslo basert p\aa\ data at
radiansen for et svart legeme skulle v\ae re proporsjonal med
$T^4$.  
Klassisk fysikk, se nedenfor, ga at 
\be
   \left\langle E\right\rangle =\frac{3k_BT}{2},
\ee
hvor $k_B$ er Boltzmanns konstant.
Dvs. 
\be
   M_{\nu}(T)=\frac{\pi\nu^2}{c^2}3k_BT.
   \label{eq:klassisk}
\ee
Dersom vi integrerer det siste uttrykket for \aa\ finne radiansen
\be 
   M(T)=\int_0^{\infty} \frac{\pi\nu^2}{c^2}3k_BT d\nu,
   \label{eq:divrad}
\ee
ser vi at radiansen divergerer, i strid med den empiriske
oppf\o rselen i Stefan-Boltzmanns lov. 

Plancks hypotese (se nedenfor) ga
\be 
   \left\langle E\right\rangle =\frac{h\nu}{e^{h\nu/k_BT}-1},
\ee
og dermed 
\be
   M_{\nu}(T)=\frac{2\pi\nu^2}{c^2}\frac{h\nu}{e^{h\nu/k_BT}-1},
   \label{eq:plankcfirst}
\ee
og samsvar med eksperiment. $h$ er Plancks konstant.
Vi ser at denne funksjonen har det riktige
forl\o p b\aa de ved sm\aa\  og store verdier av $\nu$, se
figur \ref{fig:blackbody}.
I denne figuren har vi valgt enheter eV og nm. Grunnen er
at dersom vi \o nsker \aa\ sette naturkonstantene $k_B$, $h$ og $c$
i enheter av henholdsvis J/K, Js og m/s$^2$, 
kan det lett lede til tap av presisjon i numeriske beregninger   
av frekvensfordelingen. For \aa\ konvertere til disse enhetene har vi 
multiplisert siste uttrykk med $h^2$ i teller og nevner
\be
   M_{\nu}(T)=\frac{2\pi}{h^2c^2}\frac{(h\nu)^3}{e^{h\nu/k_BT}-1},
\ee
og ved \aa\ sette $x=h\nu$ finner vi
\be
   M_{x}(T)=\frac{2\pi}{h^2c^2}\frac{x^3}{e^{x/k_BT}-1},
\ee
som gir med $hc=1240$ eVnm 
\be
   M_{x}(T)=\frac{2\pi}{(1240)^2}\frac{x^3}{e^{x/k_BT}-1}.
\ee
Temperaturen er ogs\aa\ uttrykt i eV. En temperatur p\aa\ 
1 eV (husk at $1 \mathrm{eV} = 1.6\times 10^{-19}$ J )svarer derfor til 
\[
   T=1.60\times 10^{-19}/(1.38\times 10^{-23})= 11594 \hspace{0.1cm}\mathrm{K}.
\]
Vi kan utlede 
Stefan-Boltzmanns lov 
vha.\ Plancks
kvantiseringshypotese. Vi trenger da
\be
   M(T)=\int_0^{\infty}M_{\nu}(T)d\nu=\int_0^{\infty}\frac{2\pi\nu^2}{c^2}\frac{h\nu}{e^{h\nu/k_BT}-1} d\nu,
\ee
og med variabel bytte $x=h\nu/k_BT$ f\aa r vi
\be
M(T)=\frac{2\pi k_B^4}{c^2h^3}T^4\int_0^{\infty}\frac{x^3}{e^{x}-1} dx,
\ee
og med 
\be
   \int_0^{\infty}\frac{x^3}{e^{x}-1} dx=\frac{\pi^4}{15},
\ee
f\aa r vi 
\be
M(T)=\sigma T^4,
\label{eq:korrekt_sb}
\ee
med
\be
\sigma=\frac{2\pi^5k_B^4}{15c^2h^3}=5.676\times 10^{-8}\hspace{0.1cm}\mathrm{W/m^2K^4},
\ee
i godt samsvar med verdier fra empiriske  data.



{\bf Det viktige budskapet er at Planck forlot den klassiske
m\aa ten \aa\ regne ut den midlere energien $\left\langle E\right\rangle$. 
Istedet for \aa\
tillatte alle mulige verdier av energien, krevde han at kun bestemte
diskrete verdier kunne tas med i utregningen 
av $\left\langle E\right\rangle$ .} 

Dette leder oss til en digresjon om fordelingsfunksjoner, litt
statistikk og hvorfor vi ofte betrakter kun midlere verdier av st\o rrelser
i fysikk. Mye av dette vil dere f\aa\ i st\o rre detalj i FYS2160.

Det er dog en ting som er viktig \aa\ ha klart for seg. 
Planck hadde dataene foran seg! og visste dermed hva svaret
skulle v\ae re og pr\o vde \aa\ tilpasse dataene med ulike
funksjoner. Det er p\aa\ dette viset vi ofte g\aa r fram i fysikk.
I mange tilfeller har vi data fra eksperiment som vi ikke kan forklare
med gjeldende teorier, andre ganger har vi teoretiske prediksjoner
p\aa\ fenomen som ikke er m\aa lt/observert. Fysikk representerer 
syntesen 
av eksperiment og teori med den m\aa lsetting \aa\ avdekke
bevegelseslovene til naturen. 

\subsection{Maxwells hastighets og energifordeling, klassisk}

Materialet her er bakgrunnsmateriale, og gjennomg\aa s i dybde i FYS2160.
Men vi trenger noen begrep for \aa\ forst\aa\ hvordan en kan regne
ut midlere energi $\left\langle E\right\rangle $. I tillegg, vil begrepet om sannsynlighetsfordeling
og normering v\ae re sentrale i v\aa r diskusjon av kvantemekanikken.
Det er en annen viktig grunn til at vi ofte bruker midlere verdier. Det skyldes
at fysiske systemer involver s\aa\ mange frihetsgrader, bare tenk p\aa\
Avogadros tall!, at vi ikke vil v\ae re i stand til \aa\ regne
med alle. En fullstendig mekanisk beskrivelse av et makroskopisk system som
en gass av hydrogenatomer er uoverkommelig, og heller ikke \o nskelig.
Nedskrevet p\aa\ papir m\aa tte det atskillige billass med begynnelsesbetingelser
til for \aa\ beskrive bare et gram av en slik hydrogen gass. 
Vi er egentlig bare interessert i visse midlere egenskaper ved atomenes
dynamikk. 
Derfor betrakter vi som regel et system ved likevekt og ser p\aa\
midlere st\o rrelser, slik som midlere hastighet, energi osv.
Makroskopiske systemer i likevekt beskrives f.eks.~med termodynamiske
st\o rrelser som trykk $P$ og temperatur $T$.  
For \aa\ regne ut slike st\o rrelser trenger vi en del begrep fra
sannsynlighetsl\ae re og statistikk.

 
Hvis vi tenker p\aa\ statistikk og sannsynlighetsregning,
s\aa\ innf\o rer vi begrepet sannsynlighet for at ei hending kan skje.
F.eks., sannsynligheten $P$ for \aa\ f\aa\ 1,2,3,4,5 eller 
6 ved et terningkast
er gitt ved $P=1/6$. Kaller vi et slikt utfall av terningkast for ei hending,
gir summen over alle hendinger
\be
   \sum_{i=1}^{6}P_i=\sum_{i=1}^{6}\frac{1}{6}=1,
\ee  
summen av alle sannsynligheter er lik 1. Vi sier da ogs\aa\ at
summen er normalisert. Hvis vi \o nsker \aa\ finne
gjennomsnittsverdien til en st\o rrelse $x$ er den definert ved
\be 
   \left\langle x\right\rangle =\frac{\sum_i xP_i}{\sum_i P_i},
\ee
hvor leddet i nevneren s\o rger for at fordelingen over sannsynligheter
er normert. Sannsynlighetene for terningkastene tar kun 
diskrete verdier. Vi kaller derfor fordelingen ovenfor for 
ei diskret fordeling.
Men vi kan ogs\aa\ ha kontinuerlige sannsynlighetsfordelinger.
Et viktig eksempel i fysikk er Maxwells hastighetsfordeling\footnote{Denne
fordelingen utledes ikke her, og er ikke del av pensum}.
Den er gitt ved  
             \be 
                \frac{df}{dv}=4\pi\left(\frac{m}{2\pi k_BT}\right)^{3/2}
                              v^2e^{-mv^2/2k_BT}
                \label{eq:maxwell}
             \ee
med 
             \be
                \int_0^{\infty}\frac{df}{dv}dv=1
             \ee
             dvs at den er normalisert. 
Tolkningen av $df/dvdv$ er at den gir br\o kdelen av partikler som
har en hastighet mellom $v$ og $v+dv$ ved en gitt temperatur $T$.

Dersom vi \o nsker \aa\ regne ut en midlere hastighet trenger vi \aa\ regne
ut 
\be
  \left\langle v\right\rangle = \frac{\int_0^{\infty} v df/dv dv}{\int_0^{\infty}df/dv dv}.
\ee
For Maxwells fordeling i likning (\ref{eq:maxwell}) er nevneren
allerede normalisert og lik 1. Midlere hastighet blir
\be
\left\langle v\right\rangle =\int_0^{\infty}v4\pi\left(\frac{m}{2\pi k_BT}\right)^{3/2}
                              v^2e^{-mv^2/2k_BT}dv.
\ee
Vi trenger alts\aa\ \aa\ regne ut integral av typen
\be
   \int_0^{\infty}v^ne^{-\beta v^2}dv.
\ee
Disse kan finnes slik: for $n$ ulike, velg ny variabel $t=\beta v^2$
og du f\aa r et integral av typen
\be
   \int_0^{\infty}(t/\beta)^{(n-1)/2}e^{-t}dt,
\ee
hvor $n-1$ er et like tall. For like $n$ trenger vi kun \aa\ 
foreta $n/2$ differensiasjoner m.h.p.\ $\beta$ av 
integralet $\int_0^{\infty}e^{-\beta v^2}dv=1/2\sqrt{\pi/\beta}$.
\O nsker vi midlere hastighet $\left\langle v\right\rangle $ finner vi da
\be
\left\langle v\right\rangle =\sqrt{\frac{8k_BT}{\pi m}},
\ee
eller dersom vi \o nsker $\left\langle v^2\right\rangle $ finner vi 
\be
  \left\langle v^2\right\rangle =\frac{3k_BT}{m},
\ee
noe som igjen gir en midlere energi p\aa\
\be
   \left\langle E\right\rangle =\frac{1}{2}m\left\langle v^2\right\rangle =\frac{3k_BT}{2}.
\ee
Det var dette resultatet som ga det klassiske resultat for 
frekvensfordelingen
av e.m.~str\aa ling.
Vi kan ogs\aa\ regne ut det siste uttrykk vha.\ energifordelingen
\be
    \frac{df}{dE}.
\ee
Bruker vi $E=1/2mv^2$ har vi $dE=mvdv$ noe som igjen gir
\be
\frac{df}{dE}=\frac{dv}{dE}\frac{df}{dv}=\frac{4\pi}{mv}
\left(\frac{m}{2\pi k_BT}\right)^{3/2}
                              v^2e^{-mv^2/2k_BT},
\ee
og setter vi inn at $v^2=2E/m$ f\aa r vi 
 energifordelingen
             \be
                \frac{df}{dE}=2\pi\left(\frac{1}{\pi k_BT}\right)^{3/2}
                              \sqrt{E}e^{-E/k_BT},
             \ee
             med
             \be
                \int_0^{\infty}\frac{df}{dE}dE=1,
             \ee
dvs.\ at den er normalisert.
\O nsker vi s\aa\ \aa\ finne midlere energi har vi
 \be
       \left\langle E\right\rangle =\int_0^{\infty}E\frac{df}{dE}dE=\frac{3k_BT}{2},
 \ee
som er det klassiske resultat.
Innsatt i uttrykket for frekvensfordelingen $M_{\nu}(T)$ i
likning (\ref{eq:klassisk}) f\aa r vi uttrykket til Rayleigh og Jeans.
Vi fant ogs\aa\ at radiansen divergerte, se igjen likning (\ref{eq:divrad}).

Vi kan oppsumere dette avsnittet med f\o lgende. 
Vi kan skrive energifordelingsfunksjonen som
\be 
                \frac{df}{dE}=g(E)e^{-E/k_BT},
\ee
hvor $g(E)$ er en funksjon som kalles tettheten av tilstander med en gitt
energi $E$, men eksponensial faktoren $e^{-E/k_BT}$ kalles
Maxwell-Boltzmann faktoren og uttrykker sannsynligheten for \aa\ finne
systemetet i en tilstand med gitt energi $E$.

{\bf Og n\aa\ kommer det som er viktig:}\newline
Klassisk s\aa\ tillater vi alle verdier av $E$ i beregningen
av forventningsverdien $\left\langle E\right\rangle $, dvs vi har et integral
\be
   \left\langle E\right\rangle =\frac{\int_0^{\infty}Eg(E)e^{-E/k_BT}dE}
{\int_0^{\infty}g(E)e^{-E/k_BT}dE},
\label{eq:eexpt}
\ee
hvor alle $E$ er tillatt. Det resulterte i den s\aa kalte ultrafiolette
katastrofen.
      


\subsection{Plancks hypotese}
Plancks radikale hypotese var \aa\ foresl\aa\ at kun bestemte
energier i beregningene av $\left\langle E\right\rangle $ er tillatt, dvs. 
             \be
               E_n(\nu)=n h \nu, \hspace{1cm} n=0,1,2,\dots
             \ee
             hvor $\nu$ er frekvensen til den e.m.~str\aa lingen 
             og $h$ er en universell
             konstant var tillatt . 
Vi sier da at energien er kvantisert! De tillate energitilstandene
             kalles {\bf kvantetilstander} og heltallet $n$ kalles 
             et {\bf kvantetall}.
Det betyr at forventningsverdien i likning (\ref{eq:eexpt}) blir
\be
   \left\langle E\right\rangle =\frac{\sum_{n=0}^{\infty}E_ng(E_n)e^{-E_n/k_BT}}
{\sum_{n=0}^{\infty}g(E_n)e^{-E_n/k_BT}}.
\ee
Setter vi inn $E_n=nh\nu$ og tar bort leddet $g(E_n)$ f\aa r vi
\be
   \left\langle E\right\rangle =\frac{\sum_{n=0}^{\infty}nh\nu e^{-nh\nu/k_BT}}
{\sum_{n=0}^{\infty}e^{-nh\nu/k_BT}}.
\ee
Setter vi s\aa\ $x=h\nu/k_BT$ f\aa r vi
\be
   \left\langle E\right\rangle =k_BT\frac{\sum_{n=0}^{\infty}nxe^{-nx}}
{\sum_{n=0}^{\infty}e^{-nx}}.
\ee
Summen 
\be
   \sum_{n=0}^{\infty}e^{-nx}=\frac{1}{1-e^{-x}},
\ee
og bruker vi at
\be
    \sum_{n=0}^{\infty}nxe^{-nx}=-x\frac{d}{dx}\sum_{n=0}^{\infty}e^{-nx},
\ee
f\aa r vi at
\be 
  \left\langle E\right\rangle =k_BT x\frac{e^{-x}}{1-e^{-x}}=\frac{h\nu}{e^{h\nu/k_BT}-1},
\ee
slik at frekvensfordelingen blir 
\[
   M_{\nu}(T)d\nu=\frac{2\pi\nu^2}{c^2}\frac{h\nu}{e^{h\nu/k_BT}-1}d\nu,
\]
eller dersom vi \o nsker uttrykket gitt ved b\o lgelengden $\lambda$
(vis dette)
har vi
\be  
   M_{\lambda}(T)d\lambda=\frac{2\pi hc^2}{\lambda^5}\frac{1}{e^{hc/\lambda k_BT}-1}d\lambda.
\ee
Som vist i likning (\ref{eq:korrekt_sb}), ga dette oss ogs\aa\ et korrekt
uttrykk for Stefan-Boltzmanns lov. 

Det spesielle med e.m.~str\aa ling i et hulrom er at det gir opphav
til st\aa ende e.m.~b\o lger som utviser enkle harmoniske
svingninger. Det s\ae regne  med denne type problem er at energien
antar diskrete verdier\footnote{I tilknyting Schr\"odingers likning,
skal vi vise dette for ei fj\ae r som svinger og er fastspent i begge
ender.}. 

\begin{center}
\shabox{\parbox{14cm}{Plancks hypotese (1900) kan dermed formuleres som f\o lger: Enhver 
             fysisk st\o rrelse som utviser enkle harmoniske
             svingninger har energier som tilfredsstiller 
             \[
               E_n(\nu)=n h \nu, \hspace{1cm} n=0,1,2,\dots
             \]
             hvor $\nu$ er frekvensen til svingningen og $h$ er en universell
             konstant. 
}}\end{center}
En idealisert pendel utviser ogs\aa\ enkle harmoniske svningninger,
og vi kan jo da stille sp\o rsm\aa let om hvorfor kan vi beskrive
en pendel vha.~klassisk fysikk og ikke e.m.~str\aa ling ?

Dette sp\o rsm\aa let belyser et viktig aspekt ved v\aa r forst\aa else av
fysikk og framgangsm\aa ter for \aa\ studere fysiske systemer.
{\em Det dreier seg om energiskalaer og st\o rrelsen p\aa\ systemet.}
Gang p\aa\ gang vil vi komme over eksempler p\aa\ det i kurset.
Dette dikterer igjen hvilken fysisk teori som er anvendbar.

La oss bruke pendelen til \aa\ se n\ae rmere p\aa\ dette.
Anta at vi har en idealisert pendel, vi ser bort fra luftmostand osv.
Vi gir pendelen en masse 
$m=0.01$ 
kg, en lengde p\aa\ $l=0.1$ m og vi tillater
at den kan svinge ut en vinkel p\aa\ maks $\theta=0.1$ rad.

Vi sp\o r deretter om hvor stor energiforskjellen er mellom
kvantetilstander
m\aa lt i forhold til maks potensiell energi pendelen kan ha n\aa r vi anvender
Plancks hypotese.
Maks potensiell energi $E$ er gitt ved 
\be
  E=mgs=mgl(1-cos\theta)=5\times 10^{-5}\hspace{0.1cm} \mathrm{J},
\ee
hvor $s$ er maksimal h\o yde som pendelen kan oppn\aa\ ved utsving
og $g$ er tyngdeakselerasjonen.
Svingefrekvensen $\nu$ finner vi ogs\aa\ ved \aa\ anvende en velkjent traver
\be
   \nu=\frac{1}{2\pi}\sqrt{\frac{g}{l}}=1.6\hspace{0.1cm} \mathrm{s}^{-1}.
\ee

N\aa\ skal vi anvende Plancks hypotese for \aa\ regne energiforskjellen
$\Delta E$
\be
   \Delta E=(n+1)h\nu - nh\nu=h\nu\sim 10^{-33}\hspace{0.1cm} \mathrm{J}.
\ee
Forholdet 
\be
   \frac{\Delta E}{E}\sim 10^{-29}!
\ee
viser at
vi kunne praktisk talt satt denne energiforskjellen lik null.
Det finnes ikke noe m\aa leinstrument som kan m\aa le en slik
energiforskjell.
I klassisk fysikk kan vi derfor sette $h=0$.

Problemet dukker opp n\aa r 
$\Delta E/E$ ikke er neglisjerbar. For h\o gfrekvent e.m.~str\aa ling
er dette tilfelle, og da kunnne ikke klassisk fysikk lenger forklare
fenomenene. Dersom $E$ er s\aa\ liten at 
$\Delta E=h\nu$ er p\aa\ samme st\o rrelse, da er vi p\aa\ energiskalaer
som ikke lenger kunne og kan forklares uten at ny teori anvendes.

Vi skal se p\aa\ mange flere slike eksempler i dette kurset. 


\section{Fotoelektrisk effekt}


Den fotoelektriske effekt\footnote{Svarer til kap 2-5 i boka, sidene 99-103. I diskusjonen her henviser jeg til figurer i l\ae reboka.} ble bla.~studert  av Hertz i 1886 og 1887, og ble da brukt
som en bekreftelse p\aa\ eksistensen av e.m.~b\o lger og av Maxwells e.m.~teori
for lysforplanting. Figur \ref{fig:fotoelektrisk} 
viser et oppsett for m\aa ling av fotoelektrisk
effekt.
\begin{figure}[h]
\begin{center}
{\centering
\mbox
{\psfig{figure=fotoelectric.ps,height=6cm,width=8cm}}
}
\end{center}
I tillegg til dette fant en ogs\aa\ at den kinetiske energien $K_{max}$
til de utsendte elektronene var uavhengig av intensiteten til str\aa lingen
og at det var en nedre frekvens den e.m.~str\aa lingen kunne ha 
for utsending av elektroner. I tillegg, utviste 
det p\aa satte 
potensialet $V_S$ ogs\aa\ en minste verdi for utsending av elektroner.
Figur \ref{fig:fotoelektrisk2} viser skjematisk  den resulterende 
fotostr\o mmen som funksjon av p\aa satt spenning og kinetisk energi
som funksjon av frekvensen til den innkommende e.m.~str\aa ling. 
\caption{Skjematisk oppsett for fotoelektrisk effekt. \label{fig:fotoelektrisk}}
\end{figure}
\begin{figure}[h]
\begin{center}
{\centering
\mbox
{\psfig{figure=exptfoto.ps,height=6cm,width=12cm}}
}
\end{center}
\caption{Fotostr\o m som funksjon av p\aa satt spenning og kinetisk energi
som funksjon av frekvensen til den innkommende e.m.~str\aa ling. \label{fig:fotoelektrisk2}}
\end{figure}

Oppsumert var   det
tre viktige egenskaper ved fotoelektrisk effekt som ikke
kunne forklares vha.~e.m.~b\o lgeteori for lys:
\begin{enumerate}
       \item  N\aa r intensiteten til lysstr\aa len \o kes, skal
ogs\aa\ amplituden til den oscillerende elektriske 
vektoren ${\bf E}$ \o ke. Siden kraften feltet ut\o ver p\aa\
et elektron er $e{\bf E}$, burde ogs\aa\ den kinetiske
energien til elektronene \o ke. Men eksperiment viste at
$K_{max}=eV_S$ var uavhengig av intensiteten. Dette har blitt utf\o rlig 
uttestet
for et  intesitetssprang p\aa\ $10^{7}$.  
\item I henhold til klassisk e.m.~teori, skal den fotoelektriske
effekt forekomme for enhver frekvens, gitt at lyset er intenst
nok til \aa\ gi den n\o dvendige energien til elektronene.
Dette var ikke tilfelle, jfr.~oppdagelsen av  
en nedre frekvens $\nu_0$. For lavere
frekvenser forekommer ikke fotoelektrisk effekt, et resultat som er uavhengig
av intensiteten til lyset.  
\item I henhold til klassisk teori, skulle det, n\aa r lys
      faller inn p\aa\ et materiale, ta litt tid fra n\aa r elektronene
      begynner \aa\ absorbere e.m.~str\aa ling til de slipper fri
      fra materialet. {\em Heller ikke noen slik tidsforskjell er detektert}. 
\end{enumerate}
La oss se p\aa\ det siste f\o rst. Anta at vi har ei plate av kalium
som er 1 m fra en lyskilde som sender ut e.m.~str\aa ling med
effekt 1 W$=$1 J/s. Vi antar deretter at et elektron i denne kaliumplaten
opptar et sirkelformet omr\aa de med radius $10^{-10}$ m.
Energien som trengs for \aa\ l\o srive det svakest bundne elektron
i kalium er 2.1 eV. Sp\o rsm\aa let vi stiller oss da er hvor lang tid
tar det f\o r dette ene elektronet oppn\aa r en energi p\aa\
2.1 eV n\aa r vi sender e.m.~str\aa ling fra lyskilden.

Tenker vi oss at lyskilden sender ut e.m.~str\aa ling som sf\ae riske
b\o lger, vet vi at overflaten til denne lyskjeglen ved 1 m fra lyskilden 
er gitt ved $A=4\pi 1^2$ m$^2$. Arealet elektronet opptar er 
$A_e=\pi r^2=\pi 10^{-20}$ m$^2$. Total energi $R$ som treffer $A_e$ per sekund
er da
\be
   R= 1\hspace{0.1cm}\mathrm{W}\frac{\pi 10^{-20}}{4\pi}=0.015\hspace{0.1cm}\mathrm{eV/s},
\ee
som igjen betyr at vi trenger
\be
   t=\frac{2.1}{0.015}\hspace{0.1cm}\mathrm{s}\sim 135\hspace{0.1cm}\mathrm{s},
\ee
for at dette elektronet skulle f\aa\ nok energi til \aa\ kunne l\o srives
fra metallplaten. En slik tidsforsinkelse er aldri observert,
tvertimot, elektronene blir emittert momentant.

For \aa\ l\o se de ovennevnte problemene med fotoelektrisk effekt, foreslo 
Einstein i 1905\footnote{Samme \aa r som han utviklet relativitetsteorien.} 
at den e.m.~energien er kvantisert
i konsentrerte deler (energibunter), som senere er blitt til partikler
med null masse som reiser med lysets fart, fotonene.
Han antok at energien til et slikt foton var bestemt
av dets frekvens
\be
   E=h\nu.
\ee
Han antok ogs\aa\ at i den fotoelektriske effekt
blir denne energibunten (fotonet) fullstendig
absorbert av elektronet. Energibalansen uttrykkes ved
\be
K_{max}=h\nu-w
\ee
hvor $w$ er arbeidet som kreves for \aa\ fjerne et elektron fra metallet.

Dersom vi ser p\aa\ de svakest bundne elektroner har vi
\be
K_{max}=h\nu-w_0,
\ee
hvor $w_0$ kalles arbeidsfunksjonen, den minste energi som trengs
for \aa\ fjerne det svakest bundne elektron for \aa\ unnslippe de
tiltrekkende kreftene (Coulomb) som binder et elektron til et metall.
Arbeidsfunksjonen $w_0$ er spesifikk for ethvert materiale og har typiske
verdier p\aa\ noen f\aa\ eV.

N\aa r det gjelder problemene med klassisk teori, 
s\aa\ kan vi se fra de to siste
likningene at hva ang\aa r punkt 1), s\aa\ er det n\aa\ samsvar med eksperiment
og teori. Det faktum at $K_{max}$ er proporsjonal med 
$E=h\nu$ viser at den kinetiske energien elektronene har 
er uavhengig av intensiteten til den
e.m.~str\aa lingen. Dersom vi dobler intensiteten, s\aa\ p\aa virker ikke
det energien til et foton, som er gitt ved kun $E=h\nu$.

Hva ang\aa r punkt 2), ser vi at vi kan bestemme den minste frekvensen
ved \aa\ sette den kinetiske energien til det frigjorte elektron lik null.
Da har vi 
\be
h\nu_0=w_0,
\ee
og kan dermed forklare observasjonen av en minste tillatt frekvens

Innvending nummer tre kan ogs\aa\ l\o ses dersom en tenker seg at det
er et foton som treffer elektronet, og overf\o rer det meste
av sin energi til elektronet. Da trenger en ikke \aa\ bombardere
med jevnt fordelt str\aa ling metallplaten av kalium. 
I tillegg er antall fotoner som sendes inn enormt. En enkel betraktning
kan hjelpe oss her.
Anta at vi sender inn monokromatisk (ensfarget) gult lys med
b\o lgelengde $\lambda=5890$ \AA\, hvor $1$ \AA\ $=10^{-10}$ m.
Effekt per arealenhet $M$ til flaten til lyskjeglen 1 m fra lyskilden er
da (dvs.~e.m.~energi per areal per sekund)
\be
    M= \frac{1\hspace{0.1cm}\mathrm{J/s}}{4\pi\hspace{0.1cm}\mathrm{m}^2}=
        8\times 10^{-2} \hspace{0.1cm}\mathrm{J/m^2s}=5\times 10^{17} \hspace{0.1cm}\mathrm{eV/m^2s},
\ee
og regner vi ut energien til hvert foton har vi
\be 
   E=h\nu =\frac{hc}{\lambda}=3.4\times 10^{-19} \hspace{0.1cm}\mathrm{J}=2.1 \hspace{0.1cm}\mathrm{eV}.
\ee
Totalt antall fotoner $N$ per areal per sekund blir
\be
   N=\frac{5\times 10^{17} \hspace{0.1cm}\mathrm{eV/m^2s}}{2.1 \hspace{0.1cm}\mathrm{eV}}= 2.4\times 10^{17}\hspace{0.1cm}\mathrm{foton/m^2s},
\ee
som betyr at sannsynligheten for at et foton treffer et elektron og
overf\o rer det meste av sin energi er noks\aa\ stor.

Merk at fotonene blir absorbert i den fotolektriske prosess.
Det betyr at dersom vi skal bevare bevegelsesmengde og energi, s\aa\
m\aa\ elektronene v\ae re bundet til atomet/metallet. Vi skal diskutere
dette etter avsnittet om Compton effekten.





\section{R\"ontgen str\aa ling}

R\"ontgen str\aa ling\footnote{Henvisning til boka er kap 2-6, sidene 
103-107.} svarer til det motsatte av fotoelektrisk effekt. Her sendes
energirike elektroner som akselereres gjennom et potensialfall $V_R$ p\aa\ flere
tusen V mot et metall. 
Figur \ref{fig:xray1} viser en skisse over et 
eksperimentelt oppsett for produksjon av R\"ontgenstr\aa ling. 
\begin{figure}[h]
\begin{center}
{\centering
\mbox
{\psfig{figure=xray.ps,height=6cm,width=12cm}}
}
\end{center}
\caption{Eksperimentelt oppsett for produksjon av R\"ontgenstr\aa ling. \label{fig:xray1}}
\end{figure}


Elektronene vekselvirker med atomene i metallet via Coulomb
vekselvirkningen,
og overf\o rer bevegelsesmengde til atomene. Elektronene bremses dermed ned, og i denne
deakselerasjonen sendes det ut e.m.~str\aa ling i R\"ontgen omr\aa det.
Denne str\aa lingen som skyldes nedbremsingen av elektronet kalles
Bremsstraahlung fra tysk for bremsestr\aa ling. 
Dette svarer til e.m.~str\aa ling 
med frekvenser i st\o rrelsesorden
$10^{16}-10^{21}$ Hz, b\o lgelengder 
i st\o rrelsesorden
$10^{-7}-10^{-13}$ m og energier for fotoner i st\o rrelsesorden
$10^{1}-10^{6}$ eV. 

Idet elektronet bremses ned, kan vi anta at det vekselvirker mange ganger med atomene
i materialet og dermed f\aa r vi et kontinuerlig energispektrum for
utsendt  e.m.~str\aa ling i
R\"ontgen omr\aa det, slik som vist i figur \ref{fig:xray2}.
De to toppene $K_{\alpha}$ og $K_{\beta}$ skyldes bestemte eksiterte 
tilstander i metallet brukt under eksperimentet, se igjen  
figur \ref{fig:xray1}. Vi kommer tilbake til dette under v\aa r diskusjon
om det periodiske systemet.
\begin{figure}[h]
\begin{center}
{\centering
\mbox
{\psfig{figure=xray2.ps,height=6cm,width=10cm}}
}
\end{center}
\caption{Intensitets fordeling fra R\"ontgenstr\aa ling. \label{fig:xray2}}
\end{figure} 

Dersom vi antar at atomene er mye tyngre enn det enkelte elektron, kan vi 
idealisere prosessen til \aa\ v\ae re gitt ved 
\be
   e^-\rightarrow e^-+\gamma,
\ee
hvor vi heretter i dette kurset kommer til bruke indeks $\gamma$ for fotoner
og $e^-$ for elektroner. 
Kaller vi den kinetiske energien til elektronet f\o r st\o tet for $K_e$ og den etter
for $K_e'$ har vi f\o lgende energibalanse
\be
    h\nu= \frac{hc}{\lambda}=K_e-K_e'.
\ee
Det en observerte ved R\"ontgen str\aa ling var at det fantes en minste
b\o lgelengde for utsending av  R\"ontgen str\aa ling, noe som ikke kunne forklares
vha.~klassisk e.m.~teori. Dette svarer igjen til en maksimal energi som fotonene kan ha.
Med en gitt innkommende kinetisk energi for elektronene, observerte en  ulike 
frekvens(b\o lgelengde)fordelinger
for den e.m.~str\aa lingen. Men, felles for alle kinetiske energier var en minste b\o lgelengde.

Dersom vi antar at elektronet har null kinetisk energi etter st\o tet, dvs.~at det kommer til ro,
har vi
\be
     h\nu= \frac{hc}{\lambda}=K_e,
\ee
som igjen gir oss
\be
     \frac{hc}{\lambda_{min}}=K_e,
\ee
og setter vi inn at $K_e=eV_R$ har vi
\be
     \lambda_{min}=\frac{hc}{eV_R}.
\ee

Det finnes mange eksempler p\aa\ at n\aa r ladde partikler bremses ned
s\aa\ sendes det ut h\o genergetisk e.m.~str\aa ling. Kosmisk str\aa ling er et slikt
eksempel, og hvem har ikke sett Nordlys med det blotte \o ye?

Et mer eksotisk eksempel er s\aa kalt R\"ontgenstjerner, eller dersom vi bruker
det engelske faguttrykket X-ray pulsars og bursters!
Pulsarer er hurtigroterende n\o ytronstjerner i bin\ae re stjernesystemer (to stjerner n\ae r 
hverandre).
X-ray pulsarer og bursters
er antatt \aa\ v\ae re n\o ytronstjerner som mottar masse fra en annen stjerne
i et bin\ae rsystem. Massen til den andre stjerna kan v\ae re
 flere solmasser for pulsarer ($M > 
10M_\odot$) eller ha liten masse ($M <  1.2M_\odot$) for bursters.
Den utsendte e.m.~str\aa ling i R\"ontgenomr\aa det antas \aa\ skyldes
masse som samles enten ved polene eller over hele stjerna. N\aa r masse
slynges ned mot stjerna vil ulike kjernefysiske reaksjoner sende ut 
e.m.~str\aa ling i R\"ontgenomr\aa det.

\section{Compton spredning}
%\begin{figure}
%\begin{center}
%{\centering
%\mbox
%{\psfig{figure=comptonpr.ps,height=16cm,width=10cm}}
%}
%\caption{Utdrag fra Comptons artikkel i Physical Review i 1923. Compton fikk
%Nobelprisen for effekten oppkalt etter ham i 1927.}
%\end{center}
%\end{figure}

I et eksperiment fra 1923\footnote{Dekkes av kap 2-7, sidene 107-113.}, 
sendte Compton inn h\o genergetiske fotoner
(R\"ontgen str\aa ler) mot en grafittplate og observerte bla.\
en forskjell i b\o lgelengde mellom den innkommende e.m.~str\aa ling
og den utg\aa ende. Siden energien er knytta til b\o lgelengden via
\[
  E=\frac{hc}{\lambda}=h\nu,
\]
betyr ei forandring i b\o lgelengde en energiforandring.
Figur \ref{fig:comptoneksp1} viser en skisse av oppsettet for Comptons fors\o k.
\begin{figure}[h]
\begin{center}
{\centering
\mbox
{\psfig{figure=compton3.ps,height=8cm,width=12cm}}
}
\end{center}
\caption{Eksperimentelt oppsett for Comptons fors\o k.
\label{fig:comptoneksp1}}
\end{figure}
Figur \ref{fig:comptoneksp2} viser, skjematisk, intensitetsfordelingen for ulike
spredningsvinkler $\theta$ 
for den e.m.~str\aa lingen.
\begin{figure}[h]
\begin{center}
{\centering
\mbox
{\psfig{figure=compton2.ps,height=8cm,width=12cm}}
}
\end{center}
\caption{Intensitetsfordeling for e.m.~str\aa ling i Comptons fors\o k.\label{fig:comptoneksp2}}
\end{figure}
Compton antok
at denne spredningsprosessen kunne idealiser som et st\o t mellom
et tiln\ae rma fritt elektron i grafittplata og h\o genergetiske fotoner. 
Dvs.\ at reaksjonen er gitt ved
\be
   \gamma +e^- \rightarrow  \gamma +e^-.
\ee
Dersom fotonet skulle kunne overf\o re all sin energi til elektronet, vil vi
ha en prosess av typen 
\be
   \gamma +e^- \rightarrow  e^-,
\ee 
og som vi skal nedenfor s\aa\ strider en slik prosess med bevaring av
energi og bevegelsesmengde.
Vi kunne ogs\aa\ tenke oss at fotonet  kreerte et elektron, dvs.
\be
   \gamma \rightarrow  e^-.
   \label{eq:gammaannihil}
\ee 
En slik prosess er ogs\aa\ umulig, ikke bare fordi vi ikke kan tilfredsstille
bevaring av energi og bevegelsesmengde, men ogs\aa\ fordi ladning ikke
er bevart. Et foton har null ladning, mens et elektron har ladning $-e$.

I senere eksperiment observerte en ogs\aa\ det utg\aa ende elektronet.
I slik forstand kan vi betrakte denne prosessen som en analog til
fotoelektrisk effekt, hvor innkommende e.m.~str\aa ling river l\o s de svakest
bundne elektronene. En viktig forskjell er dog at i fotoelektrisk effekt
s\aa\ blir den e.m.~str\aa ling i all hovedsak absorbert. 

Grunnen til at Compton kunne idealisere sitt spredningseksperiment
som et st\o t mellom et tiln\ae rma fritt elektron og et foton
ligger i energien til fotonene, som n\aa\ er flere st\o rrelsesordener
st\o rre enn i fotoelektrisk effekt. 
Siden energien til de innkommende fotonene $E_{\gamma}$ er s\aa\ stor, 
betyr det at arbeidsfunksjonen $w_0$, som er p\aa\ noen f\aa\ eV,
kan neglisjeres i energibalansen og vi kan betrakte det hele som et st\o t
mellom et foton og et fritt elektron.  

Hvordan kan vi forst\aa\ dette? 
La oss f\o rst sette opp resultatet av Comptons regning, dvs.\
forandringen i b\o lgelengde ved Compton spredning,
gitt ved $\Delta \lambda=\lambda'-\lambda$.
Compton viste at den kunne skrives som (se utledning nedenfor)
\be 
     \Delta \lambda=\lambda_C(1-cos\theta),
\ee
med 
\be
\lambda_C=\frac{h}{m_ec}=0.0243\times 10^{-10}\hspace{0.1cm}\mathrm{m},
\ee
og vi ser at $\Delta \lambda$ varierer fra $0$ til en maks verdi
$2\frac{h}{m_ec}$. $\lambda_C$ kalles ogs\aa\ Comptonb\o lgelengden.
{\em Legg merke til $\Delta \lambda$ er uavhengig av materiale
og b\o lgelengden til den innkommende e.m.~str\aa ling.} Det var dette,
som i motsetning til fotoelektrisk effekt med en arbeidsfunksjon $w_0$
bestemt av materialet,  som bla.~leda Compton til \aa\ anta at prosessen
kunne idealiserer vha.~et fritt elektron, og ikke et som er bundet
til et atom. 

Forklaringen ligger simpelthen i energist\o rrelsene.
I fotoelektrisk effekt har vi e.m.~str\aa ling i omr\aa det med synlig lys
til ultrafiolett lys. Det betyr at vi har frekvenser i st\o rrelsesorden
$10^{15}-10^{17}$ Hz, noe som tilsvarer b\o lgelengder
i st\o rrelsesorden
$10^{-7}-10^{-8}$ m og energier for fotoner i st\o rrelsesorden
$10^{0}-10^{2}$ eV. 
Dersom vi ser p\aa\ v\aa rt eksempel med monokromatisk gult lys,
s\aa\ er forholdet
\be
   \frac{\lambda_C}{\lambda}\sim 10^{-6},
\ee
dvs.~ at vi knapt kan observere en forandring i b\o lgelengde n\aa r vi har
med fotoelektrisk effekt \aa\ gj\o re. Legg ogs\aa\ merke til at arbeidsfunksjonen for f.eks.~kalium er p\aa\ 2.1 eV, p\aa\ st\o rrelse med energien
til fotonene. 

Med R\"ontgenstr\aa ler derimot, har vi  
e.m.~str\aa ling 
med frekvenser i st\o rrelsesorden
$10^{16}-10^{21}$ Hz, noe som tilsvarer b\o lgelengder
i st\o rrelsesorden
$10^{-7}-10^{-13}$ m og energier for fotoner i st\o rrelsesorden
$10^{1}-10^{6}$ eV. Dersom vi antar at fotonene har en frekvens
p\aa\  $10^{19}$ Hz, gir det en energi p\aa\ $41000$ eV og en 
b\o lgelengde
\be
    \lambda=\frac{c}{\nu}=3\times 10^{-11}\hspace{0.1cm}\mathrm{m}.
\ee

Ser vi f\o rst p\aa\  $\lambda_C/\lambda$ finner vi n\aa\ 
\be
   \frac{\lambda_C}{\lambda}=\frac{0.0243\times 10^{-10}\hspace{0.1cm}\mathrm{m}}{3\times 10^{-11}\hspace{0.1cm}\mathrm{m}}=0.081
\ee
som betyr at forskjellen i b\o lgelengde burde v\ae re observerbar.
\O ker vi frekvensen blir forholdet klart st\o rre, og dermed 
st\o rre sannsynlighet for \aa\ observere ei forandring i
b\o lgelengde. 
Ser vi deretter p\aa\ energien, ser vi at $41000$ eV er mye st\o rre enn 
typiske energier i fotoelektrisk effekt, og arbeidsfunksjonen $w_0$
som er p\aa\ noen f\aa\ eV. Derfor kunne Compton i dette tilfelle
idealisere spredningen av fotoner mot et materiale som spredning
av et foton mot et fritt elektron. Det vi har antatt er at den kinetiske
energien elektronet f\aa r er mye st\o rre enn $w_0$, og at vi kan 
neglisjere $w_0$. Det vil svare til en situasjon hvor elektronet 
ikke er bundet, dvs.~det er fritt.

Det vi f\o rst skal forklare er selve forandringen i
b\o lgelengde. Til slutt skal vi ogs\aa\ forklare tilfellet
med null b\o lgeforandring ogs\aa\, se igjen figur \ref{fig:comptoneksp2}.


Fotonet er en partikkel med masse null som reiser med lysets hastighet.
For \aa\ se at massen m\aa\ v\ae re null, kan vi bruke noen enkle
argument fra FY-ME100. Vi har at energien til en fri partikkel med hastighet
$v$ er gitt ved
\be
    E=\frac{m_0c^2}{\sqrt{1-v^2/c^2}}.
\ee
N\aa r vi kombinerer dette med det faktum at
fotonene reiser med lysets hastighet $c$ og at energien til et foton
er endelig og gitt ved $E=h\nu$, s\aa\ m\aa\ massen til fotonet v\ae re lik 
null, ellers vil energien divergere i det ovennevnte uttrykk. 
Bruker vi ogs\aa\ relasjonen 
\be
   E=\sqrt{p^2c^2+m^2c^4}=pc=h\nu=\frac{hc}{\lambda},
\ee
har vi at bevegelsesmengden $p$ er gitt ved
\be
   p=\frac{h\nu}{c}.
\ee


N\aa r vi skal utlede Comptons formel, trenger vi alts\aa\ \aa\ ta 
utganspunkt i to bevaringssatser, energi og bevegelsesmengde, f\o r
og etter st\o tet. 
Vi antar ogs\aa\ at f\o r st\o tet s\aa\ er elektronet i ro,
dvs.~at det har null kinetisk energi og bevegelsesmengde. Vi kan da,
se ogs\aa\ figur \ref{fig:compton1},  
\begin{figure}[h]
\begin{center}
{\centering
\mbox
{\psfig{figure=compton1.ps,height=8cm,width=12cm}}
}
\end{center}
\caption{Idealisering av kollisjonen mellom et foton og et elektron. Vi antar at det er et 
tiln\ae rmet fritt elektron i ro som kolliderer med et foton.\label{fig:compton1}}
\end{figure}
lage oss en tabell med definisjonene
av energi og bevegelsesmengde for elektroner og fotoner
\begin{table}[h]
\begin{center}
\caption{Definisjon av energi og begelsesmengde f\o r og etter st\o tet for
fotonet og elektronet.}
\begin{tabular} {lll} \\ \hline
                & F\o r & Etter \\ \hline
Bevegelsesmengde&  & \\
 $\gamma$          & $p_{\gamma}=\frac{h}{\lambda}$ & $p'_{\gamma}=\frac{h}{\lambda'}$ \\
 $e^-$          & $p_e=0$ & $p'_e$ \\
Energi&  & \\
 $\gamma$          & $E_{\gamma}=\frac{hc}{\lambda}$ & $E'_{\gamma}=\frac{hc}{\lambda'}$ \\
 $e^-$   & $E_e=m_ec^2$ & $E'_e=\sqrt{p'_ec)^2+m_e^2c^4}$    \\
\\ \hline
\end{tabular}
\end{center}
\end{table}
hvor $m_e$ er elektronets masse.

Energibevaring gir
\be
  E_{\gamma}+E_e=E'_{\gamma}+E'_e,
\ee
dvs.
\be
  \frac{hc}{\lambda}+m_ec^2=\frac{hc}{\lambda'}+\sqrt{(p'_ec)^2+m_e^2c^4}.
   \label{eq:energycompt} 
\ee
Tilsvarende har vi for bevaring av bevegelsesmengde
\be
     {\bf p}_{\gamma}   +{\bf 0 }  =   {\bf p}'_{\gamma} + {\bf p}'_e.
    \label{eq:momcompt}
\ee
Dersom vi kvadrerer likning (\ref{eq:energycompt}) f\aa r vi
\be
   (p'_ec)^2+m_e^2c^4=\left(\frac{hc}{\lambda}-\frac{hc}{\lambda'}\right)^2
   +2m_ec^2\left(\frac{hc}{\lambda}-\frac{hc}{\lambda'}\right)+m_e^2c^4,
\ee
som gir
\be
   (p'_ec)^2=\left(\frac{hc}{\lambda}\right)^2+\left(\frac{hc}{\lambda'}\right)^2-2\frac{h^2c^2}{\lambda\lambda'}+2m_ec^2\left(\frac{hc}{\lambda}-\frac{hc}{\lambda'}\right).
\label{eq:left}
\ee
Deretter kvadrerer vi likning (\ref{eq:momcompt}) og f\aa r 
\be
     \left({\bf p}_{\gamma}-{\bf p}'_{\gamma}\right)^2 =({\bf p}'_e)^2,
\ee
som gir n\aa r vi multipliserer begge sider med $c^2$ 
\be
   ({\bf p}'_ec)^2=\left(\frac{hc}{\lambda}\right)^2+\left(\frac{hc}{\lambda'}\right)^2-2\frac{h^2c^2}{\lambda\lambda'}cos\theta.
\label{eq:right}
\ee
Setter vi venstresidene i likningene (\ref{eq:left}) og (\ref{eq:right}) like
og multipliserer med
\be
    \frac{\lambda\lambda'}{2hc},
\ee
f\aa r vi 
\be 
   m_ec^2\left(\lambda'-\lambda\right)-hc=-cos\theta hc,
\ee
som igjen gir
\be
   \left(\lambda'-\lambda\right)=\Delta \lambda = \frac{hc}{m_ec^2}\left(1-cos\theta\right)=\frac{h}{m_ec}\left(1-cos\theta\right),
\ee
som er Comptons formel. Vi ser at siden b\o lgelengden til den e.m.~str\aa lingen
etter st\o tet er st\o rre enn f\o r, s\aa\ inneb\ae rer det at 
energien til fotonet etter st\o tet er mindre enn f\o r, dvs.~siden
\be
   \lambda' > \lambda,
\ee
s\aa\ har vi at
\be
   E_{\lambda'} = \frac{hc}{\lambda'} < E_{\lambda} = \frac{hc}{\lambda}.
\ee


Kan s\aa\ fotonet  overf\o re all sin energi til et fritt elektron? 
Det betyr at vi tar utgangspunkt i prosessen i likning (\ref{eq:gammaannihil}),
noe som impliserer at fotonet blir annihilert, og at dets energi og bevegelsesmengde etter
st\o tet er null. Tar vi i bruk bevaring av energi og bevegelsesmengde igjen 
finner vi at 
energibevaring gir
\be
  E_{\gamma}+E_e=E'_e,
\ee
dvs.
\be
  \frac{hc}{\lambda}+m_ec^2=\sqrt{(p'_ec)^2+m_e^2c^4}.
   \label{eq:energycompt2} 
\ee
Tilsvarende har vi for bevaring av bevegelsesmengde
\be
     {\bf p}_{\gamma}   +{\bf 0 }  =  {\bf 0 }  + {\bf p}'_e,
    \label{eq:momcompt2}
\ee
som gir n\aa r vi kvadrerer og multipliserer med $c^2$
\be
   ({\bf p}'_ec)^2=\left(\frac{hc}{\lambda}\right)^2.
\ee

Vi kvadrerer igjen likning (\ref{eq:energycompt2}) og f\aa r 
\be
   (p'_ec)^2+m_e^2c^4=\left(\frac{hc}{\lambda}\right)^2
   +2m_ec^2\frac{hc}{\lambda}+m_e^2c^4,
\ee
som gir, n\aa r vi bruker bevaring av bevegelsesmengde
\be
   \left(\frac{hc}{\lambda}\right)^2=\left(\frac{hc}{\lambda}\right)^2
   +2m_ec^2\frac{hc}{\lambda},
\ee
som impliserer at 
\be
   2m_ec^2\frac{hc}{\lambda}=0,
\ee
i strid med eksperiment, elektronet har en endelig masse $m_ec^2=0.511$ MeV.
Men vi ser at klassisk er denne prosessen fullt mulig, da setter vi nemlig
$h=0$, og vi har ikke noen inkonsistens med elektronets masse.


Det vi ikke har forklart er hvorfor vi har en topp p\aa\ Figur \ref{fig:comptoneksp2}
hvor vi tilsynelatende ikke har noen forandring
i b\o lgelengde. 
Dette kan forst\aa s p\aa\ f\o lgende vis. N\aa r vi har en forandring
i b\o lgelengde, vekselvirker egentlig fotonet med de svakest 
bundne elektronene, og vi idealiserer reaksjonen som et st\o t mellom
et foton og et elektron. Men vi kan jo tenke oss at elektronet er sterkt
bundet til atomet, eller at den innfallende e.m.~str\aa ling ikke
er sterk nok til \aa\ sparke ut et elektron. I dette tilfelle kan vi betrakte
kollisjonen mellom fotonet og materialet som en kollisjon mellom
et foton og et atom, og atomet f\aa r en rekyleffekt gjennom
kollisjonen. 
Dersom vi antar at materialet best\aa r av karbon, blir reaksjonen
v\aa r
\be
   \gamma + C \rightarrow  \gamma + C,
\ee
og 
i dette tilfelle m\aa\ vi erstatte massen til elektronet med   
den til karbonatomet, som er 22000 ganger tyngre enn elektronet.
Da blir
\be
\lambda_C=\frac{h}{22000m_ec}\sim 10^{-16}\hspace{0.1cm}\mathrm{m},
\ee
og med de aktuelle b\o lgelendge blir $\lambda_C/\lambda$
knapt observerbar!

Som en oppsumering p\aa\ fotoelektrisk effekt, R\"ontgen str\aa ling  og Compton spredning
kan vi si at e.m.~str\aa ling utviser b\aa de partikkel og 
b\o lgeegenskaper. Str\aa lingen, hva den enn m\aa tte v\ae re, utviser
i noen tilfeller partikkel egenskaper, og andre tilfelle reine
b\o lgeegenskaper slik som diffraksjon og interferens.

Comptons eksperiment utviser begge deler:\newline
1) Prinsippene bak m\aa lingen 
av den spredte e.m.~str\aa lingen ble gjort vha.\
standard b\o lgel\ae re.\newline
2) Spredningen p\aa virker b\o lgelengden p\aa\ et vis som kun
kan forst\aa s ved \aa\ behandle R\"ontgen str\aa lene som 
partikler som kolliderer  med elektronene i et atom.

Vi kan si at uttrykket
\[
   E=h\nu
\]
har i seg b\o lgebeskrivelsen ved
\be
    \lambda  \hspace{0.1cm} og  \hspace{0.1cm}  \nu
\ee
og partikkelbeskrivelsen ved
\be
    E \hspace{0.1cm} og  \hspace{0.1cm}  p=h/\lambda
\ee

\begin{center}
\shabox{\parbox{14cm}{Vi skal legge merke til at b\aa de for den fotoelektriske effekt og for Compton spredning,
s\aa\ er det av avgj\o rende betydning at vi ser p\aa\ sv\ae rt sm\aa\ b\o lgelengder. I grensa
$\lambda\rightarrow \infty$ g\aa r de kvantemekaniske resultat mot de klassiske.
N\aa r vi beveger oss inn i R\"ontgenstr\aa lingen sitt domene, begynner energiene \aa\ bli
s\aa pass store at vi kan betrakte kollisjonen som et st\o t mellom et foton og et elektron
som er tiln\ae rminsgvis fritt. Forandringen i b\o lgelengde blir s\aa pass stor 
at det er mulig \aa\ observere den. Energiskalaen og det faktum at $h$ er liten 
dikterer hva slags fysisk teori og fysisk forst\aa else vi m\aa\ ta i bruk for \aa\ forklare
eksperiment. Det kan derfor i v\aa r analyse 
v\ae re hensiktsmessig \aa\ studere st\o rrelser som
\[
   \frac{\Delta E}{E},
\]
og 
\[
   \frac{\Delta \lambda}{\lambda}.
\]
}}\end{center}

\section{Oppgaver}
\subsection{Analytiske oppgaver}

\subsubsection*{Oppgave 1.1}
%
\begin{itemize}
\item[a)] Energienheten 1~eV {(\sl elektronvolt)} er definert som
{\o}kningen i kinetisk energi n{\aa}r et elek\-tron akselereres
gjennom et potensialsprang  p{\aa} 1~volt.
Vis at 1 eV $= 1,602 \times 10^{-19}$~J.\\
Hva blir energi{\o}kningen n{\aa}r elektronet akselereres i et
potensialsprang p{\aa} 10~V,\\
 50~kV $ = 5 \times 10^4$~V
og 1~MV $ = 10^6$~V ?\\
Beregn hastigheten elektronene f{\aa}r etter akselerasjon
i potensialene nevnt ovenfor n{\aa}r utgangshastigheten
$v_0 = 0$.
%
\item[b)] Beregn den potensielle energien for to partikler
med ladning $e = 1,602 \times 10^{-19}$~C som befinner
seg i en avstand 0,1~nm $ = 10^{-10}$~m. Vis at
enheten elektronvolt er en naturlig enhet i dette tilfelle.
%
\item[c)] La oss anta at de to partiklene ovenfor er et proton
og et elektron. Diskut\'{e}r forholdet mellom gravitasjons--
og elektrostatisk potensiell energi i dette tilfelle.\\
(Gravitasjons konstanten: $\gamma = 6,67 \times 10^{-11}$
 N m$^2$~kg$^{-2}$.)
%
\item[d)] Beregn hvileenergien for elektronet og protonet.
Under hvilke betingelser m{\aa} vi regne med relativistiske
effekter?
%
\end{itemize}

\subsubsection*{Oppgave 1.2}
%
Anta at sola med radius $6,96 \times 10^8$~m str{\aa}ler som et
svart legeme. Av denne str{\aa}lingen mottar vi 1370~Wm$^{-2}$
her p{\aa} jorda i en avstand av $1,5 \times 10^{11}$~m.
Jorda har en radius $6378$~km.
%
\begin{itemize}
%
\item[a)] Beregn temperaturen til sola.
%
\item[b)] Anta at atmosf{\ae}ren rundt jorda reflekterer
$30 \%$ av den innkommende str{\aa}ling. Hvor mye energi
fra solen absorberer den per sekund og per kvadratmeter?
%
\item[c)] For {\aa} v{\ae}re i termisk likevekt, m{\aa}
jorda emittere like mye energi som den absorberer via
atmosf{\ae}ren hvert sekund. Anta at den str{\aa}ler
som et sort legeme. Finn temperaturen.
Hva betyr drivhuseffekten ?
\end{itemize}

\subsubsection*{Oppgave 1.3}
Defin\'{e}r uttrykket ``svart legeme'' (``blackbody'').
Plancks str\aa lingslov sier at utstr\aa lt effekt pr. arealenhet og 
pr. b\o lgelengde, $dR/d\lambda$, fra et sort legeme er gitt ved 
\[
  \frac{dR}{d\lambda}=\frac{2\pi hc^{2}}{\lambda^{5}(e^{hc/\lambda kT}-1)},
\]
der $k$ er Boltzmanns konstant og $T$ er legemets temperatur.  
I utledningen av likningen over gj\o res en antakelse som bryter med 
klassisk fysikk.  Forklar kort hva denne antakelsen best\aa r i.   
Bruk Plancks lov til \aa \ vise Wiens forskyvningslov
\[
 \lambda_{{\rm m}}T={\rm konstant}, 
\]
der $\lambda_{{\rm m}}$ er b\o lgelengden hvor $dR/d\lambda$ er maksimal
(konstanten skal ikke bestemmes, men har verdien 0.00290 ${\rm K}\cdot
{\rm m}$).
En del av de astrofysiske objekter som kalles pulsarer sender 
ut str\aa ling i r{\o}ntgenomr\aa det.
Ved observasjon av en slik pulsar finner man at den  
utstr\aa lte effekt pr. arealenhet og pr. b{\o}lgelengde 
er st\o rst ved en b\o lgelengde p\aa \ 
5.58 nm.  Anta at pulsaren str\aa ler som et sort legeme og 
regn ut overflatetemperaturen. 

\subsubsection*{Oppgave 1.4, Eksamen V-1994}
%
%
\begin{itemize}
%
\item[a)] Gj{\o}r rede for den fotoelektriske effekten. Tegn en figur
som viser en eksperimentell oppstilling for {\aa} studere denne effekten.
%
\item[b)] Anta at vi bruker denne oppstillingen til \aa\ bestr\aa le et metall
med monokromatisk lys, $\lambda = 2.58 \cdot 10^{-7}$ m. Figur~1.1 viser observerte
verdier av str{\o}mmen $I$ n{\aa}r spenningen varieres.
Forklar forl{\o}pet av kurven.
Hva skjer hvis intensiteten av lyset {\o}kes til det dobbelte ?
%
\begin{figure}[hbtp]
%
\setlength{\unitlength}{1cm}
%
\begin{center}
%
\begin{picture}(10,6)
%
\thicklines
%
%\put(0,0){\framebox(14,10){}}
	\put(0,0){\epsfxsize= 10cm \epsfbox{volt-1.eps}}
%

\end{picture}
%
\caption{Str{\o}m $I$ som funksjon av spenningen $V$ ved
foto--elektrisk effekt. \label{fig:figfoto}}
%
\end{center}
%
\end{figure}
%
\item[c)] Bruk figur~\ref{fig:figfoto} til {\aa} bestemme arbeidsfunksjonen for metallet.
%
\item[d)] Vi bytter n{\aa} ut  metallet foran med et nytt.  Det
bestr{\aa}les med lys med forskjellig b{\o}lgelengde. Tabell~\ref{tabfoto}
viser stoppepotensialet for noen forskjellige verdier av b{\o}lgelengden.
Bestem ut fra dette arbeidsfunksjonen for det nye metallet, den lavest mulige
frekvensen for prosessen og Plancks konstant $h$.
%
\begin{table}[htbp]
%
\begin{center}
%
\begin{tabular}{||r|r|r|r|r||}\hline
$\lambda$ (i 10$^{-7}$ m)& 2,536 & 3,132 & 3,650 & 4,047\\ \hline
$V_S$ (i volt)            & 1,95 & 0,98 & 0,50 & 0,14\\ \hline
\end{tabular}
%
\label{tabfoto}
%
\caption{Stoppepotensial for forskjellige verdier av b{\o}lgelengden.}
%
\end{center}
%
\end{table}
%
\item[e)] Forklar hvorfor et foton ikke kan overf{\o}re all
sin energi og bevegelsesmengde til et fritt elektron.
%
\end{itemize}
%

\subsubsection*{Oppgave 1.5}
En str�le av ultrafiolett lys og med en intensitet p�  $1.6 \times 10^{-12}$W blitt pluselig satt p� 
og bestr�ler en metalflate. Ved fotoelektrisk effekt blir elektroner sendt ut fra metallflaten.
Den innkommende str�len har et tverrsnitt p� 1~cm$^2$ og b�lgelengden svarer til en foton energi p� 10~eV. 
Arbeidsfunksjonen for metallet er 5~eV.  Vi skal i det f�lgende analysere hvor lang tid det vil ta f�r 
elektroner emitteres.
%
\begin{itemize}
%
\item[a)] Gj�r et klassisk estimat basert p� den tid det vil ta f�r et elektron med 
radius $\approx 1$~� har absorbert nok energi til � emitteres.
%
\item[b)] Lord Rayleigh viste imidlertid (Phil. Mag. {\bf 32}, (1916), side 188) at dette
 estimatet er for pessimistisk. Absorbsjonsarealet for et elektron i et atom er av st�rrelsesorden 
$\lambda^2$ for lys av b�lgelengde $\lambda$. Beregn  den klassiske  forsinkelsestiden
for emisjon av et elektron.
%
\item[c)] Kvantemekanisk  er det mulig for et elektron � emitteres  umiddelbart -- s� snart 
et foton med tilstrekkelig energi  treffer metalflaten. For � f� et estimat som kan sammenlignes med 
den klassiske verdien beregn det midlere tidsintervall mellom to fotoner i str�len. 
%
\end{itemize}

\subsubsection*{Oppgave 1.6}
\begin{itemize}
\item[a)] Gj{\o}r kort rede for den fotoelektriske effekten,
og skiss\'{e}r en  eksperimentell oppstilling
som kan observere og m{\aa}le denne
effekten.
%
\item[b)] Den fotoelektriske arbeidsfunksjonen for kalium (K)
er 2,0~eV. Anta at lys med en b{\o}lgelengde p{\aa}
360~nm ( 1~nm = $10^{-9}$~m)
faller p{\aa} kaliumet. Finn stoppepotensialet for fotoelektronene,
den kinetiske energien og hastigheten for de hurtigste av de emitterte
elektronene.
%
\item[c)] En uniform monokromatisk lysstr{\aa}le med b{\o}lgelengde
400~nm faller p{\aa} et materiale med arbeidsfunksjon p{\aa} 2,0~eV, og
med en intensitet p{\aa} $3,0 \times 10^{-9}$ Wm$^{-2}$.
Anta at materialet reflekterer 50~\% av den
innfallende str{\aa}le, og at 10~\% av de absorberte fotoner
f{\o}rer til et emittert elektron.
Finn antall
elektroner emittert pr. m$^2$ og pr. sec, den absorberte
energi pr. m$^2$ og pr. sec, samt den kinetiske energi for
fotoelektronene.
\end{itemize}


\subsubsection*{Oppgave 1.7}
I et r{\o}ntgenr\o r lar vi en elektronstr\aa le gjennoml\o pe et 
spenningsfall $V$ f\o r den treffer en anode av wolfram (W).  
Den resulterende str\aa lingen fra r{\o}ntgenr\o ret observeres.  
Figur~\ref{fig:xray2} viser hvordan str\aa lingens intensitet varierer med 
b\o lgelengden $\lambda$.
%
\begin{itemize}
%
\item[a)]Tegn en skjematisk skisse av et r{\o}ntgenr\o r.  Gi en kort og 
kvalitativ 
beskrivelse av hvordan r{\o}ntgenstr\aa lingen dannes og typiske trekk 
ved den jevne kontinuerlige delen av spekteret i figur~\ref{fig:xray2}.
%
\item[b)]Skriv ned og begrunn sammenhengen mellom spenningen $V$ og 
den minste 
b\o lgelengden $\lambda_{{\rm min}}$ for r\o ntgenstr\aa lingen.  Gj\o r 
det samme for $V$ og den maksimale frekvensen $\nu_{{\rm maks}}$.  

\end{itemize}

\subsubsection*{Oppgave 1.8}

\begin{itemize}
%
\item[a)] Tegn en skjematisk skisse av et r\"{o}ntgenr{\o}r. Gi
en kort og kvalitativ beskrivelse av hvordan r\"{o}ntgenstr{\aa}lingen
dannes og typiske trekk ved fotonspektret.
%
\item[b)] Hvis et elektron har kinetisk energi $E_k = 10000$~eV,
beregn den minste b{\o}lgelengden $\lambda_{min}$ for de utsendte fotonene.
%
\item[c)] Vi antar  at vi har r\"{o}ntgenstr{\aa}ler med  b{\o}lgelengde
$\lambda = 1 \mbox{nm} = 1 \times 10^{-9}\mbox{m}$.
De faller inn mot en krystall  under en
vinkel p{\aa} 45$^{\circ}$ med innfallsloddet. Dette gir
en koherent refleksjon fra krystallen.
Beskriv prosessen og beregn ut fra dette
avstanden mellom de reflekterende plan i krystallen.
%
\item[d)] N{\aa}r et foton passerer en atomkjerne, kan det omdannes
til et elektron--positron par (pardannelse).
Finn den minste energien $E_{\nu}$  fotonet kan ha
for at denne prosessen kan skje (se bort fra mulige rekyleffekter).
%
\end{itemize}

\subsubsection*{Oppgave 1.9}
%
Et foton er spredt en vinkel $\theta$  i forhold til dets opprinnelige
retning etter {\aa} ha vekselvirket med et fritt elektron som var
i ro.
%
\begin{itemize}
%
\item[a)]  Gj{\o}r rede for hvilke prinsipper som anvendes til {\aa}
	  beregne det spredte fotonets b{\o}lgelengde og utled Comptons
	  formel.
%
\item[b)]  Begrunn at b{\o}lgelengden for det spredte
	  fotonet er st{\o}rre enn b{\o}lgelengden for fotonet f{\o}r
	  vekselvirkningen (spredningsvinkelen   forutsettes {\aa} v{\ae}re
	  st{\o}rre enn null).
%
\item[c)] Et foton er spredt en vinkel $\theta$   etter {\aa} ha vekselvirket med et
          fritt elektron som f{\o}r vekselvirkningen hadde en
          bevegelsesmengde som var like stor som fotonets bevegelsesmengde, men
          motsatt rettet. Hvor stor blir fotonets b{\o}lgelengdeforandring i dette
	  tilfellet? Vi antar at elektronet kan beskrives
	  ikke--relativistisk.

\item[d)]  Kan et foton overf{\o}re hele sin energi og bevegelsesmengde
	  til et fritt elektron? Begrunn svaret.
\end{itemize}




\subsubsection*{Oppgave 1.10}
Et foton med b\o lgelengde $\lambda = 1,00\cdot 10^{- 11}$ m treffer et fritt
elektron i ro. Fotonet blir spredt under en vinkel $\theta$, og f�r
en b\o lgelengdeforandring gitt ved Comptons ligning $\Delta \lambda =
\lambda^{'} - \lambda = \lambda_{c} (1 - \cos \theta )$, hvor Comptonb\o
lgelengden er $\lambda_{c} = 2,426\cdot 10^{-12}$ m.
%
\begin{figure}[htbp]
%
\begin{center}

\setlength{\unitlength}{1cm}
%
\begin{picture}(9,6)
\thicklines
              \put(0,0){\makebox(0,0)[bl]{
              \put(3,0){\vector(0,1){6}}
              \put(3,3){\vector(1,0){4}}
              \dottedline{0.1}(0,3)(2.3,3)
              \put(2.2,3){\vector(1,0){0.5}}
              \put(3,3){\circle*{0.2}}
              \put(3,3){\vector(3,-1){3.5}}
              \put(6,2){\circle*{0.2}}
              \put(6,2.2){\small elektron} 
               \dottedline{0.1}(3,3)(4.5,5.25)
               \put(4.5,5.25){\vector(2,3){0.3}}
              \put(0.5,3.2){\makebox(0,0)[bl]{\small $\lambda$: foton}}
              \put(5,4.5){\makebox(0,0)[b]{\small $\lambda^{'}$: foton}}
              \put(3.7,3.4){\makebox(0,0)[bl]{\small $\theta = 60^{\circ}$}} 
              \put(4.5,2.6){\makebox(0,0)[bl]{\small $\phi = \;?$}}
              \put(7.1,3){\makebox(0,0)[bl]{$x$}}
              \put(3.2,5.5){\makebox(0,0)[bl]{$y$}}
         }} 
%
\end{picture}
%
\caption{\label{figcomp}}

\end{center}

\end{figure}
%
I denne oppgaven skal vi bare se p\aa ~det som observeres under en vinkel
$\theta = 60^{\circ}$ (se figur~\ref{figcomp}). Energi og bevegelsesmengde beregningene
skal uttrykkes ved enheten eV\@.
%
\begin{itemize}
%
\item[a)] Beregn energien og bevegelsesmengden til det innkommende fotonet.

\item[b)] Finn b\o lgelengden, bevegelsesmengde og den kinetiske energien
til det spredte fotonet.

\item[c)] Finn den kinetiske energien, bevegelsesmengden og
spredningsvinkelen for elektronet.
%
\end{itemize}
%

\subsubsection*{Oppgave 1.11, Eksamen V-1992}
Vi skal i denne oppgaven gj{\o}re bruk av det relativistiske uttrykk
for energien av en partikkel
%
\[
E = \sqrt{E_0^2 + (pc)^2}.
\]
%
Som energi enhet bruk $eV$ og $eV/c$ som enhet for bevegelsesmengde.
%
\begin{itemize}
\item[a)] Gj{\o}r kort rede for de st{\o}rrelser som inng{\aa}r i relasjonen
ovenfor.
%
\item[b)] Anvend relasjonen ovenfor p{\aa} et foton og finn fotonets
energi og bevegelsesmengde n{\aa}r b{\o}lgelengden er
$500 \; nm = 5,00 \times 10^{-7}\; m$.
%
\end{itemize}
%
Et foton med energi $E = h \nu_0 = h c / \lambda_0$ spres en vinkel
$\theta$ ved Comptoneffekt
mot et elektron som antas {\aa} ligge i ro f{\o}r spredning.
%
\begin{itemize}
%
\item[c)] Tegn en prinsippskisse for et Compton eksperiment
og angi spredningsvinklene for fotonet og elektronet etter
spredningen.
%
\item[d)] Formul\'{e}r de prinsipper som anvendes til {\aa} beregne
det spredte fotonets b{\o}lgelengde $\lambda^{'}$ og bruk dem
til {\aa} utlede Comptons formel
%
\[
\lambda^{'} - \lambda_0 = \frac{h}{m_e c} (1 - cos \theta )
\]
%
\end{itemize}
%
R{\o}ntgenstr{\aa}ler med en b{\o}lgelengde
$\lambda_0 = 1,21 \times 10^{-1}\; nm$
treffer en m{\aa}lskive med carbon atomer.
De spredte r{\o}ntgenstr{\aa}ler blir observert i en vinkel
p{\aa} 90$^{\circ}$.
\begin{itemize}
\item[e)] Beregn b{\o}lgelengden $\lambda^{'}$ av det spredte
fotonet ved $90^{\circ}$.
%
\item[f)] Figur~\ref{fig:comptoneksp2} viser en topp ved b{\o}lgelengden $\lambda_0$.
Hvilken prosess skyldes denne toppen?
\end{itemize}

\clearemptydoublepage
\chapter{Bohr's model of the atom}

\section{Introduksjon}
\begin{figure}[h]
\begin{center}
{\centering
\mbox
{\psfig{figure=stamp_bohr.ps,height=6cm,width=8cm}}
}
\end{center}
\caption{Dansk frimerke som markerer Bohrs atomteori.}
\end{figure}

Allerede\footnote{Lesehenvisning til l\ae reboka er 
Kap.\ 3-5, 3-6, 3-7 og 3-8, sidene 144-170.} 
mot  slutten av 1800-tallet fantes det store mengder med
eksperimentelle data om spektroskopi fra atomer. Det dreide seg i 
all hovedsak om utsendt (emittert) e.m.~str\aa ling fra atomer.
Den viktige observasjonen var  at e.m.~str\aa ling utsendt fra
atomer er konsentrert ved bestemte diskrete b\o lgelengder.
Et av de mest studerte atomer var hydrogenatomet, som best\aa r
av et proton og et elektron. 
\begin{figure}[h]
\begin{center}
{\centering
\mbox
{\psfig{figure=hydrogen1.ps,height=6cm,width=12cm}}
}
\end{center}
\caption{Fotonspekteret for hydrogen. B\o lgelengdene er i nm.}
\end{figure}

Basert p\aa\ slike diskrete verdier for b\o lgelengden 
(spektrallinjer)
til den utsendte
e.m.~str\aa ling, se Figur \ref{fig:spektrabalmer}, 
\begin{figure}[h]
\begin{center}
{\centering
\mbox
{\psfig{figure=hydrogenspekter.ps,height=8cm,width=12cm}}
}
\end{center}
\caption{Ulike mulige overganger i hydrogenatomet. Grunntilstanden
har kvantetall $n=1$ og energi $-13.6$ eV, som ogs\aa\ svarer til energien for 
\aa\ frigj\o re et elektron i hydrogenatomet. \label{fig:spektrabalmer}}
\end{figure}
laget Balmer en
parametrisering for de observerte spektrallinjene allerede i 1885,
med b\o lgelengden gitt som
\be
    \lambda =(3645.6\times 10^{-10}\hspace{0.1cm}\mathrm{m})\frac{n^2}{n^2-4}
             \hspace{0.1cm}n=3,4,\dots ,
\ee
som skal svare til overganger fra eksiterte tilstander i hydrogenatomet
til den f\o rste eksisterte tilstanden i hydrogen, dvs.\ $n=2$.
Balmerserien av overganger representerer overganger til den f\o rste
eksiterte tilstanden fra h\o yereliggende eksiterte tilstander.
Tilsvarende overganger til grunntilstanden kalles for Lymanserien.
For \aa\ foregripe den p\aa f\o lgende diskusjonen, kan vi si at
tallene $n=1,2,3,\dots$ skal svare til bestemte kvantetilstander
(merk at verdiene er diskrete) for elektronet i hydrogenatomet.
Merk at vi heretter, n\aa r vi snakker om et atoms energi, s\aa\ tenker
vi p\aa\ energien til elektronet(ne). Da opererer vi med en energi skala
p\aa\ noen f\aa\ eV. 
Tallet $n=1$ skal svare til grunntilstanden, eller normaltilstanden.
Denne tilstanden har en energi, som vi skal utlede b\aa de vha.~Bohrs atommodell og kvantemekanikk,  p\aa\ $-13.6$ eV. 
For $n=2$, som svarer til en energi p\aa\ $-3.4$ eV, har vi den f\o rste
eksiterte tilstand for hydrogenatomet. For $n=3,4,\dots$ har vi h\o yere
eksiterte tilstander, se Figur \ref{fig:spektrabalmer}. 


Grunnen til at bla.~Balmer studerte overganger fra $n=3,4,\dots$ til 
den f\o rste eksiterte tilstand med $n=2$ skyldtes enkelt og greit
at den utsendte str\aa ling var i omr\aa det for synlig lys til det 
ultrafiolette. I tillegg var det slik, og er fremdeles slik, at den e.m.~str\aa ling som blir sendt utfra fotosf\ae ren p\aa\ sola, dvs.~solas overflate,
ligger i omr\aa det for synlig lys til det ultrafiolette. Og sola har fra
forhistorisk tid alltid v\ae rt gjenstand for v\aa r nysgjerrighet og
vitebegj\ae r. I omr\aa det for synlig lys utsendt i fotosf\ae ren, 
har det negative hydrogenionet avgj\o rende betydning. Dette ionet har i kort
tid to elektroner i bane rundt kjernen og er meget viktig for emisjon
og absorpsjon av e.m.~str\aa ling i fotosf\ae ren. 
Dersom vi sammenligner data for utsendt e.m.~str\aa ling
fra solas overflate og e.m.~str\aa ling utsendt fra et svart
legeme med temperatur 5800 K (som er ca.~temperatur ved solas overflate),
ville vi f\aa tt en mer hakkete kurve med ulike fordypninger. 
Disse fordypningene
eller topper, svarer til utsendt str\aa ling fra ulike atomer i solas
atmosf\ae re. En bestemt topp er f.eks.~gitt  
ved b\o lgelengden $\lambda=656.1$ nm,
som igjen svarer til overgangen fra $n=3$ til $n=2$ i
hydrogenatomet. Denne overgangen fikk benevning  $H_{\alpha}$.  
 

I 1890 kom Rydberg med en parametrisering for alle mulige overganger
i hydrogenatomet, nemlig
\be
\frac{1}{\lambda}=R_H\left(\frac{1}{n_f^2}-\frac{1}{n_i^2}\right),
\label{eq:ryd}
\ee  
med 
\begin{enumerate}
\item $n_f=1$ og $n_i=2,3,\dots$ svarende til den s\aa kalte Lymanserien,(ultrafiolett)
dvs.~overganger fra eksiterte tilstander til grunntilstanden,
\item  $n_f=2$ og $n_i=3,4,\dots$ svarende til den s\aa kalte Balmerserien (synlig lys),
\item  $n_f=3$ og $n_i=4,5,\dots$ svarende til den s\aa kalte Paschenserien (infrar\o dt), osv.
\end{enumerate}
Konstanten $R_H$ er Rydbergkonstanten for hydrogen og gitt ved
\be
    R_H=1.0967757\times 10^7\hspace{0.1cm}\mathrm{m}^{-1}.
\ee

selv om parametriseringene reproduserte dataene, s\aa\ manglet en mer
grunnleggende teori til beskrivelsen av atomer. F\o rst med kvantemekanikken
var en i stand til \aa\ gi en tilfredsstillende matematisk beskrivelse.
Et viktig bidrag mot kvantemekanikken var Bohrs atommodell fra 1913.
Bohr utviklet en teori basert p\aa\ kombinasjonen av klassisk mekanikk
og Plancks og Einsteins kvantiseringsteorier fra e.m.~str\aa ling.

Teorien Bohr presenterte forklarte hydrogenspekteret 
og dermed Rydbergs formel. La oss se n\ae rmere p\aa\ postulatene
Bohr kom med.

\section{Bohrs postulater og hydrogenatomet}
Vi skal se p\aa\ de ulike postulatene som Bohr lanserte og diskutere
hvert av dem. Postulatene er som f\o lger 
       \begin{enumerate}
          \item Et elektron i et atom beveger seg i sirkul\ae re
                baner om en kjerne. Kreftene er gitt ved tiltrekning via Coulomb
                vekselvirkningen mellom elektron og kjerne.

          \item Klassisk kan et elektron befinne seg i en uendelighet
                av orbitaler. {\bf Bohr antok at et elektron kan kun bevege 
                seg i baner hvis banespinn $L$ er et heltalls
                multiplum av $\hbar=h/2\pi$.}

          \item Et elektron i en slik bane vil klassisk ha en 
                sentripetalakselearasjon og skal dermed 
                sende ut e.m.~str\aa ling. Ved utsending av energi
                deakselereres elektronet, og til slutt vil det
                komme helt inn til kjernen. {\bf Bohr antok 
                derimot at elektronets energi i en gitt orbital 
                er konstant.} 

          \item E.m.~str\aa ling sendes ut dersom et elektron
                som beveger seg i en orbital med energi 
                $E_i$, forandrer p\aa\ et diskontinuerlig vis
                sin bevegelse slik at energien er gitt ved 
                den til en orbital med energi $E_f$.
                \[
                   \nu=\frac{(E_i-E_f)}{h} 
                \]
      \end{enumerate}

\subsection{F\o rste postulat} 
Dette postulatet baseres seg p\aa\ eksistensen av en atomkjerne, og utledningen
av energien som et elektron kan ha er basert p\aa\ betrakninger fra
klassisk mekanikk. Siden protonet er ca 2000 ganger tyngre enn elektronet,
kan vi i f\o rste omgang beskrive systemet som et 
elektron som kretser om protonet som tyngdepunkt. Dvs.~at vi kun tar
hensyn til elektronets posisjon og energi som frihetsgrader.
Radien $r$ svarer da til avstanden mellom protonet og elektronet.
Kreftene $F$ som virker p\aa\ elektronet er gitt via Coulombtiltrekningen 
\be
    F = k\frac{(-e)(+e)}{r^2},
\label{eq:fcoul}
\ee
hvor vi heretter skal bruke
\be
    k=\frac{1}{4\pi\epsilon_0}=8.988\times 10^9\hspace{0.1cm}\mathrm{Nm}^2/\mathrm{C}^2.
\ee
Bruker vi ogs\aa\ elektronets ladning kvadrert 
har vi at $ke^2=1.44$ eVnm. Dette er en st\o rrelse som er grei \aa\ huske, da den forekommer ofte i dette kurset.
Bohr antok sirkul\ae re baner. Da er sentripetakselarasjonen gitt ved
\be
   a_r=-\frac{v^2}{r},
\ee
og bruker vi Newtons lov med $m_ea_r$ og likning (\ref{eq:fcoul}) f\aa r vi at
\be
   m_ea_r=-m_e\frac{v^2}{r}=-k\frac{e^2}{r^2}.
   \label{eq:vn}
\ee 
N\aa\ kan vi ogs\aa\ finne elektronets energi $E$ ved \aa\ summere
kinetisk energi $E_{kin}$ og potensiell energi $E_{pot}$.
Sistnevnte f\o lger fra likning (\ref{eq:fcoul}) 
\be
   F=-\frac{\partial E_{pot}}{\partial r},
\ee
slik at vi f\aa r
\be
    E=E_{kin}+E_{pot}=\frac{1}{2}mv^2-k\frac{e^2}{r},
\ee 
og bruker vi likning (\ref{eq:vn}) 
finner vi
\be
   E=\frac{1}{2}k\frac{e^2}{r}-k\frac{e^2}{r}=-k\frac{e^2}{2r}.
\ee
\subsubsection{Andre postulat} 
Vi har tidligere sett Plancks og Einsteins kvantiseringspostulat
for e.m.~str\aa ling $E=nh\nu$. Bohr antok, basert p\aa\ at elektronet
kun kunne bevege seg i bestemte baner, at det var banespinnet til
elektronet (bane angul\ae rt moment) som var kvantisert, dvs.~at banespinnet 
kunne
anta bare diskrete verdier
\be
   |{\bf L}|=m_evr=n\hbar.
\ee
med $n=1, 2, 3,\dots$
Fra  likning (\ref{eq:vn}) har vi at 
\be
   v=\sqrt{\frac{ke^2}{m_er}},
\ee
som gir
\be
   m_evr=\sqrt{ke^2m_er}=n\hbar.
\ee
Kvadrerer vi siste likning har vi definisjonen p\aa\ Bohrradier
\be
   r_n=\frac{\hbar^2n^2}{m_eke^2}=a_0n^2,
   \label{eq:bohrradius}
\ee
hvor $a_0$ kalles {\em Bohrradien}, og danner en viktig lengdeskala
i atomfysikk, faste stoffers fysikk og molekylfysikk. Denne minste
Bohrradien har verdi
\be
   a_0=\frac{\hbar^2}{m_eke^2}=0.529\times 10^{-10} \hspace{0.1cm}\mathrm{m}.
\ee

Vi ser av uttrykket for Bohrradiene at de tillatte radiene antar
diskrete verdier diktert av verdien p\aa\ heltallet $n$, som {\em vi kaller
et kvantetall}.

Setter vi uttrykket for $r_n$ i det for energien til elektronet finner vi
\be
   E=E_n=-k\frac{e^2}{2r_n}=-k\frac{e^2}{2a_0}\frac{1}{n^2}=-13.6\frac{1}{n^2} \hspace{0.1cm}\mathrm{eV}. \label{eq:energyradius}
\ee

{\em Det viktige budskapet her er at 
energikvantisering f\o lger n\aa\ fra postulatet om 
kvantisering av banespinn.}
Vi kan generalisere uttrykket for energien, radien osv.~til \aa\ gjelde
tyngre atomer ved \aa\ erstatte ladningen til et enkelt proton
$+e$ med den til $Z$ protoner. Da f\aa r vi
\be
   E_n=-k\frac{Z^2e^2}{2a_0}\frac{1}{n^2}=-13.6\frac{Z^2}{n^2} \hspace{0.1cm}\mathrm{eV},
\ee
\be
   r_n=\frac{\hbar^2n^2}{m_ekZe^2}=a_0\frac{n^2}{Z},
\ee
og
\be 
   v_n=\frac{n\hbar}{m_er_n}=\frac{kZe^2}{n\hbar}.
\ee
Dersom vi velger $n=1$ finner vi 
\be
   v_1=\frac{kZe^2}{\hbar},
\ee
og med innsetting av lyshastigheten 
har vi
\be
   v_1=\frac{ckZe^2}{\hbar c}.
\ee
N\aa\ er $ke^2=1.44$ eVnm, $\hbar c=197$ eVnm, slik at
\be
   v_1\approx Z\times 2.2\times 10^6 \hspace{0.1cm}\mathrm{m/s},
\ee
og for lette atomer er hastigheten kun noen f\aa\ prosent av 
lyshastigheten. 
Dette forteller oss at Bohrs atommodell, som er basert p\aa\ ikke
relativistisk teori, kun kan anvendes for lette atomer.
For $Z\sim 100$, begynner elektronets hastighet \aa\ n\ae rme seg
lysets.
Vi avslutter dette avsnittet med definisjonen av den s\aa kalt
finstrukturkonstanten
                \be
                  \alpha=\frac{e^2}{4\pi\epsilon_0\hbar c}\approx 1/137,
                \ee
som vi kan bruke til \aa\ skrive om uttrykkene til Bohrradien 
\be
   a_0=\frac{\hbar}{\alpha m_ec},
\ee
og energien 
\be
E_n=-\frac{\alpha^2}{2n^2}m_ec^2,
\label{eq:ealpha}
\ee
som likner p\aa\ det velkjente uttrykket $E=m_ec^2$!

Setter vi inn for ulike $n$-verdier finner vi 
$E_1=-13.6$ eV, $E_2=-3.4 $ eV, $E_3=-1.5 $ eV osv., i bra samsvar 
med de eksperimentelle verdiene.

\subsection{Tredje postulat} 

Hva tredje postulat ang\aa r, s\aa\ tar det utgangspunkt i det 
faktum at atomer tross alt er stabile. Vi skal se senere
i v\aa r diskusjon av kvantemekanikken at vi kan, om enn ikke
forklare hvorfor atomer er stabile, ihvertfall lage
en teori hvor formalismen gir som resultat at det er stor
sannsynlighet for at atomer forblir stabile. 



\subsection{Fjerde postulat} 
N\aa\ kan vi tenke oss et elektron som er i en orbital $n_i$ med
energi $E_i$ g\aa r spontant til en orbital $n_f$ hvor
$n_f < n_i$. 
I denne prosessen sendes det ut e.m.~str\aa ling og
siden energien til $E_i \ne E_f$, s\aa\ betyr det at energibevaring krever
\be
   E_i=E_f+E_{\gamma}=E_f+h\nu.
\ee   
Den utsendte e.m.~str\aa ling er gitt ved 
\be
   \Delta=h\nu=E_i-E_f,
\ee
dvs.\
\be
  \nu=\frac{(E_i-E_f)}{h}=\left(\frac{1}{4\pi\epsilon_0}\right)^2\frac{m_eZ^2e^4}{4\pi\hbar^3}\left(\frac{1}{n_f^2}- \frac{1}{n_i^2}\right)
\label{eq:emisjon}
\ee
Dersom vi n\aa\ 
bruker at $1/\lambda=\nu/c$ og 
setter inn tallverdier for de ulike konstantene og velger $Z=1$ (hydrogenatomet) finner vi
\be
  \frac{1}{\lambda}=R_H\left(\frac{1}{n_f^2}- \frac{1}{n_i^2}\right),
   \label{eq:trans}
\ee
med 
\be
   R_H=\left(\frac{1}{4\pi\epsilon_0}\right)^2\frac{m_ee^4}{4\pi\hbar^3c}=
        \frac{ke^2}{2a_0hc},
   \label{eq:rh}
\ee
som har samme form som Rydbergs formel i likning (\ref{eq:ryd}).
Setter vi deretter inn numeriske verdier finner vi
\be
   R_H=1.0973732\times 10^7\hspace{0.1cm}\mathrm{m}^{-1},
\ee 
mens den eksperimentelle verdien for hydrogenatomet er 
$R_H=1.0967757\times 10^7$ m$^{-1}$. Vi har alts\aa\ en teori
som reproduserer tilpasningen til de eksperimentelle spektrallinjene
gitt i likning (\ref{eq:ryd}). I neste avsnitt skal vi se at n\aa r vi
tar hensyn  til protonets masse vil verdien av $R_H$ n\ae rme
seg den eksperimentelle.

P\aa\ tilsvarende vis kan vi studere absorpsjon av e.m.~str\aa ling,
en prosess hvor innkommende e.m.~str\aa ling
eksiterer et elektron til en h\o yere liggende energitilstand.
Energibevaring gir 
\be
   h\nu+E_i=E_f
\ee
og den inverse b\o lgelengde blir da
\be
  \frac{1}{\lambda}=R_H\left(\frac{1}{n_i^2}- \frac{1}{n_f^2}\right).
\ee
 
Til slutt skal vi nevne at vi dersom $n\rightarrow \infty$ f\aa r vi
$E_{\infty}=0$. Dette svarer igjen til at avstanden mellom elektronet
og protonet er uendelig stor. Vi sier da at elektronet er fritt.
Den energien som trengs for \aa\ l\o srive et elektron fra
hydrogenkjernen kalles ionisasjonsenergien og er gitt ved
\be
   E_{\infty}-E_1=0-(-13.6)=13.6 \hspace{0.1cm}\mathrm{eV}.
\ee


\subsection{Tyngdepunktskorreksjon}

Vi har sett at massen til elektronet $m_e$ er  mye mindre enn massen
til protonet $m_p$, $m_p\sim 2000 m_e$. Dersom vi n\aa\ tar utgangspunkt
i bevegelse om et felles tyngdepunkt for protonet og elektronet 
s\aa\ trenger vi \aa\ introdusere en relativ masse
for hydrogenatomet $\mu_H$
\be
    \mu_H=\frac{m_em_p}{m_e+m_p}=m_e\frac{1}{1+\frac{m_e}{m_p}}=0.99946m_e.
\ee
Dette virker som en ubetydelig korreksjon. Men det 
er n\aa\ slik at 
spektroskopiske data er sv\ae rt n\o yaktige. Dersom vi f.eks.~
studerer n\ae rmere  uttrykket for $R_H$ i likning (\ref{eq:rh}) 
ser vi at det avhenger av massen til elektronet.
Tar vi forholdet mellom den utrekna verdien og den eksperimentelle
for $R_H$ finner vi
\be
   1.0973732\times 10^7/1.0967757\times 10^7=1.000545 
\ee
Verdien av $\mu_H=0.99946m_e$ gir oss dermed et hint om mulige
korreksjoner til den teoretiske verdien av $R_H$. 
La oss studere denne mulige korreksjonen i n\ae rmere detalj.

Hittil har vi alts\aa\ tenkt at siden elektronet er mye lettere enn protonet,
kan vi approksimere systemet til et elektron som kretser rundt et 
tyngdepunktsenter, som er protonet. Dette senteret definerer origo.

La oss heller betrakte systemet som et to-legeme
system, hvor vi ser p\aa\ tyngdepunktsenteret til dette to-legeme systemet
som det nye origo.
Vi definerer da en relativ avstand 
\be
   {\bf r}={\bf r_e}-{\bf r_p},
\ee
hvor   ${\bf r_e}$ er elektronets avstand til det nye origo og
${\bf r_p}$ den tilsvarende for protonet. 
Siden origo er lokalisert i massesenteret m\aa\ vi tilfredsstille f\o lgende
likning
\be
    m_e {\bf r_e} + m_p{\bf r_p}=0.
\ee

De to siste likningene hjelper oss \aa\ finne et uttrykk for 
\be
    {\bf r_e}=\frac{m_p}{m_e+m_p}{\bf r},
\ee
og 
\be
    {\bf r_p}=-\frac{m_e}{m_e+m_p}{\bf r}.
\ee
Bruker vi deretter definisjonen p\aa\ hastighet $v=dr/dt$ finner vi
\be
    {\bf v_e}=\frac{m_p}{m_e+m_p}{\bf v},
\ee
og 
\be
    {\bf v_p}=-\frac{m_e}{m_e+m_p}{\bf v}.
\ee
Vha.~de to uttrykkene for hastighetene kan vi uttrykke b\aa de kinetisk
energi og banespinnet for protonet og elektronet. Den kinetiske
energien til dette to-legeme systemet blir da
\be
   E_{kin}=\frac{1}{2}m_ev_e^2+\frac{1}{2}m_pv_p^2=
            \frac{1}{2}\frac{m_em_p}{m_e+m_p}v^2,
\ee
eller 
\be
   E_{kin}=\frac{1}{2}\mu_Hv^2,
\ee
hvor $v$ er den relative hastighet gitt ved 
$   {\bf v}={\bf v_e}-{\bf v_p}$ og $\mu_H$ er den reduserte
massen til et system som best\aa r av et proton og et elektron.

P\aa\ tilsvarende vis kan vi uttrykke det totale banespinnet som
\be
   L= m_er_ev_e + m_pr_pv_p=m_e\left(\frac{m_p}{m_e+m_p}\right)^2vr+
      m_p\left(\frac{m_e}{m_e+m_p}\right)^2vr,
\ee
eller 
\be
   L=\mu_Hvr.
\ee
Bruker vi atter en gang Bohrs kvantiseringspostulat p\aa\ banespinnet, finner
vi at Bohrradiene $r_n$ er gitt ved 
\be
    r_n=\frac{4\pi\epsilon_0\hbar^2}{e^2\mu_H}n^2.
\ee

Dette uttrykket skiller seg fra likning (\ref{eq:bohrradius}) 
ved at vi har den reduserte massen istedet for elektronets masse.
Multipliserer vi oppe og nede med massen til elektronet finner vi
\be
    r_n=\frac{m_e4\pi\epsilon_0\hbar^2}{m_ee^2\mu_H}n^2=
    \frac{m_e}{\mu_H}a_0n^2,
\ee
eller
\be
   r_n=\frac{m_e}{\mu_H}a_0\frac{n^2}{Z},
\ee
hvor Bohrradien $a_0$ er gitt ved
$a_0=\hbar^2/m_eke^2$. I tillegg 
har vi introdusert ladningen til kjernen ved $Z$. For hydrogenatomet har
vi $Z=1$.

I likning (\ref{eq:energyradius}) ga vi et uttrykk for energien 
vha.~Bohrradiene $r_n$. Setter vi inn det nye uttrykket for $r_n$ finner vi
at energien kan skrives som 
\be
   E_n=-k\frac{\mu_H}{m_e}\frac{Ze^2}{2a_0}\frac{1}{n^2}=-13.6\frac{\mu_H}{m_e}\frac{Z}{n^2} \hspace{0.1cm}\mathrm{eV},
\ee
som da gir oss den nye Rydbergkonstanten $R_H'$ for hydrogenatomet
gitt ved
\be 
   R_H'=\frac{\mu_H}{m_e}R_H=0.99946R_H=1.096781\times 10^7\hspace{0.1cm}\mathrm{m}^{-1},
\ee
n\ae rmere den eksperimentelle verdien.

En interessant anvendelse av tyngdepunktskorreksjon var oppdagelsen
av deutronet. Deutronet best\aa r av et proton, et n\o ytron og et elektron.
N\o ytronet har omtrent samme masse som protonet. Setter vi massen
til deutronets kjerne lik $2m_p=2\times 938$ MeV/c$^2$, finner vi at
den reduserte massen til deutronet (med et elektron er)
\be
   \mu_D=\frac{m_e2m_p}{m_e+2m_p}=m_e\frac{1}{1+\frac{m_e}{2m_p}}=
0.99973m_e.
\ee
En observerte at for $n_f=2$ og $n_i=3$ overgangen i Balmerserien,
den s\aa kalt $H_{\alpha}$ linjen s\aa\ fantes det to b\o lgelengder,
en gitt ved $\lambda=656.1$ nm som vi allerede har diskutert og en
ukjent ved $\lambda=656.3$ nm.
Dersom vi bruker likning (\ref{eq:trans}) og korrigerer for tyngdepunktskorreksjoner for henholdsvis hydrogenatomet og deutronet finner vi
for hydrogenatomet
\be
  \frac{1}{\lambda_H}=\frac{\mu_H}{m_e}R_H\left(\frac{1}{2^2}- \frac{1}{3^2}\right),
\ee
og for deutronet 
\be
  \frac{1}{\lambda_D}=\frac{\mu_D}{m_e}R_H\left(\frac{1}{2^2}- \frac{1}{3^2}\right).
\ee
Ser vi p\aa\ forholdet 
\be
    \frac{\lambda_D-\lambda_H}{\lambda_H}=
    \frac{\mu_H-\mu_D}{\mu_D}=\frac{\mu_H}{\mu_D}-1,
\ee
som skal gi oss forskjellen i b\o lgelengde. Setter vi inn tallverdier finner
vi 
\be
   \frac{\mu_H-\mu_D}{\mu_D}=\frac{m_e(0.99946-0.99973)}{m_e0.99973}=-0.00027,
\ee
som svarer til ei forandring p\aa\ $0.027\%$. Dette gir oss ei forandring
i b\o lgelengde mellom hydrogenatomet og deutronet for denne spesielle
overgangen p\aa\  0.177 nm, i bra samsvar med observerte verdier.
Deutronet ble oppdaga i 1932 av Urey, 3 \aa r etter oppdagelsen
av n\o ytronet. 
\subsection{Korrespondanseprinsippet}

Bohr formulerte korrespondanse prinsippet slik:
       \begin{itemize}
          \item Kvanteteoriens forutsigelser om oppf\o rselen
                til et fysisk system skal svare til de for et klassisk system i grensa hvor kvantetallene som bestemmer en tilstand
blir veldig store. 
      \end{itemize}
Teknisk sett betyr det at den klassiske grense skal n\aa s n\aa r kvantetallene
blir 'store', f.eks.~ved \aa\ la kvantetallet $n$ i Bohrs modell
bli stort. (Senere i dette kurset skal vi se at kvantemekanikken som teori
inneholder automatisk klassisk fysikk som et grensetilfelle.)
Vi kan vise dette prinsippet med et enkelt eksempel. La oss se p\aa\
frekvensen for en overgang fra en tilstand med kvantetall $n+1$ til en
med kvantetall $n$.  Bruker vi likning (\ref{eq:emisjon}) 
\be
  \nu=\frac{(E_{n+1}-E_n)}{h}=\frac{m_ec^2}{4\pi \hbar}(Z\alpha)^2\left(\frac{1}{n^2}- \frac{1}{(n+1)^2}\right),
\ee
ser vi at n\aa r $n$ er stor kan vi approksimere frekvensen med
\be
   \nu\approx \frac{m_ec^2}{2\pi \hbar}(Z\alpha)^2\frac{1}{n^3}.
    \label{eq:klassiskbohr}
\ee
Det klassiske uttrykket for frekvensen er gitt fra FY-ME100 som
\be
   \nu_{klassisk}=\frac{v}{2\pi r}.
\ee
Setter vi s\aa\ inn uttrykket for Bohrradien har vi
\be
   \nu_{klassisk}=\frac{Z\alpha m_ec}{n}\frac{Z\alpha c}{2\pi n^2\hbar}=
 \frac{m_ec^2}{2\pi \hbar}(Z\alpha)^2\frac{1}{n^3},
\ee
som svarer til likning (\ref{eq:klassiskbohr}). Legg merke til at det er
bare overganger av typen $n+1\rightarrow n$ som gir det klassiske
resultat. Str\aa ling som svarer til en overgang av typen 
$n+2\rightarrow n$ har ikke noen klassisk analog. 

\subsection{Franck-Hertz eksperimentet}

I 1914 utf\o rte Franck og Hertz et eksperiment som var 
den f\o rste bekreftelse p\aa\ eksistensen av stasjon\ae re tilstander
i atomer, dvs.~at kun bestemet eksiterte tilstander var mulig,
i motsetning til det kontinuerlige spekteret av tilstander 
som klassik fysikk ga. 
For dette fikk Franck og Hertz Nobelprisen i fysikk i  1925. Et eksperimentelt
oppsett er vist i Figur \ref{fig:fh1}. Eksperimentet ble gjennomf\o rt
ved \aa\ akselerere elektroner gjennom en gass 
av kvikks\o lv vha.~en p\aa satt spenning mellom katoden og anoden. 
Etter som elektronene 
kolliderer med kvikks\o lvatomene taper de kinetisk energi. For \aa\ kunne 
n\aa\ anoden og for at en skulle kunne m\aa le en str\o m av elektroner
m\aa\ elektronene ha en bestemt kinetisk energi. \O kes spenningsforskjellen,
vil flere og flere elektroner n\aa\ anoden, og dermed skulle en registrere en
monotont \o kende str\o m av elektroner. 
\begin{figure}[h]
\begin{center}
{\centering
\mbox
{\psfig{figure=franckhertz.ps,height=6cm,width=12cm}}
}
\end{center}
\caption{Skjematisk oppsett for Franck og Hertz sitt eksperiment med
kvkks\o lv i 1914. Et r\o r med tre elektroder fylles med kvikks\o lvdamp. Elektroner aksellereres over et spenningsfall $V$ mellom katoden og et gitter. Et mindre spenningsfall med motsatt polaritet settes opp mellom gitteret og anoden. Str\o mmen av elektroner gjennom r\o ret avhenger av hvordan de vekselvirker med kvikks\o lv.\label{fig:fh1}}
\end{figure}
\begin{figure}[t]
\begin{center}
{\centering
\mbox
{\psfig{figure=franckhertz2.ps,height=6cm,width=12cm}}
}
\end{center}
\caption{Str\o mmen av elektroner gjennom r\o ret m\aa les som funksjon av det aksellererende 
potensialet $V$. Flere og flere elektroner n\aa r anoden etter som spenningen
\o kes. Ved $V\approx 4.9$ V, $2\times 4.9$ V, $3\times 4.9$ V osv.~avtar str\o mmen. 
Det svarer til at elektronene
taper 4.9 eV per kollisjon.\label{fig:fh2}}
\end{figure}
Dette var ikke tilfelle eksperimentelt. Et plott av elektrisk str\o m 
som funksjon av det akselererende potensialet viste topper ved heltalls
verdier av en spenning p\aa\ ca 4.9 V, se Figur \ref{fig:fh2}.  
Det som skjer er at ved bestemte
kinetiske energier (og dermed spenninger) for elektronene, har elektronene
en energi som svarer til den f\o rste eksiterte tilstanden i 
kvikks\o lvatomet, se Figur \ref{fig:fh3}. 
N\aa r elektronene dermed overf\o rer sin kinetiske energi til 
kvikks\o lvatomet, bremses det ned og f\ae rre elektroner n\aa r anoden. 
Derfor faller str\o mmen for spenningsfall over 4.9 V.
\O kes spenningen over 4.9 V, \o ker str\o mmen igjen til elektronene har
en kinetiske energi p\aa\
\[
   E_{kin}= 2\times 4.9 \hspace{0.1cm} \mathrm{eV}.
\]
Da har elektronet energi nok til \aa\ eksitere to kvikks\o lvatomer og en 
observerte en ny topp i elektronstr\o mmen. 

N\aa r kvikks\o lvatomet g\aa r tilbake til grunntilstanden, sendes
det ut elektromagnetisk str\aa ling med b\o lgelengde p\aa\ 
$\lambda=254$ nm.
Det svarer til en energiforksjell gitt ved 
\be
  \Delta E=\frac{hc}{\lambda}=
   \frac{1240\hspace{0.1cm} \mathrm{eVnm}}{254\hspace{0.1cm} \mathrm{nm}}=
           4.88 \hspace{0.1cm} \mathrm{eV},
\ee
i samsvar med den m\aa lte spenningen og dermed den kinetiske energi som
elektronene har.


\subsection{Problemer med Bohrs atommodell}
Bohrs atommodell var et viktig steg i retning mot kvantemekanikken.
Det faktum at en teoretisk kunne forklare spektrene til atomer var
en aldri s\aa\ liten revolusjon. Men modellen hadde klare svakheter
etterhvert som den ble konfrontert med nye eksperimentelle data.

Modellen var i stand til \aa\ forklare flere egenskaper ved alkalimetallene,
slik som litium, natrium, kalium, rubidium og cesium. Dette er alle
metaller hvis kjemiske egenskaper henger n\o ye sammen med det faktum at de
alle har et valenselektron. Vi kan da tenke oss at de svarer til modifiserte
hydrogenatomer, hvor frihetsgradene til kjernene og de andre elektronene
er absorbert i ladningen $Z$. Men som vi skal se i v\aa r gjennomgang av
det periodiske systemet, kunne ikke alle egenskapene til 
alkaliatomene 
forklarers vha.~Bohrs atommodell. For atomer med flere valenselektroner
var samsvaret mellom teori og eksperiment d\aa rlig. 

I tillegg, inkluderte Bohrs atommodell ad hoc postulater slik som
postulatet om kvantiseringen av banespinnet, 
som i det lange l\o p ikke var intellektuelt
tilfredsstillende. 
\begin{figure}[t]
\begin{center}
{\centering
\mbox
{\psfig{figure=franckhertz3.ps,height=4cm,width=10cm}}
}
\end{center}
\caption{Eksitasjonsenergien fra grunntilstanden til den f\o rste eksiterte tilstanden svarer til at et elektron i et kvikks\o lvatom har f\aa tt tilf\o rt en energi p\aa\ $4.9$ eV.
Ved henfall tilbake til grunntilstanden sendes det ut e.m.~str\aa ling med b\o lgelengde
  $\lambda=\frac{hc}{\Delta E}=254$ nm.
\label{fig:fh3}}
\end{figure}
\section{Oppgaver}
\subsection{Analytiske oppgaver}
\subsubsection*{Oppgave 2.1}
Synlig lys har b{\o}lgelengder i omr{\aa}det 4000--7000 {\AA}.
%
\begin{itemize}
%
\item[a)] Fra Bohrs formel for energiniv{\aa}ene i H--atomet, vis at
b{\o}lgelengden til
det emitterte lyset vil ligge utenfor det synlige omr{\aa}det ved alle
overganger til laveste niv{\aa}~(Lyman--serien).
%
\item[b)] Hva blir den korteste og den lengste b{\o}lgelengden
ved overganger til
det nest laveste niv{\aa}et (Balmer--serien)?

\item[c)] Hvor mange spektrallinjer ligger i det synlige omr{\aa}det i denne
serien?
%
\end{itemize}



\subsubsection*{Oppgave 2.2}
I den enkleste versjonen av Bohrs atommodell antas det at elektronet
beveger seg
i sirkul{\ae}re baner rundt en atomkjerne som ligger i ro,
d.v.s. at den har en
masse som regnes {\aa} v{\ae}re uendelig stor. Spesielt for
de letteste atomene er
dette ikke en s{\ae}rlig god antagelse. I H--atomet best{\aa}r
kjernen kun av et
enkelt proton med en masse $m_{p}$ som er 1836 ganger tyngre
enn elek\-tron\-massen $m_{e}$.
%
\begin{itemize}
%
\item[a)] Ved {\aa} ta med bevegelsen til protonet i H--atomet,
vis at den totale
banespinnet for atom\-et i dets massesentersystem er
$L = \mu \omega r^{2}$,
hvor r er avstanden mellom elektronet og protonet, $\omega$ er
vinkelfrekvensen
til deres sirkul{\ae}re bevegelse og $\mu$ er deres reduserte masse
\[
\mu = \frac{m_{e} m_{p}}{m_{e} + m_{p}}.
\]
%
\item[b)] Finn energiene til de stasjon{\ae}re
tilstandene til atomet ved bruk av
Bohrs kvantiseringsbetingelse $L = n \hbar$, og vis at dette mer
n{\o}yaktige
resultatet er identisk med Bohrs energiformel med det unntak at den reduserte
massen $\mu$ n{\aa} inng{\aa}r i stedet for elektronmassen $m_{e}$.
%
\item[c)] Hvor mye forandres den lengste b{\o}lgelengden $H_{\alpha}$
($ n= 3 \longrightarrow n = 2$) i Balmer--serien if{\o}lge
denne mer korrekte formelen?
%
\item[d)] Spektrallinjen $H_{\alpha}$ fra deuterium (tungt hydrogen) har
b{\o}lgelengden $\lambda = $ 656,029 {\AA}. Finn massen
til atomkjernen i deuterium.
%
\end{itemize}
%

\subsubsection*{Oppgave 2.3}
\begin{itemize}
%
\item[a)] Hvilke energier har lyskvant som faller i den
synlige delen av spektret,\\
4000 {\AA} $\leq \lambda \leq $ 7000 {\AA}?
%
\item[b)] Oppgi noen spektrallinjer av atom{\ae}rt hydrogen
og enkeltionisert helium som tilsvarer synlig lys.
%
\item[c)] Finn st{\o}rrelsesorden av effekten p{\aa} disse
dersom man tar hensyn til rekylen av atomet under emisjonen,
og vis at denne effekten er langt mindre enn
isotopeffekten (effekt p.g.a. redusert masse).
%
\end{itemize}

\subsubsection*{Oppgave 2.4}
Et atom med masse M, opprinnelig i ro, emitterer et foton ved en overgang fra et
energiniv\aa ~$E_1$ til et annet niv\aa ~$E_2$. Det sies vanligvis at det
emitterte fotonets energi er gitt ved $h \nu = E_1 - E_2$, der $\nu$ er
fotonets frekvens og h er Plancks konstant. Imidlertid vil en liten del av
energien overf\o res til atomet som kinetisk energi (rekylvirkning), og fotonet
f\aa r derfor en tilsvarende mindre energi: $h \nu = E_1 - E_2 - \Delta E$.
%
\begin{itemize}
%
\item[a)] Finn et uttrykk for denne korreksjonen $\Delta E$. G\aa ~ut fra at
fotonet har en bevegelsesmengde lik $ h \nu / c$.

\item[b)] Gjennomf\o r en tilsvarende diskusjon for det tilfellet at atomet
 absorberer et foton.

\item[c)] Regn ut den numeriske verdien av $\Delta E / h \nu $ for
en overgang $E_1 - E_2 = 4,86$ eV i et kvikks\o lvatomet
som har masse lik  200 $\times $ massen til et proton.
%
\end{itemize}

\subsubsection*{Oppgave 2.5}

Bruk Bohrs kvantiseringsbetingelse  $L = n \hbar$ til � kvantisere energien av en partikkel
i et sentral potensial $V(r) = V_0 (r/a)^k$. Beregn den kvantiserte energien som 
funksjon av kvantetallet $n$. Kontroll\'{e}r $n$-avhenigheten for $k = -1$ som svarer til 
Bohrs atom modell for hydrog\'{e}n. Lag en figur som viser $V(r)$ for store verdier av $k$.
Forklar ut fra figuren verdien av elektronets bane-radius for $k \rightarrow \infty$. 
%


\clearemptydoublepage
\chapter{Matter waves}
\begin{figure}[h]
\begin{center}
{\centering
\mbox
{\psfig{figure=ride.ps,height=6cm,width=8cm}}
}
\end{center}
\caption{Dobbeltsspalt eksperiment. Nylig, se 
Phys.~Rev.~Lett.~{\bf 82} (1999) 2868, klarte fysikere \aa\ observere
interferens fenomen mellom b\o lgepakker som besto av kun to fotoner.}
\end{figure}
Hensikten med dette kapitelet, i tillegg til \aa\
diskutere de Broglies hypotese, er \aa\ gi en introduksjon
til viktige deler av b\o lgel\ae ren som vil f\o lge oss i
store av deler av kurset. B\o lgel\ae ren er i seg selv
et omfattende fagfelt, men vi skal i all hovedsak fokusere
p\aa\ de deler som er av interesse for kurset.
Dette dreier seg om interferens, diffraksjon, og gruppe
og fasehastighet. Dette er begrep som var sentrale 
i eksperiment som avdekket b\o lgeegenskapen til materien. 
For \aa\ forst\aa\ disse resultatene og \aa\ kunne relatere
dem til de Broglies hypotese, Heisenbergs uskarphetsrelasjon,
materiens b\o lge og partikkel egenskaper og Schr\"odingers
likning, trenger vi noen resultater fra b\o lgel\ae ren.
I tillegg, og kanskje minst like viktig, vil v\aa r diskusjon
av kvantemekanikken basere seg p\aa\ b\o lgeteori.
Et viktig aspekt i denne vektleggingen av b\o lgel\ae re
ligger i det rent pedagogiske, i den forstand at dere
vil se en fortsettelse og utvidelse 
av begrep fra e.m.~teori som dere har sett i kurs som FY101.
Sett utifra et slikt perspektiv kan en ogs\aa\ si at
bildet basert p\aa\ b\o lgeteori som vi kommer til
\aa\ vektlegge er kanskje mer fysisk intuitivt ved et 
introduksjonskurs som FYS2140. 
I videreg\aa ende kurs som FYS 201, FYS 303, FYS 305 og FYS 403,
 vil dere f\aa\ en introduksjon til
en mer moderne matematisk presentasjon av kvantemekanikken. 

Etter avsnittet med b\o lgel\ae re, f\o lger en diskusjon
av Heisenbergs uskarphetsrelasjon. Men f\o rst til et eksperimentelt
faktum som ikke har noen klassisk analogi. Her ligger ogs\aa\
det eneste 'mysteriet' med kvantemekanikken. 

\section{Materiens b\o lge og partikkelnatur}

Mer eller mindre vet vi hvordan dagligdagse gjenstander oppf\o rer seg
under p\aa virkninger fra ulike krefter. P\aa\ mikroniv\aa\ er dog v\aa rt 
erfaringsgrunnlag begrenset. Det vi skal ta for oss n\aa\ er et eksperimentelt
faktum uten sidestykke i den makroskopiske virkelighet, slik vi  kjenner den
fra v\aa rt erfaringsgrunnlag.
Materien, dvs.~elektroner, n\o ytroner, atomer m.m.~utviser ikke bare partikkel
egenskaper men ogs\aa\ b\o lgeegenskaper. Elektronene oppf\o rer seg akkurat
som lys. Det h\o rer ogs\aa\ med til historien at Newton
trodde lys besto av partikler, men da b\o lgeegenskapene til lys  ble avdekka
i mange eksperiment, ble denne tr\aa den forlatt, inntil Einsteins
radikale bruk av Plancks postulat for \aa\ forklare den fotoelektriske
effekt. Historisk ble elektronet tiltenkt kun partikkel egenskaper. 
Men ulike eksperiment med elektroner avdekket ogs\aa\ b\o lgeegenskaper. 
 
Det vi skal studere i dette kapitlet er alts\aa\ materiens b\o lgeegenskaper.
Vi kan ikke forklare dette vha.~klassisk teori. Det danner ogs\aa\ grunnlaget
for kvantemekanikken som teori og to av de viktige postulatene om naturen vi
nevnte innledningsvis, Heisenbergs uskarphetsrelasjon og de Broglies postulat
om materiens b\o lgeegenskaper. Kvantemekanikken som teori baserer seg p\aa\ disse postulatene ved \aa\ ta utgangspunkt i en b\o lgebeskrivelse av naturen\footnote{
Schr\"odingers likning som er v\aa r naturlov, er ikke noe annet enn en
b\o lgelikning, som, dersom vi tillater oss abstraksjonen med imagin\ae r tid,
kan skrives som en diffusjonlikning. Varmeledning er et klassisk 
eksempel p\aa\ et fysisk problem som kan modelleres vha.~en diffusjonslikning}. 
Vi er ikke i stand til \aa\ forklare {\em hvorfor} materien utviser slike
egenskaper. V\aa r innsikt begrenser seg {\em kun} til \aa\ forklare  
hva som skjer. Sagt litt annerledes, ved \aa\ fortelle deg hva som skjer,
vil vi ogs\aa\ avdekke kvantemekanikkens grunnleggende merkverdigheter.
Med siste ord mener vi egentlig sider ved v\aa r forst\aa else av naturen
som savner sidestykker i v\aa re erfaringsgrunnlag fra verden p\aa\
makroniv\aa\ . Du vil selvsagt sitte igjen med mange  store sp\o rsm\aa l
om hvorvidt v\aa r beskrivelse av den fysiske virkelighet p\aa\ mikroniv\aa\
virkelig f\o lger f.eks.~uskarphetsrelasjonen. I skrivende stund har vi ikke
en eneste eksperimentell observasjon p\aa\ brudd med de grunnleggende 
postulatene. 
Kanskje er uskarphetsrelasjonen en fundamental egenskap ved naturen?

Faktisk er det slik at idag tenker vi oss at vi kan bruke
sider av kvantemekanikken som er h\o yst ikke-trivielle 
i f.eks.~utviklingen av kvantedatamaskiner eller unders\o kelser av
systemer p\aa\ mikroniv\aa\ . Idag er vi f.eks.~istand til \aa\
fange inn  enkeltelektroner (kunstig hydrogenatom) i omr\aa der
p\aa\ noen f\aa\ nanometre. Eksperiment har blitt gjort
hvor en til og med kan lage s\aa kalte Schr\"odinger 
katt tilstander ved \aa\ manipulere enkeltioner i ionefeller vha.~laserlys. 
Bruk av 
entanglement\footnote{En mulig norsk oversettelse er sammenfiltrede tilstander.}
av kvantemekaniske systemer for sikker 
kryptering, rask s\o king i store databaser og teleportasjon er begrep
som bare for f\aa\ \aa r siden hadde et klart science fiction preg over seg.
Idag snakker vi om slike sider, om ikke med den st\o rste selvf\o lge, s\aa\
ihvertfall som spennende muligheter for \aa\ forst\aa\ naturen bedre. Men, 
mere om dette i neste kapittel. 

Tenk deg n\aa\ et eksperiment hvor vi sender inn f.eks.~en jevn str\o m
av tennisballer mot to spalte\aa pninger, slik som vist i Figuren nedenfor. 
Anta ogs\aa\ at m\aa ten disse tennisballene treffer spaltene er vilk\aa rlig.
P\aa\ baksiden av de to spalte\aa pningene har vi en detektor som m\aa ler hvor mange baller som kommer i et gitt omr\aa de. 
Anta ogs\aa\ at vi holder p\aa\ en stund slik at 
intensitetsfordelingen $I_{12}$ 
for  tennisballene (som uttrykker hvor mange
baller som kommer inn per tid per arealenhet) ser ut som vist p\aa\ 
figuren.  Siden vi ikke kan si helt n\o yaktig hvor ballene vil havne, 
uttrykker intensitetsfordelingen en sannsynlighet for hvor det er st\o rst 
eller minst sannsynlighet for at ballene treffer. 
\begin{figure}[htbp]
%
\begin{center}

\setlength{\unitlength}{1mm}
\begin{picture}(140,70)

\thicklines
                 
   \put(0,0.0){\makebox(0,0)[bl]{
              \put(0,0){\vector(1,0){50}}
              \put(-5,0){\circle*{50}}
              \put(0,0){\vector(4,1){50}}
              \put(0,0){\vector(4,-1){50}}
              \put(10,10){\makebox(0,0){$v$}}
              \put(85,-25){\makebox(0,0){skjerm}}
              \put(85,10){\makebox(0,0){$I_{12}$}}
              \put(115,25){\makebox(0,0){$I_{1}$}}
              \put(115,-25){\makebox(0,0){$I_{2}$}}
              \put(60,-20){\line(0,-1){10}}
              \put(60,-10){\line(0,1){20}}
              \put(60,20){\line(0,1){10}}
              \put(100,-30){\line(0,1){60}}
              \put(120,-30){\line(0,1){60}}
              \qbezier(100,-25)(80, 0)(100, 25)
              \qbezier(120,-25)(110, 0)(120, 10)
              \qbezier(120,-10)(110, 0)(120, 25)
         }}
\end{picture}
\end{center}
\caption{Tennisballer med hastighet $v$ sendes inn mot to spalt\aa pningen. P\aa\ baksiden har vi en detektor som registrerer hvor tennisballene treffer.
Den totale intensitetsfordelingen er gitt ved $I_{12}$.} 
\end{figure}

Dersom vi lukker en av spaltene, f.eks.~den andre, vil vi m\aa le en
intesitetsfordeling som vist i kurven $I_1$. Lukker vi spalt en, finner vi
en tilsvarende fordeling $I_2$.
Det vi skal merke oss her er at disse intensitetsfordelingene adderes opp, dvs.
\[
   I_{12}=I_1+I_2.
\]

La oss gjenta dette eksperimentet, men denne gang med lys. 
Anta at vi har en lyskilde som sender ensfarget lys med gitt b\o lgelendge
$\lambda$ mot samme type skjerm, som vist i Figur 
\ref{fig:fotonspalt}. Istedet for \aa\ m\aa le
antall baller som treffer bak spaltene har vi n\aa\ f.eks.~en fotografisk
plate som m\aa ler intensiteten til lyset som passerer de to spaltene.   
Vi kaller avstanden mellom spalte\aa pningene for $a$. Dersom forholdet mellom
$a$ og lysets b\o lgelendge 
(vi kommer til dette i de neste to avsnittene)  er gitt ved
\[
   a/\lambda\sim 1,
\]
kan vi observere et diffraksjonsm\o nster ved den fotografiske plata som vist
i Figur \ref{fig:fotonspalt}. 
Et diffraksjonsm\o nster utviser intensitetstopper og bunner.
Intensiteten $I^{\gamma}_{12}$ er n\aa\ forskjellig 
fra eksperimentet
v\aa rt med tennisballene.  Lukker vi n\aa\  den andre spalten, kan vi,
avhengig av forholdet mellom spaltens st\o rrelse og lysets b\o lgelengde
observere et diffraksjonsm\o nster eller en mer glatt kurve som vist 
i figuren med intensiteten $I^{\gamma}_1$. Lukker vi spalt en, ser vi en tilsvarende
intensitetsfordeling $I^{\gamma}_2$. 
\begin{figure}[htbp]
%
\begin{center}

\setlength{\unitlength}{1mm}
\begin{picture}(140,70)

\thicklines
                 
   \put(0,0.0){\makebox(0,0)[bl]{
              \put(0,0){\vector(1,0){50}}
              \put(-5,0){\circle*{50}}
              \put(0,0){\vector(4,1){50}}
              \put(0,0){\vector(4,-1){50}}
              \put(10,10){\makebox(0,0){$\lambda$}}
              \put(85,-25){\makebox(0,0){skjerm}}
              \put(85,10){\makebox(0,0){$I^{\gamma}_{12}$}}
              \put(115,25){\makebox(0,0){$I^{\gamma}_{1}$}}
              \put(115,-25){\makebox(0,0){$I^{\gamma}_{2}$}}
              \put(60,-20){\line(0,-1){10}}
              \put(60,-10){\line(0,1){20}}
              \put(60,20){\line(0,1){10}}
              \put(100,-30){\line(0,1){60}}
              \put(120,-30){\line(0,1){60}}
              \qbezier(100,-25)(95, -20)(100, -15)
              \qbezier(100,-15)(85, -10)(100, -5)
              \qbezier(100,-5)(70, 0)(100, 5)
              \qbezier(100,5)(85, 10)(100, 15)
              \qbezier(100,15)(95, 20)(100, 25)

              \qbezier(120,-25)(110, 0)(120, 10)
              \qbezier(120,-10)(110, 0)(120, 25)
         }}
\end{picture}
\end{center}
\caption{Lys med b\o lgelengde  $\lambda$ sendes inn mot to spalt\aa pningen. P\aa\ baksiden har vi en detektor som registrerer intensiteten.
Den totale intensitetsfordelingen er gitt ved $I^{\gamma}_{12}$.\label{fig:fotonspalt}} 
\end{figure}
Pr\o ver vi \aa\ addere de enkelte intensitetsfordelingene finner vi at
\[
   I^{\gamma}_{12}\ne I^{\gamma}_1+I^{\gamma}_2!
\]
Vi skal utlede det matematiske uttrykket i avsnittet om b\o lgel\ae ren.
Vi kan alst\aa\ ikke addere opp intensiteter for lys. Det at lys 
utviser et diffraksjonsm\o nster p\aa\ lik linje med hva f.eks.~vannb\o lger
gj\o r, f\o rte til forst\aa elsen av lys som b\o lger. 

Hva om vi erstatter lys med elektroner i Figur\ref{fig:fotonspalt}?   
Anta at du har en eller
annen kilde som sender elektroner med en gitt b\o lgelengde 
mot spaltene. I tillegg har vi en detektor ved skjermen som teller 
antall elektroner som passerer de to spaltene. Med begge spalter
\aa pne kan vi igjen observere en tilsvarende intensitets fordeling
som for lys. Elektronene utviser alts\aa\ b\o lgeegenskaper, i klar
strid med v\aa r intuitive oppfatning av partikler. 
Pr\o ver vi \aa\ lukke en av spaltene, 
ser vi en tilsvarende intensitetsfordeling som for lys. Legger vi sammen
intensitetsfordelingene 
kommer vi fram til samme
konklusjon, nemlig
\[
   I^{e^-}_{12}\ne I^{e^-}_1+I^{e^-}_2!
\]

Materien tilskrives  alts\aa\ b\o lgegenskaper, p\aa\ lik linje med 
fotonene.

La oss utvide eksperimentet med elektronene. 
Vi sender et elektron per sekund mot de to spaltene. Samtidig er
vi n\aa\ litt mer ambisi\o se. I tillegg til diffraksjonsm\o nsteret
\o nsker vi \aa\ finne ut fra hvilken spalt et enkelt elektron
gikk gjennom. Vi \o nsker \aa\ b\aa de studere b\o lgeengenskapene 
til elektronene (vi vil ha et diffraksjonsm\o nster) samtidig
som vi \o nsker \aa\ si noe om hvilken spalt det gikk gjennom,
dvs.~vi \o nsker \aa\ studere dets partikkelegenskaper.

Tenk deg da at vi plasserer en detektor ved spaltene.
Denne detektoren  sender ut fotoner som kolliderer 
med de innkommende elektronene, jfr.~Compton spredning. 
Detektoren er bygd slik 
at vi ser et skarpt lysglimt ved enten den ene eller andre spalten, 
avhengig
av hvor elektronet gikk gjennom. 
Ellers har eksperimentet v\aa rt samme oppsett som i 
Figur \ref{fig:fotonspalt}. 

Hva skjer n\aa\ ? Jo, vi observerer et lysglimt hver gang et elektron
passerer en av spaltene. Men, intensitetsfordelingen v\aa r utviser ikke
lenger et diffraksjonsm\o nster som vist i Figur \ref{fig:fotonspalt}! 
V\aa rt lille \o nske
om \aa\ studere b\aa de partikkel og b\o lgegenskaper {\em samtidig} i ett
og samme eksperiment er ikke mulig. 
Det var basert p\aa\ slike observasjoner at Heisenberg formulerte sin 
uskarphetsrelasjon. 
Per dags dato finnes det ikke noe eksperiment som bryter  med
denne relasjonen. 

Hensikten med dette kapittelet blir dermed \aa\ introdusere for dere
to viktige postulat om naturen, de Broglies hypotese om materiens
b\o lge og partikkelegenskaper og Heisenbergs uskarphetsrelasjon.


\section{De Broglies hypotese}    

I forrige
kapittel\footnote{Lesehenvisning her er kap 4-1 og 4-2.} 
diskuterte vi egenskapene til lys og
fant fra Comptons effekt, fotoelektrisk effekt og 
R\"ontgenstr\aa ling at e.m.~str\aa ling utviser b\aa de
partikkel og b\o lgeegenskaper. I noen tilfeller
kan vi avdekke b\o lgeegenskaper, i andre tilfeller ser vi
partikkelegenskapene.

Det er viktig \aa\ ha klart for seg skillet mellom partikler
og b\o lger siden de representer de eneste moder for transport
av energi. Klassisk har en partikkel
egenskaper som 
posisjon, bevegelsesmengde, kinetisk energi, masse og elektrisk
ladning. En klassisk b\o lge derimot beskrives ved st\o rrelser som
b\o lgelengde, frekvens, hastighet, amplitude, intensitet,
energi og bevegelsesmengde. Den viktigste forskjellen mellom
en klassisk partikkel og b\o lge er at partikkelen kan {\bf lokaliseres}, dvs.~har en skarpt definert posisjon, mens en b\o lge er spredt
utover rommet. Vi skal se mer av dette b\aa de n\aa r vi diskuterer
b\o lgel\ae re og i tilknytting Heisenbergs uskarphetsrelasjon,
og ikke minst i v\aa r diskusjon av kvantemekanikken.   

de Broglie tok utgangspunkt i resultatetene fra e.m.~str\aa ling og foreslo, utifra ideen om symmetrier i naturen, at ogs\aa\
materie, dvs.~partikler med endelig masse som f.eks.~elektroner,
protoner, atomer m.m.,utviser partikkel og b\o lge
egenskaper. For fotoner har vi relasjonen mellom b\o lge
og partikkel egenskaper gitt ved $E=h\nu$ og
$\lambda=h/p$, med Plancks konstant som knytter sammen disse to
relasjonene. de Broglie foreslo at liknende relasjoner
ogs\aa\ skulle gjelde for materie og postulerte at 
f.eks.~elektroner kunne tilordnes en b\o lgelengde
\be
    \lambda=\frac{h}{p}=\frac{h}{mv}.
\ee
Dette er de Broglies hypotese.

Det er en viktig ting \aa\ merke seg, som vi ogs\aa\ kommer
tilbake i neste avsnitt. For et foton har vi
en enkel relasjon som gir oss b\aa de b\o lgelengde og frekvens,
nemlig $\lambda\nu=c$, med gitt energi og bevegelsemengde.
For massive partikler derimot, trenger vi separate relasjoner
for henholdsvis b\o lgelengde $\lambda=h/p$ og 
frekvens $\nu=E/h$. 

Basert p\aa\ de Broglies hypotese foreslo Elsasser i 1926 at
b\o lgeegenskapene til materie burde kunne testes
p\aa\ lik linje med f.eks.~slik en observerte R\"ontgenstr\aa ler,
dvs.~ved \aa\ sende h\o genergetiske elektroner mot et metall
og detektere p\aa\ ei fotografisk plate et eventuelt diffraksjonsm\o nster for fotonene. Her skulle en alts\aa\ pr\o ve \aa\ detektere 
et diffraksjonsm\o nster for elektroner. Eksperimentet som ble
gjennomf\o rt av Davisson og Germer besto i \aa\ sende
elektroner mot en krystall av  nikkel for deretter \aa\ pr\o ve \aa\
detektere de utg\aa ende spredte elektroner som funksjon av
innkommende energi og spredningsvinkel $\theta$. 
Atomene i nikkel krystallen, med interatom\ae r avstand
$d$ skulle dermed fungere som diffraksjonssentra og i henhold
til b\o lgel\ae re skulle en dermed kunne observere et 
diffraksjonsm\o nster ved detektoren. Fra diffraksjonsteori,
som vi skal utlede i neste avsnitt, kan en vise at det er
en relasjon mellom b\o lgelengden til elektronene, avstanden
mellom atomene som fungerer som diffraksjonssentra og spredningsvinkelen
$\theta$ gitt ved
\be 
   n\lambda = 2dsin\theta,
\ee
hvor $n$ er et heltall\footnote{Vi utleder denne formlen i neste
avsnitt. Her tillater vi oss \aa\ bare sette opp resultatet.} 
Figur \ref{fig:xray3} viser en skisse for dette diffraksjonseksperimentet.
\begin{figure}[h]
\begin{center}
{\centering
\mbox
{\psfig{figure=xray3.ps,height=8cm,width=12cm}}
}
\end{center}
\caption{Avstanden mellom atomene som fungerer som 
diffraksjonssentra med spredningsvinkelen
$\theta$ for innkommende og utg\aa ende elektroner.\label{fig:xray3}}
\end{figure}

Det som er viktig for den videre forst\aa else, er at 
diffraksjon skyldes en forstyrrelse av en innkommende b\o lge
hvor det forstyrrende element, om det er en spalte\aa pning
eller avstanden mellom atomer i en krystall, er p\aa\ st\o rrelse
med b\o lgelengden til den innkommende b\o lge, dvs.
\be
    \lambda/d \sim 1.
\ee
Dersom $\lambda/d\rightarrow 0$ kan vi ikke observere noe
diffraksjonsm\o nster. 

For \aa\ ta et eksempel, anta at vi sender en kontinuerlig
str\o m av baller som veier et 1 kg mot en spalte\aa pning
$d$. Anta at hastigheten er gitt ved $v=10$ ms$^{-1}$.
Bruker vi de Broglies hypotese  
finner vi at b\o lgelengden til denne innkommende str\o m av
baller er gitt ved
\be
   \lambda_{ball} =\frac{h}{p}=\frac{h}{mv}=6.6\times 10^{-25}
    \times 10^{-10} \hspace{0.1cm}\mathrm{m} ,
\ee
og dersom vi overhodet skal v\ae re i stand til \aa\ observere
et diffraksjonsm\o nster for ballene som passerer \aa pningen,
b\o r vi ihvertfall ha en spalte\aa pning $d\sim 10^{-35}$ m! 
\AA\ lage en ball p\aa\ et kg som passer til en slik \aa pning
overlates hermed til ekspertene!
I praksis vil det si at $\lambda_{ball}/d \sim 0$ og vi vil 
aldri v\ae re i stand til \aa\ observere et diffraksjonsm\o nster.


Derimot, dersom vi sender elektroner med kinetisk energi
p\aa\ $E_{kin}=100$ eV mot en nikkel krystall, hvor $d=0.91$ \AA\,
har vi 
\be 
   \lambda_{e^-}=\frac{h}{p}=\frac{h}{\sqrt{2m_eE_{kin}}}=1.2\times 10^{-10} \hspace{0.1cm}\mathrm{m},
\ee
og vil v\ae re i stand til \aa\ observere et diffraksjonsm\o nster.

Det var nettopp det Davisson og Germer s\aa\ .
For en spredningsvinkel p\aa\ $50$ grader 
observerte de en interferens topp som kun kunne forklares dersom elektroner utviser ogs\aa\ 
b\o lgeegenskaper. Se avsnittene 4-1 og 4-2
i l\ae reboka for en mer utf\o rlig diskusjon av eksperimentene
til Davisson og Germer og ogs\aa\ Thompson.

F\o r vi g\aa r inn p\aa\ detaljer om interferens og diffraksjon,
kan vi kort oppsumere denne delen med f\o lgende.
For \aa\ observere et diffraksjonsm\o nster, som igjen
er en b\o lgeegenskap, b\o r vi ha et forhold
$\lambda/d\sim 1$. Har vi derimot $\lambda/d\sim 0$, kan ikke
noe slikt m\o nster observeres. 
Tenker vi deretter p\aa\ $\lambda=h/p$ har vi at
det er forholdet 
\be
   \frac{h}{pd}=\frac{h}{mvd},
\ee
som er viktig. Velger vi $h=0$ skal ikke
en partikkel utvise b\o lgeegenskaper. For massive
partikler som ballen i eksemplet ovenfor, blir $p$ s\aa\
stor at vi knapt kan detektere b\o lgelengden.
For elektronet derimot, er massen s\aa\ liten at ogs\aa\
bevegelsesmengden $p=mv$ blir liten, til tross for at hastigheten
$v$ kan bli sv\ae rt s\aa\ stor. Er i tillegg $d$ p\aa\ st\o rrelsesorden med $\lambda$ er det store muligheter for \aa\ observere
et diffraksjonsm\o nster. mer om dette i neste avsnitt. 


  
\section{Diffraksjon, fasehastighet og gruppehastighet}
\label{sec:boelgelaere}
I dette avsnittet skal\footnote{Lesehenvisning er kap 4-6, sidene 205-215.}
vi presentere en del emner fra b\o lgel\ae re som vi vil f\aa\ bruk for b\aa de
i tilknytting den p\aa f\o lgende diskusjonen om Heisenbergs uskarphetsrelasjon
og senere n\aa r vi g\aa r l\o s p\aa\ kvantemekanikken. 
\subsection{B\o lgepakker, gruppe og fasehastighet}
Her skal vi rekapitulere en del begrep fra FYS2140 samt innf\o re begrepet
gruppehastighet. Fra sistnevnte kurs har dere sett at Maxwells likninger gir 
en b\o lgelikning p\aa\ formen
\be
  \frac{\partial^2 {\cal E}({\bf x},t)}{\partial {\bf x}^2}  =
  \epsilon_0\mu_0\frac{\partial^2 {\cal E}({\bf x},t)}{\partial t^2},
\label{eq:ewave}
\ee 
hvor ${\cal E}$ er det elektriske feltet. Tilsvarende likning 
f\aa r vi for det magnetiske feltet ved \aa\ erstatte ${\cal E}$ med
${\cal B}$. 
Vi kan generalisere denne likningen ved \aa\ skrive den som
\be
  v^2 \frac{\partial^2 \psi({\bf x},t)}{\partial {\bf x}^2}  =
  \frac{\partial^2 \psi({\bf x},t)}{\partial t^2},
\label{eq:generalwave}
\ee 
hvor $\psi$ kan representere det elektriske ${\cal E}$ 
eller magnetiske ${\cal B}$ felt
og hvor
\be
   v^2=c^2=\frac{1}{\epsilon_0\mu_0}
\ee
hastigheten til lyset for det e.m.~tilfellet. 
Heretter i dette kurset kommer  vi konsekvent til \aa\ bruke 
$\psi$ som symbol for b\o lgefunksjonen. 

Slik likning (\ref{eq:generalwave})  st\aa r kan den representere
en b\o lgelikning for flere systemer, fra e.m.~felt til trykkb\o lger i en gass eller en
svingende streng. N\aa r vi kommer til Schr\"odingers likning skal vi se at
den tidsavhengige delen av 
denne likning blir forskjellig. Istedet for den andrederiverte av 
b\o lgefunksjonen mhp.~tiden $t$ har vi en f\o rste deriverte mhp.~$t$.

L\o sningen til likning (\ref{eq:generalwave}) kan skrives p\aa\  generell form som
\be
\psi(x,t)=f(x\mp v_ft),
\ee
som skal representere en b\o lge som propagerer som funksjon av tid og posisjon
med en hastighet $v_f$. Sistenevnte kalles
fasehastigheten og er hastigheten b\o lgen reiser med. Minustegnet svarer til en forover (i $x$-retning) reisende b\o lge,
mens plusstegnet svarer til ei b\o lge som brer seg bakover.
Figur \ref{fig:fasev} illustrerer
noe av dette. 
\begin{figure}[h]
   \setlength{\unitlength}{1mm}
   \begin{picture}(100,50)
   \put(25,0){\epsfxsize=12cm \epsfbox{fig6.eps}}
   \end{picture}
\label{fig:waveex}
\caption{B\o lge som propagerer forover med fasehastighet $v_f$ og
         b\o lgelengde $\lambda$.\label{fig:fasev}} 
\end{figure}
Ei spesiell l\o sning til b\o lgelikningen er gitt ved sinus eller cosinus
funksjoner, f.eks.~ved 
\be
\psi(x,t)=Asin[k(x- v_ft)]=Asin[k(x- v_ft)+2\pi],
\ee
som indikerer at n\aa r 
$x\rightarrow x+2\pi/k$ s\aa\ har $\psi$ samme verdi. St\o rrelsen $A$ er amplituden.
Alternativt kunne vi ha brukt cosinus-funksjonen som l\o sning
\be
\psi(x,t)=Acos[k(x- v_ft)],
\ee
eller uttrykke l\o sningen som
\be
   \psi_i=Ae^{i(kx-kv_f t)}.
\ee
Fra FY101 har vi sett at b\o lgelengden kan skrives som  
\be
    \lambda=\frac{2\pi}{k},
\ee
hvor $k$ 
kalles b\o lgetallet.
Vi kan da omskrive b\o lgefunksjonen til
\be
\psi(x,t)=Asin[\frac{2\pi}{\lambda}(x- v_ft)],
\ee
og med vinkelfrekvensen
\be
   \omega=2\pi\nu=kv=\frac{2\pi v_f}{\lambda},
\ee
har vi 
\be
\psi(x,t)=Asin(kx- \omega t)
\ee
Nullpunktene svarer til noder
\be
   \frac{2\pi}{\lambda}(x_n-vt)=n\pi \hspace{0.3cm} n=0,\pm 1,\pm 2, \dots
\ee
som gir
\be
   x_n=\frac{n\lambda}{2}+vt.
\ee
Fasehastigheten $v_f$ kan ogs\aa\ uttrykkes vha.~vinkelfrekvensen og b\o lgetallet ved
\be
   v_f=\frac{\omega}{k}
   \label{eq:vf}.
\ee


La oss anta at den formen vi har funnet p\aa\ l\o sningen til b\o lgefunksjonen
skal svare til b\o lgefunksjonen for et foton eller elektron. 
Dersom vi \o nsker \aa\ assosiere fasehastigheten $v_f$ 
med hastigheten $v$ til en partikkel f\aa r vi et problem. Dette ser vi fra
\be
   v_f=\lambda\nu=\frac{hE}{ph}=\frac{E}{p},
\ee
hvor vi har brukt de Broglie sin hypotese. Vi har sneket inn en ansats 
for materieb\o lger om
at $E=h\nu=\hbar\omega$, noe som vi skal komme tilbake til under 
Schr\"odingers likning. Antar vi at dette dreier seg om
ikke relativistiske partikler med energi $E=mv^2/2$ og bevegelsesmengde
 $p=mv$ finner vi at
\be
   v_f=\frac{E}{p}=\frac{mv^2/2}{mv}=\frac{v}{2}.
\ee
Dersom vi \o nsker at b\o lgen i Figur \ref{fig:fasev} 
skal representere en partikkel, b\o r b\o lgehastigheten (fasehastigheten) 
ogs\aa\ svare til partikkelens hastighet. Dette stemmer ikke helt i siste
likning.
Merk ogs\aa\ at b\o lgen i Figur \ref{fig:fasev} 
svarer til en kontinuerlig harmonisk b\o lge 
av uendelig utstrekning, med en gitt 
b\o lgelengde
og frekvens. 
En slik b\o lge 
kan ikke representere en partikkel som befinner seg p\aa\ et bestemt sted
i  rommet.
Ei heller er den egnet til \aa\ sende et signal,
siden et signal, tenk bare p\aa\ n\aa r dere roper etter noen, er noe som starter et sted
og slutter et annet sted etter et gitt tidspunkt. 
En slik b\o lge har en form som likner mer p\aa\ b\o lgen i Figur \ref{fig:fasev2}. Denne figuren viser  en s\aa kalt superposisjon (addisjon) av flere
b\o lger. Vi skal se at en slik superposisjon av b\o lger, som kalles b\o lgepakke, 
leder til begrepet 
gruppehastighet, som igjen vil gi oss en hastighet for b\o lgepakken som svarer til en
partikkels hastighet. 
\begin{figure}[h]
   \setlength{\unitlength}{1mm}
   \begin{picture}(100,80)
   \put(25,0){\epsfxsize=12cm \epsfbox{fig7.eps}}
   \end{picture}
\caption{Eksempel p\aa\ superponering av to b\o lger ved tiden $t=0$ og
$dk=k_1-k_2$.\label{fig:fasev2} }
\end{figure}
Som et eksempel for \aa\ belyse dette velger vi 
\aa\ se p\aa\ superponering av to b\o lger med ulik vinkelfrekvens
$\omega_1$ og $\omega_2$ 
og b\o lgetall $k_1$ og $k_2$. Vi antar at forskjellen mellom disse verdiene
er slik at
\be
    \omega_1-\omega_2 << 1 \hspace{0.2cm} og \hspace{0.2cm} k_1-k_2 << 1.
\ee
Legger vi sammen disse to b\o lgene har vi
\be
   \psi=\psi_1+\psi_2=A_1sin(k_1x- \omega_1 t)+A_2sin(k_2x- \omega_2 t),
\ee
og setter vi amplitudene like $A_1=A_2=A$ har vi
\be
   \psi(x,t)=A(sin(k_1x- \omega_1 t)+sin(k_2x- \omega_2 t) ),
\ee
som vha.~den trigonometriske relasjonen
\be
   sin X+sinY=2cos\left(\frac{X-Y}{2}\right)sin\left(\frac{X+Y}{2}\right),
\ee
gir
\be
   \psi(x,t)=2Acos\left(\frac{(k_1-k_2)x-(w_1-w_2)t}{2}\right)sin\left(\frac{(k_1+k_2)x-(w_1+w_2)t}{2}\right).
\ee
Setter vi s\aa\
\be
    \omega_1+\omega_2 \approx 2\omega_1=2\omega \hspace{0.2cm} og \hspace{0.2cm}k_1+k_2 \approx2k_1=2k,
\ee
har vi
\be
   \psi(x,t)=2Acos(\frac{(k_1-k_2)x-(w_1-w_2)t}{2})sin(kx-\omega t).
\ee
Leddet med cosinus funksjonen kalles for den modulerte del og vi sier at
amplituden $A$ er modulert gitt ved faktoren 
\be
   2Acos(\frac{(k_1-k_2)x-(w_1-w_2)t}{2}).
\ee
Den modulerte amplituden svarer til en hastighet 
\be
   \frac{\omega_1-\omega_2}{k_1-k_2}\rightarrow \frac{d\omega}{dk}=v_g,
\ee
som er definisjonen p\aa\ gruppehastigheten. Se Figur 2.4 for et eksempel ved 
$t=0$. 
Vi kan si at b\o lgepakkens amplitude reiser med en hastighet som svarer 
til gruppehastigheten.
Det er denne hastigheten som skal svare til partikkelens hastighet.
Det kan vi se dersom vi bruker at 
\[
   \omega=2\pi\nu=2\pi\frac{E}{h},
\]
og
\[
   k=2\pi\frac{p}{h},
\] som gir at
\be
   v_g=\frac{d\omega}{dk}=\frac{d(2\pi E/h)}{d(2\pi p/h)}=\frac{dE}{dp},
\ee
og antar vi at v\aa r partikkel kan beskrives vha.~ikke-relativistisk
mekanikk har vi
\be
   v_g=\frac{dE}{dp}=\frac{d(mv^2/2)}{d(mv)}=v,
\ee
dvs.~gruppehastigheten svarer til partikkelens hastighet. 
Gruppehastigheten til en materieb\o lge er lik hastigheten til partikkelen
hvis bevegelse den skal representere. de Broglie sitt postulat er dermed
konsistent. Vi f\aa r samme resultat dersom vi hadde antatt en relativistisk
beskrivelse.

selv om vi bare har addert to b\o lger med ulike frekvenser og b\o lgetall,
s\aa\ holder dette resultatet for mange b\o lger som adderes og har frekvenser
og b\o lgetall n\ae r hverandre.


Dersom $dv_f/dk \ne 0$ kan vi bruke $\omega =kv_f$ til \aa\ vise at
\be
   v_g=v_f+k\frac{dv_f}{dk}.
\ee
I et dispersivt medium vil et signal reise med gruppehastigheten.

For en ikke
relativistisk partikkel er $v_g=v=p/m=\hbar k/m$ slik at vi f\aa r
\be
   \frac{d\omega(k)}{dk}=v_g=v=\frac{\hbar k}{m},
\ee
som gir n\aa r vi integrer opp (og setter konstanten fra 
integrasjonen lik null) 
\be 
   \omega(k)=\frac{\hbar k^2}{2m}=\frac{\hbar^2 k^2}{2m\hbar}.
\ee
En slik relasjon kalles for en dispersjonsrelasjon.
Med $\omega=2\pi\nu$ har vi
\be
   \nu(k)=\frac{\hbar^2 k^2}{2m \hbar 2\pi}=
   \frac{\hbar^2 k^2}{2m h}=\frac{p^2}{2m}\frac{1}{h}=\frac{E}{h},
\ee
som ventet! 
Utifra likning  (\ref{eq:vf}) finner vi s\aa\ fasehastigheten $v_f$.

\subsection{Interferens}

Interferens forekommer n\aa r to eller flere b\o lger sammenfaller i tid og rom
og gir opphav til en maksimal amplitude eller en minimal amplitude.
Intensiteten til f.eks.~e.m.~str\aa ling som treffer ei fotografisk plate
er proporsjonal med kvadratet av b\o lgefunksjonen, som igjen er proporsjonal
med amplituden $A$.
{\bf Intensitet er definert som energi som kommer inn per sekund per areal.} 
Dersom amplituden, som er resultatet av summen av alle amplituder, har et 
maksimum i et gitt punkt $P$, vil en kunne observere et kraftig m\o nster
p\aa\ f.eks.~ei fotografisk plate. Vi snakker da om konstruktiv interferens.
Dersom vi ikke observerer noe har vi destruktiv interferens.

La oss n\aa\ regne ut intensiteten for lys som blir sendt fra $N$ 
lyskilder mot en skjerm ved avstand $D$. Avstanden mellom hver lyskilde er
$a$. Vi antar at vi har med koherente lyskilder \aa\ gj\o re, at
amplituden til alle b\o lgefunksjonene $A_i$ er like,
\be
   A_1=A_2=A_3\dots = A_N=A,
\ee
og at b\aa de vinkelfrekvens og b\o lgetall er like.
\begin{figure}[h]
   \setlength{\unitlength}{1mm}
   \begin{picture}(100,120)
   \put(25,0){\epsfxsize=6cm \epsfbox{fig1.eps}}
   \end{picture}
\label{fig:interferensoppsett}
\caption{Skjematisk oppsett med $N$ lyskilder som sender koherent lys mot en skjerm.
         Avstanden mellom hver lyskilde er $a$, og $r_2-r_1$ kalles den optiske
         veilengde. Skjermen antas \aa\ v\ae re langt borte fra lyskildene, slik
         at linjene som skal representere b\o lgene kan antas \aa\ 
         v\ae re parallelle.}
\end{figure}
En slik b\o lgefunksjon vil ved skjermen v\ae re gitt ved
\be
   \psi_i=Ae^{i(kr_i-\omega t)},
\ee
hvor vi n\aa\ velger \aa\ bruke den komplekse representasjonen for b\o lgefunsjonen. St\o rrelsen $r_i$ er {\bf avstanden fra lyskilde $i$ til skjermen}. 
Vi antar ogs\aa\ at skjermen er s\aa\ langt borte at vi kan betrakte 
linjene fra lyskilden til skjermen som tiln\ae rma parallelle, 
se Figur \ref{fig:interferensoppsett} 
for et mulig oppsett med $N$ lyskilder.

Den totale b\o lgefunksjonen er n\aa\ gitt ved
\be
   \psi=A\left( e^{i(kr_1-\omega t)}+e^{i(kr_2-\omega t)}+\dots +e^{i(kr_N-\omega t)}\right),
\ee
eller
\be
   \psi=Ae^{i(kr_1-\omega t)}\left(1+e^{i(k(r_2-r_1))}+\dots +e^{i(k(r_N-r_1)}\right).
\ee
Definerer vi s\aa\
\be
  \delta=k(r_2-r_1),\hspace{0.2cm}  2\delta=k(r_3-r_1),\hspace{0.2cm} \dots, (N-1)\delta=k(r_N-r_1),
\ee
har vi
\be
   \psi=Ae^{i(kr_1-\omega t)}\left(1+e^{i\delta}+\dots +e^{i\delta(N-1)}\right),
\ee
som kan summeres opp ved formelen til ei geometrisk rekke for \aa\ gi
\be
   \psi=Ae^{i(kr_1-\omega t)}\frac{e^{iN\delta}-1}{e^{i\delta}-1},
\ee
eller
\be
      \psi=Ae^{i(kr_1-\omega t)}\frac{e^{iN\delta/2}}{e^{i\delta/2}}\frac{e^{iN\delta/2}-e^{-iN\delta/2}}{e^{i\delta/2}-e^{-i\delta/2}},
\ee
som vi kan skrive om vha.~definisjonen for sinus-funksjonen som
\be
      \psi = Ae^{i(kr_1-\omega t)}e^{i(N-1)\delta/2}\frac{sin(N\delta/2)}{sin(\delta/2)}.
\ee
Siden vi har antatt at linjene som treffer skjermen er parallelle, har vi at
den {\bf optiske veilengden} $r_2-r_1$ er gitt ved 
\be
   r_2-r_1=asin\theta,
\ee
som sammen med 
\be
    k(r_2-r_1)=\frac{2\pi}{\lambda}(r_2-r_1)=\frac{2\pi}{\lambda}asin\theta =\delta,
\ee
og definisjonen 
\be
   R=\frac{1}{2}(N-1)asin\theta+r_1,
\ee
som svarer til en linje med lengde $R$ trukket fra midtpunktet til alle lyskildene,
kan vi skrive om uttrykket for $\psi $ som
\be
  \psi = Ae^{i(kR-\omega t)}\frac{sin(N\delta/2)}{sin(\delta/2)},
\ee
eller
\be
  \psi = Ae^{i(kR-\omega t)}\frac{sin(N\frac{\pi}{\lambda}asin\theta)}{sin(\frac{\pi}{\lambda}asin\theta)}.
\ee
Intensiteten $I$ er proporsjonal med kvadratet av b\o lgefunskjonen, og siden
$|e^{i(kR-\omega t)}|^2=1$ har vi
\be
    I=I_0\frac{sin^2(N\frac{\pi}{\lambda}asin\theta)}
               {sin^2(\frac{\pi}{\lambda}asin\theta)}.
\ee
Konstanten $I_0$ inneholder bla.~$|A|^2$ samt konstanter som gir oss de riktige
enhetene. 
Intensiteten har en maksimal verdi ved
\be
   asin\theta_m=m\lambda, \hspace{0.1cm} m=0,\pm 1, \pm 2, \dots ,
\ee
som svarer til konstruktiv interferens.

For $N=2$ har vi det mer velkjente uttrykket
\be
    I=4I_0cos^2(\frac{\pi}{\lambda}asin\theta),
\ee
der vi har brukt at $\sin(2x)=2\cos(x)\sin(x)$
Eksempler p\aa\ mulige interferensm\o nstre er vist i Figur 
\ref{fig:interferensmoenster} for $N=2$ og $N=8$. Vi merker oss at etterhvert som vi 
\o ker $N$ blir interferensmaksima skarpere og skarpere. 
\begin{figure}
% GNUPLOT: LaTeX picture with Postscript
\begingroup%
  \makeatletter%
  \newcommand{\GNUPLOTspecial}{%
    \@sanitize\catcode`\%=14\relax\special}%
  \setlength{\unitlength}{0.1bp}%
{\GNUPLOTspecial{!
%!PS-Adobe-2.0 EPSF-2.0
%%Title: inter2.tex
%%Creator: gnuplot 3.7 patchlevel 0.2
%%CreationDate: Fri Feb 11 10:17:40 2000
%%DocumentFonts: 
%%BoundingBox: 0 0 360 216
%%Orientation: Landscape
%%EndComments
/gnudict 256 dict def
gnudict begin
/Color false def
/Solid false def
/gnulinewidth 5.000 def
/userlinewidth gnulinewidth def
/vshift -33 def
/dl {10 mul} def
/hpt_ 31.5 def
/vpt_ 31.5 def
/hpt hpt_ def
/vpt vpt_ def
/M {moveto} bind def
/L {lineto} bind def
/R {rmoveto} bind def
/V {rlineto} bind def
/vpt2 vpt 2 mul def
/hpt2 hpt 2 mul def
/Lshow { currentpoint stroke M
  0 vshift R show } def
/Rshow { currentpoint stroke M
  dup stringwidth pop neg vshift R show } def
/Cshow { currentpoint stroke M
  dup stringwidth pop -2 div vshift R show } def
/UP { dup vpt_ mul /vpt exch def hpt_ mul /hpt exch def
  /hpt2 hpt 2 mul def /vpt2 vpt 2 mul def } def
/DL { Color {setrgbcolor Solid {pop []} if 0 setdash }
 {pop pop pop Solid {pop []} if 0 setdash} ifelse } def
/BL { stroke userlinewidth 2 mul setlinewidth } def
/AL { stroke userlinewidth 2 div setlinewidth } def
/UL { dup gnulinewidth mul /userlinewidth exch def
      10 mul /udl exch def } def
/PL { stroke userlinewidth setlinewidth } def
/LTb { BL [] 0 0 0 DL } def
/LTa { AL [1 udl mul 2 udl mul] 0 setdash 0 0 0 setrgbcolor } def
/LT0 { PL [] 1 0 0 DL } def
/LT1 { PL [4 dl 2 dl] 0 1 0 DL } def
/LT2 { PL [2 dl 3 dl] 0 0 1 DL } def
/LT3 { PL [1 dl 1.5 dl] 1 0 1 DL } def
/LT4 { PL [5 dl 2 dl 1 dl 2 dl] 0 1 1 DL } def
/LT5 { PL [4 dl 3 dl 1 dl 3 dl] 1 1 0 DL } def
/LT6 { PL [2 dl 2 dl 2 dl 4 dl] 0 0 0 DL } def
/LT7 { PL [2 dl 2 dl 2 dl 2 dl 2 dl 4 dl] 1 0.3 0 DL } def
/LT8 { PL [2 dl 2 dl 2 dl 2 dl 2 dl 2 dl 2 dl 4 dl] 0.5 0.5 0.5 DL } def
/Pnt { stroke [] 0 setdash
   gsave 1 setlinecap M 0 0 V stroke grestore } def
/Dia { stroke [] 0 setdash 2 copy vpt add M
  hpt neg vpt neg V hpt vpt neg V
  hpt vpt V hpt neg vpt V closepath stroke
  Pnt } def
/Pls { stroke [] 0 setdash vpt sub M 0 vpt2 V
  currentpoint stroke M
  hpt neg vpt neg R hpt2 0 V stroke
  } def
/Box { stroke [] 0 setdash 2 copy exch hpt sub exch vpt add M
  0 vpt2 neg V hpt2 0 V 0 vpt2 V
  hpt2 neg 0 V closepath stroke
  Pnt } def
/Crs { stroke [] 0 setdash exch hpt sub exch vpt add M
  hpt2 vpt2 neg V currentpoint stroke M
  hpt2 neg 0 R hpt2 vpt2 V stroke } def
/TriU { stroke [] 0 setdash 2 copy vpt 1.12 mul add M
  hpt neg vpt -1.62 mul V
  hpt 2 mul 0 V
  hpt neg vpt 1.62 mul V closepath stroke
  Pnt  } def
/Star { 2 copy Pls Crs } def
/BoxF { stroke [] 0 setdash exch hpt sub exch vpt add M
  0 vpt2 neg V  hpt2 0 V  0 vpt2 V
  hpt2 neg 0 V  closepath fill } def
/TriUF { stroke [] 0 setdash vpt 1.12 mul add M
  hpt neg vpt -1.62 mul V
  hpt 2 mul 0 V
  hpt neg vpt 1.62 mul V closepath fill } def
/TriD { stroke [] 0 setdash 2 copy vpt 1.12 mul sub M
  hpt neg vpt 1.62 mul V
  hpt 2 mul 0 V
  hpt neg vpt -1.62 mul V closepath stroke
  Pnt  } def
/TriDF { stroke [] 0 setdash vpt 1.12 mul sub M
  hpt neg vpt 1.62 mul V
  hpt 2 mul 0 V
  hpt neg vpt -1.62 mul V closepath fill} def
/DiaF { stroke [] 0 setdash vpt add M
  hpt neg vpt neg V hpt vpt neg V
  hpt vpt V hpt neg vpt V closepath fill } def
/Pent { stroke [] 0 setdash 2 copy gsave
  translate 0 hpt M 4 {72 rotate 0 hpt L} repeat
  closepath stroke grestore Pnt } def
/PentF { stroke [] 0 setdash gsave
  translate 0 hpt M 4 {72 rotate 0 hpt L} repeat
  closepath fill grestore } def
/Circle { stroke [] 0 setdash 2 copy
  hpt 0 360 arc stroke Pnt } def
/CircleF { stroke [] 0 setdash hpt 0 360 arc fill } def
/C0 { BL [] 0 setdash 2 copy moveto vpt 90 450  arc } bind def
/C1 { BL [] 0 setdash 2 copy        moveto
       2 copy  vpt 0 90 arc closepath fill
               vpt 0 360 arc closepath } bind def
/C2 { BL [] 0 setdash 2 copy moveto
       2 copy  vpt 90 180 arc closepath fill
               vpt 0 360 arc closepath } bind def
/C3 { BL [] 0 setdash 2 copy moveto
       2 copy  vpt 0 180 arc closepath fill
               vpt 0 360 arc closepath } bind def
/C4 { BL [] 0 setdash 2 copy moveto
       2 copy  vpt 180 270 arc closepath fill
               vpt 0 360 arc closepath } bind def
/C5 { BL [] 0 setdash 2 copy moveto
       2 copy  vpt 0 90 arc
       2 copy moveto
       2 copy  vpt 180 270 arc closepath fill
               vpt 0 360 arc } bind def
/C6 { BL [] 0 setdash 2 copy moveto
      2 copy  vpt 90 270 arc closepath fill
              vpt 0 360 arc closepath } bind def
/C7 { BL [] 0 setdash 2 copy moveto
      2 copy  vpt 0 270 arc closepath fill
              vpt 0 360 arc closepath } bind def
/C8 { BL [] 0 setdash 2 copy moveto
      2 copy vpt 270 360 arc closepath fill
              vpt 0 360 arc closepath } bind def
/C9 { BL [] 0 setdash 2 copy moveto
      2 copy  vpt 270 450 arc closepath fill
              vpt 0 360 arc closepath } bind def
/C10 { BL [] 0 setdash 2 copy 2 copy moveto vpt 270 360 arc closepath fill
       2 copy moveto
       2 copy vpt 90 180 arc closepath fill
               vpt 0 360 arc closepath } bind def
/C11 { BL [] 0 setdash 2 copy moveto
       2 copy  vpt 0 180 arc closepath fill
       2 copy moveto
       2 copy  vpt 270 360 arc closepath fill
               vpt 0 360 arc closepath } bind def
/C12 { BL [] 0 setdash 2 copy moveto
       2 copy  vpt 180 360 arc closepath fill
               vpt 0 360 arc closepath } bind def
/C13 { BL [] 0 setdash  2 copy moveto
       2 copy  vpt 0 90 arc closepath fill
       2 copy moveto
       2 copy  vpt 180 360 arc closepath fill
               vpt 0 360 arc closepath } bind def
/C14 { BL [] 0 setdash 2 copy moveto
       2 copy  vpt 90 360 arc closepath fill
               vpt 0 360 arc } bind def
/C15 { BL [] 0 setdash 2 copy vpt 0 360 arc closepath fill
               vpt 0 360 arc closepath } bind def
/Rec   { newpath 4 2 roll moveto 1 index 0 rlineto 0 exch rlineto
       neg 0 rlineto closepath } bind def
/Square { dup Rec } bind def
/Bsquare { vpt sub exch vpt sub exch vpt2 Square } bind def
/S0 { BL [] 0 setdash 2 copy moveto 0 vpt rlineto BL Bsquare } bind def
/S1 { BL [] 0 setdash 2 copy vpt Square fill Bsquare } bind def
/S2 { BL [] 0 setdash 2 copy exch vpt sub exch vpt Square fill Bsquare } bind def
/S3 { BL [] 0 setdash 2 copy exch vpt sub exch vpt2 vpt Rec fill Bsquare } bind def
/S4 { BL [] 0 setdash 2 copy exch vpt sub exch vpt sub vpt Square fill Bsquare } bind def
/S5 { BL [] 0 setdash 2 copy 2 copy vpt Square fill
       exch vpt sub exch vpt sub vpt Square fill Bsquare } bind def
/S6 { BL [] 0 setdash 2 copy exch vpt sub exch vpt sub vpt vpt2 Rec fill Bsquare } bind def
/S7 { BL [] 0 setdash 2 copy exch vpt sub exch vpt sub vpt vpt2 Rec fill
       2 copy vpt Square fill
       Bsquare } bind def
/S8 { BL [] 0 setdash 2 copy vpt sub vpt Square fill Bsquare } bind def
/S9 { BL [] 0 setdash 2 copy vpt sub vpt vpt2 Rec fill Bsquare } bind def
/S10 { BL [] 0 setdash 2 copy vpt sub vpt Square fill 2 copy exch vpt sub exch vpt Square fill
       Bsquare } bind def
/S11 { BL [] 0 setdash 2 copy vpt sub vpt Square fill 2 copy exch vpt sub exch vpt2 vpt Rec fill
       Bsquare } bind def
/S12 { BL [] 0 setdash 2 copy exch vpt sub exch vpt sub vpt2 vpt Rec fill Bsquare } bind def
/S13 { BL [] 0 setdash 2 copy exch vpt sub exch vpt sub vpt2 vpt Rec fill
       2 copy vpt Square fill Bsquare } bind def
/S14 { BL [] 0 setdash 2 copy exch vpt sub exch vpt sub vpt2 vpt Rec fill
       2 copy exch vpt sub exch vpt Square fill Bsquare } bind def
/S15 { BL [] 0 setdash 2 copy Bsquare fill Bsquare } bind def
/D0 { gsave translate 45 rotate 0 0 S0 stroke grestore } bind def
/D1 { gsave translate 45 rotate 0 0 S1 stroke grestore } bind def
/D2 { gsave translate 45 rotate 0 0 S2 stroke grestore } bind def
/D3 { gsave translate 45 rotate 0 0 S3 stroke grestore } bind def
/D4 { gsave translate 45 rotate 0 0 S4 stroke grestore } bind def
/D5 { gsave translate 45 rotate 0 0 S5 stroke grestore } bind def
/D6 { gsave translate 45 rotate 0 0 S6 stroke grestore } bind def
/D7 { gsave translate 45 rotate 0 0 S7 stroke grestore } bind def
/D8 { gsave translate 45 rotate 0 0 S8 stroke grestore } bind def
/D9 { gsave translate 45 rotate 0 0 S9 stroke grestore } bind def
/D10 { gsave translate 45 rotate 0 0 S10 stroke grestore } bind def
/D11 { gsave translate 45 rotate 0 0 S11 stroke grestore } bind def
/D12 { gsave translate 45 rotate 0 0 S12 stroke grestore } bind def
/D13 { gsave translate 45 rotate 0 0 S13 stroke grestore } bind def
/D14 { gsave translate 45 rotate 0 0 S14 stroke grestore } bind def
/D15 { gsave translate 45 rotate 0 0 S15 stroke grestore } bind def
/DiaE { stroke [] 0 setdash vpt add M
  hpt neg vpt neg V hpt vpt neg V
  hpt vpt V hpt neg vpt V closepath stroke } def
/BoxE { stroke [] 0 setdash exch hpt sub exch vpt add M
  0 vpt2 neg V hpt2 0 V 0 vpt2 V
  hpt2 neg 0 V closepath stroke } def
/TriUE { stroke [] 0 setdash vpt 1.12 mul add M
  hpt neg vpt -1.62 mul V
  hpt 2 mul 0 V
  hpt neg vpt 1.62 mul V closepath stroke } def
/TriDE { stroke [] 0 setdash vpt 1.12 mul sub M
  hpt neg vpt 1.62 mul V
  hpt 2 mul 0 V
  hpt neg vpt -1.62 mul V closepath stroke } def
/PentE { stroke [] 0 setdash gsave
  translate 0 hpt M 4 {72 rotate 0 hpt L} repeat
  closepath stroke grestore } def
/CircE { stroke [] 0 setdash 
  hpt 0 360 arc stroke } def
/Opaque { gsave closepath 1 setgray fill grestore 0 setgray closepath } def
/DiaW { stroke [] 0 setdash vpt add M
  hpt neg vpt neg V hpt vpt neg V
  hpt vpt V hpt neg vpt V Opaque stroke } def
/BoxW { stroke [] 0 setdash exch hpt sub exch vpt add M
  0 vpt2 neg V hpt2 0 V 0 vpt2 V
  hpt2 neg 0 V Opaque stroke } def
/TriUW { stroke [] 0 setdash vpt 1.12 mul add M
  hpt neg vpt -1.62 mul V
  hpt 2 mul 0 V
  hpt neg vpt 1.62 mul V Opaque stroke } def
/TriDW { stroke [] 0 setdash vpt 1.12 mul sub M
  hpt neg vpt 1.62 mul V
  hpt 2 mul 0 V
  hpt neg vpt -1.62 mul V Opaque stroke } def
/PentW { stroke [] 0 setdash gsave
  translate 0 hpt M 4 {72 rotate 0 hpt L} repeat
  Opaque stroke grestore } def
/CircW { stroke [] 0 setdash 
  hpt 0 360 arc Opaque stroke } def
/BoxFill { gsave Rec 1 setgray fill grestore } def
end
%%EndProlog
}}%
\begin{picture}(3600,2160)(0,0)%
{\GNUPLOTspecial{"
gnudict begin
gsave
0 0 translate
0.100 0.100 scale
0 setgray
newpath
1.000 UL
LTb
400 300 M
63 0 V
2987 0 R
-63 0 V
400 520 M
63 0 V
2987 0 R
-63 0 V
400 740 M
63 0 V
2987 0 R
-63 0 V
400 960 M
63 0 V
2987 0 R
-63 0 V
400 1180 M
63 0 V
2987 0 R
-63 0 V
400 1400 M
63 0 V
2987 0 R
-63 0 V
400 1620 M
63 0 V
2987 0 R
-63 0 V
400 1840 M
63 0 V
2987 0 R
-63 0 V
400 2060 M
63 0 V
2987 0 R
-63 0 V
705 300 M
0 63 V
0 1697 R
0 -63 V
1315 300 M
0 63 V
0 1697 R
0 -63 V
1925 300 M
0 63 V
0 1697 R
0 -63 V
2535 300 M
0 63 V
0 1697 R
0 -63 V
3145 300 M
0 63 V
0 1697 R
0 -63 V
1.000 UL
LTb
400 300 M
3050 0 V
0 1760 V
-3050 0 V
400 300 L
1.000 UL
LT0
3087 1947 M
263 0 V
400 300 M
31 44 V
31 127 V
30 199 V
31 249 V
31 275 V
31 274 V
31 245 V
30 192 V
31 119 V
31 36 V
31 -53 V
31 -135 V
31 -205 V
30 -253 V
31 -276 V
893 866 L
924 625 L
955 440 L
985 328 L
31 -26 V
31 61 V
31 143 V
31 210 V
30 257 V
31 277 V
31 270 V
31 236 V
31 179 V
30 103 V
31 18 V
31 -70 V
31 -150 V
31 -216 V
30 -260 V
31 -278 V
31 -268 V
31 -231 V
31 -172 V
31 -95 V
30 -9 V
31 78 V
31 158 V
31 221 V
31 263 V
30 278 V
31 266 V
31 226 V
31 165 V
31 87 V
30 0 V
31 -87 V
31 -165 V
31 -226 V
31 -266 V
30 -278 V
31 -263 V
31 -221 V
31 -158 V
31 -78 V
30 9 V
31 95 V
31 172 V
31 231 V
31 268 V
31 278 V
30 260 V
31 216 V
31 150 V
31 70 V
31 -18 V
30 -103 V
31 -179 V
31 -236 V
31 -270 V
31 -277 V
30 -257 V
31 -210 V
31 -143 V
31 -61 V
31 26 V
30 112 V
31 185 V
31 241 V
31 272 V
31 276 V
30 253 V
31 205 V
31 135 V
31 53 V
31 -36 V
31 -119 V
30 -192 V
31 -245 V
31 -274 V
31 -275 V
31 -249 V
30 -199 V
31 -127 V
31 -44 V
stroke
grestore
end
showpage
}}%
\put(3037,1947){\makebox(0,0)[r]{N=2}}%
\put(100,1180){%
\special{ps: gsave currentpoint currentpoint translate
270 rotate neg exch neg exch translate}%
\makebox(0,0)[b]{\shortstack{I}}%
\special{ps: currentpoint grestore moveto}%
}%
\put(3145,200){\makebox(0,0){2}}%
\put(2535,200){\makebox(0,0){1}}%
\put(1925,200){\makebox(0,0){0}}%
\put(1315,200){\makebox(0,0){-1}}%
\put(705,200){\makebox(0,0){-2}}%
\end{picture}%
\endgroup
\endinput

% GNUPLOT: LaTeX picture with Postscript
\begingroup%
  \makeatletter%
  \newcommand{\GNUPLOTspecial}{%
    \@sanitize\catcode`\%=14\relax\special}%
  \setlength{\unitlength}{0.1bp}%
{\GNUPLOTspecial{!
%!PS-Adobe-2.0 EPSF-2.0
%%Title: inter8.tex
%%Creator: gnuplot 3.7 patchlevel 0.2
%%CreationDate: Fri Feb 11 10:17:44 2000
%%DocumentFonts: 
%%BoundingBox: 0 0 360 216
%%Orientation: Landscape
%%EndComments
/gnudict 256 dict def
gnudict begin
/Color false def
/Solid false def
/gnulinewidth 5.000 def
/userlinewidth gnulinewidth def
/vshift -33 def
/dl {10 mul} def
/hpt_ 31.5 def
/vpt_ 31.5 def
/hpt hpt_ def
/vpt vpt_ def
/M {moveto} bind def
/L {lineto} bind def
/R {rmoveto} bind def
/V {rlineto} bind def
/vpt2 vpt 2 mul def
/hpt2 hpt 2 mul def
/Lshow { currentpoint stroke M
  0 vshift R show } def
/Rshow { currentpoint stroke M
  dup stringwidth pop neg vshift R show } def
/Cshow { currentpoint stroke M
  dup stringwidth pop -2 div vshift R show } def
/UP { dup vpt_ mul /vpt exch def hpt_ mul /hpt exch def
  /hpt2 hpt 2 mul def /vpt2 vpt 2 mul def } def
/DL { Color {setrgbcolor Solid {pop []} if 0 setdash }
 {pop pop pop Solid {pop []} if 0 setdash} ifelse } def
/BL { stroke userlinewidth 2 mul setlinewidth } def
/AL { stroke userlinewidth 2 div setlinewidth } def
/UL { dup gnulinewidth mul /userlinewidth exch def
      10 mul /udl exch def } def
/PL { stroke userlinewidth setlinewidth } def
/LTb { BL [] 0 0 0 DL } def
/LTa { AL [1 udl mul 2 udl mul] 0 setdash 0 0 0 setrgbcolor } def
/LT0 { PL [] 1 0 0 DL } def
/LT1 { PL [4 dl 2 dl] 0 1 0 DL } def
/LT2 { PL [2 dl 3 dl] 0 0 1 DL } def
/LT3 { PL [1 dl 1.5 dl] 1 0 1 DL } def
/LT4 { PL [5 dl 2 dl 1 dl 2 dl] 0 1 1 DL } def
/LT5 { PL [4 dl 3 dl 1 dl 3 dl] 1 1 0 DL } def
/LT6 { PL [2 dl 2 dl 2 dl 4 dl] 0 0 0 DL } def
/LT7 { PL [2 dl 2 dl 2 dl 2 dl 2 dl 4 dl] 1 0.3 0 DL } def
/LT8 { PL [2 dl 2 dl 2 dl 2 dl 2 dl 2 dl 2 dl 4 dl] 0.5 0.5 0.5 DL } def
/Pnt { stroke [] 0 setdash
   gsave 1 setlinecap M 0 0 V stroke grestore } def
/Dia { stroke [] 0 setdash 2 copy vpt add M
  hpt neg vpt neg V hpt vpt neg V
  hpt vpt V hpt neg vpt V closepath stroke
  Pnt } def
/Pls { stroke [] 0 setdash vpt sub M 0 vpt2 V
  currentpoint stroke M
  hpt neg vpt neg R hpt2 0 V stroke
  } def
/Box { stroke [] 0 setdash 2 copy exch hpt sub exch vpt add M
  0 vpt2 neg V hpt2 0 V 0 vpt2 V
  hpt2 neg 0 V closepath stroke
  Pnt } def
/Crs { stroke [] 0 setdash exch hpt sub exch vpt add M
  hpt2 vpt2 neg V currentpoint stroke M
  hpt2 neg 0 R hpt2 vpt2 V stroke } def
/TriU { stroke [] 0 setdash 2 copy vpt 1.12 mul add M
  hpt neg vpt -1.62 mul V
  hpt 2 mul 0 V
  hpt neg vpt 1.62 mul V closepath stroke
  Pnt  } def
/Star { 2 copy Pls Crs } def
/BoxF { stroke [] 0 setdash exch hpt sub exch vpt add M
  0 vpt2 neg V  hpt2 0 V  0 vpt2 V
  hpt2 neg 0 V  closepath fill } def
/TriUF { stroke [] 0 setdash vpt 1.12 mul add M
  hpt neg vpt -1.62 mul V
  hpt 2 mul 0 V
  hpt neg vpt 1.62 mul V closepath fill } def
/TriD { stroke [] 0 setdash 2 copy vpt 1.12 mul sub M
  hpt neg vpt 1.62 mul V
  hpt 2 mul 0 V
  hpt neg vpt -1.62 mul V closepath stroke
  Pnt  } def
/TriDF { stroke [] 0 setdash vpt 1.12 mul sub M
  hpt neg vpt 1.62 mul V
  hpt 2 mul 0 V
  hpt neg vpt -1.62 mul V closepath fill} def
/DiaF { stroke [] 0 setdash vpt add M
  hpt neg vpt neg V hpt vpt neg V
  hpt vpt V hpt neg vpt V closepath fill } def
/Pent { stroke [] 0 setdash 2 copy gsave
  translate 0 hpt M 4 {72 rotate 0 hpt L} repeat
  closepath stroke grestore Pnt } def
/PentF { stroke [] 0 setdash gsave
  translate 0 hpt M 4 {72 rotate 0 hpt L} repeat
  closepath fill grestore } def
/Circle { stroke [] 0 setdash 2 copy
  hpt 0 360 arc stroke Pnt } def
/CircleF { stroke [] 0 setdash hpt 0 360 arc fill } def
/C0 { BL [] 0 setdash 2 copy moveto vpt 90 450  arc } bind def
/C1 { BL [] 0 setdash 2 copy        moveto
       2 copy  vpt 0 90 arc closepath fill
               vpt 0 360 arc closepath } bind def
/C2 { BL [] 0 setdash 2 copy moveto
       2 copy  vpt 90 180 arc closepath fill
               vpt 0 360 arc closepath } bind def
/C3 { BL [] 0 setdash 2 copy moveto
       2 copy  vpt 0 180 arc closepath fill
               vpt 0 360 arc closepath } bind def
/C4 { BL [] 0 setdash 2 copy moveto
       2 copy  vpt 180 270 arc closepath fill
               vpt 0 360 arc closepath } bind def
/C5 { BL [] 0 setdash 2 copy moveto
       2 copy  vpt 0 90 arc
       2 copy moveto
       2 copy  vpt 180 270 arc closepath fill
               vpt 0 360 arc } bind def
/C6 { BL [] 0 setdash 2 copy moveto
      2 copy  vpt 90 270 arc closepath fill
              vpt 0 360 arc closepath } bind def
/C7 { BL [] 0 setdash 2 copy moveto
      2 copy  vpt 0 270 arc closepath fill
              vpt 0 360 arc closepath } bind def
/C8 { BL [] 0 setdash 2 copy moveto
      2 copy vpt 270 360 arc closepath fill
              vpt 0 360 arc closepath } bind def
/C9 { BL [] 0 setdash 2 copy moveto
      2 copy  vpt 270 450 arc closepath fill
              vpt 0 360 arc closepath } bind def
/C10 { BL [] 0 setdash 2 copy 2 copy moveto vpt 270 360 arc closepath fill
       2 copy moveto
       2 copy vpt 90 180 arc closepath fill
               vpt 0 360 arc closepath } bind def
/C11 { BL [] 0 setdash 2 copy moveto
       2 copy  vpt 0 180 arc closepath fill
       2 copy moveto
       2 copy  vpt 270 360 arc closepath fill
               vpt 0 360 arc closepath } bind def
/C12 { BL [] 0 setdash 2 copy moveto
       2 copy  vpt 180 360 arc closepath fill
               vpt 0 360 arc closepath } bind def
/C13 { BL [] 0 setdash  2 copy moveto
       2 copy  vpt 0 90 arc closepath fill
       2 copy moveto
       2 copy  vpt 180 360 arc closepath fill
               vpt 0 360 arc closepath } bind def
/C14 { BL [] 0 setdash 2 copy moveto
       2 copy  vpt 90 360 arc closepath fill
               vpt 0 360 arc } bind def
/C15 { BL [] 0 setdash 2 copy vpt 0 360 arc closepath fill
               vpt 0 360 arc closepath } bind def
/Rec   { newpath 4 2 roll moveto 1 index 0 rlineto 0 exch rlineto
       neg 0 rlineto closepath } bind def
/Square { dup Rec } bind def
/Bsquare { vpt sub exch vpt sub exch vpt2 Square } bind def
/S0 { BL [] 0 setdash 2 copy moveto 0 vpt rlineto BL Bsquare } bind def
/S1 { BL [] 0 setdash 2 copy vpt Square fill Bsquare } bind def
/S2 { BL [] 0 setdash 2 copy exch vpt sub exch vpt Square fill Bsquare } bind def
/S3 { BL [] 0 setdash 2 copy exch vpt sub exch vpt2 vpt Rec fill Bsquare } bind def
/S4 { BL [] 0 setdash 2 copy exch vpt sub exch vpt sub vpt Square fill Bsquare } bind def
/S5 { BL [] 0 setdash 2 copy 2 copy vpt Square fill
       exch vpt sub exch vpt sub vpt Square fill Bsquare } bind def
/S6 { BL [] 0 setdash 2 copy exch vpt sub exch vpt sub vpt vpt2 Rec fill Bsquare } bind def
/S7 { BL [] 0 setdash 2 copy exch vpt sub exch vpt sub vpt vpt2 Rec fill
       2 copy vpt Square fill
       Bsquare } bind def
/S8 { BL [] 0 setdash 2 copy vpt sub vpt Square fill Bsquare } bind def
/S9 { BL [] 0 setdash 2 copy vpt sub vpt vpt2 Rec fill Bsquare } bind def
/S10 { BL [] 0 setdash 2 copy vpt sub vpt Square fill 2 copy exch vpt sub exch vpt Square fill
       Bsquare } bind def
/S11 { BL [] 0 setdash 2 copy vpt sub vpt Square fill 2 copy exch vpt sub exch vpt2 vpt Rec fill
       Bsquare } bind def
/S12 { BL [] 0 setdash 2 copy exch vpt sub exch vpt sub vpt2 vpt Rec fill Bsquare } bind def
/S13 { BL [] 0 setdash 2 copy exch vpt sub exch vpt sub vpt2 vpt Rec fill
       2 copy vpt Square fill Bsquare } bind def
/S14 { BL [] 0 setdash 2 copy exch vpt sub exch vpt sub vpt2 vpt Rec fill
       2 copy exch vpt sub exch vpt Square fill Bsquare } bind def
/S15 { BL [] 0 setdash 2 copy Bsquare fill Bsquare } bind def
/D0 { gsave translate 45 rotate 0 0 S0 stroke grestore } bind def
/D1 { gsave translate 45 rotate 0 0 S1 stroke grestore } bind def
/D2 { gsave translate 45 rotate 0 0 S2 stroke grestore } bind def
/D3 { gsave translate 45 rotate 0 0 S3 stroke grestore } bind def
/D4 { gsave translate 45 rotate 0 0 S4 stroke grestore } bind def
/D5 { gsave translate 45 rotate 0 0 S5 stroke grestore } bind def
/D6 { gsave translate 45 rotate 0 0 S6 stroke grestore } bind def
/D7 { gsave translate 45 rotate 0 0 S7 stroke grestore } bind def
/D8 { gsave translate 45 rotate 0 0 S8 stroke grestore } bind def
/D9 { gsave translate 45 rotate 0 0 S9 stroke grestore } bind def
/D10 { gsave translate 45 rotate 0 0 S10 stroke grestore } bind def
/D11 { gsave translate 45 rotate 0 0 S11 stroke grestore } bind def
/D12 { gsave translate 45 rotate 0 0 S12 stroke grestore } bind def
/D13 { gsave translate 45 rotate 0 0 S13 stroke grestore } bind def
/D14 { gsave translate 45 rotate 0 0 S14 stroke grestore } bind def
/D15 { gsave translate 45 rotate 0 0 S15 stroke grestore } bind def
/DiaE { stroke [] 0 setdash vpt add M
  hpt neg vpt neg V hpt vpt neg V
  hpt vpt V hpt neg vpt V closepath stroke } def
/BoxE { stroke [] 0 setdash exch hpt sub exch vpt add M
  0 vpt2 neg V hpt2 0 V 0 vpt2 V
  hpt2 neg 0 V closepath stroke } def
/TriUE { stroke [] 0 setdash vpt 1.12 mul add M
  hpt neg vpt -1.62 mul V
  hpt 2 mul 0 V
  hpt neg vpt 1.62 mul V closepath stroke } def
/TriDE { stroke [] 0 setdash vpt 1.12 mul sub M
  hpt neg vpt 1.62 mul V
  hpt 2 mul 0 V
  hpt neg vpt -1.62 mul V closepath stroke } def
/PentE { stroke [] 0 setdash gsave
  translate 0 hpt M 4 {72 rotate 0 hpt L} repeat
  closepath stroke grestore } def
/CircE { stroke [] 0 setdash 
  hpt 0 360 arc stroke } def
/Opaque { gsave closepath 1 setgray fill grestore 0 setgray closepath } def
/DiaW { stroke [] 0 setdash vpt add M
  hpt neg vpt neg V hpt vpt neg V
  hpt vpt V hpt neg vpt V Opaque stroke } def
/BoxW { stroke [] 0 setdash exch hpt sub exch vpt add M
  0 vpt2 neg V hpt2 0 V 0 vpt2 V
  hpt2 neg 0 V Opaque stroke } def
/TriUW { stroke [] 0 setdash vpt 1.12 mul add M
  hpt neg vpt -1.62 mul V
  hpt 2 mul 0 V
  hpt neg vpt 1.62 mul V Opaque stroke } def
/TriDW { stroke [] 0 setdash vpt 1.12 mul sub M
  hpt neg vpt 1.62 mul V
  hpt 2 mul 0 V
  hpt neg vpt -1.62 mul V Opaque stroke } def
/PentW { stroke [] 0 setdash gsave
  translate 0 hpt M 4 {72 rotate 0 hpt L} repeat
  Opaque stroke grestore } def
/CircW { stroke [] 0 setdash 
  hpt 0 360 arc Opaque stroke } def
/BoxFill { gsave Rec 1 setgray fill grestore } def
end
%%EndProlog
}}%
\begin{picture}(3600,2160)(0,0)%
{\GNUPLOTspecial{"
gnudict begin
gsave
0 0 translate
0.100 0.100 scale
0 setgray
newpath
1.000 UL
LTb
350 300 M
63 0 V
3037 0 R
-63 0 V
350 551 M
63 0 V
3037 0 R
-63 0 V
350 803 M
63 0 V
3037 0 R
-63 0 V
350 1054 M
63 0 V
3037 0 R
-63 0 V
350 1306 M
63 0 V
3037 0 R
-63 0 V
350 1557 M
63 0 V
3037 0 R
-63 0 V
350 1809 M
63 0 V
3037 0 R
-63 0 V
350 2060 M
63 0 V
3037 0 R
-63 0 V
660 300 M
0 63 V
0 1697 R
0 -63 V
1280 300 M
0 63 V
0 1697 R
0 -63 V
1900 300 M
0 63 V
0 1697 R
0 -63 V
2520 300 M
0 63 V
0 1697 R
0 -63 V
3140 300 M
0 63 V
0 1697 R
0 -63 V
1.000 UL
LTb
350 300 M
3100 0 V
0 1760 V
-3100 0 V
350 300 L
1.000 UL
LT0
3087 1947 M
263 0 V
350 300 M
31 24 V
32 -15 V
31 3 V
31 22 V
32 -34 V
31 71 V
31 -37 V
32 93 V
31 900 V
31 574 V
31 -792 V
726 349 L
31 8 V
31 -4 V
32 -51 V
31 35 V
31 -32 V
32 11 V
31 2 V
31 -16 V
32 24 V
31 -23 V
31 18 V
32 6 V
31 -22 V
31 77 V
31 -70 V
32 233 V
31 988 V
31 301 V
32 -938 V
31 -586 V
31 65 V
32 -41 V
31 -26 V
31 27 V
32 -34 V
31 21 V
31 -10 V
32 -6 V
31 19 V
31 -25 V
31 29 V
32 -11 V
31 -1 V
31 67 V
32 -84 V
31 404 V
31 1003 V
32 0 V
1947 704 L
31 -404 V
32 84 V
31 -67 V
31 1 V
32 11 V
31 -29 V
31 25 V
31 -19 V
32 6 V
31 10 V
31 -21 V
32 34 V
31 -27 V
31 26 V
32 41 V
31 -65 V
31 586 V
32 938 V
31 -301 V
31 -988 V
32 -233 V
31 70 V
31 -77 V
31 22 V
32 -6 V
31 -18 V
31 23 V
32 -24 V
31 16 V
31 -2 V
32 -11 V
31 32 V
31 -35 V
32 51 V
31 4 V
31 -8 V
32 760 V
31 792 V
31 -574 V
31 -900 V
32 -93 V
31 37 V
31 -71 V
32 34 V
31 -22 V
31 -3 V
32 15 V
31 -24 V
stroke
grestore
end
showpage
}}%
\put(3037,1947){\makebox(0,0)[r]{N=8}}%
\put(1900,50){\makebox(0,0){$asin\theta/\lambda$}}%
\put(100,1180){%
\special{ps: gsave currentpoint currentpoint translate
270 rotate neg exch neg exch translate}%
\makebox(0,0)[b]{\shortstack{I}}%
\special{ps: currentpoint grestore moveto}%
}%
\put(3140,200){\makebox(0,0){2}}%
\put(2520,200){\makebox(0,0){1}}%
\put(1900,200){\makebox(0,0){0}}%
\put(1280,200){\makebox(0,0){-1}}%
\put(660,200){\makebox(0,0){-2}}%
\end{picture}%
\endgroup
\endinput

\caption{Eksempel p\aa\ interferensm\o nster med $N=2$ og $N=8$ lyskilder.\label{fig:interferensmoenster}} 
\end{figure}



\subsection{Diffraksjon}

Tenk dere at en sender inn en b\o lge mot en spalte\aa pning
som vist p\aa\ Figur \ref{fig:diffmonster}. 
Dersom spalte\aa pningen $d$ er p\aa\ st\o rrelse
med b\o lgelengden $\lambda$\footnote{Vi nevnte  dette ogs\aa\ i anledning 
de Broglies hypotese.}, kan b\o lgen som brer seg fra spalte\aa pningen
danne et interferens m\o nster. Vi sier at, i motsetning til interferens
hvor flere b\o lger summeres opp i et omr\aa de $P$, s\aa\ svarer diffraksjon til 
at det er  selve b\o lgen som vekselvirker med seg selv og danner et interferensm\o nster
ved skjermen. 
\begin{figure}[h]
   \setlength{\unitlength}{1mm}
   \begin{picture}(100,60)
   \put(25,0){\epsfxsize=10cm \epsfbox{fig2.eps}}
   \end{picture}
\caption{Skjematisk oppsett for lysstr\aa le som sendes mot en spalte\aa pning. Vinkelen
$\theta$ indikerer hvor f\o rste diffraksjonsminimum opptrer.\label{fig:diffmonster}} 
\end{figure}

Her skal vi all hovedsak kun anskueliggj\o re utledningen av
uttrykket for intensiteten
ved en skjerm langt borte fra spalte\aa pningen. 
Vi skal betrakte superposisjon
av b\o lger som reiser ut fra spalten og treffer skjermen ved 
et omr\aa de $P$ 
som har en avstand $R$ fra sentrum av spalten. Huygens prinsipp forteller
at hvert element $dy$ i spalte\aa pningen virker som en lyskilde, jfr.~diskusjonen
fra foreg\aa ende avsnitt om interferens. 
Igjen antar vi at avstanden er s\aa\ stor at vi kan anta tiln\ae rma parallelle
linjer for alle b\o lger som treffer omr\aa det $P$. 

Vi definerer
\be
   R=\frac{1}{2}(N-1)dsin\theta+dy,
\ee
dvs.~vi har delt opp spalten i $N$ deler $dy$. 
En b\o lge som blir utsendt i et punkt $y$ reiser derfor en 
avstand $R-ysin\theta$ for \aa\ komme til omr\aa det $P$. 
Bruker vi s\aa\ definisjonen av b\o lgefunksjonen
fra et punkt $dy$ (som svarer til $r_i$ i foreg\aa ende avsnitt)
\be
  d\psi = Ae^{i(k(R-ysin\theta)-\omega t)}\frac{dy}{d},
\ee
og integrerer vi opp bidragene fra hele spalten f\aa r vi
\be
\psi=\int_{-ad/2}^{d/2}d\psi=\int_{-d/2}^{d/2}Ae^{i(k(R-ysin\theta)-\omega t)}\frac{dy}{d},
\ee
som gir
\be
    \psi=Ae^{i(kR-\omega t)}\frac{sin(\frac{\pi}{\lambda}dsin\theta)}{\frac{\pi}{\lambda}dsin\theta},
\ee
som resulterer i en intensitet gitt ved 
\be
I=I_0\frac{sin^2(\frac{\pi}{\lambda}dsin\theta)}{(\frac{\pi}{\lambda}dsin\theta)^2},
\ee
eller
\be
I=I_0\frac{sin^2(u)}{u^2},
\ee
med
\be
    u=\frac{\pi}{\lambda}dsin\theta.
\ee
Dette forteller oss 
vi har maks intensitet n\aa r $u=0$ og min intensitet, dvs $I=0$  n\aa r
\be
    u=n\pi \hspace{0.3cm} n=\pm 1, \pm 2, \dots ,
\ee
eller
\be 
    dsin\theta_n=n\lambda.
\ee  
\begin{figure}[h]
   \setlength{\unitlength}{1mm}
   \begin{picture}(100,60)
   \put(25,0){\epsfxsize=10cm \epsfbox{fig4.eps}}
   \end{picture}
\caption{Skjematisk oppsett for lysstr\aa le som sendes mot to spalte\aa pninger
med \aa pning $d$ og avstand mellom \aa pningene gitt ved $a$.} 
\end{figure}
Dersom vi har to spalter med avstand $a$ og spalte\aa pning $d$, se Figur 2.6,
finner vi at intensiteten er gitt ved
\be
I=I_0cos^2(\frac{\pi}{\lambda}asin\theta)\frac{sin^2(\frac{\pi}{\lambda}dsin\theta)}{(\frac{\pi}{\lambda}dsin\theta)^2},
\label{eq:diffint}
\ee
som er relevant n\aa r en studerer Bragg spredning.
I dette tilfelle har vi konstruktiv interferens b\aa de n\aa r 
\[ 
    dsin\theta_n=\frac{(2n+1)\lambda}{2}.
\]
og n\aa r
\[ 
    asin\theta_n=n\lambda.
\]
Eksempler for diffraksjonsm\o nster med henholdsvis en og to spalte\aa pninger
er vist i Figur \ref{fig:diff}. 
Legg merke til at for tilfellet med
to spalter har vi satt avstanden mellom spalte\aa pningene $a$ lik st\o rrelsen
p\aa\ spalte\aa pnigen $d$.  
\begin{figure}
% GNUPLOT: LaTeX picture with Postscript
\begingroup%
  \makeatletter%
  \newcommand{\GNUPLOTspecial}{%
    \@sanitize\catcode`\%=14\relax\special}%
  \setlength{\unitlength}{0.1bp}%
{\GNUPLOTspecial{!
%!PS-Adobe-2.0 EPSF-2.0
%%Title: diff1.tex
%%Creator: gnuplot 3.7 patchlevel 0.2
%%CreationDate: Fri Feb 11 10:29:02 2000
%%DocumentFonts: 
%%BoundingBox: 0 0 360 216
%%Orientation: Landscape
%%EndComments
/gnudict 256 dict def
gnudict begin
/Color false def
/Solid false def
/gnulinewidth 5.000 def
/userlinewidth gnulinewidth def
/vshift -33 def
/dl {10 mul} def
/hpt_ 31.5 def
/vpt_ 31.5 def
/hpt hpt_ def
/vpt vpt_ def
/M {moveto} bind def
/L {lineto} bind def
/R {rmoveto} bind def
/V {rlineto} bind def
/vpt2 vpt 2 mul def
/hpt2 hpt 2 mul def
/Lshow { currentpoint stroke M
  0 vshift R show } def
/Rshow { currentpoint stroke M
  dup stringwidth pop neg vshift R show } def
/Cshow { currentpoint stroke M
  dup stringwidth pop -2 div vshift R show } def
/UP { dup vpt_ mul /vpt exch def hpt_ mul /hpt exch def
  /hpt2 hpt 2 mul def /vpt2 vpt 2 mul def } def
/DL { Color {setrgbcolor Solid {pop []} if 0 setdash }
 {pop pop pop Solid {pop []} if 0 setdash} ifelse } def
/BL { stroke userlinewidth 2 mul setlinewidth } def
/AL { stroke userlinewidth 2 div setlinewidth } def
/UL { dup gnulinewidth mul /userlinewidth exch def
      10 mul /udl exch def } def
/PL { stroke userlinewidth setlinewidth } def
/LTb { BL [] 0 0 0 DL } def
/LTa { AL [1 udl mul 2 udl mul] 0 setdash 0 0 0 setrgbcolor } def
/LT0 { PL [] 1 0 0 DL } def
/LT1 { PL [4 dl 2 dl] 0 1 0 DL } def
/LT2 { PL [2 dl 3 dl] 0 0 1 DL } def
/LT3 { PL [1 dl 1.5 dl] 1 0 1 DL } def
/LT4 { PL [5 dl 2 dl 1 dl 2 dl] 0 1 1 DL } def
/LT5 { PL [4 dl 3 dl 1 dl 3 dl] 1 1 0 DL } def
/LT6 { PL [2 dl 2 dl 2 dl 4 dl] 0 0 0 DL } def
/LT7 { PL [2 dl 2 dl 2 dl 2 dl 2 dl 4 dl] 1 0.3 0 DL } def
/LT8 { PL [2 dl 2 dl 2 dl 2 dl 2 dl 2 dl 2 dl 4 dl] 0.5 0.5 0.5 DL } def
/Pnt { stroke [] 0 setdash
   gsave 1 setlinecap M 0 0 V stroke grestore } def
/Dia { stroke [] 0 setdash 2 copy vpt add M
  hpt neg vpt neg V hpt vpt neg V
  hpt vpt V hpt neg vpt V closepath stroke
  Pnt } def
/Pls { stroke [] 0 setdash vpt sub M 0 vpt2 V
  currentpoint stroke M
  hpt neg vpt neg R hpt2 0 V stroke
  } def
/Box { stroke [] 0 setdash 2 copy exch hpt sub exch vpt add M
  0 vpt2 neg V hpt2 0 V 0 vpt2 V
  hpt2 neg 0 V closepath stroke
  Pnt } def
/Crs { stroke [] 0 setdash exch hpt sub exch vpt add M
  hpt2 vpt2 neg V currentpoint stroke M
  hpt2 neg 0 R hpt2 vpt2 V stroke } def
/TriU { stroke [] 0 setdash 2 copy vpt 1.12 mul add M
  hpt neg vpt -1.62 mul V
  hpt 2 mul 0 V
  hpt neg vpt 1.62 mul V closepath stroke
  Pnt  } def
/Star { 2 copy Pls Crs } def
/BoxF { stroke [] 0 setdash exch hpt sub exch vpt add M
  0 vpt2 neg V  hpt2 0 V  0 vpt2 V
  hpt2 neg 0 V  closepath fill } def
/TriUF { stroke [] 0 setdash vpt 1.12 mul add M
  hpt neg vpt -1.62 mul V
  hpt 2 mul 0 V
  hpt neg vpt 1.62 mul V closepath fill } def
/TriD { stroke [] 0 setdash 2 copy vpt 1.12 mul sub M
  hpt neg vpt 1.62 mul V
  hpt 2 mul 0 V
  hpt neg vpt -1.62 mul V closepath stroke
  Pnt  } def
/TriDF { stroke [] 0 setdash vpt 1.12 mul sub M
  hpt neg vpt 1.62 mul V
  hpt 2 mul 0 V
  hpt neg vpt -1.62 mul V closepath fill} def
/DiaF { stroke [] 0 setdash vpt add M
  hpt neg vpt neg V hpt vpt neg V
  hpt vpt V hpt neg vpt V closepath fill } def
/Pent { stroke [] 0 setdash 2 copy gsave
  translate 0 hpt M 4 {72 rotate 0 hpt L} repeat
  closepath stroke grestore Pnt } def
/PentF { stroke [] 0 setdash gsave
  translate 0 hpt M 4 {72 rotate 0 hpt L} repeat
  closepath fill grestore } def
/Circle { stroke [] 0 setdash 2 copy
  hpt 0 360 arc stroke Pnt } def
/CircleF { stroke [] 0 setdash hpt 0 360 arc fill } def
/C0 { BL [] 0 setdash 2 copy moveto vpt 90 450  arc } bind def
/C1 { BL [] 0 setdash 2 copy        moveto
       2 copy  vpt 0 90 arc closepath fill
               vpt 0 360 arc closepath } bind def
/C2 { BL [] 0 setdash 2 copy moveto
       2 copy  vpt 90 180 arc closepath fill
               vpt 0 360 arc closepath } bind def
/C3 { BL [] 0 setdash 2 copy moveto
       2 copy  vpt 0 180 arc closepath fill
               vpt 0 360 arc closepath } bind def
/C4 { BL [] 0 setdash 2 copy moveto
       2 copy  vpt 180 270 arc closepath fill
               vpt 0 360 arc closepath } bind def
/C5 { BL [] 0 setdash 2 copy moveto
       2 copy  vpt 0 90 arc
       2 copy moveto
       2 copy  vpt 180 270 arc closepath fill
               vpt 0 360 arc } bind def
/C6 { BL [] 0 setdash 2 copy moveto
      2 copy  vpt 90 270 arc closepath fill
              vpt 0 360 arc closepath } bind def
/C7 { BL [] 0 setdash 2 copy moveto
      2 copy  vpt 0 270 arc closepath fill
              vpt 0 360 arc closepath } bind def
/C8 { BL [] 0 setdash 2 copy moveto
      2 copy vpt 270 360 arc closepath fill
              vpt 0 360 arc closepath } bind def
/C9 { BL [] 0 setdash 2 copy moveto
      2 copy  vpt 270 450 arc closepath fill
              vpt 0 360 arc closepath } bind def
/C10 { BL [] 0 setdash 2 copy 2 copy moveto vpt 270 360 arc closepath fill
       2 copy moveto
       2 copy vpt 90 180 arc closepath fill
               vpt 0 360 arc closepath } bind def
/C11 { BL [] 0 setdash 2 copy moveto
       2 copy  vpt 0 180 arc closepath fill
       2 copy moveto
       2 copy  vpt 270 360 arc closepath fill
               vpt 0 360 arc closepath } bind def
/C12 { BL [] 0 setdash 2 copy moveto
       2 copy  vpt 180 360 arc closepath fill
               vpt 0 360 arc closepath } bind def
/C13 { BL [] 0 setdash  2 copy moveto
       2 copy  vpt 0 90 arc closepath fill
       2 copy moveto
       2 copy  vpt 180 360 arc closepath fill
               vpt 0 360 arc closepath } bind def
/C14 { BL [] 0 setdash 2 copy moveto
       2 copy  vpt 90 360 arc closepath fill
               vpt 0 360 arc } bind def
/C15 { BL [] 0 setdash 2 copy vpt 0 360 arc closepath fill
               vpt 0 360 arc closepath } bind def
/Rec   { newpath 4 2 roll moveto 1 index 0 rlineto 0 exch rlineto
       neg 0 rlineto closepath } bind def
/Square { dup Rec } bind def
/Bsquare { vpt sub exch vpt sub exch vpt2 Square } bind def
/S0 { BL [] 0 setdash 2 copy moveto 0 vpt rlineto BL Bsquare } bind def
/S1 { BL [] 0 setdash 2 copy vpt Square fill Bsquare } bind def
/S2 { BL [] 0 setdash 2 copy exch vpt sub exch vpt Square fill Bsquare } bind def
/S3 { BL [] 0 setdash 2 copy exch vpt sub exch vpt2 vpt Rec fill Bsquare } bind def
/S4 { BL [] 0 setdash 2 copy exch vpt sub exch vpt sub vpt Square fill Bsquare } bind def
/S5 { BL [] 0 setdash 2 copy 2 copy vpt Square fill
       exch vpt sub exch vpt sub vpt Square fill Bsquare } bind def
/S6 { BL [] 0 setdash 2 copy exch vpt sub exch vpt sub vpt vpt2 Rec fill Bsquare } bind def
/S7 { BL [] 0 setdash 2 copy exch vpt sub exch vpt sub vpt vpt2 Rec fill
       2 copy vpt Square fill
       Bsquare } bind def
/S8 { BL [] 0 setdash 2 copy vpt sub vpt Square fill Bsquare } bind def
/S9 { BL [] 0 setdash 2 copy vpt sub vpt vpt2 Rec fill Bsquare } bind def
/S10 { BL [] 0 setdash 2 copy vpt sub vpt Square fill 2 copy exch vpt sub exch vpt Square fill
       Bsquare } bind def
/S11 { BL [] 0 setdash 2 copy vpt sub vpt Square fill 2 copy exch vpt sub exch vpt2 vpt Rec fill
       Bsquare } bind def
/S12 { BL [] 0 setdash 2 copy exch vpt sub exch vpt sub vpt2 vpt Rec fill Bsquare } bind def
/S13 { BL [] 0 setdash 2 copy exch vpt sub exch vpt sub vpt2 vpt Rec fill
       2 copy vpt Square fill Bsquare } bind def
/S14 { BL [] 0 setdash 2 copy exch vpt sub exch vpt sub vpt2 vpt Rec fill
       2 copy exch vpt sub exch vpt Square fill Bsquare } bind def
/S15 { BL [] 0 setdash 2 copy Bsquare fill Bsquare } bind def
/D0 { gsave translate 45 rotate 0 0 S0 stroke grestore } bind def
/D1 { gsave translate 45 rotate 0 0 S1 stroke grestore } bind def
/D2 { gsave translate 45 rotate 0 0 S2 stroke grestore } bind def
/D3 { gsave translate 45 rotate 0 0 S3 stroke grestore } bind def
/D4 { gsave translate 45 rotate 0 0 S4 stroke grestore } bind def
/D5 { gsave translate 45 rotate 0 0 S5 stroke grestore } bind def
/D6 { gsave translate 45 rotate 0 0 S6 stroke grestore } bind def
/D7 { gsave translate 45 rotate 0 0 S7 stroke grestore } bind def
/D8 { gsave translate 45 rotate 0 0 S8 stroke grestore } bind def
/D9 { gsave translate 45 rotate 0 0 S9 stroke grestore } bind def
/D10 { gsave translate 45 rotate 0 0 S10 stroke grestore } bind def
/D11 { gsave translate 45 rotate 0 0 S11 stroke grestore } bind def
/D12 { gsave translate 45 rotate 0 0 S12 stroke grestore } bind def
/D13 { gsave translate 45 rotate 0 0 S13 stroke grestore } bind def
/D14 { gsave translate 45 rotate 0 0 S14 stroke grestore } bind def
/D15 { gsave translate 45 rotate 0 0 S15 stroke grestore } bind def
/DiaE { stroke [] 0 setdash vpt add M
  hpt neg vpt neg V hpt vpt neg V
  hpt vpt V hpt neg vpt V closepath stroke } def
/BoxE { stroke [] 0 setdash exch hpt sub exch vpt add M
  0 vpt2 neg V hpt2 0 V 0 vpt2 V
  hpt2 neg 0 V closepath stroke } def
/TriUE { stroke [] 0 setdash vpt 1.12 mul add M
  hpt neg vpt -1.62 mul V
  hpt 2 mul 0 V
  hpt neg vpt 1.62 mul V closepath stroke } def
/TriDE { stroke [] 0 setdash vpt 1.12 mul sub M
  hpt neg vpt 1.62 mul V
  hpt 2 mul 0 V
  hpt neg vpt -1.62 mul V closepath stroke } def
/PentE { stroke [] 0 setdash gsave
  translate 0 hpt M 4 {72 rotate 0 hpt L} repeat
  closepath stroke grestore } def
/CircE { stroke [] 0 setdash 
  hpt 0 360 arc stroke } def
/Opaque { gsave closepath 1 setgray fill grestore 0 setgray closepath } def
/DiaW { stroke [] 0 setdash vpt add M
  hpt neg vpt neg V hpt vpt neg V
  hpt vpt V hpt neg vpt V Opaque stroke } def
/BoxW { stroke [] 0 setdash exch hpt sub exch vpt add M
  0 vpt2 neg V hpt2 0 V 0 vpt2 V
  hpt2 neg 0 V Opaque stroke } def
/TriUW { stroke [] 0 setdash vpt 1.12 mul add M
  hpt neg vpt -1.62 mul V
  hpt 2 mul 0 V
  hpt neg vpt 1.62 mul V Opaque stroke } def
/TriDW { stroke [] 0 setdash vpt 1.12 mul sub M
  hpt neg vpt 1.62 mul V
  hpt 2 mul 0 V
  hpt neg vpt -1.62 mul V Opaque stroke } def
/PentW { stroke [] 0 setdash gsave
  translate 0 hpt M 4 {72 rotate 0 hpt L} repeat
  Opaque stroke grestore } def
/CircW { stroke [] 0 setdash 
  hpt 0 360 arc Opaque stroke } def
/BoxFill { gsave Rec 1 setgray fill grestore } def
end
%%EndProlog
}}%
\begin{picture}(3600,2160)(0,0)%
{\GNUPLOTspecial{"
gnudict begin
gsave
0 0 translate
0.100 0.100 scale
0 setgray
newpath
1.000 UL
LTb
400 300 M
63 0 V
2987 0 R
-63 0 V
400 476 M
63 0 V
2987 0 R
-63 0 V
400 652 M
63 0 V
2987 0 R
-63 0 V
400 828 M
63 0 V
2987 0 R
-63 0 V
400 1004 M
63 0 V
2987 0 R
-63 0 V
400 1180 M
63 0 V
2987 0 R
-63 0 V
400 1356 M
63 0 V
2987 0 R
-63 0 V
400 1532 M
63 0 V
2987 0 R
-63 0 V
400 1708 M
63 0 V
2987 0 R
-63 0 V
400 1884 M
63 0 V
2987 0 R
-63 0 V
400 2060 M
63 0 V
2987 0 R
-63 0 V
705 300 M
0 63 V
0 1697 R
0 -63 V
1315 300 M
0 63 V
0 1697 R
0 -63 V
1925 300 M
0 63 V
0 1697 R
0 -63 V
2535 300 M
0 63 V
0 1697 R
0 -63 V
3145 300 M
0 63 V
0 1697 R
0 -63 V
1.000 UL
LTb
400 300 M
3050 0 V
0 1760 V
-3050 0 V
400 300 L
1.000 UL
LT0
3087 1947 M
263 0 V
400 328 M
31 0 V
31 -1 V
30 -2 V
31 -4 V
31 -4 V
31 -5 V
31 -5 V
30 -4 V
31 -2 V
31 -1 V
31 1 V
31 4 V
31 6 V
30 9 V
31 10 V
31 11 V
31 12 V
31 10 V
30 9 V
31 6 V
31 3 V
31 -1 V
31 -6 V
30 -9 V
31 -13 V
31 -14 V
31 -15 V
31 -13 V
30 -8 V
31 -2 V
31 8 V
31 20 V
31 35 V
30 50 V
31 67 V
31 84 V
31 102 V
31 116 V
31 130 V
30 140 V
31 145 V
31 146 V
31 144 V
31 134 V
30 121 V
31 103 V
31 81 V
31 56 V
31 29 V
30 0 V
31 -29 V
31 -56 V
31 -81 V
31 -103 V
30 -121 V
31 -134 V
31 -144 V
31 -146 V
31 -145 V
30 -140 V
31 -130 V
31 -116 V
31 -102 V
31 -84 V
31 -67 V
30 -50 V
31 -35 V
31 -20 V
31 -8 V
31 2 V
30 8 V
31 13 V
31 15 V
31 14 V
31 13 V
30 9 V
31 6 V
31 1 V
31 -3 V
31 -6 V
30 -9 V
31 -10 V
31 -12 V
31 -11 V
31 -10 V
30 -9 V
31 -6 V
31 -4 V
31 -1 V
31 1 V
31 2 V
30 4 V
31 5 V
31 5 V
31 4 V
31 4 V
30 2 V
31 1 V
31 0 V
stroke
grestore
end
showpage
}}%
\put(3037,1947){\makebox(0,0)[r]{I for 1 spalt}}%
\put(100,1180){%
\special{ps: gsave currentpoint currentpoint translate
270 rotate neg exch neg exch translate}%
\makebox(0,0)[b]{\shortstack{I}}%
\special{ps: currentpoint grestore moveto}%
}%
\put(3145,200){\makebox(0,0){2}}%
\put(2535,200){\makebox(0,0){1}}%
\put(1925,200){\makebox(0,0){0}}%
\put(1315,200){\makebox(0,0){-1}}%
\put(705,200){\makebox(0,0){-2}}%
\end{picture}%
\endgroup
\endinput

% GNUPLOT: LaTeX picture with Postscript
\begingroup%
  \makeatletter%
  \newcommand{\GNUPLOTspecial}{%
    \@sanitize\catcode`\%=14\relax\special}%
  \setlength{\unitlength}{0.1bp}%
{\GNUPLOTspecial{!
%!PS-Adobe-2.0 EPSF-2.0
%%Title: diff2.tex
%%Creator: gnuplot 3.7 patchlevel 0.2
%%CreationDate: Fri Feb 11 10:31:02 2000
%%DocumentFonts: 
%%BoundingBox: 0 0 360 216
%%Orientation: Landscape
%%EndComments
/gnudict 256 dict def
gnudict begin
/Color false def
/Solid false def
/gnulinewidth 5.000 def
/userlinewidth gnulinewidth def
/vshift -33 def
/dl {10 mul} def
/hpt_ 31.5 def
/vpt_ 31.5 def
/hpt hpt_ def
/vpt vpt_ def
/M {moveto} bind def
/L {lineto} bind def
/R {rmoveto} bind def
/V {rlineto} bind def
/vpt2 vpt 2 mul def
/hpt2 hpt 2 mul def
/Lshow { currentpoint stroke M
  0 vshift R show } def
/Rshow { currentpoint stroke M
  dup stringwidth pop neg vshift R show } def
/Cshow { currentpoint stroke M
  dup stringwidth pop -2 div vshift R show } def
/UP { dup vpt_ mul /vpt exch def hpt_ mul /hpt exch def
  /hpt2 hpt 2 mul def /vpt2 vpt 2 mul def } def
/DL { Color {setrgbcolor Solid {pop []} if 0 setdash }
 {pop pop pop Solid {pop []} if 0 setdash} ifelse } def
/BL { stroke userlinewidth 2 mul setlinewidth } def
/AL { stroke userlinewidth 2 div setlinewidth } def
/UL { dup gnulinewidth mul /userlinewidth exch def
      10 mul /udl exch def } def
/PL { stroke userlinewidth setlinewidth } def
/LTb { BL [] 0 0 0 DL } def
/LTa { AL [1 udl mul 2 udl mul] 0 setdash 0 0 0 setrgbcolor } def
/LT0 { PL [] 1 0 0 DL } def
/LT1 { PL [4 dl 2 dl] 0 1 0 DL } def
/LT2 { PL [2 dl 3 dl] 0 0 1 DL } def
/LT3 { PL [1 dl 1.5 dl] 1 0 1 DL } def
/LT4 { PL [5 dl 2 dl 1 dl 2 dl] 0 1 1 DL } def
/LT5 { PL [4 dl 3 dl 1 dl 3 dl] 1 1 0 DL } def
/LT6 { PL [2 dl 2 dl 2 dl 4 dl] 0 0 0 DL } def
/LT7 { PL [2 dl 2 dl 2 dl 2 dl 2 dl 4 dl] 1 0.3 0 DL } def
/LT8 { PL [2 dl 2 dl 2 dl 2 dl 2 dl 2 dl 2 dl 4 dl] 0.5 0.5 0.5 DL } def
/Pnt { stroke [] 0 setdash
   gsave 1 setlinecap M 0 0 V stroke grestore } def
/Dia { stroke [] 0 setdash 2 copy vpt add M
  hpt neg vpt neg V hpt vpt neg V
  hpt vpt V hpt neg vpt V closepath stroke
  Pnt } def
/Pls { stroke [] 0 setdash vpt sub M 0 vpt2 V
  currentpoint stroke M
  hpt neg vpt neg R hpt2 0 V stroke
  } def
/Box { stroke [] 0 setdash 2 copy exch hpt sub exch vpt add M
  0 vpt2 neg V hpt2 0 V 0 vpt2 V
  hpt2 neg 0 V closepath stroke
  Pnt } def
/Crs { stroke [] 0 setdash exch hpt sub exch vpt add M
  hpt2 vpt2 neg V currentpoint stroke M
  hpt2 neg 0 R hpt2 vpt2 V stroke } def
/TriU { stroke [] 0 setdash 2 copy vpt 1.12 mul add M
  hpt neg vpt -1.62 mul V
  hpt 2 mul 0 V
  hpt neg vpt 1.62 mul V closepath stroke
  Pnt  } def
/Star { 2 copy Pls Crs } def
/BoxF { stroke [] 0 setdash exch hpt sub exch vpt add M
  0 vpt2 neg V  hpt2 0 V  0 vpt2 V
  hpt2 neg 0 V  closepath fill } def
/TriUF { stroke [] 0 setdash vpt 1.12 mul add M
  hpt neg vpt -1.62 mul V
  hpt 2 mul 0 V
  hpt neg vpt 1.62 mul V closepath fill } def
/TriD { stroke [] 0 setdash 2 copy vpt 1.12 mul sub M
  hpt neg vpt 1.62 mul V
  hpt 2 mul 0 V
  hpt neg vpt -1.62 mul V closepath stroke
  Pnt  } def
/TriDF { stroke [] 0 setdash vpt 1.12 mul sub M
  hpt neg vpt 1.62 mul V
  hpt 2 mul 0 V
  hpt neg vpt -1.62 mul V closepath fill} def
/DiaF { stroke [] 0 setdash vpt add M
  hpt neg vpt neg V hpt vpt neg V
  hpt vpt V hpt neg vpt V closepath fill } def
/Pent { stroke [] 0 setdash 2 copy gsave
  translate 0 hpt M 4 {72 rotate 0 hpt L} repeat
  closepath stroke grestore Pnt } def
/PentF { stroke [] 0 setdash gsave
  translate 0 hpt M 4 {72 rotate 0 hpt L} repeat
  closepath fill grestore } def
/Circle { stroke [] 0 setdash 2 copy
  hpt 0 360 arc stroke Pnt } def
/CircleF { stroke [] 0 setdash hpt 0 360 arc fill } def
/C0 { BL [] 0 setdash 2 copy moveto vpt 90 450  arc } bind def
/C1 { BL [] 0 setdash 2 copy        moveto
       2 copy  vpt 0 90 arc closepath fill
               vpt 0 360 arc closepath } bind def
/C2 { BL [] 0 setdash 2 copy moveto
       2 copy  vpt 90 180 arc closepath fill
               vpt 0 360 arc closepath } bind def
/C3 { BL [] 0 setdash 2 copy moveto
       2 copy  vpt 0 180 arc closepath fill
               vpt 0 360 arc closepath } bind def
/C4 { BL [] 0 setdash 2 copy moveto
       2 copy  vpt 180 270 arc closepath fill
               vpt 0 360 arc closepath } bind def
/C5 { BL [] 0 setdash 2 copy moveto
       2 copy  vpt 0 90 arc
       2 copy moveto
       2 copy  vpt 180 270 arc closepath fill
               vpt 0 360 arc } bind def
/C6 { BL [] 0 setdash 2 copy moveto
      2 copy  vpt 90 270 arc closepath fill
              vpt 0 360 arc closepath } bind def
/C7 { BL [] 0 setdash 2 copy moveto
      2 copy  vpt 0 270 arc closepath fill
              vpt 0 360 arc closepath } bind def
/C8 { BL [] 0 setdash 2 copy moveto
      2 copy vpt 270 360 arc closepath fill
              vpt 0 360 arc closepath } bind def
/C9 { BL [] 0 setdash 2 copy moveto
      2 copy  vpt 270 450 arc closepath fill
              vpt 0 360 arc closepath } bind def
/C10 { BL [] 0 setdash 2 copy 2 copy moveto vpt 270 360 arc closepath fill
       2 copy moveto
       2 copy vpt 90 180 arc closepath fill
               vpt 0 360 arc closepath } bind def
/C11 { BL [] 0 setdash 2 copy moveto
       2 copy  vpt 0 180 arc closepath fill
       2 copy moveto
       2 copy  vpt 270 360 arc closepath fill
               vpt 0 360 arc closepath } bind def
/C12 { BL [] 0 setdash 2 copy moveto
       2 copy  vpt 180 360 arc closepath fill
               vpt 0 360 arc closepath } bind def
/C13 { BL [] 0 setdash  2 copy moveto
       2 copy  vpt 0 90 arc closepath fill
       2 copy moveto
       2 copy  vpt 180 360 arc closepath fill
               vpt 0 360 arc closepath } bind def
/C14 { BL [] 0 setdash 2 copy moveto
       2 copy  vpt 90 360 arc closepath fill
               vpt 0 360 arc } bind def
/C15 { BL [] 0 setdash 2 copy vpt 0 360 arc closepath fill
               vpt 0 360 arc closepath } bind def
/Rec   { newpath 4 2 roll moveto 1 index 0 rlineto 0 exch rlineto
       neg 0 rlineto closepath } bind def
/Square { dup Rec } bind def
/Bsquare { vpt sub exch vpt sub exch vpt2 Square } bind def
/S0 { BL [] 0 setdash 2 copy moveto 0 vpt rlineto BL Bsquare } bind def
/S1 { BL [] 0 setdash 2 copy vpt Square fill Bsquare } bind def
/S2 { BL [] 0 setdash 2 copy exch vpt sub exch vpt Square fill Bsquare } bind def
/S3 { BL [] 0 setdash 2 copy exch vpt sub exch vpt2 vpt Rec fill Bsquare } bind def
/S4 { BL [] 0 setdash 2 copy exch vpt sub exch vpt sub vpt Square fill Bsquare } bind def
/S5 { BL [] 0 setdash 2 copy 2 copy vpt Square fill
       exch vpt sub exch vpt sub vpt Square fill Bsquare } bind def
/S6 { BL [] 0 setdash 2 copy exch vpt sub exch vpt sub vpt vpt2 Rec fill Bsquare } bind def
/S7 { BL [] 0 setdash 2 copy exch vpt sub exch vpt sub vpt vpt2 Rec fill
       2 copy vpt Square fill
       Bsquare } bind def
/S8 { BL [] 0 setdash 2 copy vpt sub vpt Square fill Bsquare } bind def
/S9 { BL [] 0 setdash 2 copy vpt sub vpt vpt2 Rec fill Bsquare } bind def
/S10 { BL [] 0 setdash 2 copy vpt sub vpt Square fill 2 copy exch vpt sub exch vpt Square fill
       Bsquare } bind def
/S11 { BL [] 0 setdash 2 copy vpt sub vpt Square fill 2 copy exch vpt sub exch vpt2 vpt Rec fill
       Bsquare } bind def
/S12 { BL [] 0 setdash 2 copy exch vpt sub exch vpt sub vpt2 vpt Rec fill Bsquare } bind def
/S13 { BL [] 0 setdash 2 copy exch vpt sub exch vpt sub vpt2 vpt Rec fill
       2 copy vpt Square fill Bsquare } bind def
/S14 { BL [] 0 setdash 2 copy exch vpt sub exch vpt sub vpt2 vpt Rec fill
       2 copy exch vpt sub exch vpt Square fill Bsquare } bind def
/S15 { BL [] 0 setdash 2 copy Bsquare fill Bsquare } bind def
/D0 { gsave translate 45 rotate 0 0 S0 stroke grestore } bind def
/D1 { gsave translate 45 rotate 0 0 S1 stroke grestore } bind def
/D2 { gsave translate 45 rotate 0 0 S2 stroke grestore } bind def
/D3 { gsave translate 45 rotate 0 0 S3 stroke grestore } bind def
/D4 { gsave translate 45 rotate 0 0 S4 stroke grestore } bind def
/D5 { gsave translate 45 rotate 0 0 S5 stroke grestore } bind def
/D6 { gsave translate 45 rotate 0 0 S6 stroke grestore } bind def
/D7 { gsave translate 45 rotate 0 0 S7 stroke grestore } bind def
/D8 { gsave translate 45 rotate 0 0 S8 stroke grestore } bind def
/D9 { gsave translate 45 rotate 0 0 S9 stroke grestore } bind def
/D10 { gsave translate 45 rotate 0 0 S10 stroke grestore } bind def
/D11 { gsave translate 45 rotate 0 0 S11 stroke grestore } bind def
/D12 { gsave translate 45 rotate 0 0 S12 stroke grestore } bind def
/D13 { gsave translate 45 rotate 0 0 S13 stroke grestore } bind def
/D14 { gsave translate 45 rotate 0 0 S14 stroke grestore } bind def
/D15 { gsave translate 45 rotate 0 0 S15 stroke grestore } bind def
/DiaE { stroke [] 0 setdash vpt add M
  hpt neg vpt neg V hpt vpt neg V
  hpt vpt V hpt neg vpt V closepath stroke } def
/BoxE { stroke [] 0 setdash exch hpt sub exch vpt add M
  0 vpt2 neg V hpt2 0 V 0 vpt2 V
  hpt2 neg 0 V closepath stroke } def
/TriUE { stroke [] 0 setdash vpt 1.12 mul add M
  hpt neg vpt -1.62 mul V
  hpt 2 mul 0 V
  hpt neg vpt 1.62 mul V closepath stroke } def
/TriDE { stroke [] 0 setdash vpt 1.12 mul sub M
  hpt neg vpt 1.62 mul V
  hpt 2 mul 0 V
  hpt neg vpt -1.62 mul V closepath stroke } def
/PentE { stroke [] 0 setdash gsave
  translate 0 hpt M 4 {72 rotate 0 hpt L} repeat
  closepath stroke grestore } def
/CircE { stroke [] 0 setdash 
  hpt 0 360 arc stroke } def
/Opaque { gsave closepath 1 setgray fill grestore 0 setgray closepath } def
/DiaW { stroke [] 0 setdash vpt add M
  hpt neg vpt neg V hpt vpt neg V
  hpt vpt V hpt neg vpt V Opaque stroke } def
/BoxW { stroke [] 0 setdash exch hpt sub exch vpt add M
  0 vpt2 neg V hpt2 0 V 0 vpt2 V
  hpt2 neg 0 V Opaque stroke } def
/TriUW { stroke [] 0 setdash vpt 1.12 mul add M
  hpt neg vpt -1.62 mul V
  hpt 2 mul 0 V
  hpt neg vpt 1.62 mul V Opaque stroke } def
/TriDW { stroke [] 0 setdash vpt 1.12 mul sub M
  hpt neg vpt 1.62 mul V
  hpt 2 mul 0 V
  hpt neg vpt -1.62 mul V Opaque stroke } def
/PentW { stroke [] 0 setdash gsave
  translate 0 hpt M 4 {72 rotate 0 hpt L} repeat
  Opaque stroke grestore } def
/CircW { stroke [] 0 setdash 
  hpt 0 360 arc Opaque stroke } def
/BoxFill { gsave Rec 1 setgray fill grestore } def
end
%%EndProlog
}}%
\begin{picture}(3600,2160)(0,0)%
{\GNUPLOTspecial{"
gnudict begin
gsave
0 0 translate
0.100 0.100 scale
0 setgray
newpath
1.000 UL
LTb
400 300 M
63 0 V
2987 0 R
-63 0 V
400 476 M
63 0 V
2987 0 R
-63 0 V
400 652 M
63 0 V
2987 0 R
-63 0 V
400 828 M
63 0 V
2987 0 R
-63 0 V
400 1004 M
63 0 V
2987 0 R
-63 0 V
400 1180 M
63 0 V
2987 0 R
-63 0 V
400 1356 M
63 0 V
2987 0 R
-63 0 V
400 1532 M
63 0 V
2987 0 R
-63 0 V
400 1708 M
63 0 V
2987 0 R
-63 0 V
400 1884 M
63 0 V
2987 0 R
-63 0 V
400 2060 M
63 0 V
2987 0 R
-63 0 V
705 300 M
0 63 V
0 1697 R
0 -63 V
1315 300 M
0 63 V
0 1697 R
0 -63 V
1925 300 M
0 63 V
0 1697 R
0 -63 V
2535 300 M
0 63 V
0 1697 R
0 -63 V
3145 300 M
0 63 V
0 1697 R
0 -63 V
1.000 UL
LTb
400 300 M
3050 0 V
0 1760 V
-3050 0 V
400 300 L
1.000 UL
LT0
3087 1947 M
263 0 V
400 300 M
31 1 V
31 2 V
30 2 V
31 2 V
31 2 V
31 -1 V
31 -2 V
30 -3 V
31 -2 V
31 -1 V
31 1 V
31 4 V
31 4 V
30 4 V
31 1 V
31 -1 V
31 -3 V
31 -5 V
30 -4 V
31 -1 V
31 3 V
31 6 V
31 9 V
30 7 V
31 3 V
31 -2 V
31 -7 V
31 -9 V
30 -8 V
31 -2 V
31 8 V
31 17 V
31 22 V
30 21 V
31 12 V
31 -3 V
31 -18 V
31 -29 V
31 -24 V
30 -3 V
31 41 V
31 100 V
31 169 V
31 233 V
30 278 V
31 292 V
31 268 V
31 205 V
31 111 V
30 0 V
31 -111 V
31 -205 V
31 -268 V
31 -292 V
30 -278 V
31 -233 V
31 -169 V
31 -100 V
31 -41 V
30 3 V
31 24 V
31 29 V
31 18 V
31 3 V
31 -12 V
30 -21 V
31 -22 V
31 -17 V
31 -8 V
31 2 V
30 8 V
31 9 V
31 7 V
31 2 V
31 -3 V
30 -7 V
31 -9 V
31 -6 V
31 -3 V
31 1 V
30 4 V
31 5 V
31 3 V
31 1 V
31 -1 V
30 -4 V
31 -4 V
31 -4 V
31 -1 V
31 1 V
31 2 V
30 3 V
31 2 V
31 1 V
31 -2 V
31 -2 V
30 -2 V
31 -2 V
31 -1 V
stroke
grestore
end
showpage
}}%
\put(3037,1947){\makebox(0,0)[r]{I for 2 spalter}}%
\put(1925,50){\makebox(0,0){$u=asin\theta/\lambda$}}%
\put(100,1180){%
\special{ps: gsave currentpoint currentpoint translate
270 rotate neg exch neg exch translate}%
\makebox(0,0)[b]{\shortstack{I}}%
\special{ps: currentpoint grestore moveto}%
}%
\put(3145,200){\makebox(0,0){2}}%
\put(2535,200){\makebox(0,0){1}}%
\put(1925,200){\makebox(0,0){0}}%
\put(1315,200){\makebox(0,0){-1}}%
\put(705,200){\makebox(0,0){-2}}%
\end{picture}%
\endgroup
\endinput

\caption{Eksempel p\aa\ interferensm\o nster med henhodsvis 1 og 2 spalte\aa pninger. Her har vi for enkelthetsskyld satt $a=d$ for tilfellet med to spalter.
\label{fig:diff}} 
\end{figure}

\subsection{Fourieranalyse og uskarphetsrelasjonen}
Her skal vi pr\o ve \aa\ bygge en bro mellom begrep som b\o lgepakker
og gruppehastighet og neste avsnitt om Heisenbergs uskarphetsrelasjon.

La oss repetere litt av diskusjonen rundt Figurene \ref{fig:fasev} og 
\ref{fig:waveex}. 
De fleste av dere er kanskje vant med at en b\o lge er noe som kan 
likne p\aa\ den harmoniske svingningen vist i Figur \ref{fig:fasev}. Problemet her er
at b\o lgen v\aa r har en utstrekning i rom som er uendelig. Det blir dermed
vanskelig \aa\ tilordne en partikkel eller f.eks.~en h\o yttaler puls, eller
en radiob\o lge. Skal vi kople b\o lgebeskrivelsen til en beskrivelse av
materie, m\aa\ b\o lgen
ha ei begrenset utstrekning i rom. Vi vil da bruke b\o lgen for \aa\ uttrykke
en sannsynlighet for \aa\ finne partikkelen et bestemt sted. Er utbredelsen
uendelig, kan vi ikke si noe som helst om hvor partikkelen er. 

Det vi ogs\aa\ la merke til var at dersom vi satte at hastigheten til 
partikkelen skulle svare til fasehastigheten, fikk vi problemer med
de Broglie sitt postulat. 
Vi \o nsker derfor at v\aa r b\o lge skal ha en form som likner mer p\aa\ 
hva vi ser i Figur \ref{fig:waveex}. Rekningen v\aa r viste at dersom vi introduserer
begrepet gruppehastighet $v_g$, dvs.~hastigheten som b\o lgepakken (som er satt
sammen av flere b\o lger) reiser med, fant vi at partikkelens hastighet svarte
til gruppehastigheten, 

For \aa\ f\aa\ til dette finnes det et teorem
fra Fourier som sier at dersom vi \o nsker at v\aa r b\o lge skal
v\ae re forskjellig fra null kun i et bestemt omr\aa de i rommet, 
m\aa\ vi integrere
over alle frekvenser $\nu$ eller b\o lgetall $k$. 

F\o lgende eksempel illustrer dette. Eksemplet tillatter 
oss ogs\aa\ \aa\ lage ei kopling til   
Heisenbergs uskarphetsrelasjon.

Anta at du er i stand til \aa\ lage en firkantpuls som vist i neste figur.
Denne pulsen er sentrert rundt et bestemt b\o lgetall $k_0$ (eller frekvens
om du foretrekker det). Den er null
n\aa r $k < k_0-\Delta $ og $k > k_0+\Delta $. 

\begin{figure}[htbp]
%
\begin{center}

\setlength{\unitlength}{1cm}
\begin{picture}(13,6)

\thicklines

   \put(0,0.5){\makebox(0,0)[bl]{
              \put(0,1){\vector(1,0){12}}
              \put(12.3,1){\makebox(0,0){k}}
              \put(5.2,0.8){\makebox(0,0){$k_0-\Delta$}}
              \put(8.2,0.8){\makebox(0,0){$k_0+\Delta$}}
              \put(6.7,0.8){\makebox(0,0){$k_0$}}
              \put(5,1){\line(0,1){3}}
              \put(5,4){\line(1,0){3}}
              \put(8,1){\line(0,1){3}}
         }}
\end{picture}
\end{center}
\caption{Eksempel p\aa\ firkantpuls sentrert rundt et b\o lgetall $k_0$.} 
\end{figure}

Vi kan n\aa\ tenke oss at vi har en b\o lgefunksjon $\psi(x)$ 
ved tida $t=0$ gitt som et 
integral over alle mulige b\o lgetall $k$.  Vi antar ogs\aa\ at pulsen v\aa r
har en amplitude lik $A(k)=1$ for hvert b\o lgetall som faller innenfor
det tillatte intervallet, slik at b\o lgefunksjonen blir da
\be
   \psi(x)=\int_{-\infty}^{\infty}A(k)cos(2\pi kx)dk=
           \int_{k_0-\Delta}^{k_0+\Delta}cos(2\pi kx)dk.
\ee
Her har vi valgt en cosinus funksjon for hver $k$-verdi. I tillegg s\aa\ er 
pulsen v\aa r forskjellig fra null for kun bestemte verdier av $k$, 
slik at vi kan redusere integralet til et lite omr\aa de. 

Dette integralet kan lett l\o ses, og vi finner
\be
   \psi(x)=\frac{1}{2\pi x}\int_{2\pi(k_0-\Delta)x}^{2\pi(k_0+\Delta)x}
           cos(2\pi kx)d(2\pi kx),
\ee
som gir
\be
  \psi(x)=\frac{1}{2\pi x}\left[sin(2\pi (k_0+\Delta)x)-
                                sin(2\pi (k_0-\Delta)x)\right],
\ee
som gir
\be
    \psi(x)=\frac{1}{\pi x}\left[cos(2\pi k_0x)sin(2\pi \Delta x)\right],
\ee
eller
\be
    \psi(x)=2\Delta cos(2\pi k_0x)\frac{sin(2\pi \Delta x)}{2\pi \Delta x}.
\ee

La oss plotte  $\psi$ som funksjon av $x$ for to ulike verdier av $\Delta$.
Velg f\o rst $\Delta =0.001$, dvs.~at v\aa r puls er skarpt
bestemt rundt verdien $k_0$. Deretter velger vi $\Delta=100$, som betyr at
vi integrerer over mange flere frekvenser. 
\begin{figure}
% GNUPLOT: LaTeX picture with Postscript
\begingroup%
  \makeatletter%
  \newcommand{\GNUPLOTspecial}{%
    \@sanitize\catcode`\%=14\relax\special}%
  \setlength{\unitlength}{0.1bp}%
{\GNUPLOTspecial{!
%!PS-Adobe-2.0 EPSF-2.0
%%Title: fourier1.tex
%%Creator: gnuplot 3.7 patchlevel 1
%%CreationDate: Thu Feb 15 20:26:47 2001
%%DocumentFonts: 
%%BoundingBox: 0 0 360 216
%%Orientation: Landscape
%%EndComments
/gnudict 256 dict def
gnudict begin
/Color false def
/Solid false def
/gnulinewidth 5.000 def
/userlinewidth gnulinewidth def
/vshift -33 def
/dl {10 mul} def
/hpt_ 31.5 def
/vpt_ 31.5 def
/hpt hpt_ def
/vpt vpt_ def
/M {moveto} bind def
/L {lineto} bind def
/R {rmoveto} bind def
/V {rlineto} bind def
/vpt2 vpt 2 mul def
/hpt2 hpt 2 mul def
/Lshow { currentpoint stroke M
  0 vshift R show } def
/Rshow { currentpoint stroke M
  dup stringwidth pop neg vshift R show } def
/Cshow { currentpoint stroke M
  dup stringwidth pop -2 div vshift R show } def
/UP { dup vpt_ mul /vpt exch def hpt_ mul /hpt exch def
  /hpt2 hpt 2 mul def /vpt2 vpt 2 mul def } def
/DL { Color {setrgbcolor Solid {pop []} if 0 setdash }
 {pop pop pop Solid {pop []} if 0 setdash} ifelse } def
/BL { stroke userlinewidth 2 mul setlinewidth } def
/AL { stroke userlinewidth 2 div setlinewidth } def
/UL { dup gnulinewidth mul /userlinewidth exch def
      10 mul /udl exch def } def
/PL { stroke userlinewidth setlinewidth } def
/LTb { BL [] 0 0 0 DL } def
/LTa { AL [1 udl mul 2 udl mul] 0 setdash 0 0 0 setrgbcolor } def
/LT0 { PL [] 1 0 0 DL } def
/LT1 { PL [4 dl 2 dl] 0 1 0 DL } def
/LT2 { PL [2 dl 3 dl] 0 0 1 DL } def
/LT3 { PL [1 dl 1.5 dl] 1 0 1 DL } def
/LT4 { PL [5 dl 2 dl 1 dl 2 dl] 0 1 1 DL } def
/LT5 { PL [4 dl 3 dl 1 dl 3 dl] 1 1 0 DL } def
/LT6 { PL [2 dl 2 dl 2 dl 4 dl] 0 0 0 DL } def
/LT7 { PL [2 dl 2 dl 2 dl 2 dl 2 dl 4 dl] 1 0.3 0 DL } def
/LT8 { PL [2 dl 2 dl 2 dl 2 dl 2 dl 2 dl 2 dl 4 dl] 0.5 0.5 0.5 DL } def
/Pnt { stroke [] 0 setdash
   gsave 1 setlinecap M 0 0 V stroke grestore } def
/Dia { stroke [] 0 setdash 2 copy vpt add M
  hpt neg vpt neg V hpt vpt neg V
  hpt vpt V hpt neg vpt V closepath stroke
  Pnt } def
/Pls { stroke [] 0 setdash vpt sub M 0 vpt2 V
  currentpoint stroke M
  hpt neg vpt neg R hpt2 0 V stroke
  } def
/Box { stroke [] 0 setdash 2 copy exch hpt sub exch vpt add M
  0 vpt2 neg V hpt2 0 V 0 vpt2 V
  hpt2 neg 0 V closepath stroke
  Pnt } def
/Crs { stroke [] 0 setdash exch hpt sub exch vpt add M
  hpt2 vpt2 neg V currentpoint stroke M
  hpt2 neg 0 R hpt2 vpt2 V stroke } def
/TriU { stroke [] 0 setdash 2 copy vpt 1.12 mul add M
  hpt neg vpt -1.62 mul V
  hpt 2 mul 0 V
  hpt neg vpt 1.62 mul V closepath stroke
  Pnt  } def
/Star { 2 copy Pls Crs } def
/BoxF { stroke [] 0 setdash exch hpt sub exch vpt add M
  0 vpt2 neg V  hpt2 0 V  0 vpt2 V
  hpt2 neg 0 V  closepath fill } def
/TriUF { stroke [] 0 setdash vpt 1.12 mul add M
  hpt neg vpt -1.62 mul V
  hpt 2 mul 0 V
  hpt neg vpt 1.62 mul V closepath fill } def
/TriD { stroke [] 0 setdash 2 copy vpt 1.12 mul sub M
  hpt neg vpt 1.62 mul V
  hpt 2 mul 0 V
  hpt neg vpt -1.62 mul V closepath stroke
  Pnt  } def
/TriDF { stroke [] 0 setdash vpt 1.12 mul sub M
  hpt neg vpt 1.62 mul V
  hpt 2 mul 0 V
  hpt neg vpt -1.62 mul V closepath fill} def
/DiaF { stroke [] 0 setdash vpt add M
  hpt neg vpt neg V hpt vpt neg V
  hpt vpt V hpt neg vpt V closepath fill } def
/Pent { stroke [] 0 setdash 2 copy gsave
  translate 0 hpt M 4 {72 rotate 0 hpt L} repeat
  closepath stroke grestore Pnt } def
/PentF { stroke [] 0 setdash gsave
  translate 0 hpt M 4 {72 rotate 0 hpt L} repeat
  closepath fill grestore } def
/Circle { stroke [] 0 setdash 2 copy
  hpt 0 360 arc stroke Pnt } def
/CircleF { stroke [] 0 setdash hpt 0 360 arc fill } def
/C0 { BL [] 0 setdash 2 copy moveto vpt 90 450  arc } bind def
/C1 { BL [] 0 setdash 2 copy        moveto
       2 copy  vpt 0 90 arc closepath fill
               vpt 0 360 arc closepath } bind def
/C2 { BL [] 0 setdash 2 copy moveto
       2 copy  vpt 90 180 arc closepath fill
               vpt 0 360 arc closepath } bind def
/C3 { BL [] 0 setdash 2 copy moveto
       2 copy  vpt 0 180 arc closepath fill
               vpt 0 360 arc closepath } bind def
/C4 { BL [] 0 setdash 2 copy moveto
       2 copy  vpt 180 270 arc closepath fill
               vpt 0 360 arc closepath } bind def
/C5 { BL [] 0 setdash 2 copy moveto
       2 copy  vpt 0 90 arc
       2 copy moveto
       2 copy  vpt 180 270 arc closepath fill
               vpt 0 360 arc } bind def
/C6 { BL [] 0 setdash 2 copy moveto
      2 copy  vpt 90 270 arc closepath fill
              vpt 0 360 arc closepath } bind def
/C7 { BL [] 0 setdash 2 copy moveto
      2 copy  vpt 0 270 arc closepath fill
              vpt 0 360 arc closepath } bind def
/C8 { BL [] 0 setdash 2 copy moveto
      2 copy vpt 270 360 arc closepath fill
              vpt 0 360 arc closepath } bind def
/C9 { BL [] 0 setdash 2 copy moveto
      2 copy  vpt 270 450 arc closepath fill
              vpt 0 360 arc closepath } bind def
/C10 { BL [] 0 setdash 2 copy 2 copy moveto vpt 270 360 arc closepath fill
       2 copy moveto
       2 copy vpt 90 180 arc closepath fill
               vpt 0 360 arc closepath } bind def
/C11 { BL [] 0 setdash 2 copy moveto
       2 copy  vpt 0 180 arc closepath fill
       2 copy moveto
       2 copy  vpt 270 360 arc closepath fill
               vpt 0 360 arc closepath } bind def
/C12 { BL [] 0 setdash 2 copy moveto
       2 copy  vpt 180 360 arc closepath fill
               vpt 0 360 arc closepath } bind def
/C13 { BL [] 0 setdash  2 copy moveto
       2 copy  vpt 0 90 arc closepath fill
       2 copy moveto
       2 copy  vpt 180 360 arc closepath fill
               vpt 0 360 arc closepath } bind def
/C14 { BL [] 0 setdash 2 copy moveto
       2 copy  vpt 90 360 arc closepath fill
               vpt 0 360 arc } bind def
/C15 { BL [] 0 setdash 2 copy vpt 0 360 arc closepath fill
               vpt 0 360 arc closepath } bind def
/Rec   { newpath 4 2 roll moveto 1 index 0 rlineto 0 exch rlineto
       neg 0 rlineto closepath } bind def
/Square { dup Rec } bind def
/Bsquare { vpt sub exch vpt sub exch vpt2 Square } bind def
/S0 { BL [] 0 setdash 2 copy moveto 0 vpt rlineto BL Bsquare } bind def
/S1 { BL [] 0 setdash 2 copy vpt Square fill Bsquare } bind def
/S2 { BL [] 0 setdash 2 copy exch vpt sub exch vpt Square fill Bsquare } bind def
/S3 { BL [] 0 setdash 2 copy exch vpt sub exch vpt2 vpt Rec fill Bsquare } bind def
/S4 { BL [] 0 setdash 2 copy exch vpt sub exch vpt sub vpt Square fill Bsquare } bind def
/S5 { BL [] 0 setdash 2 copy 2 copy vpt Square fill
       exch vpt sub exch vpt sub vpt Square fill Bsquare } bind def
/S6 { BL [] 0 setdash 2 copy exch vpt sub exch vpt sub vpt vpt2 Rec fill Bsquare } bind def
/S7 { BL [] 0 setdash 2 copy exch vpt sub exch vpt sub vpt vpt2 Rec fill
       2 copy vpt Square fill
       Bsquare } bind def
/S8 { BL [] 0 setdash 2 copy vpt sub vpt Square fill Bsquare } bind def
/S9 { BL [] 0 setdash 2 copy vpt sub vpt vpt2 Rec fill Bsquare } bind def
/S10 { BL [] 0 setdash 2 copy vpt sub vpt Square fill 2 copy exch vpt sub exch vpt Square fill
       Bsquare } bind def
/S11 { BL [] 0 setdash 2 copy vpt sub vpt Square fill 2 copy exch vpt sub exch vpt2 vpt Rec fill
       Bsquare } bind def
/S12 { BL [] 0 setdash 2 copy exch vpt sub exch vpt sub vpt2 vpt Rec fill Bsquare } bind def
/S13 { BL [] 0 setdash 2 copy exch vpt sub exch vpt sub vpt2 vpt Rec fill
       2 copy vpt Square fill Bsquare } bind def
/S14 { BL [] 0 setdash 2 copy exch vpt sub exch vpt sub vpt2 vpt Rec fill
       2 copy exch vpt sub exch vpt Square fill Bsquare } bind def
/S15 { BL [] 0 setdash 2 copy Bsquare fill Bsquare } bind def
/D0 { gsave translate 45 rotate 0 0 S0 stroke grestore } bind def
/D1 { gsave translate 45 rotate 0 0 S1 stroke grestore } bind def
/D2 { gsave translate 45 rotate 0 0 S2 stroke grestore } bind def
/D3 { gsave translate 45 rotate 0 0 S3 stroke grestore } bind def
/D4 { gsave translate 45 rotate 0 0 S4 stroke grestore } bind def
/D5 { gsave translate 45 rotate 0 0 S5 stroke grestore } bind def
/D6 { gsave translate 45 rotate 0 0 S6 stroke grestore } bind def
/D7 { gsave translate 45 rotate 0 0 S7 stroke grestore } bind def
/D8 { gsave translate 45 rotate 0 0 S8 stroke grestore } bind def
/D9 { gsave translate 45 rotate 0 0 S9 stroke grestore } bind def
/D10 { gsave translate 45 rotate 0 0 S10 stroke grestore } bind def
/D11 { gsave translate 45 rotate 0 0 S11 stroke grestore } bind def
/D12 { gsave translate 45 rotate 0 0 S12 stroke grestore } bind def
/D13 { gsave translate 45 rotate 0 0 S13 stroke grestore } bind def
/D14 { gsave translate 45 rotate 0 0 S14 stroke grestore } bind def
/D15 { gsave translate 45 rotate 0 0 S15 stroke grestore } bind def
/DiaE { stroke [] 0 setdash vpt add M
  hpt neg vpt neg V hpt vpt neg V
  hpt vpt V hpt neg vpt V closepath stroke } def
/BoxE { stroke [] 0 setdash exch hpt sub exch vpt add M
  0 vpt2 neg V hpt2 0 V 0 vpt2 V
  hpt2 neg 0 V closepath stroke } def
/TriUE { stroke [] 0 setdash vpt 1.12 mul add M
  hpt neg vpt -1.62 mul V
  hpt 2 mul 0 V
  hpt neg vpt 1.62 mul V closepath stroke } def
/TriDE { stroke [] 0 setdash vpt 1.12 mul sub M
  hpt neg vpt 1.62 mul V
  hpt 2 mul 0 V
  hpt neg vpt -1.62 mul V closepath stroke } def
/PentE { stroke [] 0 setdash gsave
  translate 0 hpt M 4 {72 rotate 0 hpt L} repeat
  closepath stroke grestore } def
/CircE { stroke [] 0 setdash 
  hpt 0 360 arc stroke } def
/Opaque { gsave closepath 1 setgray fill grestore 0 setgray closepath } def
/DiaW { stroke [] 0 setdash vpt add M
  hpt neg vpt neg V hpt vpt neg V
  hpt vpt V hpt neg vpt V Opaque stroke } def
/BoxW { stroke [] 0 setdash exch hpt sub exch vpt add M
  0 vpt2 neg V hpt2 0 V 0 vpt2 V
  hpt2 neg 0 V Opaque stroke } def
/TriUW { stroke [] 0 setdash vpt 1.12 mul add M
  hpt neg vpt -1.62 mul V
  hpt 2 mul 0 V
  hpt neg vpt 1.62 mul V Opaque stroke } def
/TriDW { stroke [] 0 setdash vpt 1.12 mul sub M
  hpt neg vpt 1.62 mul V
  hpt 2 mul 0 V
  hpt neg vpt -1.62 mul V Opaque stroke } def
/PentW { stroke [] 0 setdash gsave
  translate 0 hpt M 4 {72 rotate 0 hpt L} repeat
  Opaque stroke grestore } def
/CircW { stroke [] 0 setdash 
  hpt 0 360 arc Opaque stroke } def
/BoxFill { gsave Rec 1 setgray fill grestore } def
end
%%EndProlog
}}%
\begin{picture}(3600,2160)(0,0)%
{\GNUPLOTspecial{"
gnudict begin
gsave
0 0 translate
0.100 0.100 scale
0 setgray
newpath
1.000 UL
LTb
550 300 M
63 0 V
2837 0 R
-63 0 V
550 520 M
63 0 V
2837 0 R
-63 0 V
550 740 M
63 0 V
2837 0 R
-63 0 V
550 960 M
63 0 V
2837 0 R
-63 0 V
550 1180 M
63 0 V
2837 0 R
-63 0 V
550 1400 M
63 0 V
2837 0 R
-63 0 V
550 1620 M
63 0 V
2837 0 R
-63 0 V
550 1840 M
63 0 V
2837 0 R
-63 0 V
550 2060 M
63 0 V
2837 0 R
-63 0 V
550 300 M
0 63 V
0 1697 R
0 -63 V
1275 300 M
0 63 V
0 1697 R
0 -63 V
2000 300 M
0 63 V
0 1697 R
0 -63 V
2725 300 M
0 63 V
0 1697 R
0 -63 V
3450 300 M
0 63 V
0 1697 R
0 -63 V
1.000 UL
LTb
550 300 M
2900 0 V
0 1760 V
-2900 0 V
550 300 L
1.000 UL
LT0
3087 1947 M
263 0 V
550 1620 M
29 -309 V
609 818 L
29 16 V
29 503 V
29 282 V
30 -335 V
755 803 L
29 49 V
30 511 V
29 253 V
29 -360 V
902 789 L
29 82 V
29 517 V
29 224 V
30 -383 V
29 -452 V
29 115 V
30 520 V
29 194 V
29 -405 V
29 -434 V
30 147 V
29 521 V
29 163 V
30 -425 V
29 -415 V
29 178 V
29 521 V
30 131 V
29 -443 V
29 -394 V
30 209 V
29 519 V
29 98 V
30 -460 V
29 -371 V
29 239 V
29 513 V
30 66 V
29 -474 V
29 -348 V
30 268 V
29 507 V
29 33 V
29 -487 V
30 -323 V
29 296 V
29 498 V
30 0 V
29 -498 V
29 -296 V
30 323 V
29 487 V
29 -33 V
29 -507 V
30 -268 V
29 348 V
29 474 V
30 -66 V
29 -513 V
29 -239 V
29 371 V
30 460 V
29 -98 V
29 -519 V
30 -209 V
29 394 V
29 443 V
30 -131 V
29 -521 V
29 -178 V
29 415 V
30 425 V
29 -163 V
29 -521 V
30 -147 V
29 434 V
29 405 V
29 -194 V
30 -520 V
29 -115 V
29 452 V
30 383 V
29 -224 V
29 -517 V
29 -82 V
30 467 V
29 360 V
29 -253 V
30 -511 V
29 -49 V
29 481 V
30 335 V
29 -282 V
29 -503 V
29 -16 V
30 493 V
29 309 V
stroke
grestore
end
showpage
}}%
\put(3037,1947){\makebox(0,0)[r]{$\psi$ for $\Delta=0.001$}}%
\put(2000,50){\makebox(0,0){$x$}}%
\put(100,1180){%
\special{ps: gsave currentpoint currentpoint translate
270 rotate neg exch neg exch translate}%
\makebox(0,0)[b]{\shortstack{$\psi(x)$}}%
\special{ps: currentpoint grestore moveto}%
}%
\put(3450,200){\makebox(0,0){10}}%
\put(2725,200){\makebox(0,0){5}}%
\put(2000,200){\makebox(0,0){0}}%
\put(1275,200){\makebox(0,0){-5}}%
\put(550,200){\makebox(0,0){-10}}%
\put(500,2060){\makebox(0,0)[r]{0.004}}%
\put(500,1840){\makebox(0,0)[r]{0.003}}%
\put(500,1620){\makebox(0,0)[r]{0.002}}%
\put(500,1400){\makebox(0,0)[r]{0.001}}%
\put(500,1180){\makebox(0,0)[r]{0}}%
\put(500,960){\makebox(0,0)[r]{-0.001}}%
\put(500,740){\makebox(0,0)[r]{-0.002}}%
\put(500,520){\makebox(0,0)[r]{-0.003}}%
\put(500,300){\makebox(0,0)[r]{-0.004}}%
\end{picture}%
\endgroup
\endinput

% GNUPLOT: LaTeX picture with Postscript
\begingroup%
  \makeatletter%
  \newcommand{\GNUPLOTspecial}{%
    \@sanitize\catcode`\%=14\relax\special}%
  \setlength{\unitlength}{0.1bp}%
{\GNUPLOTspecial{!
%!PS-Adobe-2.0 EPSF-2.0
%%Title: fourier2.tex
%%Creator: gnuplot 3.7 patchlevel 1
%%CreationDate: Thu Feb 15 20:23:41 2001
%%DocumentFonts: 
%%BoundingBox: 0 0 360 216
%%Orientation: Landscape
%%EndComments
/gnudict 256 dict def
gnudict begin
/Color false def
/Solid false def
/gnulinewidth 5.000 def
/userlinewidth gnulinewidth def
/vshift -33 def
/dl {10 mul} def
/hpt_ 31.5 def
/vpt_ 31.5 def
/hpt hpt_ def
/vpt vpt_ def
/M {moveto} bind def
/L {lineto} bind def
/R {rmoveto} bind def
/V {rlineto} bind def
/vpt2 vpt 2 mul def
/hpt2 hpt 2 mul def
/Lshow { currentpoint stroke M
  0 vshift R show } def
/Rshow { currentpoint stroke M
  dup stringwidth pop neg vshift R show } def
/Cshow { currentpoint stroke M
  dup stringwidth pop -2 div vshift R show } def
/UP { dup vpt_ mul /vpt exch def hpt_ mul /hpt exch def
  /hpt2 hpt 2 mul def /vpt2 vpt 2 mul def } def
/DL { Color {setrgbcolor Solid {pop []} if 0 setdash }
 {pop pop pop Solid {pop []} if 0 setdash} ifelse } def
/BL { stroke userlinewidth 2 mul setlinewidth } def
/AL { stroke userlinewidth 2 div setlinewidth } def
/UL { dup gnulinewidth mul /userlinewidth exch def
      10 mul /udl exch def } def
/PL { stroke userlinewidth setlinewidth } def
/LTb { BL [] 0 0 0 DL } def
/LTa { AL [1 udl mul 2 udl mul] 0 setdash 0 0 0 setrgbcolor } def
/LT0 { PL [] 1 0 0 DL } def
/LT1 { PL [4 dl 2 dl] 0 1 0 DL } def
/LT2 { PL [2 dl 3 dl] 0 0 1 DL } def
/LT3 { PL [1 dl 1.5 dl] 1 0 1 DL } def
/LT4 { PL [5 dl 2 dl 1 dl 2 dl] 0 1 1 DL } def
/LT5 { PL [4 dl 3 dl 1 dl 3 dl] 1 1 0 DL } def
/LT6 { PL [2 dl 2 dl 2 dl 4 dl] 0 0 0 DL } def
/LT7 { PL [2 dl 2 dl 2 dl 2 dl 2 dl 4 dl] 1 0.3 0 DL } def
/LT8 { PL [2 dl 2 dl 2 dl 2 dl 2 dl 2 dl 2 dl 4 dl] 0.5 0.5 0.5 DL } def
/Pnt { stroke [] 0 setdash
   gsave 1 setlinecap M 0 0 V stroke grestore } def
/Dia { stroke [] 0 setdash 2 copy vpt add M
  hpt neg vpt neg V hpt vpt neg V
  hpt vpt V hpt neg vpt V closepath stroke
  Pnt } def
/Pls { stroke [] 0 setdash vpt sub M 0 vpt2 V
  currentpoint stroke M
  hpt neg vpt neg R hpt2 0 V stroke
  } def
/Box { stroke [] 0 setdash 2 copy exch hpt sub exch vpt add M
  0 vpt2 neg V hpt2 0 V 0 vpt2 V
  hpt2 neg 0 V closepath stroke
  Pnt } def
/Crs { stroke [] 0 setdash exch hpt sub exch vpt add M
  hpt2 vpt2 neg V currentpoint stroke M
  hpt2 neg 0 R hpt2 vpt2 V stroke } def
/TriU { stroke [] 0 setdash 2 copy vpt 1.12 mul add M
  hpt neg vpt -1.62 mul V
  hpt 2 mul 0 V
  hpt neg vpt 1.62 mul V closepath stroke
  Pnt  } def
/Star { 2 copy Pls Crs } def
/BoxF { stroke [] 0 setdash exch hpt sub exch vpt add M
  0 vpt2 neg V  hpt2 0 V  0 vpt2 V
  hpt2 neg 0 V  closepath fill } def
/TriUF { stroke [] 0 setdash vpt 1.12 mul add M
  hpt neg vpt -1.62 mul V
  hpt 2 mul 0 V
  hpt neg vpt 1.62 mul V closepath fill } def
/TriD { stroke [] 0 setdash 2 copy vpt 1.12 mul sub M
  hpt neg vpt 1.62 mul V
  hpt 2 mul 0 V
  hpt neg vpt -1.62 mul V closepath stroke
  Pnt  } def
/TriDF { stroke [] 0 setdash vpt 1.12 mul sub M
  hpt neg vpt 1.62 mul V
  hpt 2 mul 0 V
  hpt neg vpt -1.62 mul V closepath fill} def
/DiaF { stroke [] 0 setdash vpt add M
  hpt neg vpt neg V hpt vpt neg V
  hpt vpt V hpt neg vpt V closepath fill } def
/Pent { stroke [] 0 setdash 2 copy gsave
  translate 0 hpt M 4 {72 rotate 0 hpt L} repeat
  closepath stroke grestore Pnt } def
/PentF { stroke [] 0 setdash gsave
  translate 0 hpt M 4 {72 rotate 0 hpt L} repeat
  closepath fill grestore } def
/Circle { stroke [] 0 setdash 2 copy
  hpt 0 360 arc stroke Pnt } def
/CircleF { stroke [] 0 setdash hpt 0 360 arc fill } def
/C0 { BL [] 0 setdash 2 copy moveto vpt 90 450  arc } bind def
/C1 { BL [] 0 setdash 2 copy        moveto
       2 copy  vpt 0 90 arc closepath fill
               vpt 0 360 arc closepath } bind def
/C2 { BL [] 0 setdash 2 copy moveto
       2 copy  vpt 90 180 arc closepath fill
               vpt 0 360 arc closepath } bind def
/C3 { BL [] 0 setdash 2 copy moveto
       2 copy  vpt 0 180 arc closepath fill
               vpt 0 360 arc closepath } bind def
/C4 { BL [] 0 setdash 2 copy moveto
       2 copy  vpt 180 270 arc closepath fill
               vpt 0 360 arc closepath } bind def
/C5 { BL [] 0 setdash 2 copy moveto
       2 copy  vpt 0 90 arc
       2 copy moveto
       2 copy  vpt 180 270 arc closepath fill
               vpt 0 360 arc } bind def
/C6 { BL [] 0 setdash 2 copy moveto
      2 copy  vpt 90 270 arc closepath fill
              vpt 0 360 arc closepath } bind def
/C7 { BL [] 0 setdash 2 copy moveto
      2 copy  vpt 0 270 arc closepath fill
              vpt 0 360 arc closepath } bind def
/C8 { BL [] 0 setdash 2 copy moveto
      2 copy vpt 270 360 arc closepath fill
              vpt 0 360 arc closepath } bind def
/C9 { BL [] 0 setdash 2 copy moveto
      2 copy  vpt 270 450 arc closepath fill
              vpt 0 360 arc closepath } bind def
/C10 { BL [] 0 setdash 2 copy 2 copy moveto vpt 270 360 arc closepath fill
       2 copy moveto
       2 copy vpt 90 180 arc closepath fill
               vpt 0 360 arc closepath } bind def
/C11 { BL [] 0 setdash 2 copy moveto
       2 copy  vpt 0 180 arc closepath fill
       2 copy moveto
       2 copy  vpt 270 360 arc closepath fill
               vpt 0 360 arc closepath } bind def
/C12 { BL [] 0 setdash 2 copy moveto
       2 copy  vpt 180 360 arc closepath fill
               vpt 0 360 arc closepath } bind def
/C13 { BL [] 0 setdash  2 copy moveto
       2 copy  vpt 0 90 arc closepath fill
       2 copy moveto
       2 copy  vpt 180 360 arc closepath fill
               vpt 0 360 arc closepath } bind def
/C14 { BL [] 0 setdash 2 copy moveto
       2 copy  vpt 90 360 arc closepath fill
               vpt 0 360 arc } bind def
/C15 { BL [] 0 setdash 2 copy vpt 0 360 arc closepath fill
               vpt 0 360 arc closepath } bind def
/Rec   { newpath 4 2 roll moveto 1 index 0 rlineto 0 exch rlineto
       neg 0 rlineto closepath } bind def
/Square { dup Rec } bind def
/Bsquare { vpt sub exch vpt sub exch vpt2 Square } bind def
/S0 { BL [] 0 setdash 2 copy moveto 0 vpt rlineto BL Bsquare } bind def
/S1 { BL [] 0 setdash 2 copy vpt Square fill Bsquare } bind def
/S2 { BL [] 0 setdash 2 copy exch vpt sub exch vpt Square fill Bsquare } bind def
/S3 { BL [] 0 setdash 2 copy exch vpt sub exch vpt2 vpt Rec fill Bsquare } bind def
/S4 { BL [] 0 setdash 2 copy exch vpt sub exch vpt sub vpt Square fill Bsquare } bind def
/S5 { BL [] 0 setdash 2 copy 2 copy vpt Square fill
       exch vpt sub exch vpt sub vpt Square fill Bsquare } bind def
/S6 { BL [] 0 setdash 2 copy exch vpt sub exch vpt sub vpt vpt2 Rec fill Bsquare } bind def
/S7 { BL [] 0 setdash 2 copy exch vpt sub exch vpt sub vpt vpt2 Rec fill
       2 copy vpt Square fill
       Bsquare } bind def
/S8 { BL [] 0 setdash 2 copy vpt sub vpt Square fill Bsquare } bind def
/S9 { BL [] 0 setdash 2 copy vpt sub vpt vpt2 Rec fill Bsquare } bind def
/S10 { BL [] 0 setdash 2 copy vpt sub vpt Square fill 2 copy exch vpt sub exch vpt Square fill
       Bsquare } bind def
/S11 { BL [] 0 setdash 2 copy vpt sub vpt Square fill 2 copy exch vpt sub exch vpt2 vpt Rec fill
       Bsquare } bind def
/S12 { BL [] 0 setdash 2 copy exch vpt sub exch vpt sub vpt2 vpt Rec fill Bsquare } bind def
/S13 { BL [] 0 setdash 2 copy exch vpt sub exch vpt sub vpt2 vpt Rec fill
       2 copy vpt Square fill Bsquare } bind def
/S14 { BL [] 0 setdash 2 copy exch vpt sub exch vpt sub vpt2 vpt Rec fill
       2 copy exch vpt sub exch vpt Square fill Bsquare } bind def
/S15 { BL [] 0 setdash 2 copy Bsquare fill Bsquare } bind def
/D0 { gsave translate 45 rotate 0 0 S0 stroke grestore } bind def
/D1 { gsave translate 45 rotate 0 0 S1 stroke grestore } bind def
/D2 { gsave translate 45 rotate 0 0 S2 stroke grestore } bind def
/D3 { gsave translate 45 rotate 0 0 S3 stroke grestore } bind def
/D4 { gsave translate 45 rotate 0 0 S4 stroke grestore } bind def
/D5 { gsave translate 45 rotate 0 0 S5 stroke grestore } bind def
/D6 { gsave translate 45 rotate 0 0 S6 stroke grestore } bind def
/D7 { gsave translate 45 rotate 0 0 S7 stroke grestore } bind def
/D8 { gsave translate 45 rotate 0 0 S8 stroke grestore } bind def
/D9 { gsave translate 45 rotate 0 0 S9 stroke grestore } bind def
/D10 { gsave translate 45 rotate 0 0 S10 stroke grestore } bind def
/D11 { gsave translate 45 rotate 0 0 S11 stroke grestore } bind def
/D12 { gsave translate 45 rotate 0 0 S12 stroke grestore } bind def
/D13 { gsave translate 45 rotate 0 0 S13 stroke grestore } bind def
/D14 { gsave translate 45 rotate 0 0 S14 stroke grestore } bind def
/D15 { gsave translate 45 rotate 0 0 S15 stroke grestore } bind def
/DiaE { stroke [] 0 setdash vpt add M
  hpt neg vpt neg V hpt vpt neg V
  hpt vpt V hpt neg vpt V closepath stroke } def
/BoxE { stroke [] 0 setdash exch hpt sub exch vpt add M
  0 vpt2 neg V hpt2 0 V 0 vpt2 V
  hpt2 neg 0 V closepath stroke } def
/TriUE { stroke [] 0 setdash vpt 1.12 mul add M
  hpt neg vpt -1.62 mul V
  hpt 2 mul 0 V
  hpt neg vpt 1.62 mul V closepath stroke } def
/TriDE { stroke [] 0 setdash vpt 1.12 mul sub M
  hpt neg vpt 1.62 mul V
  hpt 2 mul 0 V
  hpt neg vpt -1.62 mul V closepath stroke } def
/PentE { stroke [] 0 setdash gsave
  translate 0 hpt M 4 {72 rotate 0 hpt L} repeat
  closepath stroke grestore } def
/CircE { stroke [] 0 setdash 
  hpt 0 360 arc stroke } def
/Opaque { gsave closepath 1 setgray fill grestore 0 setgray closepath } def
/DiaW { stroke [] 0 setdash vpt add M
  hpt neg vpt neg V hpt vpt neg V
  hpt vpt V hpt neg vpt V Opaque stroke } def
/BoxW { stroke [] 0 setdash exch hpt sub exch vpt add M
  0 vpt2 neg V hpt2 0 V 0 vpt2 V
  hpt2 neg 0 V Opaque stroke } def
/TriUW { stroke [] 0 setdash vpt 1.12 mul add M
  hpt neg vpt -1.62 mul V
  hpt 2 mul 0 V
  hpt neg vpt 1.62 mul V Opaque stroke } def
/TriDW { stroke [] 0 setdash vpt 1.12 mul sub M
  hpt neg vpt 1.62 mul V
  hpt 2 mul 0 V
  hpt neg vpt -1.62 mul V Opaque stroke } def
/PentW { stroke [] 0 setdash gsave
  translate 0 hpt M 4 {72 rotate 0 hpt L} repeat
  Opaque stroke grestore } def
/CircW { stroke [] 0 setdash 
  hpt 0 360 arc Opaque stroke } def
/BoxFill { gsave Rec 1 setgray fill grestore } def
end
%%EndProlog
}}%
\begin{picture}(3600,2160)(0,0)%
{\GNUPLOTspecial{"
gnudict begin
gsave
0 0 translate
0.100 0.100 scale
0 setgray
newpath
1.000 UL
LTb
450 300 M
63 0 V
2937 0 R
-63 0 V
450 476 M
63 0 V
2937 0 R
-63 0 V
450 652 M
63 0 V
2937 0 R
-63 0 V
450 828 M
63 0 V
2937 0 R
-63 0 V
450 1004 M
63 0 V
2937 0 R
-63 0 V
450 1180 M
63 0 V
2937 0 R
-63 0 V
450 1356 M
63 0 V
2937 0 R
-63 0 V
450 1532 M
63 0 V
2937 0 R
-63 0 V
450 1708 M
63 0 V
2937 0 R
-63 0 V
450 1884 M
63 0 V
2937 0 R
-63 0 V
450 2060 M
63 0 V
2937 0 R
-63 0 V
450 300 M
0 63 V
0 1697 R
0 -63 V
1200 300 M
0 63 V
0 1697 R
0 -63 V
1950 300 M
0 63 V
0 1697 R
0 -63 V
2700 300 M
0 63 V
0 1697 R
0 -63 V
3450 300 M
0 63 V
0 1697 R
0 -63 V
1.000 UL
LTb
450 300 M
3000 0 V
0 1760 V
-3000 0 V
450 300 L
1.000 UL
LT0
3087 1947 M
263 0 V
450 652 M
30 -8 V
31 22 V
30 -28 V
30 24 V
31 -12 V
30 -5 V
30 21 V
30 -31 V
31 30 V
30 -17 V
30 -2 V
31 21 V
30 -34 V
30 35 V
31 -23 V
30 2 V
30 21 V
30 -38 V
31 42 V
30 -31 V
30 9 V
31 18 V
30 -41 V
30 51 V
31 -43 V
30 18 V
30 16 V
30 -47 V
31 63 V
30 -58 V
30 31 V
31 11 V
30 -53 V
30 81 V
31 -82 V
30 51 V
30 3 V
31 -65 V
30 114 V
30 -128 V
30 95 V
31 -16 V
30 -93 V
30 200 V
31 -264 V
30 242 V
30 -84 V
31 -304 V
30 1610 V
30 0 V
1995 366 L
31 304 V
30 84 V
30 -242 V
31 264 V
30 -200 V
30 93 V
31 16 V
30 -95 V
30 128 V
30 -114 V
31 65 V
30 -3 V
30 -51 V
31 82 V
30 -81 V
30 53 V
31 -11 V
30 -31 V
30 58 V
31 -63 V
30 47 V
30 -16 V
30 -18 V
31 43 V
30 -51 V
30 41 V
31 -18 V
30 -9 V
30 31 V
31 -42 V
30 38 V
30 -21 V
30 -2 V
31 23 V
30 -35 V
30 34 V
31 -21 V
30 2 V
30 17 V
31 -30 V
30 31 V
30 -21 V
30 5 V
31 12 V
30 -24 V
30 28 V
31 -22 V
30 8 V
stroke
grestore
end
showpage
}}%
\put(3037,1947){\makebox(0,0)[r]{$\psi$ for $\Delta=100$}}%
\put(1950,50){\makebox(0,0){$x$}}%
\put(100,1180){%
\special{ps: gsave currentpoint currentpoint translate
270 rotate neg exch neg exch translate}%
\makebox(0,0)[b]{\shortstack{$\psi(x)$}}%
\special{ps: currentpoint grestore moveto}%
}%
\put(3450,200){\makebox(0,0){10}}%
\put(2700,200){\makebox(0,0){5}}%
\put(1950,200){\makebox(0,0){0}}%
\put(1200,200){\makebox(0,0){-5}}%
\put(450,200){\makebox(0,0){-10}}%
\put(400,2060){\makebox(0,0)[r]{1.6}}%
\put(400,1884){\makebox(0,0)[r]{1.4}}%
\put(400,1708){\makebox(0,0)[r]{1.2}}%
\put(400,1532){\makebox(0,0)[r]{1}}%
\put(400,1356){\makebox(0,0)[r]{0.8}}%
\put(400,1180){\makebox(0,0)[r]{0.6}}%
\put(400,1004){\makebox(0,0)[r]{0.4}}%
\put(400,828){\makebox(0,0)[r]{0.2}}%
\put(400,652){\makebox(0,0)[r]{0}}%
\put(400,476){\makebox(0,0)[r]{-0.2}}%
\put(400,300){\makebox(0,0)[r]{-0.4}}%
\end{picture}%
\endgroup
\endinput

\caption{Eksempel p\aa\ b\o lgefunksjon $\psi$ for ulike verdier av
$\Delta$.} 
\end{figure}

Vi legger merke til f\o lgende:
\begin{itemize}
\item Dersom vi gj\o r $\Delta$ liten, som betyr at vi bestemmer 
$k$ skarpt n\ae r $k_0$, resulterer dette i en b\o lgefunksjon 
$\psi(x) \propto cos(2\pi k_0x)$. Dette vil svare til v\aa r harmoniske
svingning i Figur \ref{fig:fasev} og en b\o lge av uendelig utstrekning. Integrerer 
vi over f\aa\ b\o lgetall, f\aa r vi alts\aa\ en b\o lge sentrert rundt
kun et b\o lgetall. Vi kan si at sinus-delen svarer til den  modulerte
b\o lgefunskjonen, mens cosinus delen representer de enkelte b\o lgene som 
settes sammen.  
\item \O nsker vi derimot en b\o lge som er sterkt lokalisert i rom og null
eller liten for de fleste verdier av $x$, m\aa\ vi \o ke intervallet over
$k$-verdier i integrasjonen v\aa r. 
\item Det sistnevnte gir oss den n\o dvendige koplingen til v\aa r diskusjon
om gruppehastighet og fasehastighet og en b\o lgepakkes utbredelse.
\item Dersom vi n\aa\ \o nsker \aa\ kople dette resultatet med fysikk
      har vi at siden
      \[ 
         k=\frac{2\pi}{\lambda}=\frac{h2\pi}{h\lambda}=\frac{p2\pi}{h}
      \]
eller 
       \[
          \hbar k=p
       \]
s\aa\ betyr det at dersom vi \o nsker \aa\ fiksere bevegelsesmengden skarpt,
dvs.~$\Delta \rightarrow 0$, resulterer det i en b\o lgefunksjon av uendelig
utbredelse. \O nsker vi at denne b\o lgefunksjonen skal representere en
partikkel, betyr det igjen at vi ikke kan fiksere posisjon og bevegelsesmengde
skarpt samtidig. Tilsvarende, \o nsker vi \aa\ fiksere posisjonen skarpt,
finner vi at uskarpheten til partikkelen \o ker.
\end{itemize}
 
Det siste leder oss til neste avsnitt om Heisenbergs 
      uskarphetsrelasjon.

\section{Heisenbergs uskarphetsrelasjon}

La oss f\o rst anvende\footnote{Lesehenvisning her er kapitlene 
4-3, 4-4 og 4-5, sidene 188-205. To sm\aa\ oppfordringer: 
den f\o rste med et snev av moralisme i seg. Jeg tar meg nemlig den frihet 
\aa\ anbefale sterkt at dere
leser disse avsnittene i boka, da de danner et viktig grunnlag for bruddet
med klassisk fysikk og introduksjonen av kvantemekanikken som teori. 
Vi kommer ikke til \aa\ g\aa\ noe i dybden p\aa\ de filosofiske aspektene
i disse notatene. Derfor den andre oppfordringen. Jeg setter pris p\aa\
dersom dere tar disse emnene opp til diskusjon enten under forelesningene
eller under oppgavel\o sning.} 
resultatet fra avsnittet om diffraksjon
p\aa\ en innkommende b\o lge av elektroner med b\o lgelengde $\lambda$
mot en spalte\aa pning med st\o rrelse $d=\Delta y$.
Et eksempel p\aa\ et slikt oppsett er vist i Figur \ref{fig:diffex}.

N\aa r monokromatiske b\o lger med b\o lgelengde
$\lambda$ passerer \aa pningen vil det dannes et diffraksjonsm\o nster p\aa\
skjermen. 
Det innfallende elektron har bevegelsesmengde kun i $x$-retningen f\o r det
treffer spalte\aa pningen. 
F\o rste punkt hvor vi har destruktiv interferens er gitt ved  
\be 
   sin\theta=\frac{\lambda}{d}=\frac{\lambda}{\Delta y}=\frac{h}{p\Delta y},
\ee
siden $\lambda=h/p$.  
Etter at elektronet har passert \aa pningen vil det ogs\aa\ ha en bevegelsemengde i 
$y$-retningen som vi ikke kjenner.
Vi veit ikke med n\o yaktighet hvor elektronet  vil treffe skjermen, det eneste
vi kan si er at det er en stor sannsynlighet for at det treffer i et omr\aa de 
hvor intensiteten er maksimal. { \bf Her har vi innf\o rt et sannsynlighetsbegrep
som vi skal diskutere n\ae rmere etter dette eksemplet.}

Vi kan anta at $y$-komponenten til bevegelsesmengden har en verdi mellom
$0$ og $psin\theta$, dvs. 
\be
   \Delta p_y=psin\theta=\frac{h\lambda}{\lambda \Delta y}=\frac{h}{\Delta y},
\ee
som gir en uskarphet 
\be
    \Delta p_y\Delta y=h,
\ee
som p\aa\ en konstant n\ae r ($h=2\pi\hbar$) er gitt ved Heisenbergs uskaperhetsrelasjon
\be
    \Delta p_y\Delta y \geq \frac{\hbar}{2},
    \label{eq:heisenberg}
\ee
eller mer generelt
\be
    \Delta {\bf p}\Delta {\bf x} \geq \frac{\hbar}{2},
    \label{eq:heisenberg}
\ee
Dette uttrykket skal vi utlede senere, under kapitlet om
kvantemekanikkens formelle grunnlag.

Det vi skal merke oss her er at dette resultatet som kun er basert p\aa\ 
b\o lgel\ae re forteller oss at dersom vi \o nsker \aa\ lokalisere
skarpt p\aa\ skjermen hvor elektronet treffer, dvs.~at dersom vi setter 
$\Delta y=0$ s\aa\ vil
$\Delta p_y$ divergere. Vi kan alts\aa\ ikke fiksere skarpt b\aa de posisjon
og bevegelsesmengde. Vi merker oss dog at dersom $h=0$, ville vi ikke hatt noe slikt
problem.   
\begin{figure}[h]
   \setlength{\unitlength}{1mm}
   \begin{picture}(100,60)
   \put(25,0){\epsfxsize=10cm \epsfbox{fig3.eps}}
   \end{picture}
\caption{Diffraksjon for materieb\o lge med b\o lgelengde $\lambda$  som sendes mot en spalte\aa pning. \label{fig:diffex}} 
\end{figure}
Klassisk er det slik at ${\bf x}$ og ${\bf p}$ er uavhengige st\o rrelser.
Kvantemekanisk derimot, siden $h\ne 0$, s\aa\ er avhengighets forholdet gitt ved
Heisenbergs uskarphets relasjon. Dette impliserer igjen at vi ikke kan 
lokalisere en partikkel og samtidig bestemme dens bevegelsesmengde skarpt. 
En ytterligere konsekvens er at vi ikke kan i et bestemt eksperiment observere
b\aa de partikkel og b\o lgeegenskaper. Vi skal se et eksempel p\aa\ dette
i diskusjonen lenger nede, men f\o rst noen ord om partikkel-b\o lge dualiteten.

\subsection{Partikkel-B\o lge dualitet}
\label{subsec:pbdualitet}
I det siste eksemplet innf\o rte vi at diffraksjonsm\o nsteret som vi ser
p\aa\ skjermen skal representer en sannsynlighet for at 
elektronet er \aa\ finne et bestemt sted. 
Vi skal n\aa\ anskueliggj\o re dette ved \aa\ se p\aa\ uttrykket
for intensiteten for det elektromagnetiske feltet.

Klassisk, dvs.~fra e.m.~teori og dermed FY101, har vi at
b\o lgefunksjonen til f.eks.~det elektriske feltet er gitt ved
\be
   {\cal E}(x,t)={\cal E}_0sin(kx-\omega t),
\ee
og at  
intensiteten til det e.m.~feltet, dvs.~energi som kommer inn per areal
per sekund, er gitt ved 
\be
    I=c\epsilon_0{\cal E}_0^2,
\ee
hvor ${\cal E}_0$ er amplituden til feltet. 
Et eksempel p\aa\ uttrykk for $I$ er gitt ved likning (\ref{eq:diffint}).
Vi kan si at intensiteten er proporsjonal med b\o lgefunskjonen
kvadrert, dvs.
\be
   I\propto  {\cal E}^2.
\ee

Dette er et resultat som baserer seg kun p\aa\ klassisk b\o lgel\ae re,
dvs.~et reint b\o lgebilde av e.m.~str\aa ling. 

Dersom vi tar utgangspunkt i Einsteins og Plancks kvantseringshypoteser for
det e.m.~felt, s\aa\ tilordner vi partikkel egenskaper til intensiteten.
I dette tilfellet har vi at 
\be
   I=h\nu N_{\gamma},
\ee
hvor $h\nu$ er det velkjente uttrykket for energien til et foton 
mens $N_{\gamma}$ er midlere antall fotoner per areal per sekund som kommer inn
i et omr\aa de. 
Denne st\o rrelsen er igjen et uttrykk for 
en sannsynlighet for \aa\ finne et gitt antall 
fotoner i et bestemt omr\aa de. {\bf Det er dette begrepet vi n\aa\ skal 
kople til det klassiske resultatet for $I$.}    

Dersom vi n\aa\ g\aa r tilbake til diffraksjonsm\o nsteret i likning
(\ref{eq:diffint}) kan vi kople til  
interferensmaksima et begrep om at det er der flest fotoner treffer plata. 

Interferensmaksima er kopla til b\o lgeegenskapen til fotonene, men n\aa r vi
setter
\be 
   I=h\nu N_{\gamma}=c\epsilon_0{\cal E}_0^2,
\ee
s\aa\ relaterer vi partikkelegenskaper til b\o lgeegenskaper og vi sier at
relasjonen
\be 
   N_{\gamma}\propto {\cal E}_0^2
\ee
skal uttrykke et forhold mellom en intensitestfordeling fra b\o lgeoppf\o rsel
og en sannsynlighet for \aa\ detektere fotoner.

Mere generelt, n\aa r vi erstatter det elektriske felt med 
b\o lgefunskjonen $\psi$, s\aa\ skal st\o rrelsen 
\be
    |\psi(x,t)|^2dx,
\ee
uttrykke sannsynligheten for \aa\ finne en partikkel 
(hvis bevegelseslikninger er
beskrevet av en b\o lge $\psi$) i et omr\aa de mellom
$x$ og $x+dx$. 
Det er denne tolkningen som danner basis for kvantemekanikken. 



\subsubsection{To spalte\aa pninger}


Anta at vi har to spalte\aa pninger som vi sender elektroner mot.
Avstanden mellom spalte\aa pningene kaller vi igjen for $a$.
Intensiteten ved skjermen var da gitt ved likning (\ref{eq:diffint}), som
igjen fortalte oss  
at vi kan ha  konstruktiv interferens n\aa r betingelsen
\be
    sin\theta_n=n\frac{\lambda}{a},
\ee
er oppfylt med gitt
b\o lgelengden $\lambda$. Dersom vi kaller
avstanden fra spalte\aa pningene til skjermen hvor vi observerer et
diffraksjonsm\o nster for $b$, s\aa\ har vi at avstanden mellom
to interferensmaksima er gitt ved
\be
    b(sin\theta_n-sin\theta_{n-1})=\frac{b\lambda}{a}.
    \label{eq:intensmaks}
\ee 
La oss anta at vi krever at vi er istand til \aa\ m\aa le elektronets
posisjon med en presisjon som er mindre en halvparten av avstanden mellom
spaltene, dvs.\
\be
    \Delta y < \frac{a}{2}.
\ee
Vi kan f.eks.~anta at vi har en eller annen monitor rett ved skjermen 
som forteller oss hvilken
spalte\aa pning elektronet passerte. Ved bruk av en slik monitor tilf\o res
elektronet en bevegelsesmengde i $y$-retningen (parallellt med skjermen) 
hvis st\o rrelse er uskarp og gitt ved
\be
    \Delta p_y > \frac{2h}{a}.
\ee
En slik uskarphet i bevegelsesmengde resulterer i en relativ uskarphet
\be
    \frac{\Delta p_y}{p}>\frac{2h}{ap}=\frac{2\lambda}{a}.
\ee
En slik uskarphet medf\o rer en uskarphet i elektronets posisjon p\aa\
skjermen som ihvertfall er gitt ved 
\be 
   \frac{2\lambda b}{a},
\ee 
som er st\o rre enn avstanden mellom interferensmaksima i likning
(\ref{eq:intensmaks}).
Det betyr igjen at dersom vi \o nsker \aa\ m\aa le elektronets posisjon presist,
vil vi komme til \aa\ \o delegge interferensm\o nsteret. 
Vi kan ikke 
observere  fra et enkeltst\aa ende eksperiment b\aa de b\o lge og partikkelegenskapene til materie,
eller e.m.~str\aa ling for den saks skyld.
B\o lge og partikkelegenskapene er alts\aa\ 
to komplent\ae re egenskaper ved materien. 

Dersom vi \o nsker \aa\ formulere matematisk konsekvensene av det siste
eksemplet s\aa\ m\aa\ vi ha 
\be
   \Delta p_y\Delta y > h.
\ee



\subsection{St\o rrelsesorden estimater, leik med uskarphetsrelasjonen}

Hensikten med dette avsnittet er \aa\ vise
at n\aa r vi kjenner til formen for energien et system har, s\aa\ kan vi bruke
uskarphetsrelasjonen til \aa\ estimere st\o rrelser slik som
bindingsenergien til et elektron i et hydrogenatom, eller hvor stor
massen til en n\o ytronstjerne er!

 
\subsubsection{``Realiteten'' til Bohrs orbitaler}

Vi s\aa\ i forrige kapittel at Bohrs atommodell ga oss
en definisjon p\aa\ Bohrradiene gitt ved
\be
    r_n=\frac{\hbar n^2}{\alpha m_ec},
\ee
hvor $\alpha$ er finstrukturkonstanten. 
Anta at vi er interessert i \aa\ m\aa le hvor elektronet
befinner seg i et atom. Vi krever at n\o yaktigheten er slik at 
\be
   \Delta x << r_n-r_{n-1}=\frac{\hbar( 2n-1)}{\alpha m_ec}, 
\ee
for at vi skal kunne fastsl\aa\ med rimelig n\o yaktighet
at et elektron er i en  bestemt orbital med kvantetall $n$.  
Men dette kravet leder til en uskarphet i bevegelsesmengde\footnote{Her bruker vi det eksakte uttrykket for 
Heisenbergs uskarphetsrelasjon gitt i likning
(\ref{eq:heisenberg}).} 
 gitt ved
\be
   \Delta p >> \frac{\alpha m_ec}{2(2n-1)}.
\ee
Dersom vi bruker resultatet fra Bohrs atommodell for hastigheten til et 
elektron i en gitt bane, $v_n$, har vi en bevegelsesmengde
\be
    p=m_ev_n=\frac{\alpha cm_e}{n}
\ee
som er p\aa\ st\o rrelse med $\Delta p$!
Vi kan s\aa\ pr\o ve \aa\ regne
ut uskarpheten i energien til elektronet. Da trenger vi det ikke-relativistiske
uttrykket for $E=p^2/2m_e$, som gir at
\be
     \Delta E=\frac{p\Delta p}{m_e},
\ee
som gir 
\be
    \Delta E =\frac{p\Delta p}{m_e} >> \frac{1}{2}\frac{\alpha^2 m_ec^2}{n^2(2n-1)}
    =13.6\frac{1}{2n^2-1},
\ee
som betyr at uskarpheten i energien er mye st\o rre enn bindingsenergien
til elektronet.  En slik m\aa ling av elektronet vil sparke 
elektronet ut av sin orbital, dvs.~vi er ikke i stand til \aa\ m\aa le
hvor det befinner seg i atomet.

\subsubsection{Bindingsenergien til elektronet i hydrogenatomet}

La oss anta at elektronet er i sin grunntilstand i hydrogenatomet.
Vi antar ogs\aa\, basert p\aa\ Heisenbergs uskarphetsrelasjon at vi kan
sette 
\be 
    p\sim \frac{\hbar}{r}.
    \label{eq:happrox}
\ee
Energien til dette elektronet vil v\ae re gitt av summen av kinetisk
og potensiell energi, dvs.
\be
   E=\frac{p^2}{2m_e}-\frac{e^2}{4\pi\epsilon_0r}.
\ee
Setter vi inn uttrykket for $p$ finner vi
\be
   E\sim
\frac{\hbar^2}{2r^2m_e}-\frac{e^2}{4\pi\epsilon_0r}.
\ee
N\aa\ kan vi betrakte $r$ som en variabel som kan varieres. Dersom vi
\o nsker \aa\ finne energiens minimum, trenger vi kun \aa\ finne radien
n\aa  r
\be 
   \frac{dE}{dr}=0.
\ee
Dette gir 
\be
    r=\frac{\hbar^24\pi\epsilon_0}{m_ee^2},
\ee
og setter vi $r$ i uttrykket for energien har vi
\be
    E\sim -\frac{1}{2}\frac{m_ee^4}{(\hbar4\pi\epsilon_0)^2}=-13.6 
    \hspace{0.1cm}\mathrm{eV},
\ee
akkurat den eksperimentelle verdien!

Det kan da ikke stemme at en s\aa\  enkel antagelse som den gitt i
likning (\ref{eq:happrox}) skal kunne gi oss den eksperimentelle
energien til hydrogenatomet. 
Selv om vi var heldige ved valget, ser vi at dersom vi hadde valgt en annen
verdi for forholdet mellom $p$, $h$ og $r$, s\aa\ ville energien avveket bare
p\aa\ en konstant n\ae r. Dvs.~at vi f\aa r ut en energi som har riktig 
st\o rrelsesorden. N\aa r vi kommer til Kap 7 i boka, skal vi faktisk
regne ut $\Delta p$ og $\Delta r$ for hydrogenatomet og se at den
antagelsen vi gjorde faktisk er fornuftig. 
Men, det viktige her er at uskarphetsrelasjonen gir oss en energi som har riktig
st\o rrelsesorden.

Vi skal ikke slippe dette eksemplet helt, fordi det er noen sider ved
uskarphetsrelasjonen og bindingsenergien til et system som er av betydning
for den videre forst\aa else. 
Det f\o rste dere skal bite dere merke i er at uskarphetsrelasjonen begrenser
nedad bindingsenergien, den kan ikke bli uendelig stor i absoluttverdi. 
Det andre er at dersom vi tenker tilbake p\aa\ Bohrs tredje
postulat og ser for oss muligheten for at elektronet skulle deaksellereres og miste
all sin energi, s\aa\ vil det bety at det til slutt ville klappe sammen p\aa\
atomkjernen. N\aa r avstanden til kjernen minsker, kan vi ogs\aa\ anta at
$\Delta x$ minsker. 
Men det betyr igjen at $\Delta p$ \o ker slik at \o kningen i potensiell
energi kompenseres ved en \o kning i kinetisk energi. 
 
\subsubsection{Massen til en n\o ytronstjerne}
N\aa\ har vi f\aa tt blod p\aa\ tann. Vi
kaster  oss derfor 
friskt og freidig 
ut i verdensrommet og p\aa st\aa r at vi kan estimere massen til en
n\o ytronstjerne. Ei n\o ytronstjerne best\aa r i all hovedsak av n\o ytroner,
samt en liten innmiksing av protoner, elektroner, og andre baryoner, og kanskje
muligens en fase av kvarkmaterie i sitt indre, hvor tettheten er ekstremt h\o y.
Her skal vi anta at den best\aa r av bare n\o ytroner.
Radiusen til ei slik stjerne kaller vi
for $R$.  Vi bruker den gjennomsnittlige massen til protoner og n\o ytroner,
og setter massen til n\o ytronet $m_n=938$ MeV/c$^2$.
Vi antar at vi har $N$ n\o ytroner i stjerna, dermed er
antallstettheten $n$ gitt ved
\be
   n\sim \frac{N}{R^3},
\ee
noe som gir et volum per n\o ytron ved $1/n$. Radien til det volumelementet et enkelt
n\o ytron da opptar er tiln\ae rmingsvis gitt 
ved $r\sim 1/n^{1/3}$. Med en slik radius kan vi n\aa\ bruke
Heisenbergs uskarphets relasjon p\aa\ nytt
\be 
    p\sim \frac{\hbar}{r}\sim \hbar n^{1/3}.
\ee
La oss s\aa\ anta at n\o ytronene  v\aa re er ekstremt relativistiske
og at de ikke vekselvirker. Da er den kinetiske energien
til et n\o ytron gitt ved 
\be
    E_{n}
=\sqrt{p^2c^2+m_n^2c^4}\approx pc\sim \hbar n^{1/3}c\sim 
\hbar c\frac{N^{1/3}}{R}.
\ee
Den gravitasjonelle energien per n\o ytron, 
n\aa r den totale massen til stjerna er gitt ved
\be
   M=Nm_n,
\ee
er gitt ved 
\be
    E_G\sim -\frac{GMm_n}{R},
\ee
hvor $G$ er gravitasjonskonstanten.
Energien m\aa\ da v\ae re gitt ved energien til n\o ytronene og den 
gravitasjonelle energien. Vi har da
\be
    E=E_n+E_G=\hbar c\frac{N^{1/3}}{R}-\frac{GNm_n^2}{R}.
\ee
Likevekt oppn\aa s n\aa r energi er et minimum, noe som forteller
om balansen mellom gravitasjonskreftene og den kinetiske energien til
n\o ytronene. Deriverer vi $E$ mhp.~$R$ finner vi 
det totale antall partikler
ved likevekt er gitt ved
\be
    N_{likevekt}\sim \left(\frac{\hbar c}{Gm_n^2}\right)^{3/2}\sim
    2\times 10^{57}.
\ee
Massen ved likevekt er da gitt ved
\be
   M_{likevekt}\sim N_{likevekt}m_n =2\times 10^{57}\dot 1.67 \times 10^{-27}
  \hspace{0.1cm}\mathrm{kg}=3.3 \times 10^{30}\hspace{0.1cm}\mathrm{kg}.
\ee
Massen til sola er $M_{\odot}=1.98 \times 10^{30}$ kg.
Dvs.~at vi finner at massen til v\aa r n\o ytronstjerne er
ca.~$1.66 M_{\odot}$. Massen til n\o ytronstjerner m\aa lt i bin\ae re systemer\footnote{Nobelpris i Fysikk 1993 til Hulse og Taylor.} ligger p\aa\
ca.~$1.4 M_{\odot}$. Ikke verst hva?

\subsubsection{Estimat for kjernekreftenes st\o rrelse}

For elektroner, og dermed for systemer som behandles i atomfysikk, faste  stoffers fysikk
og molekylfysikk, opererer vi med en energi skala p\aa\ noen f\aa\ til kanskje
noen tusen eV. Dette gir igjen lengdeskalaer p\aa\ noen nanometer.

For kjernekreftene,
dvs de kreftene som uttrykker vekselvirkningene mellom f.eks~protoner
og n\o ytroner,  veit vi at rekkevidden til kreftene er bare
noen f\aa\ femtometer, 1 fm = $10^{-15}$ m. Dette impliserer en ny energiskala
som er forskjellig fra den vi kjenner til i Bohrs atommodell. 
For \aa\ se dette kan vi igjen bruke uskarphetrelasjonen p\aa\ forma
\[
   p\sim \frac{\hbar}{r}.
\]
Setter vi $r\sim 1$ fm, f\aa r vi at den kinetiske energien som et proton eller
et n\o ytron kan ha i en atomkjerne blir
\be
   \frac{p^2}{2m_p}\sim\frac{\hbar^2}{2m_pr^2}.
\ee
Setter vi inn $m_p=938$ MeV/c$^2$ og at $\hbar c=197$ MeVfm, finner vi
\be
   \frac{\hbar^2}{2m_pr^2}=\frac{197^2}{2\times 938}\hspace{0.1cm}\mathrm{MeV}=
                          21\hspace{0.1cm}\mathrm{MeV}.
\ee
Det f\o rste vi merker oss er at energien m\aa les n\aa\ i MeV, en million eV.
Av denne grunn kalles ogs\aa\ kjernekreftene for de sterke krefter. 
N\aa\ er det slik, akkurat som for elektroner i atomer, at protoner og
n\o ytroner er bundet sammen i en atomkjerne. Dvs.~at vi ogs\aa\ m\aa\
ta med en potensiell energi, som igjen m\aa\ v\ae re st\o rre (i absoluttverdi)
enn den kinetiske energien til protoner og n\o ytroner for at disse
partiklene skal holde sammen. Dvs.~at den potensielle energien b\o r v\ae re
st\o rre en 21 MeV.   

\subsection{Energi-tid uskarphetsrelasjonen}
Gitt 
\[
       \Delta p_x\Delta x \geq \frac{\hbar}{2},
\]
har vi n\aa r vi bruker  
\[
   \Delta E=\frac{p\Delta p}{m},
\]
og 
\[
   v=\frac{\Delta x}{\Delta t},
\]
sammen med $p=mv$ at
\be
    \frac{p\Delta p}{m}\frac{\Delta xm}{p}=\Delta E\frac{\Delta x}{v}
\ee
som gir oss
\be
           \Delta E\Delta t \geq \frac{\hbar}{2},
\ee
som er den tilsvarende 
energi-tid uskarphetsrelasjonen.


\section{Oppgaver}
\subsection{Analytiske oppgaver}
\subsubsection*{Oppgave 3.1}
I et elektronmikroskop f�r elektronene stor kinetisk energi ved
at de akselereres i et elektrostatisk potensial V.
%
\begin{itemize}
%
\item[a)] Vis at de Broglie b\o lgelengden for et slikt
ikke--relativistisk elektron er gitt ved formelen
\[
\lambda = \frac{1,23}{\sqrt{V}} \; \mbox{nm}
\]
n\aa r V m\aa les i volt.
%
\item[b)] Hva blir den tilsvarende formelen hvis elektronet er
relativistisk?
%
\item[c)] For hvilken spenning V gir den ikke--relativistiske
formelen et svar som er 5 \% feil?
%
\end{itemize}

\subsubsection*{Oppgave 3.2}
En partikkel med ladning $e$ og masse $m_0$
akselereres av en elektrisk potensial $V$
til en relativistisk hastighet.
%
\begin{itemize}
%
\item[a)]Vis at de Broglie b{\o}lgelengden for
partikkelen er gitt ved
\[
\lambda = \frac{h}{\sqrt{2 m_0 e V}}
\left ( 1 + \frac{e V}{2 m_0 c^2}\right )^{-1/2}
\]
%
\item[b)] Vis at dette gir $\lambda = h / p$ i den
ikke--relativiske grense.
%
\item[c)] Vis at for en relativistisk partikkel med hvileenergi
$E_0$ er de Broglie b{\o}lgelengden gitt ved
\[
\lambda = \frac{1,24 \times 10^{-2}}{E_0 (MeV)}
\cdot \frac{\sqrt{(1-\beta^2)}}{\beta}\;\;\mbox{�},
\]
hvor $\beta = v / c$.
\end{itemize}


\subsubsection*{Oppgave 3.3}
Fra optikken vet vi at for \aa ~observere sm\aa ~objekter med
lys, m\aa ~b\o lgelengden maksimalt v\ae re av samme
st\o rrelsesorden som objektets utstrekning.
%
\begin{itemize}
%
\item[a)] Hva er den laveste frekvensen p\aa ~lys som kan
benyttes til \aa~unders\o ke et objekt med radius
0.3 nm i et mikroskop?
\item[b)] Hva er den tilsvarende energien?
%
\item[c)] Hvis man i stedet ville bruke et elektronmikroskop, hva
er da den laveste energien elek\-tronene kan ha for at
partikkelen skal kunne studeres i detalj?
%
\item[d)] Er lys-- eller elektronmikroskop \aa ~foretrekke for
denne typen arbeid?
%
\end{itemize}

\subsubsection*{Oppgave 3.4}
Lyd med frekvens 440 Hz og hastighet 340 $\mbox{ms}^{-1}$ sendes
normalt mot en smal spalte i en vegg. Spalten har en slik bredde at
lydintensiteten har avtatt til det halve i en retning
p\aa ~$45^{\circ}$ fra innfallsvinkelen bak veggen. Hvor
bred er spalten?

\subsubsection*{Oppgave 3.5}
Et fysisk system beskrevet ved hjelp av b\o lgeligninger som
tillater $y = A cos( kx - \omega t)$ som l\o sninger,
der sirkelfrekvensen $\omega$ er en reell
funksjon av b\o lgetallet k, kalles for et line\ae rt, dispersivt system.
Funksjonen $\omega (k)$ kalles  for {\em dispersjonsrelasjonen} til
systemet.
%
\begin{itemize}
%
\item[a)] Vis at dispersjonsrelasjonen for frie, relativistiske
elektronb\o lger er gitt ved

\begin{eqnarray}
\omega (k) = c \cdot \sqrt{ k^{2} + \left( \frac{mc}{\hbar }\right)^{2}},
\nonumber
\end{eqnarray}
der m er elektronets hvilemasse.

\item[b)] Finn et uttrykk for fasehastigheten $v_{f} (k)$ og gruppehastigheten
$v_{g} (k)$ til disse b\o lgene, og vis at produktet $v_{f} (k) \cdot v_{g} (k)$
er en konstant (uavhengig av k).

\item[c)] Fra uttrykket for $v_{f}$ ser vi at $v_{f} > c$ ! Kommenter dette
fenomenet og hva det har \aa ~si for tolkningene av $v_{f}$ og $v_{g}$ ut fra
den spesielle relativitetsteorien.
%
\end{itemize}


\subsubsection*{Oppgave 3.6}
\begin{itemize}
%
\item[a)] En partikkel med masse 1 g har en uskarphet i hastigheten p\aa
~$\Delta v = 10^{-6}\; ms^{-1}$. Hva blir den kvantemekaniske uskarpheten i
posisjonen?

\item[b)] Et elektron med energi 10 keV er lokalisert i et omr\aa de med
utstrekning 0,1 nm. Hva blir uskarpheten i bevegelsesmengden
til elektronet?

\item[c)] Hva blir uskarpheten i elektronets energi?

\item[d)] Et proton i en atomkjerne kan bevege seg i et omr\aa de med en
utstrekning av st{\o}r\-rel\-ses\-or\-den $10^{-15}$ m.  Hvis vi antar at protonet er
"fanget" i et uendelig potensial, hva er den minste kinetiske energien det kan
ha? Ansl\aa ~en st\o rrelsesorden for styrken av potensialet hvis dette likevel kan
antas \aa ~v\ae re endelig.
%
\end{itemize}

\subsubsection*{Oppgave 3.7. Eksamen H-1995}
%
Den relativistiske sammenhengen mellom en partikkels bevegelsesmengde $p$
og energi $E$ er gitt ved 
%
\[
E = \sqrt{E_0^2 + (pc)^2},
\]
%
hvor c er lyshastigheten.
%
\begin{itemize}
%
\item[a)] Forklar hva $E_k = E - E_0$ st{\aa}r for. Hva er $E_0$ og $E_k$ for 
et foton? Finn sammenhengen mellom  b{\o}lgelengden $\lambda$ og 
bevegelsesmengden $p$ for et foton.
%
\item[b)] Gj{\o}r \underline{kort} rede for de Broglie's ideer om materieb{\o}lger
og sett opp uttrykket for de Broglie b{\o}lgelengden for en materiell partikkel 
med bevegelsesmengde  $p$.
%
\item[c)] Finn de Broglie b{\o}lgelengden for et elektron som funksjon av elektronets 
kinetiske energi. Bruk i dette tilfelle det ikke--relativistiske uttrykk for den 
kinetiske energien.
%
\end{itemize}
%
En mono--energetisk elektronstr{\aa}le blir sendt skr{\aa}tt inn mot overflaten av en 
\'{e}n--krystall, og spres i visse retninger fra krystallen. Under behandlingen av et 
slikt spredningsproblem gj{\o}r en bruk av den s{\aa}kalte Braggbetingelsen
%
\[
2 d \sin \theta = n \lambda
\]
%
\begin{itemize}
%
\item[d)] Forklar de st{\o}rrelsene som inng{\aa}r  og lag en 
enkel skisse av det eksperimentelle oppsettet.
%
\end{itemize}
%
\subsubsection*{Kort fasit}
\begin{itemize}
%
\item[a)] $E_k = E - E_0 = \sqrt{E_0^2 + (p c)^2} = $kinetisk energi.
For et foton er $E_0 = \mbox{hvileenergi} = 0,\; E_k = p c, 
          \; \lambda = h / p.$
%
\item[b)] Se l{\ae}reboka {\sl Brehm and Mullin: Introduction to the
structure of matter}, avsnitt 4.1,  $\lambda = h/p$.
%
\item[c)] $E_k = p^2 / 2m \longrightarrow p = \sqrt{2 m E_k}, 
\mbox{innsat i} \lambda = h / p$.
%
\item[d)] Se l{\ae}reboka {\sl Brehm and Mullin: Introduction to the
structure of matter}, avsnitt 2.6
\end{itemize}
\clearemptydoublepage

\clearemptydoublepage
\part{Simple quantum mechanical systems}
	

\chapter{Introduction to quantum physics}
\begin{quotation}
Lasciate ogni speranza, voi che entrate. 
{\em Dante Alighieri, Divina Commedia,
Inferno, canto 3}
\end{quotation}

Hensikten med dette kapitlet  er \aa\ gi dere en introduksjon
til Schr\"odingers likning,
tolkning av b\o lgefunskjonen slik 
Born ga og sakte men sikkert diskutere anvendelser av Schr\"odingers likning  for systemer
hvor den potensielle energien er gitt p\aa\ enkel form slik at
vi kan finne analytiske l\o sninger av Schr\"odingers likning. Slike systemer er 
f.eks.~et harmonisk oscillator potensial eller et s\aa kalla
kassepotensial. Felles for disse systemene er at de utviser
viktige egenskaper ved et kvantemekanisk system slik som
energikvantisering. I tillegg gir disse systemene en brukbar
f\o rste  approksimasjon til virkelige systemer. 
I en slik en forstand, og dette er noe jeg virkelig vil stresse for dere,
dersom dere forst\aa r tiln\ae rmingene som vi gj\o r for 
disse enkle systemene,
og f\aa r et grep om fysikken bak, s\aa\ er min p\aa stand at dere 
langt p\aa\ vei vil n\ae rme dere en bedre forst\aa else av Schr\"odingers likning 
og kvantemekaniske systemer. 

\section{Schr\"odingers likning}
\label{sec:slintro}
I dette avsnittet\footnote{Lesehenvisning er kap 5-1 sidene
220-226.} skal vi sannsynliggj\o re hvordan vi kan komme fram 
til Schr\"odingers likning . 
Schr\"odingers likning  er en bevegelseslikning som baseres seg p\aa\ kreftene som virker p\aa\
en partikkel og som skal gi oss en b\o lgefunksjon til beskrivelsen
av en partikkels ulike egenskaper slik som posisjon, banespinn, energi
etc. 
Schr\"odingers likning kan ikke utledes fra basale prinsipper,
noe som er ogs\aa\ felles for Maxwells likninger. Derimot rettferdiggj\o r 
vi Schr\"odingers likning  ved \aa\ anvende den for \aa\ beskrive ulike fysiske systemer.

Selv om vi ikke har noen basale prinsipper (det hadde ikke
Newton heller) for utledningen av Schr\"odingers likning, kan vi si at den viten om
kvantemekaniske effekter som var akkumulert fram til presentasjonen
av Schr\"odingers likning  ga oss et sett av hypoteser og postulater for \aa\ utlede
en b\o lgelikning for materie p\aa\ mikroniv\aa\ .

Bakgrunnen kan vi oppsummere i f\o lgende fire punkter
\begin{enumerate}
\item  Konsistens med de Broglie-Einstein postulater
\begin{equation}
     \lambda =\frac{h}{p}  \hspace{1cm} \nu=\frac{E}{h}    
\end{equation}
\item Energien (ikke-relativistisk ) er gitt ved
\begin{equation}
    E=\frac{p^2}{2m}+V
\end{equation}
\item Dersom $\Psi_1(x,t)$ og $\Psi_2(x,t)$ er l\o sninger er
ogs\aa\
\begin{equation} 
 \Psi(x,t)=\Psi_1(x,t)+\Psi_2(x,t)
\end{equation}
Dette kalles for krav om  lin\ae aritet, som igjen sikrer 
interferensm\o nster og diffraksjonsm\o nster slik vi observerte i 
forrige kapittel for materieb\o lger. Dette skal vi vise eksplisitt
i neste avsnitt.
\item
Dersom $V(x,t)=V_0$ en konstant, har vi
\begin{equation}
   F=-\frac{\partial V_0}{\partial x}=0
\end{equation}
og Newtons bevegelseslov forteller at $p$ er konstant som igjen betyr
at $E$ er konstant. L\o sningen blir da av typen
\begin{equation}
    \Psi\propto sin(kx-\omega t)
\end{equation}
\end{enumerate}

Det er denne syntesen, basert p\aa\ bestemte postulater og med
en tilh\o rende  teoretisk beskrivelse som gir, sammen med eksperiment,
essensen i den vitenskapelige framgangsm\aa te i fysikk. Vi kan trygt si at
fysikk er et eksperimentelt fag. Dog, siden vi tar m\aa l av oss \aa\ avdekke
naturens lover, s\aa\ blir konsekvensene av denne syntesen ytterste
spennende, fra teorier om universets tilblivelse til  m\aa ten PC'en 
p\aa\ labben eller hjemme virker. Arbeidshesten i PC'en er en halvleder
transistor, CMOS, som igjen er et vakkert eksempel p\aa\ et kvantemekanisk
system. 

Dersom vi n\aa\ anvender
de Broglie-Einstein hypotesen p\aa\ uttrykket for energien f\aa r vi
\begin{equation}
    E=\frac{p^2}{2m}+V(x,t)=\frac{\hbar^2k^2}{2m}+V(x,t),
\end{equation}
hvor vi har brukt at $p=\hbar k$. Innf\o rer vi deretter
at $E=h\nu=\hbar\omega$ finner vi at 
\begin{equation}
    E=\frac{\hbar^2k^2}{2m}+V(x,t)=\hbar\omega.
\end{equation}
Her har vi 
overf\o rt  $E=\hbar\omega$ fra fotoner til materieb\o lger 
dvs. 
\begin{equation}
   \omega=\omega(k),
\end{equation} 
som kalles en dispersjonsrelasjon for $\omega$.

Det vi skal legge merke til er at energien er proporsjonal med
$\omega$ og $k^2$. 

I b\o lgelikningen fra klassisk fysikk har vi et ledd med den
andre deriverte av b\o lgefunksjonen~mhp.~tid og et med den andre deriverte mhp.~posisjon.
Det skulle tilsi at dersom 
vi antar at b\o lgefunksjonen~har ei l\o sning p\aa\ formen
\begin{equation}
    \Psi(x,t)=sin(kx-\omega t)
\end{equation}
og at denne b\o lgefunksjonen~kan representere en partikkel, 
har vi at 
\begin{equation}
    \frac{\partial^2 \Psi(x,t)}{\partial x^2}=-k^2 sin(kx-\omega t)=-k^2 \Psi(x,t)
\end{equation}
og 
\begin{equation}
    \frac{\partial^2 \Psi(x,t)}{\partial t^2}=\omega^2 sin(kx-\omega t)=\omega^2 \Psi(x,t).
\end{equation}

Men vi ser at dette gir et uttrykk som relaterer $k^2$ til $\omega^2$,
som passer d\aa rlig med uttrykket for $E$. 

Rekner vi ut den f\o rste deriverte mhp.~tid finner vi 
\begin{equation}
    \frac{\partial \Psi(x,t)}{\partial t}=-\omega cos(kx-\omega t)\ne -\omega \Psi(x,t).
\label{eq:first}
\end{equation}
Dette eksemplet viser at
for \aa\ f\aa\ et uttrykk for energien som g\aa r som
$E=\hbar\omega$ trenger vi alts\aa\ den f\o rste deriverte
av b\o lgefunksjonen~mhp.~tid $t$. For \aa\ f\aa\ et uttrykk for kinetisk
energi som er i samsvar med $k^2$ trenger vi den andre deriverte
av b\o lgefunksjonen~mhp.~posisjon $x$. 

Dersom vi n\aa\ setter $V(x,t)=V_0$, som er en konstant, 
kan vi alts\aa\ gjette p\aa\ ei b\o lgelikning som har form som
\be
    \alpha \frac{\partial^2 \Psi(x,t)}{\partial x^2}+V_0\Psi(x,t)=
    \beta\frac{\partial \Psi(x,t)}{\partial t},
    \label{eq:trial}
\ee
Siden likning (\ref{eq:first}) ikke er proporsjonal med b\o lgefunksjonen, pr\o ver 
vi oss med
\be
     \Psi(x,t)=cos(kx-\omega t)+\gamma sin(kx-\omega t).
\ee
Vi ser da at 
\be
    \frac{\partial^2 \Psi(x,t)}{\partial x^2}=-k^2 \Psi(x,t)
\ee
og at
\be
    \frac{\partial \Psi(x,t)}{\partial t}=
\omega sin(kx-\omega t)-\omega\gamma cos(kx-\omega t),
\ee
som innsatt i likning (\ref{eq:trial})
gir 
\begin{eqnarray}
   -\alpha k^2(cos(kx-\omega t)+\gamma sin(kx-\omega t))+
   V_0(cos(kx-\omega t)+\gamma sin(kx-\omega t))=& \nonumber \\
   \beta\omega(sin(kx-\omega t)-\gamma cos(kx-\omega t))&,
\end{eqnarray}
eller
\be
[-\alpha k^2+V_0+\beta\gamma\omega]cos(kx-\omega t)+
[-\alpha k^2\gamma+V_0\gamma-\beta\omega]sin(kx-\omega t)=0
\ee
som gir
\be
   -\alpha k^2+V_0=-\beta\gamma\omega,
\ee
og 
\be
   -\alpha k^2\gamma+V_0\gamma=\beta\omega,
\ee
som leder til
\be
    -\beta\gamma^2\omega-\beta\omega=0,
\ee
eller
\be
-\gamma^2=1,
\ee
som igjen betyr at $\gamma=\pm i$, dvs.
\be
   -\alpha k^2+V_0=\pm i\beta\omega,
\ee
som skal sammenliknes med
\[
 \frac{\hbar^2k^2}{2m}+V_0=\hbar\omega.   
\]
Det betyr at
\be
   -\alpha=\frac{\hbar^2}{2m},
\ee
og
 \be
\pm i\beta=\hbar,
\ee
eller
\be
   \beta=\pm i\hbar,
\ee
Dette betyr at
\be
     \Psi(x,t)=cos(kx-\omega t)+i sin(kx-\omega t)=
     e^{i(kx-\omega t)}, 
\ee
og at vi kan skrive likning (\ref{eq:trial}) som
\be
    -\frac{\hbar^2}{2m}\frac{\partial^2 \Psi(x,t)}{\partial x^2}+V_0\Psi(x,t)=
    i\hbar\frac{\partial \Psi(x,t)}{\partial t}.
\ee
Til n\aa\ har vi antatt at $V_0$ er konstant. Vi postulerer at siste likning
ogs\aa\ skal gjelde for et generelt potensial slik at Schr\"odingers likning  kan skrives p\aa\
generell form som
\be
    -\frac{\hbar^2}{2m}\frac{\partial^2 \Psi(x,t)}{\partial x^2}+
V(x,t)\Psi(x,t)=
    i\hbar\frac{\partial \Psi(x,t)}{\partial t}.
    \label{eq:schroed}
\ee
Dette er den kvantemekaniske bevegelseslikning som er analog til 
f.eks.~Newtons lover eller Maxwells likninger. B\o lgefunksjonen
skal alts\aa\ inneholde opplysninger om partikkelens posisjon
og bevegelsesmengde innafor den usikkerhet som er diktert av
Heisenbergs uskarphets relasjon. 

Vi merker oss at Schr\"odingers likning  ikke er en fullstendig likning for kvantemekaniske
systemer, bla.~kan vi ikke bruke den til \aa\ beskrive fotoner, som tilh\o rer
en partikkelgruppe som g\aa r under navnet bosoner.
I tillegg er den basert p\aa\ et ikke-relativistisk uttrykk for energien.
Det var 
Dirac som introduserte en relativistisk kvantemekanisk likning for b\aa de
elektroner og fotoner ved \aa\ starte med uttrykket for energien
\be
   E=\sqrt{p^2c^2+(mc^2)^2}+V.
\ee


\section{Borns sannsynlighetstolkning}
\label{sec:sltolkning}
Her skal vi introdusere Borns tolkning\footnote{Lesehenvisning er kap 5-2 sidene
226-231.}.
Vi har alts\aa\ sett at 
\[
     \Psi(x,t)=cos(kx-\omega t)+i sin(kx-\omega t)=
     e^{i(kx-\omega t)}, 
\]
dvs.~at b\o lgefunksjonen er kompleks, siden vi har at den f\o rste deriverte
av tiden forholder seg til den andre deriverte mhp.~posisjon $x$.
At det er slik f\o lger av uttrykket for energien som er utleda basert 
p\aa\ de Broglie-Einstein hypotesen, som gir oss $p^2\sim \hbar \omega$. 

En kompleks b\o lgefunksjonen~kan ikke m\aa les, ei heller er vi i stand til \aa\ trekke
paralleller til en klassisk b\o lgefunksjonen
I Schr\"odingers likning  kan vi betrakte b\o lgefunksjonen~som et abstrakt matematisk begrep, et 
beregningsverkt\o y, som har mening innenfor rammen av Schr\"odingers likning  og som skal
gi oss matematisk reelle st\o rrelser som energi, bevegelsesmengde
og posisjon, for \aa\ nevne noen. 
Den inneholder all informasjon som uskarphetsprinsippet 
tillater oss \aa\ vite om den tilordnede partikkelen. 

For \aa\ gi b\o lgefunksjonen~en tolkning dro Born veksel p\aa\ partikkel-b\o lge
dualiteten som vi diskuterte i forrige kapittel, se avsnitt \ref{subsec:pbdualitet}, og foreslo
at 
\be
   P(x,t)=\Psi(x,t)^*\Psi(x,t),
\ee
hvor stjernesymbolet skal representere komplekskonjugering,
skulle tolkes som en sannsynlighetstetthet hvor
\be
   P(x,t)dx=\Psi(x,t)^*\Psi(x,t)dx
\ee
gir oss sannsynligheten for \aa\ finne partikkelen innafor et omr\aa de
$x$ til $x+dx$. Sannsynligheten m\aa\ alltid v\ae re reell, noe vi kan
se fra f\o lgende. Antar vi at vi kan skrive 
\be 
   \Psi(x,t)=R(x,t)+i I(x,t)
\ee
hvor $R$ og $I$ er reelle funksjoner, gir det
\be 
   \Psi(x,t)^*\Psi(x,t)=(R-i I)(R+i I)=R^2+I^2
\ee 
hvor vi har brukt at $i^2=-1$.

Siden b\o lgefunksjonen~skal representere en partikkel m\aa\ den v\ae re
av endelig utstrekning, ellers kan vi ikke snakke om muligheten for \aa\
finne partikkelen et bestemt sted, se f.eks.~figur 5.3 i boka. 
Dette gir oss noen f\o ringer p\aa\ selve b\o lgefunksjonen
\begin{enumerate}
\item For at vi skal kunne gi den en sannsynlighetstolkning
m\aa\ den alts\aa\ v\ae re endelig. I tillegg m\aa\ summen
(integral her da b\o lgefunksjonen~er en kontinuerlig funksjon)  av alle
sannsynligheter v\ae re lik $1$. Vi sier da at den er normerbar.
Matematisk uttrykkes dette som 
\be
   \int_{-\infty}^{\infty}P(x,t)dx=\int_{-\infty}^{\infty}\Psi(x,t)^*\Psi(x,t)dx=1
\ee
som betyr igjen at 
\be
  \int_{-\infty}^{\infty}\Psi(x,t)^*\Psi(x,t)dx < \infty
\ee
dvs.~ at b\o lgefunksjonen~er kvadratisk normerbar, 
og $\Psi(x,t)$ m\aa\ g\aa\ raskere mot null
enn $x^{-1/2}$. 
\item $\Psi(x,t)$ m\aa\ v\ae re entydig, ellers vil vi ikke v\ae re i stand
 til \aa\ tilordne b\o lgefunksjonen~f.eks.~en bestemt bevegelsesmengde.
\item $\Psi(x,t)$ og $\partial \Psi(x,t)/\partial x$
m\aa\ v\ae re kontinuerlige, for ellers vil Schr\"odingers likning  divergere
n\aa r vi rekner ut den andre deriverte mhp.~$x$. Merk at disse to krav
ikke gjelder n\aa r $V=\pm \infty$. Vi skal se eksempler p\aa\ det i avsnitt
\ref{sec:uendeligbronn}
\end{enumerate}


Vi nevnte i forrige avsnitt at b\o lgefunksjonen~skal v\ae re line\ae r, dvs.~at 
dersom $\Psi_1$ og $\Psi_2$ er to l\o sninger av 
Schr\"odingers likning , s\aa\ er ogs\aa\ $\Psi=\Psi_1+\Psi_2$ en l\o sning.
Dette har konsekvenser dersom vi er interesserte i
\aa\ observere et interferensm\o nster.
Vi vet fra forrige kapittel at intensiteten er proporsjonal
med $\Psi(x,t)^*\Psi(x,t)$, og for bestemte verdier
av $k(x_1-x_2)$ kunne vi ha konstruktiv eller destruktiv 
interferens. La oss anta at vi har to b\o lgefunksjonen~ved $t=0$ ved et punkt
$P$. Den f\o rste b\o lgefunksjonen~er da gitt ved  
\be
   \Psi_1(x_1,t=0)=A_1e^{i kx_1},
\ee
mens den andre er gitt ved
\be
   \Psi_2(x_2,t=0)=A_2e^{i kx_2}.
\ee
Definerer vi  deretter
\be
   \Theta_1=kx_1,\hspace{0.2cm}\Theta_2=kx_2
\ee
s\aa\ har vi at
\be
   \Psi=\Psi_1+\Psi_2
\ee
gir 
\be
   |\Psi(x,t=0)|^2=
(A_1e^{-i\Theta_1}+A_2e^{-i\Theta_2})
(A_1e^{i\Theta_1}+A_2e^{i\Theta_2}),
\ee
som resulterer i 
\be 
A_1^2+A_2^2+2A_1A_2cos(\Theta_1-\Theta_2).
\ee
Vi merker oss at dette uttrykket har samme form som n\aa r vi adderte
to b\o lger i avsnitt \ref{sec:boelgelaere}. Dersom vinkelen
\be
   \Theta_1-\Theta_2=2n\pi , \hspace{0.2cm} n=0,\pm 1, \pm 2,\dots  
\ee
har vi 
\be
  |\Psi(x,t=0)|^2= A_1^2+A_2^2+2A_1A_2,
\ee 
og f\aa r dermed kontruktiv interferens. For
\be
   \Theta_1-\Theta_2=(2n+1)\pi ,\hspace{0.2cm} n=0,\pm 1, \pm 2,\dots  
\ee
f\aa s
\be
  |\Psi(x,t=0)|^2= A_1^2+A_2^2-2A_1A_2,
\ee 
som gir destruktiv interferens (dersom $A_1=A_2$ f\aa r vi null). 

Vi ser at kravet til lin\ae aritet som vi satte opp i avsnitt
\ref{sec:slintro} gir oss muligheter for interferens, noe som er \o nskelig
utifra det vi har observert om materieb\o lger. 

\section{Egenfunksjoner og egenverdier}

Vi skal\footnote{Lesehenvisning er kap 5-3 sidene
231-234.} n\aa\ ta v\aa rt f\o rste skritt i retning av \aa\ l\o se
Schr\"odingers likning  gitt ved likning (\ref{eq:schroed})
\[
    -\frac{\hbar^2}{2m}\frac{\partial^2 \Psi(x,t)}{\partial x^2}+
V(x,t)\Psi(x,t)=
    i\hbar\frac{\partial \Psi(x,t)}{\partial t}. 
\]
Dette er en s\aa kalt partiell differensiallikning hvis l\o sning 
ikke alle har sett. I et kurs som FYS 211, vil teorien for slike
likninger gjennomg\aa s i detalj. Her skal vi anvende en teknikk som heter
separasjon av variable, en teknikk som reduserer en differensiallikning
med flere variable til et sett av ordin\ae re differensiallikninger
med kun en variabel. 
Teknikken best\aa r i \aa\ pr\o ve en ansats for l\o sningen gitt ved
\be
   \Psi(x,t)=\psi (x)\phi (t).
    \label{eq:ansatz}
\ee
I tillegg, og dette er noe som dere sannsynligvis
har sett fra klassisk fysikk ogs\aa\ (tenk p\aa\ Coulomb potensialet),
skal vi anta at 
\be
   V(x,t)=V(x).
\ee
Kvantemekanikken avviker ikke noe s\ae rlig fra klassisk fysikk hva ang\aa r
den potensielle energiens variabel avhengighet. En slik forenkling
av potensialet tillater oss \aa\ finne en l\o sning p\aa\ formen
angitt i likning (\ref{eq:ansatz}). Dersom vi setter denne ansatsen for 
b\o lgefunksjonen~inn i Schr\"odingers likning  finner vi  
\be
    -\frac{\hbar^2}{2m}\frac{\partial^2 \psi (x)\phi (t)}{\partial x^2}+
V(x)\psi (x)\phi (t)=
    i\hbar\frac{\partial \psi (x)\phi (t)}{\partial t},
\ee
som gir
\be
    -\frac{\hbar^2}{2m}\phi (t)\frac{\partial^2 \psi (x)}{\partial x^2}+
V(x)\psi (x)\phi (t)=
    i\hbar\psi(x)\frac{\partial \phi (t)}{\partial t},
\ee
og dividerer vi begge sider med $\psi (x)\phi (t)$ kommer  vi fram til
\be
\frac{1}{\psi (x)}\left\{-\frac{\hbar^2}{2m}
        \frac{\partial^2 \psi (x)}{\partial x^2}+V(x)\psi (x)\right\}=\frac{i\hbar}{\phi(t)}\frac{\partial \phi (t)}{\partial t}.
\ee
Vi merker oss at h\o yre
side avhenger ikke av $x$ og venstre side avhenger ikke av $t$. 
Det betyr igjen at deres felles verdi kan ikke avhenge av $x$ eller $t$,
noe som igjen leder til at den felles verdien m\aa\ v\ae re en konstant.
Vi kan skrive henholdsvis venstre og h\o yre side som to adskilte
ordin\ae re differensiallikninger 
hvor vi n\aa\ dropper symbolet for partiell derivasjon 
\be
\frac{1}{\psi (x)}\left\{-\frac{\hbar^2}{2m}
        \frac{d^2 \psi (x)}{d x^2}+V(x)\psi (x)\right\}=C
\ee
og 
\be
   \frac{i\hbar}{\phi(t)}\frac{\partial \phi (t)}{\partial t}=C
\ee
som er en f\o rste ordens differensiallikning. $C$ kalles for 
separasjonskonstanten, av samme grunn som denne teknikken kalles
for separasjon av variable. 

Vi kan omskrive siste likning som
\[
      \frac{d \phi (t)}{d t}=C\frac{-i\phi(t)}{\hbar}
\]
hvis l\o sning er gitt ved 
\be
   \phi(t)=e^{-i Ct/\hbar}=cos(Ct/\hbar) -i sin(Ct/\hbar).
\ee
Bruker vi deretter  
\be
   \frac{C}{\hbar}=\omega,
\ee
finner vi ved \aa\ bruke
$E=\hbar\omega$, at dette impliserer at 
$C=E$, energien til systemet. 
Vi kunne osg\aa\ sett at konstanten $C$ har dimensjon energi 
fra en dimensjonsanalyse av eksponenten $Ct/\hbar$. En funskjon
slik som slik som $ln(x)$, $e^x$, $sin(x)$ osv.~m\aa\ alltid et argument
som er dimensjonsl\o st. Vi kan ikke si at energien til systemet
er $ln(10$eV)! Siden $\hbar$ har dimensjon energi-tid og $t$ har dimensjon
tid, m\aa\ $C$ ha dimensjon energi.
 
Med $C=E$  ledes vi til den tidsuavhengige
Schr\"odinger likning gitt ved
\be
\frac{1}{\psi (x)}\left\{-\frac{\hbar^2}{2m}
        \frac{d^2 \psi (x)}{d x^2}+V(x)\psi (x)\right\}=E
      \label{eq:xsl}
\ee
hvor n\aa\ $\psi(x) $ kalles egenfunksjoner
og $E$ kalles egenverdier. Merk at det er
$\Psi(x,t)$ som kalles selve b\o lgefunksjonen. Den siste 
likning inneholder ikke $i$ og gir dermed ikke n\o dvendigvis
komplekse l\o sninger. I all hovedsak likner denne likningen
p\aa\ den klassiske likning for en svingende streng.
Dette skal vi se p\aa\ i neste avsnitt n\aa r vi l\o ser
Schr\"odingers likning  for en partikkel i en en-dimensjonal potensialbr\o nn. 
Vi skal se at ogs\aa\ et klassisk system som en svingende streng
utviser kvantisering av energien, dvs.~at vi har diskrete egenenergier. 

Vi kan n\aa\ pr\o ve \aa\ anskueliggj\o re den tidsuavhengige 
Schr\"odingers likning  i likning (\ref{eq:xsl}) ved \aa\ sette
$V(x)=V_0$. Da er 
energien gitt ved, dersom vi anvender de Broglies og Einsteins 
hypotese 
\be
   E=\frac{\hbar^2k^2}{2m}+V_0,
\ee
som gir
\be
   k^2=\frac{2m}{\hbar^2}\left(E-V_0\right).
   \label{eq:test}
\ee
Antar vi deretter at den romlige delen av b\o lgefunksjonen~er gitt ved 
\be
   \psi(x)=sin(\frac{2\pi x}{\lambda})=sin (kx)
\ee
har vi 
\be
   \frac{d^2 \psi (x)}{d x^2}=-k^2\psi(x),
\ee
som innsatt i den tidsuavhengige Schr\"odingers likning  
fra  likning (\ref{eq:xsl}) gir 
\be
   \frac{d^2 \psi (x)}{d x^2}= -\frac{2m}{\hbar^2}\left(E-V_0\right)\psi(x)=-k^2\psi(x),
\ee  
som igjen gir oss 
\[
     k^2=\frac{2m}{\hbar^2}\left(E-V_0\right),
\]
i samsvar med likning (\ref{eq:test}).


\section{Schr\"odinger likning for et uendelig bokspotensial}
\label{sec:uendeligbronn}
Vi skal n\aa\ pr\o ve \aa\ l\o se\footnote{Lesehenvisning: kap 5-4.} den tidsuavhengige 
Schr\"odingers likning for noen enkle systemer. Konkret 
skal vi se p\aa\ det idealiserte tilfellet med en 
partikkel i en uendelig potensialbr\o nn, for bla.~\aa\ vise at 
Schr\"odingers likning gir kvantisering av energien. I tillegg,
likner dette eksemplet mye p\aa\ l\o sningen for en svingende
streng som er fastspent i begge ender. Men f\o rst 
skal vi se p\aa\ l\o sningen av den tidsuavhengige Schr\"odingers likning  for fri 
partikkel. 

\subsection{Fri partikkel}
Vi kan her tenke oss at vi sender monoenergetiske (med bestemt
bevegelsesmengde)  protoner fra en syklotron mot en eller
annen target, en blykjerne f.eks. Protonene har kun kinetisk energi og
den tidsuavhengige Schr\"odingers likning  reduseres til
\be
        -\frac{\hbar^2}{2m}
        \frac{d^2 \psi (x)}{d x^2}=E\psi (x),
\ee
som igjen gir en l\o sning som vi diskuterte i avsnitt \ref{sec:slintro}, nemlig
\be
    \Psi(x,t)=cos(kx-\omega t)+i sin(kx-\omega t),
\ee
eller dersom vi \o nsker kan vi skrive l\o sningen p\aa\ 
eksponensial form
\be
    \Psi(x,t)=e^{i (kx-\omega t)},
\ee
hvor vi har definert b\o lgetallet
\be
    k=\frac{\sqrt{2mE}}{\hbar},
\ee
og vinkelfrekvensen $\omega =E/\hbar$. Splitter vi eksponensialfunksjonen
i en $x$-avhengig og en $t$-avhengig del har vi
\be
    \Psi(x,t)=e^{i kx}e^{-i \omega t}=e^{i kx}e^{-i Et/\hbar}=\psi(x)e^{-i Et/\hbar},
\ee
hvor $\psi(x)=e^{i kx}$. 
Denne representer en b\o lge som reiser i retning \o kende $x$. P\aa\
tilsvarende vis har vi ei b\o lge som reiser i retning minkende $x$ gitt
ved 
\be
    \Psi(x,t)=e^{-i kx}e^{-i Et/\hbar},
\ee
slik at vi kan skrive den generelle egenfunksjonen\footnote{Merk 
at med egenfunksjonen mener vi $\psi(x)$ og ikke $\Psi(x,t)$, som 
er selve b\o lgefunksjonen.} som
\be
   \psi(x)=Ae^{i kx}+Be^{-i kx},
\ee
hvor $A$ og $B$ er to konstanter som bestemmes fra normeringsbetingelsen
diskutert i avsnitt \ref{sec:sltolkning}. 
Vi skal merke oss her at
dersom vi velger l\o sningen for en partikkel som beveger seg i positiv
$x$-retning har vi at
\be
    \Psi(x,t)^*\Psi(x,t)=A^*Ae^{-i (kx-\omega t)}e^{i (kx-\omega t)}=A^*A,
\ee
en konstant. Dvs.~at sannsynligheten for \aa\ finne partikkelen
er like stor for alle $x$, noe som igjen betyr at uskarpheten i posisjon 
$\Delta x=\infty$. Det vil igjen si at $\Delta p=0$ i henhold til
Heisenbergs uskarphetsrelasjon. 

Selv om vi sier at uskarpheten i $x$ er uendelig, s\aa\ vil vi alltid
assosiere en bestemt partikkel til en b\o lgefunksjon. I v\aa rt tilfelle
kan det v\ae re protonene som blir aksellerert i en syklotron og treffer
en target av bly etter en viss avstand. Selv om vi ikke kan si eksakt
hvor protonet er, s\aa\ har en slik protonstr\aa le en endelig utstrekning,
fra syklotronen til den treffer blykjerne. I den forstand kan vi si
at normaliseringsbetingelsen blir 
\be
    \int_{-\infty}^{\infty}\Psi^*\Psi dx=\int_{-L/2}^{L/2}\Psi^*\Psi dx=LA^*A=1,
\ee
hvor $L$ er avstanden fra syklotronen til target, og $\Delta x < L$. 
Vi setter $\Psi=0$ i omr\aa dene  $-L/2 < x$ og $ x >  L/2$.

\subsubsection{Virkem\aa ten til en syklotron}
I Michigan State University ved East Lansing, finnes det en supraledende 
syklotron\footnote{Se http://intra.nscl.msu.edu/facility/k1200/} som produserer
str\aa ler med energier fra 20 MeV per nukleon (n\o ytron eller proton)
til 200 MeV per nukleon. Radiofrekvensen (rf) til systemet s\o rger for et aksellerende
potensial med en frekvens, som svarer til de ovennevnte energiene, mellom
9 MHz og 26.5 MHz. En supraledende magnet setter opp magnetfelt mellom
3 og 5 Tesla. 
Syklotronen brukes til \aa\ aksellerere enkeltatomer, men kan ogs\aa\ brukes til \aa\
aksellerere molekyler. For \aa\ produsere en str\aa le av protoner brukes
f.eks H$_2^+$ molekyler.

En ladd partikkel i et magnetfelt p\aa virkes av en kraft $F$
\be
    {\bf F} = q {\bf v} \times {\bf B},
\ee
hvor $v$ er hastigheten, $q$ ladingen og $B$ magnetfeltet. Sentripetalaksellerasjonen
er gitt ved
\be 
   a_r= \frac{v^2}{r},
\ee
som gir
\be 
   F=m\frac{v^2}{r} = qvB.
\ee
Avstanden til sentrum av syklotronen er dermed gitt ved
\be
   r=\frac{mv}{qB}.
\ee
Dersom radien til syklotronen settes lik $R$, finner vi at den maksimale
hastigheten som en ladd partikkel kan ha er gitt ved 
\be
   v_{\mathrm{maks}}= \frac{qBR}{m},
\ee
som gir en resulterende kinetisk energi 
\be
   E_{\mathrm{kin}} = \frac{1}{2} \frac{q^2B^2R^2}{m}.
\ee
\subsection{Uendelig potensialbr\o nn}\label{subsec:342}
Vi skal n\aa\ se p\aa\ et system hvor en partikkel kan antas \aa\ bevege
seg i et s\aa kalt uendelig kassepotensial gitt ved 
\be
 V(x)=\left\{\begin{array}{cc}+\infty& x\le -a\\ 
           0&-a <  x < +a \\
           +\infty&x \ge +a, \end{array} \right.
\ee
med en skjematisk skisse som vist i Figur \ref{fig:uendeligpot}.
\begin{figure}[h]
\begin{center}

\setlength{\unitlength}{1cm}
\begin{picture}(13,6)
\thicklines
   \put(0,0.5){\makebox(0,0)[bl]{
              \put(0,1){\vector(1,0){12}}
              \put(12.3,1){\makebox(0,0){x}}
              \put(4.9,0.5){\makebox(0,0){$-a$}}
              \put(8.0,0.5){\makebox(0,0){$a$}}
              \put(6.5,0.5){\makebox(0,0){0}}
              \put(4.5,4.8){\makebox(0,0){$+\infty$}}
              \put(8.5,4.8){\makebox(0,0){$+\infty$}}
              \put(5,1){\line(0,1){4}}
              \put(8,1){\line(0,1){4}}
         }}
\end{picture}
\end{center}
\caption{Eksempel p\aa\ et uendelig kassepotensial, med
         $V(x)=\infty$ ved $x=-a$ og $x=a$. \label{fig:uendeligpot}}
\end{figure}
Den tidsuavhengige Schr\"odingers likning  blir dermed 
\be
        -\frac{\hbar^2}{2m}\frac{d^2 \psi (x)}{d x^2}=E\psi (x),
\ee 
for omr\aa det $-a <  x < +a$. 
Vi kan omskrive siste likning som
\be
            \left(\frac{d^2 }{d x^2}+
             \frac{2m}{\hbar^2}E\right)\psi (x)=0,
\ee
og med 
\be
    E=\frac{p^2}{2m}=\frac{\hbar^2k^2}{2m} 
\ee
som systemets energi har vi
\be
            \left(\frac{d^2 }{d x^2}+k^2\right)\psi (x)=0.
\ee
Vi ser at den tidsavhengige Schr\"odingers likning  reduserer seg i omr\aa det 
$-a <  x < +a$ til Schr\"odingers likning  for en fri partikkel.
En generell l\o sning til denne
likningen blir dermed (siden vi har \aa\ gj\o re med 
en andre ordens differensiallikning f\aa r vi to konstanter $A$
og $B$, se det foreg\aa ende eksemplet for fri partikkel) 
\be
   \psi(x)=Acos(kx)+Bsin(kx),
\ee
eller 
\be
   \psi(x)=A'e^{i(kx-\omega t)}+B'e^{-i( kx+\omega t)},
\ee
hvor det f\o rste leddet representerer ei b\o lge som reiser
i positiv $x$-retning mens det andre leddet
er ei b\o lge som reiser
i negativ $x$-retning.   

Fra avsnitt 4.2 har vi betingelsen
om at b\o lgefunksjonen~og dens deriverte $d\psi(x)/dx$ skal v\ae re kontinuerlig.
Men siden vi har et potensial av uendelig utstrekning ved $|x|=a$,
byr dette p\aa\ noen problemer.
I omr\aa dene $|x| \ge |a|$ 
er $V(x)=+\infty$ 
overalt.  Klassisk og konkret svarer det til \aa \ sette en 
uigjennomtrengelig vegg foran partikkelen.  Partikkelen 
kan da umulig befinne seg i dette omr\aa det.  Dette vil ogs\aa \ 
gjelde kvantemekanisk, det m\aa \ v\ae re null sannsynlighet 
for \aa \ finne partikkelen i et punkt i dette omr\aa det. 
Dette betyr at 
\[
\psi(x)\equiv 0
\]
for alle $|x| \ge |a|$.  Fysisk er dette fullstendig rimelig, 
men vi kan ogs\aa \ sannsynliggj\o re det ut i fra Schr\"odingers likning : 
\be
-\frac{\hbar^2}{2m}\frac{d^2}{dx^2}\psi(x)+V(x)\psi(x)=
E\psi(x).
\ee
Siden $V(x)=+\infty$ betyr det at venstresiden 
av likningen vil v\ae re uendelig stor.  Vi kan da tenke oss 
to m\aa ter \aa \ f\aa \ likningen til \aa \ g\aa \ opp p\aa \ : 
\begin{enumerate}
 \item Vi kan la $\psi(x)=\infty$. 
 \item Vi kan la $\psi(x)\equiv 0$.  
\end{enumerate}
Den f\o rste muligheten er fullstendig ufysikalsk, da den gir 
en uendelig sannsynlighet for \aa \ finne partikkelen i ethvert 
punkt $|x| \ge |a|$.  Bare den siste muligheten er akseptabel, og derfor er 
\be
\psi(x)=0
\ee
for alle $|x| \ge |a|$.  
I f\o lge kravene til b\o lgefunksjonen m\aa \ $\psi$ v\ae re 
kontinuerlig i $|x|=a$.  Det betyr at 
\be
\psi(x=a)=0
\ee
Siden $V(x)$ er uendelig for $|x| \ge |a|$ kan vi ikke kreve at 
$\psi$ skal v\ae re kontinuerlig deriverbar i $x=a$.

Den siste betingelsen leder til
\be
   \psi(x=a)=0=Acos(ka)+Bsin(ka),
\ee
og 
\be
   \psi(x=-a)=0=Acos(ka)-Bsin(ka),
\ee
Dersom vi subtraherer de to siste likninger finner vi
\be
    2Bsin(ka)=0,
\ee
og adderer vi har en
\be 
   2Acos(ka)=0.
\ee
Vi kan ikke ha b\aa de $A$ og $B$ lik null for da vil
$\psi$ v\ae re null for en gitt verdi av $k$. 
Ei heller klarer vi \aa\ finne ei l\o sning hvor b\aa de
$cos(ka)=0$ og $sin(ka)=0$.
Vi velger derfor
to forskjellige l\o sninger, en hvor $B=0$ som gir 
\be
\psi(x)=Acos (kx),
\ee
som er null n\aa r $ka=n\pi/2$, med $n=\pm 1, \pm 3,\dots$, dvs
\be
     k=\frac{n\pi}{2a}, \hspace{0.2cm} n=\pm 1, \pm 3,\dots,
\ee
slik at vi har
\be
   \psi(x)=Acos (x\frac{n\pi}{2a}).
\ee
Tilsvarende har vi en l\o sning n\aa r $A=0$ som gir
\be
\psi(x)=Bsin (kx),
\ee
som er null n\aa r $ka=n\pi/2$, med $n=\pm 2, \pm 4,\dots$, dvs
\be
     k=\frac{n\pi}{2a}, \hspace{0.2cm} n=\pm 2, \pm 4,\dots,
\ee
slik at vi har
\be
   \psi(x)=B sin (x\frac{n\pi}{2a}).
\ee
Merk at vi ikke tar med $n=0$ da det gir den trivielle l\o sningen
$\psi(x) =0$, som igjen betyr n\aa r vi bruker Borns tolkning av b\o lgefunksjonen at det er null sannsynlighet for \aa\ finne partikkelen. 

Med uttrykket for $k$ har vi dermed at energien blir kvantisert
siden $E$ er gitt ved
\be
    E_n=\frac{\hbar^2k^2}{2m}=\frac{\hbar^2\pi^2}{8ma^2}n^2.
    \label{eq:uendeligpote}
\ee
Den generelle b\o lgefunksjonen blir dermed
\be
    \Psi_n(x,t)=Acos (x\frac{n\pi}{2a})e^{-i E_nt/\hbar}\hspace{0.2cm} n=\pm 1, \pm 3,\dots,
\ee
og  
\be
    \Psi_n(x,t)=Asin (x\frac{n\pi}{2a})e^{-i E_nt/\hbar}\hspace{0.2cm} n=\pm 2, \pm 4,\dots,
\ee


La oss se p\aa\ normaliseringsbetingelsen for b\o lgefunksjonen~for \aa\ kunne
bestemme konstantene $A$ og $B$. La oss se p\aa\ cosinus-l\o sningen
f\o rst.
Vi krever alts\aa\ at
\be
   \int_{-\infty}^{+\infty}\psi(x)^*\psi(x)dx=\int_{-a}^{a}A^2cos^2(k_nx)dx=1,
\ee
som gir oss at
\be
  \int_{-a}^{a}A^2cos^2(k_nx)dx=\int_{-a}^{a}\frac{1}{2}A^2(1+cos(2k_nx))dx=1,
\ee
som resulterer i 
\be
    \frac{1}{2}A^2\left(2a+\frac{1}{2k_n}sin(\frac{n\pi}{a}x)\right)\left|_{-a}^a\right.=A^2a=1,
\ee
dvs.
\be
    A=\sqrt{\frac{1}{a}}.
\ee
Det gir oss en b\o lgefunksjon
\be
    \Psi_n(x,t)=\sqrt{\frac{1}{a}}cos (x\frac{n\pi}{2a})e^{-i E_nt/\hbar}\hspace{0.2cm} n=\pm 1, \pm 3,\dots.
\ee
P\aa\ tilsvarende vis finner vi (vis dette!)  at 
\be
   B=\sqrt{\frac{1}{a}},
\ee
slik at sinus-l\o sningen er gitt ved
\be
    \Psi_n(x,t)=\sqrt{\frac{1}{a}}sin (x\frac{n\pi}{2a})e^{-i E_nt/\hbar}\hspace{0.2cm} n=\pm 2, \pm 4,\dots.
\ee


Sj\o l om vi kun har tatt for oss et idealisert potensial, med liten rot
i den virkelige verden, s\aa\ utviser dette eksemplet en del viktige egenskaper
ved et kvantemekanisk system som dere b\o r  legge merke til.
\begin{itemize}
\item  Vi  finner en tilstand med 
lavest energi, gitt ved $n=1$ 
\be
    E_1=\frac{\hbar^2\pi^2}{8ma^2}.
\ee
Denne tilstanden kalles grunntilstanden. Klassisk er tilstanden med lavest energi
gitt n\aa r partikkelen er i ro. Det svarer til at  
b\aa de bevegelsesmengden $p$ og potensialet $V$ er lik null, som igjen betyr
at grunntilstandsenergien utifra et klassisk bide er lik null. 
Det faktum at denne energien ikke er null kvantemekanisk 
skyldes at vi har alltid
en uskarphet i posisjon og bevegelsesmengde. 

Vi kan vise dette dersom vi antar at vi kjenner $x$ med en uskarphet
$\Delta x\approx2a$, utstrekningen av potensialet. 
Vi kjenner ogs\aa\ bevegelsesmengden til den tilstanden som har lavest energi.
For $n=1$ har vi 
\be
   p_1=\pm\sqrt{2mE_1}=\pm\frac{\hbar\pi}{2a}.
\ee
Siden partikkelen kan ha b\aa de negativ og positiv
bevegelsesmengde har vi at 
\be
    \Delta p \approx 2|p_1|=\frac{\hbar\pi}{a},
\ee
som innsatt i Heisenbergs uskarphetsrelasjon gir
\be 
   \Delta p\Delta x\approx \frac{\hbar\pi}{a}2a=2\pi\hbar=h,
\ee
p\aa\ en konstant n\ae r $\hbar/2$. 

\item At det finnes en slik null-punkts energi (laveste energi) og 
null-punkts bevegelsesmengde, er i strid med klassisk
fysikk, hvor all bevegelse opph\o rer ved null temperatur. 
Denne null-punkts energien  er ansvarlig for flere interessante
kvantefenomener. Et sl\aa ende ekssempel er flytende helium, som forblir flytende helt ned til oppn\aa elige temperatur i labben, $\sim 0.001 $K.
Kun dersom et stort trykk anvendes vil en overgang til fast stoff forekomme.
Som en digresjon, tenk bare p\aa\ at den kosmiske bakgrunnsstr\aa lingen
er p\aa\ 3 K. Vi er alts\aa\ i stand til skape temperaturer som aldri
har eksistert i universet!

\item Legg ogs\aa\ merke til at energien er bestemt av dimensjonene
til systemet, dvs.~massen til partikkelen og utstrekningen 
av potensialet. Kun to parametre som dikterer fysikken!

\item Selv om dette systemet er fjernt fra virkeligheten, kan vi 
ha fysiske situasjoner som likner p\aa\ en uendelig potensialbr\o nn.
Mer om dette senere.
 
\end{itemize}
De l\o sningene som vi fant fram til er ortogonale, dvs.~at
%
\begin{equation}
\int \psi_{n'}^\ast \psi_{n} dx = \delta_{n',n},
\end{equation}
hvis $\psi_{n'}$ og $\psi_n$ h{\o}rer til to forskjellige
egenverdier $E_{n'}$ og $E_{n}$.



\section{Endelig bokspotensial}
V\aa rt neste eksempel er et endelig bokspotensial.
Her skal vi studere en partikkel som 
beveger seg langs $x$-aksen i et potensial 
\be
 V(x)=\left\{\begin{array}{cc}0&x>a  \\ 
                              -V_0&0< x \le a \\
                               \infty &x \le 0\end{array}\right. , 
\ee
vist i Figur \ref{fig:endeligkasse}.
\begin{figure}[h]
\begin{center}
\setlength{\unitlength}{1cm}
\begin{picture}(13,9)
\thicklines
   \put(0,0.5){\makebox(0,0)[bl]{
              \put(8,1){\vector(1,0){4}}
              \put(12.3,1){\makebox(0,0){x}}
              \put(5.2,1.5){\makebox(0,0){$0$}}
              \put(8.1,1.5){\makebox(0,0){$a$}}
              \put(4,0){\makebox(0,0){$III$}}
              \put(6,0){\makebox(0,0){$II$}}
              \put(10,0){\makebox(0,0){$I$}}
              \put(8.5,-3){\makebox(0,0){$-V_0$}}
              \put(5.5,4.8){\makebox(0,0){$V(x)$}}
              \put(5,1){\line(0,-1){4}}
              \put(5,1){\line(0,1){4}}
              \put(5,-3){\line(1,0){3}}
              \put(8,1){\line(0,-1){4}}
         }}
\end{picture}
\end{center}
\caption{Eksempel p\aa\ et endelig kassepotensial med verdi $-V_0$ i omr\aa det
         $0 <  x \le a$, uendelig for $x\le 0$ og null ellers. \label{fig:endeligkasse}}
\end{figure}

F\o rst litt notasjon.  Vi kaller omr\aa det p\aa \ $x$-aksen der 
$x > a$ for omr\aa de I, omr\aa det der $0< x \le a$ for II, og 
omr\aa det der $x \le  0$ for III.  B\o lgefunksjonen i de tre 
omr\aa dene kalles henholdsvis $\psi_{I}$, 
$\psi_{II}$ og $\psi_{III}$.  

For \aa \ bestemme l\o sningene trenger vi alts\aa \ de generelle 
krav vi har til en fysisk akseptabel b\o lgefunksjon.  Disse 
kan formuleres slik i den \'{e}n-dimensjonale utgaven: 
\begin{enumerate}
 \item $\psi(x)$ m\aa \ v\ae re kontinuerlig.
 \item $\psi(x)$ m\aa \ v\ae re normaliserbar, dvs vi m\aa \ 
       kunne finne en konstant $C_N$ slik at 
\be
   \int_{-\infty}^{+\infty}(C_N\psi(x))^*(C_N\psi(x))dx=1, 
\ee
     n\aa r fysikken i problemet er slik at systemet er begrenset 
     i rommet. 
\item $\frac{d}{dx}\psi(x)=\psi'(x)$ m\aa \ v\ae re kontinuerlig 
      n\aa r potensialet $V(x)$ er endelig. 
\end{enumerate}

Vi skal se p\aa \ tilfellet der partikkelen har 
negativ energi, $E<0$.
I klassisk fysikk betyr dette at partikkelen 
ikke kan befinne seg i omr\aa de I, der $V(x)=0$.  Dette ser vi 
med en energibetraktning:  Partikkelens energi er summen av 
kinetisk og potensiell energi:
\be
 E=\frac{p^2}{2m}+V(x),
\ee
og dersom partikkelen er i omr\aa de I vil bare det kinetiske leddet 
bidra:
\be
E=\frac{p^2}{2m}.
\ee
Men dersom $E<0$ er dette ikke mulig, det ville i s\aa \ fall bety 
at partikkelen hadde imagin\ae r bevegelsesmengde $p$.  
Klassisk er derfor omr\aa de I et forbudt omr\aa de for partikkelen. 
Vi skal se at kvantemekanisk kan partikkelen allikevel ha en 
viss sannsynlighet for \aa \ befinne seg i dette omr\aa det.


Vi ser f\o rst p\aa \ situasjonen i omr\aa de I: 
\be
-\frac{\hbar^2}{2m}\frac{d^2}{dx^2}\psi_I(x)=E\psi_I(x)
=-|E|\psi_I(x),
\ee
der den siste likheten gjelder fordi $E<0$.  Denne likningen skrives 
s\aa \ som 
\be
\psi_I''(x)=\beta^2\psi_I(x)
\ee
der $\beta^2=2m|E|/\hbar^2>0$.  Denne 2. ordens differensiallikningen 
har som generell l\o sning 
\be
\psi_I(x)=Ae^{\beta x}+Be^{-\beta x}
\ee
der $A$ og $B$ er integrasjonskonstanter.  Vi ser at leddet 
$Ae^{\beta x}$ vokser over alle grenser n\aa r $x\rightarrow \infty$.
Vi har et generelt krav om at b\o lgefunksjonen~skal v\ae re begrenset, slik   
at dersom dette leddet er med i b\o lgefunksjonen betyr det at 
sannsynligheten for \aa \ finne partikkelen uendelig langt ute 
i det klassisk forbudte omr\aa det er uendelig stor!  Dette 
er ufysikalsk og uakseptabelt.  Vi m\aa \ forlange at dette 
leddet ikke er med i b\o lgefunksjonen, dvs. $A=0$.  
Konklusjon: 
\be
\psi_I(x)=Be^{-\beta x}.
\ee

Schr\"odingers likning  i omr\aa de II er: 
\be 
-\frac{\hbar^2}{2m}\frac{d^2}{dx^2}\psi_{II}(x)-V_0\psi_{II}(x)=
-|E|\psi_{II}(x)
\ee
som vi skriver som 
\be
\psi_{II}''(x)=-\frac{2m}{\hbar^2}(V_0-|E|)\psi_{II}(x)=-k^2\psi_{II}(x),
\ee
der $k^2=2m(V_0-|E|)/\hbar^2$.  Vi har at $k^2>0$ siden $V_0>|E|$.  
At det siste er tilfellet kan vi se ved en klassisk energibetraktning:
I omr\aa de II har partikkelen energi 
\be
 E=\frac{p^2}{2m}-V_0.
\ee
Bidraget fra kinetisk energi, $p^2/2m \ge 0$, slik at 
\be
E \ge -V_0.
\ee
Siden $E=-|E|<0$ medf\o rer dette at 
\be
V_0-|E| \ge 0.
\ee
Dette gjelder ogs\aa \ kvantemekanisk.

Differensiallikningen v\aa r har da som generell l\o sning 
\be
\psi_{II}(x)=C\sin(kx)+D\cos(kx) 
\ee
der $C$ og $D$ er integrasjonskonstanter.  

La oss s\aa \ se p\aa \ omr\aa de III.  Her er $V(x)=+\infty$ 
overalt.  
Som diskutert i forrige 
avsnitt svarer det  til 
\be
\psi_{III}(x)\equiv 0
\ee
for alle $x \le 0$.  

I f\o lge krav 1 til b\o lgefunksjonen m\aa \ $\psi$ v\ae re 
kontinuerlig i $x=0$.  Det betyr at 
\be
\psi_{II}(0)=\psi_{III}(0)=0
\ee
og dermed m\aa \ $D=0$.  Dermed er l\o sningen i omr\aa de II 
gitt ved 
\be
\psi_{II}(x)=C\sin(kx).
\ee
Siden $V(x)$ er uendelig for $x\le 0$ kan vi ikke kreve at 
$\psi$ skal v\ae re kontinuerlig deriverbar i $x=0$.

I punktet $x=a$ har potensialet en {\it endelig} diskontinuitet.  
Der m\aa \ b\aa de krav 1 og 3 til b\o lgefunksjonen v\ae re 
oppfylt: 
\begin{eqnarray}
 \psi_I(a)&=&\psi_{II}(a) \nonumber \\ 
 \psi_I'(a)&=&\psi_{II}'(a), \nonumber 
\end{eqnarray}
som gir 
\be
 Be^{-\beta a}=-C\sin(ka)
\label{eq:krav1}
\ee
\be
-\beta B e^{-\beta a}= kC\cos(ka)
\label{eq:krav2}
\ee
Vi kan finne  konstantene $B$ og $C$ 
vha.~likningene (\ref{eq:krav2}) og (\ref{eq:krav1}), dvs.
\be
 B=-C\sin(ka)e^{\beta a},
\ee
som innsatt i likning (\ref{eq:krav2}) gir
\be
k\cot(ka)=-\beta.
\label{eq:egenverdi}
\ee
Denne likningen bestemmer de tillatte energiene for partikkelen.  
Siden $\hbar\beta=\sqrt{2m|E|}$ og $\hbar k=\sqrt{2m(V_0-|E|)}$ 
ser vi nemlig at den eneste ukjente st\o rrelsen som inng\aa r 
i likning (\ref{eq:egenverdi}) er energien $E$.  


N\aa \ skal vi fors\o ke \aa \ finne l\o sningene av likning 
(\ref{eq:egenverdi}).  Denne likningen er dessverre en 
s\aa kalt transcendent likning, i praksis betyr det at vi 
m\aa \ finne l\o sningene numerisk eller grafisk.  
Det er det siste vi skal fors\o ke her.  Men f\o rst 
skriver vi den om litt.  Vi multipliserer f\o rst begge sider 
med $a$ slik at 
\be
ka\cot(ka)=-\beta a.
\ee
La oss videre innf\o re $\eta=ka$.
N\aa \ ser vi at 
\be
\eta^2=(ka)^2=\frac{2ma^2V_0}{\hbar^2}-\frac{2ma^2|E|}{\hbar^2}
=\alpha-(\beta a)^2, 
\ee
der $\alpha=2ma^2V_0/\hbar^2$, og dermed 
\be
(\beta a)^2=\alpha-\eta^2.
\ee
Likningen kan derfor skrives 
\be
\eta \cot \eta=\eta\frac{\cos\eta}{\sin\eta}=-\sqrt{\alpha-\eta^2}.  
\label{eq:nyegen}
\ee
Vi kvadrerer begge sider og utnytter at $\cos^2\eta=1-\sin^2\eta$.  
Det gir 
\begin{eqnarray} 
\eta^2-\eta^2\sin^2\eta&=&\alpha\sin^2\eta-\eta^2\sin^2\eta \nonumber \\
\sin^2\eta&=&\frac{\eta^2}{\alpha} \nonumber \\
\sin\eta&=&\pm\frac{\eta}{\sqrt{\alpha}}=\pm K\eta, \label{eq:grafisketa}
\end{eqnarray} 
der vi har definert $K=1/\sqrt{\alpha}$.  N\aa r vi kvadrerer 
en likning risikerer vi imidlertid at vi f\aa r med oss falske 
l\o sninger.  Vi m\aa \ g\aa \ tilbake til den opprinnelige likningen 
(\ref{eq:nyegen}) for \aa \ se hva slags l\o sninger vi kan godta.  Siden 
$\eta=ka>0$ ser vi at vi m\aa \ ha $\cos\eta/\sin\eta<0$ for at 
likningen skal v\ae re oppfylt.  Det betyr at $\cos\eta$ og $\sin\eta$ 
m\aa \ ha motsatt fortegn.  Dette skjer n\aa r $\eta$ ligger i 
intervallene $[\pi/2,\pi]$,$[3\pi/2,2\pi]$, $[5\pi/2,3\pi]$ osv.  

I Figur \ref{fig:etafig}
er funksjonene $f(\eta)=\sin\eta$, $g^{+}(\eta)=C\eta$ 
og $g^{-}(\eta)=-C\eta$ skissert for forskjellige verdier av $C$.  
Energiegenverdiene er gitt som skj\ae ringspunktene mellom 
$f$ og $g^{\pm}$, forutsatt at de ligger innenfor de tillatte intervallene 
for $\eta$.  Vi ser at antall egenverdier varierer med $C$, n\aa r 
$C$ avtar, som svarer til \aa \ \o ke styrken p\aa \ $V_0$, 
f\aa r vi flere l\o sninger.  
\begin{figure}[hbtp]
\begin{center}
{\centering
\mbox{\psfig{figure=grafisk.ps,height=12cm,width=14cm,angle=0}}
}
\caption{Grafisk l\o sning av likning (\ref{eq:grafisketa}). \label{fig:etafig}}
\end{center}
\end{figure}
Vi skal ogs\aa \ bestemme verdien av $C$ som gir (minst) \'{e}n 
bundet tilstand.  Den minste $\eta$-verdien som er en akseptabel 
l\o sning av likningen er $\eta=\pi/2$.  I dette punktet er 
$\sin\eta=1$.   Linjen som g\aa r gjennom origo og skj\ae rer 
$f$ i punktet $(\pi/2,1)$ har likning
\be
 g^{+}(\eta)=\frac{2}{\pi}\eta, 
\ee
som alts\aa \ svarer til $C=2/\pi$.  Vi ser av figuren at 
dersom $C>2/\pi$ vil vi ikke ha noen l\o sning, mens vi for 
$C\le 2/\pi$  alltid har minst \'{e}n l\o sning.  
$C=2/\pi$ svarer til, siden $C=1/\sqrt{\alpha}$ og 
$\alpha=2ma^2V_0/\hbar^2$, 
\be
V_0=\frac{\hbar^2\pi^2}{8ma^2}.
\ee
Dersom $V_0$ er st\o rre enn denne verdien har vi minst \'{e}n 
bundet tilstand, hvis $V_0$ er mindre har vi ingen.  

I grensen $V_0\rightarrow \infty$ vil likningen for egenverdiene 
bli 
\be
\sin\eta=0
\ee
som gir 
\be
\eta=ka=n\pi,\; n=1,2,3,\ldots
\ee
som l\o sninger.  S\aa \ 
\be
k^2=\frac{2m}{\hbar^2}(V_0-|E|)=\frac{n^2\pi^2}{a^2}
\ee
som gir (husk at $E=-|E|$) 
\be
E=\frac{\hbar^2\pi^2 n^2}{2ma^2}-V_0.  
\ee
Ved \aa \ definere nullpunkt for energien i $-V_0$ kan vi sette 
$E'=E+V_0$, og ved s\aa \ \aa \ la $V_0\rightarrow \infty$ 
ser vi at den fysiske situasjon er n\o yaktig den samme som 
for en partikkel i en uendelig potensialbr\o nn med lengde $a$, 
og energiegenverdiene er 
\be
E=\frac{\hbar^2\pi^2n^2}{2ma^2},\;n=1,2,3,\ldots
\ee
slik de b\o r v\ae re. Legg merke til at i dette tilfellet er
potensialets utstrekning gitt ved $a$, mens for uendelig
potensial brukte vi en  utstrekning gitt ved $2a$. Setter vi inn
denne utstrekningen i uttrykket for energien i likning (\ref{eq:uendeligpote}),
finner vi at 
\[
E=\frac{\hbar^2\pi^2n^2}{8ma^2},\;n=1,2,3,\ldots
\]

\section{Deuterium og endelig bokspotensial}
N\aa\ kan vi pr\o ve oss p\aa\ en anvendelse
av resultatene fra den endelige potensialbr\o nnen p\aa\ et virkelig
fysisk system. Vi skal selvsagt foreta en del forenklinger, men 
resulatet v\aa rt kommer ikke til \aa\ v\ae re s\aa\ fjernt fra
eksperimentelle st\o rrelser. I tillegg, skal vi pr\o ve oss p\aa\
en numerisk l\o sning vha.~Maple.

Systement vi skal ta for oss er deuterium\footnote{Utfyllende
lesning finner dere i kap 14.8, sidene 705-710.}, som best\aa r av et 
proton og et n\o ytron. Deuterium har en bindingsenergi p\aa\
$-2.2$ MeV, dvs.~at protonet og n\o ytronet holder sammen pga.~tiltrekningen
fra de sterke kjernekreftene (ikke Coulomb, siden n\o ytronet har null ladning ).

Det f\o rste vi skal gj\o re er \aa\ forenkle vekselvirkningen mellom
et proton og et n\o ytron til \aa\ v\ae re gitt ved et kassepotensial
som vist i Figur \ref{fig:endeligkasse}. Variabelen $x$ er n\aa\ avstanden mellom
protonet og n\o ytronet. Vi har dermed valgt et massesenter mellom
protonet og n\o ytronet, slik at $x$ er n\aa\ den relative avstanden
mellom protonet og n\o ytronet. Merk at vi fremdeles betrakter dette
problemet som et en-dimensjonalt problem.

Den reduserte massen blir
\be
   \mu=\frac{m_pm_n}{m_n+m_p},
\ee
hvor $m_n$ er massen til n\o ytronet og $m_p$ er massen til protonet.
Her skal vi neglisjere den lille masseforksjellen som finnes mellom
protonet og n\o ytronet og sette $m=m_n=m_p=938$ MeV/c$^2$. Det gir en redusert
masse
\be
    \mu=\frac{m}{2}.
\ee
Siden vi n\aa\ bruker den reduserte massen og den relative avstanden $x$,
blir den en-dimensjonal tidsuavhengige Schr\"odingers likning  gitt ved
\be 
-\frac{\hbar^2}{m}\frac{d^2}{dx^2}\psi(x)-V_0\psi(x)=
-|E|\psi(x),
\ee
i omr\aa de II fra Figur \ref{fig:endeligkasse}.
Bruker vi s\aa\ resultatene fra likning (\ref{eq:egenverdi})
\[
k\cot(ka)+\beta=0
\]
hvor
$\hbar\beta=\sqrt{m|E|}$ og $\hbar k=\sqrt{m(V_0-|E|)}$, 
har vi en transcendent likning til bestemmelsen av $E$, med gitt
$V_0$ og utstrekning $a$.

F\o r vi pr\o ver oss p\aa\ en grafisk og numerisk l\o sning,
stiller vi oss sp\o rsm\aa let hva er den minste verdien $V_0$ kan
ha for at vi ihvertfall skal ha en bunden tilstand? 
Da m\aa\ $E \le 0$. Setter vi $E=0$ finner vi den minste verdien.
Det gir da
\be
k\cot(ka)=-\beta=-\frac{\sqrt{m|E|}}{\hbar}=0,
\ee
som resulterer i 
\be
   k\cot(ka)=\frac{\sqrt{mV_0}}{\hbar}\cot(\frac{\sqrt{mV_0}}{\hbar}a)=0,
\ee
eller
\be
   \frac{\sqrt{mV_0}}{\hbar}a=\frac{\pi}{2},
\ee
som gir
\be
   V_0=\frac{\hbar^2\pi^2}{4ma^2},
\ee
et uttrykk som likner mistenkelig p\aa\ likning (\ref{eq:uendeligpote})!
Setter vi inn $mc^2=938$ MeV og $\hbar c=197$ MeVfm finner vi
\be
   V_0= \frac{102}{a^2} \hspace{0.1cm} \mathrm{MeV}.
\ee
Utstrekningen  til den sterke vekselvirkningen mellom protoner
og n\o ytroner er p\aa\ noen f\aa\ femtometer. Velger vi 
$a=1.6$ fm, finner vi  $V_0=40$ MeV. 
Vi er i riktig omr\aa de utifra eksperiment
for styrken til vekselvirkningen. Dersom vi vil ha en bunden
tilstand p\aa\ $-2.2$ MeV m\aa\ $V_0$ v\ae re st\o rre enn $40$ MeV.

La oss n\aa\ velge $V_0=52$ MeV og $a=1.6$ fm og pr\o ve \aa\ l\o se
den transcendente likning $k\cot(ka)-\beta=0$.
Vi definerer alts\aa\  en funksjon
\be
   f(E)=\frac{\sqrt{m(V_0-|E|)}}{\hbar}\cot(\frac{\sqrt{m(V_0-|E|)}}{\hbar}a)-\frac{\sqrt{m|E|}}{\hbar}, 
   \label{eq:grafiskfe}
\ee
og sp\o r hvor denne funksjonen skj\ae rer $E$-aksen. Der den skj\ae rer
$E$-aksen har vi l\o sninga. 
Figuren nedenfor viser et plott av denne funksjonen for de valgte verdiene av
$V_0$ og $a$. 
\begin{figure}[hbtp]
\begin{center}
% GNUPLOT: LaTeX picture with Postscript
\begingroup%
  \makeatletter%
  \newcommand{\GNUPLOTspecial}{%
    \@sanitize\catcode`\%=14\relax\special}%
  \setlength{\unitlength}{0.1bp}%
{\GNUPLOTspecial{!
%!PS-Adobe-2.0 EPSF-2.0
%%Title: fig1chap8.tex
%%Creator: gnuplot 3.7 patchlevel 0.2
%%CreationDate: Wed Mar  1 14:23:49 2000
%%DocumentFonts: 
%%BoundingBox: 0 0 360 216
%%Orientation: Landscape
%%EndComments
/gnudict 256 dict def
gnudict begin
/Color false def
/Solid false def
/gnulinewidth 5.000 def
/userlinewidth gnulinewidth def
/vshift -33 def
/dl {10 mul} def
/hpt_ 31.5 def
/vpt_ 31.5 def
/hpt hpt_ def
/vpt vpt_ def
/M {moveto} bind def
/L {lineto} bind def
/R {rmoveto} bind def
/V {rlineto} bind def
/vpt2 vpt 2 mul def
/hpt2 hpt 2 mul def
/Lshow { currentpoint stroke M
  0 vshift R show } def
/Rshow { currentpoint stroke M
  dup stringwidth pop neg vshift R show } def
/Cshow { currentpoint stroke M
  dup stringwidth pop -2 div vshift R show } def
/UP { dup vpt_ mul /vpt exch def hpt_ mul /hpt exch def
  /hpt2 hpt 2 mul def /vpt2 vpt 2 mul def } def
/DL { Color {setrgbcolor Solid {pop []} if 0 setdash }
 {pop pop pop Solid {pop []} if 0 setdash} ifelse } def
/BL { stroke userlinewidth 2 mul setlinewidth } def
/AL { stroke userlinewidth 2 div setlinewidth } def
/UL { dup gnulinewidth mul /userlinewidth exch def
      10 mul /udl exch def } def
/PL { stroke userlinewidth setlinewidth } def
/LTb { BL [] 0 0 0 DL } def
/LTa { AL [1 udl mul 2 udl mul] 0 setdash 0 0 0 setrgbcolor } def
/LT0 { PL [] 1 0 0 DL } def
/LT1 { PL [4 dl 2 dl] 0 1 0 DL } def
/LT2 { PL [2 dl 3 dl] 0 0 1 DL } def
/LT3 { PL [1 dl 1.5 dl] 1 0 1 DL } def
/LT4 { PL [5 dl 2 dl 1 dl 2 dl] 0 1 1 DL } def
/LT5 { PL [4 dl 3 dl 1 dl 3 dl] 1 1 0 DL } def
/LT6 { PL [2 dl 2 dl 2 dl 4 dl] 0 0 0 DL } def
/LT7 { PL [2 dl 2 dl 2 dl 2 dl 2 dl 4 dl] 1 0.3 0 DL } def
/LT8 { PL [2 dl 2 dl 2 dl 2 dl 2 dl 2 dl 2 dl 4 dl] 0.5 0.5 0.5 DL } def
/Pnt { stroke [] 0 setdash
   gsave 1 setlinecap M 0 0 V stroke grestore } def
/Dia { stroke [] 0 setdash 2 copy vpt add M
  hpt neg vpt neg V hpt vpt neg V
  hpt vpt V hpt neg vpt V closepath stroke
  Pnt } def
/Pls { stroke [] 0 setdash vpt sub M 0 vpt2 V
  currentpoint stroke M
  hpt neg vpt neg R hpt2 0 V stroke
  } def
/Box { stroke [] 0 setdash 2 copy exch hpt sub exch vpt add M
  0 vpt2 neg V hpt2 0 V 0 vpt2 V
  hpt2 neg 0 V closepath stroke
  Pnt } def
/Crs { stroke [] 0 setdash exch hpt sub exch vpt add M
  hpt2 vpt2 neg V currentpoint stroke M
  hpt2 neg 0 R hpt2 vpt2 V stroke } def
/TriU { stroke [] 0 setdash 2 copy vpt 1.12 mul add M
  hpt neg vpt -1.62 mul V
  hpt 2 mul 0 V
  hpt neg vpt 1.62 mul V closepath stroke
  Pnt  } def
/Star { 2 copy Pls Crs } def
/BoxF { stroke [] 0 setdash exch hpt sub exch vpt add M
  0 vpt2 neg V  hpt2 0 V  0 vpt2 V
  hpt2 neg 0 V  closepath fill } def
/TriUF { stroke [] 0 setdash vpt 1.12 mul add M
  hpt neg vpt -1.62 mul V
  hpt 2 mul 0 V
  hpt neg vpt 1.62 mul V closepath fill } def
/TriD { stroke [] 0 setdash 2 copy vpt 1.12 mul sub M
  hpt neg vpt 1.62 mul V
  hpt 2 mul 0 V
  hpt neg vpt -1.62 mul V closepath stroke
  Pnt  } def
/TriDF { stroke [] 0 setdash vpt 1.12 mul sub M
  hpt neg vpt 1.62 mul V
  hpt 2 mul 0 V
  hpt neg vpt -1.62 mul V closepath fill} def
/DiaF { stroke [] 0 setdash vpt add M
  hpt neg vpt neg V hpt vpt neg V
  hpt vpt V hpt neg vpt V closepath fill } def
/Pent { stroke [] 0 setdash 2 copy gsave
  translate 0 hpt M 4 {72 rotate 0 hpt L} repeat
  closepath stroke grestore Pnt } def
/PentF { stroke [] 0 setdash gsave
  translate 0 hpt M 4 {72 rotate 0 hpt L} repeat
  closepath fill grestore } def
/Circle { stroke [] 0 setdash 2 copy
  hpt 0 360 arc stroke Pnt } def
/CircleF { stroke [] 0 setdash hpt 0 360 arc fill } def
/C0 { BL [] 0 setdash 2 copy moveto vpt 90 450  arc } bind def
/C1 { BL [] 0 setdash 2 copy        moveto
       2 copy  vpt 0 90 arc closepath fill
               vpt 0 360 arc closepath } bind def
/C2 { BL [] 0 setdash 2 copy moveto
       2 copy  vpt 90 180 arc closepath fill
               vpt 0 360 arc closepath } bind def
/C3 { BL [] 0 setdash 2 copy moveto
       2 copy  vpt 0 180 arc closepath fill
               vpt 0 360 arc closepath } bind def
/C4 { BL [] 0 setdash 2 copy moveto
       2 copy  vpt 180 270 arc closepath fill
               vpt 0 360 arc closepath } bind def
/C5 { BL [] 0 setdash 2 copy moveto
       2 copy  vpt 0 90 arc
       2 copy moveto
       2 copy  vpt 180 270 arc closepath fill
               vpt 0 360 arc } bind def
/C6 { BL [] 0 setdash 2 copy moveto
      2 copy  vpt 90 270 arc closepath fill
              vpt 0 360 arc closepath } bind def
/C7 { BL [] 0 setdash 2 copy moveto
      2 copy  vpt 0 270 arc closepath fill
              vpt 0 360 arc closepath } bind def
/C8 { BL [] 0 setdash 2 copy moveto
      2 copy vpt 270 360 arc closepath fill
              vpt 0 360 arc closepath } bind def
/C9 { BL [] 0 setdash 2 copy moveto
      2 copy  vpt 270 450 arc closepath fill
              vpt 0 360 arc closepath } bind def
/C10 { BL [] 0 setdash 2 copy 2 copy moveto vpt 270 360 arc closepath fill
       2 copy moveto
       2 copy vpt 90 180 arc closepath fill
               vpt 0 360 arc closepath } bind def
/C11 { BL [] 0 setdash 2 copy moveto
       2 copy  vpt 0 180 arc closepath fill
       2 copy moveto
       2 copy  vpt 270 360 arc closepath fill
               vpt 0 360 arc closepath } bind def
/C12 { BL [] 0 setdash 2 copy moveto
       2 copy  vpt 180 360 arc closepath fill
               vpt 0 360 arc closepath } bind def
/C13 { BL [] 0 setdash  2 copy moveto
       2 copy  vpt 0 90 arc closepath fill
       2 copy moveto
       2 copy  vpt 180 360 arc closepath fill
               vpt 0 360 arc closepath } bind def
/C14 { BL [] 0 setdash 2 copy moveto
       2 copy  vpt 90 360 arc closepath fill
               vpt 0 360 arc } bind def
/C15 { BL [] 0 setdash 2 copy vpt 0 360 arc closepath fill
               vpt 0 360 arc closepath } bind def
/Rec   { newpath 4 2 roll moveto 1 index 0 rlineto 0 exch rlineto
       neg 0 rlineto closepath } bind def
/Square { dup Rec } bind def
/Bsquare { vpt sub exch vpt sub exch vpt2 Square } bind def
/S0 { BL [] 0 setdash 2 copy moveto 0 vpt rlineto BL Bsquare } bind def
/S1 { BL [] 0 setdash 2 copy vpt Square fill Bsquare } bind def
/S2 { BL [] 0 setdash 2 copy exch vpt sub exch vpt Square fill Bsquare } bind def
/S3 { BL [] 0 setdash 2 copy exch vpt sub exch vpt2 vpt Rec fill Bsquare } bind def
/S4 { BL [] 0 setdash 2 copy exch vpt sub exch vpt sub vpt Square fill Bsquare } bind def
/S5 { BL [] 0 setdash 2 copy 2 copy vpt Square fill
       exch vpt sub exch vpt sub vpt Square fill Bsquare } bind def
/S6 { BL [] 0 setdash 2 copy exch vpt sub exch vpt sub vpt vpt2 Rec fill Bsquare } bind def
/S7 { BL [] 0 setdash 2 copy exch vpt sub exch vpt sub vpt vpt2 Rec fill
       2 copy vpt Square fill
       Bsquare } bind def
/S8 { BL [] 0 setdash 2 copy vpt sub vpt Square fill Bsquare } bind def
/S9 { BL [] 0 setdash 2 copy vpt sub vpt vpt2 Rec fill Bsquare } bind def
/S10 { BL [] 0 setdash 2 copy vpt sub vpt Square fill 2 copy exch vpt sub exch vpt Square fill
       Bsquare } bind def
/S11 { BL [] 0 setdash 2 copy vpt sub vpt Square fill 2 copy exch vpt sub exch vpt2 vpt Rec fill
       Bsquare } bind def
/S12 { BL [] 0 setdash 2 copy exch vpt sub exch vpt sub vpt2 vpt Rec fill Bsquare } bind def
/S13 { BL [] 0 setdash 2 copy exch vpt sub exch vpt sub vpt2 vpt Rec fill
       2 copy vpt Square fill Bsquare } bind def
/S14 { BL [] 0 setdash 2 copy exch vpt sub exch vpt sub vpt2 vpt Rec fill
       2 copy exch vpt sub exch vpt Square fill Bsquare } bind def
/S15 { BL [] 0 setdash 2 copy Bsquare fill Bsquare } bind def
/D0 { gsave translate 45 rotate 0 0 S0 stroke grestore } bind def
/D1 { gsave translate 45 rotate 0 0 S1 stroke grestore } bind def
/D2 { gsave translate 45 rotate 0 0 S2 stroke grestore } bind def
/D3 { gsave translate 45 rotate 0 0 S3 stroke grestore } bind def
/D4 { gsave translate 45 rotate 0 0 S4 stroke grestore } bind def
/D5 { gsave translate 45 rotate 0 0 S5 stroke grestore } bind def
/D6 { gsave translate 45 rotate 0 0 S6 stroke grestore } bind def
/D7 { gsave translate 45 rotate 0 0 S7 stroke grestore } bind def
/D8 { gsave translate 45 rotate 0 0 S8 stroke grestore } bind def
/D9 { gsave translate 45 rotate 0 0 S9 stroke grestore } bind def
/D10 { gsave translate 45 rotate 0 0 S10 stroke grestore } bind def
/D11 { gsave translate 45 rotate 0 0 S11 stroke grestore } bind def
/D12 { gsave translate 45 rotate 0 0 S12 stroke grestore } bind def
/D13 { gsave translate 45 rotate 0 0 S13 stroke grestore } bind def
/D14 { gsave translate 45 rotate 0 0 S14 stroke grestore } bind def
/D15 { gsave translate 45 rotate 0 0 S15 stroke grestore } bind def
/DiaE { stroke [] 0 setdash vpt add M
  hpt neg vpt neg V hpt vpt neg V
  hpt vpt V hpt neg vpt V closepath stroke } def
/BoxE { stroke [] 0 setdash exch hpt sub exch vpt add M
  0 vpt2 neg V hpt2 0 V 0 vpt2 V
  hpt2 neg 0 V closepath stroke } def
/TriUE { stroke [] 0 setdash vpt 1.12 mul add M
  hpt neg vpt -1.62 mul V
  hpt 2 mul 0 V
  hpt neg vpt 1.62 mul V closepath stroke } def
/TriDE { stroke [] 0 setdash vpt 1.12 mul sub M
  hpt neg vpt 1.62 mul V
  hpt 2 mul 0 V
  hpt neg vpt -1.62 mul V closepath stroke } def
/PentE { stroke [] 0 setdash gsave
  translate 0 hpt M 4 {72 rotate 0 hpt L} repeat
  closepath stroke grestore } def
/CircE { stroke [] 0 setdash 
  hpt 0 360 arc stroke } def
/Opaque { gsave closepath 1 setgray fill grestore 0 setgray closepath } def
/DiaW { stroke [] 0 setdash vpt add M
  hpt neg vpt neg V hpt vpt neg V
  hpt vpt V hpt neg vpt V Opaque stroke } def
/BoxW { stroke [] 0 setdash exch hpt sub exch vpt add M
  0 vpt2 neg V hpt2 0 V 0 vpt2 V
  hpt2 neg 0 V Opaque stroke } def
/TriUW { stroke [] 0 setdash vpt 1.12 mul add M
  hpt neg vpt -1.62 mul V
  hpt 2 mul 0 V
  hpt neg vpt 1.62 mul V Opaque stroke } def
/TriDW { stroke [] 0 setdash vpt 1.12 mul sub M
  hpt neg vpt 1.62 mul V
  hpt 2 mul 0 V
  hpt neg vpt -1.62 mul V Opaque stroke } def
/PentW { stroke [] 0 setdash gsave
  translate 0 hpt M 4 {72 rotate 0 hpt L} repeat
  Opaque stroke grestore } def
/CircW { stroke [] 0 setdash 
  hpt 0 360 arc Opaque stroke } def
/BoxFill { gsave Rec 1 setgray fill grestore } def
end
%%EndProlog
}}%
\begin{picture}(3600,2160)(0,0)%
{\GNUPLOTspecial{"
gnudict begin
gsave
0 0 translate
0.100 0.100 scale
0 setgray
newpath
1.000 UL
LTb
450 300 M
63 0 V
2937 0 R
-63 0 V
450 740 M
63 0 V
2937 0 R
-63 0 V
450 1180 M
63 0 V
2937 0 R
-63 0 V
450 1620 M
63 0 V
2937 0 R
-63 0 V
450 2060 M
63 0 V
2937 0 R
-63 0 V
450 300 M
0 63 V
0 1697 R
0 -63 V
1050 300 M
0 63 V
0 1697 R
0 -63 V
1650 300 M
0 63 V
0 1697 R
0 -63 V
2250 300 M
0 63 V
0 1697 R
0 -63 V
2850 300 M
0 63 V
0 1697 R
0 -63 V
3450 300 M
0 63 V
0 1697 R
0 -63 V
1.000 UL
LTb
450 300 M
3000 0 V
0 1760 V
-3000 0 V
450 300 L
1.000 UL
LT0
3087 1947 M
263 0 V
450 739 M
30 63 V
31 27 V
30 21 V
30 19 V
31 16 V
30 15 V
30 14 V
30 13 V
31 12 V
30 12 V
30 11 V
31 11 V
30 11 V
30 10 V
31 10 V
30 9 V
30 10 V
30 9 V
31 9 V
30 9 V
30 9 V
31 8 V
30 8 V
30 9 V
31 8 V
30 8 V
30 7 V
30 8 V
31 8 V
30 7 V
30 8 V
31 7 V
30 7 V
30 8 V
31 7 V
30 7 V
30 7 V
31 7 V
30 6 V
30 7 V
30 7 V
31 6 V
30 7 V
30 7 V
31 6 V
30 6 V
30 7 V
31 6 V
30 6 V
30 7 V
30 6 V
31 6 V
30 6 V
30 6 V
31 6 V
30 6 V
30 6 V
31 6 V
30 6 V
30 6 V
30 5 V
31 6 V
30 6 V
30 6 V
31 5 V
30 6 V
30 6 V
31 5 V
30 6 V
30 5 V
31 6 V
30 5 V
30 6 V
30 5 V
31 6 V
30 5 V
30 5 V
31 6 V
30 5 V
30 5 V
31 5 V
30 6 V
30 5 V
30 5 V
31 5 V
30 5 V
30 6 V
31 5 V
30 5 V
30 5 V
31 5 V
30 5 V
30 5 V
30 5 V
31 5 V
30 5 V
30 5 V
31 5 V
30 5 V
stroke
grestore
end
showpage
}}%
\put(3037,1947){\makebox(0,0)[r]{f(E)}}%
\put(1950,20){\makebox(0,0){$|E|$ [MeV]}}%
\put(100,1180){%
\special{ps: gsave currentpoint currentpoint translate
270 rotate neg exch neg exch translate}%
\makebox(0,0)[b]{\shortstack{$f(E)$ [MeV]}}%
\special{ps: currentpoint grestore moveto}%
}%
\put(3450,200){\makebox(0,0){5}}%
\put(2850,200){\makebox(0,0){4}}%
\put(2250,200){\makebox(0,0){3}}%
\put(1650,200){\makebox(0,0){2}}%
\put(1050,200){\makebox(0,0){1}}%
\put(450,200){\makebox(0,0){0}}%
\put(400,2060){\makebox(0,0)[r]{100}}%
\put(400,1620){\makebox(0,0)[r]{50}}%
\put(400,1180){\makebox(0,0)[r]{0}}%
\put(400,740){\makebox(0,0)[r]{-50}}%
\put(400,300){\makebox(0,0)[r]{-100}}%
\end{picture}%
\endgroup
\endinput

\end{center}
\caption{Grafisk l\o sning av likning (\ref{eq:grafiskfe}).}
\end{figure}

Utifra figuren tipper vi at v\aa rt valg av verdier gir en l\o sning
n\aa r $|E|\sim 2$, som ikke er s\aa\ v\ae rst n\aa r vi tenker 
at den eksperimentelle verdien er $|E|=2.2$ MeV. Verdiene vi har valgt for
$V_0$ og $a$ er ogs\aa\ n\ae r det eksperiment forteller oss.
Alt i alt kan vi si at dette forenklede potensialet, som kan sees
p\aa\ som en f\o rste approksimasjon til den virkelige vekselvirkningen,
gir resultat med riktig st\o rrelsesorden med kun to parametre, verdien
p\aa\ potensialet og dets utstrekning.

Den grafiske l\o sningen gir oss kun en pekepinn p\aa\ st\o rrelsen
til $E$. \O nsker vi et mer presist svar m\aa\ vi ty til numerisk
innsats. Det finnes et vell av metoder for \aa\ finne nullpunktet
av 
\[
  f(E)=\frac{\sqrt{m(V_0-|E|)}}{\hbar}\cot(\frac{\sqrt{m(V_0-|E|)}}{\hbar}a)-\frac{\sqrt{m|E|}}{\hbar}=0 
\]
\subsection{Eksempel p\aa\ l\o sning for harmonisk oscillator vha.~Maple}
Dersom vi \o nsker \aa\ bruke Maple p\aa\ sitt enkleste vis, holder
det med \aa\ definere funksjonen $f(E)$ med sine variable og l\o se
den ovennevnte likning vha.~Maple funksjonen {\bf fsolve}. 
Men, Maple tillater oss ogs\aa\ \aa\ definere egne algoritmer
vha.~prosedyre utsagnet {\bf proc}. I et slikt tilfelle m\aa\ vi 
velge en bestemt numerisk metode for s\o king etter nullpunkter.
Vi velger Newtons metode og lager en prosedyre i Maple.
Newtons metode kan beskrives som f\o lger
\begin{itemize}
\item Vi finner f\o rst et punkt p\aa\ E-aksen som vi antar 
er n\ae r l\o sningen, jfr.\ den grafiske l\o sningen ovenfor.
\item Vi finner krumningen til kurven i dette punktet.
\item Vi trekker tangenten ved dette punktet og ser hvor 
tangenten treffer E-aksen. Det sistnevnte gir oss en ny E verdi som
er n\ae rmere l\o sningen. Det nye punktet $E_1$ kan vi s\aa\ bruke
til \aa\ bestemme et nytt punkt.
\item Slik kan en fortsette til en finner den numeriske verdien
for $E$ hvor den ovennevnte likning blir null.
\item Matematisk kan vi uttrykke denne iterasjons prosessen som
     \[
        E_{i+1}=E_i-\frac{f(E_i)}{f'(E_i)},
     \]
hvor $f$ er selve funksjonen.
\end{itemize}
I Maple kan vi skrive dette som 
\begin{maplegroup}
\begin{mapleinput}
\mapleinline{active}{1d}{MakeIteration:= proc(expr::algebraic,
x::name)}{%
}
\end{mapleinput}

\begin{mapleinput}
\mapleinline{active}{1d}{    local iteration;}{%
}
\end{mapleinput}

\begin{mapleinput}
\mapleinline{active}{1d}{    iteration:=x-expr/diff(expr,x);}{%
}
\end{mapleinput}

\begin{mapleinput}
\mapleinline{active}{1d}{    unapply(iteration, x);}{%
}
\end{mapleinput}

\end{maplegroup}
Uttrykket {\it expr} er selve funksjonen v\aa r 
\begin{maplegroup}
\begin{mapleinput}
\mapleinline{active}{1d}{expr:=f(x);}
{%
}
\end{mapleinput}

\end{maplegroup}
hvor en m\aa\ definere f\o rst selve funksjonen $f(x)$ i Maple.
Det er ikke gjort her.
\begin{maplegroup}
\begin{mapleinput}
\mapleinline{active}{1d}{Newton:=MakeIteration(expr, xi);}{%
}
\end{mapleinput}
\end{maplegroup}
Fra grafen finner vi en startverdi for iterasjonen {\it guess}
\begin{maplegroup}
\begin{mapleinput}
\mapleinline{active}{1d}{guess:=1.5;}{%
}
\end{mapleinput}

\mapleresult
\begin{maplelatex}
\[
\mathit{guess} := 1.5
\]
\end{maplelatex}

\end{maplegroup}
\begin{maplegroup}
\begin{mapleinput}
\mapleinline{active}{1d}{to 3. do guess:=Newton(guess); od;}{%
}
\end{mapleinput}

\end{maplegroup}
Dette Maple  programmet gir oss en 
bindingsenergi p\aa\ $|E|=1.9$ MeV, som er n\ae r
den eksperimentelle energiverdien for grunntilstanden til deuterium.
Vi kan selvsagt finjustere $V_0$ til vi treffer akkurat den eksperimentelle
verdien. Men en skal ikke t\o ye dette enkle potensialet for langt.

\subsection{Eksempel p\aa\ l\o sning for harmonisk oscillator vha.~Matlab}

I Matlab kan vi bruke funksjonen {\bf fzero} som nullpunkt til en funksjon med en variabel.
Eksemplet her viser hvordan dette kan kodes i Matlab. Vi har valgt en prototyp funksjon
\[
   xcot(x)+\sqrt{x} = 0,
\]
\[
k\cot(ka)+\beta=0.
\] 
Lag deretter 
en Matlab {\bf M}-fil som ser slik ut 
\begin{verbatim}
%% Funksjonen cotx
%% funksjoner inn i samme fil)
function f = cotx
% Finn null n�r denne verdien
omtrentnull = 2.;
% @cotx betyr at matlab skal finne funksjonen med navnet cosx
f = fzero(@cotx,omtrentnull);
%% Funksjonen vi skal finne nullpunktet til
function y = cotx
y = x*cotx(x)+sqrt(x);
\end{verbatim}

\section{Harmonisk oscillator potensial}

Et potensial som ofte anvendes i fysikk er den harmoniske
oscillator. I kvantemekanikk finnes det flere fagfelt 
hvor dette potensialet inng\aa r som en god f\o rste approksimasjon,
fra kjernefysikk til modeller for vibrerende molekyler. 
Utviklingen i ionefelle innfangningsteknikker har ogs\aa\ gitt en renessanse
for den harmoniske oscillator. Slike teknikker resulterte bla.~i Nobel
prisen i fysikk i 1989 til Hans Dehmelt, Wolfgang Paul og Norman Ramsey,
se f.eks.~tidskriftet Review of Modern Physics, bind {\bf 62}, (1990)
sidene 525-541\footnote{Artiklene kan lastes ned fra nettstedet til det Amerikanske Fysiske selskap, www.aps.org, se http://prola.aps.org/}.
Ionefeller sammen med Laserkj\o ling av ioner og atomer f\o rte ogs\aa\ til 
en eksperimentell realisering av Bose-Einstein kondensasjon (Nobel pris i 1995), en fase
hvor alle partikler i en gass av f.eks fortynnete Rubidium ioner befinner
seg i den laveste mulige energitilstand. Vi kommer tilbake til dette
i kapittel \ref{chap:lasere}.
Kort fortalt, kan en ionefelle best\aa\  av elektriske felt som oscillerer
med radiofrekvenser, siden et ion ikke kan fanges inn av et statisk elektrisk
felt. Et eksempel er en s\aa kalt Paul felle, etter Nobelpris vinneren
Wolfgang Paul, som best\aa r av 
hyperbolske elektroder som setter opp 
et elektrisk kvadropol felt 
\be
\Phi(x,y) = \frac{V_0cos(\omega t)}{x_0^2+y_0^2}(x^2-2y^2),
\ee
hvor   $x_0$ og $y_0$ er dimensjonene p\aa\ ionefellen og 
$V_0cos(\omega t)$ er det 
oscillerende elektriske feltet med radiofrekvens $\omega$. 
Vi kan idealisere systemet v\aa rt til en dimensjon. Feltet setter
da opp et oscillator potensial
\[
   V(x)=\frac{1}{2}kx^2,
\]
som en partikkel beveger seg i.
I mekanikk er $k$ den s\aa kalte fj\ae rkonstanten.  

Innsatt i Schr\"odingers likning gir 
potensialet analytiske l\o sninger for energiegenfunksjoner.
Sistnevnte er gitt ved en klasse funksjoner som kalles Hermite
polynomer.
Energien er kvantisert, p\aa\ lik linje med det vi fant for en
uendelig potensialbr\o nn. 
\begin{figure}
\begin{center}
{\centering
\mbox{\psfig{figure=ho.ps,height=6cm,width=8cm,angle=0}}
}
\caption{Skisse av et harmonisk oscillator potential med tilh\o rende egenverdier.}
\end{center}

\end{figure}

Den tidsuavhengige Schr\"odingers likning  blir i dette tilfellet
\be
        -\frac{\hbar^2}{2m}\frac{d^2 \psi (x)}{d x^2}+
        \frac{1}{2}kx^2\psi (x)=E\psi (x),
\ee
hvor $V(x)=1/2kx^2$. 
Den siste likningen 
kan omformes som
\be
  \frac{d^2 \psi (x)}{d x^2}=\left(\frac{mk}{\hbar^2}x^2-\frac{2m}{\hbar^2}E\right)\psi (x).
\label{eq:homod}
\ee
Introduserer vi variabelen $\eta$ 
\be
    \eta^2=\frac{\sqrt{mk}x^2}{\hbar},
\ee
har vi at
\be
   \frac{d^2}{d x^2}=\frac{\sqrt{mk}}{\hbar}\frac{d^2}{d \eta^2},
\ee
som innsatt i likning (\ref{eq:homod}) gir
\be
  \frac{d^2 \psi (\eta)}{d \eta^2}=\frac{\hbar}{\sqrt{mk}}\left(\frac{\sqrt{mk}}{\hbar}\eta^2-\frac{2m}{\hbar^2}E\right)\psi (\eta),
\label{eq:homod1}
\ee
eller
\be
  \frac{d^2 \psi (\eta)}{d \eta^2}=\left(\eta^2-\frac{2\sqrt{m}}{\sqrt{k}\hbar}E\right)\psi (\eta).
\ee
N\aa\ kan vi velge \aa\ definere en st\o rrelse $\lambda$ gitt ved
\be
   \lambda=E\frac{2\sqrt{m}}{\sqrt{k}\hbar},
\ee
og omskrive Schr\"odingers likning  for den harmoniske oscillator som
\be
   \frac{d^2 \psi (\eta)}{d \eta^2}=\left(\eta^2-\lambda\right)\psi(\eta).
\ee


Her skal vi pr\o ve \aa\ anskueliggj\o re selve l\o sningen ved \aa\
se f\o rst p\aa\ grensen $\eta \rightarrow \infty$. Antas
\be
    \eta >> \lambda,
\ee
kan vi skrive Schr\"odingers likning  som
\be
   \frac{d^2 \psi (\eta)}{d \eta^2}=\eta^2\psi(\eta).
\ee
Dette er en differensiallikning hvis l\o sning er  p\aa\ formen
\be
   \psi(\eta)\approx e^{-\eta^2/2},
\ee 
som kan sees ved innsetting da
\be
   \frac{d\psi (\eta)}{d \eta}=-\eta e^{-\eta^2/2},
\ee
og 
\be
   \frac{d^2 \psi (\eta)}{d \eta^2}= -e^{-\eta^2/2}+\eta^2 e^{-\eta^2/2}\approx \eta^2 e^{-\eta^2/2}=\eta^2\psi(\eta),
\ee
n\aa r $\eta \rightarrow \infty$. Dvs.~vi har
\be
   \psi(\eta)_{\eta\rightarrow\infty}\rightarrow \eta^2 e^{-\eta^2/2}.
\ee


Denne framgangsm\aa ten tjener den hensikt at den viser oss at
b\o lgefunksjonen ihvertfall b\o r ha en oppf\o rsel slik som vist
i siste i likning. mer generelt kan vi da anta at den fullstendige l\o sningen
kan skrives som
\be
    \psi(\eta)=H(\eta)e^{-\eta^2/2}.
\ee
Vi kan finne l\o sningene for $H(\eta)$ p\aa\ f\o lgende vis
\be
      \frac{d\psi (\eta)}{d \eta}=\frac{dH(\eta)}{d \eta}e^{-\eta^2/2}
       -H(\eta)\eta e^{-\eta^2/2},
\ee
og
\be
      \frac{d^2 \psi (\eta)}{d \eta^2}=\frac{d^2H(\eta)}{d \eta^2}e^{-\eta^2/2}
       -H(\eta)\eta e^{-\eta^2/2}-2\eta\frac{dH(\eta)}{d \eta}e^{-\eta^2/2}+
       H(\eta)\eta^2 e^{-\eta^2/2}.
\ee
Setter vi det siste leddet inn i den modifiserte Schr\"odingers likning  gitt i likning (\ref{eq:homod1}), finner vi
\be
   \frac{d^2H(\eta)}{d \eta^2}-2\eta\frac{dH(\eta)}{d \eta}+
       (\lambda-1)H(\eta)=0.
   \label{eq:hermite}
\ee
Denne differensiallikningen har som l\o sning de s\aa kalte Hermite polynomene,
hvis formelle l\o sning st\aa r p\aa\ sidene 246-249 i tekstboka(kun orienteringsstoff). 
Vi ser straks at en l\o sning er gitt dersom $H(\eta)$ er uavhengig av 
$\eta$, da blir de to leddene med de deriverte lik null, og vi sitter
igjen med 
\be
    \lambda=1.
\ee
B\o lgefunksjonen for dette tilfellet er 
\be
    \psi(\eta)=H(\eta)e^{-\eta^2/2}=Ce^{-\eta^2/2},
\ee
hvor $C$ er en konstant vi kan bestemme fra normeringsbetingelsen
\be
   \int_{-\infty}^{+\infty}(\psi(x))^*(\psi(x))dx=1. 
\ee
I v\aa rt tilfelle trenger vi alts\aa\ \aa\ rekne ut
\be
    \int_{-\infty}^{+\infty}C^2e^{-\eta^2}dx\int_{-\infty}^{+\infty}C^2e^{-\frac{\sqrt{mk}x^2}{\hbar}}dx=1,
\ee
som gir 
\be
    C^2=\sqrt{\frac{\pi\hbar}{mk}},
\ee
slik at egenfunksjonen blir 
\be
   \psi(x)=\left(\sqrt{\frac{\pi\hbar}{mk}}\right)^{1/2}e^{-\frac{\sqrt{mk}x^2}{2\hbar}},
\ee
med en energi $E$ 
\be
   \lambda=1=E\sqrt{\frac{m}{k}}\frac{2}{\hbar},
\ee
dvs.
\be
   E=\frac{\hbar}{2}\sqrt{\frac{k}{m}}.
\ee
Dette skal svare til energien til grunntilstanden. Det er vanlig \aa\ definere
energien vha.~frekvensen $\omega$ gitt ved
\be
   \omega=\sqrt{\frac{k}{m}},
\ee
slik at energien blir
\be
   E=\frac{1}{2}\hbar\omega.
\ee
Likning (\ref{eq:hermite}) er differensiallikningen for de s\aa kalte
Hermite polynomene, som er polynom av like eller odde
potenser i $\eta$ og dermed $x$. Polynomene har et endelig antall ledd.
Et eksempel har vi allerede sett, nemlig $H_0(\eta)$. Denne er gitt ved
\be
   H_0(\eta)=1.
\ee
Flere eksempler er 
\be
    H_1(\eta)=2\eta,
\ee
\be
    H_2(\eta)=4\eta^2-2,
\ee
osv. Hvert polynom $H_n(\eta)$,
gir opphav til en egenverdi, gitt ved
\be
   E_n=\hbar\omega(n+\frac{1}{2}).
\ee
Den totale egenfunksjonen er gitt ved, for $n=0,1,2$, ved
\be
   \psi_0(\eta) = C_0e^{-\eta^2/2},
\ee
\be
    \psi_1(\eta)=C_12\eta e^{-\eta^2/2},
\ee
\be
    \psi_2(\eta)=C_2(4\eta^2-2)e^{-\eta^2/2},
\ee
hvor $C_{0,1,2}$ er normeringskonstanter. Vi legger merke til at 
for odde $n$ er egenfunksjonen antisymmetrisk om $\eta =0$, dvs.
\be 
   \psi_{n=\mathrm{oddetall}}(-\eta) = -\psi_{n=\mathrm{oddetall}}(\eta),
\ee
mens for like tall har vi en symmetrisk egenfunksjon og dermed
b\o lgefunksjon
\be 
   \psi_{n=\mathrm{liketall}}(-\eta) = \psi_{n=\mathrm{liketall}}(\eta).
\ee
For eksiterte tilstander ser vi ogs\aa\ at egenfunksjonen utviser et antall
noder som svarer til kvantetallet $n$. Dette sees fra  Figur 
\ref{fig:hoscplot}.
\begin{figure}
\begin{center}
% GNUPLOT: LaTeX picture with Postscript
\begingroup%
  \makeatletter%
  \newcommand{\GNUPLOTspecial}{%
    \@sanitize\catcode`\%=14\relax\special}%
  \setlength{\unitlength}{0.1bp}%
{\GNUPLOTspecial{!
%!PS-Adobe-2.0 EPSF-2.0
%%Title: hostates.tex
%%Creator: gnuplot 3.7 patchlevel 1
%%CreationDate: Fri Mar  1 16:06:12 2002
%%DocumentFonts: 
%%BoundingBox: 0 0 360 216
%%Orientation: Landscape
%%EndComments
/gnudict 256 dict def
gnudict begin
/Color false def
/Solid false def
/gnulinewidth 5.000 def
/userlinewidth gnulinewidth def
/vshift -33 def
/dl {10 mul} def
/hpt_ 31.5 def
/vpt_ 31.5 def
/hpt hpt_ def
/vpt vpt_ def
/M {moveto} bind def
/L {lineto} bind def
/R {rmoveto} bind def
/V {rlineto} bind def
/vpt2 vpt 2 mul def
/hpt2 hpt 2 mul def
/Lshow { currentpoint stroke M
  0 vshift R show } def
/Rshow { currentpoint stroke M
  dup stringwidth pop neg vshift R show } def
/Cshow { currentpoint stroke M
  dup stringwidth pop -2 div vshift R show } def
/UP { dup vpt_ mul /vpt exch def hpt_ mul /hpt exch def
  /hpt2 hpt 2 mul def /vpt2 vpt 2 mul def } def
/DL { Color {setrgbcolor Solid {pop []} if 0 setdash }
 {pop pop pop Solid {pop []} if 0 setdash} ifelse } def
/BL { stroke userlinewidth 2 mul setlinewidth } def
/AL { stroke userlinewidth 2 div setlinewidth } def
/UL { dup gnulinewidth mul /userlinewidth exch def
      10 mul /udl exch def } def
/PL { stroke userlinewidth setlinewidth } def
/LTb { BL [] 0 0 0 DL } def
/LTa { AL [1 udl mul 2 udl mul] 0 setdash 0 0 0 setrgbcolor } def
/LT0 { PL [] 1 0 0 DL } def
/LT1 { PL [4 dl 2 dl] 0 1 0 DL } def
/LT2 { PL [2 dl 3 dl] 0 0 1 DL } def
/LT3 { PL [1 dl 1.5 dl] 1 0 1 DL } def
/LT4 { PL [5 dl 2 dl 1 dl 2 dl] 0 1 1 DL } def
/LT5 { PL [4 dl 3 dl 1 dl 3 dl] 1 1 0 DL } def
/LT6 { PL [2 dl 2 dl 2 dl 4 dl] 0 0 0 DL } def
/LT7 { PL [2 dl 2 dl 2 dl 2 dl 2 dl 4 dl] 1 0.3 0 DL } def
/LT8 { PL [2 dl 2 dl 2 dl 2 dl 2 dl 2 dl 2 dl 4 dl] 0.5 0.5 0.5 DL } def
/Pnt { stroke [] 0 setdash
   gsave 1 setlinecap M 0 0 V stroke grestore } def
/Dia { stroke [] 0 setdash 2 copy vpt add M
  hpt neg vpt neg V hpt vpt neg V
  hpt vpt V hpt neg vpt V closepath stroke
  Pnt } def
/Pls { stroke [] 0 setdash vpt sub M 0 vpt2 V
  currentpoint stroke M
  hpt neg vpt neg R hpt2 0 V stroke
  } def
/Box { stroke [] 0 setdash 2 copy exch hpt sub exch vpt add M
  0 vpt2 neg V hpt2 0 V 0 vpt2 V
  hpt2 neg 0 V closepath stroke
  Pnt } def
/Crs { stroke [] 0 setdash exch hpt sub exch vpt add M
  hpt2 vpt2 neg V currentpoint stroke M
  hpt2 neg 0 R hpt2 vpt2 V stroke } def
/TriU { stroke [] 0 setdash 2 copy vpt 1.12 mul add M
  hpt neg vpt -1.62 mul V
  hpt 2 mul 0 V
  hpt neg vpt 1.62 mul V closepath stroke
  Pnt  } def
/Star { 2 copy Pls Crs } def
/BoxF { stroke [] 0 setdash exch hpt sub exch vpt add M
  0 vpt2 neg V  hpt2 0 V  0 vpt2 V
  hpt2 neg 0 V  closepath fill } def
/TriUF { stroke [] 0 setdash vpt 1.12 mul add M
  hpt neg vpt -1.62 mul V
  hpt 2 mul 0 V
  hpt neg vpt 1.62 mul V closepath fill } def
/TriD { stroke [] 0 setdash 2 copy vpt 1.12 mul sub M
  hpt neg vpt 1.62 mul V
  hpt 2 mul 0 V
  hpt neg vpt -1.62 mul V closepath stroke
  Pnt  } def
/TriDF { stroke [] 0 setdash vpt 1.12 mul sub M
  hpt neg vpt 1.62 mul V
  hpt 2 mul 0 V
  hpt neg vpt -1.62 mul V closepath fill} def
/DiaF { stroke [] 0 setdash vpt add M
  hpt neg vpt neg V hpt vpt neg V
  hpt vpt V hpt neg vpt V closepath fill } def
/Pent { stroke [] 0 setdash 2 copy gsave
  translate 0 hpt M 4 {72 rotate 0 hpt L} repeat
  closepath stroke grestore Pnt } def
/PentF { stroke [] 0 setdash gsave
  translate 0 hpt M 4 {72 rotate 0 hpt L} repeat
  closepath fill grestore } def
/Circle { stroke [] 0 setdash 2 copy
  hpt 0 360 arc stroke Pnt } def
/CircleF { stroke [] 0 setdash hpt 0 360 arc fill } def
/C0 { BL [] 0 setdash 2 copy moveto vpt 90 450  arc } bind def
/C1 { BL [] 0 setdash 2 copy        moveto
       2 copy  vpt 0 90 arc closepath fill
               vpt 0 360 arc closepath } bind def
/C2 { BL [] 0 setdash 2 copy moveto
       2 copy  vpt 90 180 arc closepath fill
               vpt 0 360 arc closepath } bind def
/C3 { BL [] 0 setdash 2 copy moveto
       2 copy  vpt 0 180 arc closepath fill
               vpt 0 360 arc closepath } bind def
/C4 { BL [] 0 setdash 2 copy moveto
       2 copy  vpt 180 270 arc closepath fill
               vpt 0 360 arc closepath } bind def
/C5 { BL [] 0 setdash 2 copy moveto
       2 copy  vpt 0 90 arc
       2 copy moveto
       2 copy  vpt 180 270 arc closepath fill
               vpt 0 360 arc } bind def
/C6 { BL [] 0 setdash 2 copy moveto
      2 copy  vpt 90 270 arc closepath fill
              vpt 0 360 arc closepath } bind def
/C7 { BL [] 0 setdash 2 copy moveto
      2 copy  vpt 0 270 arc closepath fill
              vpt 0 360 arc closepath } bind def
/C8 { BL [] 0 setdash 2 copy moveto
      2 copy vpt 270 360 arc closepath fill
              vpt 0 360 arc closepath } bind def
/C9 { BL [] 0 setdash 2 copy moveto
      2 copy  vpt 270 450 arc closepath fill
              vpt 0 360 arc closepath } bind def
/C10 { BL [] 0 setdash 2 copy 2 copy moveto vpt 270 360 arc closepath fill
       2 copy moveto
       2 copy vpt 90 180 arc closepath fill
               vpt 0 360 arc closepath } bind def
/C11 { BL [] 0 setdash 2 copy moveto
       2 copy  vpt 0 180 arc closepath fill
       2 copy moveto
       2 copy  vpt 270 360 arc closepath fill
               vpt 0 360 arc closepath } bind def
/C12 { BL [] 0 setdash 2 copy moveto
       2 copy  vpt 180 360 arc closepath fill
               vpt 0 360 arc closepath } bind def
/C13 { BL [] 0 setdash  2 copy moveto
       2 copy  vpt 0 90 arc closepath fill
       2 copy moveto
       2 copy  vpt 180 360 arc closepath fill
               vpt 0 360 arc closepath } bind def
/C14 { BL [] 0 setdash 2 copy moveto
       2 copy  vpt 90 360 arc closepath fill
               vpt 0 360 arc } bind def
/C15 { BL [] 0 setdash 2 copy vpt 0 360 arc closepath fill
               vpt 0 360 arc closepath } bind def
/Rec   { newpath 4 2 roll moveto 1 index 0 rlineto 0 exch rlineto
       neg 0 rlineto closepath } bind def
/Square { dup Rec } bind def
/Bsquare { vpt sub exch vpt sub exch vpt2 Square } bind def
/S0 { BL [] 0 setdash 2 copy moveto 0 vpt rlineto BL Bsquare } bind def
/S1 { BL [] 0 setdash 2 copy vpt Square fill Bsquare } bind def
/S2 { BL [] 0 setdash 2 copy exch vpt sub exch vpt Square fill Bsquare } bind def
/S3 { BL [] 0 setdash 2 copy exch vpt sub exch vpt2 vpt Rec fill Bsquare } bind def
/S4 { BL [] 0 setdash 2 copy exch vpt sub exch vpt sub vpt Square fill Bsquare } bind def
/S5 { BL [] 0 setdash 2 copy 2 copy vpt Square fill
       exch vpt sub exch vpt sub vpt Square fill Bsquare } bind def
/S6 { BL [] 0 setdash 2 copy exch vpt sub exch vpt sub vpt vpt2 Rec fill Bsquare } bind def
/S7 { BL [] 0 setdash 2 copy exch vpt sub exch vpt sub vpt vpt2 Rec fill
       2 copy vpt Square fill
       Bsquare } bind def
/S8 { BL [] 0 setdash 2 copy vpt sub vpt Square fill Bsquare } bind def
/S9 { BL [] 0 setdash 2 copy vpt sub vpt vpt2 Rec fill Bsquare } bind def
/S10 { BL [] 0 setdash 2 copy vpt sub vpt Square fill 2 copy exch vpt sub exch vpt Square fill
       Bsquare } bind def
/S11 { BL [] 0 setdash 2 copy vpt sub vpt Square fill 2 copy exch vpt sub exch vpt2 vpt Rec fill
       Bsquare } bind def
/S12 { BL [] 0 setdash 2 copy exch vpt sub exch vpt sub vpt2 vpt Rec fill Bsquare } bind def
/S13 { BL [] 0 setdash 2 copy exch vpt sub exch vpt sub vpt2 vpt Rec fill
       2 copy vpt Square fill Bsquare } bind def
/S14 { BL [] 0 setdash 2 copy exch vpt sub exch vpt sub vpt2 vpt Rec fill
       2 copy exch vpt sub exch vpt Square fill Bsquare } bind def
/S15 { BL [] 0 setdash 2 copy Bsquare fill Bsquare } bind def
/D0 { gsave translate 45 rotate 0 0 S0 stroke grestore } bind def
/D1 { gsave translate 45 rotate 0 0 S1 stroke grestore } bind def
/D2 { gsave translate 45 rotate 0 0 S2 stroke grestore } bind def
/D3 { gsave translate 45 rotate 0 0 S3 stroke grestore } bind def
/D4 { gsave translate 45 rotate 0 0 S4 stroke grestore } bind def
/D5 { gsave translate 45 rotate 0 0 S5 stroke grestore } bind def
/D6 { gsave translate 45 rotate 0 0 S6 stroke grestore } bind def
/D7 { gsave translate 45 rotate 0 0 S7 stroke grestore } bind def
/D8 { gsave translate 45 rotate 0 0 S8 stroke grestore } bind def
/D9 { gsave translate 45 rotate 0 0 S9 stroke grestore } bind def
/D10 { gsave translate 45 rotate 0 0 S10 stroke grestore } bind def
/D11 { gsave translate 45 rotate 0 0 S11 stroke grestore } bind def
/D12 { gsave translate 45 rotate 0 0 S12 stroke grestore } bind def
/D13 { gsave translate 45 rotate 0 0 S13 stroke grestore } bind def
/D14 { gsave translate 45 rotate 0 0 S14 stroke grestore } bind def
/D15 { gsave translate 45 rotate 0 0 S15 stroke grestore } bind def
/DiaE { stroke [] 0 setdash vpt add M
  hpt neg vpt neg V hpt vpt neg V
  hpt vpt V hpt neg vpt V closepath stroke } def
/BoxE { stroke [] 0 setdash exch hpt sub exch vpt add M
  0 vpt2 neg V hpt2 0 V 0 vpt2 V
  hpt2 neg 0 V closepath stroke } def
/TriUE { stroke [] 0 setdash vpt 1.12 mul add M
  hpt neg vpt -1.62 mul V
  hpt 2 mul 0 V
  hpt neg vpt 1.62 mul V closepath stroke } def
/TriDE { stroke [] 0 setdash vpt 1.12 mul sub M
  hpt neg vpt 1.62 mul V
  hpt 2 mul 0 V
  hpt neg vpt -1.62 mul V closepath stroke } def
/PentE { stroke [] 0 setdash gsave
  translate 0 hpt M 4 {72 rotate 0 hpt L} repeat
  closepath stroke grestore } def
/CircE { stroke [] 0 setdash 
  hpt 0 360 arc stroke } def
/Opaque { gsave closepath 1 setgray fill grestore 0 setgray closepath } def
/DiaW { stroke [] 0 setdash vpt add M
  hpt neg vpt neg V hpt vpt neg V
  hpt vpt V hpt neg vpt V Opaque stroke } def
/BoxW { stroke [] 0 setdash exch hpt sub exch vpt add M
  0 vpt2 neg V hpt2 0 V 0 vpt2 V
  hpt2 neg 0 V Opaque stroke } def
/TriUW { stroke [] 0 setdash vpt 1.12 mul add M
  hpt neg vpt -1.62 mul V
  hpt 2 mul 0 V
  hpt neg vpt 1.62 mul V Opaque stroke } def
/TriDW { stroke [] 0 setdash vpt 1.12 mul sub M
  hpt neg vpt 1.62 mul V
  hpt 2 mul 0 V
  hpt neg vpt -1.62 mul V Opaque stroke } def
/PentW { stroke [] 0 setdash gsave
  translate 0 hpt M 4 {72 rotate 0 hpt L} repeat
  Opaque stroke grestore } def
/CircW { stroke [] 0 setdash 
  hpt 0 360 arc Opaque stroke } def
/BoxFill { gsave Rec 1 setgray fill grestore } def
end
%%EndProlog
}}%
\begin{picture}(3600,2160)(0,0)%
{\GNUPLOTspecial{"
gnudict begin
gsave
0 0 translate
0.100 0.100 scale
0 setgray
newpath
1.000 UL
LTb
450 300 M
63 0 V
2937 0 R
-63 0 V
450 496 M
63 0 V
2937 0 R
-63 0 V
450 691 M
63 0 V
2937 0 R
-63 0 V
450 887 M
63 0 V
2937 0 R
-63 0 V
450 1082 M
63 0 V
2937 0 R
-63 0 V
450 1278 M
63 0 V
2937 0 R
-63 0 V
450 1473 M
63 0 V
2937 0 R
-63 0 V
450 1669 M
63 0 V
2937 0 R
-63 0 V
450 1864 M
63 0 V
2937 0 R
-63 0 V
450 2060 M
63 0 V
2937 0 R
-63 0 V
750 300 M
0 63 V
0 1697 R
0 -63 V
1350 300 M
0 63 V
0 1697 R
0 -63 V
1950 300 M
0 63 V
0 1697 R
0 -63 V
2550 300 M
0 63 V
0 1697 R
0 -63 V
3150 300 M
0 63 V
0 1697 R
0 -63 V
1.000 UL
LTb
450 300 M
3000 0 V
0 1760 V
-3000 0 V
450 300 L
1.000 UL
LT0
3087 1947 M
263 0 V
450 1082 M
30 0 V
31 0 V
30 0 V
30 0 V
31 0 V
30 0 V
30 0 V
30 0 V
31 0 V
30 0 V
30 0 V
31 1 V
30 0 V
30 0 V
31 0 V
30 0 V
30 1 V
30 1 V
31 1 V
30 1 V
30 1 V
31 2 V
30 3 V
30 3 V
31 5 V
30 5 V
30 6 V
30 7 V
31 9 V
30 10 V
30 12 V
31 14 V
30 16 V
30 17 V
31 19 V
30 21 V
30 21 V
31 23 V
30 24 V
30 24 V
30 24 V
31 23 V
30 21 V
30 20 V
31 18 V
30 15 V
30 11 V
31 8 V
30 4 V
30 0 V
30 -4 V
31 -8 V
30 -11 V
30 -15 V
31 -18 V
30 -20 V
30 -21 V
31 -23 V
30 -24 V
30 -24 V
30 -24 V
31 -23 V
30 -21 V
30 -21 V
31 -19 V
30 -17 V
30 -16 V
31 -14 V
30 -12 V
30 -10 V
31 -9 V
30 -7 V
30 -6 V
30 -5 V
31 -5 V
30 -3 V
30 -3 V
31 -2 V
30 -1 V
30 -1 V
31 -1 V
30 -1 V
30 -1 V
30 0 V
31 0 V
30 0 V
30 0 V
31 -1 V
30 0 V
30 0 V
31 0 V
30 0 V
30 0 V
30 0 V
31 0 V
30 0 V
30 0 V
31 0 V
30 0 V
1.000 UL
LT1
3087 1847 M
263 0 V
450 1082 M
30 0 V
31 0 V
30 0 V
30 0 V
31 0 V
30 0 V
30 0 V
30 0 V
31 -1 V
30 0 V
30 0 V
31 -1 V
30 -1 V
30 -1 V
31 -2 V
30 -2 V
30 -4 V
30 -4 V
31 -5 V
30 -6 V
30 -9 V
31 -10 V
30 -12 V
30 -15 V
31 -17 V
30 -21 V
30 -23 V
30 -26 V
31 -30 V
30 -31 V
30 -34 V
31 -35 V
30 -35 V
30 -34 V
31 -33 V
30 -29 V
30 -24 V
31 -18 V
30 -10 V
30 0 V
30 9 V
31 19 V
30 31 V
30 42 V
31 52 V
30 60 V
30 69 V
31 74 V
30 78 V
30 79 V
30 77 V
31 75 V
30 68 V
30 61 V
31 52 V
30 41 V
30 31 V
31 20 V
30 9 V
30 -1 V
30 -10 V
31 -18 V
30 -24 V
30 -29 V
31 -32 V
30 -35 V
30 -35 V
31 -35 V
30 -33 V
30 -32 V
31 -29 V
30 -26 V
30 -24 V
30 -20 V
31 -18 V
30 -15 V
30 -12 V
31 -10 V
30 -8 V
30 -7 V
31 -5 V
30 -4 V
30 -3 V
30 -2 V
31 -2 V
30 -2 V
30 -1 V
31 0 V
30 -1 V
30 0 V
31 0 V
30 0 V
30 -1 V
30 0 V
31 0 V
30 0 V
30 0 V
31 0 V
30 0 V
1.000 UL
LT2
3087 1747 M
263 0 V
450 1082 M
30 0 V
31 1 V
30 0 V
30 0 V
31 0 V
30 1 V
30 1 V
30 1 V
31 2 V
30 3 V
30 3 V
31 5 V
30 6 V
30 8 V
31 11 V
30 14 V
30 18 V
30 22 V
31 26 V
30 33 V
30 38 V
31 46 V
30 51 V
30 58 V
31 64 V
30 68 V
30 72 V
30 72 V
31 71 V
30 65 V
30 57 V
31 44 V
30 28 V
30 7 V
31 -15 V
30 -41 V
30 -68 V
31 -95 V
30 -120 V
30 -141 V
30 -159 V
31 -170 V
30 -174 V
30 -169 V
31 -158 V
30 -137 V
30 -110 V
31 -77 V
30 -39 V
30 0 V
30 39 V
31 77 V
30 110 V
30 137 V
31 158 V
30 169 V
30 174 V
31 170 V
30 159 V
30 141 V
30 120 V
31 95 V
30 68 V
30 41 V
31 15 V
30 -7 V
30 -28 V
31 -44 V
30 -57 V
30 -65 V
31 -71 V
30 -72 V
30 -72 V
30 -68 V
31 -64 V
30 -58 V
30 -51 V
31 -46 V
30 -38 V
30 -33 V
31 -26 V
30 -22 V
30 -18 V
30 -14 V
31 -11 V
30 -8 V
30 -6 V
31 -5 V
30 -3 V
30 -3 V
31 -2 V
30 -1 V
30 -1 V
30 -1 V
31 0 V
30 0 V
30 0 V
31 -1 V
30 0 V
stroke
grestore
end
showpage
}}%
\put(3037,1747){\makebox(0,0)[r]{$\psi_3(\eta)$}}%
\put(3037,1847){\makebox(0,0)[r]{$\psi_2(\eta)$}}%
\put(3037,1947){\makebox(0,0)[r]{$\psi_1(\eta)$}}%
\put(1950,50){\makebox(0,0){$\eta$ }}%
\put(100,1180){%
\special{ps: gsave currentpoint currentpoint translate
270 rotate neg exch neg exch translate}%
\makebox(0,0)[b]{\shortstack{$\psi(\eta)$}}%
\special{ps: currentpoint grestore moveto}%
}%
\put(3150,200){\makebox(0,0){4}}%
\put(2550,200){\makebox(0,0){2}}%
\put(1950,200){\makebox(0,0){0}}%
\put(1350,200){\makebox(0,0){-2}}%
\put(750,200){\makebox(0,0){-4}}%
\put(400,2060){\makebox(0,0)[r]{2.5}}%
\put(400,1864){\makebox(0,0)[r]{2}}%
\put(400,1669){\makebox(0,0)[r]{1.5}}%
\put(400,1473){\makebox(0,0)[r]{1}}%
\put(400,1278){\makebox(0,0)[r]{0.5}}%
\put(400,1082){\makebox(0,0)[r]{0}}%
\put(400,887){\makebox(0,0)[r]{-0.5}}%
\put(400,691){\makebox(0,0)[r]{-1}}%
\put(400,496){\makebox(0,0)[r]{-1.5}}%
\put(400,300){\makebox(0,0)[r]{-2}}%
\end{picture}%
\endgroup
\endinput

\end{center}
\caption{Plott av egenfunksjonene (unormerte) 
for en harmonisk oscillator som funksjon av 
$\eta$. \label{fig:hoscplot}}
\end{figure}

Bruker vi den eksakte energien, kan vi dermed sette opp den eksakte
b\o lgefunksjonen. For de samme tilstandene har vi 
\be
   \Psi_0(\eta,t) = C_0e^{-\eta^2/2}e^{-it\omega\frac{1}{2}},
\ee
\be
    \Psi_1(\eta,t)=C_12\eta e^{-\eta^2/2}e^{-it\omega\frac{3}{2}},
\ee
og
\be
    \Psi_2(\eta,t)=C_2(4\eta^2-2)e^{-\eta^2/2}e^{-it\omega\frac{5}{2}},
\ee
hvor $C_{0,1,2}$ igjen er normeringskonstanter. 


\subsection{Eksempel p\aa\ l\o sning for harmonisk oscillator vha.~Maple}

Her skal vi ta for oss hvordan vi kan finne b\o lgefunksjonen 
for et harmonisk oscillator potensial med gitt energi. 
F\o rst skal vi forenkle Schr\"odingers likning ved \aa\ sette
$\hbar=1$ (energien blir da uttrykt i enheter av $\hbar$)  og velge
$m=k=1$ slik at likningen tar forma
\be
  \frac{d^2 \psi_n (x)}{d x^2}-x^2\psi_n (x)+E_n\psi_n (x),
\ee
hvor energien n\aa\ er gitt ved
\[
   E_n=2(n+\frac{1}{2}),
\]
istedet for $E_n=\hbar\omega(n+\frac{1}{2})$.

Potensialet v\aa rt er symmetrisk, dvs.~$V(x)=V(-x)$ men b\o lgefunksjonene
kan v\ae re enten symmetriske eller antisymmetriske. Ved origo utviser
disse b\o lgefunksjonene ulike egenskaper. For en symmetrisk b\o lgefunksjon
har vi
\[
   \psi(0)=konstant\ne 0, \hspace{1cm} \frac{d\psi}{dx}|_{x=0}=0,
\]
mens en antisymmetrisk b\o lgefunksjon er gitt ved 
\[
   \psi(0)=0, \hspace{1cm} \frac{d\psi}{dx}|_{x=0}=konstant\ne 0.
\]

Vi kan velge konstanten lik $1$ og deretter normalisere b\o lgefunksjonen
v\aa r. 

Her skal vi vise hvordan vi kan finne $\psi$ med en gitt egenverdi $E_n$ 
vha.~Maple. 

Som eksempel velger vi her grunntilstanden som i dette tilfellet har energien
$E_0=1$. Vi kan da skrive  Schr\"odingers likning p\aa\ f\o lgende vis
\begin{maplegroup}
\begin{mapleinput}
\mapleinline{active}{1d}{SE:=diff(psi(x),x$2)-x^2*psi(x)+psi(x)=0;}{%
}
\end{mapleinput}
der vi har brukt kommandoen {\bf diff} for \aa\ definere den deriverte. 
\mapleresult
\begin{maplelatex}
\[
\mathit{SE} := ({\frac {\partial ^{2}}{\partial x^{2}}}\,\psi (x)
) - x^{2}\,\psi (x) + \psi (x)=0
\]
\end{maplelatex}

\end{maplegroup}
Siden grunntilstanden er symmetrisk om origo, definerer vi
\begin{maplegroup}
\begin{mapleinput}
\mapleinline{active}{1d}{icsym:=psi(0)=1,D(psi)(0)=0;}{%
}
\end{mapleinput}

\mapleresult
\begin{maplelatex}
\[
\mathit{icsym} := \psi (0)=1, \,\mathrm{D}(\psi )(0)=0
\]
\end{maplelatex}

\end{maplegroup}
N\aa r vi deretter skal l\o se difflikningen numerisk, bruker
vi Maple kommandoen {\bf dsolve}, med {\bf numeric} som opsjon,
samt at vi legger til f\o ringene p\aa\ egenfunksjonen ved $x=0$. 
\begin{maplegroup}
\begin{mapleinput}
\mapleinline{active}{1d}{solveSe:=dsolve(\{subs(SE),icsym
\},psi(x),numeric);}{%
}
\end{mapleinput}
Maple bruker Runge-Kutta metoden til \aa\ l\o se denne difflikningen,
som en kan se fra den neste outputlinje  fra Maple.
\mapleresult
\begin{maplelatex}
\[
\mathit{solveSe} := \textbf{proc} (\mathit{rkf45\_x})\,\ldots \,
\textbf{end} 
\]
\end{maplelatex}

\end{maplegroup}
Vi \o nsker deretter \aa\ plotte egenfunksjonen v\aa r. I dette tilfellet
velger vi \aa\ plotte sannsynligheten $P(x)$ uttrykt vha.~$\psi(x)^*\psi(x)$.
Koplingen mellom en numerisk l\o sning og figurer i Maple kan vi lett
f\aa\ til vha.~kommandoen {\bf odeplot}. Her spesifiserer vi hva som skal
plottes samt grenseverdiene for b\aa de $x$ og $y$ aksene. I eksemplet
nedenfor har vi begrenset oss til $x$-aksen.
\begin{maplegroup}
\begin{mapleinput}
\mapleinline{active}{1d}{with(plots):}{%
}
\end{mapleinput}
\end{maplegroup}
\begin{maplegroup}
\begin{mapleinput}
\mapleinline{active}{1d}{odeplot(solveSe,[x,psi(x)^2],-3..3);}{%
}
\end{mapleinput}
Den f\o rste figuren viser b\o lgefunksjonen kvadrert for grunntilstanden.
\mapleresult
\begin{figure}
\begin{center}
\mapleplot{ho01.eps}
\end{center}
\caption{Sannsynlighetsfordeling for grunntilstanden til en harmonisk oscillator.}
\end{figure}
\end{maplegroup}
\begin{maplegroup}
\begin{mapleinput}
\mapleinline{active}{1d}{p1:=odeplot(solveSe,[x,psi(x)^2],-3..3);}{%
}
\end{mapleinput}
\end{maplegroup}
I neste figur \o nsker vi ogs\aa\ \aa\ plotte potensialet sammen
med sannsynligheten for \aa\ finne partikkelen i et bestemt omr\aa de.
Det kan vi f\aa\ til vha.~{\bf display} og {\bf plot} kommandoene i Maple. 
\begin{maplegroup}
\begin{mapleinput}
\mapleinline{active}{1d}{p2:=plot(x^2,x=-3..3,y=0..3);}{%
}
\end{mapleinput}
\end{maplegroup}
\begin{maplegroup}
\begin{mapleinput}
\mapleinline{active}{1d}{display(p1,p2);}{%
}
\end{mapleinput}
\mapleresult
\begin{figure}
\begin{center}
\mapleplot{ho02.eps}
\end{center}
\caption{Plott av harmonisk oscillator potensial og sannsynlighetsfordeling for grunntilstanden. Legg merke til at det finnes en viss sannsynlighet for at partikkelen kan befinne seg utenfor potensialet.}
\end{figure}

\end{maplegroup}
\begin{maplegroup}
\begin{mapleinput}
\end{mapleinput}
\end{maplegroup}


Vi legger merke til at det er en viss sannsynlighet for \aa\ finne partikkelen
utenfor potensialet! Dette er forskjellig fra det vi har sett tidligere for
en klassisk harmonisk oscillator. Klassisk er det ikke mulig at 
partikkelen kan trenge gjennom potensialet.

\subsection{Eksempel p\aa\ l\o sning for harmonisk oscillator vha.~Matlab}
I Matlab finnes det flere funksjoner for \aa\ l\o ordin\ae re differensial likninger.
Men i motsetning til Maple m\aa\ vi f\o rst skrive om likningen v\aa r som et sett med
kopla f\o rste ordens differensiallikninger.
Det betyr at vi definerer
\be 
    y_1(x) = \psi_n(x),
\ee 
og 
\be
   \frac{ dy_1}{dx} = y_2
\ee
og vi kan dermed omskrive den andre ordens differensiallikningen
\[
  \frac{d^2 \psi_n (x)}{d x^2}-x^2\psi_n (x)+E_n\psi_n (x)=0,
\]
som 
\be 
    \frac{d y_2(x)}{d x}-x^2y_1(x)+\lambda y_1(x)=0,
\ee 
og 
\be
      \frac{ dy_1}{dx} = y_2,
\ee
hvor $\lambda$ er den ukjente energien. 
Men Schr\"odingers likning m\aa\ l\o ses med tanke p\aa\ randbetingelsene. I v\aa rt tilfelle
har vi at  for en symmetrisk l\o sning om $x=0$ at 
\[
   \psi(0)=konstant\ne 0, \hspace{1cm} \frac{d\psi}{dx}|_{x=0}=0,
\]
mens en antisymmetrisk b\o lgefunksjon er gitt ved 
\[
   \psi(0)=0, \hspace{1cm} \frac{d\psi}{dx}|_{x=0}=konstant\ne 0.
\]

Vi kan velge konstanten lik $1$ og deretter normalisere b\o lgefunksjonen
v\aa r. I tillegg m\aa\ b\o lgefunksjonen g\aa\ mot null i grensa $x\rightarrow \pm \infty$.

Matlab tillatter oss \aa\ l\o se denne type problem med sine innebygde funksjoner
av typen {\bf bvp4c} som l\o ser s\aa kalte topunkts randbetingelseproblemer
for ordin\ae re differensiallikninger. Denne funksjonen har to viktige hjelpefunksjoner
hvor blant annet {\bf bvpinit} er sv\ae rt viktig da den tillater oss \aa\ gi gjetninger
p\aa\ egenverdier og form p\aa\ l\o sning. Funksjonen {\bf deval} gir oss muligheten til
\aa\ evaluere den numeriske l\o sningen fra {\bf bvp4c}.
Eksemplet nedenfor viser hvordan vi kan l\o se likningsettet v\aa rt med Matlab.
\begin{verbatim}
function harmonic_oscillator

%   Vi velger et l�sningsintervall [0, 5] med randbetingelser y'(0) = 0, 
%   og y(5) = 0
%   Egenfunskjonen y(x) er bestemt ved en multiplikativ konstant n�r.
%   N�r vi setter y(0) = 1 er det for � velge ut den symmetriske l�sningen
%   om x = 0.
%   Her setter vi inn den eksakte egenverdien for grunntilstanden.

lambda = 1;
solinit = bvpinit(linspace(0,5,10),@mat4init,lambda)

% BVP4C returnerer strukturen 'sol'. Den beregnede egenverdien finnes 
% i feltet sol.parameters. 
sol = bvp4c(@mat4ode,@mat4bc,solinit);

fprintf('Egenverdien er %7.3f.\n',sol.parameters)
% Her plotter vi egenfunksjonen
xint = linspace(0,5);
Sxint = deval(sol,xint);
figure;
plot(xint,Sxint(1,:));
axis([0 5 -2 2.1]);
title('Egenfunksjon for Harmonisk oscillator'); 
xlabel('x');
ylabel('L�sning y');

%difflikningen 
function dydx = mat4ode(x,y,lambda)

dydx = [              y(2)
         (x*x-lambda)*y(1) ];

% randbetingelser, ya(2) betyr y_2(0) = 0, yb(1) betyr y_1(5)=0
% og ya(1)-1 betyr y_1(0)-1=0 
function res = mat4bc(ya,yb,lambda)
res = [  ya(2) 
         yb(1) 
         ya(1)-1];

% pr�veform for l�sningen og dens deriverte
function yinit = mat4init(x)
yinit = [   exp(-x*x)
    -2.*x*exp(-x*x)];
\end{verbatim}


\subsection{Viktig l\ae rdom}  
For en partikkel i en potensialbr\o nn med uendelig potensial, s\aa\ vi at
energien er kvantisert og at energien er bestemt av utstrekningen av 
potensialet samt partikkelens masse. 

Et harmonisk oscillator potensial, i tillegg til \aa\ v\ae re et meget 
popul\ae rt potensial for modellering av fysiske systemer, utviser ogs\aa\
kvantisering av energien, samt at energien er bestemt av massen til partikkelen
og potensialkonstanten $k$. 
Et harmonisk oscillator potensial har en uendelighet av diskrete tilstander,
p\aa\ lik linje med en uendelig potensialbr\o nn. 
Men, det nye her er at det finnes en viss
sannsynlighet for at partikkelen kan v\ae re utenfor selve potensialet,
dvs.~i et omr\aa de som er klassisk forbudt. 
Dette f\aa r store konsekvenser for kvantemekaniske systemer, som vi bla.~skal
studere i avsnittet om tunneling. 


\section{Tre-dimensjonalt bokspotensial}

Vi skal n\aa\ studere
et uendelig potensial i tre dimensjoner\footnote{Lesehenvisning er
Kap. 5-11, sidene 288-293.}, dvs.~at potensialet
tar forma
\be
 V(x,y,z)=\left\{\begin{array}{cc}0& -a< x < +a, -a< y < +a, -a< z < +a \\
           \infty&|x| \ge a, |y| \ge a, |z| \ge a \end{array} \right.
\ee
Dersom vi n\aa\ ser p\aa\ den tidsuavhengige Schr\"odingers likning , s\aa\ trenger vi \aa\
introdusere operatorer for bevegelsesmengden i $x$, $y$ og $z$ retning, dvs.
\be
   \widehat{p}_x = -i\hbar \frac{\partial}{\partial x},
\ee
\be
   \widehat{p}_y = -i\hbar \frac{\partial}{\partial y},
\ee
og
\be
   \widehat{p}_z = -i\hbar \frac{\partial}{\partial z}.
\ee
I tillegg har vi at
\be
    p^2=p_x^2+p_y^2+p_z^2, 
\ee
slik at vi kan skrive den tidsuavhengige Schr\"odingers likning  p\aa\ f\o lgende operatorform
\be
   -\frac{\hbar^2}{2m}\left(\frac{\partial^2}{\partial x^2}+\frac{\partial^2}{\partial y^2}+\frac{\partial^2}{\partial z^2}\right)+V(x,y,z)=i\hbar \frac{\partial}{\partial t},
\ee
og vi har antatt at potensialet er uavhengig av tiden $t$.
Setter vi inn b\o lgefunksjonen~har vi
\be
   -\frac{\hbar^2}{2m}\left(\frac{\partial^2}{\partial x^2}+\frac{\partial^2}{\partial y^2}+\frac{\partial^2}{\partial z^2}\right)\Psi(x,y,z,t)+V(x,y,z)\Psi(x,y,z,t)=i\hbar \frac{\partial\Psi(x,y,z,t)}{\partial t}.
\ee
Denne likningen kan vi omskrive som
\be
   -\frac{\hbar^2}{2m}\nabla^2\Psi(x,y,z,t)+V(x,y,z)\Psi(x,y,z,t)=i\hbar \frac{\partial\Psi(x,y,z,t)}{\partial t},
\ee
og siden potensialet er uavhengig av tiden $t$ 
kan vi anta at b\o lgefunksjonen~er separabel i tid
\be
   \Psi(x,y,z,t)=\psi(x,y,z)e^{-i Et/\hbar}.
\ee
Den romlige delen av Schr\"odingers likning  reduseres dermed til en egenverdilikning i tre dimensjoner
\be
   -\frac{\hbar^2}{2m}\nabla^2\psi(x,y,z)+V(x,y,z)\psi(x,y,z)=E\psi(x,y,z),
\ee
\begin{table}[tp]
\begin{center}
\begin{tabular}{llll}\hline\\
$n_x$ & $n_y$ & $n_z$ & Energi\\ \hline
1 & 1 & 1& $ E_0\times 3$ \\
2 & 1 & 1& $E_0 \times 6$ \\
1 & 2 & 1& $E_0 \times 6$ \\
1 & 1 & 2& $E_0 \times 6$ \\
2 & 2 & 1& $E_0 \times 9$ \\
1 & 2 & 2& $E_0 \times 9$ \\
2 & 1 & 2& $ E_0\times 9$ \\
3 & 1 & 1& $E_0\times 11$ \\
1 & 3 & 1& $E_0\times 11$ \\
1 & 1 & 3& $E_0 \times 11$ \\
$\dots$ & $\dots$ & $\dots$ & $\dots$  \\\hline
\end{tabular}
\end{center}
\caption{Mulige verdier for energien for den tre-dimensjonale potensialbr\o nnen som funksjon av $n_x$, $n_y$ og $n_z$. I tabellen har vi definert 
$E_0=\pi^2\hbar^2/8ma^2$.}
\end{table}
I tillegg har vi at potensialet er adskilt i tre deler som 
funksjoner av $x$, $y$ og $z$, som ikke avhenger av hverandre og
\be
V(x,y,z)=V(x)+V(y)+V(z)=0 \hspace{0.2cm}-a< x < +a, -a< y < +a, -a< z < +a.
\ee
Vi kan da pr\o ve som l\o sningansats
\be
    \psi(x,y,z)=\psi(x)\psi(y)\psi(z).
\ee 
Det gir oss tre separate likninger for egenverdilikningen
\be
   -\frac{\hbar^2}{2m}\frac{\partial^2}{\partial x^2}\psi(x)=E_x\psi(x),
\ee
\be
   -\frac{\hbar^2}{2m}\frac{\partial^2}{\partial y^2}\psi(y)=E_y\psi(y),
\ee
og
\be
   -\frac{\hbar^2}{2m}\frac{\partial^2}{\partial z^2}\psi(z)=E_z\psi(z).
\ee
I avsnitt \ref{sec:uendeligbronn} 
har vi allerede l\o st likningen mhp.~$x$, med l\o sningen 
\[
    \psi_{n_x}(x)=\sqrt{\frac{1}{a}}cos (x\frac{n_x\pi}{2a})\hspace{0.2cm} n=\pm 1, \pm 3,\dots.
\]
og 
\[
    \psi_{n_x}(x)=\sqrt{\frac{1}{a}}sin (x\frac{n_x\pi}{2a})\hspace{0.2cm} n=\pm 2, \pm 4,\dots.
\]
Energien er gitt ved $E_{n_x}=\pi^2\hbar^2n_x^2/8ma^2$.
Tilsvarende l\o sning har vi for $\psi(y)$ og $\psi(z)$, slik at 
den totale egenfunksjonen blir
\be
   \psi(x,y,z)=\sqrt{\frac{1}{a}}\left\{\begin{array}{c}cos\\sin\end{array}\right\}(x\frac{n_x\pi}{2a})\sqrt{\frac{1}{a}}\left\{\begin{array}{c}cos\\sin\end{array}\right\}(y\frac{n_y\pi}{2a})\sqrt{\frac{1}{a}}\left\{\begin{array}{c}cos\\sin\end{array}\right\}(z\frac{n_z\pi}{2a}),
\ee
med egenverdi
\be
    E_{n_xn_yn_z}=\frac{\pi^2\hbar^2}{8ma^2}\left(n_x^2+n_y^2+n_z^2\right),
\ee
hvor $n_x$, $n_y$ og $n_z$ er heltall.
Vi kan dermed lage oss en tabell over energiegenverdiene for ulike
kombinasjoner av $n_x$, $n_y$ og $n_z$.


Vi legger merke til at det finnes flere egenverdier som er like, men at
egenfunksjonene er forskjellige, dvs.~kombinasjonene av
$n_x$, $n_y$ og $n_z$ har ulike egenfunksjoner, men resulterer i lik
energi. Det kalles for degenerasjon, og gjenspeiler en underliggende 
symmetri til systemet. Vi skal studere dette n\o yere n\aa r vi l\o ser
Schr\"odingers likning  for et sentralsymmetrisk potensial slik som Coulomb potensialet.
 
\section{Oppgaver}
\subsection{Analytiske oppgaver}
\subsubsection*{Oppgave 4.1}
En partikkel med masse m og energi $E < 0$ beveger seg
langs x-aksen i potensialet
%
\[
V(x) = \left \{
\begin{array}{c}
     0  \\
  - V_{0}\\
  + \infty
\end{array}
%
\;\;\; \mbox{for} \;\;\;
%
\begin{array}{r}
 x < -a \\
 -a \leq x < 0 \\
 x \geq  0
\end{array}
%
\right .
%
\]
%
\begin{itemize}
%
\item[a)] L\o s Schr\"{o}dingerligningen for systemet og vis at de tilsvarende
energiniv\aa ene er implisitt gitt ved ligningen $k \cot (ka) = -\beta $ hvor
$\hbar \beta = \sqrt{2m|E|}$ og $\hbar k = \sqrt{2m (V_{0} - |E|)}$.

\item[b)] Finn r\o ttene i ligningen grafisk.
(L{\o}sningstips: Kvadr\'{e}r ligningen overfor og finn et uttrykk
for $\sin (ka)$. L{\o}s denne ligningen grafisk).
Ansl\aa ~ut fra grafen hvor
sterkt potensialet $V_{0}$ m\aa ~v\ae re for \aa ~ha minst en bundet tilstand?

\item[c)] Hva blir energiniv\aa ene i grensen $V_{0} \rightarrow \infty $?
Hvorfor tilsvarer dette niv\aa ene til til\-stand\-ene
i uendelig dypt og symmetrisk kassepotensial?
%
\end{itemize}
%

%
\subsubsection*{Oppgave 4.2}
Vi studerer igjen en partikkel som beveger seg langs x-aksen i potensialet
%
\[
V(x) = \left \{
\begin{array}{c}
     0  \\
  - V_{0}\\
  + \infty
\end{array}
%
\;\;\; \mbox{for} \;\;\;
%
\begin{array}{r}
 x < -a \\
 -a \leq x < 0 \\
 x \geq  0
\end{array}
%
\right .
%
\]
%
men antar n\aa ~at den har positiv energi, $E > 0$.
%
\begin{itemize}
%
\item[a)] Hvordan ser l\o sningen av Schr\"{o}dingerligningen n\aa ~ut i de to
omr\aa dene $x < - a$ og\\
$-a \leq x < 0$?

\item[b)] Finn ved utregning at av en innkommende partikkelb\o lge fra venstre,
vil 100\% bli re\-flek\-tert. Hvorfor var dette \aa ~vente?

\item[c)] Vis at b\o lgefunksjonen for $x < -a$ kan skrives som $\psi (x) = C
sin(kx + \delta )$, og uttrykk faseforskjellen $\delta$ ved energien E og
spredningspotensialet $V_{0}$.

\item[d)] I motsetning til i oppgave 4.1 er ikke energien kvantisert i dette
tilfellet. Hvorfor?
%
\end{itemize}
%
\subsubsection*{Oppgave 4.3}
Et elektron er beskrevet ved b\o{}lgefunksjonen
%
\begin{eqnarray*}
%
\psi(x) =\left \{ \begin{array}{ll}
				 0 & for \;\; x \leq 0;\\
				 N\exp(-\alpha x) (1 - \exp(-\alpha x) ) & for \;\; x > 0,
				 \end{array}
				 \right .
\end{eqnarray*}
%
hvor koordinaten er $x$  og $N$ er en
normaliseringskonstant.
%
\begin{itemize}
%
\item[a)] Skiss\'{e}r b{\o}lgefunksjonen og forklar dens fysiske
innhold. Hvis den er en l{\o}sning av Schr\"{o}dingerligningen
for et elektron i et potensial $V(x)$, hva er formen p{\aa}
dette potensialet?
%
\item[b)] Finn normaliseringskonstanten $N$.
\item[c)] For hvilken verdi av $x$ er det
mest sannsynlig {\aa} finne elektronet?
\end{itemize}
\subsubsection*{L\o sning}
%
\begin{itemize}
% 
\item[a)] Egenskaper ved b�lgefunksjonen
% 
\begin{eqnarray*}
   \psi(x = 0) &=& 0, \quad  \psi(x) 
        \longrightarrow 0 \quad \mbox{for}
          \quad x \rightarrow +\infty\\
    \frac{d}{dx} \psi(x) &=& -\alpha \psi(x) + \alpha e^{-2\alpha x} = 0 \quad  
    \mbox{som gir} \quad x_{max} = \frac{\ln 2}{\alpha}
\end{eqnarray*}
%
Vi bestemmer elektronets potensielle energi ved � sette
b�lgefunksjonen $\psi(x)$ in i Schr\"{o}dinger ligningen
% 
\[
- \frac{\hbar^2}{2m} \frac{d^2}{dx^2} \psi(x) + V(x) \psi(x) = E \psi(x)
\]
% 
Beregning gir
%
\begin{eqnarray*}
\frac{d^2}{dx^2} \psi(x) &=& \frac{d}{dx} 
          \left ( -\alpha \psi(x) + \alpha e^{-2\alpha x} \right )\\
        &=& \alpha^2 \psi(x) - 3 \alpha^2 e^{-2\alpha x}
\end{eqnarray*}
%
som innsatt i Schr\"{o}dinger ligningen blir 
%
\begin{eqnarray*}
\frac{\hbar^2}{2m}\left ( \alpha^2 \psi(x) -3 \alpha^2 e^{-2\alpha x}
\right ) + V(x) \psi(x) &=& E \psi(x)\\
\left ( \frac{3 \hbar^2 \alpha^2}{2 m} \frac{e^{-2\alpha x}}{\psi(x)} +
V(x) \right ) &=& E + \frac{\hbar^2 \alpha^2}{2m}
\end{eqnarray*}
%
% 
H�yre siden i dette uttrykket er en konstant, uavhengig av $x$. Da m�
venstre siden ogs� v�re en konstant. Dette gir l�sningen
%
\begin{eqnarray*}
E &=& - \frac{\hbar^2\alpha^2}{2m} + E_0\\
V(x) &=& - \frac{3\hbar^2\alpha^2}{2m} \frac{e^{-\alpha x }}{1 -
e^{-\alpha x}} + E_0
\end{eqnarray*}
% 
og uten tap av generalitet kan vi velge $E_0 = 0$.
% 

\item[b)] Normalisering
% 
\begin{eqnarray*}
N^2 \int_0^{+\infty} e^{-2\alpha x} \left ( 1 - e^{-\alpha x} 
                    \right )^2 dx
	&=& N^2 \int_0^{+\infty} 
   	\left ( e^{-2\alpha x} - 2 e^{-3 \alpha x} + e^{-4 \alpha x} \right ) dx\\
	&=& N^2 \left ( \frac{1}{2\alpha}
	- \frac{2}{3\alpha} + \frac{1}{4\alpha} 
	\right ) = \frac{N^2}{12\alpha} = 1
\end{eqnarray*}
% 
som gir $ N = 2\sqrt{3 \alpha}$.
% 
\item[c)] Sannsynlighetstettheten er gitt ved 
$P(x) = \psi^{*}(x) \psi(x)$
Maksimumsverdien er beregnet i punkt a) til � v�re  
$x_{maks} = \frac{\ln 2}{\alpha}$.
%
\item[d)] 
\begin{eqnarray*}
\langle x \rangle &=& \int_0^{+\infty} \psi^*(x) x \psi(x) dx 
    = N^2  \int_0^{+\infty} x e^{-2\alpha x} \left ( 1 - e^{\alpha x}
    \right )^2  dx\\
&=& N^2 \int_0^{+\infty} \left ( x e^{-2\alpha x} - 2 x e^{-3\alpha x}
                 + x e^{-4\alpha x} \right ) dx
\end{eqnarray*}
% 
Vi l�ser integralene ved hjelp av  formelen $ u\cdot v' = (u \cdot v)'
- u'\cdot v$. Dette gir
%
\[
\int_0^{+\infty} x e^{\beta x} dx = + \frac{1}{\beta}\int_0^{+\infty} e^{-\beta x}
dx = \frac{1}{\beta^2}
\]
% 
og 
% 
\[ 
\langle x \rangle = N^2 \left ( \frac{1}{4 \alpha^2} 
           - \frac{2}{4 \alpha^2}+ \frac{1}{16 \alpha^2} \right )=
           \frac{13}{12 \alpha}
\]
\end{itemize}
%
Sammenlignet med resultatet i c) gir
 $x_{maks} = 0,693 < \langle x \rangle = 1,083$. Dett skyldes at i
 $\langle x \rangle$
veies store $x$ bidrag mer enn i $P(x)$.  

\subsubsection*{Oppgave 4.4}
Vi skal i denne oppgaven behandle en \'{e}n--dimensjonal, harmonisk
oscillator med masse m og stivhet k.

%
\begin{itemize}
%
\item[a)] Skriv ned uttrykket for oscillatorens potensielle energi V(x)
og dens klassiske vinkelfrekvens $\omega$.

\item[b)] Skriv ned den tidsavhengige og den tidsuavhengige
Schr\"{o}dingerligningen for oscillatoren. Gj\o r kort rede for
hvordan, og under hvilke betingelser vi f\aa r den tidsuavhengige
Schr\"{o}dingerligningen fra den tidsavhengige.
%
\end{itemize}
%
Gitt to normerte tilstandsfunksjoner $\psi(x)$  og $\phi(x)$ for oscillatoren
ved tiden t = 0,
\begin{eqnarray*}
\psi(x) &=& \left( \frac{\alpha }{\sqrt{\pi }} \right)^{\frac{1}{2}} \exp\left( -
\frac{1}{2}\alpha^{2} x^{2} \right),\\
\phi(x) &=& \left( \frac{2\alpha^{3}}{\sqrt{\pi }} \right)^{\frac{1}{2}} x
\exp \left( -\frac{1}{2}\alpha^{2} x^{2} \right),
\end{eqnarray*}
der $\alpha \equiv \left( mk / \hbar^{2} \right)^{\frac{1}{4}}$.

%
\begin{itemize}
%
\item[c]) Vis at $\psi(x)$ og $\phi(x)$ er energi egentilstander for
oscillatoren. Finn de tilh\o rende energi egenverdiene. Skriv ned uttrykket
for de to energi egentilstandene ved tiden t.
%
\end{itemize}
%
\subsubsection*{Oppgave 4.5, Eksamen V-1997}
Vi skal i denne oppgaven studere det kvantemekaniske problem med 
en partikkel med masse $m$ som beveger seg i et potensial av formen 
%
\[
V(x) = \left \{
\begin{array}{c}
     V_0  \\
         0\\
    V_0
\end{array}
%
\;\;\; \mbox{for} \;\;\;
%
\begin{array}{r}
 x < -a \\
 -a \leq x < +a \\
 x \geq  +a
\end{array}
%
\right .
%
\]
%
\begin{itemize}
%
\item[a)] For $0< E < V_0$ l�s den tidsuavhengige egenverdiligningen for hvert av de 
tre omr�dene hvor den potensielle energien er konstant.
%
\item[b)] Gi en kort oversikt over de krav som stilles til den kvantemekaniske
b�lgefunksjonen. Bruk dette til � finne de betingelsesligningene 
som egenfunksjonene i punkt a) m� tilfredsstille.
%
\item[c)] Vi antar n� at grunntilstands energien $E_0 = V_0 / 2$. Bruk dette til 
          � l�se ligningene i b) og vis at 
%
\[
  a = \frac{1}{4} \sqrt{\frac{\hbar^2 \pi^2}{m V_0}}
\]
%
\item[d)] Bestem alle de andre  konstantene  for grunntilstands 
b�lgefunksjonen. Lag en skisse av funksjonen
og vis at den tilfredsstiller $\Psi(-x) = + \Psi(x)$.
%
\item[e)] Den f�rste eksiterte tilstand m� tilfredsstille $\Psi(-x) = - \Psi(x)$.
Bruk dette og uttrykkene i a) til � bestemme formen p� den f�rste eksiterte
tilstanden. Skiss\'{e}r funksjonen.
%
\item[f)] Uten � l�se problemet eksakt, pr�v � finne argumenter som viser at en slik
eksitert tilstand er mulig.
% 
\end{itemize}

\subsubsection*{L\o sning}
\begin{itemize}
% 
\item[a)] Omr�det I: $-\infty < x < -a$.
%
\[
-\frac{\hbar^2}{2 m} \frac{d^2 }{dx^2}\psi_{I}(x)
  + V_0 \psi_{I}(x) = E \psi_{I}(x),
\]
%
med l{\o}sning
%
\[
%
\psi_{I}(x) =  A e^{+\alpha x} 
   \quad \mbox{hvor} \quad  \alpha = \sqrt{\frac{2m}{\hbar^2}\left (
   V_0 - E\right ) }
\]
%
Komponenten $e^{-\alpha x}$ vokser over alle grenser for $ x
\longrightarrow -\infty$ og m� fjernes.

Omr�det II: $-a< x < +a$.
%
\[
-\frac{\hbar^2}{2 m} \frac{d^2 }{dx^2}\psi_{II}(x) = E \psi_{II}(x),
\]
%
med l{\o}sning
%
\begin{equation}
%
\psi_{II}(x) =  B e^{+ik x} + C e^{-ikx} 
   \quad \mbox{hvor} \quad  k = \sqrt{\frac{2m}{\hbar^2} E }
\end{equation}
%
Omr�det III: $+a< x < + \infty$.
%
\[
-\frac{\hbar^2}{2 m} \frac{d^2 }{dx^2}\psi_{III}(x)
  + V_0 \psi_{III}(x) = E \psi_{III}(x),
\]
%
med l{\o}sning
%
\[
%
\psi_{III}(x) =  D e^{-\alpha x} 
   \quad \mbox{hvor} \quad  \alpha = \sqrt{\frac{2m}{\hbar^2}\left (
   V_0 - E\right ) }
\]
%
Komponenten $e^{+\alpha x}$ vokser over alle grenser for $ x
\longrightarrow +\infty$ og m� fjernes.
% 
\item[b)] L�sningen $\psi(x)$ av Schr\"{o}dinger ligningen m�
tilfredsstille f�lgende krav
   \begin{enumerate}
   % 
   \item $\psi(x)$ m� v�re \'{e}ntydig, kontinuerlig og begrenset overalt.
   % 
   \item For bundne systemer m� funksjonen v�re norm\'{e}rbar.
   %
   \item Overalt hvor den potensielle energien  $V(x)$ er endelig, m� den
   deriverte av  $\psi(x)$ v�re kontinuerlig.
   \end{enumerate}
%
I   punktet $ x = -a$ f�r vi betingelsene
% 
\begin{eqnarray}
A e^{-\alpha a} &=& B e^{-ika} + C e^{+ika}\\
\alpha A e^{-\alpha a} &=& ik \left (B e^{-ika} - C e^{+ika} \right )
\end{eqnarray}
%
og i punktet $ x = +a$ betingelsene
% 
\begin{eqnarray}
D e^{-\alpha a} &=& B e^{+ika} + C e^{-ika}\\
-\alpha D e^{-\alpha a} &=& ik \left (B e^{+ika} - C e^{-ika} \right )
\end{eqnarray}
%
\item[c)] For $E_0 = V_0/2$ f�r vi 
% 
\[
k = \sqrt{\frac{2 m E}{\hbar^2}} = \sqrt{\frac{mV_0}{\hbar^2}}; 
\quad 
\alpha = \sqrt{\frac{2 m}{\hbar^2}\left ( V_0 - E \right )}
                                     = \sqrt{\frac{mV_0}{\hbar^2}} = k 
\]
%
Det gir 
% 
\[
B + C e^{2ika} = i \left ( B - C e^{+2ika} \right )
\]
%
Realdelen og imagin�rdelen gir to separate ligninger
% 
\begin{eqnarray*}
B + C \cos (2ka) &=& C \sin (2ka)\\ 
B - C \cos (2ka) &=& C \sin (2ka)
\end{eqnarray*}
med l�sning $2 C \cos (2ka) = 0$ og $ka = \pi/4$ for
grunntilstanden. Dette gir 
% 
\[
a = \frac{\pi}{4k} = \frac{1}{4}\sqrt{\frac{\hbar^2 \pi^2}{m V_0}}
\]
% 
Ligningsettet ovenfor gir muligheter for andre l�sninger, men de  vil
ha $ka > \pi/4$ og kan ikke svare til grunntilstanden av systemet som
forutsatt.

Fra ligningene (4) og (5) for $x = +a$ f�r vi 
% 
\[
B + C e^{-2ika} = -i \left ( B - C e^{-2ika} \right )
\]
%
Realdelen og imagin�rdelen gir to separate ligninger
% 
\begin{eqnarray*}
B + C \cos (2ka) &=& C \sin (2ka)\\ 
-B + C \cos (2ka) &=& -C \sin (2ka)
\end{eqnarray*}
med samme l�sning for $a$.
%
\item[d)] Fra l�sningen i c) f�lger at $C = B$ og fra lign~(2) kan vi
bestemme $B$ og $C$ som funksjon av A
%
\[
B = C = \sqrt{\frac{1}{2}} e^{-\alpha a} A
\]
%
Fra lign~(4) f�r vi $ D = A$. Grunntilstands b�lgefunksjonen blir da
% 
\begin{eqnarray}
x < -a :&& \psi(x) = A e^{\alpha x} \nonumber \\
-a < x < +a : && \psi(x) = A e^{-\alpha a} \sqrt{2} \cos (ka) \nonumber\\
+a < x : && \psi(x) = A e^{-\alpha x}
\end{eqnarray}
% 
og denne funksjonen har paritet$=+1$.
% 
Konstanten $A$ bestemmes ved normalisering
% 
\begin{eqnarray*}
1 &=& 2 \int_{0}^{+\infty} \psi^{\ast}(x) \psi(x) dx 
        = \int_0^a \psi_{II}^{\ast}(x) \psi_{II}(x) dx 
          + \int_a^{+\infty} \psi_{III}^{\ast}(x) \psi_{II}(x) dx\\
  &=& 4 |A|^2 e^{-2\alpha a} \int_0^a \cos^2(kx)dx 
     + 2 |A|^2 \int_a^{+\infty} e^{-2\alpha x} dx\\
 &=& 4 |A|^2 e^{-2\alpha a}\frac{1}{k} \left ( \frac{1}{2} \sin (ka)
        \cos (ka) + \frac{1}{2} k a \right ) 
       + |A|^2 \frac{1}{\alpha} e^{- 2 \alpha}\\
&=& |A|^2 \left ( \frac{1}{2} + \frac{3}{\pi} \right ) e^{-\pi/2} a
\end{eqnarray*}
% 
som gir 
% 
\[
|A|^2 = A^2 = \frac{1}{a} \left ( \frac{1}{2} + \frac{3}{\pi} \right )
e^{\pi/2}
\]
%
\item[e)] F�rste eksiterte tilstand har paritet$=-1$. Det gir
$\psi_1(x = 0) = 0$. Ser vi p� den generelle l�sningen $\psi_{II}(x)$
ovenfor m� det bety at  $B = - C$ og formen $\psi_1(x) = 2Bi \sin
(k_1x)$.
For $x < -a$ og $x > +a$ blir formen $\pm konstant \;e^{\pm\alpha_1 x}$.
%
%\begin{figure}[hbtp]
%
%\setlength{\unitlength}{1cm}
%
%\begin{center}
%
%\begin{picture}(7,5)
%
%\thicklines
%
%\put(0,0){\framebox(5,5){}}
%\put(0,0){\epsfxsize= 5cm \epsfbox{oblig-2-23-fig1.ps}}
%
%\label{fig1}
%\end{picture}
%
%\caption{Skisse av de tre laveste egenefunksjonene}
%
%\end{center}
%
%\end{figure}

\item[f)] Den f�rste bundne eksiterte tilstand m� ha $E_1 <
V_0$. Dette gir
%
\[
 k_1 = \sqrt{\frac{2m}{\hbar^2} E_1} < \sqrt{\frac{2m}{\hbar^2} V_0}
\]
%
og 
%
\[
(k_1a)_{maks} < a \sqrt{\frac{2m}{\hbar^2} V_0} 
              = a \sqrt{2} k = \frac{\pi}{4}\sqrt{2}
\]
%
Ser vi n�rmere p� b�lgefunksjonen skissert i fig.1 for $\psi_1(x)$ i punktet $x = a$
svarer det til en verdi p� argumentet $ka > \pi/2$. Dette betyr at
betingelsene ovenfor ikke oppfylles, og f�lgelig har vi kun \'{e}n
bunden egentilstand.
%
\end{itemize}

%
\noindent {\large \bf  Alternativ l�sning:}\\
%
\begin{itemize}
% 
\item[a)] Vi kan alternativt skrive l�sningen i omr�det $ -a < x < +a$
p� formen  
%
\begin{equation}
%
\psi_{II}(x) =  B \cos (kx) + C \sin (kx) 
\end{equation}
%
%
\item[b)] Betingelsesligningene blir da:\\
I punktet $x = -a$
% 
\begin{eqnarray}
A e^{-\alpha a} &=& B \cos (ka) - C \sin(ka)\\
\alpha A e^{-\alpha a} &=& k \left (B \sin (ka) + C \cos (ka) \right )
\end{eqnarray}
%
og i punktet $ x = +a$
% 
\begin{eqnarray}
D e^{-\alpha a} &=& B \cos (ka) + C \sin (ka)\\
-\alpha D e^{-\alpha a} &=& k \left (- B \sin (ka) + C \cos (ka) \right )
\end{eqnarray}
%
\item[c)] Dette gir i punktet $x = -a$
%
\[
B \cos (ka) - C \sin (ka) = B \sin (ka) + C \cos (ka)
\]
% 
eller
% 
\[
(B - C)\cos (ka) - (B + C) \sin (ka) = 0
\]
% 
med fire mulige l�sninger
% 
\begin{eqnarray*}
B = 0:&& \quad
\cos (ka) + \sin (ka) = 0;
\;\longrightarrow \;
  ka = \frac{3\pi}{4}\\
 C = 0: &&\quad \cos (ka) - \sin (ka) = 0;
\;\longrightarrow \;
  ka = \frac{\pi}{4}\\
B = C: && \quad
\sin (ka) = 0;
\;\longrightarrow \;
  ka = \pi\\
B = C:&& \quad \cos (ka) = 0;
\;\longrightarrow \;
  ka = \frac{\pi}{2}
\end{eqnarray*}
%
I punktet $x = +a$ f�r vi betingelsen 
%
\[
B \cos (ka) + C \sin (ka) = B \sin (ka) - C \cos (ka)
\]
% 
eller
% 
\[
(B + C)\cos (ka) - (B - C) \sin (ka) = 0
\]
% 
igjen med fire  mulige l�sninger
% 
\begin{eqnarray*}
%
B = 0:&& \quad \cos (ka) + \sin (ka) = 0;
\;\longrightarrow \;
  ka = \frac{3\pi}{4}\\
C = 0: && \quad \cos (ka) - \sin (ka) = 0;
\;\longrightarrow \;
  ka = \frac{\pi}{4}\\
B = C: && \quad \cos (ka) = 0;
\;\longrightarrow \;
  ka = \frac{\pi}{2}\\
B = -C: && \quad \sin (ka) = 0;
\;\longrightarrow \;
  ka = \pi
\end{eqnarray*}
%
To l�sninger er mulige n�r vi skal  tilfredsstille betingelsene i
begge punktene $x = \pm a$, $B = 0$ og $ka = 3\pi/4$ eller $C = 0$ og
$ka = \pi/4$.

L�sningen $B = 0$ svarer til f�rste eksiterte tilstand. Det skulle g�
frem av fig.1. Mens l�sningen $C = 0$ er grunntilstanden -- den vi
�nsker. Og vi f�r samme svar som tidligere.

%
\end{itemize}


\subsection{Numeriske oppgaver}

\subsubsection*{Oppgave 4.6}

Her skal vi studere en partikkel som beveger seg langs $x$-aksen i et
potensial
\begin{equation}
 V(x)=\left\{
              \begin{array}{cc} 0    & x>a         \\ 
                                -V_0 &-a < x \le a \\
                                0    & x \le -a 
              \end{array} \right. , 
\label{eq:l1}
\end{equation}
vist i figur \ref{fig:f1} med energi $E$ og $V_0>0$. Klassisk vil en
slik partikkel v�re fanget i boksen (omr�de $II$) og skli frem og
tilbake med jevn hastighet dersom $E<0$. Det er alts� null
sannsynlighet for at partikkelen befinner seg utenfor denne
boksen. Kvantemekanisk blir denne sannsynligheten ulik null. 

De matematiske krav som m� stilles til en fysisk akseptabel l�sning
$\psi(x)$ av egenverdilikningen i \'en dimensjon er:

\begin{enumerate}
 \item $\psi(x)$ m�\ v�re kontinuerlig.
 \item $\psi(x)$ m�\ v�re normaliserbar, dvs vi m�\ 
       kunne finne en konstant $C_N$ slik at 
\begin{equation}
   \int_{-\infty}^{+\infty}(C_N\psi(x))^*(C_N\psi(x))dx=1, 
\label{eq:l2}
\end{equation}
     n�r fysikken i problemet er slik at systemet er begrenset 
     i rommet. 
\item $\frac{d}{dx}\psi(x)=\psi'(x)$ m� v�re kontinuerlig 
      n�r potensialet $V(x)$ er endelig. 
\end{enumerate}

Vi skal se p� tilfellet der partikkelen har 
negativ energi, $E<0$, dvs.~vi �nsker � studere bundne tilstander.  

\begin{figure}[h]
\begin{center}
\setlength{\unitlength}{0.8cm}
\begin{picture}(13,6)
\thicklines
   \put(0,0.5){\makebox(0,0)[bl]{
              \put(8,1){\vector(1,0){4}}
              \put(12.3,1){\makebox(0,0){x}}
              \put(5.1,1.5){\makebox(0,0){$-a$}}
              \put(8.1,1.5){\makebox(0,0){$a$}}
              \put(4,0){\makebox(0,0){$I$}}
              \put(6,0){\makebox(0,0){$II$}}
              \put(10,0){\makebox(0,0){$III$}}
              \put(8.5,-3){\makebox(0,0){$-V_0$}}
              \put(5,1){\line(0,-1){4}}
              \put(5,1){\line(-1,0){3}}
              \put(5,-3){\line(1,0){3}}
              \put(8,1){\line(0,-1){4}}
         }}
\end{picture}
\end{center}
\caption{Partikkel i endelig boks}
\label{fig:f1}
\end{figure}

\begin{itemize}
%
\item[a)] 
Sett opp den tidsuavhengige Schr\"odingers likning for de tre omr�dene
$I$, $II$ og $III$ som vist p� figuren. Hvorfor kan vi skille ut
tidsavhengigheten, og hvilket analytisk uttrykk har denne?

\item[b)] 
Sett opp de generelle b�lgefunksjonene for de tre omr�dene og vis
at de tilsvarende energiniv�ene er gitt ved likningene 
\begin{equation} \label{eq:l3}
   k\cot(ka)=-\beta 
\ee
\be \label{eq:l4}   k\tan(ka)=\beta
\ee
for hhv. den odde og den like b�lgefunksjonen i omr�de $II$.
Her har vi definert $\beta=\sqrt{2m|E|}/\hbar$ og
$k=\sqrt{2m(V_0-|E|)}/\hbar$.

\item[c)] 
Hvor sterkt m� potensialet $V_0$ v�re for at vi skal
ha minst en bunden tilstand, dvs.~$E\le0$?
\end{itemize}
\subsubsection*{Numerisk del}

Vi skal her studere relasjonene (\ref{eq:l3}) og (\ref{eq:l4}) gitt i
oppgave \emph{1b} og finne energien $E$ for de to lavest bundne
tilstandene. 
\begin{itemize}
\item[d)] 
Vi skal n� l�se numerisk de
\emph{transcendentale}\footnote{Transcendentale likninger er likninger
som inneholder transcendentale funksjoner (exp, sin, cos et.c.), og
som sjelden har analytiske l�sninger, slik at vi m� ty til numeriske metoder.} likningene
\[ 
   k\cot(ka)=-\beta.
\]
for den odde b�lgefunksjonen i omr�de $II$ og 
\[ 
   k\tan(ka)=\beta.
\]
for den like b�lgefunksjonen i omr�de $II$. Som f�r er
$\beta=\sqrt{2m|E|}/\hbar$ og $k=\sqrt{2m(V_0-|E|)}/\hbar$. Anta at
partikkelen som er fanget i potensialet er et elektron med masse
$m_e=0.511 \textrm{MeV}/c^2$. Parametrene $V_0$ og $a$ velger du slik
at du f�r minst to bundne tilstander.

Finn energien til grunntilstanden og til neste tilstand med de valgte
parametrene.

\end{itemize}

\subsubsection*{Oppgave 4.7}
Vi repeterer kort betingelsene fra forrige oppgave. Vi studerer alts�
et elektron som beveger seg langs $x$-aksen i et potensial 
\be
 V(x)=\left\{\begin{array}{cc}0&x>a  \\ 
                              -V_0&-a < x \le a \\
                               0 &x \le -a\end{array}\right. , 
\label{eq:l1}
\ee
vist i figur \ref{fig:f1} med energi $E$ og $V_0>0$. 

De matematiske krav som m� stilles til en fysisk akseptabel l�sning
$\psi(x)$ av egenverdilikningen i \'en dimensjon er:

\begin{enumerate}
 \item $\psi(x)$ m�\ v�re kontinuerlig.
 \item $\psi(x)$ m�\ v�re normaliserbar, dvs vi m�\ 
       kunne finne en konstant $C_N$ slik at 
\be
   \int_{-\infty}^{+\infty}(C_N\psi(x))^*(C_N\psi(x))dx=1, 
	\label{eq:l2}
\ee
     n�r fysikken i problemet er slik at systemet er begrenset 
     i rommet. 
\item $\frac{d}{dx}\psi(x)=\psi'(x)$ m� v�re kontinuerlig 
      n�r potensialet $V(x)$ er endelig. 
\end{enumerate}

Vi skal se p�\ tilfellet der partikkelen har 
negativ energi, $E<0$, dvs.~vi �nsker � studere
bundne tilstander. Klassisk er partikkelen da fanget i omr�de $II$.  

Potensialet v�rt er symmetrisk, dvs.~$V(x)=V(-x)$ men b�lgefunksjonene
kan v�re enten symmetriske eller antisymmetriske. Ved origo utviser
disse b�lgefunksjonene ulike egenskaper. For en symmetrisk (like) b�lgefunksjon
har vi
\[
   \psi(0)=konstant\ne 0, \hspace{1cm} \frac{d\psi}{dx}|_{x=0}=0,
\]
mens en antisymmetrisk (uilike) b�lgefunksjon har 
\[
   \psi(0)=0, \hspace{1cm} \frac{d\psi}{dx}|_{x=0}=konstant\ne 0.
\]

Vi kan velge konstanten lik $1$ og deretter normalisere b�lgefunksjonen
v�r. 

\begin{itemize}
%
\item[a)] 
Du skal l�se den tidsuavhengige schr\"odingerlikning for
partikkel-i-endelig-boks med energiegenverdiene du fant i forrige numeriske
oppgave. Dersom du ikke gjorde forrige oppgave, kan du bruke verdiene $V_0=10$
eV, $a=1 \mathrm{nm}$, $E_0 = 9.916640434\mathrm{eV}$ og $E_1 =
9.666726218\mathrm{eV}$.

L�s schr\"odingerlikningen i omr�de $II$ (der $V(x)=-V_0$) numerisk for de to
bundne tilstandene med lavest energi. Fremstill de resulterende
b�lgefunksjonene grafisk i omr�det $-a$ til $a$. Tegn ogs� ogs�
sannsynlighetstettheten. 

\item[b)] 
Norm\'er de to b�lgefunksjonene (denne gangen med l�sningene
i $I$ og $III$) med Maple eller Matlab 
og fremstill funksjonene i en graf sammen
med potensialet. Reflekt\'er omkring sannsynligheten for � finne
elektronet p� ulike steder i potensialfella.

Bruk de analytiske uttrykkene for b�lgefunksjonen med ukjente koeffisienter
som du fant i forrige numeriske oppgave. 
Husk ogs� de matematiske betingelsene for en b�lgefunksjon fra
forrige numeriske oppgave.

\item[c)] 
Hva skjer med de to egenfunksjonene og egenverdiene dersom du lar $a
\rightarrow \infty$, mens $V_0$ holdes konstant? Pr�v deg gjerne frem b�de
analytisk og numerisk. Hvordan vil du tolke resultatet?
\end{itemize}

\clearemptydoublepage
\chapter{Quantum mechanics}
\begin{quotation}
How empty is theory in the presence of fact. {\em Mark Twain}
\end{quotation}

\section{Viktige postulater}
Schr\"{o}dingerlikningen bestemmer b{\o}lgefunksjonen $\Psi$ for det fysiske system, 
og kvantemekanikken postulerer at $\Psi$ inneholder all informasjon vi kan f{\aa}
om systemet.
I avsnitt \ref{sec:sltolkning} diskuterte vi Borns forslag til tolkning
av b\o lgefunksjonen,
\[
   P(x,t)dx=\Psi(x,t)^*\Psi(x,t)dx
\]
hvor sannsynlighetstettheten $P$ 
gir oss sannsynligheten for \aa\ finne partikkelen innafor et omr\aa de
$x$ til $x+dx$.  

Denne tolkningen leder da til innf\o ringen av begrep fra
sannsynlighetsl\ae re og statistikk. Forventningsverdien,
eller middelverdien,
av en st\o rrelse $A$ kan da reknes ut ifra
\be
   \langle A \rangle =\frac{\sum_iA_iP_i}{\sum_iP_i},
\ee
hvor vi har antatt en diskret sannsynlightesfordeling,
f.eks.~sannsynligheten for utkomme av terningkast med
to terninger. $P_i$ representerer da sannsynligheten for en bestemt
hending, f.eks.~muligheten for \aa\ f\aa\ tallet seks i et kast
med to terninger. Dersom summen over alle sannsynlighetene
er normert til en, kan vi droppe nevneren i det siste
uttrykket.
Helt analogt, dersom vi har en kontinuerlig fordeling
$P(x)$ kan vi definere forventningsverdien som
(med normert fordeling)
\be
   \langle A \rangle =\int_{-\infty}^{\infty}A(x)P(x)dx.
\ee

I kvantemekanikken kan vi betrakte
b{\o}lgefunksjonen $\Psi(x,t)$ som en fordelingsfunksjon
til bestemmelse av posisjonen $x$. 
Den kan ikke bestemme partikkelens 
posisjon eksakt, men kun den midlere posisjon. Helt analogt 
med forrige likning definerer vi midlere posisjon som  
%
\begin{equation}
\langle x \rangle = \int \Psi^{\ast}(x,t) x \Psi(x,t) dx .
\label{eq1}
\end{equation}

For at dere ikke skal komme helt bort i den kvantemekaniske
skogen, la oss f\o rst samle opp en del egenskaper og prinsipper
om kvantemekaniske systemer. Den innramma delen representer ogs\aa\
deler av det som kan kalles for K\o benhavner tolkningen av 
kvantemekanikken, etter hovedsaklig Bohr og Heisenberg.

\begin{center}
\shabox{\parbox{14cm}{\begin{itemize}\item Tilstanden til et kvantesystem bestemmes av l\o sningen av Schr\"odingers likning , b\o lgefunksjonen. B\o lgefunksjonen kvadrert tolkes som
en sannsynlighetstetthet. Schr\"odingers likning  er v\aa r bevegelseslov for kvantesystemer.
\item Kvanteobjekter 'styres' av Heisenbergs uskarphetsrelasjon, det er
ikke mulig \aa\ m\aa le b\aa de posisjon og bevegelsesmengde skarpt samtidig.
\item Det faktum at materie har partikkel og b\o lgenatur, og at vi ikke
er i stand til \aa\ observere begge disse egenskapene samtidig, leder til
{\em komplementaritetsprinsippet}, b\o lge og partikkel egenskaper er to
komplement\ae re side av materie.
\item {\bf Korrespondanseprinsippet} forteller oss at i grensa store
kvantetall skal vi gjenvinne resultater fra klassisk fysikk.
\end{itemize}}}\end{center}

Det er klart at det er ikke lett \aa\ tilskreve en b\o lgefunksjon 
som er kompleks
en intuitiv forklaring. Flere av kvantemekanikkens postulater
har ogs\aa\ leda til interessante filosofiske tolkninger av teoriens
konsekvenser. Kan vi bruke b\o lgefunksjonen~til \aa\ si noe om systemet, eller 
gjenspeiler den bare v\aa r begrensning i kunnskap?

Det vi skal legge til de innramma egenskapene ved kvantemekanikken
i dette avsnittet, er en formell 
matematisk beskrivelse av fysiske systemer basert p\aa\ ulike postulater.

Vi studerer f\o rst problemet i \`{e}n dimensjon.
Den kvantemekaniske energi egenverdilikningen er
da gitt ved
%
\begin{equation}
\left [ \frac{1}{2 m}
	  \left ( -\hbar^2 \frac{d^2}{dx^2}\right )
		+ V(x) \right ] \psi(x)
		= E \psi(x).
\label{a1}
\end{equation}
%
Venstre side er skrevet som et produkt av to
faktorer. Dette skal forst{\aa}s slik at hvert ledd i
parentesen $ [ \ldots ] $ skal virke p{\aa} b{\o}lgefunksjonen
$ \psi(x)$. N{\aa}r $d^2/dx^2$ opererer p{\aa} $\psi(x)$
blir resultatet den annen
deriverte av $\psi(x)$, mens $V(x)$ multipliseres med $\psi(x)$.
I matematisk spr{\aa}kbruk kan vi si at uttrykket
%
\begin{equation}
\OP{H}
  = \frac{1}{2 m} \left ( -\hbar^2\frac{d^2}{dx^2} \right ) + V(x),
\label{a2}
\end{equation}
%
er en operator som n{\aa}r den virker p{\aa} en funksjon $\psi(x)$
produserer en ny funksjon som et
resultat av en serie matematiske operasjoner
fastlagt gjennom definisjonen av $\OP{H}$. Vi vil i det
f{\o}lgende bruke skrivem{\aa}ten $\OP{A}$ for {\aa}
markere at A er en kvantemekanisk operator.
Med innf{\o}ring av operator begrepet kan vi skrive likning (\ref{a1})
p{\aa} formen
%
\begin{equation}
\OP{H} \psi(x) = E \psi(x).
\label{a3}
\end{equation}
%
Dette betyr at virkningen av $\OP{H}$ p{\aa}
b{\o}lgefunksjonen $\psi(x)$ er det samme
som {\aa} multiplisere $\psi(x)$ med en konstant $E$.

I alminnelighet vil det ikke v{\ae}re slik at n{\aa}r $\OP{H}$
opererer p{\aa}
en vilk{\aa}rlig funksjon, vil resultatet bli den samme
funksjonen multiplisert med en konstant.
Funksjoner som tilfredsstiller likning (\ref{a3}) kalles egenfunksjoner for
operatoren $\OP{H}$, og den tilh{\o}rende verdien av $E$ er egenverdien
for operatoren.

For en vilk{\aa}rlig operator $\OP{A}$ vil de tilh{\o}rende egenfunksjonene
og egenverdiene tilfredsstille likningen
%
\begin{equation}
\OP{A} \Phi_{\nu}(x) = a_{\nu} \Phi_{\nu},
\label{a4}
\end{equation}
%
med en serie egenverdier $ a_1, a_2, a_3 \ldots $
og  med tilh{\o}rende egenfunksjoner $ \Phi_1(x)$, $ \Phi_2(x),
\Phi_3(x), \ldots$ . Dette avhenger av formen p{\aa} operatoren
og de matematiske randbetingelsene. Egenfunksjonene  er en
konsekvens av de fysiske egenskapene for systemet. I
noen tilfeller vil  flere
egenfunksjoner ha  samme egenverdi. I slike
tilfeller sier vi at systemet er {\sl degenerert}.

I kvantemekanikken vil alle operatorer som representerer
fysiske st{\o}rrelser v{\ae}re av en bestemt
matematisk type -- {\sl Hermite'ske operatorer} (navngitt
etter en fransk matematiker C.~Hermite). Slike operatorer
tilfredsstiller betingelsen
%
\begin{equation}
\int \left (\Phi_{\mu} \right )^{\ast}
	 \left (\OP{A} \Phi_{\nu}\right ) d\tau
= \int \left (\OP{A} \Phi_{\mu} \right )^{\ast} \Phi_{\nu} d\tau,
\label{a5}
\end{equation}
%
for alle funksjoner $\Phi_{\mu}$ og $\Phi_{\nu}$ som oppfyller
randbetingelsene for systemet.

\noindent Vi har n{\aa} f{\o}lgende:
%
\begin{itemize}
%
\item[a)] {\sl Egenverdiene for  Hermite`ske operator er
reelle,
%
\begin{equation}
(a_{\mu})^{\ast} = a_{\mu},
\label{a6}
\end{equation}
%
\item[b)] og egenfunksjonene er ortogonale,
%
\begin{equation}
\int \Phi_{\mu}^\ast \Phi_{\nu} d\tau = \delta_{\mu,\nu} N_{\nu},
\label{a7}
\end{equation}
%
hvis $\Phi_\mu$ og $\Phi_\nu$ h{\o}rer til to forskjellige
egenverdier $a_{\nu}$ og $a_{\mu}$ av operatoren $\OP{A}$.
St{\o}rrelsen $N_{\nu}$ er en normaliseringskonstant.}
%
\end{itemize}
Dette kan vises p{\aa} f{\o}lgende m{\aa}te:
Vi skriver likning (\ref{a4})
for to forskjellige egentilstander, kompleks konjugerer
den siste og f{\aa}r
%
\begin{eqnarray*}
\OP{A} \Phi_{\nu} = a_{\nu} \Phi_{\nu},\\
\left ( \OP{A} \Phi_{\mu} \right )^{\ast}
  = \left ( a_{\mu} \right )^{\ast} \left ( \Phi_{\mu} \right)^{\ast}.
\end{eqnarray*}
%
Multipliser f{\o}rste likning fra venstre med
$(\Phi_{\mu})^{\ast}$ og den andre likningen med $ \Phi_{\nu}$ fra
h{\o}yre og subtrah\'{e}rer. Dette gir
%
\[
\left ( \Phi_{\mu} \right )^{\ast} \OP{A} \Phi_{\nu}
	 - \left ( \OP{A} \Phi_{\mu} \right )^{\ast}\Phi_{\nu}
  = ( a_{\nu} - a_{\mu}^{\ast} )
	 \left ( \Phi_{\mu} \right )^{\ast} \Phi_{\nu}.
\]
%
Uttrykket integreres over hele konfigurasjonsrommet.
Ved bruk av likning (\ref{a5}) gir dette
%
\[
(a_{\mu} - a_{\nu}^{\ast} )
 \int \left ( \Phi_{\mu} \right )^{\ast} \Phi_{\nu} d\tau = 0.
\]
%
For $\mu = \nu$ er integralet lik 1 n{\aa}r funksjonene er
normerte. Dette gir
%
\[
a_{\mu} - a_{\mu}^{\ast} = 0 \; \;
			  \longrightarrow \;\; alle \;\; a_{\mu} \;\; er \;\; reelle.
\]
%
For $\mu \neq \nu$ m{\aa} integralet v{\ae}re null. Dette gir likning (\ref{a7}),
og derved er relasjonene ovenfor bevist.
%

Operatoren $\OP{H}$ i likning (\ref{a2}) spiller en viktig rolle
i kvantemekanikken. Den kalles {\sl Hamilton} operatoren
for systemet. Uttrykket er hentet fra klassisk mekanikk
hvor systemets totale energi kalles {\sl Hamilton} funksjonen.
Den skrives som en funksjon av koordinat og
bevegelsesmengde for systemet.
For en partikkel som beveger seg i \'{e}n dimensjon, er
den klassiske {\sl Hamilton} funksjonen gitt ved
%
\begin{equation}
H_{klassisk} = \frac{1}{2 m} p_x^2 + V(x).
\label{a8}
\end{equation}
%
Vi kan n{\aa} overf{\o}re denne klassiske {\sl Hamilton} funksjonen
til kvantemekanisk form p{\aa} en ganske enkel m{\aa}te.
Ved {\aa} sammenligne likning (\ref{a2}) og likning (\ref{a8}) ser vi at hvis den
klassiske bevegelsesmengden $p_x$ byttes ut med operatoren
%
\begin{equation}
p_x \longrightarrow -i \hbar \frac{d}{dx},
\label{a9}
\end{equation}
%
f{\aa}r vi den kvantemekaniske {\sl Hamilton} operatoren $\OP{H}$.
For bevegelse av en partikkel i tre dimensjoner
vil den klassiske {\sl Hamilton funksjonen} ha
formen
%
\begin{equation}
H_{klassisk} = \frac{1}{2 m} (\, \vec{p}\, )^2 + V(\vec{r}\, ),
\label{a10}
\end{equation}
%
hvor $\vec{r}$ er posisjonsvektoren for partikkelen
og $(\, \vec{p}\, )^2 = p_x^2 + p_y^2 + p_z^2$. I dette tilfelle
blir operator transformasjonene
%
\begin{equation}
p_x \longrightarrow -i \hbar\frac{\partial}{\partial x},\; \;
p_y \longrightarrow -i \hbar\frac{\partial}{\partial y},\; \;
p_z \longrightarrow -i \hbar\frac{\partial}{\partial z}.
\label{a11}
\end{equation}
%
Med bruk av Laplace operatoren
%
\begin{equation}
\vec{\bigtriangledown} = \vec{u}_x\frac{\partial}{\partial x}
						+ \vec{u}_y\frac{\partial}{\partial y}
						+ \vec{u}_z\frac{\partial}{\partial z},
\label{a12}
\end{equation}
%
kan vi sammenfatte likning (\ref{a11}) til
%
\begin{equation}
\vec{p} \longrightarrow -i \hbar \vec{\bigtriangledown}.
\label{13}
\end{equation}
%
Den kvantemekaniske
{\sl Hamilton} operatoren for det tredimensjonale tilfelle
blir da
%
\begin{equation}
\OP{H} = -\frac{\hbar^2}{2 m} (\vec{\bigtriangledown})^2
			+ V(\vec{r})
	= -\frac{\hbar^2}{2 m}
				\left ( \frac{\partial^2}{\partial x^2}
			  + 	\frac{\partial^2}{\partial y^2}
			  +	\frac{\partial^2}{\partial z^2} \right )
			  + V(\vec{r}).
\label{a14}
\end{equation}
%
Lar vi $\OP{H}$ virke p{\aa} funksjonen $\psi(\vec{r})$,
f{\aa}r vi
%
\begin{equation}
\OP{H} \psi(\vec{r}) = -\frac{\hbar^2}{2 m}
				\left ( \frac{\partial^2 \psi}{\partial x^2}
			  + 	\frac{\partial^2 \psi}{\partial y^2}
			  +	\frac{\partial^2\psi}{\partial z^2} \right )
			  + V(\vec{r}) \psi.
\label{a15}
\end{equation}
%
Hvis $\psi$ er en egenfunksjon til $\OP{H}$, har vi
$\OP{H} \psi = E \psi$, hvor $E$ er egenverdiene til $\OP{H}$.
Skrevet helt ut, gir dette
%
\begin{equation}
-\frac{\hbar^2}{2 m}
				\left ( \frac{\partial^2 \psi}{\partial x^2}
			  + 	\frac{\partial^2 \psi}{\partial y^2}
			  +	\frac{\partial^2\psi}{\partial z^2} \right )
			  + V(\vec{r}) \psi
			  = E \psi.
\label{16}
\end{equation}
%
%
Vi kan n{\aa} formulere kvantemekanikkens
f{\o}rste postulat:
\subsection{F\o rste postulat}
%
\begin{itemize}
\item[I] {\sl Til enhver fysisk st{\o}rrelse $A(\vec{r},\vec{p})$
som er en funksjon av posisjonen $\vec{r}$ og bevegelsesmengden
$\vec{p}$ til systemet, vil det svare en kvantemekanisk operator.
Den fremkommer ved {\aa} bytte $\vec{p}$  med
$-i\hbar \vec{\bigtriangledown}$. Dette gir den tilh{\o}rende
kvantemekaniske operatoren
%
\[
\OP{A} = A(\vec{r},-i\hbar \vec{\bigtriangledown)}.
\]
}
\end{itemize}
%
Hamilton operatoren i likning (\ref{a2}) eller likning (\ref{a14}) er den
kvantemekaniske operatoren som svarer til systemets
totale energi i henholdsvis \'{e}n og tre dimensjoner.
P{\aa} tilsvarende m{\aa}te vil systemets kinetiske energi
$T = p^2 / 2 m$ gi operatoren
$ - (\hbar^2 / 2 m ) (\vec{\bigtriangledown})^2$ i tre dimensjoner
og $ (- \hbar^2 / 2 m ) d^2 / dx^2$ i \'{e}n dimensjon.
%

I tabell~\ref{t1} har vi samlet en del fysiske st{\o}rrelser
og de tilh{\o}rende kvantemekaniske ope\-ra\-to\-rene.
%

\begin{table}[htbp]
\caption{Kvantemekaniske operatorer.}

\vspace{0.2cm}

\begin{center}
%
\begin{tabular}{|l|l|l|}  \hline
Fysisk st{\o}rrelse & Klassisk definisjon & Kvantemekanisk operator\\
\hline
Posisjon            & $\vec{r}$           & $\OP{\vec{r}} = \vec{r}$\\
Bevegelsesmengde    & $\vec{p}$
						  & $\OP{\vec{p}} = -i \hbar \vec{\bigtriangledown}$\\
Banespinn           & $\vec{L} = \vec{r} \times \vec{p}$
		  & $\OP{\vec{L}} = \vec{r} \times (-i\hbar \vec{\bigtriangledown})$\\
Kinetisk energi     & $T = (\vec{p})^2 / 2 m$
						  & $\OP{T} = - (\hbar^2 / 2 m) (\vec{\bigtriangledown})^2$\\
Total energi 		  & $H = (p^2 / 2 m) + V(\vec{r})$
						  & $\OP{H} = - ( \hbar^2 / 2 m )(\vec{\bigtriangledown})^2
										  + V(\vec{r})$\\
\hline
\end{tabular}
%
\end{center}
%
\label{t1}
\end{table}
%
\subsection{Andre postulat}
Det neste kvantemekaniske postulat er f{\o}lgende:
%
\begin{itemize}
\item[II] {\sl De eneste mulige verdier vi kan f{\aa}
ved en ideell m{\aa}ling av den fysiske st{\o}rrelsen $A$
er egenverdiene for den tilsvarende kvantemekaniske
operatoren $\OP{A}$.}
\end{itemize}
%

Ut fra dette kan vi bestemme ikke bare energien, men ogs{\aa}
andre fysiske st{\o}rrelser for systemet, og postulatet fastlegger
hvilke fysiske informasjoner vi er i stand til {\aa}
finne. For en operator $\OP{A}$
kan vi formulere egenverdi likningen
%
\begin{equation}
\OP{A} \psi_{\nu}
	 = a_{\nu} \psi_{\nu},
\label{a17}
\end{equation}
%
som gir egenverdiene $ a_1, a_2, a_3,\cdots$
som de eneste mulige m{\aa}leresultater for den
fysiske st{\o}rrelsen $A$. De tilh{\o}rende
egentilstander $ \psi_1, \psi_2, \psi_3 \cdots$
angir systemets kvantemekaniske tilstander.
Dette betyr at hvis systemet er i en tilstand
beskrevet ved egenfunksjonen $\psi_{\nu}$ for
operatoren $\OP{A}$, er verdien av den fysiske st{\o}rrelsen
A lik $a_{\nu}$.


V{\aa}rt neste sp{\o}rsm{\aa}l blir da: Hvordan vil det v{\ae}re hvis
systemet er i tilstanden  $\Phi$ som ikke er
en l{\o}sning av likning (\ref{a17})?  I slike tilfeller vil m{\aa}ling
av den fysiske st{\o}rrelsen A  gi \'{e}n av egenverdiene
i likning (\ref{a17}). La oss skrive b{\o}lgefunksjonen $\Phi$ som
en line{\ae}r kombinasjon av egenfunksjonene $\psi_{\nu}$
for $\OP{A}$. Dette gir
%
\begin{equation}
\Phi = c_1 \psi_1 + c_2 \psi_2 + \cdots
  = \sum_{\nu} c_{\nu} \psi_{\nu}.
\label{a18}
\end{equation}
%
Siden funksjonene er ortogonale ($\OP{A}$ er Hermite'sk),
f{\aa}r vi
%
\begin{equation}
c_{\nu} = \int (\Phi)^{\ast} \psi_{\nu} d\tau.
\label{a19}
\end{equation}
%
\subsection{Tredje postulat}
Av dette kan vi formulere et tredje kvantemekaniske postulat:
%
\begin{itemize}
\item[III]
\sl
N{\aa}r b{\o}lgefunksjonen til et system er $\Phi$, vil
sannsynligheten for {\aa} f{\aa} verdien $a_{\nu}$ som et
resultat av en ideell m{\aa}ling av den fysiske st{\o}rrelsen
$A$ v{\ae}re $|c_{\nu}|^2$, hvor $c_{\nu}$
er gitt i likning (\ref{a19}), og $\psi_{\nu}$ er en egenfunksjon til
$\OP{A}$ som svarer til egenverdien $a_{\nu}$.
\end{itemize}
%
Hvis b{\o}lgefunksjonen $\Phi$ ikke er en egenfunksjon
for $\OP{A}$, vil vi i f{\o}lge dette tredje postulat
ikke kunne finne noen eksakt verdi for $A$. Gjentar vi m{\aa}lingen av den fysiske 
st{\o}rrelsen $A$ p{\aa} identiske systemer og de alle er 
i samme kvantemekaniske tilstand beskrevet
ved  b{\o}lgefunksjonen $\Phi$, vil vi f{\aa}
forskjellige resultater.
Hvert resultat vil komme med en viss sannsynlighet.
Vi kan imidlertid snakke om den {\sl midlere} verdien av
st{\o}rrelsen A for et system beskrevet ved tilstanden
$\Phi$.\\
%
Som et resultat av det tredje postulatet kan vi vise
at
%
\begin{itemize}
\item
{\sl
N{\aa}r et kvantemekanisk system er i tilstanden $\Phi$,
er middelverdien -- eller forventningsverdien --
for den fysisk st{\o}rrelsen $A(\vec{r}, \vec{p})$
gitt ved
%
\begin{equation}
\langle A \rangle 
	= \int (\Phi)^{\ast} \OP{A}(\vec{r}, -i \hbar\vec{\bigtriangledown})
		 \Phi d\tau.
\label{a20}
\end{equation}
%
%
Her m{\aa} vi f{\o}rst la $\OP{A}$ virke p{\aa}
tilstanden $\Phi$ til h{\o}yre. Resultatet multipliseres med
$(\Phi)^{\ast}$ og integreres. Vi forutsetter at funksjonen
$\Phi$ er normalisert, dvs. $\int (\Phi)^{\ast} \Phi d\tau = 1$.
Hvis $\Phi$ ikke er normalisert, f{\aa}r lign.(\ref{a20})
formen
%
\begin{equation}
 \langle A \rangle = \frac{\int (\Phi)^{\ast} \OP{A} \Phi d\tau}
			  {\int (\Phi)^{\ast} \Phi d\tau}.
\label{a21}
\end{equation}
%
}
\end{itemize}
%
\subsection{Fjerde postulat}
Postulatene vi har formulert s{\aa} langt beskriver en statisk
situasjon. Den gir ingen informasjon om tidsutviklingen
av systemet. For f{\aa} en dynamisk teori trenger vi
et fjerde kvantemekanisk postulat:
%
\begin{itemize}
\item[IV]
\sl Tidsutviklingen av et fysisk system er
bestemt av likningen
%
\begin{equation}
i \hbar \frac{\partial \Psi}{\partial t} = \OP{H} \Psi,
\label{a22}
\end{equation}
%
hvor $\OP{H}$ er den kvantemekaniske Hamilton operatoren
for systemet.
\end{itemize}
%
Denne likningen gir en b{\o}lgefunksjon $\Psi(\vec{r},t)$
som funksjon av posisjonen $\vec{r}$ og tiden $t$.
Hvis operatoren $\OP{V}$ for den potensielle energien
er eksplisitt uavhengig av tiden, kan vi skrive
$\Psi(\vec{r}, t)$ p{\aa} formen
%
\begin{equation}
\Psi(\vec{r}, t) = \psi(\vec{r}) \exp (-(i / \hbar) E t),
\label{a23}
\end{equation}
%
hvor $\psi$ er l{\o}sning av likning (\ref{a15}).
I dette tilfellet har vi $|\Psi(\vec{r}, t)|^2 = |\psi(\vec{r})|^2$,
dvs. sannsynlighetstettheten er uavhengig av tiden t.
Dette representerer {\sl stasjon{\ae}re} tilstander for et
kvantemekanisk system. En generell l{\o}sning av likning (\ref{a22})
vil v{\ae}re av formen
%
\[
\Phi(\vec{r},t) = \sum_n c_n \Psi_n(\vec{r},t)
		 = \sum_n c_n \psi_n(\vec{r}) \exp(-(i / \hbar) E_n t).
\]
%
Denne tilstanden vil ikke ha presis (skarp) energi, men v{\ae}re
en line{\ae}r kombinasjon av tilstander med forskjellig energi.
Den vil heller ikke v{\ae}re stasjon{\ae}r, i.e. $|\Phi(\vec{r},t)|^2$
vil v{\ae}re tidsavhengig.

\subsection{Kommutatorer}

Som der fremg{\aa}r at det vi hittil har sagt
vil alle fysiske st{\o}rrelser i kvantemekanikken
bli formulert som operatorer.
Noen eksempler er gitt i tabell~\ref{t1}.
Bruken av operatorer i kvantemekanikken
skaper imidlertid et nytt problem.
Hvis vi har et sammensatt
uttrykk, vil rekkef{\o}lgen av operatorene ha betydning.
Dette kan vi se av f{\o}lgende eksempel:
%
\begin{eqnarray}
\OP{x} \OP{p}_x \Phi &=& x ( -i\hbar ) \frac{d}{d x}\Phi
			= - i \hbar x \frac{d \Phi}{dx},\nonumber\\
\OP{p}_x \OP{x} \Phi &=& (-i \hbar ) \frac{d}{dx}( x \Phi )
			= -i \hbar ( \Phi + x \frac{d \Phi}{dx}.).
\label{a24}
\end{eqnarray}
%
Virkningen av operatorene $\OP{x} \OP{p}_x$ og $\OP{p}_x \OP{x}$
p{\aa} en vilk{\aa}rlig tilstand $\Phi$ blir forskjellige.
Begrepet {\bf kommutator} st{\aa}r sentralt
i kvantemekanikken. Den forteller hva som skjer n{\aa}r
vi bytter om rekkef{\o}lgen av to operatorer i et uttrykk.
Kommutatoren mellom operatorene $\OP{A}$ og $\OP{B}$
defineres ved
%
\begin{equation}
\left [\OP{A}, \OP{B} \right ]  \equiv  \OP{A} \OP{B} - \OP{B} \OP{A}.
\label{a25}
\end{equation}
%
Eksemplet i likning (\ref{a24}) ovenfor gir
%
\begin{equation}
\left [ \OP{x}, \OP{p}_x \right ]
	  = \OP{x} \OP{p}_x -  \OP{p}_x \OP{x} =  i \hbar.
\label{a26}
\end{equation}
%
Dette er en operator relasjon, og det betyr at n{\aa}r
det sammensatte operatorutrykket p{\aa} venstre siden
virker p{\aa} en b{\o}lgefunksjon er resultatet det samme
som n{\aa}r vi multipliserer b{\o}lgefunksjonen med konstanten
$ i \hbar $.\\
%
Andre eksempler er
%
\begin{eqnarray}
\left [ \OP{x}, \OP{p}_y \right ]
	  &=& \OP{x} \OP{p}_y -  \OP{p}_y \OP{x} = 0,\nonumber\\
\left [ \OP{x}, \OP{p}_z \right ]
	  &=& \OP{x} \OP{p}_z -  \OP{p}_z \OP{x} = 0.
\label{a27}
\end{eqnarray}
%

Vi skal diskutere \'{e}n viktig anvendelse av kommutator relasjoner.
La oss vende tilbake til postulat II
og egenverdilikningen~(\ref{a17}). Hvis systemet er i en egentilstand
$\psi_{\nu}$, vil en ideell m{\aa}ling av den fysiske st{\o}rrelsen A
gi som resultat den presise (skarpe) verdien $a_{\nu}$. Kan det
i en slik situasjon tenkes at vi ogs{\aa} kan f{\aa} en presis
(skarp) verdi
for en annen fysisk st{\o}rrelse? La oss kalle den B. Hvis det
er mulig, m{\aa}tte f{\o}lgende  egenverdilikninger
v{\ae}re oppfylt samtidig:
%
\begin{eqnarray}
\OP{A} \psi_{\nu, \mu} &=& a_{\nu} \psi_{\nu, \mu},\nonumber\\
\OP{B} \psi_{\nu, \mu} &=& b_{\mu} \psi_{\nu, \mu}.
\label{a28}
\end{eqnarray}
%
B{\o}lgefunksjonen er her karakterisert  med to indekser
$\nu$ og $\mu$ siden den er en
egenfunksjon for to forskjellige operatorer $\OP{A}$ og $\OP{B}$
med egenverdiene $a_{\nu}$ og $b_{\mu}$.
Vi kan vise at dette er mulig hvis de to operatorene
kommuterer.

Multiplis\'{e}r den f{\o}rste likningen med $\OP{B}$ og
den andre med $\OP{A}$ og subtrah\'{e}r.
Dette gir
%
\[
\left ( \OP{B} \OP{A} - \OP{A} \OP{B} \right ) \psi_{\nu, \mu}
= 0,
\]
%
dvs. hvis likning (\ref{a28}) er oppfylt, m{\aa} de to operatorene
kommutere. Det omvendte er ogs{\aa} tilfelle: Hvis de to
operatorene $\OP{A}$ og $\OP{B}$ kommuterer, kan vi finne
egentilstander for $\OP{A}$ og $\OP{B}$  samtidig, og ved
m{\aa}ling av de tilsvarende dynamiske variable vil vi f{\aa}
presise (skarpe) verdier for b{\aa}de A og B.

La oss ta  ett eksempel p{\aa} anvendelse av denne
regelen. Vi har vist i likning (\ref{a26}) at operatorene
$\OP{x}$ og $\OP{p}_x$ ikke kommuterer. Dette betyr at  vi
ikke kan finne en egentilstand for en partikkel hvor
de fysiske st{\o}rrelsene $x$ og $p_x$ har presise (skarpe)
verdier samtidig. Dette stemmer med hva vi tidligere er
kommet frem
til i forbindelse med analysen av Heisenbergs uskarphets
relasjon. Derimot er det mulig {\aa} m{\aa}le presise (skarpe)
verdier av $x$, $p_y$ og $p_z$ samtidig.

Det at en operator har en egenverdi er gitt ved uttrykket fra 
postulat nummer 2, dvs.
\[
  \OP{A}\psi_n = a_n\psi_n,
\]
hvor $a_n$ er et tall. 

Som et konkret eksempel kan vi ta grunntilstanden for den en-dimensjonale harmoniske
oscillator, som har en b\o lgefunksjon gitt ved 
\[
   \psi_{0}(x)=Ce^{-x^2\alpha^2},
\]
hvor $C$ og $\alpha$ er konstanter. Virker vi p\aa\ denne
b\o lgefunksjonen med hamiltonfunksjonen v\aa r 
\[
\OP{H}=-\frac{\hbar^2}{2m}\frac{d^2}{dx^2}+\frac{1}{2}kx^2,
\]
finner vi en veldefinert energi $E_0$ gitt ved
\[
   \OP{H}\psi_0=E_0\psi_0,
\]
hvor $E_0$  er en konstant. Dersom vi virker p\aa\ $\psi_0$ med
operatoren for bevegelsesmengden $\OP{p}$ har vi
\[
   \OP{p}\psi_0=-i\hbar\frac{d}{dx}\psi_0=-i\hbar 2x\alpha^2\psi_0\ne 
     \mathrm{konstant}\times \psi_0.
\]
Dette eksemplet forteller oss at vi kan ikke bestemme bevegelsesmengde
og energi skarpt samtidig. 
Operatorene $\OP{H}$ og $\OP{p}$ 
kommuterer ikke og har ikke felles egenfunskjoner.


I denne oversikten over det formelle grunnlaget for
kvantemekanikken har vi behandlet et system med en
partikkel. Men formalismen kan lett utvides til et
mer komplisert system av mange partikler.




\section{Forventningsverdier og operatorer}


Som et eksempel p\aa\ diskusjonen i forrige avsnitt, 
skal vi studere igjen 
en partikkel med masse $m$ som beveger seg i et uendelig bokspotensial
%
\begin{equation}
%
V(x) = \left \{
\begin{array}{c}
   + \infty  \\
  0\\
  + \infty
\end{array}
%
\;\;\; \mbox{for} \;\;\;
%
\begin{array}{r}
 x < -a \\
 -a \leq x < +a \\
 x \geq  +a
\end{array}
%
\right .
%
\label{eq2}
\end{equation}
%        
kan vi l{\o}se Schr\"{o}dingerligningen for partikkelens bevegelse
i omr{\aa}det $ -a < x < +a$ (se forrige kapittel).
B{\o}lgefunksjonene f{\aa}r formen
%
\begin{eqnarray}
\Psi_n(x,t) &=& \sqrt{\frac{1}{a}} \cos k_n x \;e^{(-i/\hbar) E_n t},\\
\Psi_n(x,t) &=& \sqrt{\frac{1}{a}} \sin k_n x \;e^{(-i/\hbar) E_n t},
%
\label{eq3}
\end{eqnarray}
%
Middelverdien av $x$ gir 
%
\begin{eqnarray}
\langle x \rangle 
        &=& \frac{1}{a} \int_{-a}^{+a} x \cos^2 k_n x dx = 0, \nonumber \\
\langle x \rangle 
        &=& \frac{1}{a} \int_{-a}^{+a} x \sin^2 k_n x dx = 0.
\label{eq4}
\end{eqnarray}
%
Generalisert til middelverdien av en vilk{\aa}rlig funksjon av $x$ f{\aa}r vi 
 f{\o}lgende uttrykk
%
\begin{equation}
\langle f \rangle = \int \Psi^{\ast}(x,t) f(x)\Psi(x,t) dx . 
\label{eq5}
\end{equation}
%
Som  eksempel kunne
vi beregne den midlere verdien av den potensielle energien
%
\begin{equation}
\langle V \rangle = \int \Psi^{\ast}(x,t) V(x) \Psi(x,t) dx .
\label{eq6}
\end{equation}
% 
I tillegg til den midlere verdien av en dynamisk variabel kan 
vi ogs{\aa} beregne spredningen ved bruk av standardavviket 
{\bf $\sigma$}. Tar vi for oss en diskret fordelingsfunksjon
$P$ har vi
%
\be
\sigma = \sqrt{\frac{1}{N}\left ( \sum_{i=1}^{N} (x_{i} 
                                        - \langle x \rangle )^2\right )}
\ee
eller mer generelt som
\be
  \sigma= \sqrt{\langle x^2 \rangle - \langle x \rangle^2} = \Delta x .
\label{eq7}
\ee
%
Her er $\langle x \rangle $ gitt i likning (\ref{eq1}) og $<x^2>$ beregnes ved bruk 
av formelen i likning (\ref{eq5})
%
%
\begin{equation}
\langle x^2 \rangle = \int \Psi^{\ast}(x,t) x^2\Psi(x,t) dx .
\label{eq8}
\end{equation}
%
Legg merke til at dette gir oss en mulighet til, med en gitt
b\o lgefunksjonen~som l\o sning av den tidsuavhengige Schr\"odingers likning , \aa\ rekne ut 
uskarpheten i posisjon. 
Observer ogs\aa\ rekkef\o lgen av leddene i integranden
i uttrykkene for den kvantemekaniske forventningsverdien.
I uttrykket for forventningsverdien av bevegelsesmengden
skal vi se at denne rekkef\o lgen ikke er vilk\aa rlig. 
Dvs.~at dersom vi hadde satt
\begin{equation}
\langle f \rangle = \int f(x) \Psi^{\ast}(x,t)\Psi(x,t) dx , 
\end{equation}
for $p$ vil det gi et forskjellig resultat. 
 
For egenfunksjonene med cosinus-l\o sningene, se likningen (\ref{eq4}),
f{\aa}r vi
%
\begin{equation}
\langle x^2 \rangle= \frac{1}{a}\int_{-a}^{+a} x^2 \cos^2 k_n x dx 
   = \frac{1}{3} a^2 - \frac{1}{2 k_n^2}
   = a^2 \left ( \frac{1}{3} - \frac{2}{(n\pi)^2} \right ),
\label{eq9}
\end{equation}
%
med $n$ odde, og tilsvarende for funksjonene med sinus-funksjonen,
se likning (\ref{eq4}), f{\aa}r vi 
%
\begin{equation}
\langle x^2 \rangle = \frac{1}{a}\int_{-a}^{+a} x^2 \sin^2 k_n x dx 
   = \frac{1}{3} a^2 - \frac{1}{2 k_n^2}
   = a^2 \left ( \frac{1}{3} - \frac{2}{(n\pi)^2} \right ) ,
\label{eq10}
\end{equation}
%
med $n$ like. ``Uskarpheten'' i bestemmelsen av posisjonen for 
grunntilstanden, $n = 1$ blir
%
\begin{eqnarray}
\Delta x &=& \sqrt{\langle x^2 \rangle - \langle x \rangle^2} 
                    = \sqrt{\langle x^2 \rangle}\nonumber \\
             &=& a\; \sqrt{\frac{1}{3} - \frac{2}{\pi^2}} .
\label{eq11}
\end{eqnarray}
%

Schr\"{o}dingers teori gir en oppskrift p{\aa} hvordan
man fra b{\o}lgefunksjonen $\Psi(x,t)$ ekstraherer verdier for 
dynamiske variable for et system.
Vi har hittil sett hvordan vi finner verdien for den 
dynamiske variable $x$ i lign.~(\ref{eq1}).
En annen viktig st{\o}rrelse er bevegelsesmengden
$p$. I analogi med uttrykket for $x$ kunne vi sette 
%
\begin{equation}
\langle p \rangle = \int \Psi^{\ast}(x,t) p \Psi(x,t) dx .
\label{eq12}
\end{equation}
%
Men hvordan skal vi n{\aa} regne ut dette? Klassisk vil $p$ 
v{\ae}re en funksjon av $x$, $p_{kl} = p_{kl}(x)$. Dette strider mot
Heisenbergs uskarphetsrelasjon, siden en skarp verdi av $x$ ville i s{\aa} fall
gi en skarp verdi av $p$.
La oss igjen utnytte tilfellet med en fri partikkel hvor vi 
kjenner b{\o}lgefunksjonen
%
\be
\Psi(x,t) = A\; e^{i(kx - \omega t)} ,
\ee
%
og beregne 
% 
\be
\frac{\partial}{\partial x} \Psi(x,t) = ik \Psi(x,t)
                         = \frac{i}{\hbar} p \Psi(x,t).
\ee
Her har  vi  brukt $p = k \hbar$. Dette uttrykket kan omformes til
%
\begin{equation}
p \Psi(x,t) 
  = \left ( -i\hbar \frac{\partial}{\partial x}\right ) \Psi(x,t) .
\label{eq13}
\end{equation}
%
Denne  ligningen kan vi tolke slik: Virkningen av $p$ p{\aa}
b{\o}lgefunksjonen er det samme som {\aa} anvende 
\underline{ \bf operatoren} $-i\hbar \partial / \partial x$ p{\aa}
$\Psi(x,t)$. La oss analogt sette opp f{\o}lgende uttrykk
% 
\be
\frac{\partial}{\partial t} \Psi(x,t) = -i\omega \Psi(x,t)
                         =-\frac{i}{\hbar} E \Psi(x,t),
\ee
%
som gir 
%
\begin{equation}
E \Psi(x,t) 
  = \left ( i\hbar \frac{\partial}{\partial t}\right ) \Psi(x,t).
\label{eq14}
\end{equation}
%
Dersom vi kvadrerer operatoren for bevegelsesmengden kan vi skrive
Schr\"odingers likning  p\aa\ operatorform
\be
   -\frac{\hbar^2}{2m}\frac{\partial^2}{\partial x^2}+V(x)=i\hbar \frac{\partial}{\partial t}.
\ee

 
Fra lign.~(\ref{eq12}) og (\ref{eq13}) kan vi etablere f{\o}lgende oppskrift
p{\aa} hvordan vi i kvantemekanikken finner verdier av  dynamisk variable
utfra kjennskap til systemets b{\o}lgefunksjon:
%
\begin{itemize}
%
\item[a)] Posisjon
%
\begin{equation}
\langle \OP{x} \rangle = \int \Psi^{\ast}(x,t) x \Psi(x,t) dx .
\label{eq15}
\end{equation}
%
%
\item[b)] Bevegelsesmengde
%
\begin{equation}
\langle \OP{p} \rangle = \int \Psi^{\ast}(x,t) (-i\hbar \frac{\partial}{\partial x})
                          \Psi(x,t) dx .
\label{eq16}
\end{equation}
%
\item[c)] Energi
%
\begin{equation}
\langle \OP{E} \rangle = \int \Psi^{\ast}(x,t) (i\hbar \frac{\partial}{\partial t})
                           \Psi(x,t) dx .
\label{eq17}
\end{equation}
%
Vi bruke her symbolene $\OP{x}, \; \OP{p}$ og $\OP{E}$ for � markere at det n� 
er middelverdi av en kvantemekanisk operator.
\end{itemize}
%

La oss s\aa\ vise at rekkef\o lgen av leddene i uttrykket
for den kvantemekaniske forventningsverdien er viktig.
Sp\o rsm\aa let vi stiller oss er hvorfor kan vi ikke bare
skrive
\be
\langle \OP{p} \rangle = \int_{-\infty}^{\infty} (-i\hbar \frac{\partial}{\partial x})\Psi^{\ast}(x,t)\Psi(x,t) dx ?
\ee
Dette integralet gir oss
\be
\langle \OP{p} \rangle = \Psi^{\ast}(x,t) \Psi(x,t)\left|\right._{-\infty}^{\infty},
\ee
som er null da vi har krevd at b\o lgefunksjonen~skal ha en endelig utstrekning,
ellers er vi ikke i stand til \aa\ tilordne partikkelbegrepet
til b\o lgefunksjonen Med andre ord, vi har satt $\Psi(x=\pm \infty,t)=0$,
per definisjon.
Det andre alternativet
\be
\langle \OP{p} \rangle = \int_{-\infty}^{\infty} \Psi^{\ast}(x,t)\Psi(x,t)(-i\hbar \frac{\partial}{\partial x}) dx,
\ee
er meningsl\o st. Vi sitter igjen derfor kun med en bestemt 
tilordning av leddene i integranden. 

Oppsumert kan vi si at
de klassiske dynamiske st{\o}rrelsene byttes ut med operatorer.
Fra klassisk mekanikk har vi at alle dynamiske st{\o}rrelser kan uttrykkes 
som funksjon av systemets koordinat $x$ og bevegelsesmengde $p$. Derved kan vi sette opp 
f{\o}lgende oppskrift p{\aa} en kvantemekanisk behandling:
%
\begin{itemize}
%
\item Sett opp systemets klassisk mekaniske uttrykk som funksjon av $x$ og $p$.
%
\item Finn de tilsvarende  kvantemekaniske uttrykkene ved {\aa} bytte
\begin{eqnarray}
x &\longrightarrow& \widehat{x} = x . \nonumber \\
p &\longrightarrow& \widehat{p} = -i\hbar \frac{\partial}{\partial x} .
\end{eqnarray}
%
\end{itemize}




%\section{Schr\"odingers katt paradokset og entanglement}

%\section{Tolkninger av kvantemekanikken}

\section{Oppgaver}
\subsection{Analytiske oppgaver}
\subsubsection*{Oppgave 5.1}
En partikkel med masse $m$ befinner seg i et \'{e}n--dimensjonalt harmonisk
oscillator potensial $V = \frac{1}{2} k x^{2}$.
%
\begin{itemize}
%
\item[a)] Vis at b\o lgefunksjonen $\psi_{0}(x) = \exp^{-\alpha x^{2}}$
beskriver
en egentilstand for systemet. Bestem parameteren $\alpha$ og den tilsvarende
egenverdien $E_{0}$. Bruk dette til \aa ~finne normaliseringskonstanten til
b\o lgefunksjonen.

\item[b)] N\aa r partikkelen befinner seg i tilstanden 
$\psi_{0}(x)$, beregn de
kvantemekaniske forventningsverdiene $\langle \OP{x} \rangle$ 
og $\langle \OP{x}^2 \rangle$,
der $\OP{x}$ er operatoren for partikkelens posisjon. 
Finn uskarpheten $\Delta x$ ut fra
definisjonen $\Delta x = \sqrt{\langle \OP{x}^2  \rangle 
- \langle \OP{x} \rangle^2}$.
%
\item[c)] Hva blir forventningsverdiene $\langle \OP{p} \rangle$
og $\langle \OP{p}^2 \rangle$,
der $\OP{p}$ er operatoren for bevegelsesmengden til 
partikkelen i tilstanden $\psi_{0}(x)$?
%
\item[d)] Finn herav uskarpheten $\Delta p$ i partikkelens bevegelsesmengde.
Hva blir
produktet $\Delta x \Delta p$? Sammenlign med Heisenbergs uskarphetsrelasjon.
%
\end{itemize}
%

\subsubsection*{Oppgave 5.2}
Et system best{\aa}r av en partikkel med masse m i en
\'{e}n--dimensjonal
potensialboks. Partikkelens potensielle energi $V$ er:
%
\[
V(x) = \left \{
\begin{array}{c}
     0  \\
  + \infty
\end{array}
%
\;\;\; \mbox{for} \;\;\;
%
\begin{array}{r}
 -a \leq x < a \\
 x \leq -a \;\;\mbox{og}\;\; x \geq  a
\end{array}
%
\right .
%
\]
%

\begin{itemize}
%
\item[a)] Bestem de normerte energi egenfunksjonene og
de tilh{\o}rende energi egenverdiene for systemet.
%
\item[b)] Vis at disse b\o lgefunksjonene er ortogonale,
dvs.
%
\begin{eqnarray*}
\int_{-a}^{+a} \psi_{m}^{*}(x) \psi_{n}(x) dx
= \delta_{m,n}
= \left \{
\begin{array}{rcl}
1&\;\;\; & \mbox{for} \;\; m = n\\
0&       & \mbox{ellers}\\
\end{array}
\right .
\end{eqnarray*}
%
\item[c)] Bestem middelverdien (forventningsverdien) for
bevegelsesmengden $p$ og for $p^2$ n{\aa}r systemet er i en
energi egentilstand.
%
\item[d)] Vis at  uskarphetsrelasjonen for posisjon og bevegelsesmengde
gir som konsekvens en nedre grense for energien til partikkelen.
Sammenlign dette med resultatet i pkt.~a).
%
\end{itemize}
%
La n{\aa} systemet v{\ae}re i en tilstand som ved tiden $ t = 0$
er beskrevet ved b{\o}lgefunksjonen
%
\begin{eqnarray*}
\Phi (x,0) = sin \left ( \frac{m\pi }{a} x \right )
				 cos \left ( \frac{n\pi }{a} x \right )
\end{eqnarray*}
hvor m og n er hele, positive tall.
%
\begin{itemize}
%
\item[e)] Bestem tilstandsfunksjonen $\Phi (x,t)$ ved tiden t (se bort fra
normeringen).
%
\end{itemize}
%
Systemet best{\aa}r n{\aa} av en partikkel med masse m
i en tre--dimensjonal
potensialboks. Partikkelens potensielle energi er:
%
\begin{eqnarray*}
V = &0 \hspace{1cm} &\mbox{for} \;\; -a < x < a, \;\; -2a < y < 2a
					\;\; \mbox{og} \;\; -2a  < z < 2a,\\
V = &+\infty   \;\;\;\;        &\mbox{ellers}.
\end{eqnarray*}
%
\begin{itemize}
%
\item[f)] Bestem de normerte energi egenfunksjonene og de tilh{\o}rende
energi egenverdiene for systemet.
%
\item[g)] Bestem degenerasjonsgraden for de tre laveste gruppene
av energi egentilstandene. Diskut\'{e}r pariteten av de
tilsvarende energi egenfunksjonene.
%
\end{itemize}
\subsubsection*{L\o sning}
\begin{itemize}
%
%
\item[a)] B�lgefunksjonen i omr�dene $ x < -a$ og $x > +a$
          er gitt ved 
% 
\[
\psi_0(x) = 0
\]
% 
I omr�det $ -a \leq x \leq +a$ kan 
ligningen for $\psi_1(x)$ kan skriver p{\aa} formen
%
\[
\left ( \frac{d^2}{dx^2} + k^2 \right ) \psi_1(x) = 0,
\]
%
med $k = \sqrt{2mE / \hbar^2}$.
Den generelle l{\o}sningen har formen
%
\[
\psi_1(x) = A e^{ikx} + B e^{-ikx}.
\]
%
Betingelsene om at $\psi(x)$ er kontinuerlig overalt gir 
% 
\[ 
\lim_{x\to -a} \psi_1(x) = 0 \quad \mbox{og}
 \lim_{x\to +a} \psi_1(x) = 0
\]
%
Siden $V(x) = \infty$ for $ |x| > a$ har vi ingen
kontinuitetskrav til  $\frac{d\psi(x)}{dx}$ i punktene $x = \pm a$.
%

Dette gir
% 
\begin{eqnarray*}
x = -a: \quad A e^{-ka} + B e^{ka} &=& 0\\
x = +a: \quad A e^{ka} + B e^{-ka} &=& 0
\end{eqnarray*}
% 
eller 
% 
\begin{eqnarray*}
(A+B) (e^{-ka} + B e^{ka}) &=& (A+B)\cos(ka) = 0\\
(A-B) (e^{-ka} + B e^{ka}) &=& (A+B)\sin(ka) = 0
\end{eqnarray*}
%
med l�sninger
%
\begin{enumerate}
% 
\item $A = B\; \longrightarrow \; \cos(ka) = 0$ med l�sning
\[
  k_na = n \frac{\pi}{2} \quad n = 1,3,5, \dots
\]  
% 
med energi egenverdiene
% 
\[
E_n = \frac{\hbar^2 \pi^2}{8 m a^2} n^2
\]
%
og b�lgefunksjonen inkludert tidsfaktoren f�r formen
% 
\[
\Psi(x,t) = A_n\cos(k_n x) e^{(-i/\hbar) E_n t}
\]
 % 
\item $A = -B\; \longrightarrow \; \sin(ka) = 0$ med l�sning
\[
  k_na = n \frac{\pi}{2} \quad n = 2, 4,6, \dots
\]  
% 
med energi egenverdiene
% 
\[
E_n = \frac{\hbar^2 \pi^2}{8 m a^2} n^2
\]
%
og b�lgefunksjonen inkludert tidsfaktoren f�r formen
% 
\[
\Psi(x,t) = A_n\sin(k_n x) e^{(-i/\hbar) E_n t}
\]
% 
\end{enumerate}
%
\item[b)] B�lgefunksjonene er ortogonale
%
\[
\int_{-\infty}^{+\infty} \Psi_n^{*}(x,t) \Psi_m(x,t) dx 
 = \int_{-a}^{+a} \Psi_n^{*}(x,t) \Psi_m(x,t) dx
      = \left \{  \begin{array}{c}
                   1 \quad n = m\\
                   0 \quad n \ne m
                 \end{array} \right .  
          = \delta_{n,m} 
\]
%
Med utgangspunkt i egenfunksjonene funnet i a) f�r 
vi tre typer integrander 
%
\begin{eqnarray*}
\sin(k_m x) \sin(k_n x) &=& \frac{1}{2} 
            \left ( \cos((k_m -k_n) x) - \cos((k_m+k_n) x)\right )\\
\cos(k_m x) \cos(k_n x) &=& \frac{1}{2} 
            \left ( \cos((k_m -k_n) x) + \cos((k_m+k_n) x)\right )\\
\sin(k_m x) \cos(k_n x) &=& \frac{1}{2}
             \left ( \sin((k_m + k_n) x) + \sin((k_m - k_n) x) \right )
\end{eqnarray*}
%
Argumentene p� h�yre side i uttrykkene ovenfor er av typen
$(k_m \pm k_n) = (m \pm n)\frac{\pi}{2 a} = 2\nu\frac{\pi}{2 a}$
hvor $\nu$ er et helt tall. Dette gir 
%
\begin{eqnarray*}
\int_{-a}^{+a} \cos(\nu \frac{\pi x}{a}) dx &=& 0\\
\int_{-a}^{+a} \sin(\nu \frac{\pi x}{a}) dx &=& 0
\end{eqnarray*}
%
som viser at alle kombinasjoner av egenfunksjoner er ortogonale.
%

\item[c)] 
% 
\[
<p> = \int_{-a}^{+a} \Psi_n^{*}(x,t)\OP{p} \Psi_n(x,t) dx 
    = \int_{-a}^{+a} \Psi_n^{*}(x,t)(-i\hbar \frac{d}{dx} ) \Psi_n(x,t) dx  
    = 0
\]
%
siden integranden har paritet minus.
%
\begin{eqnarray*}
<p^2> &=&  \int_{-a}^{+a} \Psi_n^{*}(x,t)\OP{p}^2 \Psi_n(x,t) dx
 = 2m  \int_{-a}^{+a} \Psi_n^{*}(x,t)
      \left ( \frac{-\hbar^2}{2m} \frac{d^2}{dx^2} \right )
        \Psi_n(x,t) dx\\
  &=&  2m  \int_{-a}^{+a} \Psi_n^{*}(x,t)E_n \Psi_n(x,t) dx 
   = 2m E_n  = \frac{\hbar^2\pi^2}{4 a^2} n^2
\end{eqnarray*}
%
\item[d)] Nullpunktsenergi: Et kvantemekanisk system har  $E_{min} > 0$.
En estimering av nullpunktsenergien fra Heisenbergs uskarphetsrelasjon 
$\Delta x \Delta p > h$. Partikkelens bevegelse er begrenset til dimensjonen
p� potensialet. $ \Delta x = a$. Bevegelsesmengden har konstant tallverdi, men kan 
skifte retning
% 
\[ 
   \Delta p = 2 p_{min}  = 2 \sqrt{2 m  E_{min}}
\]
% 
Dette gir   
% 
\[
E_{min} > \frac{h^2}{32 m a^2} = \frac{(\hbar \pi)^2}{8 m a^2} = E_{n = 1}
\]
%
i overensstemmelse med den beregnede grunntilstands energien i a),
% 
\item[e)] Den generelle tidsavhengige l�sning av Schr\"{o}dinger ligningen 
i denne oppgaven kan skrives p� formen 
% 
\[
\Psi(x,t) = \sum_{\nu} \left ( A_{\nu} \cos(k_{\nu}x) e^{-\frac{i}{\hbar} E_{\nu} t}
               + B_{\nu} \sin(k_{\nu}x) e^{-\frac{i}{\hbar} E_{\nu} t} \right )
\]
%
For en fullstendig bestemmelse av  $\Psi(x,t)$  kreves en initialbetingelse.
I v�rt tilfelle er den gitt ved 
%
\[
\Psi(x.t= 0) = \sin \left ( \frac{m\pi}{a}x \right ) \cos \left (
              \frac{n\pi}{a}x \right ))  
 \]
%
Konstantene $A_{\nu}$ og $B_{\nu}$ kan bestemmes p� f�lgende m�te
% 
\[
\Psi(x.t= 0) =  \sin \left ( \frac{m\pi}{a}x \right )
                  \cos \left ( \frac{n\pi}{a}x \right )
             = \frac{1}{2} \sin \left ( \frac{(m + n)\pi}{a}x \right )
           + \frac{1}{2} \sin \left ( \frac{(m - n)\pi}{a}x \right )
\]
% 
som er en sum av to egenfunksjonene. Dette bestemmer konstantene til 
% 
\[
B_{m+n} = \frac{1}{2}, \quad    B_{m-n} = \frac{1}{2} 
\]
% 
og de resterende koeffisientene er null. Den tidsavhengige l�sningen blir da 
% 
\[
\Psi(x.t) = \frac{1}{2} \sin \left ( \frac{(m + n)\pi}{a}x \right )
                       e^{-\frac{i}{\hbar} E_{m+n} t} 
             + \frac{1}{2} \sin \left ( \frac{(m - n)\pi}{a}x  \right )
                   e^{-\frac{i}{\hbar} E_{m-n} t}
\]
%

\item[f)] Egenverdiligningen i det tre--dimensjonale tilfelle blir 
% 
\[ 
 - \frac{\hbar^2}{2m}
       \left ( \frac{d^2}{dx^2} 
              + \frac{d^2}{dy^2}
              + \frac{d^2}{dz^2} \right ) \psi(x,y,z) = E
       \psi(x,y,z)
\]
% 
Ligningen er separabel $ \psi(x,y,z) = \psi_x(x)\psi_y(y)\psi_z(z)$ 
og $ E = E_x + E_y + E_z$
som gir f�lgende tre energi egenverdi ligninger
% 
\begin{eqnarray*}
 -\frac{\hbar^2}{2m} \frac{d^2}{dx^2}\psi_x(x) &=& E_x \psi_x(x) \quad
                                           -a \le x \le +a\\
 -\frac{\hbar^2}{2m} \frac{d^2}{dy^2}\psi_y(y) &=& E_y \psi_y(y) \quad
                                           -2a \le x \le +2a\\
 -\frac{\hbar^2}{2m} \frac{d^2}{dz^2}\psi_z(z) &=& E_z \psi_z(z) \quad
                                           -2a \le x \le +2a
\end{eqnarray*}
%
Disse ligninger gir samme type l�sninger som i det
\`{e}n--dimensjonale tilfelle.
% 
\begin{eqnarray*}
E_x = \frac{(\hbar \pi)^2}{8ma^2}\quad 
    \psi_x(x) = \left \{ \begin{array}{c}
                                   \sqrt{\frac{1}{a}} \sin \left (
                                    \frac{\pi}{2a} n_x x\right ) 
                                    \quad n_x\; \mbox{even}\\ 
                                   \sqrt{\frac{1}{a}} \cos \left (
                                    \frac{\pi}{2a} n_x x \right ) 
                                    \quad n_x\; \mbox{odd}  
                        \end{array} \right . \\ 
E_y = \frac{(\hbar \pi)^2}{32ma^2}\quad 
    \psi_y(y) = \left \{ \begin{array}{c}
                                   \sqrt{\frac{1}{2a}} \sin \left (
                                    \frac{\pi}{4a} n_y y \right ) 
                                    \quad n_y\; \mbox{even}\\ 
                                   \sqrt{\frac{1}{2a}} \cos \left (
                                    \frac{\pi}{4a} n_y y \right ) 
                                    \quad n_y\; \mbox{odd}  
                        \end{array} \right .  \\
E_z = \frac{(\hbar \pi)^2}{32ma^2}\quad 
    \psi_z(z) = \left \{ \begin{array}{c}
                                   \sqrt{\frac{1}{2a}} \sin \left (
                                    \frac{\pi}{4a} n_z z \right ) 
                                    \quad n_z\; \mbox{even}\\ 
                                   \sqrt{\frac{1}{2a}} \cos \left (
                                    \frac{\pi}{4a} n_z z \right ) 
                                    \quad n_z \; \mbox{odd}  
                        \end{array} \right .  
\end{eqnarray*}
%
Den totale energien blir 
% 
\[
   E = E_x + E_y + E_z = \frac{(\hbar \pi)^2}{32 m a^2}\left ( 4 n_x^2
                          + n_y^2 + n_z^2 \right ) 
\]

\item[g)] Vi setter $ E_0 =\frac{(\hbar \pi)^2}{32 m a^2}$. Energi degenerasjon
          og paritet for de tre laveste gruppene blir da  
% 
\begin{center}

\begin{tabular}{|l|ccc|c| c|c|} \hline  
Kvantetall:& $n_x$ & $n_y$ & $n_z$ & Energi: $E/E_0$ & Degenerasjon & Paritet\\ \hline
 gruppe 1  &   1   &   1   &   1   &          6      &    1         & + \\ \hline
 gruppe 2  &   1   &   2   &   1   &          9      &    2         & - \\
           &   1   &   1   &   2   &          9      &    2         & - \\ \hline
 gruppe 3  &   1   &   2   &   2   &         12      &    1         & + \\ \hline
\end{tabular}
\end{center}

\end{itemize}

\subsubsection*{Oppgave 5.3}
En kvantemekanisk kommutator er definert ved
%
\[
\left [ \OP{A}, \OP{B} \right ]
\equiv  \OP{A} \OP{B} - \OP{B} \OP{A}.
\]
%
Bruk dette til {\aa} vise at
%
\[
\left [ \OP{A} \OP{B}, \OP{C} \right ]
=  \OP{A} \left [ \OP{B}, \OP{C} \right ]
  + \left [ \OP{A}, \OP{C} \right ] \OP{B}.
\]
%
Den kvantemekaniske operatoren for en partikkels
banespinn er definert ved
%
\[
\OP{\vec{L}} = \vec{r} \times \OP{\vec{p}}
\]
%
Vis at
%
\[
\OP{L}_x = y \OP{p}_z - z \OP{p}_y, \;\;\;
\OP{L}_y = z \OP{p}_x - x \OP{p}_z, \;\;\;
\OP{L}_z = x \OP{p}_y - y \OP{p}_x,
\]
%
og utled f{\o}lgende kommutator relasjoner
%
\[
\left [ \OP{L}_x, \OP{L}_y \right ] = i \hbar \OP{L}_z, \;\;\;
\left [ \OP{L}_y, \OP{L}_z \right ] = i \hbar \OP{L}_x, \;\;\;
\left [ \OP{L}_z, \OP{L}_x \right ] = i \hbar \OP{L}_y.
\]
%

\subsubsection*{Oppgave 5.4}
For l{\o}sningen av det kvantemekaniske problemet med en
partikkel i et oscillator potensialet 
\mbox{$V = \frac{1}{2} m \omega^2 x^2$}
er det meget
nyttig \aa ~innf\o re operatorene
\begin{eqnarray*}
\OP{a} &=& \sqrt{ m\omega / 2\hbar }\; \OP{x} +
i \sqrt{ 1 / 2m \hbar \omega }\; \OP{p}\\
\OP{a}^{\dagger} &=&  \sqrt{ m \omega / 2\hbar }\; \OP{x} -
i \sqrt{ 1 / 2m\hbar \omega }\; \OP{p}.
\end{eqnarray*}
%
\begin{itemize}
%
\item[a)] Vis at kommutatoren
$\left [ \OP{a},\OP{a}^{\dagger}\right ] = 1$.
%
\item[b)] Vis at Hamilton operatoren for systemet n\aa ~kan skrives som
\begin{eqnarray*}
\OP{H} = \hbar \omega \left ( \OP{a}^{\dagger} \OP{a}
					  + \frac{1}{2} \right ).
\end{eqnarray*}
%
\item[c)] Finn kommutatorene $\left [ \OP{H}, \OP{a} \right ]$
og $\left [ \OP{H}, \OP{a}^{\dagger} \right ]$.
%
\item[d)] Vis at b\o lgefunksjonen for grunntilstanden tilfredsstiller
relasjonen $\OP{a} \psi_{0}(x) = 0$,
og at $\OP{a}^{\dagger} \psi_{0}(x)$ er den f\o rste eksiterte tilstanden
$\psi_{1}(x)$ i oscillatoren.
%
\item[e)] Vis ved hjelp av resultatet i c) at
$\OP{a}^{\dagger} \OP{a}^{\dagger} \psi_{0}(x)$ er en egentilstand
for Hamilton operatoren med egenverdi
$E_{2} = \frac{5}{2} \hbar \omega$.
Hvordan kan n'te
eksiterte tilstand i oscillatoren utledes fra grunntilstanden?

\item[f)] Verifis\'{e}r dette ved \aa ~generere egenfunksjonene
for de fire laveste
energitilstandene ved hjelp av operatoren
$\OP{a}^{\dagger}$
og vis at de er egentilstander for
$\OP{H}$. Bruk som utgangspunkt grunntilstandsfunksjonen
$\psi_{0}$ fra oppgave~4.3.
%
\end{itemize}
%

\subsubsection*{Oppgave 5.5}
Vi skal igjen studere systemet som er beskrevet
i oppgave~5.2 og antar at ved tiden $ t = 0$ er
det i tilstanden
%
\begin{eqnarray*}
\Psi(x, 0) = \sqrt{\frac{1}{2}} \left ( \psi_1(x) + \psi_2(x) \right )
\end{eqnarray*}
%
\begin{itemize}
%
\item[a)] Vis at denne funksjonen er normert til 1.
Bestem $\Psi(x, t)$ og beregn forventningsverdien
$\langle \OP{x} \rangle$ som funksjon av tiden t.
%
\item[b)] Finn ogs\aa ~forventningsverdien $\langle \OP{p} \rangle$
og vis at vi har relasjonen
\begin{eqnarray*}
m \frac{d}{dt} \langle \OP{x} \rangle = \langle \OP{p} \rangle.
\end{eqnarray*}
%
\item[c)] Hvor generelt tror du dette resultatet er?
(Hjelp: Bruk Schr\"{o}dingerligningen og dens 
kompleks konjugerte til \aa ~vise at
for generelle operatorer
$\OP{Q}$ er
%
\begin{eqnarray*}
\frac{d}{dt} \langle \OP{Q} \rangle = - \frac{i}{\hbar }
\langle \left [ \OP{Q}, \OP{H} \right ] \rangle,
\end{eqnarray*}
der $\OP{H}$ er Hamilton operatoren. Vi  antar at
$\frac{\partial }{\partial t} \OP{Q} = 0$.
Spesialiser s\aa ~til
$\OP{Q} = \OP{x}$ og \mbox{$\OP{H} =
\frac{1}{2m} \OP{p}^2 + V\left (\OP{x} \right )$.}
%
\end{itemize}
%

\subsubsection*{Oppgave 5.6}
Vi skal i denne oppgaven behandle en \'{e}n--dimensjonal, harmonisk
oscillator med masse m og stivhet k.

%
\begin{itemize}
%
\item[a)] Skriv ned uttrykket for oscillatorens potensielle energi V(x)
og dens klassiske vinkelfrekvens $\omega$.

\item[b)] Skriv ned den tidsavhengige og den tidsuavhengige
Schr\"{o}dingerligningen for oscillatoren. Gj\o r kort rede for
hvordan, og under hvilke betingelser vi f\aa r den tidsuavhengige
Schr\"{o}dingerligningen fra den tidsavhengige.
%
\end{itemize}
%
Gitt to normerte tilstandsfunksjoner $\psi(x)$  og $\phi(x)$ for oscillatoren
ved tiden t = 0,
\begin{eqnarray*}
\psi(x) &=& \left( \frac{\alpha }{\sqrt{\pi }} \right)^{\frac{1}{2}} \exp\left( -
\frac{1}{2}\alpha^{2} x^{2} \right),\\
\phi(x) &=& \left( \frac{2\alpha^{3}}{\sqrt{\pi }} \right)^{\frac{1}{2}} x
\exp \left( -\frac{1}{2}\alpha^{2} x^{2} \right),
\end{eqnarray*}
der $\alpha \equiv \left( mk / \hbar^{2} \right)^{\frac{1}{4}}$.

%
\begin{itemize}
%
\item[c]) Vis at $\psi(x)$ og $\phi(x)$ er energi egentilstander for
oscillatoren. Finn de tilh\o rende energi egenverdiene. Skriv ned uttrykket
for de to energi egentilstandene ved tiden t.
%
\end{itemize}
%
For oscillatorens energi egentilstander $\phi_{n}(x)$ gjelder generelt at
\begin{eqnarray*}
\phi_{n}(- x) = (- 1)^{n} \phi_{n}(x).
\end{eqnarray*}
%
\begin{itemize}
%
\item[d)] Bruk dette til \aa ~vise at for oscillatoren er middelverdien
$\langle \OP{x} \rangle$ av $\OP{x}$ og middelverdien 
$\langle \OP{p}\rangle_{n}$ av $\OP{p}$
i tilstanden $\phi_{n}(x)$ lik null.
%
\end{itemize}
%
Uskarpheten av en operatoren $\OP{A}$ i tilstanden $\phi_{n}(x)$ er definert 
ved $\Delta A \equiv 
\sqrt{\langle \OP{A}^{2}\rangle_{n} - \langle\OP{A}\rangle_{n}^{2}}$.
For harmonisk oscillator tilstander gjelder
\[
\Delta x \cdot \Delta p_{x} = \left( n
+ \frac{1}{2} \right) \hbar.
\]
%
\begin{itemize}
%
\item[e)] Vis at dette stemmer for de to tilstandene $ \psi(x)$
og $\phi(x)$.
\end{itemize}
%
\subsubsection*{L\o sning}
\begin{itemize}
%
\item[a)]
     Den potensielle energien for en harmonisk oscillator er gitt ved 
$V(x) = \frac{1}{2}kx^2$
og med klassisk vinkelhastighet
$\omega = \sqrt{\frac{k}{m}}$
% 
\item[b)]
   Den tidsavhengige Schr\"{o}dingerligning er 
%
\begin{equation}
\left ( -\frac{\hbar^2}{2m} \frac{\partial^2}{\partial x^2} 
       + \frac{1}{2} k x^2 \right ) \Psi(x,t) = i \hbar \frac{\partial}{\partial t} \Psi(x,t)
\end{equation}
%
og den tidsuavhengige Schr\"{o}dingerligning er 
%
\begin{equation}
\left ( -\frac{\hbar^2}{2m} \frac{d^2}{dx^2} 
       + \frac{1}{2} k x^2 \right ) \psi(x) =  E \psi(x)
\end{equation}
%
Hvis den potensielle energien er eksplisitt uavhengig av tiden,
$V(x,t) = V(x)$ kan $\Psi(x,t)$ i lign(1) skrives som et produkt 
$\Psi(x,t) = \psi(x) \exp (-(i/\hbar)E t)$ og $\psi(x)$ er l�sning av
lign(2).
% 
\item[c)]
F�rst beregner vi for $\psi_{\mu}(x)$ 
% 
\[
\frac{d}{dx} \psi_{\mu}(x) = (-\alpha^2x) \psi_{\mu}(x),\;\; \mbox{og}\;\;
\frac{d^2}{dx^2} \psi_{\mu}(x) = (-\alpha^2 + \alpha^4 x^2) \psi_{\mu}(x)
%
\]
%
Innsatt gir dette
%
\[
 \left ( -\frac{\hbar^2}{2m} ( -\alpha^2 + \alpha^4 x^2)
                  + \frac{1}{2}kx^2 \right )  \psi_{\mu}(x) =  E \psi_{\mu}(x)
\]
%
For at venstre side skal v�re lik h�yre side for alle $x$ f�r vi
betingelsene
% 
\[
\frac{\hbar^2 \alpha^4}{2m} = \frac{1}{2} k 
        \longrightarrow
 \alpha = \left ( \frac{km}{\hbar^2} \right )^{1/4}
%
\;\; \mbox{og}\;\;
E = \frac{\hbar^2 \alpha^2}{2m} = \frac{\hbar^2}{2m}
\sqrt{\frac{km}{\hbar^2}} = \frac{1}{2} \hbar \omega
% 
\]
% 
Deretter  beregner vi for $\psi_{\nu}(x)$ 
% 
\[
\frac{d}{dx} \psi_{\nu}(x) = (\frac{1}{x} - \alpha^2x) \psi_{\nu}(x)
\;\;\mbox{og}\;\;
\frac{d^2}{dx^2} \psi_{\nu}(x) = (-3\alpha^2 + \alpha^4 x^2) \psi_{\nu}(x)
%
\]
%
som gir
%
\[
 \left (-\frac{\hbar^2}{2m} ( -3\alpha^2 + \alpha^4 x^2)
                       + \frac{1}{2}k x^2 \right )  \psi_{\nu}(x) =  E \psi_{\nu}(x)
\]
%
og betingelsene blir 
% 
\[
\frac{\hbar^2 \alpha^4}{2m} = \frac{1}{2} k 
         \longrightarrow
 \alpha = \left ( \frac{km}{\hbar^2} \right )^{1/4}
\;\; \mbox{og}\;\;
%
E = \frac{3\hbar^2 \alpha^2}{2m} = \frac{3}{2} \hbar \omega
% 
\]
% 

\item[d)]
% 
Vi har f�lgende uttrykk
%
\[
\langle x\rangle_{n} = \int_{-\infty}^{\infty} \phi_{n}^{\ast}(x)_n x \phi_{n}(x)_n dx
\]
%
Integranden har  paritetssymmetri
% 
\[
 \left \{ \phi_{n}^\ast(-x) (-x) \phi_{n}(-x) \right \} 
           = - \left \{ \phi_{n}^\ast(x) (x) \phi_{n}(x) \right \}
\]
%
Siden vi integrerer symmetrisk om origo blir $\langle x\rangle_{n} = 0$.

Uttrykket for $\langle p\rangle_{n} $ blir 
%
\[
\langle p\rangle_{n} = \int_{-\infty}^{\infty} \phi_{n}^\ast(x)_n 
                  (-i\hbar\frac{d}{dx}) \phi_{n}(x)_n dx
\]
%
og paritetssymmetrien for integranden blir 
%
\[
 \left \{ \phi_{n}^\ast(-x) (-i\hbar\frac{d}{d(-x)}) \phi_{n}(-x) \right \} 
     = -\left \{ \phi_{n}^\ast(x) (-i\hbar\frac{d}{dx}(x) \phi_{n}(x)) \right \}
\]
%
og  $\langle p\rangle_{n} = 0$.
%
\item[e)] For � finne $\Delta x \cdot \Delta p$ m� vi f�rst 
beregne $\langle x^2\rangle $ og $\langle p^2\rangle $.
%
\begin{eqnarray*}
\langle x^2\rangle_{\mu} &=& \int_{-\infty}^{\infty} \psi_{\mu}^\ast(x) x^2 \psi_{\mu}(x) dx
                    = \frac{\alpha}{\sqrt{\pi}}
             \int_{-\infty}^{\infty}x^2 \exp (-\alpha^2 x^2) dx\\ 
              &=& \frac{\alpha}{\sqrt{\pi}} 2 \frac{1}{2 \alpha^3}
                   \Gamma(3/2) = \frac{1}{2 \alpha^2}\\
%
\langle x^2\rangle_{\nu} &=& \int_{-\infty}^{\infty} \phi_{\nu}^\ast(x) x^2 \phi_{\nu}(x) dx
                 = \frac{2 \alpha^3}{\sqrt{\pi}}
             \int_{-\infty}^{\infty}x^4 \exp (-\alpha^2 x^2) dx\\ 
           &=& \frac{2\alpha^3}{\sqrt{\pi}} 2 \frac{1}{2 \alpha^5}
           \Gamma(5/2) = \frac{3}{2 \alpha^2}\\
%
\langle p^2\rangle_{\mu} &=& \int_{-\infty}^{\infty} \psi_{\mu}^\ast(x)
                      (-\hbar^2 \frac{d^2}{dx^2}) \psi_{\mu}(x) dx
             = -\hbar^2 \int_{-\infty}^{\infty} \psi_{\mu}^\ast(x)
                      (-\alpha^2 + \alpha^4 x^2) \psi_{\mu}(x) dx\\
           &=& \hbar^2 \alpha^2 
              - \hbar^2 \alpha^4  \frac{\alpha}{\sqrt{\pi}}
             \int_{-\infty}^{\infty}x^2 \exp (-\alpha^2 x^2) dx\\ 
           &=& \hbar^2 \alpha^2 
              - \hbar^2 \alpha^4  \frac{\alpha}{\sqrt{\pi}}
            2 \frac{1}{2 \alpha^3} \Gamma(3/2)
           = \frac{1}{2} \hbar^2 \alpha^2\\
\langle p^2\rangle_{\nu} &= & \int_{-\infty}^{\infty} \phi_{\nu}^\ast(x)
                      (-\hbar^2 \frac{d^2}{dx^2}) \phi_{\nu}(x) dx\\
             &= & -\hbar^2\int_{-\infty}^{\infty} \phi_{\nu}^\ast(x)
                      (-3\alpha^2 + \alpha^4 x^2) \phi_{\nu}(x) dx\\
           &=& 3\hbar^2 \alpha^2 
              - \hbar^2 \alpha^4  \frac{2\alpha^3}{\sqrt{\pi}}
             \int_{-\infty}^{\infty}x^4 \exp (-\alpha^2 x^2) dx\\ 
           &=& 3\hbar^2 \alpha^2 
              -  \hbar^2 \alpha^4  \frac{2\alpha^3}{\sqrt{\pi}}
            2 \frac{1}{2 \alpha^5} \Gamma(5/2)
           = \frac{3}{2} \hbar^2 \alpha^2
%
\end{eqnarray*}
%
som gir 
% 
\begin{eqnarray*}
\Delta x_{\mu} \Delta p_{\mu} = \sqrt{\langle x^2\rangle_{\mu} \langle p^2\rangle_{\mu}} 
      = \sqrt{\frac{1}{2 \alpha^2} \frac{1}{2} \hbar^2 \alpha^2}
      = \frac{1}{2} \hbar\\
\Delta x_{\nu} \Delta p_{\nu} = \sqrt{\langle x^2\rangle_{\nu} \langle p^2\rangle_{\nu}} 
      = \sqrt{\frac{3}{2 \alpha^2} \frac{3}{2} \hbar^2 \alpha^2}
      = \frac{3}{2} \hbar\\
%
\end{eqnarray*}

\end{itemize}

\subsubsection*{Oppgave 5.7, Eksamen V-1993}
Vi skal i denne oppgaven studere det kvantemekaniske problem
med en partikkel som beveger seg langs x--aksen
med en potensiell energi $V(x)$.
Partikkelen har masse $m$.
%
\begin{itemize}
%
\item[a)] Sett opp den tidsavhengige Schr\"{o}dingerligningen
for systemet som bestemmer b{\o}lgefunk- sjonen $\Psi(x,t)$.
%
\item[b)] Vis at $\Psi(x,t)$ kan skrives p{\aa}
formen $ \Psi(x,t) = \psi(x) \phi(t)$.
Finn $\phi(t)$ og ligningen som bestemmer $\psi(x)$.
%
\end{itemize}
%
Vi velger n{\aa} den potensielle energi som
%
\[
V(x) = \left \{
\begin{array}{c}
    +\infty  \\
       0\\
  + \infty
\end{array}
%
\;\;\; \mbox{for} \;\;\;
%
\begin{array}{r}
 x < 0 \\
 0 \leq x < a \\
 x \geq  a
\end{array}
%
\right .
%
\]
%
\begin{itemize}
%
\item[c)] Finn energi egenverdiene og de normerte
energi egenfunksjonene $\psi(x)$.
%
%\item[d)] Hva er den minste energien systemet kan ha
%if{\o}lge klassisk mekanikk? Forklar hvorfor denne energien
%ikke er mulig fra et kvantemekanisk synspunkt.
%
\item[d)] Bestem middelverdien (forventningsverdien)
for $\OP{x}$  og $\OP{x}^2$ samt middelverdien for
bevegelsesmengden $\OP{p}$ og $\OP{p}^2$.
%
\item[e)] Formul\'{e}r Heisenbergs uskarphetsrelasjon
og regn den ut for grunntilstanden i systemet. Vis at dette
eksemplet er i overensstemmelse med det generelle prinsipp.
%
\end{itemize}
%
Ved tiden $t = 0$ er systemet i en tilstand beskrevet ved
b{\o}lgefunksjonen
\[
\Psi(x,0) = \frac{1}{\sqrt{2}}\left ( \psi_1(x) - \psi_2(x) \right ).
\]
hvor $\psi_1(x)$ og $\psi_2(x)$ er de to laveste normerte
egentilstandene for systemet.
%
\begin{itemize}
%
\item[f)] Bestem systemets tilstand $\Psi(x,t)$ og beregn
forventningsverdiene av $\OP{x}$ i denne tilstanden.
%
\end{itemize}
%
\subsubsection*{L\o sning}
%
\begin{itemize}
%
\item[c)] Ligningen for $\psi(x)$ kan skriver p{\aa} formen
%
\[
\left ( \frac{d^2}{dx^2} + k^2 \right ) \psi(x) = 0,
\]
%
for $0 \leq x \leq a$ og med $k = \sqrt{2mE / \hbar^2}$.
L{\o}sningen har formen
%
\[
\psi(x) = A \exp (ikx) + B \exp (-ikx).
\]
%
Randbetingelsen for $x = 0$ gir $ A + B = 0$, eller $ A = -B$,
og
%
\[
\psi(x) = A \left ( \exp (ikx) - \exp (-ikx) \right ) = A^{'} \sin (k x).
\]
%
Randbetingelsen for $x = a$ gir $\psi(a) = 0 \;\longrightarrow \;
\sin (ka) = 0$ med resultat
$k_n a = n \pi$, $ n = 1, 2, 3, \ldots$.
Kvantebetingelsen for energi egenverdiene blir
%
\[
E_n = \frac{\hbar^2 \pi^2}{2 a^2 m} n^2\;\;\;\;\; n = 1, 2, 3, \ldots.
\]
%
Normering av energi egenfunksjonene
%
\[
\left ( A^{'} \right )^2 \int_{0}^{a} \sin^2 (k_n x) dx
= \frac{a}{2}\left ( A^{'} \right )^2.
\]
%
Dette gir $A^{'} = \sqrt{2 / a}$, og de normerte egenfunksjonene blir
%
\[
\psi_n(x) =  \sqrt{\frac{2}{a}} \sin (k_n x).
\]
%
\item[d)] Middelverdien av $x$:
%
\[
 <x> = \int_{-\infty}^{\infty} \psi(x)^{*} x \psi(x) dx = \frac{a}{2}
\]
%
Middelverdien av $x^2$:
%
\[
 <x^2> = \int_{-\infty}^{\infty} \psi(x)^{*} x^2 \psi(x) dx 
     =  a^2 \left ( \frac{1}{3} - \frac{1}{2 \pi^2 n^2} \right )
\]
%
Middelverdien av $p$ er
%
\[
 <p> = \int_{-\infty}^{\infty} \psi(x)^{*} 
           \left (-i\hbar \frac{d}{dx} \right )
           \psi(x) dx = 0
\]
%
og middelverdien av $p^2$
\[
 <p^2> = \int_{-\infty}^{\infty} \psi(x)^{*} 
           \left (-i\hbar \frac{d}{dx} \right )^2
           \psi(x) dx =  = \hbar^2 k_n^2
\]
%

\item[e)] Heisenbergs uskarphetsrelasjon er
$\Delta x \Delta p \geq (1/ 2 ) \hbar$,
hvor $\Delta x = \sqrt{ <x^2> - (<x>)^2}$ og
$\Delta p = \sqrt{ <p^2> - (<p>)^2}$. Fra d) f{\aa}r vi for
grunntilstanden $(n = 1)$
%
\[
\Delta x = a \sqrt{\frac{1}{12} - \frac{1}{2 \pi^2}},
%
\;\;\;\Delta p = \frac{\hbar \pi}{a},
\;\; og\;\;
\Delta x \Delta p
= \hbar \sqrt{\frac{\pi^{2}}{12} - \frac{1}{2}}
\approx 0.57 \hbar > \frac{\hbar}{2}
\]
%
\item[f)] Systemets tilstand ved tiden $t$ er
\begin{eqnarray*}
\Psi(x, t) &=& \sqrt{\frac{1}{a}}
\left ( \sin (k_{1} x) \exp (-\frac{i}{\hbar} E_{1} t)
 - \sin (k_{2} x) \exp (-\frac{i}{\hbar} E_{2} t) \right )\\
 &=&\sqrt{\frac{1}{a}}
 \left ( \sin (\frac{\pi}{a} x) \exp (-\frac{i}{\hbar} E_{1} t)
 - \sin (\frac{2 \pi}{a}x) \exp (-\frac{i}{\hbar} 4 E_{1} t) \right ).
\end{eqnarray*}
%
Middelverdien ax $x$ er gitt ved
%
\[
<x> = \frac{a}{2} + \frac{16 a}{9 \pi^{2}} \cos (\frac{3 E_1}{\hbar} t ).
\]
%
\end{itemize}

\subsubsection*{Oppgave 5.8, Eksamen V-1996}
Vi skal i denne oppgaven studere det kvantemekaniske problem med 
en partikkel i et harmonisk oscillator potensial. 
Partikkelen har masse $m$ og den  kvantemekaniske hamilton 
operatoren kan skrives p� formen
% 
\[
\OP{H} = -\frac{\hbar^2}{2 m} \frac{d^2}{dx^2} 
        + \frac{1}{2} m \omega^2 x^2.
\]
%
To av egentilstandene er gitt ved 
%
\[
\psi_k(x) = N \exp (- \alpha x^2), \;\;\;\;
\psi_l(x) = 2 \sqrt{\alpha} x \psi_k(x)
\]
%
\begin{itemize}
%
\item[a)] Vis at $\psi_k(x)$ og  $\psi_l(x)$ er egenfunksjoner
av $\OP{H}$ og bestem $\alpha$ uttrykt ved oscillator parameteren 
$\omega$.
Finn de tilh�rende energi egenverdier og vis at de 
f�lger formelen $E_n = (n + 1/2)\hbar\omega$.
Bestem hvilke $n$--verdier som svarer til  $\psi_k(x)$ og  $\psi_l(x)$.
Beregn normeringskonstanten $N$.
%
\end{itemize}
%
For egentilstander til den harmoniske oscillator gjelder generelt
%
\[
\phi_n(-x) = (-1)^{n} \phi_n(x).
\]
%
\begin{itemize}
%
\item[b)] Bruk relasjonen til � vise at middelverdien av 
posisjonen $\langle \OP{x} \rangle$ og bevegelsesmengden 
$\langle \OP{p}_x \rangle$ begge er null.
%
\end{itemize}
%
I det f�lgende skal vi studere hvordan systemet utvikler seg som 
funksjon av tiden $t$. 
%
\begin{itemize}
%
\item[c)] Systemets tilstand ved $t = 0$
er gitt ved b�lgefunksjonen  $\psi_k(x, t = 0) = \psi_k(x)$. 
Forklar hvordan vi g�r frem for � bestemme  
funksjonen $\psi_k(x, t)$ for systemet og finn denne funksjonen.
%
\end{itemize}
%
Vi antar n� at systemet er i tilstanden $\psi_k(x, t)$ for $t \leq 0$.
Ved tiden $t = 0$ f�r vi en momentan forandring av systemets potensielle 
energi $ \frac{1}{2} m \omega^2 x^2 
    \longrightarrow  \frac{1}{2} m (2\omega)^2 x^2$ slik at for $t > 0$
vil systemet v�re beskrevet  ved en hamilton operator av formen 
%
\[
\OP{H}_{ny} = -\frac{\hbar^2}{2 m} \frac{d^2}{dx^2} 
        + \frac{1}{2} m (2\omega)^2 x^2.
\]
%
\begin{itemize}
%
\item[d)] Finn de to laveste energi egentilstandene for $\OP{H}_{ny}$.
%
\end{itemize}
%
La oss angi alle energi egenfunksjonene for den nye hamilton operatoren
$\OP{H}_{ny}$ som $\phi_n(x)$.
%
\begin{itemize}
%
\item[e)] I pkt.~d) fant vi funksjonen $\psi_k(x,t)$ for $t \leq 0$.
Sett opp  $\psi_k(x,t)$ for $t > 0 $ uttrykt ved energi egenfunksjonene
$\phi_n(x)$  for $\OP{H}_{ny}$. 
%
\item[f)] Beregn sannsynligheten  for at systemet 
ved tiden $t = 0^{+}$
(et infinitesimalt tidspunkt etter $ t = 0$) er i grunntilstanden for 
$\OP{H}_{ny}$. 
Hva er sannsynligheten for at systemet er i den f�rste  eksiterte tilstand 
for $\OP{H}_{ny}$? 
% 
\item[g)] Beregn  $\langle \OP{x}^2 \rangle$,  
$ \langle \OP{x} \rangle$ og den midlere energi 
$\langle \OP{E} \rangle$ ved tiden $t = 0^{+}$.
%
\item[h)] Hvilke av de tre st�rrelsene beregnet i pkt.~g) er uavhengig av 
tiden $t$ for $ t > 0$? Begrunn svaret.
%
\end{itemize}
%
\subsubsection*{Kort fasit}
\begin{itemize}
% 
\item[a)]  $\psi_k(x)$ svarer til  $n = 0$ og  $\psi_l(x)$ til  $n = 1$.
 $\;\alpha = (m\omega) / 2 \hbar)$ og $N = ((m \omega) / (\hbar \pi))^{1/4}$.
%
\item[c)]Se l{\ae}reboka {\sl Brehm and Mullin: Introduction to the
           structure of matter}, eksempel side 233,
           $\; \psi_k(x, t) = \psi_k(x) \exp(-(i/\hbar)(\hbar\omega/2)\; t)$ 
%
\item[d)] Vi f{\aa}r de to laveste l{\o}sningene ved {\aa} ta $\psi_k$ og $\psi_l$ 
fra pkt. a) og bytte ut $\omega$ med $2 \omega$.
%
\item[e)] Se l{\ae}reboka {\sl Brehm and Mullin: Introduction to the
           structure of matter}, eksempel side 233.
 Funksjonen $\psi_k(x)$ har paritet +1.
For $ t > 0$ f{\aa}r vi derfor
\[
\psi_k(x,t) = \sum_{like\; n} c_n \phi_n(x) \exp(-(i/\hbar)E_n^{ny}\; t)
\]
%
\item[f)] Sannsynlighetet for {\aa} finne systemet i grunntilstanden for 
$\OP{H}^{ny}$ ved $t = 0^{+}$ blir 
\[
P_0 = \left (\int_{-\infty}^{+\infty} 
             \psi_k(x, 0^{+})^{*} \phi_0(x, 0^{+})dx \right )
    = (8/9)^{1/2}
\]
Sannsynlighetet for {\aa} finne systemet i 1. eksiterte tilstand er null.
%
\item[g)] $<x^2> = 1 /(4 \alpha)$, $<x> = 0$ og 
\[
<E> = \sqrt{\frac{2 \alpha}{\pi}} 
    \int_{-\infty}^{+\infty} \exp(-\alpha x^2) \OP{H}^{ny} \exp(-\alpha x^2) dx = \frac{7}{4} \hbar \omega
\]
%
\item[h)] Vi bruker svaret i pkt. e) og f{\aa} at middelverdien av 
en fysisk st{\o}rrelse (operator) blir
\[
<\OP{F}>
= \sum_{n,n^{'}} c_n^{*} c_{n^{'}} 
         \int_{-\infty}^{+\infty}\phi_n(x,t)^{*} \OP{F} \phi_{n^{'}}(x,t) dx
 \exp(-\frac{i}{\hbar} (E_{n^{'}} - E_n) \; t)
\]
For $\OP{F} = \OP{H}^{ny}$ f{\aa}r vi 
\[
\int_{-\infty}^{+\infty}\phi_n(x,t)^{*} \OP{F} \phi_{n^{'}}(x,t) dx
= E_n^{ny} \delta_{n,n^{'}}
\]
og
\[
< \OP{H}^{ny}> = \sum_n |c_n|^{2} E_n^{ny}, 
\]
dvs. uavhengig av $t$.
For $\OP{F} = x$ blir
\[ 
\int_{-\infty}^{+\infty}\phi_n(x,t)^{*} x \phi_{n^{'}}(x,t) dx = 0
\]
siden integranden har paritet -1 og $<x>$ blir tidsuavhengig.
For $\OP{F} = x^2$ blir middelverdien avhengig av $t$.
%
\end{itemize}
%

\subsubsection*{Oppgave 5.9}
En partikkel med masse m og energi $E$ beveger seg
langs x-aksen i potensialet
%
\[
V(x) = \left \{
\begin{array}{c}
   +\infty  \\
      0\\
  + \infty
\end{array}
%
\;\;\; \mbox{for} \;\;\;
%
\begin{array}{r}
 x < 0 \\
 0  \leq x < L / 2 \\
 x \geq  L / 2
\end{array}
%
\right .
%
\]
%

\begin{itemize}
%
\item[a)] Vis at den laveste energi egenfunksjonen er gitt ved 
\[
\psi(x) = \frac{2}{\sqrt{L}} \sin \frac{2 \pi x}{L}
\]
%
for $0 < x < L/2$ og null ellers.
%
\end{itemize}
%
Plutselig blir den potensielle energien for partikkelen forandret til 
%
\[
V(x) = \left \{
\begin{array}{c}
   + \infty  \\
     0\\
  + \infty
\end{array}
%
\;\;\; \mbox{for} \;\;\;
%
\begin{array}{r}
 x < 0 \\
 0 \leq x < a \\
 x \geq  a
\end{array}
%
\right .
%
\]
%
Vi antar at b�lgefunksjonen umiddelbart etter forandringen er den samme 
som f�r forandringen (b�lgefunksjonen er kontinuerlig i tiden).
%
\begin{itemize}
%
\item[b)] Hvis vi gj�r en m�ling av systemets energi f�r forandringen, 
vil vi da f� en skarp verdi (samme verdi om vi m�ler p� mange identiske systemer)?
F�r vi en skarp verdi for energien om vi gj�r m�lingen etter forandringen?
%
\item[c)] Beregn sannsynligheten for at partikkelen er i den f�rste eksiterte egentilstand
($n = 2$) for systemet etter forandringen. 
%
\item[d)] Hva er sannsynligheten for at partikkelen er i grunntilstanden
($n = 1$) for systemet etter forandringen?
%
\item[e)] Vil energien av systemet v�re bevart ved en slik forandring av den 
potensielle energien? Hva skjer med  $\langle \OP{E} \rangle$ 
ved forandringen?
%
\end{itemize}
%

\subsubsection*{Oppgave 5.10, Eksamen V-1999}
En partikkel med masse $m$ beveger seg i et tre--dimensjonalt sentralpotensial
av formen
%
\[ 
   V(r) = V_0 r^k a^{-k},
\]
%
hvor $V_0$, $a$ og $k$ er konstanter og $r$ er partikkelens avstand fra origo.
Konstanten $k $ er heltallig og kan anta verdiene $k=0,\pm 1,\pm 2 \dots$.
%
\begin{itemize}
%
\item[a)]  Hvilken dimensjon m\aa\ $a$ og $V_0$ ha dersom vi forlanger at 
   disse dimensjonene skal gjelde for alle
               verdier av $k$? 
%
\end{itemize}
%
Vi skal f�rst betrakte systemet ved bruk av klassisk mekanikk
og antar at partikkelen beveger seg  i sirkul�re  baner.
%
\begin{itemize}
%
\item[b)] Vis at partikkelens hastighet er gitt ved 
%
   \[
      v^2= V_0 \frac{k}{m}  \left (\frac{r}{a} \right )^k.
   \]
%
For denne type potensiell energi har vi f�lgende
sammenheng mellom partik\-kelens kinetiske energi 
$T$ og den potensielle energi $V$   
%
\[
 2T = kV.
\] 
%
\item[c)] Sett opp Bohrs kvantiseringsbetingelse.
Bruk denne betingelsen og finn partik\-kelens totale
energi som funksjon av kvantetallet $n$.
%
\end{itemize}
%
I resten av denne oppgaven skal vi bruke Schr\"{o}dingers
kvantemekaniske beskrivelse.
Anta at vi n� har et \'{e}n--dimensjonalt system og erstatt
$r$ med $x$.
%
\begin{itemize}
%
\item[d)] Sett opp Schr\"{o}dingers energi egenverdiligning for
dette tilfellet og med  $k=2$ (harmonisk oscillator).
%
\end{itemize}
%
B�lgefunksjonen for grunntilstanden er i dette tilfelle
gitt p� formen 
%
\[
     \Psi (x) = C \exp{(-\alpha x^2)},
\]

hvor $\alpha$ er en konstant og $C$ er normaliseringskonstanten 
bestemt av ligningen
% 
\[
    \int_{-\infty}^{\infty} \Psi^*(x)\Psi(x) dx =1.
\]
%

\begin{itemize}
%
\item[e)]  Normalis\'{e}r $\Psi (x)$ og beregn 
   grunntilstandsenergien $E_0(\alpha)$ som funksjon av $\alpha$. 
   Finn den verdien av $\alpha$ som gir lavest energi og den 
   tilsvarende grunntilstands\-energien.
%
\item[f)] Vis at $<{{\bf p}}>=<{{\bf x}}>=0$ i dette tilfelle.
%
\end{itemize}
%
Uskarpheten for den fysiske st�rrelsen $A$ er gitt ved
$\Delta A =\sqrt{ <{A}^2> - <{A}>^2}$
%
\begin{itemize}
%
\item[g)] Uttrykk grunntilstandsenergien for den harmoniske
oscillator  fra d) ved hjelp av $\Delta {p}$ og $\Delta {x}$.
Formul\'{e}r Heisenbergs uskarphetsrelasjon og gi en fysisk forklaring
p� de st�rrelser som inng�r. Bruk denne relasjonen  til � vise at den laveste
 energien som systemet kan ha er gitt ved
%
               \[
                   E_{min}=\frac{\hbar}{a}\sqrt{\frac{V_0}{2m}}.
                \]
%
Sammenlign med resultat fra e) og komment\'{e}r.
%
\end{itemize}              
\subsubsection*{L\o sning}
%
\begin{itemize}
% 
\item[a)] Dimensjonen av $V_0$ og $a$ skal v�re uavhengig av k.
Hvis vi skriver den potensielle energien p� formen 
% 
\[
V(r) = V_0 \left ( \frac{r}{a} \right )^k
\]
% 
m� $(r/a)$ v�re ubenevnt. Dette gi  $a$ dimensjon \underline{lengde}.
Siden $V(r)$ skal ha dimensjon energi, f�lger at $V_0$ m� ha dimensjon
\underline{energi}.
%
\item[b)] For klassisk sirkul�r bevegelse har vi 
%
\[
K_r = -\frac{dV(r)}{dr} = - \frac{m v^2}{r}
\]
% 
Dette gir
% 
\[
k\cdot V_0 r^{k - 1} \cdot a^{-k} = \frac{m v^2}{r}
          \quad   \longrightarrow \quad
         v^2 = V_0 \frac{k}{m} \left ( \frac{r}{a} \right )^k
\]
% 
\item[c)] Bohrs kvantiseringsbetingelse: $L = n \hbar$. For sirkul�r
bevegelse gir dette $L = r m v = n \hbar$. Partikkelens totale energi
bli da 
%
\begin{eqnarray*}
E &=& T + V = \frac{1}{2} m v^2 + V_0 \left ( \frac{r}{a} \right )^k
       = \frac{1}{2}m V_0 \frac{k}{m} \left ( \frac{r}{a} \right )^k
           +  V_0 \left ( \frac{r}{a} \right )^k\\
  &=& \left ( \frac{k}{2} + 1 \right ) V_0  
          \left ( \frac{r}{a} \right )^k
\end{eqnarray*}
% 
Vi eliminerer $r$ og $v$ fra Bohr kvantiseringsbetingelse
% 
\begin{eqnarray*}
r \cdot m \cdot \left ( V_0 \frac{k}{m} \right ) ^{1/2} 
            \left ( \frac{r}{a} \right )^{k/2} &=& n \hbar\\
\left ( \frac{r}{a} \right )^{\frac{k}{k + 2}} 
        \left (V_0 m k \right )^{1/2} &=& \frac{n \hbar}{a}\\
\left ( \frac{r}{a} \right )^{k + 2} 
       &=& \left (\frac{n \hbar}{a} \right )^2 \frac{1}{V_0 m k}\\
\left ( \frac{r}{a} \right )^k  
       &=&\left ( \frac{n^2 \hbar^2}{a^2 V_0 m k} 
                     \right )^{\frac{k}{k + 2}}
\end{eqnarray*}
% 
Dette gir 
% 
\[
E = \left ( \frac{k + 2}{2} \right ) V_0 
         \left ( \frac{n^2 \hbar^2}{a^2 V_0 m k} 
                     \right )^{\frac{k}{k + 2}}
\]
%
\item[d)] Schr\"{o}dingers energi egenverdiligning blir 
% 
\begin{eqnarray*}
-\frac{\hbar^2}{2m} \frac{d^2}{dx^2} \Psi(x) + V_0 \left (\frac{x}{a}
\right )^k \Psi(x) &=& E \Psi(x)\\
 \frac{d^2}{dx^2} \Psi(x) + \left ( \frac{2mE}{\hbar^2} 
   - \frac{2 m V_0}{\hbar^2 a^2} x^2 \right ) \Psi(x) &=& 0
\end{eqnarray*}
% 
\item[e)] Normalisering
% 
\[
1 = C^2 \int_{-\infty}^{+\infty} e^{-2 \alpha x^2} dx 
      = C^2 \sqrt{\frac{1}{2 \alpha}} \int_{-\infty}^{+\infty} e^{-t^2} dt
      = C^2 \sqrt{\frac{\pi}{2 \alpha}}
        \quad \quad \quad \mbox{(Rottman side 155)}
\]
% 
som gir $ C = \left ( \frac{2 \alpha}{\pi} \right ) ^{1/4}$.

B�lgefunksjonen er avhengig av parameteren $\alpha$. Partikkelens
energi blir da en funksjon av $\alpha$.
% 
\begin{eqnarray*}
E(\alpha) &=&  \int_{-\infty}^{+\infty} \Psi^{*}(x) \OP{H} \Psi(x) dx
          = \int_{-\infty}^{+\infty} \Psi^{*}(x)
            \left \{ -\frac{\hbar^2}{2m} \frac{d^2}{dx^2} 
             + V_0 \left ( \frac{x}{a} \right )^2 \right \} \Psi(x) dx\\
    &=& \int_{-\infty}^{+\infty} \left \{ - \frac{\hbar^2}{2 m} \left (
          4 \alpha^2 x^2 - 2 \alpha \right )
        + \frac{V_0}{a^2} x^2 \right \} \left |\Psi(x) \right |^2 dx\\
    &=& \frac{\hbar^2 \alpha}{m} + \left ( \frac{V_0}{a^2} 
         - \frac{2 \hbar^2 \alpha^2}{m} \right ) C^2
               \int_{-\infty}^{+\infty} x^2 e^{-2 \alpha x^2} dx\\
   &=&  \frac{\hbar^2 \alpha}{m} + \left ( \frac{V_0}{a^2} 
         - \frac{2 \hbar^2 \alpha^2}{m} \right ) \sqrt{\frac{2
          \alpha}{\pi}}
\cdot \frac{1}{4 \alpha} \sqrt{\frac{\pi}{2 \alpha}}\\
   &=& \frac{\hbar^2 \alpha}{m} + \left ( \frac{V_0}{a^2} 
         - \frac{2 \hbar^2\alpha^2}{m} \right ) \frac{1}{4 \alpha}
           = \frac{\hbar^2 \alpha}{2m} + \frac{V_0}{4 a^2 \alpha}
\end{eqnarray*}

%
Minimum $E(\alpha)$ bestemmes ved 
% 
\[
\frac{d}{d\alpha} E(\alpha) = \frac{\hbar^2}{2 m} 
- \frac{V_0}{4 a^2 \alpha^2} = 0
\]
% 
med l�sning
% 
\[
\alpha_{min} = \frac{1}{\hbar a} \sqrt{\frac{m V_0}{2}}
\quad \longrightarrow \quad E_{min} = \frac{\hbar}{a} \sqrt{
\frac{V_0}{2m}}
\]
% 
\item[f)] For enkelte funksjoner gjelder f�lgende
% 
\[
f(x) = - f(-x) \quad \mbox{som gir} 
   \quad \int_{-\infty}^{+\infty} f(x)= 0 \quad \mbox{odde  funksjon
                                              -- minus paritet}\\
\]
%
Dette kan vi bruke til � beregne $<x>$ og $<p>$. Vi f�r
\begin{eqnarray*}
<x> &=& \int_{-\infty}^{+\infty} \Psi^{*}(x) \cdot x \cdot \Psi(x)dx
= \int_{-\infty}^{+\infty} x \left |\Psi(x)\right |^2dx = 0\\
<p> &=& \int_{-\infty}^{+\infty} \Psi^{*}(x) (-i\hbar)
                      \frac{d\Psi(x)}{dx} dx = 0
\end{eqnarray*}
% 
siden begge integrandene er odde funksjoner.
%
\item[g)] Fra f) f�lger at $\Delta x = \sqrt{<x^2> - <x>^2} = \sqrt{<x^2>}$ og
% 
\[
<x^2> = \int_{-\infty}^{+\infty} \Psi^{*}(x) \cdot x^2 \cdot \Psi(x) dx
    = C^2  \int_{-\infty}^{+\infty} x^2 e^{-2\alpha x^2} dx
    = \sqrt{\frac{2 \alpha}{\pi}} \frac{1}{4 \alpha} 
       \sqrt{\frac{\pi}{2 \alpha}}
   = \frac{1}{4 \alpha}
\]
% 
som gir 
% 
\[
\Delta x = \frac{1}{2\sqrt{\alpha}}
\]
%
og
%
\begin{eqnarray*}
<p^2> &=& \int_{-\infty}^{+\infty} \Psi^{*}(x) \left ( -i\hbar
    \frac{d}{dx} \right )^2 \Psi(x) dx
= -\hbar^2 \int_{-\infty}^{+\infty} (4\alpha^2 x^2 - 2 \alpha) \left
    |\Psi(x) \right|^2  dx\\
   &=& 2 \alpha \hbar^2  \int_{-\infty}^{+\infty} \left |\Psi(x) \right|^2 dx
    -4 \alpha^2 \hbar^2 C^2 \int_{-\infty}^{+\infty} x^2 e^{-2\alpha x^2}dx
    =  2 \alpha \hbar^2  - \alpha \hbar^2 = \alpha \hbar^2
\end{eqnarray*}
%
som gir 
% 
\[
\Delta p = \hbar \sqrt{\alpha}
\]
% 
Heisenbergs uskarphetsrelasjon blir da 
% 
\[
\Delta p \cdot \Delta x = \frac{1}{2} \sqrt{\frac{1}{\alpha}}
                         \cdot \hbar \sqrt{\alpha}
                         = \frac{\hbar}{2}
\]
% 
Middelverdien av partikkelens energi 
% 
\begin{eqnarray*}
<E> &=& < \frac{p^2}{2 m} + \frac{V_0}{a^2} x^2 >
  = \frac{1}{2 m} <p^2> + \frac{V_0}{a^2}<x^2>\\
  &=& \frac{1}{2 m} \left (\Delta p \right )^2 
  + \frac{V_0}{a^2}\left ( \Delta x \right )^2
  \ge \frac{1}{2m} \left ( \frac{\hbar}{2}\frac{1}{\Delta x} \right )^2
      + \frac{V_0}{a^2} \left (\Delta x \right )^2
\end{eqnarray*}
% 
Vi minimaliserer $<E>$ med hensyn p� $\Delta x$. Dette gir 
% 
\[
\frac{d<E>}{d (\Delta x)} = \frac{\hbar^2}{8m} \left ( \frac{-2}{\Delta
x^3} \right ) + 2 \frac{V_0}{a^2} \Delta x = 0
\]
% 
som gir 
%
\[
\left ( \Delta x_{min} \right )^4 = \frac{\hbar^2 a^2}{8 m V_0}
\quad \longrightarrow \quad
\left ( \Delta x_{min} \right )^2 = \frac{\hbar a}{2} \sqrt{\frac{1}{2 m
V_0}}
\]
%
og energi minimum blir
% 
\[
E_{min} = \frac{\hbar^2}{8 m} \frac{2}{\hbar a} \sqrt{ 2 m V_0}
           + \frac{V_0}{a^2} \frac{\hbar a}{2} \frac{1}{\sqrt{2 m V_0}}
        = \frac{\hbar}{a} \sqrt{\frac{V_0}{2m}}
\]
%   
Denne verdien svarer til $E_{min}$ beregnet i punkt e) og som er systemets
grunntilstandsenergi. Vi kan da si at grunntilstanden svarer til en
maksimal lokalisering av partikkelens posisjon som Heisenbergs
uskarphetsrelasjon tillater.

\end{itemize}


%\subsubsection*{Oppgave 5.11}

%\subsubsection*{Oppgave 5.12}

%\subsubsection*{Oppgave 5.13}



\clearemptydoublepage


\chapter{Quantum mechanical tunneling}

    Vi skal i dette avsnittet presentere
    en detaljert l{\o}sning av Schr\"{o}dinger\-lik\-ningen
    anvendt p{\aa} partikler som tref\-fer en  potensialbarri\'{e}re.
    Problemet dekker et spesielt kvantemekanisk fenomen
    som kalles {\bf tunneling}\footnote{Lesehenvisning: kap 5-9 i boka, sidene 271-283.} og har mange anvendelser i moderne
    mikrofysikk. Spesielt skal vi anvende resultatene p\aa\
    henfall av $\alpha$-partikler og relatere dette til 
levetid for kjerner som henfaller ved denne type prosesser.

Tunneling har avf\o dt flere Nobel-priser i fysikk, til Leo
Esaki for tunneling i halvledere, Ivar Gj\ae ver (eneste
norske Nobelpris i fysikk) for tunneling i supraledere,
Brian Josephson for tunneling av parrede elektroner 
i supraledere (alle
i 1973) og Gerd Binning og Heinrich Rohrer for oppfinnelsen
av 'scanning-tunneling' elektron mikroskop i 1986. 
I tillegg er tunneling et viktig tema i halvlederteknologi.

%
\section{Klassisk spredning mot potensialbarriere}
%
Vi starter med {\aa} formulere et klassisk fysisk problem
hvor en str{\o}m av  partikler hver med masse $m$ beveger seg 
langs x--aksen.
De starter i $x = -\infty$ med energi $E$ 
og hastighet $v = + \sqrt{2E / m}$.
Partiklene beveger seg uten vekselvirkning i  omr�det $-\infty < x < 0$.
I $ x = 0$ treffer partiklene en potensialbarri\'{e}re
gitt ved 
%
\[
V(x) = \left \{
\begin{array}{c}
 0  \\
 \frac{V_{0}}{2}\left (
  \sin (\frac{2\pi}{L}x - \frac{\pi}{2}) + 1 \right ) \\
 0 
\end{array}
%
\;\;\; \mbox{for} \;\;\;
%
\begin{array}{r}
 x < 0 \\
 0\leq x < L \\
 x \geq  L
\end{array}
%
\right .
%
\]
%

Det potensielle energien har sitt maksimum ved $ x = \pi / 2$,
$V(\pi / 2) = V_{0}$. For den videre bevegelse av partiklene m{\aa}
vi skille mellom to tilfeller:
%
\begin{itemize}
%
\item $ E < V_0$ : Ingen partikler har nok kinetisk energi til
{\aa} passere potensialbarri\'{e}ren. De vil stoppe og returnere mot 
$ x = -\infty$ med hastighet $v = -\sqrt{2E / m}$. 
Vi f{\aa}r en str{\aa}le av partikler mot h{\o}yre med 
$v = +\sqrt{2E / m}$
og en reflektert str{\aa}le av partikler mot venstre med 
$v = -\sqrt{2E / m}$. Ingen partikler passerer  potensialbarri\'{e}ren.

%
\item $ E > V_0$ : Partiklene har nok kinetisk energi til
{\aa} passere potensialbarri\'{e}ren. De vil fortsette inn i 
omr{\aa}det $x > L$ med samme hastighet $v = +\sqrt{2E / m}$.
Vi f{\aa}r ingen reflektert str{\aa}le av partikler
for $ x < 0$. 
%
\end{itemize}
%
Dette er en velkjent fysisk situasjon og lett {\aa} analysere ut fra 
klassisk fysikk. Vi skal n{\aa} studere den samme situasjonen
kvantemekanisk og analysere den ut fra Schr\"{o}dingerlikningen.
%


\section{Kvantemekanisk barrierespredning}
%
Vi forenkler situasjonen noe  ved {\aa} velge en
rektangul{\ae}r potensialbarri\'{e}re
gitt p{\aa} formen
%
\begin{equation}
V(x) = \left \{
\begin{array}{c}
 0  \\
 V_0\\
 0
\end{array}
%
\;\;\; \mbox{for} \;\;\;
%
\begin{array}{r}
 x < 0 \\
 0\leq x < a \\
 x \geq  a
\end{array}
%
\right .
%
\label{er1}
\end{equation}
%
som skissert i Figur \ref{fig2}.
Den tidsavhengige Schr\"{o}dingerlikningen  blir
%
\begin{equation}
-\frac{\hbar^2}{2 m} \frac{\partial^2 \Psi(x,t)}{\partial x^2}
+ V(x) \Psi(x,t) = i \hbar \frac{\partial \Psi(x,t)}{\partial t} .
\label{er2}
\end{equation}
%
Vi l{\o}ser likningen ved {\aa} dele opp intervallet for $x$ 
i tre omr{\aa}der som angitt i Figur \ref{fig2}.
%
\begin{figure}[htbp]
%
\begin{center}

\setlength{\unitlength}{1cm}
%
\begin{picture}(13,6)

\thicklines

   \put(0,0.5){\makebox(0,0)[bl]{
              \put(0,1){\vector(1,0){12}}
              \put(12.3,1){\makebox(0,0){x}}

               \put(2,1.2){\makebox(0,0){$V(x) = 0$}}
               \put(9,1.2){\makebox(0,0){$V(x) = 0$}}


              \put(5.2,0.8){\makebox(0,0){0}}
              \put(7.2,0.8){\makebox(0,0){a}}
%              \put(7,0){\line(1,0){5}}

              \put(5,1){\vector(0,1){4}}
              \put(5.2,5){\makebox(0,0)[tl]{$V(x)$}}
      
              \put(0,1){\line(1,0){5}}
%              \put(5.5,0){\makebox(0,0){$0$}}

              \put(5,4){\line(1,0){2}}
              \put(7.3,4){\makebox(0,0){$V_0$}}

              \put(7,1){\line(0,1){3}}

              \put(1,2){\makebox(0,0){$Ae^{+ikx}$}}
               \put(2,2){\vector(1,0){2.5}}

              \put(4,3.5){\makebox(0,0){$Be^{-ikx}$}}
               \put(3,3.5){\vector(-1,0){2.5}}

              \put(7.9,2){\makebox(0,0){$Fe^{+ik\! \! x}$}}
               \put(9,2){\vector(1,0){2.5}}

              \put(2,-0.5){\makebox(0,0){Omr�de: \hspace*{1cm} \bf (I)}}
              \put(6,-0.5){\makebox(0,0){\bf (II)}}
              \put(9,-0.5){\makebox(0,0){\bf (III)}}

         }}
%
\end{picture}
%
\caption{\label{fig2}}
\end{center}
\end{figure}




\subsection{Omr{\aa}de I: $-\infty < x < 0$.}

I dette tilfelle f{\aa} likning (\ref{er2}) formen
%
\begin{equation}
-\frac{\hbar^2}{2 m} \frac{\partial^2 }{\partial x^2}\Psi_{I}(x,t)
 = i \hbar \frac{\partial }{\partial t}\Psi_{I}(x,t),
\label{er3}
\end{equation}
%
med l{\o}sning
%
\begin{equation}
%
\Psi_{I}(x,t) =  A e^{i(kx - \omega t)} 
               + B e^{-i(kx +\omega t)} .
\label{er4}
\end{equation}
%

B{\o}lgefunksjonen $\Psi_{I}(x,t)$ er en sum av to plane 
b{\o}lger, begge har frekvens $\omega$ og energi 
$ E = \hbar \omega = \hbar^2 k^2 / 2 m$. Den f{\o}rste komponenten,
 proporsjonal med $A$, beveger seg fra venstre mot h{\o}yre 
(b{\o}lgetall $k$) og er den innkommende prim{\ae}re b{\o}lge.
Den andre komponenten, proporsjonal med $B$, beveger seg fra 
h{\o}yre mot venstre (b{\o}lgetall $-k$) og representerer en 
re\-flek\-tert b{\o}lge fra potensialbarri\'{e}ren $V(x)$ i $x = 0$.
Begge komponentene er angitt i Figur \ref{fig2} for tiden $t = 0$.
Dette er analogt med bevegelse av elektromagnetiske 
b{\o}lger hvor vi f{\aa}r refleksjon i en overgang mellom to
forskjellige medier, her representert ved en potensialbarri\'{e}re.
 
For {\aa} beskrive intensiteten av  partikler i str{\aa}len
bruker vi begrepet fluks, dvs. antall partikler som passerer
en enhetsflate per sekund.  Antall partikler som reflekteres i $x = 0$
i forhold til antall partikler i den innkommende str{\aa}le
defineres som {\bf refleksjonskoeffisienten} og betegnes 
med symbolet $R$. Dette gir 
%
\begin{equation}
R = \frac{\mbox{fluks mot venstre}}{\mbox{fluks mot h{\o}yre}}
 = \frac{v|B|^2}{v|A|^2} = \frac{|B|^2}{|A|^2}.
\label{er5}
\end{equation}
%
I en b{\o}lgeterminologi er $R$ den delen av den innkommende 
b{\o}lgen som blir reflektert av poten\-sial\-bar\-ri\'{e}ren.
I et partikkelbilde blir $R$ {\aa} tolke som sannsynligheten
for at en partikkel som kommer inn fra venstre, blir reflektert.




\subsection{Omr\aa de II: $0 < x < a$.}

I dette tilfelle f{\aa}r likning (\ref{er2}) formen

\begin{equation}
-\frac{\hbar^2}{2 m} \frac{\partial^2 }{\partial x^2}\Psi_{II}(x,t)
 + V_0 \Psi_{II}(x,t)
  = i \hbar \frac{\partial }{\partial t}\Psi_{II}(x,t) , 
\label{er6}
\end{equation}
%
med l{\o}sning
%
\begin{equation}
%
\Psi_{II}(x,t) = C e^{i(k^{'}x - \omega t)} 
                 + D e^{-i(k^{'}x +\omega t)} .
\label{er7}
\end{equation}
%

Her m{\aa} vi skille mellom : a) $E < V_0$ og b) $E > V_0$.
I tilfelle a) blir $k^{'}$ rent imagin{\ae}r, 
dvs.~at $k^{'} = \sqrt{2m(E - V_0 )/\hbar^2} = i\alpha$, med 
$ \alpha = \sqrt{2m(V_0 - E )/\hbar^2}$.

I tilfelle b) er $ k^{'} = \sqrt{2m(E - V_0 )/\hbar^2}$, 
og l{\o}sningene
blir av samme type som for omr{\aa}de~I, men med b{\o}lgetall $k^{'}$.
Vi vil i det f{\o}lgende begrense oss til tilfelle a).
Da kan vi omforme b{\o}lgefunksjonen i likning (\ref{er7}) til 
%
\begin{equation}
%
\Psi_{II}(x,t) = C e^{-\alpha x -i\omega t}
                 + D e^{\alpha x -i\omega t} .
\label{er8}
\end{equation}
% 
Vi har n{\aa} f{\aa}tt l{\o}sningene av Schr\"odingers likning 
for omr{\aa}dene I og II. I overgangen
$x = 0$ m{\aa} den endelige b{\o}lgefunksjonen v{\ae}re kontinuerlig
og ha en kontinuerlig derivert. Dette gir f{\o}lgende betingelser
for $t = 0$
%
\begin{equation}
A + B = C + D \;\;\; \mbox{og}\;\;\; ik(A - B) = - \alpha (C - D) .
\label{eq:er9}
\end{equation}
%
B�lgefunksjonen $\Psi_{II}$ er en eksponentielt
avtagende funksjon i det klassiske forbudte omr{\aa}det.
F{\o}r vi diskuterer konsekvensene av disse betingelsene,
skal vi  l{\o}se Schr\"{o}dingerlikningen i 
omr{\aa}det~III.
%
%
\subsection{Omr{\aa}det~III: $a< x < +\infty$}
%
Her f{\aa}r likning (\ref{er2}) formen
%
\begin{equation}
-\frac{\hbar^2}{2 m} \frac{\partial^2 }{\partial x^2}
\Psi_{III}(x,t)
  = i \hbar \frac{\partial }{\partial t}\Psi_{III}(x,t) , 
\label{er10}
\end{equation}
%
med l{\o}sning
%
\begin{equation}
%
\Psi_{III}(x,t) = F e^{i(kx - \omega t)} 
                 + G e^{-i(kx +\omega t)} ,
\label{er11}
\end{equation}
%
med $ k = \sqrt{2m(E)/\hbar^2}$. 
Legg merke til $k$ i omr\aa de I og III er like. Til slutt i dette
avsnittet skal vi ogs\aa\ se p\aa\ tilfellet hvor de er ulike.
B{\o}lgefunksjonen $\Psi_{III}(x,t)$ 
m{\aa} forst{\aa}s analogt til $\Psi_{I}(x,t)$ som en sum av 
to plane b{\o}lger. Den f{\o}rste komponenten,
proporsjonal med $F$, beveger seg fra venstre mot h{\o}yre 
(b{\o}lgetall $k$).
Den andre b{\o}lgen, proporsjonal med $G$, beveger seg fra 
h{\o}yre mot venstre (b{\o}lgetall $-k$).
Vi har ingen potensialbarri\'{e}re til h{\o}yre for $x = a$.
Det vil da ikke kunne oppst{\aa} noen reflektert b{\o}lge, 
og vi m{\aa} kreve  $G = 0$. likning (\ref{er11}) reduseres til
%
\begin{equation}
%
\Psi_{III}(x,t) = F e^{i(kx - \omega t)} .
 \label{er12}
\end{equation}
%
% 
Vi har n{\aa} l{\o}sningene av Schr\"odingers likning  for omr{\aa}det III. I overgangen
$x = a$ m{\aa} den endelige b{\o}lgefunksjonen v{\ae}re kontinuerlig
og ha en kontinuerlig derivert. Dette gir f{\o}lgende betingelser
for $t = 0$
%
\begin{equation}
Ce^{-\alpha a} + De^{\alpha a}  
   = Fe^{ika} 
\;\;\; \mbox{og}\;\;\;  
\alpha (-Ce^{-\alpha a} + De^{\alpha a}) = ikFe^{ika} .
\label{er13}
\end{equation}
%
\subsection{Det totale l{\o}sning.}
%
Vi har l{\o}st Schr\"odingers likning  i alle tre intervallene. 
I overgangene mellom de tre omr{\aa}dene f{\aa}r vi betingelsene 
gitt i likningene (\ref{eq:er9}) og (\ref{er13}). 
Disse likningene kan vi sammenfatte 
til
%
%
\begin{equation}
\begin{array}{rclcrcl}
%
A + B &=& C + D & \mbox{og} & ik(A - B) &=& -\alpha (C - D)\\
Ce^{-\alpha a} + De^{\alpha a}  
   &=& Fe^{ika}& 
\;\;\; \mbox{og}\;\;\;&  
\alpha (-Ce^{-\alpha a} + De^{\alpha a}) &=& ikFe^{ika} .
\end{array}
\label{er14}
\end{equation}
%
Dette er fire likninger med fem ukjente, og vi kan kun  
bestemme fire av de ukjente, da i forhold til den femte. I v{\aa}rt 
tilfelle vil det v{\ae}re naturlig {\aa} bestemme $B,C,D,F$ i 
forhold til $A$ som angir intensiteten av den prim{\ae}re partikkelstr{\aa}len. 
Dette gir 
%
\begin{equation}
\begin{array}{rclcrcl}
1 + \frac{B}{A} &=& \frac{C}{A} + \frac{D}{A} & \mbox{og} 
                   & ik(1 - \frac{B}{A}) &=& -\alpha (\frac{C}{A} - \frac{D}{A})\\[2ex]
\frac{C}{A}e^{-\alpha a} + \frac{D}{A}e^{\alpha a}  
   &=& \frac{F}{A}e^{ika}& 
\;\;\; \mbox{og}\;\;\; & 
\alpha (-\frac{C}{A}e^{-\alpha a} + \frac{D}{A}e^{\alpha a}) 
&=& ik\frac{F}{A}e^{ika} .
\end{array}
\label{er15}
\end{equation}
%

Til {\aa} angi hvor mange partikler som trenger gjennom barri\'{e}ren defineres
{\bf Transmi\-sjons\-koef\-fisienten T} ved 
%
\begin{equation}
T = \frac{\mbox{fluks mot h{\o}yre i omr{\aa}det III}}
         {\mbox{fluks mot h{\o}yre i omr{\aa}det I}}
  = \frac{v_{III}|F|^2}{v_I|A|^2}= \frac{|F|^2}{|A|^2},
%
\label{er16}
\end{equation}
%
siden vi har at $k$ i omr\aa de I og III er like, noe som medf\o rer at
hastighetene i omr{\aa}dene I og III er like.

\section{Spesialtilfeller}

I de foreg\aa ende underavsnittene har vi sett p\aa\ de generelle
likningene for barrieregjennomtrengning.
La oss n\aa\ se p\aa\ et spesialtilfelle hvor $a\rightarrow \infty$,
dvs.~at potensialbarrieren har uendelig utstrekning. 

I dette tilfelle er det kun omr\aa dene I og II som er av interesse.

\subsection{Tilfellet $E< V_0$}

For omr\aa de II kan vi ikke ha l\o sningene med $De^{\alpha x}$ da
det inneb\ae rer at egenfunksjonen $\psi_{II}$ vil divergere.
Det betyr igjen at konstanten $D=0$ slik at vi f\aa r
\begin{equation}
\Psi_{II}(x,t) = C e^{-\alpha x -i\omega t}.
\end{equation}
Siden $E< V_0$ 
blir $k^{'}$ rent imagin{\ae}r,
\mbox{$ k^{'} = \sqrt{2m(E - V_0 )/\hbar^2} = i\alpha$} med 
$ \alpha = \sqrt{2m(V_0 - E )/\hbar^2}$.
Likning (\ref{eq:er9}) blir da
%
\begin{equation}
A + B = C \;\;\; \mbox{og}\;\;\; ik(A - B) = - \alpha C 
\end{equation}
som kan omskrives som
\be
    \frac{C}{A}=\frac{B}{A}+1,
\ee
og 
\be
    \frac{C}{A}=i\frac{k}{\alpha}(\frac{B}{A}-1),
\ee
som gir
\be
    \frac{C}{A}=\frac{2k^2+2i k\alpha}{k^2+\alpha^2},
\ee
og 
\be
    \frac{B}{A}=\frac{k^2-\alpha^2+2i k\alpha}{k^2+\alpha^2}.
\ee
Bruker vi definisjonen p\aa\ refleksjonkoeffisienten gitt ved
\be
   R=\frac{|B|^2}{|A|^2}=\frac{B^*B}{A^*A},
\ee
finner vi at $R=1$, dvs.~alle partiklene blir reflektert, som forventa
utifra klassisk fysikk. Men, b\o lgefunksjonen i omr\aa de II 
trenger inn i det klassisk forbudte omr\aa det. Det er nytt, og er
sj\o lve \aa rsaken til at vi kan ha kvantemekanisk barrieregjennomtrengning.

\subsection{Tilfellet $E> V_0$}

La oss gjenta beregningen, men denne gangen for $E> V_0$.
Da blir $\alpha$ imagin\ae r, og vi setter $\alpha=i\gamma$.
Det gir oss
\be
    \frac{C}{A}=\frac{2k^2-2k\gamma}{k^2-\gamma^2},
\ee
og 
\be
    \frac{B}{A}=\frac{k^2+\gamma^2-2k\gamma}{k^2-\gamma^2}.
\ee
 
I dette tilfelle blir refleksjonskoeffisienten
\be
   R=\frac{|B|^2}{|A|^2}=\left(\frac{k-\gamma}{k+\gamma}\right)^2 <1,
\ee
dvs.~at ikke alle partiklene blir reflektert. Hva har skjedd?
Partiklene kan ikke ha blitt borte!
For \aa\ l\o se dette mysteriet trenger vi ogs\aa\ \aa\ bruke
transmisjonskoeffisienten, dvs.~hvor mange partikler som passerer barrieren
ved $x=0$ i \o kende $x$-retning. 
Vi definerer denne koeffisienten som
\be
   T=\frac{\mathrm{fluks\hspace{0.1cm} ut}}{\mathrm{fluks\hspace{0.1cm} inn}}=\frac{v'|C|^2}{v|A|^2},
\ee
med
\be
    v=\frac{p}{m}=\frac{\hbar k}{m},
\ee
og
\be
   v'=\frac{\hbar \gamma}{m},
\ee
slik at $T$ blir
\be
    T=\frac{\gamma}{k}\left(\frac{2k}{k+\gamma}\right)^2=\frac{4k\gamma}{(k+\gamma)^2}.
\ee

Summen $R+T$ b\o r helst v\ae re 1, ellers er ikke partikkeltallet bevart.
Setter vi inn for $R$ finner vi
\be
   R+T=\left(\frac{k-\gamma}{k+\gamma}\right)^2+\frac{4k\gamma}{(k+\gamma)^2}=1,
\ee
som berger oss fra brudd p\aa\ partikkelbevaring. 

\subsection{Transmisjon gjennom hele barrieren for $E< V_0$}


N\aa\ skal vi fors\o ke \aa\ l\o se likning (\ref{er15}). 
I dette tilfelle kan vi ikke sette $D=0$, da barrieren har ei 
endelig utstrekning. M\aa lsettinga er \aa\ finne b\aa de 
transmisjonskoeffisienten $T$ og refleksjonskoeffisienten $R$. 
Som tidligere nevnt s\aa\ uttrykker f.eks.~$T$ sannsynlighten
for at partikkelen kan tunnelere gjennom barrieren.

Vi skal ogs\aa\ anta, reint midlertig for symmetrien
i likningene,  at konstanten $G$ ogs\aa\ er forskjellig fra 
null. 
Vi skal omskrive likning (\ref{er14}) som en matriselikning,
noe som gir en elegant m\aa te \aa\ l\o se sistnevnte
likning. La oss f\o rst ta utgangspunkt i likningene
som kopler sammen egenfunksjonene i omr\aa de I og II,
dvs.~ved $x=0$. Vi har da
\be
\begin{array}{rclcrcl}
A + B &=& C + D & \mbox{og} & ik(A - B) &=& -\alpha (C - D),
\end{array}
\end{equation}
hvor vi kan finne en likning for hhv.~$A$ og $B$ ved \aa\ sette
f\o rst
\be
   B=C+D-A,
\ee
som gir 
\be
   2i k A=(i k-\alpha)C+(i k+\alpha)D.
\ee
Tilsvarende finner vi 
\be
      -2i k B=-(i k+\alpha)C+(\alpha-i k)D,
\ee
n\aa r vi bruker $A=C+D-B$.
De nye likningene for $A$ og $B$ kan vi sammenfatte
som en matrise gitt ved
\be
 \left[\begin{array}{c} A \\ B\end{array}\right]=
  \frac{1}{2i k}  \left[\begin{array}{cc}i k -\alpha
                        & i k +\alpha\\
                        i k +\alpha  
                        &i k -\alpha
                        \end{array}\right]
 \left[\begin{array}{c} C \\ D\end{array}\right].
 \label{eq:abcd}
 \ee
P\aa\ tilsvarende vis kan vi omforme
likningene ved $x=a$ 
\be
Ce^{-\alpha a} + De^{\alpha a} =Fe^{ika}+Ge^{ika}
\ee
og 
\be 
\alpha (-Ce^{-\alpha a} + De^{\alpha a}) = ikFe^{ika}-
ikGe^{-ika},
\ee
til
\be
 \left[\begin{array}{c} C \\ D\end{array}\right]=
  \frac{1}{2\alpha}\left[\begin{array}{cc}-(i k -\alpha)e^{(i k+\alpha)a}  
   &(i k +\alpha)e^{-(i k-\alpha)a}  \\
     (i k +\alpha)e^{(i k-\alpha)a}    
                        &-(i k -\alpha)e^{-(i k+\alpha)a}  
                        \end{array}\right]
 \left[\begin{array}{c} F \\ G\end{array}\right].
 \label{eq:cdfg}
 \ee

Insetting av likning (\ref{eq:cdfg}) i likning (\ref{eq:abcd})
gir en likning som relaterer $A$ og $B$ til $F$ og $G$.
Vi finner en matrise likning gitt ved
\be
 \left[\begin{array}{c} A \\ B\end{array}\right]=
  \left[\begin{array}{cc}M_{11} & M_{12} \\
                      M_{21}  &M_{22}\end{array}\right]
 \left[\begin{array}{c} F \\ G\end{array}\right].
 \label{eq:abfg}
 \ee
Setter vi s\aa\ $G=0$ siden vi ikke har en reflektert b\o lge 
i omr\aa de III, har vi
\be 
   A=M_{11}F,
\ee
og 
\be
   B=M_{21}F,
\ee
som igjen betyr at transmisjonkoeffisienten er gitt ved
\be
T = \frac{|F|^2}{|A|^2}=\frac{1}{|M_{11}|^2}=
     \frac{1}{M_{11}^*M_{11}},
\ee
siden vi har satt $k_I=k_{III}$, som f\o lger av at
potensialbarrieren er null i omr\aa dene I og III.
Utifra likningene (\ref{eq:abcd}) og (\ref{eq:cdfg})
kan vi f.eks.~bestemme $M_{11}$
\be
   M_{11}=-\frac{i k -\alpha}{2\alpha}
           \frac{i k -\alpha}{2i k}
           e^{(i k+\alpha)a}  
           +\frac{i k +\alpha}{2i k} 
            \frac{i k +\alpha}{2\alpha}
          e^{(i k-\alpha)a},
\ee
som kan omskrives til
\be
   M_{11}=\left(cosh(\alpha a)-\frac{i}{2}
           \frac{k^2 -\alpha^2}{k\alpha}sinh(\alpha a)\right)e^{i ka}.
\ee
Transmisjonkoeffisienten blir dermed
\be
   T=\frac{1}{\left(cosh^2(\alpha a)+
           \frac{(k^2 -\alpha^2)^2}{4k^2\alpha^2}
           sinh^2(\alpha a)\right)},
\ee
eller
\be
   T=\frac{1}{1+\frac{(k^2 +\alpha^2)^2}{4k^2\alpha^2}
           sinh^2(\alpha a)}.
\label{eq:tbarriere}
\ee
P\aa\ tilsvarende vis kan vi finne refleksjonskoeffisienten
gitt ved (vis dette) 
\be
R = \frac{|B|^2}{|A|^2}=
\frac{\frac{(\alpha^2+k^2)^2}{4k^2\alpha^2}
           sinh^2(\alpha a)}{1+\frac{(k^2 +\alpha^2)^2}{4k^2\alpha^2}
           sinh^2(\alpha a)},
\ee
slik at $T+R=1$, slik det b\o r v\ae re.


\subsection{Transmisjon gjennom hele barrieren for $E> V_0$}

Hva med tilfellet hvor den kinetiske energien til partikkelen 
er st\o rre enn verdien p\aa\ barrieren? 

Vi kan bruke resultatet fra likning (\ref{eq:tbarriere}) med det unntak
at n\aa\ er $\alpha$ gitt ved $\alpha=-i k'$. 
Innsetting av b\o lgetallet $k'$ i  likning (\ref{eq:tbarriere})
gj\o r at $sinh$-leddet g\aa r over til et $sin$-ledd.
Det eneste vi trenger \aa\ gj\o re er alts\aa\ \aa\ erstatte $\alpha$ med
$-i k'$!   Transmisjonskoeffisienten er dermed 
\be
T=\frac{1}{1+\frac{(k^2 -k'^2)^2}{4k^2k'^2}
           sin^2(k' a)}.
\ee
N\aa r $sin (k'a)=0$, dvs.~$k'a=n\pi$, $n$ heltall, ser vi at 
$T$ har sin maksimale verdi $T=1$. Det betyr igjen at ingen partikler 
blir reflektert siden summen av $R+T=1$. Slike egenskaper har konkrete
anvendelser i halvlederteknologi. 

Figur \ref{transcoeff} viser et plott av $T$ for ulike
forhold mellom $E$ og $V_0$. I plottet har vi valgt \aa\ sette
$a=1$, $m=1$, $\hbar=1$ og $V_0=1$, for enkelthetsskyld. Da reduserer
likning (\ref{eq:tbarriere}) seg til for $E< V_0$ 
\be
   T(E)=\frac{1}{1+\frac{1}{4E(1-E)}
           sinh^2(\sqrt{2(1-E)}},
\ee
og for $E>V_0$ til
\be
   T(E)=\frac{1}{1+\frac{1}{4E(E-1)}sin^2(\sqrt{2(E-1)}}.
\ee
\begin{figure}
\begin{center}
% GNUPLOT: LaTeX picture with Postscript
\begingroup%
  \makeatletter%
  \newcommand{\GNUPLOTspecial}{%
    \@sanitize\catcode`\%=14\relax\special}%
  \setlength{\unitlength}{0.1bp}%
{\GNUPLOTspecial{!
%!PS-Adobe-2.0 EPSF-2.0
%%Title: trans.tex
%%Creator: gnuplot 3.7 patchlevel 0.2
%%CreationDate: Wed Mar  8 10:08:54 2000
%%DocumentFonts: 
%%BoundingBox: 0 0 360 216
%%Orientation: Landscape
%%EndComments
/gnudict 256 dict def
gnudict begin
/Color false def
/Solid false def
/gnulinewidth 5.000 def
/userlinewidth gnulinewidth def
/vshift -33 def
/dl {10 mul} def
/hpt_ 31.5 def
/vpt_ 31.5 def
/hpt hpt_ def
/vpt vpt_ def
/M {moveto} bind def
/L {lineto} bind def
/R {rmoveto} bind def
/V {rlineto} bind def
/vpt2 vpt 2 mul def
/hpt2 hpt 2 mul def
/Lshow { currentpoint stroke M
  0 vshift R show } def
/Rshow { currentpoint stroke M
  dup stringwidth pop neg vshift R show } def
/Cshow { currentpoint stroke M
  dup stringwidth pop -2 div vshift R show } def
/UP { dup vpt_ mul /vpt exch def hpt_ mul /hpt exch def
  /hpt2 hpt 2 mul def /vpt2 vpt 2 mul def } def
/DL { Color {setrgbcolor Solid {pop []} if 0 setdash }
 {pop pop pop Solid {pop []} if 0 setdash} ifelse } def
/BL { stroke userlinewidth 2 mul setlinewidth } def
/AL { stroke userlinewidth 2 div setlinewidth } def
/UL { dup gnulinewidth mul /userlinewidth exch def
      10 mul /udl exch def } def
/PL { stroke userlinewidth setlinewidth } def
/LTb { BL [] 0 0 0 DL } def
/LTa { AL [1 udl mul 2 udl mul] 0 setdash 0 0 0 setrgbcolor } def
/LT0 { PL [] 1 0 0 DL } def
/LT1 { PL [4 dl 2 dl] 0 1 0 DL } def
/LT2 { PL [2 dl 3 dl] 0 0 1 DL } def
/LT3 { PL [1 dl 1.5 dl] 1 0 1 DL } def
/LT4 { PL [5 dl 2 dl 1 dl 2 dl] 0 1 1 DL } def
/LT5 { PL [4 dl 3 dl 1 dl 3 dl] 1 1 0 DL } def
/LT6 { PL [2 dl 2 dl 2 dl 4 dl] 0 0 0 DL } def
/LT7 { PL [2 dl 2 dl 2 dl 2 dl 2 dl 4 dl] 1 0.3 0 DL } def
/LT8 { PL [2 dl 2 dl 2 dl 2 dl 2 dl 2 dl 2 dl 4 dl] 0.5 0.5 0.5 DL } def
/Pnt { stroke [] 0 setdash
   gsave 1 setlinecap M 0 0 V stroke grestore } def
/Dia { stroke [] 0 setdash 2 copy vpt add M
  hpt neg vpt neg V hpt vpt neg V
  hpt vpt V hpt neg vpt V closepath stroke
  Pnt } def
/Pls { stroke [] 0 setdash vpt sub M 0 vpt2 V
  currentpoint stroke M
  hpt neg vpt neg R hpt2 0 V stroke
  } def
/Box { stroke [] 0 setdash 2 copy exch hpt sub exch vpt add M
  0 vpt2 neg V hpt2 0 V 0 vpt2 V
  hpt2 neg 0 V closepath stroke
  Pnt } def
/Crs { stroke [] 0 setdash exch hpt sub exch vpt add M
  hpt2 vpt2 neg V currentpoint stroke M
  hpt2 neg 0 R hpt2 vpt2 V stroke } def
/TriU { stroke [] 0 setdash 2 copy vpt 1.12 mul add M
  hpt neg vpt -1.62 mul V
  hpt 2 mul 0 V
  hpt neg vpt 1.62 mul V closepath stroke
  Pnt  } def
/Star { 2 copy Pls Crs } def
/BoxF { stroke [] 0 setdash exch hpt sub exch vpt add M
  0 vpt2 neg V  hpt2 0 V  0 vpt2 V
  hpt2 neg 0 V  closepath fill } def
/TriUF { stroke [] 0 setdash vpt 1.12 mul add M
  hpt neg vpt -1.62 mul V
  hpt 2 mul 0 V
  hpt neg vpt 1.62 mul V closepath fill } def
/TriD { stroke [] 0 setdash 2 copy vpt 1.12 mul sub M
  hpt neg vpt 1.62 mul V
  hpt 2 mul 0 V
  hpt neg vpt -1.62 mul V closepath stroke
  Pnt  } def
/TriDF { stroke [] 0 setdash vpt 1.12 mul sub M
  hpt neg vpt 1.62 mul V
  hpt 2 mul 0 V
  hpt neg vpt -1.62 mul V closepath fill} def
/DiaF { stroke [] 0 setdash vpt add M
  hpt neg vpt neg V hpt vpt neg V
  hpt vpt V hpt neg vpt V closepath fill } def
/Pent { stroke [] 0 setdash 2 copy gsave
  translate 0 hpt M 4 {72 rotate 0 hpt L} repeat
  closepath stroke grestore Pnt } def
/PentF { stroke [] 0 setdash gsave
  translate 0 hpt M 4 {72 rotate 0 hpt L} repeat
  closepath fill grestore } def
/Circle { stroke [] 0 setdash 2 copy
  hpt 0 360 arc stroke Pnt } def
/CircleF { stroke [] 0 setdash hpt 0 360 arc fill } def
/C0 { BL [] 0 setdash 2 copy moveto vpt 90 450  arc } bind def
/C1 { BL [] 0 setdash 2 copy        moveto
       2 copy  vpt 0 90 arc closepath fill
               vpt 0 360 arc closepath } bind def
/C2 { BL [] 0 setdash 2 copy moveto
       2 copy  vpt 90 180 arc closepath fill
               vpt 0 360 arc closepath } bind def
/C3 { BL [] 0 setdash 2 copy moveto
       2 copy  vpt 0 180 arc closepath fill
               vpt 0 360 arc closepath } bind def
/C4 { BL [] 0 setdash 2 copy moveto
       2 copy  vpt 180 270 arc closepath fill
               vpt 0 360 arc closepath } bind def
/C5 { BL [] 0 setdash 2 copy moveto
       2 copy  vpt 0 90 arc
       2 copy moveto
       2 copy  vpt 180 270 arc closepath fill
               vpt 0 360 arc } bind def
/C6 { BL [] 0 setdash 2 copy moveto
      2 copy  vpt 90 270 arc closepath fill
              vpt 0 360 arc closepath } bind def
/C7 { BL [] 0 setdash 2 copy moveto
      2 copy  vpt 0 270 arc closepath fill
              vpt 0 360 arc closepath } bind def
/C8 { BL [] 0 setdash 2 copy moveto
      2 copy vpt 270 360 arc closepath fill
              vpt 0 360 arc closepath } bind def
/C9 { BL [] 0 setdash 2 copy moveto
      2 copy  vpt 270 450 arc closepath fill
              vpt 0 360 arc closepath } bind def
/C10 { BL [] 0 setdash 2 copy 2 copy moveto vpt 270 360 arc closepath fill
       2 copy moveto
       2 copy vpt 90 180 arc closepath fill
               vpt 0 360 arc closepath } bind def
/C11 { BL [] 0 setdash 2 copy moveto
       2 copy  vpt 0 180 arc closepath fill
       2 copy moveto
       2 copy  vpt 270 360 arc closepath fill
               vpt 0 360 arc closepath } bind def
/C12 { BL [] 0 setdash 2 copy moveto
       2 copy  vpt 180 360 arc closepath fill
               vpt 0 360 arc closepath } bind def
/C13 { BL [] 0 setdash  2 copy moveto
       2 copy  vpt 0 90 arc closepath fill
       2 copy moveto
       2 copy  vpt 180 360 arc closepath fill
               vpt 0 360 arc closepath } bind def
/C14 { BL [] 0 setdash 2 copy moveto
       2 copy  vpt 90 360 arc closepath fill
               vpt 0 360 arc } bind def
/C15 { BL [] 0 setdash 2 copy vpt 0 360 arc closepath fill
               vpt 0 360 arc closepath } bind def
/Rec   { newpath 4 2 roll moveto 1 index 0 rlineto 0 exch rlineto
       neg 0 rlineto closepath } bind def
/Square { dup Rec } bind def
/Bsquare { vpt sub exch vpt sub exch vpt2 Square } bind def
/S0 { BL [] 0 setdash 2 copy moveto 0 vpt rlineto BL Bsquare } bind def
/S1 { BL [] 0 setdash 2 copy vpt Square fill Bsquare } bind def
/S2 { BL [] 0 setdash 2 copy exch vpt sub exch vpt Square fill Bsquare } bind def
/S3 { BL [] 0 setdash 2 copy exch vpt sub exch vpt2 vpt Rec fill Bsquare } bind def
/S4 { BL [] 0 setdash 2 copy exch vpt sub exch vpt sub vpt Square fill Bsquare } bind def
/S5 { BL [] 0 setdash 2 copy 2 copy vpt Square fill
       exch vpt sub exch vpt sub vpt Square fill Bsquare } bind def
/S6 { BL [] 0 setdash 2 copy exch vpt sub exch vpt sub vpt vpt2 Rec fill Bsquare } bind def
/S7 { BL [] 0 setdash 2 copy exch vpt sub exch vpt sub vpt vpt2 Rec fill
       2 copy vpt Square fill
       Bsquare } bind def
/S8 { BL [] 0 setdash 2 copy vpt sub vpt Square fill Bsquare } bind def
/S9 { BL [] 0 setdash 2 copy vpt sub vpt vpt2 Rec fill Bsquare } bind def
/S10 { BL [] 0 setdash 2 copy vpt sub vpt Square fill 2 copy exch vpt sub exch vpt Square fill
       Bsquare } bind def
/S11 { BL [] 0 setdash 2 copy vpt sub vpt Square fill 2 copy exch vpt sub exch vpt2 vpt Rec fill
       Bsquare } bind def
/S12 { BL [] 0 setdash 2 copy exch vpt sub exch vpt sub vpt2 vpt Rec fill Bsquare } bind def
/S13 { BL [] 0 setdash 2 copy exch vpt sub exch vpt sub vpt2 vpt Rec fill
       2 copy vpt Square fill Bsquare } bind def
/S14 { BL [] 0 setdash 2 copy exch vpt sub exch vpt sub vpt2 vpt Rec fill
       2 copy exch vpt sub exch vpt Square fill Bsquare } bind def
/S15 { BL [] 0 setdash 2 copy Bsquare fill Bsquare } bind def
/D0 { gsave translate 45 rotate 0 0 S0 stroke grestore } bind def
/D1 { gsave translate 45 rotate 0 0 S1 stroke grestore } bind def
/D2 { gsave translate 45 rotate 0 0 S2 stroke grestore } bind def
/D3 { gsave translate 45 rotate 0 0 S3 stroke grestore } bind def
/D4 { gsave translate 45 rotate 0 0 S4 stroke grestore } bind def
/D5 { gsave translate 45 rotate 0 0 S5 stroke grestore } bind def
/D6 { gsave translate 45 rotate 0 0 S6 stroke grestore } bind def
/D7 { gsave translate 45 rotate 0 0 S7 stroke grestore } bind def
/D8 { gsave translate 45 rotate 0 0 S8 stroke grestore } bind def
/D9 { gsave translate 45 rotate 0 0 S9 stroke grestore } bind def
/D10 { gsave translate 45 rotate 0 0 S10 stroke grestore } bind def
/D11 { gsave translate 45 rotate 0 0 S11 stroke grestore } bind def
/D12 { gsave translate 45 rotate 0 0 S12 stroke grestore } bind def
/D13 { gsave translate 45 rotate 0 0 S13 stroke grestore } bind def
/D14 { gsave translate 45 rotate 0 0 S14 stroke grestore } bind def
/D15 { gsave translate 45 rotate 0 0 S15 stroke grestore } bind def
/DiaE { stroke [] 0 setdash vpt add M
  hpt neg vpt neg V hpt vpt neg V
  hpt vpt V hpt neg vpt V closepath stroke } def
/BoxE { stroke [] 0 setdash exch hpt sub exch vpt add M
  0 vpt2 neg V hpt2 0 V 0 vpt2 V
  hpt2 neg 0 V closepath stroke } def
/TriUE { stroke [] 0 setdash vpt 1.12 mul add M
  hpt neg vpt -1.62 mul V
  hpt 2 mul 0 V
  hpt neg vpt 1.62 mul V closepath stroke } def
/TriDE { stroke [] 0 setdash vpt 1.12 mul sub M
  hpt neg vpt 1.62 mul V
  hpt 2 mul 0 V
  hpt neg vpt -1.62 mul V closepath stroke } def
/PentE { stroke [] 0 setdash gsave
  translate 0 hpt M 4 {72 rotate 0 hpt L} repeat
  closepath stroke grestore } def
/CircE { stroke [] 0 setdash 
  hpt 0 360 arc stroke } def
/Opaque { gsave closepath 1 setgray fill grestore 0 setgray closepath } def
/DiaW { stroke [] 0 setdash vpt add M
  hpt neg vpt neg V hpt vpt neg V
  hpt vpt V hpt neg vpt V Opaque stroke } def
/BoxW { stroke [] 0 setdash exch hpt sub exch vpt add M
  0 vpt2 neg V hpt2 0 V 0 vpt2 V
  hpt2 neg 0 V Opaque stroke } def
/TriUW { stroke [] 0 setdash vpt 1.12 mul add M
  hpt neg vpt -1.62 mul V
  hpt 2 mul 0 V
  hpt neg vpt 1.62 mul V Opaque stroke } def
/TriDW { stroke [] 0 setdash vpt 1.12 mul sub M
  hpt neg vpt 1.62 mul V
  hpt 2 mul 0 V
  hpt neg vpt -1.62 mul V Opaque stroke } def
/PentW { stroke [] 0 setdash gsave
  translate 0 hpt M 4 {72 rotate 0 hpt L} repeat
  Opaque stroke grestore } def
/CircW { stroke [] 0 setdash 
  hpt 0 360 arc Opaque stroke } def
/BoxFill { gsave Rec 1 setgray fill grestore } def
end
%%EndProlog
}}%
\begin{picture}(3600,2160)(0,0)%
{\GNUPLOTspecial{"
gnudict begin
gsave
0 0 translate
0.100 0.100 scale
0 setgray
newpath
1.000 UL
LTb
450 300 M
63 0 V
2937 0 R
-63 0 V
450 652 M
63 0 V
2937 0 R
-63 0 V
450 1004 M
63 0 V
2937 0 R
-63 0 V
450 1356 M
63 0 V
2937 0 R
-63 0 V
450 1708 M
63 0 V
2937 0 R
-63 0 V
450 2060 M
63 0 V
2937 0 R
-63 0 V
450 300 M
0 63 V
0 1697 R
0 -63 V
825 300 M
0 63 V
0 1697 R
0 -63 V
1200 300 M
0 63 V
0 1697 R
0 -63 V
1575 300 M
0 63 V
0 1697 R
0 -63 V
1950 300 M
0 63 V
0 1697 R
0 -63 V
2325 300 M
0 63 V
0 1697 R
0 -63 V
2700 300 M
0 63 V
0 1697 R
0 -63 V
3075 300 M
0 63 V
0 1697 R
0 -63 V
3450 300 M
0 63 V
0 1697 R
0 -63 V
1.000 UL
LTb
450 300 M
3000 0 V
0 1760 V
-3000 0 V
450 300 L
1.000 UL
LT0
3087 1947 M
263 0 V
650 300 M
0 6 V
1 21 V
1 20 V
0 21 V
1 20 V
1 20 V
1 20 V
0 19 V
1 20 V
1 19 V
1 19 V
0 19 V
1 18 V
1 18 V
1 19 V
0 17 V
1 18 V
1 18 V
1 17 V
0 17 V
1 17 V
1 17 V
1 17 V
0 16 V
1 16 V
1 16 V
1 16 V
0 16 V
1 16 V
1 15 V
1 15 V
0 15 V
1 15 V
1 15 V
1 14 V
0 15 V
1 14 V
1 14 V
1 14 V
0 14 V
1 13 V
1 14 V
1 13 V
0 13 V
1 13 V
1 13 V
1 13 V
0 12 V
1 13 V
1 12 V
1 12 V
0 12 V
1 12 V
1 12 V
1 12 V
0 11 V
1 12 V
1 11 V
1 11 V
0 11 V
1 11 V
1 11 V
1 10 V
0 11 V
1 10 V
1 11 V
1 10 V
0 10 V
1 10 V
1 10 V
1 9 V
0 10 V
1 10 V
1 9 V
1 9 V
0 9 V
1 10 V
1 9 V
1 8 V
0 9 V
1 9 V
1 8 V
1 9 V
0 8 V
1 9 V
1 8 V
1 8 V
0 8 V
1 8 V
1 8 V
1 7 V
0 8 V
1 8 V
1 7 V
1 7 V
0 8 V
1 7 V
1 7 V
1 7 V
0 7 V
1 7 V
1 7 V
1 7 V
0 6 V
1 7 V
1 6 V
1 7 V
0 6 V
1 6 V
1 6 V
1 7 V
0 6 V
1 6 V
1 6 V
1 5 V
0 6 V
1 6 V
1 5 V
1 6 V
0 6 V
1 5 V
1 5 V
1 6 V
0 5 V
1 5 V
1 5 V
1 5 V
0 5 V
1 5 V
1 5 V
1 5 V
0 5 V
1 4 V
1 5 V
1 4 V
0 5 V
1 4 V
1 5 V
1 4 V
0 4 V
1 5 V
1 4 V
1 4 V
0 4 V
1 4 V
1 4 V
1 4 V
0 4 V
1 4 V
1 4 V
1 3 V
0 4 V
1 4 V
1 3 V
1 4 V
0 3 V
1 4 V
1 3 V
1 4 V
0 3 V
1 3 V
1 4 V
1 3 V
0 3 V
1 3 V
1 3 V
1 3 V
0 3 V
1 3 V
1 3 V
1 3 V
0 3 V
1 3 V
1 2 V
1 3 V
0 3 V
1 2 V
1 3 V
1 3 V
0 2 V
1 3 V
1 2 V
1 3 V
0 2 V
1 2 V
1 3 V
1 2 V
0 2 V
1 3 V
1 2 V
1 2 V
0 2 V
1 2 V
1 2 V
1 2 V
0 2 V
1 2 V
1 2 V
1 2 V
0 2 V
1 2 V
1 2 V
1 2 V
0 2 V
1 1 V
1 2 V
1 2 V
0 2 V
1 1 V
1 2 V
1 1 V
0 2 V
1 2 V
1 1 V
1 2 V
0 1 V
1 2 V
1 1 V
1 2 V
0 1 V
1 1 V
1 2 V
1 1 V
0 1 V
1 2 V
1 1 V
1 1 V
0 1 V
1 2 V
1 1 V
1 1 V
0 1 V
1 1 V
1 1 V
1 1 V
0 1 V
1 1 V
1 2 V
1 1 V
0 1 V
1 1 V
1 0 V
1 1 V
0 1 V
1 1 V
1 1 V
1 1 V
0 1 V
1 1 V
1 1 V
1 0 V
0 1 V
1 1 V
1 1 V
1 0 V
0 1 V
1 1 V
1 1 V
1 0 V
0 1 V
1 1 V
1 0 V
1 1 V
1 1 V
1 1 V
1 0 V
0 1 V
1 0 V
1 1 V
1 0 V
0 1 V
1 0 V
1 1 V
1 0 V
0 1 V
1 0 V
1 1 V
1 0 V
1 1 V
1 0 V
1 1 V
1 0 V
1 1 V
1 0 V
1 1 V
1 0 V
1 0 V
1 1 V
1 0 V
1 0 V
1 1 V
1 0 V
1 0 V
1 1 V
1 0 V
1 0 V
1 0 V
1 0 V
1 1 V
1 0 V
1 0 V
1 0 V
1 0 V
1 0 V
1 0 V
0 1 V
1 0 V
1 0 V
1 0 V
1 0 V
1 0 V
1 0 V
1 0 V
1 0 V
1 0 V
1 0 V
1 0 V
1 0 V
1 0 V
1 0 V
1 0 V
1 0 V
1 0 V
1 0 V
1 -1 V
1 0 V
1 0 V
1 0 V
1 0 V
1 0 V
1 0 V
1 0 V
1 -1 V
1 0 V
1 0 V
1 0 V
1 0 V
1 0 V
1 -1 V
1 0 V
1 0 V
1 0 V
1 -1 V
1 0 V
1 0 V
1 0 V
1 -1 V
1 0 V
1 0 V
1 0 V
1 -1 V
1 0 V
1 0 V
1 0 V
0 -1 V
1 0 V
1 0 V
1 0 V
1 -1 V
1 0 V
1 0 V
1 -1 V
1 0 V
1 0 V
1 -1 V
1 0 V
1 0 V
1 -1 V
1 0 V
1 0 V
1 -1 V
1 0 V
1 0 V
1 -1 V
1 0 V
1 0 V
0 -1 V
1 0 V
1 0 V
1 0 V
0 -1 V
1 0 V
1 0 V
1 -1 V
1 0 V
1 0 V
1 -1 V
1 0 V
1 -1 V
1 0 V
1 0 V
1 -1 V
1 0 V
1 -1 V
1 0 V
1 0 V
1 -1 V
currentpoint stroke M
1 0 V
1 0 V
0 -1 V
1 0 V
1 0 V
1 -1 V
1 0 V
1 0 V
1 -1 V
1 0 V
1 -1 V
1 0 V
1 -1 V
1 0 V
1 0 V
1 -1 V
1 0 V
1 0 V
0 -1 V
1 0 V
1 0 V
1 -1 V
1 0 V
1 0 V
1 -1 V
1 0 V
1 -1 V
1 0 V
1 -1 V
1 0 V
1 0 V
0 -1 V
1 0 V
1 0 V
1 0 V
0 -1 V
1 0 V
1 0 V
1 -1 V
1 0 V
1 -1 V
1 0 V
1 0 V
1 -1 V
1 0 V
1 -1 V
1 0 V
1 0 V
1 -1 V
1 0 V
1 0 V
0 -1 V
1 0 V
1 0 V
1 0 V
0 -1 V
1 0 V
1 0 V
1 -1 V
1 0 V
1 0 V
1 -1 V
1 0 V
1 -1 V
1 0 V
1 0 V
1 -1 V
1 0 V
1 -1 V
1 0 V
1 0 V
1 -1 V
1 0 V
1 0 V
1 -1 V
1 0 V
1 0 V
1 -1 V
1 0 V
1 0 V
0 -1 V
1 0 V
1 0 V
1 0 V
0 -1 V
1 0 V
1 0 V
1 0 V
0 -1 V
1 0 V
1 0 V
1 0 V
0 -1 V
1 0 V
1 0 V
1 0 V
0 -1 V
1 0 V
1 0 V
1 0 V
1 -1 V
1 0 V
1 0 V
1 -1 V
1 0 V
1 0 V
1 -1 V
1 0 V
1 0 V
1 0 V
1 -1 V
1 0 V
1 0 V
1 -1 V
1 0 V
1 0 V
1 0 V
1 -1 V
1 0 V
1 0 V
1 0 V
0 -1 V
1 0 V
1 0 V
1 0 V
1 -1 V
1 0 V
1 0 V
1 0 V
1 -1 V
1 0 V
1 0 V
1 0 V
1 -1 V
1 0 V
1 0 V
1 0 V
1 -1 V
1 0 V
1 0 V
1 0 V
1 -1 V
1 0 V
1 0 V
1 0 V
1 0 V
0 -1 V
1 0 V
1 0 V
1 0 V
1 0 V
1 -1 V
1 0 V
1 0 V
1 0 V
1 0 V
1 -1 V
1 0 V
1 0 V
1 0 V
1 0 V
1 0 V
0 -1 V
1 0 V
1 0 V
1 0 V
1 0 V
1 0 V
1 -1 V
1 0 V
1 0 V
1 0 V
1 0 V
1 0 V
1 -1 V
1 0 V
1 0 V
1 0 V
1 0 V
1 0 V
1 0 V
0 -1 V
1 0 V
1 0 V
1 0 V
1 0 V
1 0 V
1 0 V
1 -1 V
1 0 V
1 0 V
1 0 V
1 0 V
1 0 V
1 0 V
1 0 V
1 0 V
0 -1 V
1 0 V
1 0 V
1 0 V
1 0 V
1 0 V
1 0 V
1 0 V
1 0 V
1 0 V
1 -1 V
1 0 V
1 0 V
1 0 V
1 0 V
1 0 V
1 0 V
1 0 V
1 0 V
1 0 V
1 0 V
1 0 V
1 -1 V
1 0 V
1 0 V
1 0 V
1 0 V
1 0 V
1 0 V
1 0 V
1 0 V
1 0 V
1 0 V
1 0 V
1 0 V
1 0 V
1 0 V
1 0 V
1 0 V
1 0 V
1 0 V
1 0 V
1 -1 V
1 0 V
1 0 V
1 0 V
1 0 V
1 0 V
1 0 V
1 0 V
1 0 V
1 0 V
1 0 V
1 0 V
1 0 V
1 0 V
1 0 V
1 0 V
1 0 V
1 0 V
1 0 V
1 0 V
1 0 V
1 0 V
1 0 V
1 0 V
1 0 V
1 0 V
1 0 V
1 0 V
1 0 V
1 0 V
1 0 V
1 0 V
1 0 V
1 0 V
1 0 V
1 0 V
1 0 V
1 0 V
1 0 V
1 0 V
0 1 V
1 0 V
1 0 V
1 0 V
1 0 V
1 0 V
1 0 V
1 0 V
1 0 V
1 0 V
1 0 V
1 0 V
1 0 V
1 0 V
1 0 V
1 0 V
1 0 V
1 0 V
1 0 V
1 0 V
1 0 V
1 0 V
0 1 V
1 0 V
1 0 V
1 0 V
1 0 V
1 0 V
1 0 V
1 0 V
1 0 V
1 0 V
1 0 V
1 0 V
1 0 V
1 0 V
1 0 V
1 1 V
1 0 V
1 0 V
1 0 V
1 0 V
1 0 V
1 0 V
1 0 V
1 0 V
1 0 V
1 0 V
1 0 V
1 1 V
1 0 V
1 0 V
1 0 V
1 0 V
1 0 V
1 0 V
1 0 V
1 0 V
1 0 V
1 1 V
1 0 V
1 0 V
1 0 V
1 0 V
1 0 V
1 0 V
1 0 V
1 0 V
1 1 V
1 0 V
1 0 V
1 0 V
1 0 V
1 0 V
1 0 V
1 0 V
1 0 V
1 1 V
1 0 V
1 0 V
1 0 V
1 0 V
1 0 V
1 0 V
1 0 V
1 1 V
1 0 V
1 0 V
1 0 V
1 0 V
1 0 V
1 0 V
1 0 V
1 1 V
1 0 V
1 0 V
1 0 V
1 0 V
1 0 V
1 0 V
1 0 V
0 1 V
1 0 V
1 0 V
1 0 V
1 0 V
1 0 V
1 0 V
1 1 V
1 0 V
1 0 V
1 0 V
1 0 V
1 0 V
1 0 V
1 0 V
1 1 V
1 0 V
1 0 V
1 0 V
1 0 V
currentpoint stroke M
1 0 V
1 0 V
0 1 V
1 0 V
1 0 V
1 0 V
1 0 V
1 0 V
1 0 V
0 1 V
1 0 V
1 0 V
1 0 V
1 0 V
1 0 V
1 0 V
1 1 V
1 0 V
1 0 V
1 0 V
1 0 V
1 0 V
1 1 V
1 0 V
1 0 V
1 0 V
1 0 V
1 0 V
1 0 V
1 1 V
1 0 V
1 0 V
1 0 V
1 0 V
1 0 V
1 1 V
1 0 V
1 0 V
1 0 V
1 0 V
1 0 V
1 1 V
1 0 V
1 0 V
1 0 V
1 0 V
1 0 V
1 1 V
1 0 V
1 0 V
1 0 V
1 0 V
1 0 V
1 1 V
1 0 V
1 0 V
1 0 V
1 0 V
1 0 V
1 1 V
1 0 V
1 0 V
1 0 V
1 0 V
1 0 V
1 1 V
1 0 V
1 0 V
1 0 V
1 0 V
1 0 V
1 1 V
1 0 V
1 0 V
1 0 V
1 0 V
1 1 V
1 0 V
1 0 V
1 0 V
1 0 V
1 0 V
1 1 V
1 0 V
1 0 V
1 0 V
1 0 V
1 0 V
1 1 V
1 0 V
1 0 V
1 0 V
1 0 V
1 0 V
0 1 V
1 0 V
1 0 V
1 0 V
1 0 V
1 0 V
1 0 V
0 1 V
1 0 V
1 0 V
1 0 V
1 0 V
1 0 V
1 0 V
0 1 V
1 0 V
1 0 V
1 0 V
1 0 V
1 0 V
1 1 V
1 0 V
1 0 V
1 0 V
1 0 V
1 0 V
1 1 V
1 0 V
1 0 V
1 0 V
1 0 V
1 0 V
1 1 V
1 0 V
1 0 V
1 0 V
1 0 V
1 0 V
1 1 V
1 0 V
1 0 V
1 0 V
1 0 V
1 1 V
1 0 V
1 0 V
1 0 V
1 0 V
1 0 V
1 1 V
1 0 V
1 0 V
1 0 V
1 0 V
1 0 V
1 1 V
1 0 V
1 0 V
1 0 V
1 0 V
1 0 V
1 1 V
1 0 V
1 0 V
1 0 V
1 0 V
1 0 V
1 1 V
1 0 V
1 0 V
1 0 V
1 0 V
1 0 V
1 1 V
1 0 V
1 0 V
1 0 V
1 0 V
1 0 V
1 1 V
1 0 V
1 0 V
1 0 V
1 0 V
1 0 V
1 1 V
1 0 V
1 0 V
1 0 V
1 0 V
1 0 V
1 1 V
1 0 V
1 0 V
1 0 V
1 0 V
1 0 V
1 1 V
1 0 V
1 0 V
1 0 V
1 0 V
1 0 V
1 0 V
1 1 V
1 0 V
1 0 V
1 0 V
1 0 V
1 0 V
1 1 V
1 0 V
1 0 V
1 0 V
1 0 V
1 0 V
1 0 V
0 1 V
1 0 V
1 0 V
1 0 V
1 0 V
1 0 V
1 0 V
0 1 V
1 0 V
1 0 V
1 0 V
1 0 V
1 0 V
1 0 V
1 1 V
1 0 V
1 0 V
1 0 V
1 0 V
1 0 V
1 0 V
1 1 V
1 0 V
1 0 V
1 0 V
1 0 V
1 0 V
1 0 V
1 1 V
1 0 V
1 0 V
1 0 V
1 0 V
1 0 V
1 0 V
0 1 V
1 0 V
1 0 V
1 0 V
1 0 V
1 0 V
1 0 V
1 1 V
1 0 V
1 0 V
1 0 V
1 0 V
1 0 V
1 0 V
1 0 V
1 1 V
1 0 V
1 0 V
1 0 V
1 0 V
1 0 V
1 0 V
1 1 V
1 0 V
1 0 V
1 0 V
1 0 V
1 0 V
1 0 V
1 0 V
1 1 V
1 0 V
1 0 V
1 0 V
1 0 V
1 0 V
1 0 V
1 1 V
1 0 V
1 0 V
1 0 V
1 0 V
1 0 V
1 0 V
1 0 V
1 1 V
1 0 V
1 0 V
1 0 V
1 0 V
1 0 V
1 0 V
1 0 V
1 1 V
1 0 V
1 0 V
1 0 V
1 0 V
1 0 V
1 0 V
1 0 V
1 1 V
1 0 V
1 0 V
1 0 V
1 0 V
1 0 V
1 0 V
1 0 V
1 0 V
0 1 V
1 0 V
1 0 V
1 0 V
1 0 V
1 0 V
1 0 V
1 0 V
1 0 V
1 0 V
0 1 V
1 0 V
1 0 V
1 0 V
1 0 V
1 0 V
1 0 V
1 0 V
1 0 V
1 0 V
0 1 V
1 0 V
1 0 V
1 0 V
1 0 V
1 0 V
1 0 V
1 0 V
1 0 V
1 0 V
0 1 V
1 0 V
1 0 V
1 0 V
1 0 V
1 0 V
1 0 V
1 0 V
1 0 V
1 0 V
1 1 V
1 0 V
1 0 V
1 0 V
1 0 V
1 0 V
1 0 V
1 0 V
1 0 V
1 0 V
1 0 V
1 1 V
1 0 V
1 0 V
1 0 V
1 0 V
1 0 V
1 0 V
1 0 V
1 0 V
1 0 V
1 1 V
1 0 V
1 0 V
1 0 V
1 0 V
1 0 V
1 0 V
1 0 V
1 0 V
1 0 V
1 0 V
1 0 V
0 1 V
1 0 V
1 0 V
1 0 V
1 0 V
1 0 V
1 0 V
1 0 V
1 0 V
1 0 V
1 0 V
currentpoint stroke M
1 0 V
1 1 V
1 0 V
1 0 V
1 0 V
1 0 V
1 0 V
1 0 V
1 0 V
1 0 V
1 0 V
1 0 V
1 0 V
1 0 V
0 1 V
1 0 V
1 0 V
1 0 V
1 0 V
1 0 V
1 0 V
1 0 V
1 0 V
1 0 V
1 0 V
1 0 V
1 0 V
1 0 V
1 1 V
1 0 V
1 0 V
1 0 V
1 0 V
1 0 V
1 0 V
1 0 V
1 0 V
1 0 V
1 0 V
1 0 V
1 0 V
1 0 V
1 1 V
1 0 V
1 0 V
1 0 V
1 0 V
1 0 V
1 0 V
1 0 V
1 0 V
1 0 V
1 0 V
1 0 V
1 0 V
1 0 V
1 0 V
1 0 V
1 1 V
1 0 V
1 0 V
1 0 V
1 0 V
1 0 V
1 0 V
1 0 V
1 0 V
1 0 V
1 0 V
1 0 V
1 0 V
1 0 V
1 0 V
1 0 V
1 0 V
1 0 V
1 1 V
1 0 V
1 0 V
1 0 V
1 0 V
1 0 V
1 0 V
1 0 V
1 0 V
1 0 V
1 0 V
1 0 V
1 0 V
1 0 V
1 0 V
1 0 V
1 0 V
1 0 V
1 0 V
1 0 V
1 1 V
1 0 V
1 0 V
1 0 V
1 0 V
1 0 V
1 0 V
1 0 V
1 0 V
1 0 V
1 0 V
1 0 V
1 0 V
1 0 V
1 0 V
1 0 V
1 0 V
1 0 V
1 0 V
1 0 V
1 0 V
1 0 V
1 0 V
1 0 V
0 1 V
1 0 V
1 0 V
1 0 V
1 0 V
1 0 V
1 0 V
1 0 V
1 0 V
1 0 V
1 0 V
1 0 V
1 0 V
1 0 V
1 0 V
1 0 V
1 0 V
1 0 V
1 0 V
1 0 V
1 0 V
1 0 V
1 0 V
1 0 V
1 0 V
1 0 V
1 0 V
1 0 V
1 0 V
1 0 V
1 0 V
0 1 V
1 0 V
1 0 V
1 0 V
1 0 V
1 0 V
1 0 V
1 0 V
1 0 V
1 0 V
1 0 V
1 0 V
1 0 V
1 0 V
1 0 V
1 0 V
1 0 V
1 0 V
1 0 V
1 0 V
1 0 V
1 0 V
1 0 V
1 0 V
1 0 V
1 0 V
1 0 V
1 0 V
1 0 V
1 0 V
1 0 V
1 0 V
1 0 V
1 0 V
1 0 V
1 0 V
1 0 V
1 0 V
1 0 V
1 0 V
1 0 V
1 0 V
1 0 V
1 0 V
1 0 V
1 0 V
1 0 V
1 1 V
1 0 V
1 0 V
1 0 V
1 0 V
1 0 V
1 0 V
1 0 V
1 0 V
1 0 V
1 0 V
1 0 V
1 0 V
1 0 V
1 0 V
1 0 V
1 0 V
1 0 V
1 0 V
1 0 V
1 0 V
1 0 V
1 0 V
1 0 V
1 0 V
1 0 V
1 0 V
1 0 V
1 0 V
1 0 V
1 0 V
1 0 V
1 0 V
1 0 V
1 0 V
1 0 V
1 0 V
1 0 V
1 0 V
1 0 V
1 0 V
1 0 V
1 0 V
1 0 V
1 0 V
1 0 V
1 0 V
1 0 V
1 0 V
1 0 V
1 0 V
1 0 V
1 0 V
1 0 V
1 0 V
1 0 V
1 0 V
1 0 V
1 0 V
1 0 V
1 0 V
1 0 V
1 0 V
1 0 V
1 0 V
1 0 V
1 0 V
1 0 V
1 0 V
1 0 V
1 0 V
1 0 V
1 0 V
1 0 V
1 0 V
1 0 V
1 0 V
1 0 V
1 0 V
1 0 V
1 0 V
1 0 V
1 0 V
1 0 V
1 0 V
1 0 V
1 0 V
1 0 V
1 0 V
1 0 V
1 0 V
1 0 V
1 0 V
1 0 V
1 0 V
1 0 V
1 0 V
1 0 V
1 0 V
1 0 V
1 0 V
1 0 V
1 0 V
1 0 V
1 0 V
1 0 V
1 0 V
1 0 V
1 0 V
1 0 V
1 0 V
1 0 V
1 0 V
1 0 V
1 0 V
1 0 V
1 0 V
1 0 V
1 0 V
1 0 V
1 0 V
1 0 V
1 0 V
1 0 V
1 0 V
1 0 V
1 0 V
1 0 V
1 0 V
1 0 V
1 0 V
1 0 V
1 0 V
1 0 V
1 0 V
1 0 V
1 0 V
1 0 V
1 0 V
1 0 V
1 0 V
1 0 V
1 0 V
1 0 V
1 0 V
1 -1 V
1 0 V
1 0 V
1 0 V
1 0 V
1 0 V
1 0 V
1 0 V
1 0 V
1 0 V
1 0 V
1 0 V
1 0 V
1 0 V
1 0 V
1 0 V
1 0 V
1 0 V
1 0 V
1 0 V
1 0 V
1 0 V
1 0 V
1 0 V
1 0 V
1 0 V
1 0 V
1 0 V
1 0 V
1 0 V
1 0 V
1 0 V
1 0 V
1 0 V
1 0 V
1 0 V
1 0 V
1 0 V
1 0 V
1 0 V
1 0 V
1 0 V
1 0 V
1 0 V
1 0 V
1 0 V
1 0 V
1 0 V
1 0 V
1 0 V
1 0 V
1 0 V
1 0 V
1 0 V
1 0 V
1 0 V
1 0 V
currentpoint stroke M
1 0 V
1 0 V
1 0 V
1 0 V
1 0 V
1 0 V
1 0 V
1 -1 V
1 0 V
1 0 V
1 0 V
1 0 V
1 0 V
1 0 V
1 0 V
1 0 V
1 0 V
1 0 V
1 0 V
1 0 V
1 0 V
1 0 V
1 0 V
1 0 V
1 0 V
1 0 V
1 0 V
1 0 V
1 0 V
1 0 V
1 0 V
1 0 V
1 0 V
1 0 V
1 0 V
1 0 V
1 0 V
1 0 V
1 0 V
1 0 V
1 0 V
1 0 V
1 0 V
1 0 V
1 0 V
1 0 V
1 0 V
1 0 V
1 0 V
1 0 V
1 0 V
1 0 V
1 0 V
1 0 V
1 0 V
1 0 V
1 0 V
1 0 V
0 -1 V
1 0 V
1 0 V
1 0 V
1 0 V
1 0 V
1 0 V
1 0 V
1 0 V
1 0 V
1 0 V
1 0 V
1 0 V
1 0 V
1 0 V
1 0 V
1 0 V
1 0 V
1 0 V
1 0 V
1 0 V
1 0 V
1 0 V
1 0 V
1 0 V
1 0 V
1 0 V
1 0 V
1 0 V
1 0 V
1 0 V
1 0 V
1 0 V
1 0 V
1 0 V
1 0 V
1 0 V
1 0 V
1 0 V
1 0 V
1 0 V
1 0 V
1 0 V
1 0 V
1 0 V
1 0 V
1 -1 V
1 0 V
1 0 V
1 0 V
1 0 V
1 0 V
1 0 V
1 0 V
1 0 V
1 0 V
1 0 V
1 0 V
1 0 V
1 0 V
1 0 V
1 0 V
1 0 V
1 0 V
1 0 V
1 0 V
1 0 V
1 0 V
1 0 V
1 0 V
1 0 V
1 0 V
1 0 V
1 0 V
1 0 V
1 0 V
1 0 V
1 0 V
1 0 V
1 0 V
1 0 V
1 0 V
1 0 V
1 0 V
1 0 V
1 0 V
1 0 V
1 0 V
1 0 V
1 0 V
1 0 V
1 -1 V
1 0 V
1 0 V
1 0 V
1 0 V
1 0 V
1 0 V
1 0 V
1 0 V
1 0 V
1 0 V
1 0 V
1 0 V
1 0 V
1 0 V
1 0 V
1 0 V
1 0 V
1 0 V
1 0 V
1 0 V
1 0 V
1 0 V
1 0 V
1 0 V
1 0 V
1 0 V
1 0 V
1 0 V
1 0 V
1 0 V
1 0 V
1 0 V
1 0 V
1 0 V
1 0 V
1 0 V
1 0 V
1 0 V
1 0 V
1 0 V
1 0 V
1 0 V
1 0 V
1 0 V
1 0 V
1 0 V
1 -1 V
1 0 V
1 0 V
1 0 V
1 0 V
1 0 V
1 0 V
1 0 V
1 0 V
1 0 V
1 0 V
1 0 V
1 0 V
1 0 V
1 0 V
1 0 V
1 0 V
1 0 V
1 0 V
1 0 V
1 0 V
1 0 V
1 0 V
1 0 V
1 0 V
1 0 V
1 0 V
1 0 V
1 0 V
1 0 V
1 0 V
1 0 V
1 0 V
1 0 V
1 0 V
1 0 V
1 0 V
1 0 V
1 0 V
1 0 V
1 0 V
1 0 V
1 0 V
1 0 V
1 0 V
1 0 V
1 0 V
1 0 V
1 0 V
1 -1 V
1 0 V
1 0 V
1 0 V
1 0 V
1 0 V
1 0 V
1 0 V
1 0 V
1 0 V
1 0 V
1 0 V
1 0 V
1 0 V
1 0 V
1 0 V
1 0 V
1 0 V
1 0 V
1 0 V
1 0 V
1 0 V
1 0 V
1 0 V
1 0 V
1 0 V
1 0 V
1 0 V
1 0 V
1 0 V
1 0 V
1 0 V
1 0 V
1 0 V
1 0 V
1 0 V
1 0 V
1 0 V
1 0 V
1 0 V
1 0 V
1 0 V
1 0 V
1 0 V
1 0 V
1 0 V
1 0 V
1 0 V
1 0 V
1 0 V
1 0 V
1 0 V
1 0 V
1 0 V
1 0 V
1 0 V
1 0 V
0 -1 V
1 0 V
1 0 V
1 0 V
1 0 V
1 0 V
1 0 V
1 0 V
1 0 V
1 0 V
1 0 V
1 0 V
1 0 V
1 0 V
1 0 V
1 0 V
1 0 V
1 0 V
1 0 V
1 0 V
1 0 V
1 0 V
1 0 V
1 0 V
1 0 V
1 0 V
1 0 V
1 0 V
1 0 V
1 0 V
1 0 V
1 0 V
1 0 V
1 0 V
1 0 V
1 0 V
1 0 V
1 0 V
1 0 V
1 0 V
1 0 V
1 0 V
1 0 V
1 0 V
1 0 V
1 0 V
1 0 V
1 0 V
1 0 V
1 0 V
1 0 V
1 0 V
1 0 V
1 0 V
1 0 V
1 0 V
1 0 V
1 0 V
1 0 V
1 0 V
1 0 V
1 0 V
1 0 V
1 0 V
1 0 V
1 0 V
1 0 V
1 0 V
1 0 V
1 0 V
1 0 V
1 0 V
1 -1 V
1 0 V
1 0 V
1 0 V
1 0 V
1 0 V
1 0 V
1 0 V
1 0 V
1 0 V
1 0 V
1 0 V
1 0 V
1 0 V
1 0 V
1 0 V
1 0 V
1 0 V
1 0 V
1 0 V
1 0 V
1 0 V
1 0 V
1 0 V
1 0 V
1 0 V
currentpoint stroke M
1 0 V
1 0 V
1 0 V
1 0 V
1 0 V
1 0 V
1 0 V
1 0 V
1 0 V
1 0 V
1 0 V
1 0 V
1 0 V
1 0 V
1 0 V
1 0 V
1 0 V
1 0 V
1 0 V
1 0 V
1 0 V
1 0 V
1 0 V
1 0 V
1 0 V
1 0 V
1 0 V
1 0 V
1 0 V
1 0 V
1 0 V
1 0 V
1 0 V
1 0 V
1 0 V
1 0 V
1 0 V
1 0 V
1 0 V
1 0 V
1 0 V
1 0 V
1 0 V
1 0 V
1 0 V
1 0 V
1 0 V
1 0 V
1 0 V
1 0 V
1 0 V
1 0 V
1 0 V
1 0 V
1 0 V
1 0 V
1 0 V
1 0 V
1 0 V
1 0 V
1 0 V
1 0 V
1 0 V
1 0 V
1 0 V
1 0 V
1 0 V
1 0 V
1 0 V
1 0 V
1 0 V
1 0 V
1 0 V
1 0 V
1 0 V
1 0 V
1 0 V
1 0 V
1 0 V
1 0 V
1 0 V
1 0 V
1 0 V
1 0 V
1 0 V
1 0 V
1 0 V
1 0 V
1 0 V
1 0 V
1 0 V
1 0 V
1 0 V
1 0 V
1 0 V
1 0 V
1 0 V
1 0 V
1 0 V
1 0 V
1 0 V
1 0 V
1 0 V
1 0 V
1 0 V
1 0 V
1 0 V
1 0 V
1 0 V
1 0 V
1 0 V
1 0 V
1 0 V
1 -1 V
1 0 V
1 0 V
1 0 V
1 0 V
1 0 V
1 0 V
1 0 V
1 0 V
1 0 V
1 0 V
1 0 V
1 0 V
1 0 V
1 0 V
1 0 V
1 0 V
1 0 V
1 0 V
1 0 V
1 0 V
1 0 V
1 0 V
1 0 V
1 0 V
1 0 V
1 0 V
1 0 V
1 0 V
1 0 V
1 0 V
1 0 V
1 0 V
1 0 V
1 0 V
1 0 V
1 0 V
1 0 V
1 0 V
1 0 V
1 0 V
1 0 V
1 0 V
1 0 V
1 0 V
1 0 V
1 0 V
1 0 V
1 0 V
1 0 V
1 0 V
1 0 V
1 0 V
1 0 V
1 0 V
1 0 V
1 0 V
1 0 V
1 0 V
1 0 V
1 0 V
1 0 V
1 0 V
1 0 V
1 0 V
1 0 V
1 0 V
1 0 V
1 0 V
1 0 V
1 0 V
1 0 V
1 0 V
1 0 V
1 0 V
1 0 V
1 0 V
1 0 V
1 0 V
1 0 V
1 0 V
1 0 V
1 0 V
1 1 V
1 0 V
1 0 V
1 0 V
1 0 V
1 0 V
1 0 V
1 0 V
1 0 V
1 0 V
1 0 V
1 0 V
1 0 V
1 0 V
1 0 V
1 0 V
1 0 V
1 0 V
1 0 V
1 0 V
1 0 V
1 0 V
1 0 V
1 0 V
1 0 V
1 0 V
1 0 V
1 0 V
1 0 V
1 0 V
1 0 V
1 0 V
1 0 V
1 0 V
1 0 V
1 0 V
1 0 V
1 0 V
1 0 V
1 0 V
1 0 V
1 0 V
1 0 V
1 0 V
1 0 V
1 0 V
1 0 V
1 0 V
1 0 V
1 0 V
1 0 V
1 0 V
1 0 V
1 0 V
1 0 V
1 0 V
1 0 V
1 0 V
1 0 V
1 0 V
1 0 V
1 0 V
1 0 V
1 0 V
1 0 V
1 0 V
1 0 V
1 0 V
1 0 V
1 0 V
1 0 V
1 0 V
1 0 V
1 0 V
1 0 V
1 0 V
1 0 V
1 0 V
1 0 V
1 0 V
1 0 V
1 0 V
1 0 V
1 0 V
1 0 V
1 0 V
1 0 V
1 0 V
1 0 V
1 0 V
1 0 V
1 0 V
1 0 V
1 0 V
1 0 V
1 0 V
1 0 V
1 0 V
1 0 V
1 0 V
1 0 V
1 0 V
1 0 V
1 0 V
1 0 V
1 0 V
1 0 V
1 0 V
1 0 V
1 0 V
1 0 V
1 0 V
1 0 V
1 0 V
1 0 V
1 0 V
1 0 V
1 0 V
1 0 V
1 0 V
1 0 V
1 0 V
1 0 V
1 0 V
1 0 V
1 0 V
1 0 V
1 0 V
1 0 V
1 0 V
1 0 V
1 0 V
1 0 V
1 0 V
1 0 V
1 0 V
1 0 V
1 0 V
1 0 V
1 0 V
1 0 V
1 0 V
1 0 V
1 0 V
1 0 V
1 0 V
1 0 V
1 0 V
1 0 V
1 0 V
1 0 V
1 0 V
1 0 V
1 0 V
1 0 V
1 0 V
1 0 V
1 1 V
1 0 V
1 0 V
1 0 V
1 0 V
1 0 V
1 0 V
1 0 V
1 0 V
1 0 V
1 0 V
1 0 V
1 0 V
1 0 V
1 0 V
1 0 V
1 0 V
1 0 V
1 0 V
1 0 V
1 0 V
1 0 V
1 0 V
1 0 V
1 0 V
1 0 V
1 0 V
1 0 V
1 0 V
1 0 V
1 0 V
1 0 V
1 0 V
1 0 V
1 0 V
1 0 V
1 0 V
1 0 V
1 0 V
1 0 V
1 0 V
1 0 V
1 0 V
1 0 V
1 0 V
1 0 V
1 0 V
currentpoint stroke M
1 0 V
1 0 V
1 0 V
1 0 V
1 0 V
1 0 V
1 0 V
1 0 V
1 0 V
1 0 V
1 0 V
1 0 V
1 0 V
1 0 V
1 0 V
1 0 V
1 0 V
1 0 V
1 0 V
1 0 V
1 0 V
1 0 V
1 0 V
1 0 V
1 0 V
1 0 V
1 0 V
1 0 V
1 0 V
1 0 V
1 0 V
1 0 V
1 0 V
1 0 V
1 0 V
1 0 V
1 0 V
1 0 V
1 0 V
1 0 V
1 0 V
1 0 V
1 0 V
1 0 V
1 0 V
1 1 V
1 0 V
1 0 V
1 0 V
1 0 V
1 0 V
1 0 V
1 0 V
1 0 V
1 0 V
1 0 V
1 0 V
1 0 V
1 0 V
1 0 V
1 0 V
1 0 V
1 0 V
1 0 V
1 0 V
1 0 V
1 0 V
1 0 V
1 0 V
1 0 V
1 0 V
1 0 V
1 0 V
1 0 V
1 0 V
1 0 V
1 0 V
1 0 V
1 0 V
1 0 V
1 0 V
1 0 V
1 0 V
1 0 V
1 0 V
1 0 V
1 0 V
1 0 V
1 0 V
1 0 V
1 0 V
1 0 V
1 0 V
1 0 V
1 0 V
1 0 V
1 0 V
1 0 V
1 0 V
1 0 V
1 0 V
1 0 V
1 0 V
1 0 V
1 0 V
1 0 V
1 0 V
1 0 V
1 0 V
1 0 V
1 0 V
1 0 V
1 0 V
1 0 V
1 0 V
1 0 V
1 0 V
1 0 V
1 0 V
1 0 V
1 0 V
1 0 V
1 0 V
1 0 V
0 1 V
1 0 V
1 0 V
1 0 V
1 0 V
1 0 V
1 0 V
1 0 V
1 0 V
1 0 V
1 0 V
1 0 V
1 0 V
1 0 V
1 0 V
1 0 V
1 0 V
1 0 V
1 0 V
1 0 V
1 0 V
1 0 V
1 0 V
1 0 V
1 0 V
1 0 V
1 0 V
1 0 V
1 0 V
1 0 V
1 0 V
1 0 V
1 0 V
1 0 V
1 0 V
1 0 V
1 0 V
1 0 V
1 0 V
1 0 V
1 0 V
1 0 V
1 0 V
1 0 V
1 0 V
1 0 V
1 0 V
1 0 V
1 0 V
1 0 V
1 0 V
1 0 V
1 0 V
1 0 V
1 0 V
1 0 V
1 0 V
1 0 V
1 0 V
1 0 V
1 0 V
1 0 V
1 0 V
1 0 V
1 0 V
1 0 V
1 0 V
1 0 V
1 0 V
1 0 V
1 0 V
1 0 V
1 0 V
1 0 V
1 0 V
1 1 V
1 0 V
1 0 V
1 0 V
1 0 V
1 0 V
1 0 V
1 0 V
1 0 V
1 0 V
1 0 V
1 0 V
1 0 V
1 0 V
1 0 V
1 0 V
1 0 V
1 0 V
1 0 V
1 0 V
1 0 V
1 0 V
1 0 V
1 0 V
1 0 V
1 0 V
1 0 V
1 0 V
1 0 V
1 0 V
1 0 V
1 0 V
1 0 V
1 0 V
1 0 V
1 0 V
1 0 V
1 0 V
1 0 V
1 0 V
1 0 V
1 0 V
1 0 V
1 0 V
1 0 V
1 0 V
1 0 V
1 0 V
1 0 V
1 0 V
1 0 V
1 0 V
1 0 V
1 0 V
1 0 V
1 0 V
1 0 V
1 0 V
1 0 V
1 0 V
1 0 V
1 0 V
1 0 V
1 0 V
1 0 V
1 0 V
1 0 V
1 0 V
1 0 V
1 0 V
1 0 V
1 0 V
1 0 V
1 0 V
1 0 V
1 1 V
1 0 V
1 0 V
1 0 V
1 0 V
1 0 V
1 0 V
1 0 V
1 0 V
1 0 V
1 0 V
1 0 V
1 0 V
1 0 V
1 0 V
1 0 V
1 0 V
1 0 V
1 0 V
1 0 V
1 0 V
1 0 V
1 0 V
1 0 V
1 0 V
1 0 V
1 0 V
1 0 V
1 0 V
1 0 V
1 0 V
1 0 V
1 0 V
1 0 V
1 0 V
1 0 V
1 0 V
1 0 V
1 0 V
1 0 V
1 0 V
1 0 V
1 0 V
1 0 V
1 0 V
1 0 V
1 0 V
1 0 V
1 0 V
1 0 V
1 0 V
1 0 V
1 0 V
1 0 V
1 0 V
1 0 V
1 0 V
1 0 V
1 0 V
1 0 V
1 0 V
1 0 V
1 0 V
1 0 V
1 0 V
1 0 V
1 0 V
1 0 V
1 0 V
1 0 V
1 0 V
1 0 V
1 0 V
1 0 V
1 0 V
1 0 V
1 0 V
1 0 V
1 1 V
1 0 V
1 0 V
1 0 V
1 0 V
1 0 V
1 0 V
1 0 V
1 0 V
1 0 V
1 0 V
1 0 V
1 0 V
1 0 V
1 0 V
1 0 V
1 0 V
1 0 V
1 0 V
1 0 V
1 0 V
1 0 V
1 0 V
1 0 V
1 0 V
1 0 V
1 0 V
1 0 V
1 0 V
1 0 V
1 0 V
1 0 V
1 0 V
1 0 V
1 0 V
1 0 V
1 0 V
1 0 V
1 0 V
1 0 V
1 0 V
1 0 V
1 0 V
1 0 V
1 0 V
1 0 V
1 0 V
1 0 V
currentpoint stroke M
1 0 V
1 0 V
1 0 V
1 0 V
1 0 V
1 0 V
1 0 V
1 0 V
1 0 V
1 0 V
1 0 V
1 0 V
1 0 V
1 0 V
1 0 V
1 0 V
1 0 V
1 0 V
1 0 V
1 0 V
1 0 V
1 0 V
1 0 V
1 0 V
1 0 V
1 0 V
1 0 V
1 0 V
1 0 V
1 0 V
1 0 V
1 0 V
1 0 V
1 0 V
1 0 V
1 0 V
1 0 V
1 0 V
1 0 V
1 1 V
1 0 V
1 0 V
1 0 V
1 0 V
1 0 V
1 0 V
1 0 V
1 0 V
1 0 V
1 0 V
1 0 V
1 0 V
1 0 V
1 0 V
1 0 V
1 0 V
1 0 V
1 0 V
1 0 V
1 0 V
1 0 V
1 0 V
1 0 V
1 0 V
1 0 V
1 0 V
1 0 V
1 0 V
1 0 V
1 0 V
1 0 V
1 0 V
1 0 V
1 0 V
1 0 V
1 0 V
1 0 V
1 0 V
1 0 V
1 0 V
1 0 V
1 0 V
1 0 V
1 0 V
1 0 V
1 0 V
1 0 V
1 0 V
1 0 V
1 0 V
1 0 V
1 0 V
1 0 V
1 0 V
1 0 V
1 0 V
1 0 V
1 0 V
1 0 V
1 0 V
1 0 V
1 0 V
1 0 V
1 0 V
1 0 V
1 0 V
1 0 V
1 0 V
1 0 V
1 0 V
1 0 V
1 0 V
1 0 V
1 0 V
stroke
grestore
end
showpage
}}%
\put(3037,1947){\makebox(0,0)[r]{$T(E)$}}%
\put(1950,50){\makebox(0,0){$E$ [MeV]}}%
\put(100,1180){%
\special{ps: gsave currentpoint currentpoint translate
270 rotate neg exch neg exch translate}%
\makebox(0,0)[b]{\shortstack{$T(E)$}}%
\special{ps: currentpoint grestore moveto}%
}%
\put(3450,200){\makebox(0,0){40}}%
\put(3075,200){\makebox(0,0){35}}%
\put(2700,200){\makebox(0,0){30}}%
\put(2325,200){\makebox(0,0){25}}%
\put(1950,200){\makebox(0,0){20}}%
\put(1575,200){\makebox(0,0){15}}%
\put(1200,200){\makebox(0,0){10}}%
\put(825,200){\makebox(0,0){5}}%
\put(450,200){\makebox(0,0){0}}%
\put(400,2060){\makebox(0,0)[r]{1}}%
\put(400,1708){\makebox(0,0)[r]{0.99}}%
\put(400,1356){\makebox(0,0)[r]{0.98}}%
\put(400,1004){\makebox(0,0)[r]{0.97}}%
\put(400,652){\makebox(0,0)[r]{0.96}}%
\put(400,300){\makebox(0,0)[r]{0.95}}%
\end{picture}%
\endgroup
\endinput

\caption{Transmisjonskoeffisienten som funksjon av kinetisk energi $E$.
Legg merke til at vi kun plotter $T$ for verdier n\ae r 1. For $E< 1$,
gjelder likning (\ref{eq:tbarriere}).}
\end{center}
\label{transcoeff}
\end{figure}
Fra det analytiske uttrykket ser vi at f\o rste maksimum for $T$,
kalles
for resonans, forekommer n\aa r
\be
    (k'a)^2=2(E-1)=\pi^2,
\ee
dvs.
\be
    E=\frac{\pi^2+1}{2}=5.43,
\ee
som stemmer overens med figuren.
Neste resonans skjer ved $E=20.24$. 
For st\o rre verdier av $E$ vil leddet 
\[
   \frac{1}{4E(E-1)},
\]
i nevneren for $T(E)$ g\aa\ mot null, og dermed f\aa r vi
at $T(E)\rightarrow 1$.  

\section{Anvendelser}

\subsection{Henfall av $\alpha$-partikler} 

Konkret skal vi p\aa\ henfall av $\alpha$-partikler\footnote{For mer
informasjon om henfall av $\alpha$-partikler, se kap 15-3 i boka,
sidene 751-759.}.

I denne analysen skal vi anta at $\alpha a >> 1$, som betyr
at vi ser p\aa\ kinetiske energier $E$ som er 
tilstrekkelig mindre enn
potensialbarrieren. For henfall av $\alpha$-partikler
er dette tilfelle. Typiske kinetiske energier for den emitterte 
$\alpha$-partikkelen  er
p\aa\ ca.~4-9 MeV, mens potensialbarrieren som
settes opp av Coulombpotensialet (se nedenfor for beskrivelse
av prosessen) er p\aa\ ca.~30 MeV. 

Vi definerer gjennomtrengningskonstanten $\delta$ ved
\be
\delta=\frac{1}{\alpha}=\frac{\hbar}{\sqrt{2m(V_0-E)}},
\ee
slik at uttrykket for $T$ kan forenkles n\aa r vi antar $\alpha a >> 1$ til
\be
T\approx\left(\frac{4k\delta}{1+\delta^2k^2}\right)e^{-2a/\delta}
\ee
med
\be
   k^2=\frac{2mE}{\hbar^2},
\ee
og
\be
   \delta^2k^2=\frac{E}{V_0-E}.
\ee
Som oppvarming for studiet av $\alpha$-partikkel henfall, ser vi 
p\aa\ 
et eksempel med en idealisert
potensialbarriere. To ledende koppertr\aa der
er adskilt av et isolerende oksydlag av CuO.
Vi modellerer dette oksydlaget som en kasseforma barriere 
med verdien 
$V_0=10$ eV. Deretter gir vi elektronene en kinetisk energi
$7$ eV.
Velger vi en barriere p\aa\ henholdsvis a) 5 nm og b) 1 nm,
stiller vi oss sp\o rsm\aa let om hva $T$ blir.
I dette tilfellet ser vi at $\alpha a >> 1$ slik at vi 
finner
\be
    (k\delta)^2=\frac{7}{10-7}=2.33,
\ee
og med $\hbar c=197$ eVnm og $m_ec^2=511$ keV,
finner vi at $\delta=0.113$ nm, dvs.~at gjennomtrengningsdybden
er alltid mindre enn barrieren. Ser vi p\aa\ tilfellet
a) finner vi at $T=0.98\times 10^{-38}$ mens b) gir
$T=0.66\times 10^{-7}$. Ei forandring p\aa\ utstrekning
av barrierern p\aa\ 4 nm gir ei forandring i sannsynlighet
for tunneling p\aa\ 31 st\o rrelsesordener.
Dette resultatet gir oss et aldri s\aa\ lite 
hint om mulige forklaringer
p\aa\ de store forskjellene i levetid for kjerner
som henfaller ved utsending av $\alpha$-partikkel.
\AA rsaken er at halveringstida for en kjerne er omvendt
proporsjonal med transmisjonkoeffisienten. 

Vi foretar en ytterlige approksimasjon i v\aa r analyse
av $\alpha$-partikkel henfall, nemlig
\be
T\approx e^{-2a/\delta}
\ee
Vi kan ogs\aa\ skrive $T$ fullt ut
\be
T\approx e^{-2a\sqrt{2m(V_0-E)}/\hbar}.
\ee
Vi ser at transmisjonskoeffisienten avhenger
av b\aa de potensialbarrierens utstrekning og 
gjennomtrengningskonstanten $\delta$. 

Har vi et potensial som avhenger av $x$ over et omr\aa de
$[a,b]$ kan vi approksimere siste likning med
\be
T\approx e^{-2\int_a^b\sqrt{2m(V(x)-E)}/\hbar dx}.
\ee

Men vi f\o r vi g\aa r videre med v\aa r analyse, er det p\aa\ tide
\aa\ si litt mer om sj\o lve prosessen.
$\alpha$-henfall
kan summeres opp ved f\o lgende reaksjon
\be
  X\rightarrow \alpha + Y,
\ee
hvor $X$ er kjernen som henfaller, og $Y$ kalles for datterkjernen.
$X$ har da $Z$ protoner, mens $Y$ har $Z-2$ protoner, og ogs\aa\
2 mindre n\o ytroner, siden $\alpha$-partikkelen best\aa r av
to protoner og to n\o ytroner. 

Denne prosessen kan vi tenke oss foreg\aa r p\aa\ f\o lgende vis.
I kjernen $X$ har vi en viss sannsynlighet for at vi kan 
danne oss en midlertig tilstand som best\aa r av $\alpha + Y$. Disse to 
kjernene 
henger sammen slik at vi fremdeles har $X$. 
Siden $\alpha$-partikkelen har st\o rre bindingsenergi i absoluttverdi
enn deutronet pga.~de sterke kjernekreftene, vil en slik
midlertig tilstand med $\alpha + Y$ ha en lavere energi enn
en tilstand hvor vi erstatter $\alpha$-partikkelen med deutronet.

Systemet v\aa rt best\aa r n\aa\ av en $\alpha$-partikkel
pluss $Y$. Ved korte avstander mellom disse to partiklene
vil de sterke kjernekreftene virke og holde kjernene sammen. 
Ved st\o rre avstander vil derimot det frast\o tende Coulombpotensialet
ta over. Den potensielle energien er vist i Figur 15.10 i boka. 

Vi idealiserer potensialet vist i Figur 15.10 til et
kassepotensial for de sterke kjernekreftene pluss Coulombpotensialet, 
som vist i figuren nedenfor. 
Kassepotensialet skal alts\aa\ representere de sterke kjernekreftene
som holder $\alpha$-partikkelen til $Y$.
Vi antar at $\alpha$-partikkelen 
har kun kinetisk energi $E$ innafor kassepotensialet. 
\begin{figure}
\begin{center}
%
\setlength{\unitlength}{1cm}
%
\begin{picture}(10,6)
\thicklines
\put(0,0){\makebox(0,0)[bl]{
		\put(0,0){\line(1,0){2}}
		\put(2,0){\line(0,1){4}}
		\put(0,2){\line(1,0){8}}
                \put(-0.1,4){\line(1,0){0.2}}

                \put(2,2){\dashbox{0.2}(2,2){}}

		\put(0,0){\vector(0,1){5}}

                \qbezier(2,4)(4,2)(8,2)
                \put(2.1,1.9){\makebox(0,0)[tl]{\small $R$}}
                \put(3.5,1.8){\makebox(0,0)[cl]{\small $R + b$}}
                \put(8.2,2){\makebox(0,0)[cl]{\small $r$}}
                \put(0.2,4.5){\makebox(0,0)[bl]{ E}}
         }}
%
\end{picture}
%
\end{center}
\caption{Idealisert framstilling av den potensielle energien
mellom en $\alpha$-partikkel og kjernen $Y$.}
\end{figure} 

Coulombpotensialet er da gitt ved 
\be
   V(r)=\frac{ZZ'e^2}{4\pi\epsilon_0 r}
\ee
hvor $Z=2$, ladningen til $\alpha$-partikkelen mens $Z'$ er ladningen til
datterkjernen $Y$. 

Kvantemekanisk kan $\alpha$-partikkelen tunnelere gjennom
barrieren som er satt opp av Coulombpotensialet. Det er denne tunneleringen
som gir opphav til $\alpha$-decay. Vi g\aa r dermed fra en midlertig
tilstand som best\aa r av $\alpha +Y$ til to separate kjerner
$\alpha$ og $Y$.
 
Transmisjonskoeffisienten blir 
\be
   T(E)=exp\left(-2\int_R^{R+b}\frac{\sqrt{2m(V(r)-E)}}{\hbar} dr\right)=
       exp\left(-\frac{2\sqrt{2m}}{\hbar}\int_R^{R+b}\frac{\sqrt{\frac{ZZ'e^2}{4\pi\epsilon_0 r}-E)}}{\hbar} dr\right).
\ee

Utrekning gir 
\be
      T(E)=exp\left(-4\pi Z
           \sqrt{\frac{E_0}{E}}+8\sqrt{\frac{ZR}{r_0}}\right),
\ee
hvor vi har introdusert konstantene
\be
   r_0=\frac{4\pi\epsilon_0\hbar^2}{m_{\alpha}e^2},
\ee
hvor $r_0$ er en slags Bohr radius med $m_{\alpha}=7295m_e$.
Konstanten $E_0$ er definert som
\be
   E_0=\frac{e^2}{8\pi\epsilon_0 r_0}=0.0995 \hspace{0.1cm}
   \mathrm{MeV}.
\ee
Halvveringstida for en kjerne er gitt ved
\be
    t_{1/2}=\frac{ln2}{\lambda},
\ee
hvor 
\be
   \lambda=fT(E)\approx 10^{21}e^{\left\{-4\pi Z
      \sqrt{\frac{E_0}{E}}+8\sqrt{\frac{ZR}{r_0}}\right\}}.
\ee
St\o rrelsen $f$ skal representere antall ganger $\alpha$-partikkelen
treffer potensialveggen per sekund, og er typisk av st\o rrelsesorden 
$10^{21}$
Vi skal se p\aa\ anvendelser av dette uutrykket 
for kjernene Thorium og Polonium. For Thorium har den utsendte
$\alpha$-partikkelen en kinetisk energi p\aa\
$4.05$ MeV mens for Polonium har vi $8.95$ MeV.

Thorium har 90 protoner, noe som inneb\ae rer at datterkjernen
har $Z=88$. Verdien p\aa\ $R$ er 9 fm. Det innb\ae rer ei halvveringstid p\aa\ 
\be
    t_{1/2}=\frac{ln2}{1.29\times 10^{-18}}=
    5.37\times 10^{17} \hspace{0.1cm}
   \mathrm{s}=1.70\times 10^{10} \hspace{0.1cm}
   \mathrm{yr},
\ee
som kan sammenholdes med den eksperimentelle verdien
p\aa\ $1.3\times 10^{10}$ yr.

Tilsvarende har Polonium 84 protoner, noe som inneb\ae rer at datterkjernen
har $Z=82$. Verdien p\aa\ $R$ er 9 fm. Det innb\ae rer ei halvveringstid p\aa\ 
\be
    t_{1/2}=\frac{ln2}{8.23\times 10^{8}}=
    8.42\times 10^{-10} \hspace{0.1cm}
   \mathrm{s},
\ee
som igjen sammenholdes med den eksperimentelle verdien
p\aa\ $3\times 10^{-7}$ s.

Sj\o l om vi har antatt en sv\ae rt s\aa\ skjematisk form
for potensialbarrieren, s\aa\ utviser begge disse
eksemplene et bra samsvar med eksperiment og viser at
en forskjell p\aa\ en faktor $2$ i kinetisk energi er nok
til \aa\ gi en forskjell i levetid p\aa\
26 st\o rrelsesordener!

Eksemplet viser klart forholdet mellom barrierens potensielle energi
og utstrekning,
$\alpha$-partikkelens kinetiske energi og sannsynligheten for
tunneling. 


%\subsection{Scanning Tunneling mikroskop}



\section{Oppgaver}
\subsection{Analytiske oppgaver}
\subsubsection*{Oppgave 6.1, Eksamen H-1992}
\begin{itemize}
\item[a)] Forklar kort hvilken fysisk betydning vi tillegger
en b{\o}lgefunksjon
$\Psi(\vec{r},t)$ i kvantemekanikken. Hvilke matematiske krav
m{\aa} vi stille til en slik
b{\o}lgefunksjon? Forklar hva	vi mener med en operator og en egenfunksjon
for en operator.
%
\item[b)] Skriv ned den tidsavhengige Schr\"{o}dingerligningen
for en partikkel med potensiell energi $ V(\vec{r})$.
%
\item[c)] Vis at l{\o}sningene kan skrives p{\aa} formen
%
\[
\Psi(\vec{r}, t) = \psi(\vec{r}) \exp(-iEt /\hbar).
\]
%
og finn den ligningen som $\psi(\vec{r})$ tilfredsstiller.
%
\item[d)]   Anta at $A(\vec{r},\vec{p})$ er en fysisk st{\o}rrelse
(funksjon av koordinat og bevegelsesmengde) tilordnet en partikkel.
Hvordan kommer vi i kvantemekanikken fram til de verdier som
er mulig {\aa} finne n{\aa}r vi gj{\o}r en ideell m{\aa}ling
av den fysisk st{\o}rrelsen A?
%
\end{itemize}
%

I det f\o lgende skal vi studere en partikkel med masse $m$ som beveger seg i en
potensialbr\o nn gitt ved
%
\[
V(x) = \left \{
\begin{array}{c}
     0  \\
  - V_{0}\\
    0
\end{array}
%
\;\;\; \mbox{for} \;\;\;
%
\begin{array}{r}
 x < 0 \\
 0 \leq x < a \\
 x \geq  a
\end{array}
%
\right .
%
\]
%
hvor $V_0 > 0$ og med energi $E > 0$.
%
\begin{itemize}
%
\item[e)] Anta at partikkelen kommer inn fra $x = -\infty$ og beveger seg mot
potensialbr\o nnen. Beskriv uten utregning hva som skjer i punktene $x=0$ og
$x=a$. Skiss\'{e}r b\o lgefunksjonen for partikkelen i
omr\aa dene $x<0,\;\; 0\leq x \leq a$ og $x>a$, og finn et uttrykk
for b\o lgelengden til partikkelen i disse
omr\aa dene.
%
\item[f)] Skriv ned l\o sningen av Schr\"{o}dingerligningen
for problemet samt
de randkravene denne m\aa ~oppfylle.
%
\item[g)] Vis at ligningene for randkravene kan skrives p{\aa}
formen
%
\begin{eqnarray*}
1 + \frac{B}{A} &=& \frac{C}{A} + \frac{D}{A}\\
1 - \frac{B}{A} &=& \lambda \left ( \frac{C}{A} - \frac{D}{A} \right )\\
\frac{C}{A} \exp (ik^{'}a) + \frac{D}{A} \exp (-ik^{'}a )
&=& \frac{F}{A} \exp (ika)\\
\lambda \left (\frac{C}{A} \exp (ik^{'}a) - \frac{D}{A} \exp (-ik^{'}a )
											\right )
&=& \frac{F}{A} \exp (ika)
\end{eqnarray*}
%
hvor $\lambda = \sqrt{1 + V_0 / E}$. Forklar betydningen av konstantene
$A, B, C, D, F, k$ og $k^{'}$.
%
\item[h)] Vi antar n{\aa} at $V_0 = 3 E$.
Vis at sannsynligheten, uttrykt ved $E$, for at partikler
reflekteres tilbake til omr\aa det $x<0$ er gitt ved
\begin{eqnarray*}
{\bf R} = \frac{1}{25 / 9
+ 16 / 9 \left [ cotg \left( \frac{a}{\hbar }\sqrt{8mE}\right)\right ]^2 }.
\end{eqnarray*}
%
\end{itemize}
%
\subsubsection*{Kort fasit}
%
\begin{itemize}
%
\item[a),b),c)]   Se l{\ae}reboka {\sl Brehm and Mullin: Introduction to the
           structure of matter}, avsnitt 5.1, 5.2
%
\item[d)] Den fysiske st{\o}rrelsen overf{\o}res til en kvantemekanisk
operator
$ A(\vec{r}, \vec{p}) \Longrightarrow A(\OP{\vec{r}},\OP{\vec{p}})$,
hvor $\OP{\vec{r}} = \vec{r}$ og
$\OP{\vec{p}} = -i \hbar \vec{\bigtriangledown} $.

Vi l{\o}ser egenverdiligningen
%
\[
A(\OP{\vec{r}},\OP{\vec{p}})\Phi_{\nu} = A_{\nu} \Phi_{\nu}
\]
%
og de eneste mulige verdiene vi kan f{\aa} ved en ideell m{\aa}ling
er en av egenverdiene $A_{\nu}$.
%
\item[e)]  En plan b{\o}lge med b{\o}lgetall
$k = \sqrt{(2m/ \hbar^2)E}$ kommer inn fra $x = -\infty$.
I punktet $x = 0$ dannes det en reflektert b{\o}lge
med b{\o}lgetall $-k$. En ny b{\o}lge fortsetter mot positiv x
med b{\o}lgetall $k^{'} = \sqrt{(2m/ \hbar^2)(E + V)}$.
I punktet $x = a$ dannes det en reflektert b{\o}lge
med b{\o}lgetall $-k^{'}$. En ny b{\o}lge fortsetter mot
$x = \infty$ med b{\o}lgetall $k$.
Dette er vist i figur~2:
%
\item[f)] L{\o}sningen av Schr\"{o}dingerligningen blir
%
\begin{eqnarray*}
\psi_{I}(x) &=& A \exp(ikx) + B \exp(-ikx)\;\;\; x < a\\
\psi_{II}(x) &=& C\exp(ik^{'}x) + D \exp(-ik^{'}x)\;\;\; 0 < x < a\\
\psi_{III}(x) &=& F \exp(ikx)\;\;\;\;\;\;\;\;\;\;\; x > a
\end{eqnarray*}
%
Randkravene i $x = 0$ og $x = a$ gir
%
\begin{eqnarray*}
\psi_I(x = 0) = \psi_{II}(x = 0) &\longrightarrow& A + B = C + D\\
\psi_I^{'}(x = 0) = \psi_{II}^{'}(x = 0) &\longrightarrow&
k ( A - B ) = k^{'} ( C - D)\\
\psi_{II}(x = a) = \psi_{III}(x = a) &\longrightarrow&
 C \exp(ik^{'}a) + D \exp(-ik^{'}a) = F \exp(ika)\\
\psi_{II}^{'}(x = a) = \psi_{III}^{'}(x = a) &\longrightarrow&
k^{'} (C \exp(ik^{'}a) - D \exp(-k^{'}a) = k F \exp(ika)
\end{eqnarray*}
%
Ved omforming f{\aa}r vi ligningene som er gitt i oppgaven hvor
%
\[
\lambda = \frac{k^{'}}{k}
		  = \sqrt{\frac{E + V_0}{E}} = \sqrt{1 + \frac{V_0}{E}}
\]
%
\item[h)] Refleksjonsfaktoren er definert ved
%
\[
R = \left | \frac{B}{A} \right |^2
\]
%
L{\o}sning av ligningene under g) m.h.p. B/A gir svaret for
refleksjonskoeffisienten R.
%
\end{itemize}
\subsubsection*{Oppgave 6.2}
Vi betrakter en str{\o}m av partikler med masse $m$ og med kinetisk
energi $E_p \ll m c^2$ som beveger seg mot et potensialtrinn
som vist p{\aa} figur~\ref{fig2.1}:
%
\begin{figure}[htbp]
%
\begin{center}

\setlength{\unitlength}{0.7cm}
%
\begin{pspicture}(0,0)(11,6)
%
    \psline(0,2)(5,2)
    \psline[linestyle=dashed]{->}(5,2)(10,2)
    \rput(10.2,2){$x$}
    \psline{->}(5,0)(5,6)
    \rput(5.7,5.8){$V(x)$}
    \multiput(0,4)(0.5,0){10}{\psline{->}(0,0)(0.3,0)}
    \psline(5,0)(10,0)
    \rput(7,0.2){$V(x) = -V_0$}
    \rput(2.0,2.2){$V(x) = 0$}
%
      \end{pspicture}

\caption{\label{fig2.1}}
\end{center}

\end{figure}
%

%
\noindent
Den potensielle energien er
%
\[
V(x) = \left \{
\begin{array}{c}
     0  \\
  - V_{0}
\end{array}
%
\;\;\; \mbox{for} \;\;\;
%
\begin{array}{r}
 x < 0 \\
 x \geq  0
\end{array}
%
\right .
%
\]
%

\begin{itemize}
%
\item[a)] Forklar kort og kvalitativt partiklenes videre
bevegelse hvis vi behandler systemet med klassisk
mekanikk.
%
\item[b)] Sett opp Schr\"{o}dingerligningen og vis at den har
l{\o}sningen
%
\begin{eqnarray*}
%
\psi_I &=& A \exp (i k_1 x) + B \exp (-i k_1 x)\;\;\;\; for \;\;\; x < 0\\
\psi_{II} &=& C \exp (i k_2 x) + D \exp (-i k_2 x)\;\;\;\; for \;\;\; x > 0
%
\end{eqnarray*}
%
hvor $A, B, C$ og $D$ er integrasjonskonstanter.\\
Hva er $k_1$ og $k_2$ i disse funksjonene? Hvorfor m{\aa} n{\o}dvendigvis
$D$ v{\ae}re lik null for dette systemet?
%
\item[c)] Hvilke betingelser m{\aa} $\psi_I$ og $\psi_{II}$ oppfylle
ved $x = 0$?\\
Finn $| B / A |^2$ og $| C / A |^2$ uttrykt ved $k_1$ og $k_2$.
%
\item[d)] Hvorfor kan vi kalle $R = | B / A |^2$ for
refleksjonskoeffisienten?. Defin\'{e}r og finn et uttrykk for
transmisjonskoeffisienten $T$.
%
\item[e)] Vis at $ R + T = 1$. Forklar det fysiske innholdet
i denne enkle relasjonen.
%
\end{itemize}
%
%
\subsubsection*{Oppgave 6.3}
Et elektron med energi E = 5 eV beveger seg langs x-aksen i et potensial
%
\[
V(x) = \left \{
\begin{array}{c}
     0  \\
    V_{0}
\end{array}
%
\;\;\; \mbox{for} \;\;\;
%
\begin{array}{r}
 x < 0 \\
 x \geq  0
\end{array}
%
\right .
%
\]
%

der konstanten $V_{0}$ forel\o big er ukjent.

%
\begin{itemize}
%
\item[a)] Hva er de Broglie b\o lgelengden for elektronet
n\aa r det befinner
seg i omr\aa det $x < 0$?

\item[b)] Anta at elektronet kommer inn fra $x = -\infty$
og beveger seg mot
potensialtrinnet. Hvordan ser b\o lgefunksjonen for elektronet ut i
omr\aa dene $x < 0$ og $x > 0$? Beskriv uten utregning hva som skjer b\aa de
n\aa r $E < V_{0}$ og n\aa r $E > V_{0}$.

\item[c)] Beregn sannsynlighetsamplituden uttrykt ved $E$ og $V_{0}$ for at
elektronet reflekteres tilbake fra potensialtrinnet.

\item[d)] Vis at n\aa r $V_{0}$ = 10 eV vil alle slike innkommende
elektroner reflekteres tilbake.

\item[e)] For denne verdien av $V_{0}$, beregn sannsynlighetstettheten
i punktet $x$ = 1 nm n\aa r sannsynligheten er 100 \% for \aa ~finne
den i $x$ = 0.
\end{itemize}
%
\subsubsection*{Oppgave 6.4}
La $\psi_{v}$ v{\ae}re en tidsavhengig b{\o}lgefunksjon for
en partikkel med masse $m$ og potensiell energi
$E_{p} = 0$,
\[
\psi_{v}(x,t)
	= \exp \left ( \frac{i}{\hbar} m v ( x - \frac{v}{2} t) \right ),
\]
hvor $v$ er konstant.
%
\begin{itemize}
%
\item[a)] Vis at $\psi_{v}$ lyder b{\o}lgeligningen (Schr\"{o}dingers
tidsavhengige ligning), og at partikkelens hastighet er $v$.
\end{itemize}
%

Vi skal studere (\'{e}n--dimensjonalt) partikler
som st{\o}ter mot et uendelig h{\o}yt potensialtrinn.
Det nye ved situasjonen er at potensialtrinnet
beveger seg (i x--retningen) med en gitt hastighet $u > 0$.

%
\noindent
\begin{minipage}{0.3\textwidth}
%
Den potensielle energien er gitt ved
%
\[
V_{p} = \left \{
\begin{array}{rcl}
%
0        & \mbox{for}  & x < x_{1}\\
+\infty  & \mbox{for}  & x > x_{1}\\
\end{array}
\right .
\]
%
der $x_{1} = ut$.\\

\end{minipage}
%
\hspace*{0.05\textwidth}
%
\begin{minipage}{0.65\textwidth}
%
\begin{pspicture}(0.65\textwidth,3cm)
%
\psline{->}(0,0.5)(8,0.5)
\rput(8.2,0.5){$x$}

\rput(3,0){$x = 0$}
\rput(5,0){$x_1 = u t$}

\rput(2,0.7){$V(x) = 0$}

\psline{->}(4,1)(6,1)
\rput(6.3,1){$u$}

\rput(4.5,2.5){$V(x)$}
\psframe[fillstyle=solid](4,0.5)(4.2,2)
\psline{->}(4,0.5)(4,3)
%
\end{pspicture}
%
\end{minipage}
%
P{\aa} grunn av tidsavhengigheten i $V_{p}$ m{\aa} vi
bruke tidsavhengig b{\o}lgefunksjon for
inn\-kom\-men\-de og reflektert partikkel.
Vi lar $v$ v{\ae}re hastigheten av innkommende
partikkel, og forutsetter at $ v > 2 u $.
%
\begin{itemize}
%
\item[b)] Hastigheten $v$ av reflektert partikkel blir
$ w = -v + 2 u$. Gj{\o}r rede for at dette er rimelig
(for~eks. ved {\aa} se p{\aa} en tilsvarende  \underline{klassisk}
situasjon, -- eller p{\aa} en annen m{\aa}te).
\end{itemize}
%
B{\o}lgefunksjonen for st{\o}tprosessen kan n{\aa} uttrykkes
%
\[
\psi(x,t) = A \psi_{v}(x,t) + B\psi_{w}(x,t), \;\;\;\;  x < x_{1}
\]
der $A$ og $B$ er konstanter.
%
\begin{itemize}
%
\item[c)] Vis at kravet om riktig randverdi i $ x = x_{1}$
f{\o}rer til $ B = -A$.
%
\item[d)] Ved {\aa} sette opp partikkelfluksen av innkommende
og reflektert str{\aa}le, skal man finne refleksjonsfaktoren $R$.
Gi en kommentar til svaret.
\end{itemize}
%

\subsubsection*{Oppgave 6.5}
Et elektron beveger seg med energi $E > 0$ fra venstre mot et kassepotensial
%
\[
V(x) = \left \{
\begin{array}{c}
     0  \\
  - V_{0}\\
     0
\end{array}
%
\;\;\; \mbox{for} \;\;\;
%
\begin{array}{r}
 x < 0 \\
 0 \leq x < a \\
 x \geq  a
\end{array}
%
\right .
%
\]
%
I det f\o lgende antas at energien $E < V_{0}$.
%
\begin{itemize}
%
\item[a)] Regn ut refleksjonskoeffisienten R og
transmisjonskoeffisienten T for
dette potensialet. Vis at T kan skrives som
%
\begin{eqnarray*}
T = \frac{1}{1 + \frac{1}{4} \left( \frac{\beta }{k} +
\frac{k}{\beta }\right)^{2} \: sinh^{2}(\beta a)}.
\end{eqnarray*}
%
\item[b)] Vis at i grensen $\beta a \gg 1$ kan dette
uttrykket forenkles til
%
\begin{eqnarray*}
T = 16 \frac{E}{V_{0}} (1 - \frac{E}{V_{0}}) e^{-2\beta a}.
\end{eqnarray*}

\item[c)] Beregn T ut fra den eksakte formelen i a) n\aa r E = 2 eV,
$V_{0}$ = 10 eV og a = 0,1 nm. Hvor stor blir den relative
feilen ved bruk av
den approksimative formelen i dette tilfellet?
%
\end{itemize}
%
Vi antar n\aa ~at energien til elektronet er $E > V_{0}$.
%
\begin{itemize}
%
\item[d)] Argumenter for at transmisjonskoeffisienten i dette tilfellet kan
avledes direkte fra formelen i punkt a), og finn ved en enkel utregning det nye
uttrykket for T.

\item[e)] For hvilke energier f\aa r vi ingen reflekterte partikler? Pr\o v \aa
~forst\aa ~dette ved \aa ~finne b\o lgelengden til b\o lgene i omr\aa det der
V = $V_{0}$ uttrykt ved potensialbredden a.
%
\end{itemize}
%
\subsubsection*{Oppgave 6.6, Eksamen V-1994}
En jevn str{\o}m av partikler, hver med masse $m$, beveger seg
langs x--aksen i positiv retning.
Den potensielle energien er gitt ved
%
\[
V(x) = \left \{
\begin{array}{c}
     V_1  \\
         0\\
    V_3
\end{array}
%
\;\;\; \mbox{for} \;\;\;
%
\begin{array}{r}
 x < 0 \\
 0 \leq x < a \\
 x \geq  a
\end{array}
%
\right .
%
\]
%
hvor partiklenes energi $E > V_1 > V_3 > 0$.
%
\begin{itemize}
%
\item[a)] L{\o}s den tidsuavhengige energi egenverdiligningen
for hvert av de tre omr{\aa}dene hvor den potensielle energien
er konstant.
%
\item[b)] Formul\'{e}r betingelsene  egenfunksjonene m{\aa}
tilfredsstille i dette tilfelle. Gi et begrunnet svar.
%
\item[c)] Finn  transmisjonskoeffisienten $T$
for fluksen gjennom det totale system fra $x < 0$ til $x > a$
n{\aa}r det forutsettes
at
\[
k_2 a = 3 \pi,\;\;\; \mbox{hvor:}\;\;\;
k_2  = \frac{1}{\hbar}\sqrt{2 m E}\;\;\; \mbox{og} \;\;\;
k_3 a =  \pi,\;\;\; \mbox{hvor:}\;\;\;
k_3  = \frac{1}{\hbar}\sqrt{2 m(E - V_3)}
\]
%
\end{itemize}
%
\subsubsection*{Kort fasit}
\begin{itemize}
\item[a)]
%
\begin{eqnarray}
x < 0, \;\;\; \psi_1(x) &= A \exp (i k_1 x) + B \exp (- i k_1 x),
\;\; k_1 &= \frac{\sqrt{2 m ( E - V_1)}}{\hbar^2} \nonumber \\
0 < x < a, \;\;\; \psi_2(x) &= C \exp (i k_2 x) + D \exp (- i k_2 x),
\;\; k_2 &= \frac{\sqrt{2 m E}}{\hbar^2} \nonumber\\
x > 0, \;\;\; \psi_3(x) &= F \exp (i k_1 x)
\;\; k_3 &= \frac{\sqrt{2 m ( E - V_3)}}{\hbar^2} \nonumber
\end{eqnarray}
%
\item[b)]  Se l{\ae}reboka {\sl Brehm and Mullin: Introduction to the
           structure of matter}, avsnitt 5.1 og 5.2
%
\item[c)] Transmisjonen fra $x < 0$ til $x > a$
\begin{equation}
T = \frac{k_3}{k_1} \left | \frac{F}{A} \right |^2
  = \frac{4 k_1 k_3}{(k_1 + k_3)^2}\nonumber
%
\end{equation}
%
\end{itemize}

\subsubsection*{Oppgave 6.7, Eksamen H-1996}
I denne oppgaven skal vi analysere et system best�ende av en str{\o}m partikler med 
masse $m$ og  kinetisk energi $E_k \ll m c^2$ som beveger seg fra $ x = - \infty$ mot et 
potensialtrinn gitt ved
%
\[
V(x) = \left \{
\begin{array}{c}
     0  \\
  - V_{0}
\end{array}
%
\;\;\; \mbox{for} \;\;\;
%
\begin{array}{r}
 x < 0 \\
 x \geq  0
\end{array}
%
\right .
%
\]
%
Vi skal f�rst studere problemet utfra klassisk mekanikk.
%
\begin{itemize}
%
\item[a)] Forklar kort partiklenes bevegelse og hva som skjer n�r 
de passerer potensialspranget i $x = 0$. Finn partiklenes
kinetiske energi for $x > 0$.
%
\end{itemize}

\noindent
La oss s� studere problemet med utgangspunkt i Schr\"{o}dingers kvantemekanikk.

%
\begin{itemize}
%
\item[b)] Sett opp Schr\"{o}dingerligningen og finn  
l�sningene i omr�det I $(x < 0)$ og i omr�det II  $(x > 0)$. 
%
\item[c)] En kvantemekanisk b�lgefunksjon m� tilfredsstille visse generelle krav.
Skriv ned og begrunn de krav som gjelder for den totale b�lgefunksjonen 
i dette tilfelle.
%
\item[d)] Bruk prinsippene under c) til � finne de fullstendige b�lgefunksjonene for 
systemet i hele omr�det  $-\infty < x < +\infty$.
%
\item[e)] Bestem  refleksjonskoeffisienten $R$ og 
transmisjonskoeffisienten $T$ i punktet
\mbox{$x = 0$} som funksjon av partiklenes energi $E$.
Vis at $R + T = 1$ og forklar det fysiske innholdet i denne enkle relasjonen.
%
\end{itemize}
%
Vi forandrer n�  den potensielle energien til 
%
\[
V(x) = \left \{
\begin{array}{c}
     0  \\
  - V_{0}\\
  + V_1
\end{array}
%
\;\;\; \mbox{for} \;\;\;
%
\begin{array}{r}
 x < 0 \\
 0\leq x < a \\
 x \geq  a
\end{array}
%
\right .
%
\]
%
hvor  $V_0 > V_1 > 0$ og $E > V_1$.\\

\begin{itemize}
%
\item[f)] Ta utgangspunkt i l�sningen under b) og finn den generelle l�sningen
for hvert av de tre omr{\aa}dene.
%
\item[g)] Bruk kravene  under c) til � finne den totale b�lgefunksjon
i hele omr�det\\ 
 $ -\infty < x  < + \infty$ 
%
\item[h)] Finn  den nye transmisjonskoeffisienten $T$
for fluksen gjennom det totale system fra \mbox{$x < 0$} til $x > a$
n{\aa}r det forutsettes
at
\[
\sqrt{\frac{2 m (E + V_0)}{\hbar^2}} = \frac{3 \pi}{a}\;\;\; \mbox{og} \;\;\;
\sqrt{\frac{2 m (E + V_1)}{\hbar^2}} = \frac{\pi}{a}
\]
%
Finn refleksjonskoeffisienten $R$ for det totale systemet. Beregn $R + T$
og gi en fysikalsk forklaring p� svaret.
\end{itemize}
%
\subsubsection*{Kort fasit}
\begin{itemize}
 \item[a)] Klassisk mekanikk: Partiklene fortsetter rett frem, men med �ket hastighet.
Den kinetiske energi for $x > 0$ blir $E_k + V_0$.
%
 \item[b)] Schr\"{o}dingerligningen for et intervall med konstant potensiell
energi $V$ er gitt ved
%
\[
\left ( -\frac{\hbar^2}{2 m} \frac{d^2}{dx^2} + V \right ) \psi(x)
   = E_k \psi(x)
\]
%
Dette gir l�sningen
%
\[
\begin{array}{lll}
%
\mbox{Omr�det I:} & \psi_I(x) = A \exp (ik_I x) + B \exp (-ik_I x),
\;\;\; &k_I = \sqrt{\frac{2 m E_k}{\hbar^2}}\\
\mbox{Omr�det II:} & \psi_{II}(x) = C \exp (ik_{II} x),
\;\;\; &k_{II} = \sqrt{\frac{2 m}{\hbar^2}(E_k + V_0)}
%
\end{array}
%
\]
%
\item[c)] Se l{\ae}reboka {\sl Brehm and Mullin: Introduction to the
           structure of matter}, avsnitt 5.1, 5.2
%
\item[d)] Den totale $\psi(x)$ m� v�re kontinuerlig og med 
kontinuerlig derivert i punktet $x = 0$. Dette gir 
f�lgende betingelser for konstantene
%
\[
\frac{B}{A} = \frac{k_I - k_{II}}{k_I + k_{II}}, \;\;\;\;
\frac{C}{A} = \frac{2 k_I}{k_I + k_{II}}
\]
%
\item[e)] $R = (B / A)^2 = \left (\frac{k_I - k_{II}}{k_I + k_{II}}\right )^2$ og
$T = (v_{II} / v_I) (C / A)^2 = \frac{4 k_I k_{II}}{(k_I + k_{II})^2}$.
Dette gir $R + T = 1$.
%
\item[f)] 
%
\[
\begin{array}{lll}
%
\mbox{Omr�det I:}   &\psi_I(x) = A \exp (ik_I x) + B \exp (-ik_I x),
                                     \;\;\; &k_I = \sqrt{\frac{2 m E_k}{\hbar^2}}\\
\mbox{Omr�det II:}  &\psi_{II}(x) = C \exp (ik_{II} x) + D \exp (-ik_{II} x),
                             &k_{II} = \sqrt{\frac{2 m}{\hbar^2}(E_k + V_0)}\\
\mbox{Omr�det III:} &\psi_{III}(x) = F\exp (ik_{III} x),
                                            &k_{III} = \sqrt{\frac{2 m}{\hbar^2}(E_k + V_1)}
%
\end{array}
%
\]
%
\item[g)] Den totale $\psi(x)$ m� v�re kontinuerlig og med 
kontinuerlig derivert i punktene $x = 0$ og $x = a$. Dette gir 
f�lgende betingelser for konstantene
%
\[
\frac{B}{A} = \frac{k_{I} - k_{III}}{k_I + k_{III}}, \;\;\;\;\;\;\;\;\;\;
	\frac{C}{A} = \frac{k_I}{k_{II}} \frac{k_{II} + k_{III}}{k_I + k_{III}}
\]
\[
\frac{D}{A} = \frac{k_I}{k_{II}} \frac{k_{II} - k_{III}}{k_I + k_{III}}\;\;\;\;\;\;\;
	\frac{F}{A} = \frac{2 k_I}{k_I + k_{III}}
\]
%

\item[h)] $R = (B / A)^2 = \left ( \frac{k_I - k_{III}}{k_I + k_{III}}\right)^2$ og
$T = \frac{4 k_I k_{III}}{(k_I + k_{III})^2}$
Dette gir $R + T = 1$.
%


\end{itemize}

\subsubsection*{Oppgave 6.8, Eksamen V-1999}
En jevn str{\o}m av partikler, hver med masse $m$ og  kinetisk energi
$E_k$  beveger seg fra $ x = - \infty$ mot et 
potensialtrinn gitt ved
%
\[
V(x) = \left \{
\begin{array}{c}
     0  \\
  + V_{0}
\end{array}
%
\;\;\; \mbox{for} \;\;\;
%
\begin{array}{r}
 x < 0 \\
 x \geq  0
\end{array}
%
\right .
%
\]
%
%
Vi skal f�rst studere problemet utfra klassisk mekanikk og antar at
$E_k > V_0 > 0$
%
\begin{itemize}
%
\item[a)] Forklar kort partiklenes bevegelse og hva som skjer n�r 
de passerer potensialspranget i $x = 0$. Finn partiklenes
kinetiske energi for $x > 0$.
%
\end{itemize}


\noindent
La oss s� studere problemet med utgangspunkt i Schr\"{o}dingers kvantemekanikk.

%
\begin{itemize}
%
\item[b)] Sett opp Schr\"{o}dingerligningen og finn  
l�sningene i omr�det I $(x < 0)$ og i omr�det II  $(x > 0)$. 
%
\item[c)] En kvantemekanisk b�lgefunksjon m� tilfredsstille visse generelle krav.
Skriv ned de krav som gjelder for den totale b�lgefunksjonen 
i dette tilfelle.
%
\item[d)] Bruk prinsippene under c) til � finne den fullstendige b�lgefunksjonen for 
sy\-stemet i hele omr�det  $-\infty < x < +\infty$.
%
\item[e)] Bestem  refleksjonskoeffisienten $R$ og 
transmisjonskoeffisienten $T$ i punktet
\mbox{$x = 0$} som funksjon av partiklenes energi $E_k$.
Vis at $R + T = 1$ og forklar det fysiske innholdet i denne relasjonen.
%
\end{itemize}
%
Vi forandrer n�  den potensielle energien til 
%
\[
V(x) = \left \{
\begin{array}{c}
     0  \\
  +V_{0}\\
  + V_1
\end{array}
%
\;\;\; \mbox{for} \;\;\;
%
\begin{array}{r}
 x < 0 \\
 0\leq x < a \\
 x \geq  a
\end{array}
%
\right .
%
\]
%
hvor  $E_k > V_1 > V_0 > 0$.\\

\begin{itemize}
%
\item[f)] Ta utgangspunkt i l�sningen under b) og finn den generelle l�sningen
for hvert av de tre omr{\aa}dene.
%
\item[g)] Bruk kravene  under c) til � finne betingelsene den
 fullstendige b�lgefunksjonen m� tilfredsstille.
%
\end{itemize}
%
Ligningene i g) er vanskelig � l�se. La oss derfor innf�re f�lgende
forenklinger
\[
V_0 = E_k -  \frac{\hbar^2}{2m} \frac{9\pi^2}{a^2} \;\;\; \mbox{og} \;\;\;
V_1 = E_k - \frac{\hbar^2}{2m} \frac{\pi^2}{a^2} 
\]
%
\begin{itemize}

\item[h)] Finn n� den fullstendige b�lgefunksjonen for 
systemet i hele omr�det\\
  $-\infty < x < +\infty$
%
\item[i)] Finn refleksjonskoeffisienten $R$ og
transmisjonskoeffisientet $T$ for det totale sy\-stemet. Beregn $R + T$
og gi en fysikalsk forklaring p� svaret.
%
\end{itemize}
\subsubsection*{Kort fasit}
\begin{itemize}
%
\item[a)] I punktet $ x = 0$ treffer partikkelen en potensialbarriere 
$V = +V_0$ og mister kinetisk energi. Den fortsetter for $x > 0$
med $E(\mbox{ny}) = E_k - V_0$.
%
\item[b)]
%
\begin{eqnarray}
%
\mbox{Omr�de I}:& \quad -\frac{\hbar}{2m} \frac{d^2}{dx^2}\psi(x)
      &= E_k \psi(x)\nonumber\\
\mbox{Omr�de II}:& \quad -\frac{\hbar}{2m} \frac{d^2}{dx^2}\psi(x)
                       + V_0 \psi(x) &= E_k \psi(x)\nonumber\nonumber
%
\end{eqnarray}
%
\item[c)]  Se l{\ae}reboka {\sl Brehm and Mullin: Introduction to the
           structure of matter}, avsnitt 5.1, 5.2
%
\item[d)] 
%
\begin{eqnarray}
\mbox{Omr�de 1:} \quad 
\psi_1 &=& A \left( \exp (i k_1 x) 
            + \frac{k_1 - k_2}{k_1 + k_2} \exp (-i k_1 x) \right )\nonumber \\
%
\mbox{Omr�de 2:} \quad 
\psi_2 &=& A \frac{2k_1}{k_1 + k_2} \exp (i k_2 x)  \nonumber
%
\end{eqnarray}
%
\item[e)] 
%
\[
R = \left | \frac{B}{A} \right |^2 
  = \left (\frac{k_1 - k_2}{k_1 + k_2} \right )^2
\quad \quad
T  = \left | \frac{C}{A} \right |^2 \frac{k_2}{k_1}
            = \frac{4 k_1 k_2}{(k_1 + k_2)^2}
%
\]
%
\item[f)]
%
\begin{eqnarray}
%
\mbox{Omr�de 1:} \quad 
\psi_1 =& A \exp (i k_1 x) +  B \exp (-i k_1 x)
         \quad 
        &k_1 = \sqrt{\frac{2m}{\hbar^2} E_k} \nonumber \\
%
\mbox{Omr�de 2:} \quad
\psi_2 =& C \exp (i k_2 x) +  D \exp (-i k_2 x)
         \quad 
        &k_2 = \sqrt{\frac{2m}{\hbar^2} (E_k - V_0)} \nonumber \\
%
\mbox{Omr�de 3:} \quad
\psi_3 = &F \exp (i k_3 x)
         \quad 
        &k_3 = \sqrt{\frac{2m}{\hbar^2} (E_k - V_1)} \nonumber 
%
\end{eqnarray}
%
\item[g)]
%
\begin{eqnarray}
%
A + B &=& C + D \nonumber \\
k_1 \left ( A - B \right ) &=& k_2 \left ( C - D \right ) \nonumber \\
C \exp (i k_2 a) + D \exp (-i k_2 a) 
 &=& F \exp (i k_3 a) \nonumber \\ 
k_2 \left ( C \exp (i k_2 a) - D \exp (-i k_2 a) \right )
 &=& k_3 F \exp (i k_3 a) \nonumber 
%
\end{eqnarray}
%

\item[h)] \begin{eqnarray}
%
\mbox{Omr�de 1:} 
\psi_1 &=& A \left (\exp (i k_1 x) 
             +  \frac{3k_1 -k_2}{3k_1 + k_2} \exp (-i k_1 x) \right )
                                                 \nonumber \\
%
\mbox{Omr�de 2:}
\psi_2 &= & A \left ( \frac{4k_1}{3k_1 + k_2} \exp (i k_2 x) 
            +  \frac{2k_1}{3k_1 + k_2} \exp (-i k_2 x) \right )
                               \nonumber \\
%
\mbox{Omr�de 3:}
\psi_3 &= &A \frac{6k_1}{3k_1 + k_2} \exp (i k_3 x)
                      \nonumber 
%
\end{eqnarray}
%
 
\item[i)]
 \[
            R = \left |\frac{B}{A} \right |^2 
              = \left ( \frac{3k_1 -k_2}{3k_1 + k_2} \right )^2 
            \quad 
            T = \left |\frac{F}{A}\right |^2 \frac{k_3}{k_1} 
              = \frac{36k_1 k_3}{(3k_1 + k_2)^2} 
          \]
%
\end{itemize}


%
%\subsubsection*{Oppgave 6.9}

%\subsubsection*{Oppgave 6.10}

%\subsubsection*{Oppgave 6.11}





\clearemptydoublepage
\chapter{Quantization of angular momentum}
\label{chap:banespinnkvantisering}
\begin{quotation}
Though this be madness, yet there is method in it.
{\em William Shakespeare, Hamlet, act.~II, sec.~2}
\end{quotation}




\section{Introduksjon og motivasjon}

Vi skal n\aa\ bevege oss vekk fra de enkle skjematiske
potensialene til 'real life' slik bare Coulomb potensialet
kan berette oss om. 
Hydrogenatomet og dets ve og vel blir dermed sentralt, ikke
bare fordi det er et en-partikkel problem (elektronet som kretser
rundt et proton) som kan l\o ses eksakt og dermed gi oss
nyttig informasjon om et kvantemekanisk system, men ogs\aa\ fordi
vi skal bruke 
den nyvunne innsikt til \aa\ bevege oss mot en forst\aa else av
det periodiske systemet, med Schr\"odingers likning
som v\aa r naturlov.

Vi har flere m\aa lsettinger med dette kapittelet.
\begin{itemize}
  \item Vi skal omskrive den tredimensjonale Schr\"odingers likning  
        til sf\ae riske koordinater,
     \be
        x=rsin\theta cos\phi,  
      \ee
      \be
        y=rsin\theta sin\phi,
     \ee
     \be
        z=rcos\theta.
     \ee
Grunnen til dette skyldes at dersom vi \o nsker \aa\ l\o se Schr\"odingers likning  med
kartesiske koordinater, blir Coulombpotensialet
\be
    \frac{e^2}{4\pi\epsilon_0r}=\frac{e^2}{4\pi\epsilon_0\sqrt{x^2+y^2+z^2}},
\ee
siden $r=\sqrt{x^2+y^2+z^2}$. Det betyr igjen at n\aa r vi skal
l\o se Schr\"odingers likning , er vi ikke istand til \aa\ finne en separabel l\o sning av
typen
\be
    \psi(x,y,z)=\psi(x)\psi(y)\psi(z).
\ee
Derimot, velger vi sf\ae riske koordinater er vi istand til
\aa\ finne egenfunksjonen p\aa\ forma
\be
    \psi(r,\theta,\phi)=R(r)\Theta(\theta)\Phi(\phi).
\ee
Det skyldes det faktum at Coulombpotensialet avhenger kun av $r$.
Slike potensialer kalles sentralsymmetriske.  
\item N\aa r vi l\o ser Schr\"odingers likning  for hydrogenatomet, skal vi vise at 
banespinnet er automatisk kvantisert. Dette var et av de store 
gjennomslag for Schr\"odingers likning . Vi trenger ikke \aa\ postulere
slik Bohr gjorde at $L=n\hbar$.
\item L\o sningen av Schr\"odingers likning  gir oss ogs\aa\ b\o lgefunksjonen, og i motsetning til Bohrs
modell, er vi stand til \aa\ rekne ut andre egenskaper ved systemet enn
bare dets energi. Vi skal ogs\aa\ vise at Schr\"odingers likning  gir oss de korrekte energiene.
\end{itemize}




\section{Kvantisering av banespinn}

I dette avsnittet skal vi se at n\aa r vi g\aa r vi over
til polarkoordinater kan vi skrive Schr\"odingers likning  for hydrogenatomet 
som en likning som er separabel i polarkoordinatene $r$, $\theta$ og
$\phi$. Grunnen var, som vi nevnte innledningsvis i dette
kapitlet, at Coulombpotensialet avhenger kun av radien $r$.
I tillegg skal vi vise at Schr\"odingers likning  gir oss kvantisering av banespinnet $L$.
Det er helt nytt i forhold til Bohrs teori, som rett og slett
postulerte at banespinnet var kvantisert.

F\o r vi kommer s\aa\ langt, skal vi f\o rst benytte 
oss av kommuteringsrelasjonene som vi diskuterte i kapittel 5
for \aa\ vise at banespinnets projeksjon langs ulike akser,
dvs.~$L_x$, $L_y$ og $L_z$ 
ikke er uavhengige st\o rrelser slik vi ville ha forventa fra klassisk
fysikk.
Roten til dette ligger i Heisenbergs uskarphets relasjon. For \aa\ se dette
la oss huske f\o rst 
at rekkef{\o}lgen av operatorene i et uttrykk kan ha betydning.
Det s\aa\ vi fra virkningen av operatorene $\OP{x} \OP{p}_x$ 
og $\OP{p}_x \OP{x}$
p{\aa} en vilk{\aa}rlig tilstand $\Phi$. Resultatene  blir forskjellige.
Dette kan vi uttrykke vha.~kommuteringsrelasjoner som
\[
    [ \OP{x},\OP{p}_x]=i\hbar,
\]
som impliserer  
at dersom to operatorer ikke kommuter,
s\aa\ kan vi ikke bestemme begge skarpt. 
Med skarpt meiner vi at uskarpheten til en st\o rrelse $\OP{A}$ er null,
dvs.~$\Delta A=\sqrt{\langle \OP{A}^2\rangle - \langle \OP{A}\rangle^2}=0$.
Sagt litt annerledes, dersom to fysiske operatorer kan ha
veldefinerte forventningsverdier uten uskarphet samtidig {\bf og samme
b\o lgefunksjon}, da kommuterer operatorene.

{\em Motsatt vei, kan vi si at dersom to operatorer  kommuterer,
da har de samme egenfunksjoner og skarpe egenverdier.}



Banespinnet er definert som ${\bf L}={\bf r}\times {\bf p}$, noe som igjen
gir oss banespinnets ulike projeksjoner
\be
L_x=-i\hbar(y\frac{\partial }{\partial z}-z\frac{\partial }{\partial y})=
yp_z-zp_y,
\ee
\be
L_y=-i\hbar(z\frac{\partial }{\partial x}-x\frac{\partial }{\partial z})= zp_x-xp_z,
\ee
\be
L_z=-i\hbar(x\frac{\partial }{\partial y}-y\frac{\partial }{\partial x})=xp_y-yp_x.
\ee
Dersom vi n\aa\ skal anvende kvantemekanikk, bruker vi at den kvantemekaniske
operatoren for banespinnet er gitt ved
$\OP{L}=-i\hbar \OP{r}\times \OP{p}$. 
Sp\o rsm\aa let som vi stiller oss er: hvordan ser kommuteringsrelasjonen
ut for de ulike projeksjonene av banespinnet? 
La oss se p\aa\ f\o lgende kommutator
\be
   [\OP{L}_x,\OP{L}_y]=\OP{L}_x\OP{L}_y-\OP{L}_y\OP{L}_x,
\ee
som vi kan omskrive som
\be
   (\OP{y}\OP{p}_z-\OP{z}\OP{p}_y)(\OP{z}\OP{p}_x-\OP{x}\OP{p}_z)-
    (\OP{z}\OP{p}_x-\OP{x}\OP{p}_z)(\OP{y}\OP{p}_z-\OP{z}\OP{p}_y),
\ee
og pynter vi litt til (se oppgave 5.3)
\be
   \OP{p}_z\OP{z}(\OP{y}\OP{p}_x-\OP{x}\OP{p}_y)+
   \OP{z}\OP{p}_z(\OP{p}_y\OP{x}-\OP{p}_x\OP{y})=
   [\OP{p}_z,\OP{z}](\OP{y}\OP{p}_x-\OP{x}\OP{p}_y)=i\hbar \OP{L}_z=[\OP{L}_x,\OP{L}_y].
\ee
Det betyr at siden $L_x$ og $L_y$ ikke kommuterer, s\aa\ kan vi ikke
bestemme begge skarpt.
P\aa\ tilsvarende vis finner vi at
\be
   [\OP{L}_z,\OP{L}_x]=i\hbar \OP{L}_y,
\ee
og 
\be
   [\OP{L}_y,\OP{L}_z]=i\hbar \OP{L}_x.
\ee

Her skiller vi lag med klassisk fysikk, Kvantemekanisk betyr det at 
siden de ulike projeksjonene ikke kommuterer, s\aa\ kan ikke
$L_x$, $L_y$ og $L_z$ ha samme egenfunskjoner, ei heller kunne
bestemmes skarpt samtidig. Kun en av dem kan bestemmes skarpt. 


Dersom vi bruker banespinnet kvadrert gitt ved
\be
   \OP{L}^2=\OP{L}^2_x+\OP{L}^2_y+\OP{L}^2_z
\ee
kan vi vise at $z$-komponenten av $\OP{L}$ kommuterer med $\OP{L}^2$.
Det betyr at disse to operatorene har felles egenfunksjoner. Dette ser vi
ved \aa\ rekne ut kommutatoren (se ogs\aa\ oppgave 2.6)
\begin{eqnarray} 
   [\OP{L}_z,\OP{L}^2]&=&[\OP{L}_z,\OP{L}^2_x]+[\OP{L}_z,\OP{L}^2_y]+[\OP{L}_z,\OP{L}^2_z] \\ \nonumber
   &=&[\OP{L}_z,\OP{L}_x]\OP{L}_x+\OP{L}_x[\OP{L}_z,\OP{L}_x]+
    [\OP{L}_z,\OP{L}_y]\OP{L}_y+\OP{L}_y[\OP{L}_z,\OP{L}_y]\\ \nonumber 
   &=&   i\hbar(\OP{L}_y\OP{L}_x+\OP{L}_x\OP{L}_y)-
   i\hbar(\OP{L}_x\OP{L}_y+\OP{L}_y\OP{L}_x)=0.
\end{eqnarray}

Vi skal n\aa\ g\aa\ over til polarkoordinater og vise at Schr\"odingers likning  kan separeres
i tre ulike likninger som kun avhenger av $r$, $\theta$ og $\phi$ 
definert ved
\[
   x=rsin\theta cos\phi,  
\]
\[
   y=rsin\theta sin\phi,
\]
\[
   z=rcos\theta.
\]

I polarkoordinater\footnote{Ulike uttrykk for $\nabla$, og banespinnets
projeksjoner finner dere i boka kap.~6-2.} har vi at operatoren
$\nabla^2$ som inng\aa r i leddet for den kinetiske energien 
\[
   -\frac{\hbar^2\nabla^2}{2m},
\]
kan skrives som (se Rottmann) 
\be
\nabla^2=\frac{1}{r^2sin\theta}\left\{\frac{\partial }{\partial r}
          (r^2sin(\theta)\frac{\partial }{\partial r}) +
          \frac{\partial }{\partial \theta}(sin(\theta)
          \frac{\partial }{\partial \theta})+
          \frac{\partial }{\partial \phi}(\frac{1}{sin(\theta)}
          \frac{\partial }{\partial \phi}) \right\}.
   \label{eq:nabla}
\ee
Helt tilsvarende finner vi at $z$, $x$ og $y$ komponentene til banespinnet
er gitt ved
\be
\OP{L}_z=-i\hbar\frac{\partial }{\partial \phi}
\ee
\be
  \OP{L}_x=i\hbar\left(sin(\phi) \frac{\partial }{\partial\theta}
                   +cot(\theta)cos(\phi)
                  \frac{\partial }{\partial \phi}\right)
\ee
\be
  \OP{L}_y=i\hbar\left(-cos(\phi) \frac{\partial }{\partial\theta}
                   +cot(\theta)sin(\phi)
                  \frac{\partial }{\partial \phi}\right)
\ee
som gir at $\OP{L}^2$ blir
\be
   \OP{L}^2=-\hbar^2\frac{1}{sin(\theta)}\left\{
          \frac{\partial }{\partial \theta}(sin(\theta)
          \frac{\partial }{\partial \theta})+
          \frac{\partial }{\partial \phi}(\frac{1}{sin(\theta)}
          \frac{\partial }{\partial \phi}) \right\}  
      \label{eq:l2}
\ee
{\em Legg merke} til at likning (\ref{eq:nabla}) for $\nabla^2$ 
inneholder $\OP{L}^2$! Det skal vi benytte oss av n\aa r vi n\aa\ skal
omskrive Schr\"odingers likning .
Schr\"odingers likning er gitt ved 
\be
   \left(-\frac{\hbar^2}{2m}\nabla^2+V(r)\right)\psi(r,\theta,\phi)=
    E\psi(r,\theta,\phi),
\ee
hvor $\psi(r,\theta,\phi)$ er egenfunksjonen. 
Innsetting av uttrykket for $\nabla^2$ i polarkoordinater gir
\begin{eqnarray}
-\frac{\hbar^2}{2m}\frac{1}{r^2sin\theta}\left\{\frac{\partial }{\partial r}
          (r^2sin(\theta)\frac{\partial }{\partial r}) +
          \frac{\partial }{\partial \theta}(sin(\theta)
          \frac{\partial }{\partial \theta})+
          \frac{\partial }{\partial \phi}(\frac{1}{sin(\theta)}
          \frac{\partial }{\partial \phi}) \right\}\psi(r,\theta,\phi)& &\nonumber \\
+V(r)\psi(r,\theta,\phi)=
    E\psi(r,\theta,\phi)& &
\end{eqnarray}
som vi kan omskrive til 
\be
\left(-\frac{\hbar^2}{2m}\frac{1}{r^2}\frac{\partial }{\partial r}
          (r^2\frac{\partial }{\partial r})
    +\frac{\OP{L}^2}{2mr^2}+V(r)\right)\psi(r,\theta,\phi)=
    E\psi(r,\theta,\phi).
\ee
Legg merke til at $L^2$ er p\aa\ operatorform. Vi har heller ikke
spesifisert potensialet $V(r)$.
Vi kan definere hamiltonfunksjonen v\aa r som 
\be
   \OP{H}=-\frac{\hbar^2}{2m}\frac{1}{r^2}\frac{\partial }{\partial r}
          (r^2\frac{\partial }{\partial r})
    +\frac{\OP{L}^2}{2mr^2}+V(r)=-\frac{\hbar^2}{2m}\left(\frac{\partial^2 }{\partial r^2}+\frac{2}{r}\frac{\partial}{\partial r}\right)
    +\frac{\OP{L}^2}{2mr^2}+V(r).
\ee
Vi legger merke til at $\OP{H}$ og $\OP{L}^2$ kommuterer
\be
   \left[\OP{H},\OP{L}^2\right]=\left[-\frac{\hbar^2}{2m}\left(
      \frac{\partial^2 }{\partial r^2}+\frac{2}{r}\frac{\partial}{\partial r}
     \right),\OP{L}^2\right]+\frac{1}{2mr^2}\left[\OP{L}^2,\OP{L}^2\right]+\left[V(r),\OP{L}^2\right]=0.
\ee
P\aa\ tilsvarende vis, siden $\OP{L}^2$ og $\OP{L}_z$ kommuterer, har vi 
ogs\aa\ at
\be
   \left[\OP{H},\OP{L}_z\right]=0.
\ee
Det betyr at $\OP{H}$, $\OP{L}^2$ og $\OP{L}_z$
har felles egenfunksjon $\psi$ og skarpe egenverdier gitt ved f\o lgende
likninger
\begin{center}
\shabox{\parbox{14cm}{\be
   \OP{H}\psi=E\psi,
\ee
\be
  \OP{L}^2\psi=E_L\psi,
\ee
og 
\be
  \OP{L}_z\psi=E_{L_{z}}\psi.
\ee
}}\end{center}
Det er disse egenverdiene vi skal bestemme. Spesielt skal vi vise
at energien $E$ er gitt ved Bohrs formel og at banespinnets egenverdi
$E_L$ og dets projeksjon p\aa\ $z$-aksen er kvantisert. Det siste
faller automagisk ut fra Schr\"odingers likning , og vi trenger ikke \aa\ postulere
kvantisering av banespinn slik Bohr gjorde ved \aa\ sette
$L=n\hbar$. 

At vi f\aa r tre kvantetall, et fra hver egenverdi, skyldes at vi trenger
tre uavhengige koordinater for \aa\ beskrive elektronets posisjon.
Det at vi har tre skarpe egenverdier betyr ogs\aa\ at disse 
st\o rrelsene er s\aa kalla bevegelseskonstanter, de er bevarte.
Klassisk har vi og\aa\ at banespinnet og energien er bevart
for et sentralsymmetrisk potensial. 


Vi antar s\aa\ at egenfunksjonen er separabel
\be
   \psi(r,\theta,\phi)=R(r)P(\theta)F(\phi)=RPF,
\ee
som innsatt i Schr\"odingers likning  gir 
\be
\left(-\frac{\hbar^2}{2m}\frac{1}{r^2}\frac{\partial }{\partial r}
          (r^2\frac{\partial }{\partial r})
    +\frac{\OP{L}^2}{2mr^2}+V(r)\right)RPF=ERPF
\ee
Vi multipliserer begge sider med 
\be
-\frac{2mr^2}{RPF\hbar^2},
\ee
og finner n\aa r vi setter inn for $\OP{L}^2$ at  
\be
\frac{1}{R}\frac{\partial }{\partial r}
          (r^2\frac{\partial R}{\partial r}) +
          \frac{1}{Psin(\theta)}\frac{\partial }{\partial \theta}(sin(\theta)
          \frac{\partial P}{\partial \theta})+
          \frac{1}{Fsin^2(\theta)}
          \frac{\partial^2 F }{\partial \phi^2}
-\frac{2mr^2}{\hbar^2}V(r)+\frac{2mr^2}{\hbar^2}E=0.
\ee
Den radielle delen er n\aa\ skilt ut. Det samme gjelder for vinkeldelen, slik at n\aa r vi setter inn Coulombpotensialet $V(r)=-ke^2/r$,
hvor $k=1/4\pi\epsilon_0$ er en konstant, s\aa\ finner vi 
\be
\frac{1}{R}\frac{\partial }{\partial r}
          (r^2\frac{\partial R}{\partial r}) 
+\frac{2mrke^2}{\hbar^2}+\frac{2mr^2}{\hbar^2}E
=
          -\frac{1}{Psin(\theta)}\frac{\partial }{\partial \theta}(sin(\theta)
          \frac{\partial P}{\partial \theta})-
          \frac{1}{Fsin^2(\theta)}
          \frac{\partial^2 F }{\partial \phi^2}.
\ee
Vi ser at venstre og h\o yre side er uavhengige av hverandre, og vi kan sette
\be
\frac{1}{R}\frac{\partial }{\partial r}
          (r^2\frac{\partial R}{\partial r}) 
+\frac{2mrke^2}{\hbar^2}+\frac{2mr^2}{\hbar^2}E
=C_r,
\ee
og
\be 
    C_r=-\frac{1}{Psin(\theta)}\frac{\partial }{\partial \theta}(sin(\theta)
     \frac{\partial P}{\partial \theta})-
     \frac{1}{Fsin^2(\theta)}
     \frac{\partial^2 F }{\partial \phi^2}),
\ee
som vi omskriver til
\be
   -\frac{1}{F}\frac{\partial^2 F }{\partial \phi^2}=
 C_rsin^2(\theta)+\frac{sin(\theta)}{P}\frac{\partial }{\partial \theta}(sin(\theta)
     \frac{\partial P}{\partial \theta}).
\ee
$C_r$ er en konstant. 
Vi ser igjen at sistnevnte likning har ei venstre og h\o yre side
som er uavhengige av hverandre slik at vi kan sette opp tre uavhengige
likninger
\begin{center}
\shabox{\parbox{14cm}{\be
   \frac{1}{F}\frac{\partial^2 F }{\partial \phi^2}=-C^2_{\phi}
\ee
\be
   C_rsin^2(\theta)P+sin(\theta)\frac{\partial }{\partial \theta}(sin(\theta)
     \frac{\partial P}{\partial \theta})=C_{\phi}^2P,
\ee
og 
\be
\frac{1}{R}\frac{\partial }{\partial r}
          (r^2\frac{\partial R}{\partial r}) 
+\frac{2mrke^2}{\hbar^2}+\frac{2mr^2}{\hbar^2}E
=C_r
\label{eq:radiell}
\ee
}}\end{center}


Disse tre likningene skal gi oss tre kvantetall.
L\o sningen for $F(\phi)$ er p\aa\ forma
\be
F(\phi)=Aexp\left(im_l\phi\right),
\ee
med $m_l^2=C_{\phi}^2$ og $A$ en normeringskonstant.
Husker vi tilbake til definisjonen p\aa\ $L_z=-i\hbar\partial/\partial\phi$,
har vi
\be
   L_zF(\phi)=-i\hbar\frac{\partial }{\partial \phi}F(\phi)=\hbar m_lF(\phi),
\ee
som er p\aa\ forma $A\psi = a\psi$. 

Vi har ikke sagt noe om hvilke verdier $m_l$ kan ha. Kan de f.eks.~v\ae re
kontinuerlige eller tar de kun diskrete verdier? 
Den $\phi$-avhengige delen av den totale egenfunksjonen 
gir oss et hint. Siden det
samme punktet i rommet er representert ved 
\be
   \phi=\phi+2k\pi,
\ee
med $k$ en konstant, 
m\aa\ vi ha
\be
   F(\phi)=F(\phi+2\pi),
\ee
som betyr 
\be
    exp\left(im_l\phi\right)=exp\left(im_l(\phi+2\pi) \right),
\ee
eller at
\be
   exp\left(im_l2\pi \right)=1.
\ee
Sistnevnte relasjon er oppfylt kun dersom $m_l$ er et heltall. 
Projeksjonen av banespinn langs $z$-aksen er alts\aa\ kvantisert! Vi kan 
ikke tillate alle mulige verdier for $m_l$. 

Normalisering gir
\be
   \int_0^{2\pi}A^*exp\left(-im_l\phi\right)Aexp\left(im_l\phi\right)d\phi=1,
\ee
som gir
\be
   A^*A\int_0^{2\pi}d\phi=2\pi A^*A=1,
\ee
slik at funksjonen $F$ blir
\be
   F(\phi)=\frac{1}{\sqrt{2\pi}}exp\left(im_l\phi\right).
\ee
Vi skal s\aa\ se p\aa\ l\o sninga av likninga som kun avhenger
av $\theta$. Definerer vi $x=cos(\theta)$, med tilh\o rende
\be
    \frac{dx}{d\theta}=-sin(\theta)=-\sqrt{1-x^2},
\ee
har vi at 
\be
   \frac{dP}{d\theta}=\frac{dx}{d\theta}\frac{dP}{dx}=-\sqrt{1-x^2}\frac{dP}{dx},
\ee
slik at vi kan omskrive likning (4.74) til
\be
C_r(1-x^2)P-m_l^2P+(1-x^2)\frac{d}{dx}\left((1-x^2)\frac{dP}{dx}\right)=0.
\ee
Denne differensiallikningen finner dere bla.~i Rottman (se under spesielle
funksjoner, avsnitt 6) og har som l\o sning de s\aa kalte Legendre
og tilordna Legendre polynomer).
Tar vi tilfellet $m_l=0$, kan vi utvikle $P$ som en potensrekke
\be
    P(x)=\sum_{n=0}^{\infty}a_nx^n\hspace{0.1cm} -1 \le x \le 1,
\ee
som ved innsetting gir en rekursjonsformel for koeffisientene $a_n$
gitt ved
\be
    \frac{a_{n+2}}{a_n}=\frac{n(n+1)+C_r}{(n+1)(n+2)}.
\ee
Dette forholdet g\aa r som $n/(n+2)$ n\aa r $n$ er stor, dvs.~samme forhold
som for funksjonen $\sum_nx^n/n=ln(1-x)$. Funksjonen divergerer n\aa r 
$x=1$, som svarer til $\theta=0$. Dette kan bare unng\aa s dersom
rekka bryter av for en endelig verdi av $n=l$. Det skjer bare dersom
\be
    C_r=-l(l+1) ,
\ee
hvor $l$ svarer alts\aa\ til den gitte verdien av $n$ hvor rekka
bryter av. Merk at $l$ er et ikke-negativt heltall!
Det tilh\o rende polynomet blir av grad $l$ og innsetting av 
$C_r=-l(l+1)$ i likning (4.87) gir oss
\be
l(l+1)(1-x^2)P-(1-x^2)\frac{d}{dx}\left((1-x^2)\frac{dP}{dx}\right)=0,
\ee
som kalles Legendre differensiallikning og har som l\o sning de 
ovennevnte Legendre polynomene. 
Den generelle likningen blir da en funksjon av $l$ og $m_l$, dvs
\be
m_l^2P+l(l+1)(1-x^2)P-(1-x^2)\frac{d}{dx}\left((1-x^2)\frac{dP}{dx}\right)=0,
\ee
hvis l\o sning er gitt ved de s\aa kalla sf\ae riske harmoniske
funksjoner
\be
    Y_{lm_l}(\theta,\phi)=P(\theta)F(\phi)=\sqrt{\frac{(2l+1)(l-m_l)!}{4\pi (l+m_l)!}}
                      P_l^{m_l}(cos(\theta))\exp{(im_l\phi)},
\ee
hvor polynomet $P_l^{m_l}$ er det s\aa kalte tilordna Legendre polynomet.
Vi omskriver siste likning som
\be
   Y_{lm_l}(\theta,\phi)=sin^{|m_l|}(\theta) \times (\mathrm{polynom}(cos\theta))\exp{(im_l\phi)},
\ee
med eksempler 
\be
   Y_{00}=\sqrt{\frac{1}{4\pi}},
\ee
for $l=m_l=0$, 
\be
   Y_{10}=\sqrt{\frac{3}{4\pi}}cos(\theta),
\ee
for $l=1$ og $m_l=0$, 
\be
   Y_{1\pm 1}=\sqrt{\frac{3}{8\pi}}sin(\theta)exp(\pm i\phi),
\ee
for  $l=1$ og $m_l=\pm 1$, 
\be
   Y_{20}=\sqrt{\frac{5}{16\pi}}(3cos^2(\theta)-1)
\ee
for $l=2$ og $m_l=0$ osv. 

Vi kan oppsummere resultatene hittil p\aa\ f\o lgende vis
\begin{center}
\shabox{\parbox{14cm}{
For den delen av egenfunksjonen som avhenger av vinklene $\theta$ og
$\phi$ ser vi at banespinnets projeksjon langs $z$-aksen er kvantisert
og gitt ved kvantetallet $m_l$. Tilsvarende er banespinnet gitt ved
kvantetallet $l$. Vi har f\o lgende egenskaper
\begin{enumerate}\item  \[ l \ge 0\]
\item \[ l=0,1,2,\dots \]
\item \[ m_l=-l,-l+1,\dots, l-1,l \]
\item \[ \OP{L}^2P(\theta)F(\phi)=\OP{L}^2Y_{lm_l}(\theta,\phi)=l(l+1)\hbar^2Y_{lm_l}(\theta,\phi) \]
\item \[ \OP{L}_zP(\theta)F(\phi)=\OP{L}_zY_{lm_l}(\theta,\phi)=m_l\hbar Y_{lm_l}(\theta,\phi) \]
\end{enumerate}}}\end{center}

Merk ogs\aa\ at $l$ og $m_l$ kan v\ae re null! Vi kan ha null banespinn!
I Bohrs atommodell
kan vi ikke ha null banespinn, siden heltallet $n$ i $L=n\hbar$ er st\o rre
eller lik 1. I klassisk fysikk, dersom vi tenker p\aa\ en planetbevegelse
(jorda rundt sola) kan vi heller ikke ha null banespinn. Dersom det var
tilfelle ville det bety at jorda kunne oscillert fram og tilbake langs ei rett
linje 
med sola som sentrum. Jorda m\aa tte da passere massesenterpunktet, 
som er sola. At banespinnet er null er et kvantemekanisk
resultat, og det finnes et vell av eksperiment som st\o tter opp
om dette resultatet. 

Dersom vi vil lage oss et bilde av null banespinn for hydrogenatomet,
kan vi forestille oss at elektronet oscillerer fram og tilbake gjennom
protonet i et plan. For \aa\ f\aa\ til det m\aa\ vi nok bruke
b\o lgeegenskapen til materien, og forestille oss protonet og elektronet
som materieb\o lger som passerer hverandre. 

Vi har n\aa\ fiksert to kvantetall som svarer 
til bevaring av to st\o rrelser, 
det totale banespinnet og dets projeksjon langs $z$-aksen. 
L\o sning av den radielle Schr\"odingers likning  vil gi oss ytterligere et kvantetall.

\section{Den radielle Schr\"odinger likning}

Dersom vi n\aa\ ser tilbake p\aa\ likning (\ref{eq:radiell}), s\aa\ kan vi
omskrive den vha.~$L^2\psi(r,\theta,\phi)=\hbar^2l(l+1)\psi(r,\theta,\phi)$ til 
\be
-\frac{\hbar^2 r^2}{2m}\left(\frac{\partial }{\partial r}
          (r^2\frac{\partial R(r)}{\partial r})\right) 
-\frac{ke^2}{r}R(r)+\frac{\hbar^2l(l+1)}{2mr^2}R(r)=ER(r).
\ee
Det er vanlig \aa\ l\o se den med hensyn p\aa\ $u(r)=rR(r)$ slik at vi kan
omskrive forrige likning til
\be
-\frac{\hbar^2}{2m}\frac{\partial^2 u(r)}{\partial r^2}-
\left(\frac{ke^2}{r}-\frac{\hbar^2l(l+1)}{2mr^2}\right)u(r)=Eu(r).
\ee
Sammenlikner vi  med det endimensjonale tilfellet kan vi si at den radielle Schr\"odingers likning 
er ekvivalent med en en-dimensjonal bevegelse under et effektivt potensial
\be
V_{eff}(r)=-\frac{ke^2}{r}+\frac{\hbar^2l(l+1)}{2mr^2},
\ee
hvor leddet med $l(l+1)$ er et sentrifugal potensial da den
tilsvarende 'kraft' 
\[
    F=-\frac{\partial}{\partial r}\left(\frac{\hbar^2l(l+1)}{2mr^2}\right)
\]
er positiv og peker dermed bort fra origo. 
Figurene \ref{41} og \ref{42} viser dette potensialet for ulike verdier av
$l$. Figur \ref{41} er for sm\aa\ verdier av $r$, mens \ref{42} 
gir potensialet for
st\o rre verdier av $r$.
\begin{figure}[h]
\begin{center}
% GNUPLOT: LaTeX picture with Postscript
\begingroup%
  \makeatletter%
  \newcommand{\GNUPLOTspecial}{%
    \@sanitize\catcode`\%=14\relax\special}%
  \setlength{\unitlength}{0.1bp}%
{\GNUPLOTspecial{!
%!PS-Adobe-2.0 EPSF-2.0
%%Title: veff.tex
%%Creator: gnuplot 3.7 patchlevel 0.2
%%CreationDate: Fri Mar 17 12:34:22 2000
%%DocumentFonts: 
%%BoundingBox: 0 0 360 216
%%Orientation: Landscape
%%EndComments
/gnudict 256 dict def
gnudict begin
/Color false def
/Solid false def
/gnulinewidth 5.000 def
/userlinewidth gnulinewidth def
/vshift -33 def
/dl {10 mul} def
/hpt_ 31.5 def
/vpt_ 31.5 def
/hpt hpt_ def
/vpt vpt_ def
/M {moveto} bind def
/L {lineto} bind def
/R {rmoveto} bind def
/V {rlineto} bind def
/vpt2 vpt 2 mul def
/hpt2 hpt 2 mul def
/Lshow { currentpoint stroke M
  0 vshift R show } def
/Rshow { currentpoint stroke M
  dup stringwidth pop neg vshift R show } def
/Cshow { currentpoint stroke M
  dup stringwidth pop -2 div vshift R show } def
/UP { dup vpt_ mul /vpt exch def hpt_ mul /hpt exch def
  /hpt2 hpt 2 mul def /vpt2 vpt 2 mul def } def
/DL { Color {setrgbcolor Solid {pop []} if 0 setdash }
 {pop pop pop Solid {pop []} if 0 setdash} ifelse } def
/BL { stroke userlinewidth 2 mul setlinewidth } def
/AL { stroke userlinewidth 2 div setlinewidth } def
/UL { dup gnulinewidth mul /userlinewidth exch def
      10 mul /udl exch def } def
/PL { stroke userlinewidth setlinewidth } def
/LTb { BL [] 0 0 0 DL } def
/LTa { AL [1 udl mul 2 udl mul] 0 setdash 0 0 0 setrgbcolor } def
/LT0 { PL [] 1 0 0 DL } def
/LT1 { PL [4 dl 2 dl] 0 1 0 DL } def
/LT2 { PL [2 dl 3 dl] 0 0 1 DL } def
/LT3 { PL [1 dl 1.5 dl] 1 0 1 DL } def
/LT4 { PL [5 dl 2 dl 1 dl 2 dl] 0 1 1 DL } def
/LT5 { PL [4 dl 3 dl 1 dl 3 dl] 1 1 0 DL } def
/LT6 { PL [2 dl 2 dl 2 dl 4 dl] 0 0 0 DL } def
/LT7 { PL [2 dl 2 dl 2 dl 2 dl 2 dl 4 dl] 1 0.3 0 DL } def
/LT8 { PL [2 dl 2 dl 2 dl 2 dl 2 dl 2 dl 2 dl 4 dl] 0.5 0.5 0.5 DL } def
/Pnt { stroke [] 0 setdash
   gsave 1 setlinecap M 0 0 V stroke grestore } def
/Dia { stroke [] 0 setdash 2 copy vpt add M
  hpt neg vpt neg V hpt vpt neg V
  hpt vpt V hpt neg vpt V closepath stroke
  Pnt } def
/Pls { stroke [] 0 setdash vpt sub M 0 vpt2 V
  currentpoint stroke M
  hpt neg vpt neg R hpt2 0 V stroke
  } def
/Box { stroke [] 0 setdash 2 copy exch hpt sub exch vpt add M
  0 vpt2 neg V hpt2 0 V 0 vpt2 V
  hpt2 neg 0 V closepath stroke
  Pnt } def
/Crs { stroke [] 0 setdash exch hpt sub exch vpt add M
  hpt2 vpt2 neg V currentpoint stroke M
  hpt2 neg 0 R hpt2 vpt2 V stroke } def
/TriU { stroke [] 0 setdash 2 copy vpt 1.12 mul add M
  hpt neg vpt -1.62 mul V
  hpt 2 mul 0 V
  hpt neg vpt 1.62 mul V closepath stroke
  Pnt  } def
/Star { 2 copy Pls Crs } def
/BoxF { stroke [] 0 setdash exch hpt sub exch vpt add M
  0 vpt2 neg V  hpt2 0 V  0 vpt2 V
  hpt2 neg 0 V  closepath fill } def
/TriUF { stroke [] 0 setdash vpt 1.12 mul add M
  hpt neg vpt -1.62 mul V
  hpt 2 mul 0 V
  hpt neg vpt 1.62 mul V closepath fill } def
/TriD { stroke [] 0 setdash 2 copy vpt 1.12 mul sub M
  hpt neg vpt 1.62 mul V
  hpt 2 mul 0 V
  hpt neg vpt -1.62 mul V closepath stroke
  Pnt  } def
/TriDF { stroke [] 0 setdash vpt 1.12 mul sub M
  hpt neg vpt 1.62 mul V
  hpt 2 mul 0 V
  hpt neg vpt -1.62 mul V closepath fill} def
/DiaF { stroke [] 0 setdash vpt add M
  hpt neg vpt neg V hpt vpt neg V
  hpt vpt V hpt neg vpt V closepath fill } def
/Pent { stroke [] 0 setdash 2 copy gsave
  translate 0 hpt M 4 {72 rotate 0 hpt L} repeat
  closepath stroke grestore Pnt } def
/PentF { stroke [] 0 setdash gsave
  translate 0 hpt M 4 {72 rotate 0 hpt L} repeat
  closepath fill grestore } def
/Circle { stroke [] 0 setdash 2 copy
  hpt 0 360 arc stroke Pnt } def
/CircleF { stroke [] 0 setdash hpt 0 360 arc fill } def
/C0 { BL [] 0 setdash 2 copy moveto vpt 90 450  arc } bind def
/C1 { BL [] 0 setdash 2 copy        moveto
       2 copy  vpt 0 90 arc closepath fill
               vpt 0 360 arc closepath } bind def
/C2 { BL [] 0 setdash 2 copy moveto
       2 copy  vpt 90 180 arc closepath fill
               vpt 0 360 arc closepath } bind def
/C3 { BL [] 0 setdash 2 copy moveto
       2 copy  vpt 0 180 arc closepath fill
               vpt 0 360 arc closepath } bind def
/C4 { BL [] 0 setdash 2 copy moveto
       2 copy  vpt 180 270 arc closepath fill
               vpt 0 360 arc closepath } bind def
/C5 { BL [] 0 setdash 2 copy moveto
       2 copy  vpt 0 90 arc
       2 copy moveto
       2 copy  vpt 180 270 arc closepath fill
               vpt 0 360 arc } bind def
/C6 { BL [] 0 setdash 2 copy moveto
      2 copy  vpt 90 270 arc closepath fill
              vpt 0 360 arc closepath } bind def
/C7 { BL [] 0 setdash 2 copy moveto
      2 copy  vpt 0 270 arc closepath fill
              vpt 0 360 arc closepath } bind def
/C8 { BL [] 0 setdash 2 copy moveto
      2 copy vpt 270 360 arc closepath fill
              vpt 0 360 arc closepath } bind def
/C9 { BL [] 0 setdash 2 copy moveto
      2 copy  vpt 270 450 arc closepath fill
              vpt 0 360 arc closepath } bind def
/C10 { BL [] 0 setdash 2 copy 2 copy moveto vpt 270 360 arc closepath fill
       2 copy moveto
       2 copy vpt 90 180 arc closepath fill
               vpt 0 360 arc closepath } bind def
/C11 { BL [] 0 setdash 2 copy moveto
       2 copy  vpt 0 180 arc closepath fill
       2 copy moveto
       2 copy  vpt 270 360 arc closepath fill
               vpt 0 360 arc closepath } bind def
/C12 { BL [] 0 setdash 2 copy moveto
       2 copy  vpt 180 360 arc closepath fill
               vpt 0 360 arc closepath } bind def
/C13 { BL [] 0 setdash  2 copy moveto
       2 copy  vpt 0 90 arc closepath fill
       2 copy moveto
       2 copy  vpt 180 360 arc closepath fill
               vpt 0 360 arc closepath } bind def
/C14 { BL [] 0 setdash 2 copy moveto
       2 copy  vpt 90 360 arc closepath fill
               vpt 0 360 arc } bind def
/C15 { BL [] 0 setdash 2 copy vpt 0 360 arc closepath fill
               vpt 0 360 arc closepath } bind def
/Rec   { newpath 4 2 roll moveto 1 index 0 rlineto 0 exch rlineto
       neg 0 rlineto closepath } bind def
/Square { dup Rec } bind def
/Bsquare { vpt sub exch vpt sub exch vpt2 Square } bind def
/S0 { BL [] 0 setdash 2 copy moveto 0 vpt rlineto BL Bsquare } bind def
/S1 { BL [] 0 setdash 2 copy vpt Square fill Bsquare } bind def
/S2 { BL [] 0 setdash 2 copy exch vpt sub exch vpt Square fill Bsquare } bind def
/S3 { BL [] 0 setdash 2 copy exch vpt sub exch vpt2 vpt Rec fill Bsquare } bind def
/S4 { BL [] 0 setdash 2 copy exch vpt sub exch vpt sub vpt Square fill Bsquare } bind def
/S5 { BL [] 0 setdash 2 copy 2 copy vpt Square fill
       exch vpt sub exch vpt sub vpt Square fill Bsquare } bind def
/S6 { BL [] 0 setdash 2 copy exch vpt sub exch vpt sub vpt vpt2 Rec fill Bsquare } bind def
/S7 { BL [] 0 setdash 2 copy exch vpt sub exch vpt sub vpt vpt2 Rec fill
       2 copy vpt Square fill
       Bsquare } bind def
/S8 { BL [] 0 setdash 2 copy vpt sub vpt Square fill Bsquare } bind def
/S9 { BL [] 0 setdash 2 copy vpt sub vpt vpt2 Rec fill Bsquare } bind def
/S10 { BL [] 0 setdash 2 copy vpt sub vpt Square fill 2 copy exch vpt sub exch vpt Square fill
       Bsquare } bind def
/S11 { BL [] 0 setdash 2 copy vpt sub vpt Square fill 2 copy exch vpt sub exch vpt2 vpt Rec fill
       Bsquare } bind def
/S12 { BL [] 0 setdash 2 copy exch vpt sub exch vpt sub vpt2 vpt Rec fill Bsquare } bind def
/S13 { BL [] 0 setdash 2 copy exch vpt sub exch vpt sub vpt2 vpt Rec fill
       2 copy vpt Square fill Bsquare } bind def
/S14 { BL [] 0 setdash 2 copy exch vpt sub exch vpt sub vpt2 vpt Rec fill
       2 copy exch vpt sub exch vpt Square fill Bsquare } bind def
/S15 { BL [] 0 setdash 2 copy Bsquare fill Bsquare } bind def
/D0 { gsave translate 45 rotate 0 0 S0 stroke grestore } bind def
/D1 { gsave translate 45 rotate 0 0 S1 stroke grestore } bind def
/D2 { gsave translate 45 rotate 0 0 S2 stroke grestore } bind def
/D3 { gsave translate 45 rotate 0 0 S3 stroke grestore } bind def
/D4 { gsave translate 45 rotate 0 0 S4 stroke grestore } bind def
/D5 { gsave translate 45 rotate 0 0 S5 stroke grestore } bind def
/D6 { gsave translate 45 rotate 0 0 S6 stroke grestore } bind def
/D7 { gsave translate 45 rotate 0 0 S7 stroke grestore } bind def
/D8 { gsave translate 45 rotate 0 0 S8 stroke grestore } bind def
/D9 { gsave translate 45 rotate 0 0 S9 stroke grestore } bind def
/D10 { gsave translate 45 rotate 0 0 S10 stroke grestore } bind def
/D11 { gsave translate 45 rotate 0 0 S11 stroke grestore } bind def
/D12 { gsave translate 45 rotate 0 0 S12 stroke grestore } bind def
/D13 { gsave translate 45 rotate 0 0 S13 stroke grestore } bind def
/D14 { gsave translate 45 rotate 0 0 S14 stroke grestore } bind def
/D15 { gsave translate 45 rotate 0 0 S15 stroke grestore } bind def
/DiaE { stroke [] 0 setdash vpt add M
  hpt neg vpt neg V hpt vpt neg V
  hpt vpt V hpt neg vpt V closepath stroke } def
/BoxE { stroke [] 0 setdash exch hpt sub exch vpt add M
  0 vpt2 neg V hpt2 0 V 0 vpt2 V
  hpt2 neg 0 V closepath stroke } def
/TriUE { stroke [] 0 setdash vpt 1.12 mul add M
  hpt neg vpt -1.62 mul V
  hpt 2 mul 0 V
  hpt neg vpt 1.62 mul V closepath stroke } def
/TriDE { stroke [] 0 setdash vpt 1.12 mul sub M
  hpt neg vpt 1.62 mul V
  hpt 2 mul 0 V
  hpt neg vpt -1.62 mul V closepath stroke } def
/PentE { stroke [] 0 setdash gsave
  translate 0 hpt M 4 {72 rotate 0 hpt L} repeat
  closepath stroke grestore } def
/CircE { stroke [] 0 setdash 
  hpt 0 360 arc stroke } def
/Opaque { gsave closepath 1 setgray fill grestore 0 setgray closepath } def
/DiaW { stroke [] 0 setdash vpt add M
  hpt neg vpt neg V hpt vpt neg V
  hpt vpt V hpt neg vpt V Opaque stroke } def
/BoxW { stroke [] 0 setdash exch hpt sub exch vpt add M
  0 vpt2 neg V hpt2 0 V 0 vpt2 V
  hpt2 neg 0 V Opaque stroke } def
/TriUW { stroke [] 0 setdash vpt 1.12 mul add M
  hpt neg vpt -1.62 mul V
  hpt 2 mul 0 V
  hpt neg vpt 1.62 mul V Opaque stroke } def
/TriDW { stroke [] 0 setdash vpt 1.12 mul sub M
  hpt neg vpt 1.62 mul V
  hpt 2 mul 0 V
  hpt neg vpt -1.62 mul V Opaque stroke } def
/PentW { stroke [] 0 setdash gsave
  translate 0 hpt M 4 {72 rotate 0 hpt L} repeat
  Opaque stroke grestore } def
/CircW { stroke [] 0 setdash 
  hpt 0 360 arc Opaque stroke } def
/BoxFill { gsave Rec 1 setgray fill grestore } def
end
%%EndProlog
}}%
\begin{picture}(3600,2160)(0,0)%
{\GNUPLOTspecial{"
gnudict begin
gsave
0 0 translate
0.100 0.100 scale
0 setgray
newpath
1.000 UL
LTb
450 300 M
63 0 V
2937 0 R
-63 0 V
450 740 M
63 0 V
2937 0 R
-63 0 V
450 1180 M
63 0 V
2937 0 R
-63 0 V
450 1620 M
63 0 V
2937 0 R
-63 0 V
450 2060 M
63 0 V
2937 0 R
-63 0 V
608 300 M
0 63 V
0 1697 R
0 -63 V
924 300 M
0 63 V
0 1697 R
0 -63 V
1239 300 M
0 63 V
0 1697 R
0 -63 V
1555 300 M
0 63 V
0 1697 R
0 -63 V
1871 300 M
0 63 V
0 1697 R
0 -63 V
2187 300 M
0 63 V
0 1697 R
0 -63 V
2503 300 M
0 63 V
0 1697 R
0 -63 V
2818 300 M
0 63 V
0 1697 R
0 -63 V
3134 300 M
0 63 V
0 1697 R
0 -63 V
3450 300 M
0 63 V
0 1697 R
0 -63 V
1.000 UL
LTb
450 300 M
3000 0 V
0 1760 V
-3000 0 V
450 300 L
1.000 UL
LT0
3087 1947 M
263 0 V
520 300 M
21 76 V
30 87 V
31 70 V
30 58 V
30 48 V
30 41 V
31 35 V
30 31 V
30 27 V
31 23 V
30 21 V
30 19 V
31 17 V
30 16 V
30 14 V
30 13 V
31 11 V
30 11 V
30 10 V
31 9 V
30 9 V
30 8 V
31 7 V
30 7 V
30 7 V
30 6 V
31 6 V
30 5 V
30 6 V
31 5 V
30 4 V
30 5 V
31 4 V
30 4 V
30 4 V
31 3 V
30 4 V
30 3 V
30 3 V
31 3 V
30 3 V
30 3 V
31 2 V
30 3 V
30 3 V
31 2 V
30 2 V
30 2 V
30 3 V
31 2 V
30 2 V
30 1 V
31 2 V
30 2 V
30 2 V
31 2 V
30 1 V
30 2 V
30 1 V
31 2 V
30 1 V
30 2 V
31 1 V
30 1 V
30 2 V
31 1 V
30 1 V
30 1 V
31 1 V
30 1 V
30 2 V
30 1 V
31 1 V
30 1 V
30 1 V
31 1 V
30 1 V
30 1 V
31 0 V
30 1 V
30 1 V
30 1 V
31 1 V
30 1 V
30 0 V
31 1 V
30 1 V
30 1 V
31 0 V
30 1 V
30 1 V
30 0 V
31 1 V
30 1 V
30 0 V
31 1 V
30 1 V
1.000 UL
LT1
3087 1847 M
263 0 V
640 2060 M
22 -139 V
30 -146 V
31 -113 V
30 -90 V
30 -72 V
31 -58 V
30 -48 V
30 -40 V
31 -33 V
30 -27 V
30 -24 V
30 -20 V
31 -17 V
30 -14 V
30 -13 V
31 -11 V
30 -9 V
30 -8 V
31 -7 V
30 -6 V
30 -6 V
30 -5 V
31 -4 V
30 -3 V
30 -4 V
31 -2 V
30 -3 V
30 -2 V
31 -2 V
30 -2 V
30 -1 V
31 -1 V
30 -2 V
30 -1 V
30 0 V
31 -1 V
30 -1 V
30 0 V
31 -1 V
30 0 V
30 0 V
31 -1 V
30 0 V
30 0 V
30 0 V
31 0 V
30 0 V
30 0 V
31 0 V
30 0 V
30 0 V
31 0 V
30 1 V
30 0 V
30 0 V
31 0 V
30 0 V
30 1 V
31 0 V
30 0 V
30 0 V
31 0 V
30 1 V
30 0 V
31 0 V
30 1 V
30 0 V
30 0 V
31 0 V
30 1 V
30 0 V
31 0 V
30 1 V
30 0 V
31 0 V
30 0 V
30 1 V
30 0 V
31 0 V
30 1 V
30 0 V
31 0 V
30 0 V
30 1 V
31 0 V
30 0 V
30 1 V
30 0 V
31 0 V
30 0 V
30 1 V
31 0 V
30 0 V
1.000 UL
LT2
3087 1747 M
263 0 V
961 2060 M
4 -15 V
30 -85 V
31 -74 V
30 -66 V
30 -57 V
31 -51 V
30 -46 V
30 -40 V
31 -36 V
30 -33 V
30 -29 V
30 -27 V
31 -24 V
30 -22 V
30 -20 V
31 -18 V
30 -17 V
30 -15 V
31 -15 V
30 -13 V
30 -12 V
31 -11 V
30 -10 V
30 -10 V
30 -8 V
31 -9 V
30 -7 V
30 -7 V
31 -7 V
30 -6 V
30 -6 V
31 -6 V
30 -5 V
30 -4 V
30 -5 V
31 -4 V
30 -4 V
30 -4 V
31 -3 V
30 -4 V
30 -3 V
31 -3 V
30 -3 V
30 -2 V
30 -3 V
31 -2 V
30 -3 V
30 -2 V
31 -2 V
30 -2 V
30 -1 V
31 -2 V
30 -2 V
30 -1 V
31 -2 V
30 -1 V
30 -2 V
30 -1 V
31 -1 V
30 -1 V
30 -1 V
31 -2 V
30 -1 V
30 -1 V
31 0 V
30 -1 V
30 -1 V
30 -1 V
31 -1 V
30 0 V
30 -1 V
31 -1 V
30 0 V
30 -1 V
31 -1 V
30 0 V
30 -1 V
30 0 V
31 -1 V
30 0 V
30 -1 V
31 0 V
30 -1 V
1.000 UL
LT3
3087 1647 M
263 0 V
1279 2060 M
19 -39 V
31 -54 V
30 -50 V
30 -45 V
31 -41 V
30 -38 V
30 -35 V
31 -33 V
30 -30 V
30 -28 V
31 -26 V
30 -24 V
30 -22 V
30 -21 V
31 -19 V
30 -19 V
30 -17 V
31 -16 V
30 -15 V
30 -14 V
31 -13 V
30 -13 V
30 -11 V
30 -12 V
31 -10 V
30 -10 V
30 -9 V
31 -9 V
30 -9 V
30 -8 V
31 -7 V
30 -8 V
30 -7 V
30 -6 V
31 -6 V
30 -6 V
30 -6 V
31 -5 V
30 -6 V
30 -5 V
31 -4 V
30 -5 V
30 -4 V
31 -4 V
30 -4 V
30 -4 V
30 -4 V
31 -3 V
30 -3 V
30 -4 V
31 -3 V
30 -3 V
30 -3 V
31 -2 V
30 -3 V
30 -2 V
30 -3 V
31 -2 V
30 -3 V
30 -2 V
31 -2 V
30 -2 V
30 -2 V
31 -2 V
30 -2 V
30 -1 V
30 -2 V
31 -2 V
30 -1 V
30 -2 V
31 -1 V
30 -2 V
stroke
grestore
end
showpage
}}%
\put(3037,1647){\makebox(0,0)[r]{$l=3$}}%
\put(3037,1747){\makebox(0,0)[r]{$l=2$}}%
\put(3037,1847){\makebox(0,0)[r]{$l=1$}}%
\put(3037,1947){\makebox(0,0)[r]{$l=0$}}%
\put(1950,50){\makebox(0,0){$r$ [nm]}}%
\put(100,1180){%
\special{ps: gsave currentpoint currentpoint translate
270 rotate neg exch neg exch translate}%
\makebox(0,0)[b]{\shortstack{$V(r)$ [eV]}}%
\special{ps: currentpoint grestore moveto}%
}%
\put(3450,200){\makebox(0,0){0.2}}%
\put(3134,200){\makebox(0,0){0.18}}%
\put(2818,200){\makebox(0,0){0.16}}%
\put(2503,200){\makebox(0,0){0.14}}%
\put(2187,200){\makebox(0,0){0.12}}%
\put(1871,200){\makebox(0,0){0.1}}%
\put(1555,200){\makebox(0,0){0.08}}%
\put(1239,200){\makebox(0,0){0.06}}%
\put(924,200){\makebox(0,0){0.04}}%
\put(608,200){\makebox(0,0){0.02}}%
\put(400,2060){\makebox(0,0)[r]{100}}%
\put(400,1620){\makebox(0,0)[r]{50}}%
\put(400,1180){\makebox(0,0)[r]{0}}%
\put(400,740){\makebox(0,0)[r]{-50}}%
\put(400,300){\makebox(0,0)[r]{-100}}%
\end{picture}%
\endgroup
\endinput

\caption{Det effektive potensialet for ulike $l$-verdier men for sm\aa\ verdier av $r$.\label{41}}
\end{center}
\end{figure}
Vi skal merke oss f\o lgende i tilknytting disse to figurene.
\begin{figure}[h]
\begin{center}
% GNUPLOT: LaTeX picture with Postscript
\begingroup%
  \makeatletter%
  \newcommand{\GNUPLOTspecial}{%
    \@sanitize\catcode`\%=14\relax\special}%
  \setlength{\unitlength}{0.1bp}%
{\GNUPLOTspecial{!
%!PS-Adobe-2.0 EPSF-2.0
%%Title: veff1.tex
%%Creator: gnuplot 3.7 patchlevel 0.2
%%CreationDate: Fri Mar 17 12:40:00 2000
%%DocumentFonts: 
%%BoundingBox: 0 0 360 216
%%Orientation: Landscape
%%EndComments
/gnudict 256 dict def
gnudict begin
/Color false def
/Solid false def
/gnulinewidth 5.000 def
/userlinewidth gnulinewidth def
/vshift -33 def
/dl {10 mul} def
/hpt_ 31.5 def
/vpt_ 31.5 def
/hpt hpt_ def
/vpt vpt_ def
/M {moveto} bind def
/L {lineto} bind def
/R {rmoveto} bind def
/V {rlineto} bind def
/vpt2 vpt 2 mul def
/hpt2 hpt 2 mul def
/Lshow { currentpoint stroke M
  0 vshift R show } def
/Rshow { currentpoint stroke M
  dup stringwidth pop neg vshift R show } def
/Cshow { currentpoint stroke M
  dup stringwidth pop -2 div vshift R show } def
/UP { dup vpt_ mul /vpt exch def hpt_ mul /hpt exch def
  /hpt2 hpt 2 mul def /vpt2 vpt 2 mul def } def
/DL { Color {setrgbcolor Solid {pop []} if 0 setdash }
 {pop pop pop Solid {pop []} if 0 setdash} ifelse } def
/BL { stroke userlinewidth 2 mul setlinewidth } def
/AL { stroke userlinewidth 2 div setlinewidth } def
/UL { dup gnulinewidth mul /userlinewidth exch def
      10 mul /udl exch def } def
/PL { stroke userlinewidth setlinewidth } def
/LTb { BL [] 0 0 0 DL } def
/LTa { AL [1 udl mul 2 udl mul] 0 setdash 0 0 0 setrgbcolor } def
/LT0 { PL [] 1 0 0 DL } def
/LT1 { PL [4 dl 2 dl] 0 1 0 DL } def
/LT2 { PL [2 dl 3 dl] 0 0 1 DL } def
/LT3 { PL [1 dl 1.5 dl] 1 0 1 DL } def
/LT4 { PL [5 dl 2 dl 1 dl 2 dl] 0 1 1 DL } def
/LT5 { PL [4 dl 3 dl 1 dl 3 dl] 1 1 0 DL } def
/LT6 { PL [2 dl 2 dl 2 dl 4 dl] 0 0 0 DL } def
/LT7 { PL [2 dl 2 dl 2 dl 2 dl 2 dl 4 dl] 1 0.3 0 DL } def
/LT8 { PL [2 dl 2 dl 2 dl 2 dl 2 dl 2 dl 2 dl 4 dl] 0.5 0.5 0.5 DL } def
/Pnt { stroke [] 0 setdash
   gsave 1 setlinecap M 0 0 V stroke grestore } def
/Dia { stroke [] 0 setdash 2 copy vpt add M
  hpt neg vpt neg V hpt vpt neg V
  hpt vpt V hpt neg vpt V closepath stroke
  Pnt } def
/Pls { stroke [] 0 setdash vpt sub M 0 vpt2 V
  currentpoint stroke M
  hpt neg vpt neg R hpt2 0 V stroke
  } def
/Box { stroke [] 0 setdash 2 copy exch hpt sub exch vpt add M
  0 vpt2 neg V hpt2 0 V 0 vpt2 V
  hpt2 neg 0 V closepath stroke
  Pnt } def
/Crs { stroke [] 0 setdash exch hpt sub exch vpt add M
  hpt2 vpt2 neg V currentpoint stroke M
  hpt2 neg 0 R hpt2 vpt2 V stroke } def
/TriU { stroke [] 0 setdash 2 copy vpt 1.12 mul add M
  hpt neg vpt -1.62 mul V
  hpt 2 mul 0 V
  hpt neg vpt 1.62 mul V closepath stroke
  Pnt  } def
/Star { 2 copy Pls Crs } def
/BoxF { stroke [] 0 setdash exch hpt sub exch vpt add M
  0 vpt2 neg V  hpt2 0 V  0 vpt2 V
  hpt2 neg 0 V  closepath fill } def
/TriUF { stroke [] 0 setdash vpt 1.12 mul add M
  hpt neg vpt -1.62 mul V
  hpt 2 mul 0 V
  hpt neg vpt 1.62 mul V closepath fill } def
/TriD { stroke [] 0 setdash 2 copy vpt 1.12 mul sub M
  hpt neg vpt 1.62 mul V
  hpt 2 mul 0 V
  hpt neg vpt -1.62 mul V closepath stroke
  Pnt  } def
/TriDF { stroke [] 0 setdash vpt 1.12 mul sub M
  hpt neg vpt 1.62 mul V
  hpt 2 mul 0 V
  hpt neg vpt -1.62 mul V closepath fill} def
/DiaF { stroke [] 0 setdash vpt add M
  hpt neg vpt neg V hpt vpt neg V
  hpt vpt V hpt neg vpt V closepath fill } def
/Pent { stroke [] 0 setdash 2 copy gsave
  translate 0 hpt M 4 {72 rotate 0 hpt L} repeat
  closepath stroke grestore Pnt } def
/PentF { stroke [] 0 setdash gsave
  translate 0 hpt M 4 {72 rotate 0 hpt L} repeat
  closepath fill grestore } def
/Circle { stroke [] 0 setdash 2 copy
  hpt 0 360 arc stroke Pnt } def
/CircleF { stroke [] 0 setdash hpt 0 360 arc fill } def
/C0 { BL [] 0 setdash 2 copy moveto vpt 90 450  arc } bind def
/C1 { BL [] 0 setdash 2 copy        moveto
       2 copy  vpt 0 90 arc closepath fill
               vpt 0 360 arc closepath } bind def
/C2 { BL [] 0 setdash 2 copy moveto
       2 copy  vpt 90 180 arc closepath fill
               vpt 0 360 arc closepath } bind def
/C3 { BL [] 0 setdash 2 copy moveto
       2 copy  vpt 0 180 arc closepath fill
               vpt 0 360 arc closepath } bind def
/C4 { BL [] 0 setdash 2 copy moveto
       2 copy  vpt 180 270 arc closepath fill
               vpt 0 360 arc closepath } bind def
/C5 { BL [] 0 setdash 2 copy moveto
       2 copy  vpt 0 90 arc
       2 copy moveto
       2 copy  vpt 180 270 arc closepath fill
               vpt 0 360 arc } bind def
/C6 { BL [] 0 setdash 2 copy moveto
      2 copy  vpt 90 270 arc closepath fill
              vpt 0 360 arc closepath } bind def
/C7 { BL [] 0 setdash 2 copy moveto
      2 copy  vpt 0 270 arc closepath fill
              vpt 0 360 arc closepath } bind def
/C8 { BL [] 0 setdash 2 copy moveto
      2 copy vpt 270 360 arc closepath fill
              vpt 0 360 arc closepath } bind def
/C9 { BL [] 0 setdash 2 copy moveto
      2 copy  vpt 270 450 arc closepath fill
              vpt 0 360 arc closepath } bind def
/C10 { BL [] 0 setdash 2 copy 2 copy moveto vpt 270 360 arc closepath fill
       2 copy moveto
       2 copy vpt 90 180 arc closepath fill
               vpt 0 360 arc closepath } bind def
/C11 { BL [] 0 setdash 2 copy moveto
       2 copy  vpt 0 180 arc closepath fill
       2 copy moveto
       2 copy  vpt 270 360 arc closepath fill
               vpt 0 360 arc closepath } bind def
/C12 { BL [] 0 setdash 2 copy moveto
       2 copy  vpt 180 360 arc closepath fill
               vpt 0 360 arc closepath } bind def
/C13 { BL [] 0 setdash  2 copy moveto
       2 copy  vpt 0 90 arc closepath fill
       2 copy moveto
       2 copy  vpt 180 360 arc closepath fill
               vpt 0 360 arc closepath } bind def
/C14 { BL [] 0 setdash 2 copy moveto
       2 copy  vpt 90 360 arc closepath fill
               vpt 0 360 arc } bind def
/C15 { BL [] 0 setdash 2 copy vpt 0 360 arc closepath fill
               vpt 0 360 arc closepath } bind def
/Rec   { newpath 4 2 roll moveto 1 index 0 rlineto 0 exch rlineto
       neg 0 rlineto closepath } bind def
/Square { dup Rec } bind def
/Bsquare { vpt sub exch vpt sub exch vpt2 Square } bind def
/S0 { BL [] 0 setdash 2 copy moveto 0 vpt rlineto BL Bsquare } bind def
/S1 { BL [] 0 setdash 2 copy vpt Square fill Bsquare } bind def
/S2 { BL [] 0 setdash 2 copy exch vpt sub exch vpt Square fill Bsquare } bind def
/S3 { BL [] 0 setdash 2 copy exch vpt sub exch vpt2 vpt Rec fill Bsquare } bind def
/S4 { BL [] 0 setdash 2 copy exch vpt sub exch vpt sub vpt Square fill Bsquare } bind def
/S5 { BL [] 0 setdash 2 copy 2 copy vpt Square fill
       exch vpt sub exch vpt sub vpt Square fill Bsquare } bind def
/S6 { BL [] 0 setdash 2 copy exch vpt sub exch vpt sub vpt vpt2 Rec fill Bsquare } bind def
/S7 { BL [] 0 setdash 2 copy exch vpt sub exch vpt sub vpt vpt2 Rec fill
       2 copy vpt Square fill
       Bsquare } bind def
/S8 { BL [] 0 setdash 2 copy vpt sub vpt Square fill Bsquare } bind def
/S9 { BL [] 0 setdash 2 copy vpt sub vpt vpt2 Rec fill Bsquare } bind def
/S10 { BL [] 0 setdash 2 copy vpt sub vpt Square fill 2 copy exch vpt sub exch vpt Square fill
       Bsquare } bind def
/S11 { BL [] 0 setdash 2 copy vpt sub vpt Square fill 2 copy exch vpt sub exch vpt2 vpt Rec fill
       Bsquare } bind def
/S12 { BL [] 0 setdash 2 copy exch vpt sub exch vpt sub vpt2 vpt Rec fill Bsquare } bind def
/S13 { BL [] 0 setdash 2 copy exch vpt sub exch vpt sub vpt2 vpt Rec fill
       2 copy vpt Square fill Bsquare } bind def
/S14 { BL [] 0 setdash 2 copy exch vpt sub exch vpt sub vpt2 vpt Rec fill
       2 copy exch vpt sub exch vpt Square fill Bsquare } bind def
/S15 { BL [] 0 setdash 2 copy Bsquare fill Bsquare } bind def
/D0 { gsave translate 45 rotate 0 0 S0 stroke grestore } bind def
/D1 { gsave translate 45 rotate 0 0 S1 stroke grestore } bind def
/D2 { gsave translate 45 rotate 0 0 S2 stroke grestore } bind def
/D3 { gsave translate 45 rotate 0 0 S3 stroke grestore } bind def
/D4 { gsave translate 45 rotate 0 0 S4 stroke grestore } bind def
/D5 { gsave translate 45 rotate 0 0 S5 stroke grestore } bind def
/D6 { gsave translate 45 rotate 0 0 S6 stroke grestore } bind def
/D7 { gsave translate 45 rotate 0 0 S7 stroke grestore } bind def
/D8 { gsave translate 45 rotate 0 0 S8 stroke grestore } bind def
/D9 { gsave translate 45 rotate 0 0 S9 stroke grestore } bind def
/D10 { gsave translate 45 rotate 0 0 S10 stroke grestore } bind def
/D11 { gsave translate 45 rotate 0 0 S11 stroke grestore } bind def
/D12 { gsave translate 45 rotate 0 0 S12 stroke grestore } bind def
/D13 { gsave translate 45 rotate 0 0 S13 stroke grestore } bind def
/D14 { gsave translate 45 rotate 0 0 S14 stroke grestore } bind def
/D15 { gsave translate 45 rotate 0 0 S15 stroke grestore } bind def
/DiaE { stroke [] 0 setdash vpt add M
  hpt neg vpt neg V hpt vpt neg V
  hpt vpt V hpt neg vpt V closepath stroke } def
/BoxE { stroke [] 0 setdash exch hpt sub exch vpt add M
  0 vpt2 neg V hpt2 0 V 0 vpt2 V
  hpt2 neg 0 V closepath stroke } def
/TriUE { stroke [] 0 setdash vpt 1.12 mul add M
  hpt neg vpt -1.62 mul V
  hpt 2 mul 0 V
  hpt neg vpt 1.62 mul V closepath stroke } def
/TriDE { stroke [] 0 setdash vpt 1.12 mul sub M
  hpt neg vpt 1.62 mul V
  hpt 2 mul 0 V
  hpt neg vpt -1.62 mul V closepath stroke } def
/PentE { stroke [] 0 setdash gsave
  translate 0 hpt M 4 {72 rotate 0 hpt L} repeat
  closepath stroke grestore } def
/CircE { stroke [] 0 setdash 
  hpt 0 360 arc stroke } def
/Opaque { gsave closepath 1 setgray fill grestore 0 setgray closepath } def
/DiaW { stroke [] 0 setdash vpt add M
  hpt neg vpt neg V hpt vpt neg V
  hpt vpt V hpt neg vpt V Opaque stroke } def
/BoxW { stroke [] 0 setdash exch hpt sub exch vpt add M
  0 vpt2 neg V hpt2 0 V 0 vpt2 V
  hpt2 neg 0 V Opaque stroke } def
/TriUW { stroke [] 0 setdash vpt 1.12 mul add M
  hpt neg vpt -1.62 mul V
  hpt 2 mul 0 V
  hpt neg vpt 1.62 mul V Opaque stroke } def
/TriDW { stroke [] 0 setdash vpt 1.12 mul sub M
  hpt neg vpt 1.62 mul V
  hpt 2 mul 0 V
  hpt neg vpt -1.62 mul V Opaque stroke } def
/PentW { stroke [] 0 setdash gsave
  translate 0 hpt M 4 {72 rotate 0 hpt L} repeat
  Opaque stroke grestore } def
/CircW { stroke [] 0 setdash 
  hpt 0 360 arc Opaque stroke } def
/BoxFill { gsave Rec 1 setgray fill grestore } def
end
%%EndProlog
}}%
\begin{picture}(3600,2160)(0,0)%
{\GNUPLOTspecial{"
gnudict begin
gsave
0 0 translate
0.100 0.100 scale
0 setgray
newpath
1.000 UL
LTb
350 300 M
63 0 V
3037 0 R
-63 0 V
350 652 M
63 0 V
3037 0 R
-63 0 V
350 1004 M
63 0 V
3037 0 R
-63 0 V
350 1356 M
63 0 V
3037 0 R
-63 0 V
350 1708 M
63 0 V
3037 0 R
-63 0 V
350 2060 M
63 0 V
3037 0 R
-63 0 V
532 300 M
0 63 V
0 1697 R
0 -63 V
897 300 M
0 63 V
0 1697 R
0 -63 V
1262 300 M
0 63 V
0 1697 R
0 -63 V
1626 300 M
0 63 V
0 1697 R
0 -63 V
1991 300 M
0 63 V
0 1697 R
0 -63 V
2356 300 M
0 63 V
0 1697 R
0 -63 V
2721 300 M
0 63 V
0 1697 R
0 -63 V
3085 300 M
0 63 V
0 1697 R
0 -63 V
3450 300 M
0 63 V
0 1697 R
0 -63 V
1.000 UL
LTb
350 300 M
3100 0 V
0 1760 V
-3100 0 V
350 300 L
1.000 UL
LT0
3087 1947 M
263 0 V
460 300 M
15 33 V
32 61 V
31 56 V
31 52 V
32 47 V
31 44 V
31 40 V
31 38 V
32 35 V
31 33 V
31 31 V
32 29 V
31 27 V
31 26 V
32 24 V
31 23 V
31 21 V
32 21 V
31 19 V
31 19 V
32 17 V
31 17 V
31 16 V
31 15 V
32 15 V
31 14 V
31 13 V
32 13 V
31 12 V
31 12 V
32 12 V
31 10 V
31 11 V
32 10 V
31 10 V
31 9 V
32 9 V
31 9 V
31 9 V
31 8 V
32 8 V
31 7 V
31 8 V
32 7 V
31 7 V
31 7 V
32 7 V
31 6 V
31 6 V
32 6 V
31 6 V
31 6 V
32 5 V
31 6 V
31 5 V
31 5 V
32 5 V
31 5 V
31 5 V
32 4 V
31 5 V
31 4 V
32 4 V
31 5 V
31 4 V
32 4 V
31 4 V
31 3 V
32 4 V
31 4 V
31 3 V
31 4 V
32 3 V
31 4 V
31 3 V
32 3 V
31 3 V
31 3 V
32 3 V
31 3 V
31 3 V
32 3 V
31 3 V
31 2 V
32 3 V
31 3 V
31 2 V
31 3 V
32 2 V
31 3 V
31 2 V
32 3 V
31 2 V
31 2 V
32 2 V
31 3 V
1.000 UL
LT1
3087 1847 M
263 0 V
350 331 M
31 59 V
32 54 V
31 50 V
31 46 V
32 43 V
31 41 V
31 37 V
32 35 V
31 33 V
31 31 V
31 29 V
32 27 V
31 26 V
31 24 V
32 24 V
31 21 V
31 21 V
32 20 V
31 18 V
31 18 V
32 17 V
31 16 V
31 16 V
32 15 V
31 14 V
31 13 V
31 14 V
32 12 V
31 12 V
31 12 V
32 11 V
31 10 V
31 11 V
32 10 V
31 9 V
31 9 V
32 9 V
31 9 V
31 8 V
32 8 V
31 8 V
31 8 V
31 7 V
32 7 V
31 7 V
31 7 V
32 6 V
31 6 V
31 7 V
32 5 V
31 6 V
31 6 V
32 5 V
31 6 V
31 5 V
32 5 V
31 5 V
31 5 V
31 4 V
32 5 V
31 4 V
31 5 V
32 4 V
31 4 V
31 4 V
32 4 V
31 4 V
31 4 V
32 3 V
31 4 V
31 4 V
32 3 V
31 3 V
31 4 V
31 3 V
32 3 V
31 3 V
31 3 V
32 3 V
31 3 V
31 3 V
32 3 V
31 3 V
31 2 V
32 3 V
31 3 V
31 2 V
32 3 V
31 2 V
31 3 V
31 2 V
32 2 V
31 3 V
31 2 V
32 2 V
31 2 V
31 2 V
32 3 V
31 2 V
1.000 UL
LT2
3087 1747 M
263 0 V
350 957 M
31 -7 V
32 -2 V
31 2 V
31 5 V
32 7 V
31 8 V
31 10 V
32 11 V
31 11 V
31 11 V
31 12 V
32 11 V
31 12 V
31 11 V
32 12 V
31 11 V
31 11 V
32 11 V
31 10 V
31 10 V
32 10 V
31 10 V
31 10 V
32 9 V
31 9 V
31 9 V
31 8 V
32 8 V
31 8 V
31 8 V
32 8 V
31 7 V
31 8 V
32 7 V
31 6 V
31 7 V
32 7 V
31 6 V
31 6 V
32 6 V
31 6 V
31 6 V
31 5 V
32 6 V
31 5 V
31 5 V
32 5 V
31 5 V
31 5 V
32 4 V
31 5 V
31 4 V
32 5 V
31 4 V
31 4 V
32 4 V
31 4 V
31 4 V
31 4 V
32 4 V
31 3 V
31 4 V
32 3 V
31 4 V
31 3 V
32 3 V
31 4 V
31 3 V
32 3 V
31 3 V
31 3 V
32 3 V
31 3 V
31 3 V
31 2 V
32 3 V
31 3 V
31 2 V
32 3 V
31 2 V
31 3 V
32 2 V
31 3 V
31 2 V
32 2 V
31 2 V
31 3 V
32 2 V
31 2 V
31 2 V
31 2 V
32 2 V
31 2 V
31 2 V
32 2 V
31 2 V
31 2 V
32 2 V
31 2 V
1.000 UL
LT3
3087 1647 M
263 0 V
350 1896 M
31 -107 V
32 -86 V
31 -70 V
31 -57 V
32 -47 V
31 -38 V
31 -32 V
32 -27 V
31 -21 V
31 -18 V
31 -15 V
32 -12 V
31 -10 V
31 -7 V
32 -7 V
31 -4 V
31 -4 V
32 -3 V
31 -2 V
31 -1 V
32 0 V
31 0 V
31 0 V
32 1 V
31 2 V
31 1 V
31 2 V
32 2 V
31 2 V
31 2 V
32 3 V
31 2 V
31 3 V
32 3 V
31 3 V
31 2 V
32 3 V
31 3 V
31 3 V
32 3 V
31 3 V
31 3 V
31 3 V
32 2 V
31 3 V
31 3 V
32 3 V
31 3 V
31 3 V
32 2 V
31 3 V
31 3 V
32 3 V
31 2 V
31 3 V
32 2 V
31 3 V
31 2 V
31 3 V
32 2 V
31 3 V
31 2 V
32 3 V
31 2 V
31 2 V
32 3 V
31 2 V
31 2 V
32 2 V
31 2 V
31 3 V
32 2 V
31 2 V
31 2 V
31 2 V
32 2 V
31 2 V
31 2 V
32 2 V
31 1 V
31 2 V
32 2 V
31 2 V
31 2 V
32 2 V
31 1 V
31 2 V
32 2 V
31 1 V
31 2 V
31 2 V
32 1 V
31 2 V
31 1 V
32 2 V
31 1 V
31 2 V
32 1 V
31 2 V
stroke
grestore
end
showpage
}}%
\put(3037,1647){\makebox(0,0)[r]{$l=3$}}%
\put(3037,1747){\makebox(0,0)[r]{$l=2$}}%
\put(3037,1847){\makebox(0,0)[r]{$l=1$}}%
\put(3037,1947){\makebox(0,0)[r]{$l=0$}}%
\put(1900,50){\makebox(0,0){$r$ [nm]}}%
\put(100,1180){%
\special{ps: gsave currentpoint currentpoint translate
270 rotate neg exch neg exch translate}%
\makebox(0,0)[b]{\shortstack{$V(r)$ [eV]}}%
\special{ps: currentpoint grestore moveto}%
}%
\put(3450,200){\makebox(0,0){2}}%
\put(3085,200){\makebox(0,0){1.8}}%
\put(2721,200){\makebox(0,0){1.6}}%
\put(2356,200){\makebox(0,0){1.4}}%
\put(1991,200){\makebox(0,0){1.2}}%
\put(1626,200){\makebox(0,0){1}}%
\put(1262,200){\makebox(0,0){0.8}}%
\put(897,200){\makebox(0,0){0.6}}%
\put(532,200){\makebox(0,0){0.4}}%
\put(300,2060){\makebox(0,0)[r]{1}}%
\put(300,1708){\makebox(0,0)[r]{0}}%
\put(300,1356){\makebox(0,0)[r]{-1}}%
\put(300,1004){\makebox(0,0)[r]{-2}}%
\put(300,652){\makebox(0,0)[r]{-3}}%
\put(300,300){\makebox(0,0)[r]{-4}}%
\end{picture}%
\endgroup
\endinput

\caption{Det effektive potensialet for ulike $l$-verdier men for store verdier av $r$ Merk at for store $r$ tar Coulombleddet over slik vi at ender opp
med tiltrekning til slutt.\label{42}}
\end{center}
\end{figure}

For sm\aa\ verdier av $r$ og $l\ne 0$, ser vi at potensialet er frast\o tende.
Det er sentrifugaldelen av potensialet som 'skyver' elektronet bort
fra protonet. For $l=0$ har vi kun Coulomb potensialet, som da tiltrekker
elektronet. Siden vi ikke har noe frast\o ting her, betyr det at
elektroner med $l=0$ har en st\o rre sannsynlighet for \aa\ 
v\ae re n\ae r kjerner enn elektroner i tilstander med $l>0$. 
At dette er tilfelle skal vi se p\aa\ n\aa r vi har den radielle
egenfunksjonen, se f.eks.~oppgave 7.4.

N\aa r vi \o ker $r$, tar Coulombleddet over sentrifugalleddet, og vi f\aa r 
et tiltrekkende potensial ogs\aa\ for $l > 0$. Husk hele tiden at
den totale bindingsenergien (den energien som binder elektronet til kjernen)
er summen av potensiell og kinetisk energi. 

En klassisk analog til dette potensialet finner vi ved \aa\ se p\aa\ en
satelitt som g\aa r i bane rundt jorda. Den vil ha et banespinnledd som g\aa r
som $L^2$ samt en gravitasjonsenergi som trekker satelitten mot jorda.
Coulombpotensialet spiller samme rolle som gravitasjonsenergien.   

N\aa\ definerer vi en ny variabel 
\be
  \rho=r\frac{\sqrt{8m|E|}}{\hbar},
\ee
samt parameteren
\be
\lambda=\frac{ke^2}{\hbar}\sqrt{\frac{m}{2|E|}},
\ee
slik at den radielle Schr\"odingers likning  kan skrives 
\be
\frac{d^2 u(\rho)}{d\rho^2}-
\left(\frac{\lambda}{\rho}-\frac{1}{4}-\frac{l(l+1)}{\rho^2}\right)u(\rho)=0.
\ee
I grensa $\rho$ stor reduseres siste likning til
\be
   \frac{d^2 u(\rho)}{d\rho^2})\approx \frac{1}{4}u(\rho)
\ee
hvis l\o sning er p\aa\ forma
\be
   u(\rho)=e^{\pm \rho/2},
\ee
hvor bare minustegnet kan aksepteres, for ellers vil $u(\rho)$ divergere,
noe som igjen leder til uendelig stor sannsynlighetstetthet.

Vi f\o lger n\aa\ samme oppskrift som for den harmoniske
oscillatoren og antar at den radielle funksjonen $u(\rho)$ er gitt
ved 
\be
   u(\rho)=e^{-\rho/2}v(\rho),
\ee
hvor $v$ er et polynom 
\be
   v(\rho)=\sum_{k=0}^{\infty}a_k\rho^{s+k},
\ee
med $a_0 \ne 0$. Innsetting gir oss
\be
\left(\frac{d^2 }{d\rho^2}-\frac{d }{d\rho}+\frac{\lambda}{\rho}-\frac{l(l+1)}{\rho^2}\right)v(\rho)=0.
\ee
Denne differensiallikningen har som l\o sning de s\aa kalte
Laguerre polynomene (se igjen Rottmann under spesielle funksjoner), 
og gir ved innsetting av polynomuttrykket at 
for $k=0$ (n\aa r vi sammenlikner ledd med samme potens i $\rho$)  
\be  
   a_0(s(s-1)-l(l+1))\rho^{s-2}=0,
\ee
som gir at $s=-l$ eller $s=l+1$. Dersom vi har $s=-l$, med 
$l$ et endelig positivt heltall, vil b\o lgefunksjonen~divergere ved $r=0$.
Vi m\aa\ ha $s=l+1$.
Det gir
\be
   v(\rho)=\sum_{k=0}^{\infty}a_k\rho^{k+l+1}.
\ee    
Det er mulig \aa\ vise at betingelsen for at rekka for Laguerre polynomet
ikke skal divergere er gitt ved
\be
   \lambda=n\geq l+1,
\ee
hvor $n$ er et heltall.
Benytter vi deretter definisjonen p\aa\ $\lambda$ og $\rho$ har vi
\be
   |E|=\frac{ke^2}{2a_0}\frac{1}{n^2},
\ee
som er samme uttrykk vi fant for Bohrs atommodell for energien.
I tillegg har vi
\be
    \rho=r\sqrt{\frac{8ke^2}{2ma_0\hbar^2}}=\frac{2r}{na_0}
\ee
med Bohrradien
\[
   a_0=\frac{\hbar^2}{mke^2}.
\]

Den radielle b\o lgefunksjonen~kan skrives som
\be
    u_n(r)=e^{-r/na_0}v(\frac{r}{a_0}),
\ee
som igjen betyr at 
\be
   R_{nl}=re^{-r/na_0}v(\frac{r}{a_0}).
\ee
eller dersom vi forenkler kan den skrives som 
\be
   R_{nl}=r^le^{-r/na_0}\times (\mathrm{polynom}(r)).
\ee

Den totale egenfunskjonen, med radiell og vinkelavhengighet er gitt ved
\be
\psi_{nlm_l}(r,\theta,\phi)=\psi_{nlm_l}=R_{nl}(r)Y_{lm_l}(\theta,\phi)=
         R_{nl}Y_{lm_l}
\ee
Her lister vi noen eksempler p\aa\ b\o lgefunksjonen~for hydrogenatomet,
med b\aa de vinkel og radiell del.
For $n=1$ og $l=m_l=0$ (grunntilstanden) har vi 
\be
   \psi_{100}=\frac{1}{a_0^{3/2}\sqrt{\pi}}e^{-r/a_0},
\ee
mens den f\o rste eksiterte tilstanden for $l=0$ er
\be
   \psi_{200}=\frac{1}{4a_0^{3/2}\sqrt{2\pi}}
   \left(2-\frac{r}{a_0}\right)e^{-r/2a_0}.
\ee
For tilstander med $l=1$ har vi for $n=2$ og $m_l=0$
\be
   \psi_{210}=\frac{1}{4a_0^{3/2}\sqrt{2\pi}}
   \left(\frac{r}{a_0}\right)e^{-r/2a_0}cos(\theta).
\ee
For $m_l=\pm 1$ finner vi 
\be
   \psi_{21\pm 1}=\frac{1}{8a_0^{3/2}\sqrt{\pi}}
   \left(\frac{r}{a_0}\right)e^{-r/2a_0}sin(\theta)e^{\pm i\phi}.
\ee

Historisk kalles tilstander med $l=0$ for $s$-tilstander ($s$ for sharp),
$l=1$ for $p$-tilstander ($p$ for pure),  
$l=2$ for $d$-tilstander ($d$ for diffuse),
$l=3$ for $f$-tilstander, $l=4$ for $g$-tilstander, $l=5$ for $h$-tilstander
osv.~i alfabetisk rekkef\o lge med stigende $l$-verdi.
Det vil si at en tilstand med $n=1$ og $l=0$ har den spektroskopiske
notasjonen $1s$. Tilsvarende for $n=2$ har vi $2s$. 
P\aa\ samme vis kan vi lage oss en spektroskopisk notasjon for
$l=1$ tilstandene. For  
$n=2$ og $l=1$ har vi den spektroskopiske
notasjonen $2p$. Tilsvarende for $n=3$ har vi $3p$. Slik kan vi fortsette
for andre verdier av $l$ og $n$.

Det vi skal legge merke til er at energien avhenger kun av kvantetallet
$n$, og er gitt ved 
\be
    E_n=\frac{E_0}{n^2},
\ee
hvor $E_0=-13.6$ eV. 
Det betyr at $2s$ og $2p$ tilstandene som begge har $n=2$ men ulikt
banespinn, har samme energi. Det at to tilstander med ulike egenfunksjoner
gir opphav til samme energi kalles degenerasjon. Vi har allerede sett
det i tilknytting det tre-dimensjonale bokspotensialet.
At to eller flere tilstander med ulike kvantetall gir samme energi skyldes
symmetrien i vekselvirkningen, her den sf\ae riske symmetrien til 
Coulombpotensialet. I neste avsnitt skal vi bryte denne symmetrien ved
\aa\ sette p\aa\ et magnetfelt.
\begin{figure}
\begin{center}
% GNUPLOT: LaTeX picture with Postscript
\begingroup%
  \makeatletter%
  \newcommand{\GNUPLOTspecial}{%
    \@sanitize\catcode`\%=14\relax\special}%
  \setlength{\unitlength}{0.1bp}%
{\GNUPLOTspecial{!
%!PS-Adobe-2.0 EPSF-2.0
%%Title: wave1.tex
%%Creator: gnuplot 3.7 patchlevel 0.2
%%CreationDate: Fri Mar 17 10:43:00 2000
%%DocumentFonts: 
%%BoundingBox: 0 0 360 216
%%Orientation: Landscape
%%EndComments
/gnudict 256 dict def
gnudict begin
/Color false def
/Solid false def
/gnulinewidth 5.000 def
/userlinewidth gnulinewidth def
/vshift -33 def
/dl {10 mul} def
/hpt_ 31.5 def
/vpt_ 31.5 def
/hpt hpt_ def
/vpt vpt_ def
/M {moveto} bind def
/L {lineto} bind def
/R {rmoveto} bind def
/V {rlineto} bind def
/vpt2 vpt 2 mul def
/hpt2 hpt 2 mul def
/Lshow { currentpoint stroke M
  0 vshift R show } def
/Rshow { currentpoint stroke M
  dup stringwidth pop neg vshift R show } def
/Cshow { currentpoint stroke M
  dup stringwidth pop -2 div vshift R show } def
/UP { dup vpt_ mul /vpt exch def hpt_ mul /hpt exch def
  /hpt2 hpt 2 mul def /vpt2 vpt 2 mul def } def
/DL { Color {setrgbcolor Solid {pop []} if 0 setdash }
 {pop pop pop Solid {pop []} if 0 setdash} ifelse } def
/BL { stroke userlinewidth 2 mul setlinewidth } def
/AL { stroke userlinewidth 2 div setlinewidth } def
/UL { dup gnulinewidth mul /userlinewidth exch def
      10 mul /udl exch def } def
/PL { stroke userlinewidth setlinewidth } def
/LTb { BL [] 0 0 0 DL } def
/LTa { AL [1 udl mul 2 udl mul] 0 setdash 0 0 0 setrgbcolor } def
/LT0 { PL [] 1 0 0 DL } def
/LT1 { PL [4 dl 2 dl] 0 1 0 DL } def
/LT2 { PL [2 dl 3 dl] 0 0 1 DL } def
/LT3 { PL [1 dl 1.5 dl] 1 0 1 DL } def
/LT4 { PL [5 dl 2 dl 1 dl 2 dl] 0 1 1 DL } def
/LT5 { PL [4 dl 3 dl 1 dl 3 dl] 1 1 0 DL } def
/LT6 { PL [2 dl 2 dl 2 dl 4 dl] 0 0 0 DL } def
/LT7 { PL [2 dl 2 dl 2 dl 2 dl 2 dl 4 dl] 1 0.3 0 DL } def
/LT8 { PL [2 dl 2 dl 2 dl 2 dl 2 dl 2 dl 2 dl 4 dl] 0.5 0.5 0.5 DL } def
/Pnt { stroke [] 0 setdash
   gsave 1 setlinecap M 0 0 V stroke grestore } def
/Dia { stroke [] 0 setdash 2 copy vpt add M
  hpt neg vpt neg V hpt vpt neg V
  hpt vpt V hpt neg vpt V closepath stroke
  Pnt } def
/Pls { stroke [] 0 setdash vpt sub M 0 vpt2 V
  currentpoint stroke M
  hpt neg vpt neg R hpt2 0 V stroke
  } def
/Box { stroke [] 0 setdash 2 copy exch hpt sub exch vpt add M
  0 vpt2 neg V hpt2 0 V 0 vpt2 V
  hpt2 neg 0 V closepath stroke
  Pnt } def
/Crs { stroke [] 0 setdash exch hpt sub exch vpt add M
  hpt2 vpt2 neg V currentpoint stroke M
  hpt2 neg 0 R hpt2 vpt2 V stroke } def
/TriU { stroke [] 0 setdash 2 copy vpt 1.12 mul add M
  hpt neg vpt -1.62 mul V
  hpt 2 mul 0 V
  hpt neg vpt 1.62 mul V closepath stroke
  Pnt  } def
/Star { 2 copy Pls Crs } def
/BoxF { stroke [] 0 setdash exch hpt sub exch vpt add M
  0 vpt2 neg V  hpt2 0 V  0 vpt2 V
  hpt2 neg 0 V  closepath fill } def
/TriUF { stroke [] 0 setdash vpt 1.12 mul add M
  hpt neg vpt -1.62 mul V
  hpt 2 mul 0 V
  hpt neg vpt 1.62 mul V closepath fill } def
/TriD { stroke [] 0 setdash 2 copy vpt 1.12 mul sub M
  hpt neg vpt 1.62 mul V
  hpt 2 mul 0 V
  hpt neg vpt -1.62 mul V closepath stroke
  Pnt  } def
/TriDF { stroke [] 0 setdash vpt 1.12 mul sub M
  hpt neg vpt 1.62 mul V
  hpt 2 mul 0 V
  hpt neg vpt -1.62 mul V closepath fill} def
/DiaF { stroke [] 0 setdash vpt add M
  hpt neg vpt neg V hpt vpt neg V
  hpt vpt V hpt neg vpt V closepath fill } def
/Pent { stroke [] 0 setdash 2 copy gsave
  translate 0 hpt M 4 {72 rotate 0 hpt L} repeat
  closepath stroke grestore Pnt } def
/PentF { stroke [] 0 setdash gsave
  translate 0 hpt M 4 {72 rotate 0 hpt L} repeat
  closepath fill grestore } def
/Circle { stroke [] 0 setdash 2 copy
  hpt 0 360 arc stroke Pnt } def
/CircleF { stroke [] 0 setdash hpt 0 360 arc fill } def
/C0 { BL [] 0 setdash 2 copy moveto vpt 90 450  arc } bind def
/C1 { BL [] 0 setdash 2 copy        moveto
       2 copy  vpt 0 90 arc closepath fill
               vpt 0 360 arc closepath } bind def
/C2 { BL [] 0 setdash 2 copy moveto
       2 copy  vpt 90 180 arc closepath fill
               vpt 0 360 arc closepath } bind def
/C3 { BL [] 0 setdash 2 copy moveto
       2 copy  vpt 0 180 arc closepath fill
               vpt 0 360 arc closepath } bind def
/C4 { BL [] 0 setdash 2 copy moveto
       2 copy  vpt 180 270 arc closepath fill
               vpt 0 360 arc closepath } bind def
/C5 { BL [] 0 setdash 2 copy moveto
       2 copy  vpt 0 90 arc
       2 copy moveto
       2 copy  vpt 180 270 arc closepath fill
               vpt 0 360 arc } bind def
/C6 { BL [] 0 setdash 2 copy moveto
      2 copy  vpt 90 270 arc closepath fill
              vpt 0 360 arc closepath } bind def
/C7 { BL [] 0 setdash 2 copy moveto
      2 copy  vpt 0 270 arc closepath fill
              vpt 0 360 arc closepath } bind def
/C8 { BL [] 0 setdash 2 copy moveto
      2 copy vpt 270 360 arc closepath fill
              vpt 0 360 arc closepath } bind def
/C9 { BL [] 0 setdash 2 copy moveto
      2 copy  vpt 270 450 arc closepath fill
              vpt 0 360 arc closepath } bind def
/C10 { BL [] 0 setdash 2 copy 2 copy moveto vpt 270 360 arc closepath fill
       2 copy moveto
       2 copy vpt 90 180 arc closepath fill
               vpt 0 360 arc closepath } bind def
/C11 { BL [] 0 setdash 2 copy moveto
       2 copy  vpt 0 180 arc closepath fill
       2 copy moveto
       2 copy  vpt 270 360 arc closepath fill
               vpt 0 360 arc closepath } bind def
/C12 { BL [] 0 setdash 2 copy moveto
       2 copy  vpt 180 360 arc closepath fill
               vpt 0 360 arc closepath } bind def
/C13 { BL [] 0 setdash  2 copy moveto
       2 copy  vpt 0 90 arc closepath fill
       2 copy moveto
       2 copy  vpt 180 360 arc closepath fill
               vpt 0 360 arc closepath } bind def
/C14 { BL [] 0 setdash 2 copy moveto
       2 copy  vpt 90 360 arc closepath fill
               vpt 0 360 arc } bind def
/C15 { BL [] 0 setdash 2 copy vpt 0 360 arc closepath fill
               vpt 0 360 arc closepath } bind def
/Rec   { newpath 4 2 roll moveto 1 index 0 rlineto 0 exch rlineto
       neg 0 rlineto closepath } bind def
/Square { dup Rec } bind def
/Bsquare { vpt sub exch vpt sub exch vpt2 Square } bind def
/S0 { BL [] 0 setdash 2 copy moveto 0 vpt rlineto BL Bsquare } bind def
/S1 { BL [] 0 setdash 2 copy vpt Square fill Bsquare } bind def
/S2 { BL [] 0 setdash 2 copy exch vpt sub exch vpt Square fill Bsquare } bind def
/S3 { BL [] 0 setdash 2 copy exch vpt sub exch vpt2 vpt Rec fill Bsquare } bind def
/S4 { BL [] 0 setdash 2 copy exch vpt sub exch vpt sub vpt Square fill Bsquare } bind def
/S5 { BL [] 0 setdash 2 copy 2 copy vpt Square fill
       exch vpt sub exch vpt sub vpt Square fill Bsquare } bind def
/S6 { BL [] 0 setdash 2 copy exch vpt sub exch vpt sub vpt vpt2 Rec fill Bsquare } bind def
/S7 { BL [] 0 setdash 2 copy exch vpt sub exch vpt sub vpt vpt2 Rec fill
       2 copy vpt Square fill
       Bsquare } bind def
/S8 { BL [] 0 setdash 2 copy vpt sub vpt Square fill Bsquare } bind def
/S9 { BL [] 0 setdash 2 copy vpt sub vpt vpt2 Rec fill Bsquare } bind def
/S10 { BL [] 0 setdash 2 copy vpt sub vpt Square fill 2 copy exch vpt sub exch vpt Square fill
       Bsquare } bind def
/S11 { BL [] 0 setdash 2 copy vpt sub vpt Square fill 2 copy exch vpt sub exch vpt2 vpt Rec fill
       Bsquare } bind def
/S12 { BL [] 0 setdash 2 copy exch vpt sub exch vpt sub vpt2 vpt Rec fill Bsquare } bind def
/S13 { BL [] 0 setdash 2 copy exch vpt sub exch vpt sub vpt2 vpt Rec fill
       2 copy vpt Square fill Bsquare } bind def
/S14 { BL [] 0 setdash 2 copy exch vpt sub exch vpt sub vpt2 vpt Rec fill
       2 copy exch vpt sub exch vpt Square fill Bsquare } bind def
/S15 { BL [] 0 setdash 2 copy Bsquare fill Bsquare } bind def
/D0 { gsave translate 45 rotate 0 0 S0 stroke grestore } bind def
/D1 { gsave translate 45 rotate 0 0 S1 stroke grestore } bind def
/D2 { gsave translate 45 rotate 0 0 S2 stroke grestore } bind def
/D3 { gsave translate 45 rotate 0 0 S3 stroke grestore } bind def
/D4 { gsave translate 45 rotate 0 0 S4 stroke grestore } bind def
/D5 { gsave translate 45 rotate 0 0 S5 stroke grestore } bind def
/D6 { gsave translate 45 rotate 0 0 S6 stroke grestore } bind def
/D7 { gsave translate 45 rotate 0 0 S7 stroke grestore } bind def
/D8 { gsave translate 45 rotate 0 0 S8 stroke grestore } bind def
/D9 { gsave translate 45 rotate 0 0 S9 stroke grestore } bind def
/D10 { gsave translate 45 rotate 0 0 S10 stroke grestore } bind def
/D11 { gsave translate 45 rotate 0 0 S11 stroke grestore } bind def
/D12 { gsave translate 45 rotate 0 0 S12 stroke grestore } bind def
/D13 { gsave translate 45 rotate 0 0 S13 stroke grestore } bind def
/D14 { gsave translate 45 rotate 0 0 S14 stroke grestore } bind def
/D15 { gsave translate 45 rotate 0 0 S15 stroke grestore } bind def
/DiaE { stroke [] 0 setdash vpt add M
  hpt neg vpt neg V hpt vpt neg V
  hpt vpt V hpt neg vpt V closepath stroke } def
/BoxE { stroke [] 0 setdash exch hpt sub exch vpt add M
  0 vpt2 neg V hpt2 0 V 0 vpt2 V
  hpt2 neg 0 V closepath stroke } def
/TriUE { stroke [] 0 setdash vpt 1.12 mul add M
  hpt neg vpt -1.62 mul V
  hpt 2 mul 0 V
  hpt neg vpt 1.62 mul V closepath stroke } def
/TriDE { stroke [] 0 setdash vpt 1.12 mul sub M
  hpt neg vpt 1.62 mul V
  hpt 2 mul 0 V
  hpt neg vpt -1.62 mul V closepath stroke } def
/PentE { stroke [] 0 setdash gsave
  translate 0 hpt M 4 {72 rotate 0 hpt L} repeat
  closepath stroke grestore } def
/CircE { stroke [] 0 setdash 
  hpt 0 360 arc stroke } def
/Opaque { gsave closepath 1 setgray fill grestore 0 setgray closepath } def
/DiaW { stroke [] 0 setdash vpt add M
  hpt neg vpt neg V hpt vpt neg V
  hpt vpt V hpt neg vpt V Opaque stroke } def
/BoxW { stroke [] 0 setdash exch hpt sub exch vpt add M
  0 vpt2 neg V hpt2 0 V 0 vpt2 V
  hpt2 neg 0 V Opaque stroke } def
/TriUW { stroke [] 0 setdash vpt 1.12 mul add M
  hpt neg vpt -1.62 mul V
  hpt 2 mul 0 V
  hpt neg vpt 1.62 mul V Opaque stroke } def
/TriDW { stroke [] 0 setdash vpt 1.12 mul sub M
  hpt neg vpt 1.62 mul V
  hpt 2 mul 0 V
  hpt neg vpt -1.62 mul V Opaque stroke } def
/PentW { stroke [] 0 setdash gsave
  translate 0 hpt M 4 {72 rotate 0 hpt L} repeat
  Opaque stroke grestore } def
/CircW { stroke [] 0 setdash 
  hpt 0 360 arc Opaque stroke } def
/BoxFill { gsave Rec 1 setgray fill grestore } def
end
%%EndProlog
}}%
\begin{picture}(3600,2160)(0,0)%
{\GNUPLOTspecial{"
gnudict begin
gsave
0 0 translate
0.100 0.100 scale
0 setgray
newpath
1.000 UL
LTb
400 300 M
63 0 V
2987 0 R
-63 0 V
400 593 M
63 0 V
2987 0 R
-63 0 V
400 887 M
63 0 V
2987 0 R
-63 0 V
400 1180 M
63 0 V
2987 0 R
-63 0 V
400 1473 M
63 0 V
2987 0 R
-63 0 V
400 1767 M
63 0 V
2987 0 R
-63 0 V
400 2060 M
63 0 V
2987 0 R
-63 0 V
400 300 M
0 63 V
0 1697 R
0 -63 V
1163 300 M
0 63 V
0 1697 R
0 -63 V
1925 300 M
0 63 V
0 1697 R
0 -63 V
2688 300 M
0 63 V
0 1697 R
0 -63 V
3450 300 M
0 63 V
0 1697 R
0 -63 V
1.000 UL
LTb
400 300 M
3050 0 V
0 1760 V
-3050 0 V
400 300 L
1.000 UL
LT0
3087 1947 M
263 0 V
400 300 M
31 320 V
31 534 V
30 428 V
31 240 V
31 66 V
31 -62 V
31 -139 V
30 -177 V
31 -188 V
31 -180 V
739 980 L
770 841 L
801 724 L
30 -96 V
31 -77 V
31 -60 V
31 -47 V
31 -36 V
30 -28 V
31 -21 V
31 -15 V
31 -12 V
31 -9 V
30 -6 V
31 -5 V
31 -3 V
31 -3 V
31 -1 V
30 -2 V
31 -1 V
31 0 V
31 -1 V
31 0 V
30 0 V
31 -1 V
31 0 V
31 0 V
31 0 V
31 0 V
30 0 V
31 0 V
31 0 V
31 0 V
31 0 V
30 0 V
31 0 V
31 0 V
31 0 V
31 0 V
30 0 V
31 0 V
31 0 V
31 0 V
31 0 V
30 0 V
31 0 V
31 0 V
31 0 V
31 0 V
30 0 V
31 0 V
31 0 V
31 0 V
31 0 V
31 0 V
30 0 V
31 0 V
31 0 V
31 0 V
31 0 V
30 0 V
31 0 V
31 0 V
31 0 V
31 0 V
30 0 V
31 0 V
31 0 V
31 0 V
31 0 V
30 0 V
31 0 V
31 0 V
31 0 V
31 0 V
30 0 V
31 0 V
31 0 V
31 0 V
31 0 V
31 0 V
30 0 V
31 0 V
31 0 V
31 0 V
31 0 V
30 0 V
31 0 V
31 0 V
1.000 UL
LT1
3087 1847 M
263 0 V
400 300 M
31 40 V
31 62 V
30 41 V
31 9 V
31 -18 V
31 -34 V
31 -39 V
30 -33 V
31 -21 V
31 -7 V
31 10 V
31 24 V
31 38 V
30 47 V
31 54 V
31 57 V
31 57 V
31 55 V
30 51 V
31 45 V
31 39 V
31 31 V
31 24 V
30 16 V
31 10 V
31 2 V
31 -3 V
31 -9 V
30 -13 V
31 -17 V
31 -20 V
31 -22 V
31 -24 V
30 -26 V
31 -26 V
31 -26 V
31 -26 V
31 -25 V
31 -25 V
30 -24 V
31 -23 V
31 -22 V
31 -20 V
31 -19 V
30 -18 V
31 -17 V
31 -15 V
31 -14 V
31 -13 V
30 -12 V
31 -11 V
31 -10 V
31 -9 V
31 -8 V
30 -7 V
31 -6 V
31 -6 V
31 -5 V
31 -5 V
30 -4 V
31 -4 V
31 -3 V
31 -3 V
31 -2 V
31 -2 V
30 -2 V
31 -2 V
31 -2 V
31 -1 V
31 -1 V
30 -1 V
31 -1 V
31 -1 V
31 -1 V
31 0 V
30 -1 V
31 0 V
31 -1 V
31 0 V
31 0 V
30 0 V
31 -1 V
31 0 V
31 0 V
31 0 V
30 0 V
31 0 V
31 0 V
31 -1 V
31 0 V
31 0 V
30 0 V
31 0 V
31 0 V
31 0 V
31 0 V
30 0 V
31 0 V
31 0 V
1.000 UL
LT2
3087 1747 M
263 0 V
400 300 M
31 12 V
31 18 V
30 11 V
31 2 V
31 -6 V
31 -11 V
31 -12 V
30 -9 V
31 -5 V
31 1 V
31 6 V
31 10 V
31 13 V
30 15 V
31 16 V
31 15 V
31 14 V
31 11 V
30 7 V
31 5 V
31 0 V
31 -2 V
31 -6 V
30 -8 V
31 -10 V
31 -11 V
31 -12 V
31 -13 V
30 -11 V
31 -11 V
31 -10 V
31 -8 V
31 -6 V
30 -4 V
31 -1 V
31 1 V
31 3 V
31 5 V
31 8 V
30 9 V
31 11 V
31 13 V
31 14 V
31 15 V
30 15 V
31 16 V
31 17 V
31 16 V
31 17 V
30 15 V
31 16 V
31 14 V
31 14 V
31 13 V
30 12 V
31 11 V
31 9 V
31 9 V
31 7 V
30 6 V
31 4 V
31 4 V
31 2 V
31 2 V
31 0 V
30 -1 V
31 -2 V
31 -2 V
31 -4 V
31 -4 V
30 -5 V
31 -5 V
31 -6 V
31 -6 V
31 -7 V
30 -7 V
31 -8 V
31 -7 V
31 -8 V
31 -8 V
30 -8 V
31 -8 V
31 -8 V
31 -8 V
31 -8 V
30 -8 V
31 -8 V
31 -8 V
31 -7 V
31 -7 V
31 -7 V
30 -7 V
31 -7 V
31 -6 V
31 -6 V
31 -6 V
30 -6 V
31 -6 V
31 -5 V
stroke
grestore
end
showpage
}}%
\put(3037,1747){\makebox(0,0)[r]{$n=3$ $l=0$}}%
\put(3037,1847){\makebox(0,0)[r]{$n=2$ $l=0$}}%
\put(3037,1947){\makebox(0,0)[r]{$n=1$ $l=0$}}%
\put(1925,50){\makebox(0,0){$r/a_0$}}%
\put(100,1180){%
\special{ps: gsave currentpoint currentpoint translate
270 rotate neg exch neg exch translate}%
\makebox(0,0)[b]{\shortstack{$a_0P(r)$}}%
\special{ps: currentpoint grestore moveto}%
}%
\put(3450,200){\makebox(0,0){20}}%
\put(2688,200){\makebox(0,0){15}}%
\put(1925,200){\makebox(0,0){10}}%
\put(1163,200){\makebox(0,0){5}}%
\put(400,200){\makebox(0,0){0}}%
\put(350,2060){\makebox(0,0)[r]{0.6}}%
\put(350,1767){\makebox(0,0)[r]{0.5}}%
\put(350,1473){\makebox(0,0)[r]{0.4}}%
\put(350,1180){\makebox(0,0)[r]{0.3}}%
\put(350,887){\makebox(0,0)[r]{0.2}}%
\put(350,593){\makebox(0,0)[r]{0.1}}%
\put(350,300){\makebox(0,0)[r]{0}}%
\end{picture}%
\endgroup
\endinput

\caption{Plot av $a_0P(r)$ for $l=0$ tilstander opp til $n=3$. \label{43}}
\end{center}
\end{figure}
\begin{figure}
\begin{center}
% GNUPLOT: LaTeX picture with Postscript
\begingroup%
  \makeatletter%
  \newcommand{\GNUPLOTspecial}{%
    \@sanitize\catcode`\%=14\relax\special}%
  \setlength{\unitlength}{0.1bp}%
{\GNUPLOTspecial{!
%!PS-Adobe-2.0 EPSF-2.0
%%Title: wave2.tex
%%Creator: gnuplot 3.7 patchlevel 0.2
%%CreationDate: Fri Mar 17 11:47:06 2000
%%DocumentFonts: 
%%BoundingBox: 0 0 360 216
%%Orientation: Landscape
%%EndComments
/gnudict 256 dict def
gnudict begin
/Color false def
/Solid false def
/gnulinewidth 5.000 def
/userlinewidth gnulinewidth def
/vshift -33 def
/dl {10 mul} def
/hpt_ 31.5 def
/vpt_ 31.5 def
/hpt hpt_ def
/vpt vpt_ def
/M {moveto} bind def
/L {lineto} bind def
/R {rmoveto} bind def
/V {rlineto} bind def
/vpt2 vpt 2 mul def
/hpt2 hpt 2 mul def
/Lshow { currentpoint stroke M
  0 vshift R show } def
/Rshow { currentpoint stroke M
  dup stringwidth pop neg vshift R show } def
/Cshow { currentpoint stroke M
  dup stringwidth pop -2 div vshift R show } def
/UP { dup vpt_ mul /vpt exch def hpt_ mul /hpt exch def
  /hpt2 hpt 2 mul def /vpt2 vpt 2 mul def } def
/DL { Color {setrgbcolor Solid {pop []} if 0 setdash }
 {pop pop pop Solid {pop []} if 0 setdash} ifelse } def
/BL { stroke userlinewidth 2 mul setlinewidth } def
/AL { stroke userlinewidth 2 div setlinewidth } def
/UL { dup gnulinewidth mul /userlinewidth exch def
      10 mul /udl exch def } def
/PL { stroke userlinewidth setlinewidth } def
/LTb { BL [] 0 0 0 DL } def
/LTa { AL [1 udl mul 2 udl mul] 0 setdash 0 0 0 setrgbcolor } def
/LT0 { PL [] 1 0 0 DL } def
/LT1 { PL [4 dl 2 dl] 0 1 0 DL } def
/LT2 { PL [2 dl 3 dl] 0 0 1 DL } def
/LT3 { PL [1 dl 1.5 dl] 1 0 1 DL } def
/LT4 { PL [5 dl 2 dl 1 dl 2 dl] 0 1 1 DL } def
/LT5 { PL [4 dl 3 dl 1 dl 3 dl] 1 1 0 DL } def
/LT6 { PL [2 dl 2 dl 2 dl 4 dl] 0 0 0 DL } def
/LT7 { PL [2 dl 2 dl 2 dl 2 dl 2 dl 4 dl] 1 0.3 0 DL } def
/LT8 { PL [2 dl 2 dl 2 dl 2 dl 2 dl 2 dl 2 dl 4 dl] 0.5 0.5 0.5 DL } def
/Pnt { stroke [] 0 setdash
   gsave 1 setlinecap M 0 0 V stroke grestore } def
/Dia { stroke [] 0 setdash 2 copy vpt add M
  hpt neg vpt neg V hpt vpt neg V
  hpt vpt V hpt neg vpt V closepath stroke
  Pnt } def
/Pls { stroke [] 0 setdash vpt sub M 0 vpt2 V
  currentpoint stroke M
  hpt neg vpt neg R hpt2 0 V stroke
  } def
/Box { stroke [] 0 setdash 2 copy exch hpt sub exch vpt add M
  0 vpt2 neg V hpt2 0 V 0 vpt2 V
  hpt2 neg 0 V closepath stroke
  Pnt } def
/Crs { stroke [] 0 setdash exch hpt sub exch vpt add M
  hpt2 vpt2 neg V currentpoint stroke M
  hpt2 neg 0 R hpt2 vpt2 V stroke } def
/TriU { stroke [] 0 setdash 2 copy vpt 1.12 mul add M
  hpt neg vpt -1.62 mul V
  hpt 2 mul 0 V
  hpt neg vpt 1.62 mul V closepath stroke
  Pnt  } def
/Star { 2 copy Pls Crs } def
/BoxF { stroke [] 0 setdash exch hpt sub exch vpt add M
  0 vpt2 neg V  hpt2 0 V  0 vpt2 V
  hpt2 neg 0 V  closepath fill } def
/TriUF { stroke [] 0 setdash vpt 1.12 mul add M
  hpt neg vpt -1.62 mul V
  hpt 2 mul 0 V
  hpt neg vpt 1.62 mul V closepath fill } def
/TriD { stroke [] 0 setdash 2 copy vpt 1.12 mul sub M
  hpt neg vpt 1.62 mul V
  hpt 2 mul 0 V
  hpt neg vpt -1.62 mul V closepath stroke
  Pnt  } def
/TriDF { stroke [] 0 setdash vpt 1.12 mul sub M
  hpt neg vpt 1.62 mul V
  hpt 2 mul 0 V
  hpt neg vpt -1.62 mul V closepath fill} def
/DiaF { stroke [] 0 setdash vpt add M
  hpt neg vpt neg V hpt vpt neg V
  hpt vpt V hpt neg vpt V closepath fill } def
/Pent { stroke [] 0 setdash 2 copy gsave
  translate 0 hpt M 4 {72 rotate 0 hpt L} repeat
  closepath stroke grestore Pnt } def
/PentF { stroke [] 0 setdash gsave
  translate 0 hpt M 4 {72 rotate 0 hpt L} repeat
  closepath fill grestore } def
/Circle { stroke [] 0 setdash 2 copy
  hpt 0 360 arc stroke Pnt } def
/CircleF { stroke [] 0 setdash hpt 0 360 arc fill } def
/C0 { BL [] 0 setdash 2 copy moveto vpt 90 450  arc } bind def
/C1 { BL [] 0 setdash 2 copy        moveto
       2 copy  vpt 0 90 arc closepath fill
               vpt 0 360 arc closepath } bind def
/C2 { BL [] 0 setdash 2 copy moveto
       2 copy  vpt 90 180 arc closepath fill
               vpt 0 360 arc closepath } bind def
/C3 { BL [] 0 setdash 2 copy moveto
       2 copy  vpt 0 180 arc closepath fill
               vpt 0 360 arc closepath } bind def
/C4 { BL [] 0 setdash 2 copy moveto
       2 copy  vpt 180 270 arc closepath fill
               vpt 0 360 arc closepath } bind def
/C5 { BL [] 0 setdash 2 copy moveto
       2 copy  vpt 0 90 arc
       2 copy moveto
       2 copy  vpt 180 270 arc closepath fill
               vpt 0 360 arc } bind def
/C6 { BL [] 0 setdash 2 copy moveto
      2 copy  vpt 90 270 arc closepath fill
              vpt 0 360 arc closepath } bind def
/C7 { BL [] 0 setdash 2 copy moveto
      2 copy  vpt 0 270 arc closepath fill
              vpt 0 360 arc closepath } bind def
/C8 { BL [] 0 setdash 2 copy moveto
      2 copy vpt 270 360 arc closepath fill
              vpt 0 360 arc closepath } bind def
/C9 { BL [] 0 setdash 2 copy moveto
      2 copy  vpt 270 450 arc closepath fill
              vpt 0 360 arc closepath } bind def
/C10 { BL [] 0 setdash 2 copy 2 copy moveto vpt 270 360 arc closepath fill
       2 copy moveto
       2 copy vpt 90 180 arc closepath fill
               vpt 0 360 arc closepath } bind def
/C11 { BL [] 0 setdash 2 copy moveto
       2 copy  vpt 0 180 arc closepath fill
       2 copy moveto
       2 copy  vpt 270 360 arc closepath fill
               vpt 0 360 arc closepath } bind def
/C12 { BL [] 0 setdash 2 copy moveto
       2 copy  vpt 180 360 arc closepath fill
               vpt 0 360 arc closepath } bind def
/C13 { BL [] 0 setdash  2 copy moveto
       2 copy  vpt 0 90 arc closepath fill
       2 copy moveto
       2 copy  vpt 180 360 arc closepath fill
               vpt 0 360 arc closepath } bind def
/C14 { BL [] 0 setdash 2 copy moveto
       2 copy  vpt 90 360 arc closepath fill
               vpt 0 360 arc } bind def
/C15 { BL [] 0 setdash 2 copy vpt 0 360 arc closepath fill
               vpt 0 360 arc closepath } bind def
/Rec   { newpath 4 2 roll moveto 1 index 0 rlineto 0 exch rlineto
       neg 0 rlineto closepath } bind def
/Square { dup Rec } bind def
/Bsquare { vpt sub exch vpt sub exch vpt2 Square } bind def
/S0 { BL [] 0 setdash 2 copy moveto 0 vpt rlineto BL Bsquare } bind def
/S1 { BL [] 0 setdash 2 copy vpt Square fill Bsquare } bind def
/S2 { BL [] 0 setdash 2 copy exch vpt sub exch vpt Square fill Bsquare } bind def
/S3 { BL [] 0 setdash 2 copy exch vpt sub exch vpt2 vpt Rec fill Bsquare } bind def
/S4 { BL [] 0 setdash 2 copy exch vpt sub exch vpt sub vpt Square fill Bsquare } bind def
/S5 { BL [] 0 setdash 2 copy 2 copy vpt Square fill
       exch vpt sub exch vpt sub vpt Square fill Bsquare } bind def
/S6 { BL [] 0 setdash 2 copy exch vpt sub exch vpt sub vpt vpt2 Rec fill Bsquare } bind def
/S7 { BL [] 0 setdash 2 copy exch vpt sub exch vpt sub vpt vpt2 Rec fill
       2 copy vpt Square fill
       Bsquare } bind def
/S8 { BL [] 0 setdash 2 copy vpt sub vpt Square fill Bsquare } bind def
/S9 { BL [] 0 setdash 2 copy vpt sub vpt vpt2 Rec fill Bsquare } bind def
/S10 { BL [] 0 setdash 2 copy vpt sub vpt Square fill 2 copy exch vpt sub exch vpt Square fill
       Bsquare } bind def
/S11 { BL [] 0 setdash 2 copy vpt sub vpt Square fill 2 copy exch vpt sub exch vpt2 vpt Rec fill
       Bsquare } bind def
/S12 { BL [] 0 setdash 2 copy exch vpt sub exch vpt sub vpt2 vpt Rec fill Bsquare } bind def
/S13 { BL [] 0 setdash 2 copy exch vpt sub exch vpt sub vpt2 vpt Rec fill
       2 copy vpt Square fill Bsquare } bind def
/S14 { BL [] 0 setdash 2 copy exch vpt sub exch vpt sub vpt2 vpt Rec fill
       2 copy exch vpt sub exch vpt Square fill Bsquare } bind def
/S15 { BL [] 0 setdash 2 copy Bsquare fill Bsquare } bind def
/D0 { gsave translate 45 rotate 0 0 S0 stroke grestore } bind def
/D1 { gsave translate 45 rotate 0 0 S1 stroke grestore } bind def
/D2 { gsave translate 45 rotate 0 0 S2 stroke grestore } bind def
/D3 { gsave translate 45 rotate 0 0 S3 stroke grestore } bind def
/D4 { gsave translate 45 rotate 0 0 S4 stroke grestore } bind def
/D5 { gsave translate 45 rotate 0 0 S5 stroke grestore } bind def
/D6 { gsave translate 45 rotate 0 0 S6 stroke grestore } bind def
/D7 { gsave translate 45 rotate 0 0 S7 stroke grestore } bind def
/D8 { gsave translate 45 rotate 0 0 S8 stroke grestore } bind def
/D9 { gsave translate 45 rotate 0 0 S9 stroke grestore } bind def
/D10 { gsave translate 45 rotate 0 0 S10 stroke grestore } bind def
/D11 { gsave translate 45 rotate 0 0 S11 stroke grestore } bind def
/D12 { gsave translate 45 rotate 0 0 S12 stroke grestore } bind def
/D13 { gsave translate 45 rotate 0 0 S13 stroke grestore } bind def
/D14 { gsave translate 45 rotate 0 0 S14 stroke grestore } bind def
/D15 { gsave translate 45 rotate 0 0 S15 stroke grestore } bind def
/DiaE { stroke [] 0 setdash vpt add M
  hpt neg vpt neg V hpt vpt neg V
  hpt vpt V hpt neg vpt V closepath stroke } def
/BoxE { stroke [] 0 setdash exch hpt sub exch vpt add M
  0 vpt2 neg V hpt2 0 V 0 vpt2 V
  hpt2 neg 0 V closepath stroke } def
/TriUE { stroke [] 0 setdash vpt 1.12 mul add M
  hpt neg vpt -1.62 mul V
  hpt 2 mul 0 V
  hpt neg vpt 1.62 mul V closepath stroke } def
/TriDE { stroke [] 0 setdash vpt 1.12 mul sub M
  hpt neg vpt 1.62 mul V
  hpt 2 mul 0 V
  hpt neg vpt -1.62 mul V closepath stroke } def
/PentE { stroke [] 0 setdash gsave
  translate 0 hpt M 4 {72 rotate 0 hpt L} repeat
  closepath stroke grestore } def
/CircE { stroke [] 0 setdash 
  hpt 0 360 arc stroke } def
/Opaque { gsave closepath 1 setgray fill grestore 0 setgray closepath } def
/DiaW { stroke [] 0 setdash vpt add M
  hpt neg vpt neg V hpt vpt neg V
  hpt vpt V hpt neg vpt V Opaque stroke } def
/BoxW { stroke [] 0 setdash exch hpt sub exch vpt add M
  0 vpt2 neg V hpt2 0 V 0 vpt2 V
  hpt2 neg 0 V Opaque stroke } def
/TriUW { stroke [] 0 setdash vpt 1.12 mul add M
  hpt neg vpt -1.62 mul V
  hpt 2 mul 0 V
  hpt neg vpt 1.62 mul V Opaque stroke } def
/TriDW { stroke [] 0 setdash vpt 1.12 mul sub M
  hpt neg vpt 1.62 mul V
  hpt 2 mul 0 V
  hpt neg vpt -1.62 mul V Opaque stroke } def
/PentW { stroke [] 0 setdash gsave
  translate 0 hpt M 4 {72 rotate 0 hpt L} repeat
  Opaque stroke grestore } def
/CircW { stroke [] 0 setdash 
  hpt 0 360 arc Opaque stroke } def
/BoxFill { gsave Rec 1 setgray fill grestore } def
end
%%EndProlog
}}%
\begin{picture}(3600,2160)(0,0)%
{\GNUPLOTspecial{"
gnudict begin
gsave
0 0 translate
0.100 0.100 scale
0 setgray
newpath
1.000 UL
LTb
450 300 M
63 0 V
2937 0 R
-63 0 V
450 740 M
63 0 V
2937 0 R
-63 0 V
450 1180 M
63 0 V
2937 0 R
-63 0 V
450 1620 M
63 0 V
2937 0 R
-63 0 V
450 2060 M
63 0 V
2937 0 R
-63 0 V
450 300 M
0 63 V
0 1697 R
0 -63 V
986 300 M
0 63 V
0 1697 R
0 -63 V
1521 300 M
0 63 V
0 1697 R
0 -63 V
2057 300 M
0 63 V
0 1697 R
0 -63 V
2593 300 M
0 63 V
0 1697 R
0 -63 V
3129 300 M
0 63 V
0 1697 R
0 -63 V
1.000 UL
LTb
450 300 M
3000 0 V
0 1760 V
-3000 0 V
450 300 L
1.000 UL
LT0
3087 1947 M
263 0 V
450 300 M
30 1 V
31 15 V
30 45 V
30 84 V
31 122 V
30 151 V
30 165 V
30 167 V
31 156 V
30 134 V
30 108 V
31 77 V
30 47 V
30 17 V
31 -9 V
30 -31 V
30 -49 V
30 -64 V
31 -73 V
30 -79 V
30 -83 V
31 -83 V
30 -81 V
30 -79 V
31 -74 V
30 -69 V
30 -64 V
30 -58 V
31 -52 V
30 -47 V
30 -41 V
31 -37 V
30 -31 V
30 -28 V
31 -24 V
30 -21 V
30 -18 V
31 -15 V
30 -13 V
30 -11 V
30 -9 V
31 -8 V
30 -7 V
30 -5 V
31 -5 V
30 -3 V
30 -4 V
31 -2 V
30 -2 V
30 -2 V
30 -2 V
31 -1 V
30 -1 V
30 -1 V
31 0 V
30 -1 V
30 0 V
31 -1 V
30 0 V
30 0 V
30 0 V
31 0 V
30 0 V
30 -1 V
31 0 V
30 0 V
30 0 V
31 0 V
30 0 V
30 0 V
31 0 V
30 0 V
30 0 V
30 0 V
31 0 V
30 0 V
30 0 V
31 0 V
30 0 V
30 0 V
31 0 V
30 0 V
30 0 V
30 0 V
31 0 V
30 0 V
30 0 V
31 0 V
30 0 V
30 0 V
31 0 V
30 0 V
30 0 V
30 0 V
31 0 V
30 0 V
30 0 V
31 0 V
30 0 V
1.000 UL
LT1
3087 1847 M
263 0 V
450 300 M
30 0 V
31 6 V
30 15 V
30 28 V
31 39 V
30 45 V
30 45 V
30 39 V
31 29 V
30 16 V
30 2 V
31 -12 V
30 -24 V
30 -32 V
31 -37 V
30 -39 V
30 -37 V
30 -33 V
31 -26 V
30 -17 V
30 -7 V
31 3 V
30 13 V
30 22 V
31 31 V
30 38 V
30 43 V
30 48 V
31 50 V
30 52 V
30 51 V
31 50 V
30 48 V
30 44 V
31 40 V
30 35 V
30 31 V
31 25 V
30 19 V
30 15 V
30 9 V
31 4 V
30 0 V
30 -5 V
31 -8 V
30 -12 V
30 -15 V
31 -17 V
30 -20 V
30 -22 V
30 -23 V
31 -24 V
30 -25 V
30 -25 V
31 -25 V
30 -26 V
30 -25 V
31 -25 V
30 -24 V
30 -23 V
30 -23 V
31 -21 V
30 -21 V
30 -20 V
31 -18 V
30 -18 V
30 -17 V
31 -15 V
30 -15 V
30 -14 V
31 -12 V
30 -12 V
30 -11 V
30 -10 V
31 -9 V
30 -9 V
30 -8 V
31 -7 V
30 -6 V
30 -6 V
31 -6 V
30 -5 V
30 -4 V
30 -4 V
31 -4 V
30 -3 V
30 -3 V
31 -3 V
30 -3 V
30 -2 V
31 -2 V
30 -2 V
30 -1 V
30 -2 V
31 -1 V
30 -1 V
30 -1 V
31 -1 V
30 -1 V
stroke
grestore
end
showpage
}}%
\put(3037,1847){\makebox(0,0)[r]{$n=3$ $l=1$ $m_l=0$}}%
\put(3037,1947){\makebox(0,0)[r]{$n=2$ $l=1$ $m_l=0$}}%
\put(1950,50){\makebox(0,0){$r/a_0$}}%
\put(100,1180){%
\special{ps: gsave currentpoint currentpoint translate
270 rotate neg exch neg exch translate}%
\makebox(0,0)[b]{\shortstack{$a_0P(r)$}}%
\special{ps: currentpoint grestore moveto}%
}%
\put(3129,200){\makebox(0,0){25}}%
\put(2593,200){\makebox(0,0){20}}%
\put(2057,200){\makebox(0,0){15}}%
\put(1521,200){\makebox(0,0){10}}%
\put(986,200){\makebox(0,0){5}}%
\put(450,200){\makebox(0,0){0}}%
\put(400,2060){\makebox(0,0)[r]{0.2}}%
\put(400,1620){\makebox(0,0)[r]{0.15}}%
\put(400,1180){\makebox(0,0)[r]{0.1}}%
\put(400,740){\makebox(0,0)[r]{0.05}}%
\put(400,300){\makebox(0,0)[r]{0}}%
\end{picture}%
\endgroup
\endinput

\caption{Plot av $a_0P(r)$ for $l=1$ tilstander opp til $n=3$ for 
$m_l=0$.\label{44}}
\end{center}
\end{figure}

Vi definerer n\aa\ sannsynlighetstettheten for den radielle delen
av egenfunksjonen som
\be
   P_{nl}dr=R_{nl}^*R_{nl}r^2dr,
\ee
som uttrykker sannsynligheten for \aa\ finne elektronet i et bestemt
omr\aa de mellom $r$ og $r+dr$, dvs.~en bestemt avstand fra protonet.
Denne sannsynligheten er ogs\aa\ normert til 1, akkurat som den tilsvarende
sannsynligheten for de sf\ae riske harmoniske funksjonene.
Den totale sannsynligheten blir dermed 
\be
   P_{nlm_l}(r,\theta,\phi)d\tau=R_{nl}(r)^*R_{nl}(r)r^2dr
   Y_{lm_l}(\theta,\phi)^*Y_{lm_l}(\theta,\phi)d\Omega,
\ee
med romvinkelen $d\Omega=sin\theta d\theta d\phi$.

I figur \ref{43} og \ref{44} har 
vi plottet $P_{nl}$ som funksjon av $r/a_0$ for
ulike $l$ og $n$ verdier.
\begin{figure}
\begin{center}
% GNUPLOT: LaTeX picture with Postscript
\begingroup%
  \makeatletter%
  \newcommand{\GNUPLOTspecial}{%
    \@sanitize\catcode`\%=14\relax\special}%
  \setlength{\unitlength}{0.1bp}%
{\GNUPLOTspecial{!
%!PS-Adobe-2.0 EPSF-2.0
%%Title: wave3.tex
%%Creator: gnuplot 3.7 patchlevel 0.2
%%CreationDate: Fri Mar 17 10:48:54 2000
%%DocumentFonts: 
%%BoundingBox: 0 0 360 216
%%Orientation: Landscape
%%EndComments
/gnudict 256 dict def
gnudict begin
/Color false def
/Solid false def
/gnulinewidth 5.000 def
/userlinewidth gnulinewidth def
/vshift -33 def
/dl {10 mul} def
/hpt_ 31.5 def
/vpt_ 31.5 def
/hpt hpt_ def
/vpt vpt_ def
/M {moveto} bind def
/L {lineto} bind def
/R {rmoveto} bind def
/V {rlineto} bind def
/vpt2 vpt 2 mul def
/hpt2 hpt 2 mul def
/Lshow { currentpoint stroke M
  0 vshift R show } def
/Rshow { currentpoint stroke M
  dup stringwidth pop neg vshift R show } def
/Cshow { currentpoint stroke M
  dup stringwidth pop -2 div vshift R show } def
/UP { dup vpt_ mul /vpt exch def hpt_ mul /hpt exch def
  /hpt2 hpt 2 mul def /vpt2 vpt 2 mul def } def
/DL { Color {setrgbcolor Solid {pop []} if 0 setdash }
 {pop pop pop Solid {pop []} if 0 setdash} ifelse } def
/BL { stroke userlinewidth 2 mul setlinewidth } def
/AL { stroke userlinewidth 2 div setlinewidth } def
/UL { dup gnulinewidth mul /userlinewidth exch def
      10 mul /udl exch def } def
/PL { stroke userlinewidth setlinewidth } def
/LTb { BL [] 0 0 0 DL } def
/LTa { AL [1 udl mul 2 udl mul] 0 setdash 0 0 0 setrgbcolor } def
/LT0 { PL [] 1 0 0 DL } def
/LT1 { PL [4 dl 2 dl] 0 1 0 DL } def
/LT2 { PL [2 dl 3 dl] 0 0 1 DL } def
/LT3 { PL [1 dl 1.5 dl] 1 0 1 DL } def
/LT4 { PL [5 dl 2 dl 1 dl 2 dl] 0 1 1 DL } def
/LT5 { PL [4 dl 3 dl 1 dl 3 dl] 1 1 0 DL } def
/LT6 { PL [2 dl 2 dl 2 dl 4 dl] 0 0 0 DL } def
/LT7 { PL [2 dl 2 dl 2 dl 2 dl 2 dl 4 dl] 1 0.3 0 DL } def
/LT8 { PL [2 dl 2 dl 2 dl 2 dl 2 dl 2 dl 2 dl 4 dl] 0.5 0.5 0.5 DL } def
/Pnt { stroke [] 0 setdash
   gsave 1 setlinecap M 0 0 V stroke grestore } def
/Dia { stroke [] 0 setdash 2 copy vpt add M
  hpt neg vpt neg V hpt vpt neg V
  hpt vpt V hpt neg vpt V closepath stroke
  Pnt } def
/Pls { stroke [] 0 setdash vpt sub M 0 vpt2 V
  currentpoint stroke M
  hpt neg vpt neg R hpt2 0 V stroke
  } def
/Box { stroke [] 0 setdash 2 copy exch hpt sub exch vpt add M
  0 vpt2 neg V hpt2 0 V 0 vpt2 V
  hpt2 neg 0 V closepath stroke
  Pnt } def
/Crs { stroke [] 0 setdash exch hpt sub exch vpt add M
  hpt2 vpt2 neg V currentpoint stroke M
  hpt2 neg 0 R hpt2 vpt2 V stroke } def
/TriU { stroke [] 0 setdash 2 copy vpt 1.12 mul add M
  hpt neg vpt -1.62 mul V
  hpt 2 mul 0 V
  hpt neg vpt 1.62 mul V closepath stroke
  Pnt  } def
/Star { 2 copy Pls Crs } def
/BoxF { stroke [] 0 setdash exch hpt sub exch vpt add M
  0 vpt2 neg V  hpt2 0 V  0 vpt2 V
  hpt2 neg 0 V  closepath fill } def
/TriUF { stroke [] 0 setdash vpt 1.12 mul add M
  hpt neg vpt -1.62 mul V
  hpt 2 mul 0 V
  hpt neg vpt 1.62 mul V closepath fill } def
/TriD { stroke [] 0 setdash 2 copy vpt 1.12 mul sub M
  hpt neg vpt 1.62 mul V
  hpt 2 mul 0 V
  hpt neg vpt -1.62 mul V closepath stroke
  Pnt  } def
/TriDF { stroke [] 0 setdash vpt 1.12 mul sub M
  hpt neg vpt 1.62 mul V
  hpt 2 mul 0 V
  hpt neg vpt -1.62 mul V closepath fill} def
/DiaF { stroke [] 0 setdash vpt add M
  hpt neg vpt neg V hpt vpt neg V
  hpt vpt V hpt neg vpt V closepath fill } def
/Pent { stroke [] 0 setdash 2 copy gsave
  translate 0 hpt M 4 {72 rotate 0 hpt L} repeat
  closepath stroke grestore Pnt } def
/PentF { stroke [] 0 setdash gsave
  translate 0 hpt M 4 {72 rotate 0 hpt L} repeat
  closepath fill grestore } def
/Circle { stroke [] 0 setdash 2 copy
  hpt 0 360 arc stroke Pnt } def
/CircleF { stroke [] 0 setdash hpt 0 360 arc fill } def
/C0 { BL [] 0 setdash 2 copy moveto vpt 90 450  arc } bind def
/C1 { BL [] 0 setdash 2 copy        moveto
       2 copy  vpt 0 90 arc closepath fill
               vpt 0 360 arc closepath } bind def
/C2 { BL [] 0 setdash 2 copy moveto
       2 copy  vpt 90 180 arc closepath fill
               vpt 0 360 arc closepath } bind def
/C3 { BL [] 0 setdash 2 copy moveto
       2 copy  vpt 0 180 arc closepath fill
               vpt 0 360 arc closepath } bind def
/C4 { BL [] 0 setdash 2 copy moveto
       2 copy  vpt 180 270 arc closepath fill
               vpt 0 360 arc closepath } bind def
/C5 { BL [] 0 setdash 2 copy moveto
       2 copy  vpt 0 90 arc
       2 copy moveto
       2 copy  vpt 180 270 arc closepath fill
               vpt 0 360 arc } bind def
/C6 { BL [] 0 setdash 2 copy moveto
      2 copy  vpt 90 270 arc closepath fill
              vpt 0 360 arc closepath } bind def
/C7 { BL [] 0 setdash 2 copy moveto
      2 copy  vpt 0 270 arc closepath fill
              vpt 0 360 arc closepath } bind def
/C8 { BL [] 0 setdash 2 copy moveto
      2 copy vpt 270 360 arc closepath fill
              vpt 0 360 arc closepath } bind def
/C9 { BL [] 0 setdash 2 copy moveto
      2 copy  vpt 270 450 arc closepath fill
              vpt 0 360 arc closepath } bind def
/C10 { BL [] 0 setdash 2 copy 2 copy moveto vpt 270 360 arc closepath fill
       2 copy moveto
       2 copy vpt 90 180 arc closepath fill
               vpt 0 360 arc closepath } bind def
/C11 { BL [] 0 setdash 2 copy moveto
       2 copy  vpt 0 180 arc closepath fill
       2 copy moveto
       2 copy  vpt 270 360 arc closepath fill
               vpt 0 360 arc closepath } bind def
/C12 { BL [] 0 setdash 2 copy moveto
       2 copy  vpt 180 360 arc closepath fill
               vpt 0 360 arc closepath } bind def
/C13 { BL [] 0 setdash  2 copy moveto
       2 copy  vpt 0 90 arc closepath fill
       2 copy moveto
       2 copy  vpt 180 360 arc closepath fill
               vpt 0 360 arc closepath } bind def
/C14 { BL [] 0 setdash 2 copy moveto
       2 copy  vpt 90 360 arc closepath fill
               vpt 0 360 arc } bind def
/C15 { BL [] 0 setdash 2 copy vpt 0 360 arc closepath fill
               vpt 0 360 arc closepath } bind def
/Rec   { newpath 4 2 roll moveto 1 index 0 rlineto 0 exch rlineto
       neg 0 rlineto closepath } bind def
/Square { dup Rec } bind def
/Bsquare { vpt sub exch vpt sub exch vpt2 Square } bind def
/S0 { BL [] 0 setdash 2 copy moveto 0 vpt rlineto BL Bsquare } bind def
/S1 { BL [] 0 setdash 2 copy vpt Square fill Bsquare } bind def
/S2 { BL [] 0 setdash 2 copy exch vpt sub exch vpt Square fill Bsquare } bind def
/S3 { BL [] 0 setdash 2 copy exch vpt sub exch vpt2 vpt Rec fill Bsquare } bind def
/S4 { BL [] 0 setdash 2 copy exch vpt sub exch vpt sub vpt Square fill Bsquare } bind def
/S5 { BL [] 0 setdash 2 copy 2 copy vpt Square fill
       exch vpt sub exch vpt sub vpt Square fill Bsquare } bind def
/S6 { BL [] 0 setdash 2 copy exch vpt sub exch vpt sub vpt vpt2 Rec fill Bsquare } bind def
/S7 { BL [] 0 setdash 2 copy exch vpt sub exch vpt sub vpt vpt2 Rec fill
       2 copy vpt Square fill
       Bsquare } bind def
/S8 { BL [] 0 setdash 2 copy vpt sub vpt Square fill Bsquare } bind def
/S9 { BL [] 0 setdash 2 copy vpt sub vpt vpt2 Rec fill Bsquare } bind def
/S10 { BL [] 0 setdash 2 copy vpt sub vpt Square fill 2 copy exch vpt sub exch vpt Square fill
       Bsquare } bind def
/S11 { BL [] 0 setdash 2 copy vpt sub vpt Square fill 2 copy exch vpt sub exch vpt2 vpt Rec fill
       Bsquare } bind def
/S12 { BL [] 0 setdash 2 copy exch vpt sub exch vpt sub vpt2 vpt Rec fill Bsquare } bind def
/S13 { BL [] 0 setdash 2 copy exch vpt sub exch vpt sub vpt2 vpt Rec fill
       2 copy vpt Square fill Bsquare } bind def
/S14 { BL [] 0 setdash 2 copy exch vpt sub exch vpt sub vpt2 vpt Rec fill
       2 copy exch vpt sub exch vpt Square fill Bsquare } bind def
/S15 { BL [] 0 setdash 2 copy Bsquare fill Bsquare } bind def
/D0 { gsave translate 45 rotate 0 0 S0 stroke grestore } bind def
/D1 { gsave translate 45 rotate 0 0 S1 stroke grestore } bind def
/D2 { gsave translate 45 rotate 0 0 S2 stroke grestore } bind def
/D3 { gsave translate 45 rotate 0 0 S3 stroke grestore } bind def
/D4 { gsave translate 45 rotate 0 0 S4 stroke grestore } bind def
/D5 { gsave translate 45 rotate 0 0 S5 stroke grestore } bind def
/D6 { gsave translate 45 rotate 0 0 S6 stroke grestore } bind def
/D7 { gsave translate 45 rotate 0 0 S7 stroke grestore } bind def
/D8 { gsave translate 45 rotate 0 0 S8 stroke grestore } bind def
/D9 { gsave translate 45 rotate 0 0 S9 stroke grestore } bind def
/D10 { gsave translate 45 rotate 0 0 S10 stroke grestore } bind def
/D11 { gsave translate 45 rotate 0 0 S11 stroke grestore } bind def
/D12 { gsave translate 45 rotate 0 0 S12 stroke grestore } bind def
/D13 { gsave translate 45 rotate 0 0 S13 stroke grestore } bind def
/D14 { gsave translate 45 rotate 0 0 S14 stroke grestore } bind def
/D15 { gsave translate 45 rotate 0 0 S15 stroke grestore } bind def
/DiaE { stroke [] 0 setdash vpt add M
  hpt neg vpt neg V hpt vpt neg V
  hpt vpt V hpt neg vpt V closepath stroke } def
/BoxE { stroke [] 0 setdash exch hpt sub exch vpt add M
  0 vpt2 neg V hpt2 0 V 0 vpt2 V
  hpt2 neg 0 V closepath stroke } def
/TriUE { stroke [] 0 setdash vpt 1.12 mul add M
  hpt neg vpt -1.62 mul V
  hpt 2 mul 0 V
  hpt neg vpt 1.62 mul V closepath stroke } def
/TriDE { stroke [] 0 setdash vpt 1.12 mul sub M
  hpt neg vpt 1.62 mul V
  hpt 2 mul 0 V
  hpt neg vpt -1.62 mul V closepath stroke } def
/PentE { stroke [] 0 setdash gsave
  translate 0 hpt M 4 {72 rotate 0 hpt L} repeat
  closepath stroke grestore } def
/CircE { stroke [] 0 setdash 
  hpt 0 360 arc stroke } def
/Opaque { gsave closepath 1 setgray fill grestore 0 setgray closepath } def
/DiaW { stroke [] 0 setdash vpt add M
  hpt neg vpt neg V hpt vpt neg V
  hpt vpt V hpt neg vpt V Opaque stroke } def
/BoxW { stroke [] 0 setdash exch hpt sub exch vpt add M
  0 vpt2 neg V hpt2 0 V 0 vpt2 V
  hpt2 neg 0 V Opaque stroke } def
/TriUW { stroke [] 0 setdash vpt 1.12 mul add M
  hpt neg vpt -1.62 mul V
  hpt 2 mul 0 V
  hpt neg vpt 1.62 mul V Opaque stroke } def
/TriDW { stroke [] 0 setdash vpt 1.12 mul sub M
  hpt neg vpt 1.62 mul V
  hpt 2 mul 0 V
  hpt neg vpt -1.62 mul V Opaque stroke } def
/PentW { stroke [] 0 setdash gsave
  translate 0 hpt M 4 {72 rotate 0 hpt L} repeat
  Opaque stroke grestore } def
/CircW { stroke [] 0 setdash 
  hpt 0 360 arc Opaque stroke } def
/BoxFill { gsave Rec 1 setgray fill grestore } def
end
%%EndProlog
}}%
\begin{picture}(3600,2160)(0,0)%
{\GNUPLOTspecial{"
gnudict begin
gsave
0 0 translate
0.100 0.100 scale
0 setgray
newpath
1.000 UL
LTb
1.000 UL
LT0
3124 1796 M
263 0 V
522 893 M
17 6 V
16 21 V
16 31 V
17 37 V
16 41 V
16 42 V
17 42 V
16 41 V
16 38 V
17 34 V
16 31 V
17 28 V
16 23 V
16 19 V
17 15 V
16 11 V
16 8 V
17 4 V
16 2 V
16 -2 V
17 -4 V
16 -7 V
17 -9 V
16 -10 V
16 -13 V
17 -13 V
16 -15 V
16 -15 V
17 -17 V
16 -17 V
16 -17 V
17 -18 V
16 -18 V
16 -18 V
17 -18 V
16 -18 V
17 -18 V
16 -17 V
16 -17 V
17 -17 V
16 -17 V
16 -16 V
17 -16 V
16 -15 V
16 -14 V
17 -15 V
16 -14 V
16 -13 V
17 -13 V
16 -12 V
17 -12 V
16 -12 V
16 -11 V
17 -10 V
16 -10 V
16 -10 V
17 -10 V
16 -9 V
16 -8 V
17 -9 V
16 -8 V
16 -7 V
17 -8 V
16 -7 V
17 -6 V
16 -7 V
16 -6 V
17 -6 V
16 -6 V
16 -6 V
17 -5 V
16 -5 V
16 -5 V
17 -5 V
16 -5 V
16 -5 V
17 -4 V
16 -4 V
17 -5 V
16 -4 V
16 -4 V
17 -3 V
16 -4 V
16 -4 V
17 -3 V
16 -4 V
16 -3 V
17 -4 V
16 -3 V
17 -3 V
16 -4 V
16 -3 V
17 -3 V
16 -3 V
16 -3 V
17 -3 V
16 -3 V
16 -3 V
17 -3 V
626 941 M
17 6 V
16 21 V
16 31 V
17 37 V
16 41 V
16 42 V
17 42 V
16 41 V
16 38 V
17 35 V
16 31 V
16 27 V
17 23 V
16 19 V
17 15 V
16 12 V
16 7 V
17 5 V
16 1 V
16 -2 V
17 -4 V
16 -7 V
16 -9 V
17 -10 V
16 -12 V
16 -14 V
17 -15 V
16 -15 V
17 -17 V
16 -17 V
16 -17 V
17 -18 V
16 -18 V
16 -18 V
17 -18 V
16 -18 V
16 -17 V
17 -18 V
16 -17 V
16 -17 V
17 -16 V
16 -17 V
17 -15 V
16 -15 V
16 -15 V
17 -14 V
16 -14 V
16 -14 V
17 -12 V
16 -13 V
16 -12 V
17 -11 V
16 -11 V
17 -11 V
16 -10 V
16 -10 V
17 -9 V
16 -9 V
16 -9 V
17 -8 V
16 -8 V
16 -8 V
17 -7 V
16 -7 V
16 -7 V
17 -7 V
16 -6 V
17 -6 V
16 -6 V
16 -5 V
17 -6 V
16 -5 V
16 -5 V
17 -5 V
16 -5 V
16 -4 V
17 -5 V
16 -4 V
16 -4 V
17 -4 V
16 -4 V
17 -4 V
16 -4 V
16 -4 V
17 -3 V
16 -4 V
16 -3 V
17 -4 V
16 -3 V
16 -3 V
17 -3 V
16 -4 V
16 -3 V
17 -3 V
16 -3 V
17 -3 V
16 -3 V
16 -3 V
17 -3 V
730 989 M
16 7 V
17 20 V
16 31 V
16 37 V
17 41 V
16 43 V
17 42 V
16 40 V
16 38 V
17 35 V
16 31 V
16 27 V
17 23 V
16 20 V
16 15 V
17 11 V
16 8 V
16 4 V
17 1 V
16 -1 V
17 -5 V
16 -6 V
16 -9 V
17 -11 V
16 -12 V
16 -14 V
17 -14 V
16 -16 V
16 -16 V
17 -17 V
16 -18 V
17 -17 V
16 -18 V
16 -18 V
17 -18 V
16 -18 V
16 -18 V
17 -18 V
16 -17 V
16 -17 V
17 -16 V
16 -16 V
16 -16 V
17 -15 V
16 -15 V
17 -14 V
16 -14 V
16 -13 V
17 -13 V
16 -13 V
16 -12 V
17 -11 V
16 -11 V
16 -11 V
17 -10 V
16 -10 V
16 -9 V
17 -9 V
16 -9 V
17 -8 V
16 -8 V
16 -8 V
17 -7 V
16 -7 V
16 -7 V
17 -6 V
16 -7 V
16 -6 V
17 -6 V
16 -5 V
16 -6 V
17 -5 V
16 -5 V
17 -5 V
16 -4 V
16 -5 V
17 -4 V
16 -5 V
16 -4 V
17 -4 V
16 -4 V
16 -4 V
17 -4 V
16 -3 V
16 -4 V
17 -3 V
16 -4 V
17 -3 V
16 -4 V
16 -3 V
17 -3 V
16 -3 V
16 -3 V
17 -3 V
16 -3 V
16 -3 V
17 -3 V
16 -3 V
17 -3 V
834 1038 M
16 6 V
17 21 V
16 31 V
16 37 V
17 41 V
16 42 V
16 42 V
17 41 V
16 37 V
17 35 V
16 31 V
16 28 V
17 23 V
16 19 V
16 15 V
17 11 V
16 8 V
16 4 V
17 2 V
16 -2 V
16 -4 V
17 -7 V
16 -9 V
17 -10 V
16 -13 V
16 -13 V
17 -15 V
16 -16 V
16 -16 V
17 -17 V
16 -17 V
16 -18 V
17 -18 V
16 -18 V
16 -18 V
17 -18 V
16 -18 V
17 -17 V
16 -17 V
16 -17 V
17 -17 V
16 -16 V
16 -16 V
17 -15 V
16 -15 V
16 -14 V
17 -14 V
16 -13 V
16 -13 V
17 -12 V
16 -12 V
17 -12 V
16 -11 V
16 -10 V
17 -11 V
16 -9 V
16 -10 V
17 -9 V
16 -8 V
16 -9 V
17 -8 V
16 -7 V
17 -8 V
16 -7 V
16 -7 V
17 -6 V
16 -6 V
16 -6 V
17 -6 V
16 -6 V
16 -5 V
17 -5 V
16 -5 V
16 -5 V
17 -5 V
16 -5 V
17 -4 V
16 -4 V
16 -5 V
17 -4 V
16 -4 V
16 -4 V
17 -3 V
16 -4 V
16 -4 V
17 -3 V
16 -4 V
16 -3 V
17 -3 V
16 -4 V
17 -3 V
16 -3 V
16 -3 V
17 -3 V
16 -3 V
16 -3 V
17 -3 V
currentpoint stroke M
16 -3 V
16 -3 V
938 1086 M
16 6 V
17 21 V
16 31 V
16 37 V
17 41 V
16 42 V
16 42 V
17 41 V
16 38 V
16 35 V
17 31 V
16 27 V
16 23 V
17 19 V
16 15 V
17 12 V
16 7 V
16 5 V
17 1 V
16 -2 V
16 -4 V
17 -7 V
16 -9 V
16 -10 V
17 -12 V
16 -14 V
16 -15 V
17 -15 V
16 -17 V
17 -17 V
16 -17 V
16 -18 V
17 -18 V
16 -18 V
16 -18 V
17 -18 V
16 -17 V
16 -18 V
17 -17 V
16 -17 V
17 -17 V
16 -16 V
16 -15 V
17 -16 V
16 -14 V
16 -15 V
17 -13 V
16 -14 V
16 -13 V
17 -12 V
16 -12 V
16 -11 V
17 -11 V
16 -11 V
17 -10 V
16 -10 V
16 -9 V
17 -9 V
16 -9 V
16 -8 V
17 -8 V
16 -8 V
16 -7 V
17 -7 V
16 -7 V
16 -7 V
17 -6 V
16 -6 V
17 -6 V
16 -6 V
16 -5 V
17 -5 V
16 -5 V
16 -5 V
17 -5 V
16 -4 V
16 -5 V
17 -4 V
16 -4 V
16 -4 V
17 -4 V
16 -4 V
17 -4 V
16 -4 V
16 -3 V
17 -4 V
16 -3 V
16 -4 V
17 -3 V
16 -3 V
16 -3 V
17 -4 V
16 -3 V
16 -3 V
17 -3 V
16 -3 V
17 -3 V
16 -3 V
16 -3 V
1042 1134 M
16 7 V
16 20 V
17 31 V
16 37 V
17 41 V
16 43 V
16 42 V
17 40 V
16 38 V
16 35 V
17 31 V
16 27 V
16 23 V
17 19 V
16 16 V
16 11 V
17 8 V
16 4 V
17 1 V
16 -2 V
16 -4 V
17 -7 V
16 -8 V
16 -11 V
17 -12 V
16 -14 V
16 -14 V
17 -16 V
16 -16 V
16 -17 V
17 -18 V
16 -18 V
17 -17 V
16 -19 V
16 -18 V
17 -17 V
16 -18 V
16 -18 V
17 -17 V
16 -17 V
16 -16 V
17 -16 V
16 -16 V
16 -15 V
17 -15 V
16 -14 V
17 -14 V
16 -13 V
16 -13 V
17 -13 V
16 -12 V
16 -11 V
17 -11 V
16 -11 V
16 -10 V
17 -10 V
16 -9 V
16 -9 V
17 -9 V
16 -8 V
17 -8 V
16 -8 V
16 -7 V
17 -7 V
16 -7 V
16 -6 V
17 -7 V
16 -6 V
16 -6 V
17 -5 V
16 -6 V
17 -5 V
16 -5 V
16 -5 V
17 -4 V
16 -5 V
16 -5 V
17 -4 V
16 -4 V
16 -4 V
17 -4 V
16 -4 V
16 -4 V
17 -3 V
16 -4 V
17 -3 V
16 -4 V
16 -3 V
17 -4 V
16 -3 V
16 -3 V
17 -3 V
16 -3 V
16 -4 V
17 -3 V
16 -3 V
16 -3 V
17 -2 V
16 -3 V
1146 1183 M
16 6 V
16 21 V
17 30 V
16 38 V
16 41 V
17 42 V
16 42 V
16 40 V
17 38 V
16 35 V
17 31 V
16 27 V
16 24 V
17 19 V
16 15 V
16 11 V
17 8 V
16 4 V
16 1 V
17 -1 V
16 -5 V
16 -6 V
17 -9 V
16 -11 V
17 -12 V
16 -13 V
16 -15 V
17 -16 V
16 -16 V
16 -17 V
17 -17 V
16 -18 V
16 -18 V
17 -18 V
16 -18 V
16 -18 V
17 -18 V
16 -17 V
17 -18 V
16 -16 V
16 -17 V
17 -16 V
16 -16 V
16 -15 V
17 -15 V
16 -14 V
16 -14 V
17 -13 V
16 -13 V
17 -12 V
16 -12 V
16 -12 V
17 -11 V
16 -10 V
16 -11 V
17 -9 V
16 -10 V
16 -9 V
17 -8 V
16 -9 V
16 -8 V
17 -7 V
16 -8 V
17 -7 V
16 -7 V
16 -6 V
17 -6 V
16 -6 V
16 -6 V
17 -6 V
16 -5 V
16 -6 V
17 -5 V
16 -4 V
16 -5 V
17 -5 V
16 -4 V
17 -5 V
16 -4 V
16 -4 V
17 -4 V
16 -4 V
16 -3 V
17 -4 V
16 -4 V
16 -3 V
17 -4 V
16 -3 V
16 -3 V
17 -4 V
16 -3 V
17 -3 V
16 -3 V
16 -3 V
17 -3 V
16 -3 V
16 -3 V
17 -3 V
16 -3 V
1249 1231 M
17 6 V
16 21 V
17 31 V
16 37 V
16 41 V
17 42 V
16 42 V
16 41 V
17 38 V
16 34 V
16 32 V
17 27 V
16 23 V
17 19 V
16 15 V
16 12 V
17 7 V
16 5 V
16 1 V
17 -2 V
16 -4 V
16 -7 V
17 -9 V
16 -10 V
16 -12 V
17 -14 V
16 -15 V
17 -15 V
16 -17 V
16 -17 V
17 -17 V
16 -18 V
16 -18 V
17 -18 V
16 -18 V
16 -18 V
17 -18 V
16 -17 V
16 -17 V
17 -17 V
16 -17 V
17 -16 V
16 -15 V
16 -16 V
17 -14 V
16 -15 V
16 -13 V
17 -14 V
16 -13 V
16 -12 V
17 -12 V
16 -11 V
16 -12 V
17 -10 V
16 -10 V
17 -10 V
16 -10 V
16 -9 V
17 -8 V
16 -8 V
16 -8 V
17 -8 V
16 -7 V
16 -8 V
17 -6 V
16 -7 V
16 -6 V
17 -6 V
16 -6 V
17 -6 V
16 -5 V
16 -5 V
17 -5 V
16 -5 V
16 -5 V
17 -4 V
16 -5 V
16 -4 V
17 -4 V
16 -4 V
17 -4 V
16 -4 V
16 -4 V
17 -4 V
16 -3 V
16 -4 V
17 -3 V
16 -4 V
16 -3 V
17 -3 V
16 -4 V
16 -3 V
17 -3 V
16 -3 V
17 -3 V
16 -3 V
16 -3 V
currentpoint stroke M
17 -3 V
16 -3 V
1353 1279 M
17 7 V
16 20 V
16 31 V
17 37 V
16 41 V
17 43 V
16 42 V
16 40 V
17 38 V
16 35 V
16 31 V
17 27 V
16 23 V
16 19 V
17 16 V
16 11 V
16 8 V
17 4 V
16 1 V
17 -2 V
16 -4 V
16 -7 V
17 -8 V
16 -11 V
16 -12 V
17 -14 V
16 -14 V
16 -16 V
17 -16 V
16 -17 V
16 -18 V
17 -18 V
16 -18 V
17 -18 V
16 -18 V
16 -18 V
17 -17 V
16 -18 V
16 -17 V
17 -17 V
16 -16 V
16 -16 V
17 -16 V
16 -15 V
16 -15 V
17 -14 V
16 -14 V
17 -13 V
16 -13 V
16 -13 V
17 -12 V
16 -11 V
16 -11 V
17 -11 V
16 -10 V
16 -10 V
17 -9 V
16 -9 V
17 -9 V
16 -8 V
16 -8 V
17 -8 V
16 -7 V
16 -7 V
17 -7 V
16 -7 V
16 -6 V
17 -6 V
16 -6 V
16 -5 V
17 -6 V
16 -5 V
17 -5 V
16 -5 V
16 -5 V
17 -4 V
16 -5 V
16 -4 V
17 -4 V
16 -4 V
16 -4 V
17 -4 V
16 -4 V
16 -3 V
17 -4 V
16 -4 V
17 -3 V
16 -3 V
16 -4 V
17 -3 V
16 -3 V
16 -3 V
17 -4 V
16 -3 V
16 -3 V
17 -3 V
16 -3 V
16 -3 V
17 -2 V
1457 1327 M
17 7 V
16 21 V
16 30 V
17 38 V
16 40 V
16 43 V
17 42 V
16 40 V
16 38 V
17 35 V
16 31 V
17 27 V
16 24 V
16 19 V
17 15 V
16 11 V
16 8 V
17 4 V
16 1 V
16 -1 V
17 -5 V
16 -6 V
17 -9 V
16 -11 V
16 -12 V
17 -13 V
16 -15 V
16 -16 V
17 -16 V
16 -17 V
16 -18 V
17 -17 V
16 -18 V
16 -18 V
17 -18 V
16 -18 V
17 -18 V
16 -17 V
16 -18 V
17 -17 V
16 -16 V
16 -16 V
17 -16 V
16 -15 V
16 -15 V
17 -14 V
16 -14 V
16 -13 V
17 -13 V
16 -12 V
17 -12 V
16 -12 V
16 -11 V
17 -11 V
16 -10 V
16 -10 V
17 -9 V
16 -9 V
16 -9 V
17 -8 V
16 -8 V
16 -8 V
17 -7 V
16 -7 V
17 -7 V
16 -6 V
16 -7 V
17 -6 V
16 -5 V
16 -6 V
17 -5 V
16 -6 V
16 -5 V
17 -5 V
16 -4 V
16 -5 V
17 -4 V
16 -5 V
17 -4 V
16 -4 V
16 -4 V
17 -4 V
16 -3 V
16 -4 V
17 -4 V
16 -3 V
16 -4 V
17 -3 V
16 -3 V
17 -4 V
16 -3 V
16 -3 V
17 -3 V
16 -3 V
16 -3 V
17 -3 V
16 -3 V
16 -3 V
17 -3 V
522 893 M
10 4 V
9 4 V
10 5 V
9 4 V
9 5 V
10 4 V
9 4 V
10 5 V
9 4 V
10 5 V
9 4 V
10 4 V
9 5 V
9 4 V
10 5 V
9 4 V
10 4 V
9 5 V
10 4 V
9 5 V
10 4 V
9 4 V
9 5 V
10 4 V
9 4 V
10 5 V
9 4 V
10 5 V
9 4 V
10 4 V
9 5 V
9 4 V
10 5 V
9 4 V
10 4 V
9 5 V
10 4 V
9 5 V
10 4 V
9 4 V
9 5 V
10 4 V
9 5 V
10 4 V
9 4 V
10 5 V
9 4 V
10 4 V
9 5 V
9 4 V
10 5 V
9 4 V
10 4 V
9 5 V
10 4 V
9 5 V
10 4 V
9 4 V
9 5 V
10 4 V
9 5 V
10 4 V
9 4 V
10 5 V
9 4 V
10 5 V
9 4 V
9 4 V
10 5 V
9 4 V
10 4 V
9 5 V
10 4 V
9 5 V
10 4 V
9 4 V
9 5 V
10 4 V
9 5 V
10 4 V
9 4 V
10 5 V
9 4 V
10 5 V
9 4 V
9 4 V
10 5 V
9 4 V
10 5 V
9 4 V
10 4 V
9 5 V
10 4 V
9 5 V
9 4 V
10 4 V
9 5 V
10 4 V
9 4 V
702 1257 M
10 5 V
9 4 V
10 5 V
9 4 V
9 4 V
10 5 V
9 4 V
10 5 V
9 4 V
10 4 V
9 5 V
10 4 V
9 5 V
9 4 V
10 4 V
9 5 V
10 4 V
9 5 V
10 4 V
9 4 V
10 5 V
9 4 V
9 4 V
10 5 V
9 4 V
10 5 V
9 4 V
10 4 V
9 5 V
10 4 V
9 5 V
9 4 V
10 4 V
9 5 V
10 4 V
9 5 V
10 4 V
9 4 V
10 5 V
9 4 V
9 5 V
10 4 V
9 4 V
10 5 V
9 4 V
10 4 V
9 5 V
10 4 V
9 5 V
9 4 V
10 4 V
9 5 V
10 4 V
9 5 V
10 4 V
9 4 V
10 5 V
9 4 V
9 5 V
10 4 V
9 4 V
10 5 V
9 4 V
10 5 V
9 4 V
10 4 V
9 5 V
9 4 V
10 4 V
9 5 V
10 4 V
9 5 V
10 4 V
9 4 V
10 5 V
9 4 V
9 5 V
10 4 V
9 4 V
10 5 V
9 4 V
10 5 V
9 4 V
10 4 V
9 5 V
9 4 V
10 5 V
9 4 V
10 4 V
9 5 V
10 4 V
9 4 V
10 5 V
9 4 V
9 5 V
10 4 V
9 4 V
currentpoint stroke M
10 5 V
9 4 V
882 1354 M
10 4 V
9 5 V
9 4 V
10 5 V
9 4 V
10 4 V
9 5 V
10 4 V
9 4 V
10 5 V
9 4 V
9 5 V
10 4 V
9 4 V
10 5 V
9 4 V
10 5 V
9 4 V
10 4 V
9 5 V
9 4 V
10 5 V
9 4 V
10 4 V
9 5 V
10 4 V
9 5 V
10 4 V
9 4 V
9 5 V
10 4 V
9 4 V
10 5 V
9 4 V
10 5 V
9 4 V
10 4 V
9 5 V
9 4 V
10 5 V
9 4 V
10 4 V
9 5 V
10 4 V
9 5 V
10 4 V
9 4 V
9 5 V
10 4 V
9 5 V
10 4 V
9 4 V
10 5 V
9 4 V
10 4 V
9 5 V
9 4 V
10 5 V
9 4 V
10 4 V
9 5 V
10 4 V
9 5 V
10 4 V
9 4 V
9 5 V
10 4 V
9 5 V
10 4 V
9 4 V
10 5 V
9 4 V
10 5 V
9 4 V
9 4 V
10 5 V
9 4 V
10 4 V
9 5 V
10 4 V
9 5 V
10 4 V
9 4 V
9 5 V
10 4 V
9 5 V
10 4 V
9 4 V
10 5 V
9 4 V
10 5 V
9 4 V
9 4 V
10 5 V
9 4 V
10 5 V
9 4 V
10 4 V
9 5 V
1062 1192 M
10 4 V
9 5 V
9 4 V
10 5 V
9 4 V
10 4 V
9 5 V
10 4 V
9 5 V
10 4 V
9 4 V
9 5 V
10 4 V
9 4 V
10 5 V
9 4 V
10 5 V
9 4 V
10 4 V
9 5 V
9 4 V
10 5 V
9 4 V
10 4 V
9 5 V
10 4 V
9 5 V
10 4 V
9 4 V
9 5 V
10 4 V
9 5 V
10 4 V
9 4 V
10 5 V
9 4 V
10 4 V
9 5 V
9 4 V
10 5 V
9 4 V
10 4 V
9 5 V
10 4 V
9 5 V
10 4 V
9 4 V
9 5 V
10 4 V
9 5 V
10 4 V
9 4 V
10 5 V
9 4 V
10 5 V
9 4 V
9 4 V
10 5 V
9 4 V
10 4 V
9 5 V
10 4 V
9 5 V
10 4 V
9 4 V
9 5 V
10 4 V
9 5 V
10 4 V
9 4 V
10 5 V
9 4 V
10 5 V
9 4 V
9 4 V
10 5 V
9 4 V
10 5 V
9 4 V
10 4 V
9 5 V
10 4 V
9 4 V
9 5 V
10 4 V
9 5 V
10 4 V
9 4 V
10 5 V
9 4 V
10 5 V
9 4 V
9 4 V
10 5 V
9 4 V
10 5 V
9 4 V
10 4 V
9 5 V
1242 1005 M
9 5 V
10 4 V
9 4 V
10 5 V
9 4 V
10 5 V
9 4 V
10 4 V
9 5 V
9 4 V
10 5 V
9 4 V
10 4 V
9 5 V
10 4 V
9 5 V
10 4 V
9 4 V
9 5 V
10 4 V
9 4 V
10 5 V
9 4 V
10 5 V
9 4 V
10 4 V
9 5 V
9 4 V
10 5 V
9 4 V
10 4 V
9 5 V
10 4 V
9 5 V
10 4 V
9 4 V
9 5 V
10 4 V
9 5 V
10 4 V
9 4 V
10 5 V
9 4 V
10 4 V
9 5 V
9 4 V
10 5 V
9 4 V
10 4 V
9 5 V
10 4 V
9 5 V
10 4 V
9 4 V
9 5 V
10 4 V
9 5 V
10 4 V
9 4 V
10 5 V
9 4 V
10 5 V
9 4 V
9 4 V
10 5 V
9 4 V
10 4 V
9 5 V
10 4 V
9 5 V
10 4 V
9 4 V
9 5 V
10 4 V
9 5 V
10 4 V
9 4 V
10 5 V
9 4 V
10 5 V
9 4 V
9 4 V
10 5 V
9 4 V
10 5 V
9 4 V
10 4 V
9 5 V
10 4 V
9 4 V
9 5 V
10 4 V
9 5 V
10 4 V
9 4 V
10 5 V
9 4 V
10 5 V
9 4 V
1422 869 M
9 4 V
10 4 V
9 5 V
10 4 V
9 4 V
10 5 V
9 4 V
10 5 V
9 4 V
9 4 V
10 5 V
9 4 V
10 5 V
9 4 V
10 4 V
9 5 V
10 4 V
9 5 V
9 4 V
10 4 V
9 5 V
10 4 V
9 5 V
10 4 V
9 4 V
10 5 V
9 4 V
9 4 V
10 5 V
9 4 V
10 5 V
9 4 V
10 4 V
9 5 V
10 4 V
9 5 V
9 4 V
10 4 V
9 5 V
10 4 V
9 5 V
10 4 V
9 4 V
10 5 V
9 4 V
9 5 V
10 4 V
9 4 V
10 5 V
9 4 V
10 5 V
9 4 V
10 4 V
9 5 V
9 4 V
10 4 V
9 5 V
10 4 V
9 5 V
10 4 V
9 4 V
10 5 V
9 4 V
9 5 V
10 4 V
9 4 V
10 5 V
9 4 V
10 5 V
9 4 V
10 4 V
9 5 V
9 4 V
10 5 V
9 4 V
10 4 V
9 5 V
10 4 V
9 4 V
10 5 V
9 4 V
9 5 V
10 4 V
9 4 V
10 5 V
9 4 V
10 5 V
9 4 V
10 4 V
9 5 V
9 4 V
10 5 V
9 4 V
10 4 V
9 5 V
10 4 V
9 5 V
currentpoint stroke M
10 4 V
9 4 V
1602 780 M
9 4 V
10 5 V
9 4 V
10 5 V
9 4 V
10 4 V
9 5 V
9 4 V
10 5 V
9 4 V
10 4 V
9 5 V
10 4 V
9 5 V
10 4 V
9 4 V
9 5 V
10 4 V
9 4 V
10 5 V
9 4 V
10 5 V
9 4 V
10 4 V
9 5 V
9 4 V
10 5 V
9 4 V
10 4 V
9 5 V
10 4 V
9 5 V
10 4 V
9 4 V
9 5 V
10 4 V
9 5 V
10 4 V
9 4 V
10 5 V
9 4 V
10 4 V
9 5 V
9 4 V
10 5 V
9 4 V
10 4 V
9 5 V
10 4 V
9 5 V
10 4 V
9 4 V
9 5 V
10 4 V
9 5 V
10 4 V
9 4 V
10 5 V
9 4 V
10 5 V
9 4 V
9 4 V
10 5 V
9 4 V
10 4 V
9 5 V
10 4 V
9 5 V
10 4 V
9 4 V
9 5 V
10 4 V
9 5 V
10 4 V
9 4 V
10 5 V
9 4 V
10 5 V
9 4 V
9 4 V
10 5 V
9 4 V
10 5 V
9 4 V
10 4 V
9 5 V
10 4 V
9 4 V
9 5 V
10 4 V
9 5 V
10 4 V
9 4 V
10 5 V
9 4 V
10 5 V
9 4 V
9 4 V
10 5 V
1782 722 M
9 4 V
10 5 V
9 4 V
10 5 V
9 4 V
10 4 V
9 5 V
9 4 V
10 4 V
9 5 V
10 4 V
9 5 V
10 4 V
9 4 V
10 5 V
9 4 V
9 5 V
10 4 V
9 4 V
10 5 V
9 4 V
10 5 V
9 4 V
10 4 V
9 5 V
9 4 V
10 5 V
9 4 V
10 4 V
9 5 V
10 4 V
9 4 V
10 5 V
9 4 V
9 5 V
10 4 V
9 4 V
10 5 V
9 4 V
10 5 V
9 4 V
10 4 V
9 5 V
9 4 V
10 5 V
9 4 V
10 4 V
9 5 V
10 4 V
9 5 V
10 4 V
9 4 V
9 5 V
10 4 V
9 4 V
10 5 V
9 4 V
10 5 V
9 4 V
10 4 V
9 5 V
9 4 V
10 5 V
9 4 V
10 4 V
9 5 V
10 4 V
9 5 V
10 4 V
9 4 V
9 5 V
10 4 V
9 5 V
10 4 V
9 4 V
10 5 V
9 4 V
10 5 V
9 4 V
9 4 V
10 5 V
9 4 V
10 4 V
9 5 V
10 4 V
9 5 V
10 4 V
9 4 V
9 5 V
10 4 V
9 5 V
10 4 V
9 4 V
10 5 V
9 4 V
10 5 V
9 4 V
9 4 V
10 5 V
1962 680 M
9 5 V
10 4 V
9 4 V
10 5 V
9 4 V
9 4 V
10 5 V
9 4 V
10 5 V
9 4 V
10 4 V
9 5 V
10 4 V
9 5 V
9 4 V
10 4 V
9 5 V
10 4 V
9 5 V
10 4 V
9 4 V
10 5 V
9 4 V
9 5 V
10 4 V
9 4 V
10 5 V
9 4 V
10 4 V
9 5 V
10 4 V
9 5 V
9 4 V
10 4 V
9 5 V
10 4 V
9 5 V
10 4 V
9 4 V
10 5 V
9 4 V
9 5 V
10 4 V
9 4 V
10 5 V
9 4 V
10 5 V
9 4 V
10 4 V
9 5 V
9 4 V
10 4 V
9 5 V
10 4 V
9 5 V
10 4 V
9 4 V
10 5 V
9 4 V
9 5 V
10 4 V
9 4 V
10 5 V
9 4 V
10 5 V
9 4 V
10 4 V
9 5 V
9 4 V
10 5 V
9 4 V
10 4 V
9 5 V
10 4 V
9 4 V
10 5 V
9 4 V
9 5 V
10 4 V
9 4 V
10 5 V
9 4 V
10 5 V
9 4 V
10 4 V
9 5 V
9 4 V
10 5 V
9 4 V
10 4 V
9 5 V
10 4 V
9 5 V
10 4 V
9 4 V
9 5 V
10 4 V
9 4 V
10 5 V
2142 646 M
9 4 V
10 5 V
9 4 V
10 5 V
9 4 V
9 4 V
10 5 V
9 4 V
10 5 V
9 4 V
10 4 V
9 5 V
10 4 V
9 5 V
9 4 V
10 4 V
9 5 V
10 4 V
9 5 V
10 4 V
9 4 V
10 5 V
9 4 V
9 4 V
10 5 V
9 4 V
10 5 V
9 4 V
10 4 V
9 5 V
10 4 V
9 5 V
9 4 V
10 4 V
9 5 V
10 4 V
9 5 V
10 4 V
9 4 V
10 5 V
9 4 V
9 5 V
10 4 V
9 4 V
10 5 V
9 4 V
10 4 V
9 5 V
10 4 V
9 5 V
9 4 V
10 4 V
9 5 V
10 4 V
9 5 V
10 4 V
9 4 V
10 5 V
9 4 V
9 5 V
10 4 V
9 4 V
10 5 V
9 4 V
10 5 V
9 4 V
10 4 V
9 5 V
9 4 V
10 4 V
9 5 V
10 4 V
9 5 V
10 4 V
9 4 V
10 5 V
9 4 V
9 5 V
10 4 V
9 4 V
10 5 V
9 4 V
10 5 V
9 4 V
10 4 V
9 5 V
9 4 V
10 5 V
9 4 V
10 4 V
9 5 V
10 4 V
9 4 V
10 5 V
9 4 V
9 5 V
10 4 V
currentpoint stroke M
9 4 V
10 5 V
1.000 UL
LTb
3077 787 M
2142 352 L
522 603 M
2142 352 L
522 603 M
935 435 V
3077 787 M
1457 1038 L
522 603 M
0 869 V
935 -434 R
0 289 V
3077 787 M
0 294 V
2142 352 M
0 294 V
522 603 M
57 26 V
878 409 R
-58 -27 V
846 553 M
57 26 V
878 408 R
-58 -27 V
1170 502 M
57 26 V
878 409 R
-58 -27 V
1494 452 M
57 26 V
878 409 R
-58 -27 V
1818 402 M
57 26 V
878 409 R
-58 -27 V
2142 352 M
57 26 V
878 409 R
-58 -27 V
2142 352 M
-63 9 V
522 603 M
62 -10 V
2291 421 M
-63 9 V
671 672 M
62 -10 V
2439 490 M
-63 9 V
820 741 M
62 -10 V
2588 559 M
-63 9 V
969 810 M
62 -10 V
2737 629 M
-63 9 V
1118 880 M
62 -10 V
2886 698 M
-63 9 V
1266 949 M
62 -10 V
3035 767 M
-63 9 V
1415 1018 M
62 -10 V
522 893 M
63 0 V
-63 96 R
63 0 V
-63 97 R
63 0 V
-63 97 R
63 0 V
-63 96 R
63 0 V
-63 97 R
63 0 V
-63 96 R
63 0 V
stroke
grestore
end
showpage
}}%
\put(522,1690){\makebox(0,0){$a_0P(r,	\theta)$}}%
\put(396,1472){\makebox(0,0)[r]{0.6}}%
\put(396,1376){\makebox(0,0)[r]{0.5}}%
\put(396,1279){\makebox(0,0)[r]{0.4}}%
\put(396,1183){\makebox(0,0)[r]{0.3}}%
\put(396,1086){\makebox(0,0)[r]{0.2}}%
\put(396,989){\makebox(0,0)[r]{0.1}}%
\put(396,893){\makebox(0,0)[r]{0}}%
\put(3014,507){\makebox(0,0){$	\theta$}}%
\put(3084,751){\makebox(0,0)[l]{3}}%
\put(2935,682){\makebox(0,0)[l]{2.5}}%
\put(2786,613){\makebox(0,0)[l]{2}}%
\put(2637,543){\makebox(0,0)[l]{1.5}}%
\put(2488,474){\makebox(0,0)[l]{1}}%
\put(2340,405){\makebox(0,0)[l]{0.5}}%
\put(2191,336){\makebox(0,0)[l]{0}}%
\put(1098,369){\makebox(0,0){$r/a_0$}}%
\put(2096,309){\makebox(0,0)[r]{5}}%
\put(1772,359){\makebox(0,0)[r]{4}}%
\put(1448,409){\makebox(0,0)[r]{3}}%
\put(1124,459){\makebox(0,0)[r]{2}}%
\put(800,510){\makebox(0,0)[r]{1}}%
\put(476,560){\makebox(0,0)[r]{0}}%
\put(3074,1796){\makebox(0,0)[r]{$n=1$ $l=0$ $m_l=0$}}%
\end{picture}%
\endgroup
\endinput

\caption{Romlig fordeling av $a_0P(r,\theta)$ for grunntilstande i hydrogenatomet.\label{45}}
\end{center}
\end{figure}
Med sannsynlighetstettheten kan vi rekne ut midlere radius gitt ved
\be
   \langle r_{nl}\rangle =\int_0^{\infty} rP_{nlm_l}(r,\theta,\phi)d\tau=\int_0^{\infty} rP_{nl}dr,
\ee
som gir
\be
   \langle r_{nl}\rangle=\frac{a_0}{2}\left(3n^2-l(l+1)\right).
\ee
Vi kan ogs\aa\ vise at 
\be
   \langle r_{nl}^2\rangle =\int_0^{\infty} r^2P_{nl}dr,
\ee   
kan skrives 
\be
   \langle r_{nl}^2\rangle=\frac{a_0n^2}{2}\left(5n^2+1-3l(l+1)\right).
\ee
Vi kan dermed finne standardavviket
\be
   \Delta r_{nl}=\sqrt{\langle r_{nl}^2\rangle-\langle r_{nl}\rangle^2},
\ee
som ved innsetting gir
\be
   \Delta r_{nl}=\frac{a_0}{2}\sqrt{n^4+2n^2-l^2(l+1)^2}.
\ee
Utifra Heisenbergs uskarphetsrelasjon har vi dermed
\be
   \Delta p_{nl}\ge \frac{\hbar}{2\Delta r_{nl}}=\frac{\hbar}{a_0\sqrt{n^4+2n^2-l^2(l+1)^2}}.
\ee

Utifra disse resultatene kan vi n\aa\ pr\o ve \aa\ tolke formen til egenfunksjonene i figurene \ref{43} og \ref{44} utifra uskarphetsrelasjonen. 

La oss se p\aa\ figur \ref{43}, da den samme diskusjon gjelder for figur \ref{44}.
For $n=1$ og $l=0$ har vi en uskarphet i $r$ og $p$ gitt ved henholdsvis
\be
   \Delta r_{10}=\frac{a_0}{2}\sqrt{3},
\ee 
og
\be
   \Delta p_{10}\ge \frac{\hbar}{a_0\sqrt{3}}.
\ee
Ser vi p\aa\ tilfellet med $n=3$ men fremdeles $l=0$, finner vi
\be
   \Delta r_{30}=\frac{3a_0}{2}\sqrt{10},
\ee 
og
\be
   \Delta p_{30}\ge \frac{\hbar}{3a_0\sqrt{10}}.
\ee

Legg merke til f\o lgende. N\aa r vi \o ker $n$, \o ker $\Delta r$
mens $\Delta p$ minsker. At $\Delta r$ \o ker ser vi p\aa\ utstrekningen 
til b\o lgefunksjonen. For $n=1$ strekker den seg ut til ca.~$5-6a_0$,
mens for $n=3$ strekker den seg ut til ca.~$25a_0$.

Hvorfor minsker bevegelsesmengden? For $n=1$ er energien $E_0=-13.6$ eV.
For $n=3$ er energien $E_0/9$. N\aa r energien minsker, minsker 
ogs\aa\ uskarpheten i bevegelsesmengde. 
N\aa r uskarpheten   i bevegelsesmengde minsker, m\aa\ uskarpheten
i posisjon \o ke. Utifra Heisenbergs uskarphetsrelasjon.

Utstrekningen av b\o lgefunksjonen er dermed n\ae rt knytta til Heisenbergs
uskarphetsrelasjon.

Tilsvarende argument gjelder for elektronet med banespinn $l=1$ og ulike
$n$ verdier, eller andre banespinn.

Legg ogs\aa\ merke til det faktum at i figur \ref{43} (\ref{44}) ser vi at for
tilstander som har mindre energi (i absoluttverdi) enn den f\o rste,
har vi nullpunkter for sannsynlightestettheten. Det skyldes at egenfunksjonen
har noder, dvs.~nullpunkter. Dette s\aa\ vi ogs\aa\ for de eksiterte
tilstandene til den en-dimensjonale potensialbr\o nnen i avsnitt 
\ref{subsec:342}.
Klassisk finner vi det samme for en streng som svinger og er fastspent
i begge ender. Desto st\o rre energi strengen har, desto flere
noder har b\o lgefunksjonen. For bundne tilstander i kvantemekanikk
har vi det samme, desto h\o yere vi g\aa r opp i eksitasjonsenergi,
desto flere noder utviser b\o lgefunksjonen. 

La oss \aa\ se p\aa\ sannsynlightesfordelingen for elektronet i 
grunntilstanden fra figur \ref{43} (tilfellet med $n=1$ og $l=0$.
Vi s\aa\ ovenfor at midlere radius var gitt ved  
 $\langle r_{10}\rangle =3a_0/2$ med en fluktuasjon
$\Delta r_{10}=a_0\sqrt{3}/2$. I midddel befinner alts\aa\
elektronet seg lenger borte fra protonet enn hva vi finner
fra Bohrs atommodell. Fra sistnevnte modell, g\aa r elektronet i
en bane med avstand $a_0$ fra protonet.

Men dersom vi ser p\aa\ sannsynlighetstetthen for $n=1$ og $l=0$ finner
vi at den maksimale verdier avstanden fra protonet kan ha er gitt ved
\be
    \frac{dP_{10}(r)}{dr}=0=2re^{-2r/a_0}-\frac{2}{a_0}r^2e^{-2r/a_0}
\ee
som gir 
\be
   r_{\mathrm{maks}}=a_0.
\ee
Formlen for Bohrradien er gitt ved $a_0n^2$.
Deriverer vi sannsynlighetstettheten for $P_{21}$, dvs.~$n=2$ 
og $l=1$, se figur \ref{44}, finner vi igjen samsvar med
Bohrs atommodell. Men den midlere radius for 
$\langle r_{21}\rangle =5a_0$.

Den kvantemekaniske tolkning av egenfunksjonen som vha.~begrepet
sannsynlighetstetthet, leder oss til midlere verdier som er
forskjellige fra den enkle Bohrmodellen, som baserte seg p\aa\
et 'planetbilde', dvs.~elektronet i bane rundt en kjerne.

Kvantemekanikken gitt ved Schr\"odingers likning  gir oss ikke bare energien og kvantisering
av spinn, men ogs\aa\ muligheten til \aa\ rekne ut andre st\o rrelser.

Som bevegelseslov (naturlov) har dermed Schr\"odingers likning  muligheten til
\aa\ si mer om et system enn mer enkle modeller.

Vi kan f.eks.~bruke egenfunskjonene til \aa\ rekne ut midlere kinetisk 
energi for en gitt tilstand $nl$ gitt ved
\be
  \langle \frac{p^2}{2m}\rangle_{nl} =-\frac{\hbar^2}{2m}\int_0^{\infty}
  \int_0^{\pi}\int_0^{2\pi}\psi_{nlm_l}^*(r,\theta,\phi)\nabla^2 \psi_{nlm_l}(r,\theta,\phi)  r^2sin(\theta)drd\theta d\phi,
\ee
eller en forventningsverdi for en vilk\aa rlig operator $\OP{A}$
gitt ved
\be
  \langle \OP{A}\rangle_{nl} =\int_0^{\infty}
  \int_0^{\pi}\int_0^{2\pi}\psi_{nlm_l}^*\OP{A} 
   \psi_{nlm_l}r^2sin(\theta)drd\theta d\phi.
\ee

Vi kan ogs\aa\, n\aa r vi f.eks.~skal se p\aa\ elektriske dipoloverganger
i neste kapittel (avsnitt \ref{sec:dipolregler}, 
studere overgangssannsyligheten fra en tilstand
$\psi_{n'l'}$ til en tilstand gitt ved $\psi_{nl}$. Denne overgangen
kan vi skrive som
\be
  \langle \OP{A}\rangle_{n'l'\rightarrow nl} =\int_0^{\infty}
  \int_0^{\pi}\int_0^{2\pi}\psi_{nlm_l}^*\OP{A} 
  \psi_{n'l'm_l'}r^2sin(\theta)drd\theta d\phi.
\ee

Slike beregninger var ikke mulig innafor ramma av Bohrs atommodell.
Vi f\aa r litt igjen med alt strevet med \aa\ sette opp Schr\"odingers likning .

Vi avslutter dette avsnittet med figurer om den romlige fordelingen
av sannsynlighetstettheten $P$. Siden avhengigheten av vinkelen
$\phi$ forsvinner n\aa r vi rekner ut $|\psi|^2$, trenger vi kun \aa\
plotte $|\psi|^2$ som funksjon av $r$ og $\theta$. 
Plottet av  $|\psi|^2$ for grunntilstanden er gitt i figur \ref{45}. 
Siden vinkelen $\phi$ representerer projeksjonen av $r$ i $xy$-planet
kan dere tenke dere figur \ref{45} som et snitt av en tre-dimensjonal
funksjon i $xz$-planet. Denne figuren vil, siden $|\psi|^2$ ikke avhenger
av hvor vi er $xy$-planet, v\ae re den samme for alle $\phi$-verdier. 
\begin{figure}[h]
\begin{center}
% GNUPLOT: LaTeX picture with Postscript
\begingroup%
  \makeatletter%
  \newcommand{\GNUPLOTspecial}{%
    \@sanitize\catcode`\%=14\relax\special}%
  \setlength{\unitlength}{0.1bp}%
{\GNUPLOTspecial{!
%!PS-Adobe-2.0 EPSF-2.0
%%Title: wave4.tex
%%Creator: gnuplot 3.7 patchlevel 0.2
%%CreationDate: Fri Mar 17 10:50:58 2000
%%DocumentFonts: 
%%BoundingBox: 0 0 360 216
%%Orientation: Landscape
%%EndComments
/gnudict 256 dict def
gnudict begin
/Color false def
/Solid false def
/gnulinewidth 5.000 def
/userlinewidth gnulinewidth def
/vshift -33 def
/dl {10 mul} def
/hpt_ 31.5 def
/vpt_ 31.5 def
/hpt hpt_ def
/vpt vpt_ def
/M {moveto} bind def
/L {lineto} bind def
/R {rmoveto} bind def
/V {rlineto} bind def
/vpt2 vpt 2 mul def
/hpt2 hpt 2 mul def
/Lshow { currentpoint stroke M
  0 vshift R show } def
/Rshow { currentpoint stroke M
  dup stringwidth pop neg vshift R show } def
/Cshow { currentpoint stroke M
  dup stringwidth pop -2 div vshift R show } def
/UP { dup vpt_ mul /vpt exch def hpt_ mul /hpt exch def
  /hpt2 hpt 2 mul def /vpt2 vpt 2 mul def } def
/DL { Color {setrgbcolor Solid {pop []} if 0 setdash }
 {pop pop pop Solid {pop []} if 0 setdash} ifelse } def
/BL { stroke userlinewidth 2 mul setlinewidth } def
/AL { stroke userlinewidth 2 div setlinewidth } def
/UL { dup gnulinewidth mul /userlinewidth exch def
      10 mul /udl exch def } def
/PL { stroke userlinewidth setlinewidth } def
/LTb { BL [] 0 0 0 DL } def
/LTa { AL [1 udl mul 2 udl mul] 0 setdash 0 0 0 setrgbcolor } def
/LT0 { PL [] 1 0 0 DL } def
/LT1 { PL [4 dl 2 dl] 0 1 0 DL } def
/LT2 { PL [2 dl 3 dl] 0 0 1 DL } def
/LT3 { PL [1 dl 1.5 dl] 1 0 1 DL } def
/LT4 { PL [5 dl 2 dl 1 dl 2 dl] 0 1 1 DL } def
/LT5 { PL [4 dl 3 dl 1 dl 3 dl] 1 1 0 DL } def
/LT6 { PL [2 dl 2 dl 2 dl 4 dl] 0 0 0 DL } def
/LT7 { PL [2 dl 2 dl 2 dl 2 dl 2 dl 4 dl] 1 0.3 0 DL } def
/LT8 { PL [2 dl 2 dl 2 dl 2 dl 2 dl 2 dl 2 dl 4 dl] 0.5 0.5 0.5 DL } def
/Pnt { stroke [] 0 setdash
   gsave 1 setlinecap M 0 0 V stroke grestore } def
/Dia { stroke [] 0 setdash 2 copy vpt add M
  hpt neg vpt neg V hpt vpt neg V
  hpt vpt V hpt neg vpt V closepath stroke
  Pnt } def
/Pls { stroke [] 0 setdash vpt sub M 0 vpt2 V
  currentpoint stroke M
  hpt neg vpt neg R hpt2 0 V stroke
  } def
/Box { stroke [] 0 setdash 2 copy exch hpt sub exch vpt add M
  0 vpt2 neg V hpt2 0 V 0 vpt2 V
  hpt2 neg 0 V closepath stroke
  Pnt } def
/Crs { stroke [] 0 setdash exch hpt sub exch vpt add M
  hpt2 vpt2 neg V currentpoint stroke M
  hpt2 neg 0 R hpt2 vpt2 V stroke } def
/TriU { stroke [] 0 setdash 2 copy vpt 1.12 mul add M
  hpt neg vpt -1.62 mul V
  hpt 2 mul 0 V
  hpt neg vpt 1.62 mul V closepath stroke
  Pnt  } def
/Star { 2 copy Pls Crs } def
/BoxF { stroke [] 0 setdash exch hpt sub exch vpt add M
  0 vpt2 neg V  hpt2 0 V  0 vpt2 V
  hpt2 neg 0 V  closepath fill } def
/TriUF { stroke [] 0 setdash vpt 1.12 mul add M
  hpt neg vpt -1.62 mul V
  hpt 2 mul 0 V
  hpt neg vpt 1.62 mul V closepath fill } def
/TriD { stroke [] 0 setdash 2 copy vpt 1.12 mul sub M
  hpt neg vpt 1.62 mul V
  hpt 2 mul 0 V
  hpt neg vpt -1.62 mul V closepath stroke
  Pnt  } def
/TriDF { stroke [] 0 setdash vpt 1.12 mul sub M
  hpt neg vpt 1.62 mul V
  hpt 2 mul 0 V
  hpt neg vpt -1.62 mul V closepath fill} def
/DiaF { stroke [] 0 setdash vpt add M
  hpt neg vpt neg V hpt vpt neg V
  hpt vpt V hpt neg vpt V closepath fill } def
/Pent { stroke [] 0 setdash 2 copy gsave
  translate 0 hpt M 4 {72 rotate 0 hpt L} repeat
  closepath stroke grestore Pnt } def
/PentF { stroke [] 0 setdash gsave
  translate 0 hpt M 4 {72 rotate 0 hpt L} repeat
  closepath fill grestore } def
/Circle { stroke [] 0 setdash 2 copy
  hpt 0 360 arc stroke Pnt } def
/CircleF { stroke [] 0 setdash hpt 0 360 arc fill } def
/C0 { BL [] 0 setdash 2 copy moveto vpt 90 450  arc } bind def
/C1 { BL [] 0 setdash 2 copy        moveto
       2 copy  vpt 0 90 arc closepath fill
               vpt 0 360 arc closepath } bind def
/C2 { BL [] 0 setdash 2 copy moveto
       2 copy  vpt 90 180 arc closepath fill
               vpt 0 360 arc closepath } bind def
/C3 { BL [] 0 setdash 2 copy moveto
       2 copy  vpt 0 180 arc closepath fill
               vpt 0 360 arc closepath } bind def
/C4 { BL [] 0 setdash 2 copy moveto
       2 copy  vpt 180 270 arc closepath fill
               vpt 0 360 arc closepath } bind def
/C5 { BL [] 0 setdash 2 copy moveto
       2 copy  vpt 0 90 arc
       2 copy moveto
       2 copy  vpt 180 270 arc closepath fill
               vpt 0 360 arc } bind def
/C6 { BL [] 0 setdash 2 copy moveto
      2 copy  vpt 90 270 arc closepath fill
              vpt 0 360 arc closepath } bind def
/C7 { BL [] 0 setdash 2 copy moveto
      2 copy  vpt 0 270 arc closepath fill
              vpt 0 360 arc closepath } bind def
/C8 { BL [] 0 setdash 2 copy moveto
      2 copy vpt 270 360 arc closepath fill
              vpt 0 360 arc closepath } bind def
/C9 { BL [] 0 setdash 2 copy moveto
      2 copy  vpt 270 450 arc closepath fill
              vpt 0 360 arc closepath } bind def
/C10 { BL [] 0 setdash 2 copy 2 copy moveto vpt 270 360 arc closepath fill
       2 copy moveto
       2 copy vpt 90 180 arc closepath fill
               vpt 0 360 arc closepath } bind def
/C11 { BL [] 0 setdash 2 copy moveto
       2 copy  vpt 0 180 arc closepath fill
       2 copy moveto
       2 copy  vpt 270 360 arc closepath fill
               vpt 0 360 arc closepath } bind def
/C12 { BL [] 0 setdash 2 copy moveto
       2 copy  vpt 180 360 arc closepath fill
               vpt 0 360 arc closepath } bind def
/C13 { BL [] 0 setdash  2 copy moveto
       2 copy  vpt 0 90 arc closepath fill
       2 copy moveto
       2 copy  vpt 180 360 arc closepath fill
               vpt 0 360 arc closepath } bind def
/C14 { BL [] 0 setdash 2 copy moveto
       2 copy  vpt 90 360 arc closepath fill
               vpt 0 360 arc } bind def
/C15 { BL [] 0 setdash 2 copy vpt 0 360 arc closepath fill
               vpt 0 360 arc closepath } bind def
/Rec   { newpath 4 2 roll moveto 1 index 0 rlineto 0 exch rlineto
       neg 0 rlineto closepath } bind def
/Square { dup Rec } bind def
/Bsquare { vpt sub exch vpt sub exch vpt2 Square } bind def
/S0 { BL [] 0 setdash 2 copy moveto 0 vpt rlineto BL Bsquare } bind def
/S1 { BL [] 0 setdash 2 copy vpt Square fill Bsquare } bind def
/S2 { BL [] 0 setdash 2 copy exch vpt sub exch vpt Square fill Bsquare } bind def
/S3 { BL [] 0 setdash 2 copy exch vpt sub exch vpt2 vpt Rec fill Bsquare } bind def
/S4 { BL [] 0 setdash 2 copy exch vpt sub exch vpt sub vpt Square fill Bsquare } bind def
/S5 { BL [] 0 setdash 2 copy 2 copy vpt Square fill
       exch vpt sub exch vpt sub vpt Square fill Bsquare } bind def
/S6 { BL [] 0 setdash 2 copy exch vpt sub exch vpt sub vpt vpt2 Rec fill Bsquare } bind def
/S7 { BL [] 0 setdash 2 copy exch vpt sub exch vpt sub vpt vpt2 Rec fill
       2 copy vpt Square fill
       Bsquare } bind def
/S8 { BL [] 0 setdash 2 copy vpt sub vpt Square fill Bsquare } bind def
/S9 { BL [] 0 setdash 2 copy vpt sub vpt vpt2 Rec fill Bsquare } bind def
/S10 { BL [] 0 setdash 2 copy vpt sub vpt Square fill 2 copy exch vpt sub exch vpt Square fill
       Bsquare } bind def
/S11 { BL [] 0 setdash 2 copy vpt sub vpt Square fill 2 copy exch vpt sub exch vpt2 vpt Rec fill
       Bsquare } bind def
/S12 { BL [] 0 setdash 2 copy exch vpt sub exch vpt sub vpt2 vpt Rec fill Bsquare } bind def
/S13 { BL [] 0 setdash 2 copy exch vpt sub exch vpt sub vpt2 vpt Rec fill
       2 copy vpt Square fill Bsquare } bind def
/S14 { BL [] 0 setdash 2 copy exch vpt sub exch vpt sub vpt2 vpt Rec fill
       2 copy exch vpt sub exch vpt Square fill Bsquare } bind def
/S15 { BL [] 0 setdash 2 copy Bsquare fill Bsquare } bind def
/D0 { gsave translate 45 rotate 0 0 S0 stroke grestore } bind def
/D1 { gsave translate 45 rotate 0 0 S1 stroke grestore } bind def
/D2 { gsave translate 45 rotate 0 0 S2 stroke grestore } bind def
/D3 { gsave translate 45 rotate 0 0 S3 stroke grestore } bind def
/D4 { gsave translate 45 rotate 0 0 S4 stroke grestore } bind def
/D5 { gsave translate 45 rotate 0 0 S5 stroke grestore } bind def
/D6 { gsave translate 45 rotate 0 0 S6 stroke grestore } bind def
/D7 { gsave translate 45 rotate 0 0 S7 stroke grestore } bind def
/D8 { gsave translate 45 rotate 0 0 S8 stroke grestore } bind def
/D9 { gsave translate 45 rotate 0 0 S9 stroke grestore } bind def
/D10 { gsave translate 45 rotate 0 0 S10 stroke grestore } bind def
/D11 { gsave translate 45 rotate 0 0 S11 stroke grestore } bind def
/D12 { gsave translate 45 rotate 0 0 S12 stroke grestore } bind def
/D13 { gsave translate 45 rotate 0 0 S13 stroke grestore } bind def
/D14 { gsave translate 45 rotate 0 0 S14 stroke grestore } bind def
/D15 { gsave translate 45 rotate 0 0 S15 stroke grestore } bind def
/DiaE { stroke [] 0 setdash vpt add M
  hpt neg vpt neg V hpt vpt neg V
  hpt vpt V hpt neg vpt V closepath stroke } def
/BoxE { stroke [] 0 setdash exch hpt sub exch vpt add M
  0 vpt2 neg V hpt2 0 V 0 vpt2 V
  hpt2 neg 0 V closepath stroke } def
/TriUE { stroke [] 0 setdash vpt 1.12 mul add M
  hpt neg vpt -1.62 mul V
  hpt 2 mul 0 V
  hpt neg vpt 1.62 mul V closepath stroke } def
/TriDE { stroke [] 0 setdash vpt 1.12 mul sub M
  hpt neg vpt 1.62 mul V
  hpt 2 mul 0 V
  hpt neg vpt -1.62 mul V closepath stroke } def
/PentE { stroke [] 0 setdash gsave
  translate 0 hpt M 4 {72 rotate 0 hpt L} repeat
  closepath stroke grestore } def
/CircE { stroke [] 0 setdash 
  hpt 0 360 arc stroke } def
/Opaque { gsave closepath 1 setgray fill grestore 0 setgray closepath } def
/DiaW { stroke [] 0 setdash vpt add M
  hpt neg vpt neg V hpt vpt neg V
  hpt vpt V hpt neg vpt V Opaque stroke } def
/BoxW { stroke [] 0 setdash exch hpt sub exch vpt add M
  0 vpt2 neg V hpt2 0 V 0 vpt2 V
  hpt2 neg 0 V Opaque stroke } def
/TriUW { stroke [] 0 setdash vpt 1.12 mul add M
  hpt neg vpt -1.62 mul V
  hpt 2 mul 0 V
  hpt neg vpt 1.62 mul V Opaque stroke } def
/TriDW { stroke [] 0 setdash vpt 1.12 mul sub M
  hpt neg vpt 1.62 mul V
  hpt 2 mul 0 V
  hpt neg vpt -1.62 mul V Opaque stroke } def
/PentW { stroke [] 0 setdash gsave
  translate 0 hpt M 4 {72 rotate 0 hpt L} repeat
  Opaque stroke grestore } def
/CircW { stroke [] 0 setdash 
  hpt 0 360 arc Opaque stroke } def
/BoxFill { gsave Rec 1 setgray fill grestore } def
end
%%EndProlog
}}%
\begin{picture}(3600,2160)(0,0)%
{\GNUPLOTspecial{"
gnudict begin
gsave
0 0 translate
0.100 0.100 scale
0 setgray
newpath
1.000 UL
LTb
1.000 UL
LT0
3124 1796 M
263 0 V
522 893 M
17 1 V
16 7 V
16 8 V
17 7 V
16 6 V
16 3 V
17 0 V
16 -3 V
16 -4 V
17 -7 V
16 -7 V
17 -9 V
16 -9 V
16 -9 V
17 -8 V
16 -8 V
16 -6 V
17 -6 V
16 -4 V
16 -3 V
17 -2 V
16 0 V
17 1 V
16 2 V
16 3 V
17 4 V
16 5 V
16 6 V
17 6 V
16 7 V
16 6 V
17 7 V
16 7 V
16 7 V
17 7 V
16 6 V
17 6 V
16 6 V
16 5 V
17 5 V
16 4 V
16 3 V
17 3 V
16 3 V
16 1 V
17 1 V
16 1 V
16 0 V
17 -1 V
16 -1 V
17 -2 V
16 -3 V
16 -2 V
17 -4 V
16 -4 V
16 -4 V
17 -4 V
16 -5 V
16 -5 V
17 -6 V
16 -5 V
16 -6 V
17 -7 V
16 -6 V
17 -6 V
16 -7 V
16 -7 V
17 -6 V
16 -7 V
16 -7 V
17 -7 V
16 -7 V
16 -6 V
17 -7 V
16 -7 V
16 -7 V
17 -6 V
16 -7 V
17 -6 V
16 -7 V
16 -6 V
17 -7 V
16 -6 V
16 -6 V
17 -6 V
16 -6 V
16 -5 V
17 -6 V
16 -5 V
17 -6 V
16 -5 V
16 -5 V
17 -6 V
16 -5 V
16 -5 V
17 -4 V
16 -5 V
16 -5 V
17 -4 V
626 941 M
17 1 V
16 7 V
16 8 V
17 7 V
16 6 V
16 3 V
17 0 V
16 -2 V
16 -5 V
17 -6 V
16 -8 V
16 -9 V
17 -9 V
16 -8 V
17 -9 V
16 -7 V
16 -7 V
17 -5 V
16 -5 V
16 -3 V
17 -1 V
16 -1 V
16 1 V
17 2 V
16 4 V
16 4 V
17 5 V
16 5 V
17 6 V
16 7 V
16 7 V
17 7 V
16 6 V
16 7 V
17 7 V
16 6 V
16 6 V
17 6 V
16 5 V
16 5 V
17 4 V
16 4 V
17 3 V
16 2 V
16 2 V
17 1 V
16 0 V
16 0 V
17 -1 V
16 -1 V
16 -2 V
17 -2 V
16 -3 V
17 -4 V
16 -3 V
16 -4 V
17 -5 V
16 -5 V
16 -5 V
17 -5 V
16 -6 V
16 -6 V
17 -6 V
16 -7 V
16 -6 V
17 -7 V
16 -6 V
17 -7 V
16 -7 V
16 -7 V
17 -6 V
16 -7 V
16 -7 V
17 -7 V
16 -7 V
16 -6 V
17 -7 V
16 -7 V
16 -6 V
17 -7 V
16 -6 V
17 -6 V
16 -6 V
16 -6 V
17 -6 V
16 -6 V
16 -6 V
17 -6 V
16 -5 V
16 -6 V
17 -5 V
16 -5 V
16 -5 V
17 -5 V
16 -5 V
17 -5 V
16 -5 V
16 -4 V
17 -5 V
730 989 M
16 2 V
17 6 V
16 8 V
16 8 V
17 5 V
16 3 V
17 0 V
16 -2 V
16 -5 V
17 -6 V
16 -8 V
16 -8 V
17 -9 V
16 -9 V
16 -8 V
17 -8 V
16 -7 V
16 -5 V
17 -4 V
16 -3 V
17 -2 V
16 0 V
16 1 V
17 2 V
16 3 V
16 4 V
17 5 V
16 6 V
16 6 V
17 6 V
16 7 V
17 7 V
16 7 V
16 7 V
17 6 V
16 7 V
16 6 V
17 5 V
16 6 V
16 4 V
17 4 V
16 4 V
16 3 V
17 2 V
16 2 V
17 1 V
16 0 V
16 0 V
17 -1 V
16 -1 V
16 -2 V
17 -2 V
16 -3 V
16 -3 V
17 -4 V
16 -4 V
16 -5 V
17 -5 V
16 -5 V
17 -5 V
16 -6 V
16 -6 V
17 -6 V
16 -6 V
16 -7 V
17 -6 V
16 -7 V
16 -7 V
17 -6 V
16 -7 V
16 -7 V
17 -7 V
16 -7 V
17 -7 V
16 -6 V
16 -7 V
17 -7 V
16 -6 V
16 -7 V
17 -6 V
16 -7 V
16 -6 V
17 -6 V
16 -6 V
16 -6 V
17 -6 V
16 -6 V
17 -5 V
16 -6 V
16 -5 V
17 -6 V
16 -5 V
16 -5 V
17 -5 V
16 -5 V
16 -5 V
17 -4 V
16 -5 V
17 -5 V
834 1038 M
16 1 V
17 7 V
16 8 V
16 7 V
17 5 V
16 3 V
16 1 V
17 -3 V
16 -4 V
17 -7 V
16 -7 V
16 -9 V
17 -9 V
16 -9 V
16 -8 V
17 -8 V
16 -6 V
16 -6 V
17 -4 V
16 -3 V
16 -2 V
17 0 V
16 1 V
17 2 V
16 3 V
16 4 V
17 5 V
16 6 V
16 6 V
17 6 V
16 7 V
16 7 V
17 7 V
16 7 V
16 7 V
17 6 V
16 6 V
17 6 V
16 5 V
16 5 V
17 4 V
16 3 V
16 3 V
17 2 V
16 2 V
16 1 V
17 1 V
16 0 V
16 -1 V
17 -2 V
16 -1 V
17 -3 V
16 -3 V
16 -3 V
17 -4 V
16 -4 V
16 -4 V
17 -5 V
16 -5 V
16 -6 V
17 -6 V
16 -6 V
17 -6 V
16 -6 V
16 -6 V
17 -7 V
16 -7 V
16 -6 V
17 -7 V
16 -7 V
16 -7 V
17 -7 V
16 -6 V
16 -7 V
17 -7 V
16 -7 V
17 -6 V
16 -7 V
16 -7 V
17 -6 V
16 -6 V
16 -7 V
17 -6 V
16 -6 V
16 -6 V
17 -6 V
16 -5 V
16 -6 V
17 -6 V
16 -5 V
17 -5 V
16 -6 V
16 -5 V
17 -5 V
16 -5 V
16 -4 V
17 -5 V
currentpoint stroke M
16 -5 V
16 -4 V
938 1086 M
16 1 V
17 7 V
16 8 V
16 7 V
17 6 V
16 3 V
16 0 V
17 -2 V
16 -5 V
16 -6 V
17 -8 V
16 -9 V
16 -9 V
17 -8 V
16 -9 V
17 -7 V
16 -7 V
16 -6 V
17 -4 V
16 -3 V
16 -1 V
17 -1 V
16 1 V
16 2 V
17 4 V
16 4 V
16 5 V
17 5 V
16 6 V
17 7 V
16 7 V
16 6 V
17 7 V
16 7 V
16 7 V
17 6 V
16 6 V
16 6 V
17 5 V
16 5 V
17 4 V
16 4 V
16 2 V
17 3 V
16 1 V
16 2 V
17 0 V
16 0 V
16 -1 V
17 -1 V
16 -2 V
16 -2 V
17 -3 V
16 -4 V
17 -3 V
16 -5 V
16 -4 V
17 -5 V
16 -5 V
16 -6 V
17 -5 V
16 -6 V
16 -6 V
17 -7 V
16 -6 V
16 -7 V
17 -6 V
16 -7 V
17 -7 V
16 -7 V
16 -6 V
17 -7 V
16 -7 V
16 -7 V
17 -7 V
16 -6 V
16 -7 V
17 -7 V
16 -6 V
16 -7 V
17 -6 V
16 -6 V
17 -6 V
16 -7 V
16 -5 V
17 -6 V
16 -6 V
16 -6 V
17 -5 V
16 -6 V
16 -5 V
17 -5 V
16 -5 V
16 -5 V
17 -5 V
16 -5 V
17 -5 V
16 -4 V
16 -5 V
1042 1134 M
16 2 V
16 6 V
17 8 V
16 8 V
17 5 V
16 3 V
16 0 V
17 -2 V
16 -5 V
16 -6 V
17 -8 V
16 -8 V
16 -9 V
17 -9 V
16 -8 V
16 -8 V
17 -7 V
16 -5 V
17 -5 V
16 -2 V
16 -2 V
17 0 V
16 1 V
16 2 V
17 3 V
16 4 V
16 5 V
17 5 V
16 7 V
16 6 V
17 7 V
16 7 V
17 7 V
16 7 V
16 6 V
17 7 V
16 6 V
16 5 V
17 6 V
16 4 V
16 4 V
17 4 V
16 3 V
16 2 V
17 2 V
16 1 V
17 0 V
16 0 V
16 -1 V
17 -1 V
16 -2 V
16 -2 V
17 -3 V
16 -3 V
16 -4 V
17 -4 V
16 -5 V
16 -5 V
17 -5 V
16 -5 V
17 -6 V
16 -6 V
16 -6 V
17 -6 V
16 -7 V
16 -6 V
17 -7 V
16 -7 V
16 -7 V
17 -6 V
16 -7 V
17 -7 V
16 -7 V
16 -7 V
17 -6 V
16 -7 V
16 -7 V
17 -6 V
16 -7 V
16 -6 V
17 -7 V
16 -6 V
16 -6 V
17 -6 V
16 -6 V
17 -6 V
16 -6 V
16 -5 V
17 -6 V
16 -5 V
16 -6 V
17 -5 V
16 -5 V
16 -5 V
17 -5 V
16 -5 V
16 -5 V
17 -4 V
16 -5 V
1146 1183 M
16 1 V
16 6 V
17 9 V
16 7 V
16 5 V
17 3 V
16 1 V
16 -3 V
17 -4 V
16 -7 V
17 -8 V
16 -8 V
16 -9 V
17 -9 V
16 -8 V
16 -8 V
17 -6 V
16 -6 V
16 -4 V
17 -3 V
16 -2 V
16 0 V
17 1 V
16 2 V
17 3 V
16 4 V
16 5 V
17 6 V
16 6 V
16 6 V
17 7 V
16 7 V
16 7 V
17 7 V
16 7 V
16 6 V
17 6 V
16 6 V
17 5 V
16 5 V
16 4 V
17 3 V
16 3 V
16 2 V
17 2 V
16 1 V
16 1 V
17 -1 V
16 0 V
17 -2 V
16 -1 V
16 -3 V
17 -3 V
16 -3 V
16 -4 V
17 -4 V
16 -4 V
16 -5 V
17 -5 V
16 -6 V
16 -6 V
17 -6 V
16 -6 V
17 -6 V
16 -6 V
16 -7 V
17 -7 V
16 -6 V
16 -7 V
17 -7 V
16 -7 V
16 -7 V
17 -7 V
16 -6 V
16 -7 V
17 -7 V
16 -7 V
17 -6 V
16 -7 V
16 -6 V
17 -6 V
16 -7 V
16 -6 V
17 -6 V
16 -6 V
16 -6 V
17 -5 V
16 -6 V
16 -6 V
17 -5 V
16 -5 V
17 -6 V
16 -5 V
16 -5 V
17 -5 V
16 -4 V
16 -5 V
17 -5 V
16 -4 V
1249 1231 M
17 1 V
16 7 V
17 8 V
16 7 V
16 6 V
17 3 V
16 0 V
16 -3 V
17 -4 V
16 -7 V
16 -7 V
17 -9 V
16 -9 V
17 -8 V
16 -9 V
16 -7 V
17 -7 V
16 -6 V
16 -4 V
17 -3 V
16 -1 V
16 -1 V
17 1 V
16 2 V
16 3 V
17 5 V
16 4 V
17 6 V
16 6 V
16 7 V
17 6 V
16 7 V
16 7 V
17 7 V
16 7 V
16 6 V
17 6 V
16 6 V
16 5 V
17 5 V
16 4 V
17 3 V
16 3 V
16 3 V
17 1 V
16 1 V
16 1 V
17 0 V
16 -1 V
16 -1 V
17 -2 V
16 -3 V
16 -2 V
17 -4 V
16 -3 V
17 -5 V
16 -4 V
16 -5 V
17 -5 V
16 -6 V
16 -5 V
17 -6 V
16 -6 V
16 -7 V
17 -6 V
16 -7 V
16 -6 V
17 -7 V
16 -7 V
17 -7 V
16 -7 V
16 -6 V
17 -7 V
16 -7 V
16 -7 V
17 -7 V
16 -6 V
16 -7 V
17 -6 V
16 -7 V
17 -6 V
16 -6 V
16 -7 V
17 -6 V
16 -6 V
16 -5 V
17 -6 V
16 -6 V
16 -5 V
17 -6 V
16 -5 V
16 -5 V
17 -5 V
16 -5 V
17 -5 V
16 -5 V
16 -5 V
currentpoint stroke M
17 -5 V
16 -4 V
1353 1279 M
17 2 V
16 6 V
16 8 V
17 8 V
16 5 V
17 3 V
16 0 V
16 -2 V
17 -5 V
16 -6 V
16 -8 V
17 -8 V
16 -9 V
16 -9 V
17 -9 V
16 -7 V
16 -7 V
17 -5 V
16 -5 V
17 -3 V
16 -1 V
16 0 V
17 1 V
16 2 V
16 3 V
17 4 V
16 5 V
16 5 V
17 6 V
16 7 V
16 7 V
17 7 V
16 7 V
17 6 V
16 7 V
16 7 V
17 6 V
16 5 V
16 5 V
17 5 V
16 4 V
16 4 V
17 3 V
16 2 V
16 2 V
17 1 V
16 0 V
17 0 V
16 -1 V
16 -1 V
17 -2 V
16 -2 V
16 -3 V
17 -3 V
16 -4 V
16 -4 V
17 -5 V
16 -5 V
17 -5 V
16 -5 V
16 -6 V
17 -6 V
16 -6 V
16 -6 V
17 -7 V
16 -6 V
16 -7 V
17 -7 V
16 -7 V
16 -6 V
17 -7 V
16 -7 V
17 -7 V
16 -7 V
16 -7 V
17 -6 V
16 -7 V
16 -6 V
17 -7 V
16 -6 V
16 -7 V
17 -6 V
16 -6 V
16 -6 V
17 -6 V
16 -6 V
17 -6 V
16 -6 V
16 -5 V
17 -5 V
16 -6 V
16 -5 V
17 -5 V
16 -5 V
16 -5 V
17 -5 V
16 -5 V
16 -4 V
17 -5 V
1457 1327 M
17 2 V
16 6 V
16 8 V
17 8 V
16 5 V
16 3 V
17 0 V
16 -2 V
16 -5 V
17 -6 V
16 -8 V
17 -8 V
16 -9 V
16 -9 V
17 -8 V
16 -8 V
16 -7 V
17 -5 V
16 -4 V
16 -3 V
17 -2 V
16 0 V
17 1 V
16 2 V
16 3 V
17 4 V
16 5 V
16 6 V
17 6 V
16 6 V
16 7 V
17 7 V
16 7 V
16 7 V
17 6 V
16 7 V
17 6 V
16 6 V
16 5 V
17 4 V
16 5 V
16 3 V
17 3 V
16 2 V
16 2 V
17 1 V
16 1 V
16 -1 V
17 0 V
16 -2 V
17 -2 V
16 -2 V
16 -3 V
17 -3 V
16 -4 V
16 -4 V
17 -5 V
16 -4 V
16 -6 V
17 -5 V
16 -6 V
16 -6 V
17 -6 V
16 -6 V
17 -7 V
16 -6 V
16 -7 V
17 -7 V
16 -6 V
16 -7 V
17 -7 V
16 -7 V
16 -7 V
17 -6 V
16 -7 V
16 -7 V
17 -7 V
16 -6 V
17 -7 V
16 -6 V
16 -6 V
17 -7 V
16 -6 V
16 -6 V
17 -6 V
16 -6 V
16 -6 V
17 -5 V
16 -6 V
17 -5 V
16 -6 V
16 -5 V
17 -5 V
16 -5 V
16 -5 V
17 -5 V
16 -4 V
16 -5 V
17 -4 V
522 893 M
10 4 V
9 4 V
10 5 V
9 4 V
9 5 V
10 4 V
9 4 V
10 5 V
9 4 V
10 5 V
9 4 V
10 4 V
9 5 V
9 4 V
10 5 V
9 4 V
10 4 V
9 5 V
10 4 V
9 5 V
10 4 V
9 4 V
9 5 V
10 4 V
9 4 V
10 5 V
9 4 V
10 5 V
9 4 V
10 4 V
9 5 V
9 4 V
10 5 V
9 4 V
10 4 V
9 5 V
10 4 V
9 5 V
10 4 V
9 4 V
9 5 V
10 4 V
9 5 V
10 4 V
9 4 V
10 5 V
9 4 V
10 4 V
9 5 V
9 4 V
10 5 V
9 4 V
10 4 V
9 5 V
10 4 V
9 5 V
10 4 V
9 4 V
9 5 V
10 4 V
9 5 V
10 4 V
9 4 V
10 5 V
9 4 V
10 5 V
9 4 V
9 4 V
10 5 V
9 4 V
10 4 V
9 5 V
10 4 V
9 5 V
10 4 V
9 4 V
9 5 V
10 4 V
9 5 V
10 4 V
9 4 V
10 5 V
9 4 V
10 5 V
9 4 V
9 4 V
10 5 V
9 4 V
10 5 V
9 4 V
10 4 V
9 5 V
10 4 V
9 5 V
9 4 V
10 4 V
9 5 V
10 4 V
9 4 V
702 904 M
10 4 V
9 4 V
10 5 V
9 4 V
9 5 V
10 4 V
9 4 V
10 5 V
9 4 V
10 5 V
9 4 V
10 4 V
9 5 V
9 4 V
10 4 V
9 5 V
10 4 V
9 5 V
10 4 V
9 4 V
10 5 V
9 4 V
9 5 V
10 4 V
9 4 V
10 5 V
9 4 V
10 5 V
9 4 V
10 4 V
9 5 V
9 4 V
10 5 V
9 4 V
10 4 V
9 5 V
10 4 V
9 4 V
10 5 V
9 4 V
9 5 V
10 4 V
9 4 V
10 5 V
9 4 V
10 5 V
9 4 V
10 4 V
9 5 V
9 4 V
10 5 V
9 4 V
10 4 V
9 5 V
10 4 V
9 5 V
10 4 V
9 4 V
9 5 V
10 4 V
9 4 V
10 5 V
9 4 V
10 5 V
9 4 V
10 4 V
9 5 V
9 4 V
10 5 V
9 4 V
10 4 V
9 5 V
10 4 V
9 5 V
10 4 V
9 4 V
9 5 V
10 4 V
9 5 V
10 4 V
9 4 V
10 5 V
9 4 V
10 4 V
9 5 V
9 4 V
10 5 V
9 4 V
10 4 V
9 5 V
10 4 V
9 5 V
10 4 V
9 4 V
9 5 V
10 4 V
9 5 V
currentpoint stroke M
10 4 V
9 4 V
882 840 M
10 5 V
9 4 V
9 4 V
10 5 V
9 4 V
10 4 V
9 5 V
10 4 V
9 5 V
10 4 V
9 4 V
9 5 V
10 4 V
9 5 V
10 4 V
9 4 V
10 5 V
9 4 V
10 5 V
9 4 V
9 4 V
10 5 V
9 4 V
10 5 V
9 4 V
10 4 V
9 5 V
10 4 V
9 4 V
9 5 V
10 4 V
9 5 V
10 4 V
9 4 V
10 5 V
9 4 V
10 5 V
9 4 V
9 4 V
10 5 V
9 4 V
10 5 V
9 4 V
10 4 V
9 5 V
10 4 V
9 5 V
9 4 V
10 4 V
9 5 V
10 4 V
9 4 V
10 5 V
9 4 V
10 5 V
9 4 V
9 4 V
10 5 V
9 4 V
10 5 V
9 4 V
10 4 V
9 5 V
10 4 V
9 5 V
9 4 V
10 4 V
9 5 V
10 4 V
9 5 V
10 4 V
9 4 V
10 5 V
9 4 V
9 4 V
10 5 V
9 4 V
10 5 V
9 4 V
10 4 V
9 5 V
10 4 V
9 5 V
9 4 V
10 4 V
9 5 V
10 4 V
9 5 V
10 4 V
9 4 V
10 5 V
9 4 V
9 5 V
10 4 V
9 4 V
10 5 V
9 4 V
10 4 V
9 5 V
1062 894 M
10 5 V
9 4 V
9 4 V
10 5 V
9 4 V
10 4 V
9 5 V
10 4 V
9 5 V
10 4 V
9 4 V
9 5 V
10 4 V
9 5 V
10 4 V
9 4 V
10 5 V
9 4 V
10 5 V
9 4 V
9 4 V
10 5 V
9 4 V
10 5 V
9 4 V
10 4 V
9 5 V
10 4 V
9 4 V
9 5 V
10 4 V
9 5 V
10 4 V
9 4 V
10 5 V
9 4 V
10 5 V
9 4 V
9 4 V
10 5 V
9 4 V
10 5 V
9 4 V
10 4 V
9 5 V
10 4 V
9 5 V
9 4 V
10 4 V
9 5 V
10 4 V
9 4 V
10 5 V
9 4 V
10 5 V
9 4 V
9 4 V
10 5 V
9 4 V
10 5 V
9 4 V
10 4 V
9 5 V
10 4 V
9 5 V
9 4 V
10 4 V
9 5 V
10 4 V
9 5 V
10 4 V
9 4 V
10 5 V
9 4 V
9 4 V
10 5 V
9 4 V
10 5 V
9 4 V
10 4 V
9 5 V
10 4 V
9 5 V
9 4 V
10 4 V
9 5 V
10 4 V
9 5 V
10 4 V
9 4 V
10 5 V
9 4 V
9 5 V
10 4 V
9 4 V
10 5 V
9 4 V
10 4 V
9 5 V
1242 949 M
9 4 V
10 4 V
9 5 V
10 4 V
9 5 V
10 4 V
9 4 V
10 5 V
9 4 V
9 4 V
10 5 V
9 4 V
10 5 V
9 4 V
10 4 V
9 5 V
10 4 V
9 5 V
9 4 V
10 4 V
9 5 V
10 4 V
9 5 V
10 4 V
9 4 V
10 5 V
9 4 V
9 5 V
10 4 V
9 4 V
10 5 V
9 4 V
10 4 V
9 5 V
10 4 V
9 5 V
9 4 V
10 4 V
9 5 V
10 4 V
9 5 V
10 4 V
9 4 V
10 5 V
9 4 V
9 5 V
10 4 V
9 4 V
10 5 V
9 4 V
10 5 V
9 4 V
10 4 V
9 5 V
9 4 V
10 4 V
9 5 V
10 4 V
9 5 V
10 4 V
9 4 V
10 5 V
9 4 V
9 5 V
10 4 V
9 4 V
10 5 V
9 4 V
10 5 V
9 4 V
10 4 V
9 5 V
9 4 V
10 5 V
9 4 V
10 4 V
9 5 V
10 4 V
9 4 V
10 5 V
9 4 V
9 5 V
10 4 V
9 4 V
10 5 V
9 4 V
10 5 V
9 4 V
10 4 V
9 5 V
9 4 V
10 5 V
9 4 V
10 4 V
9 5 V
10 4 V
9 5 V
10 4 V
9 4 V
1422 935 M
9 5 V
10 4 V
9 5 V
10 4 V
9 4 V
10 5 V
9 4 V
10 5 V
9 4 V
9 4 V
10 5 V
9 4 V
10 5 V
9 4 V
10 4 V
9 5 V
10 4 V
9 4 V
9 5 V
10 4 V
9 5 V
10 4 V
9 4 V
10 5 V
9 4 V
10 5 V
9 4 V
9 4 V
10 5 V
9 4 V
10 5 V
9 4 V
10 4 V
9 5 V
10 4 V
9 5 V
9 4 V
10 4 V
9 5 V
10 4 V
9 4 V
10 5 V
9 4 V
10 5 V
9 4 V
9 4 V
10 5 V
9 4 V
10 5 V
9 4 V
10 4 V
9 5 V
10 4 V
9 5 V
9 4 V
10 4 V
9 5 V
10 4 V
9 5 V
10 4 V
9 4 V
10 5 V
9 4 V
9 4 V
10 5 V
9 4 V
10 5 V
9 4 V
10 4 V
9 5 V
10 4 V
9 5 V
9 4 V
10 4 V
9 5 V
10 4 V
9 5 V
10 4 V
9 4 V
10 5 V
9 4 V
9 5 V
10 4 V
9 4 V
10 5 V
9 4 V
10 4 V
9 5 V
10 4 V
9 5 V
9 4 V
10 4 V
9 5 V
10 4 V
9 5 V
10 4 V
9 4 V
currentpoint stroke M
10 5 V
9 4 V
1602 874 M
9 5 V
10 4 V
9 4 V
10 5 V
9 4 V
10 4 V
9 5 V
9 4 V
10 5 V
9 4 V
10 4 V
9 5 V
10 4 V
9 5 V
10 4 V
9 4 V
9 5 V
10 4 V
9 5 V
10 4 V
9 4 V
10 5 V
9 4 V
10 5 V
9 4 V
9 4 V
10 5 V
9 4 V
10 4 V
9 5 V
10 4 V
9 5 V
10 4 V
9 4 V
9 5 V
10 4 V
9 5 V
10 4 V
9 4 V
10 5 V
9 4 V
10 5 V
9 4 V
9 4 V
10 5 V
9 4 V
10 5 V
9 4 V
10 4 V
9 5 V
10 4 V
9 4 V
9 5 V
10 4 V
9 5 V
10 4 V
9 4 V
10 5 V
9 4 V
10 5 V
9 4 V
9 4 V
10 5 V
9 4 V
10 5 V
9 4 V
10 4 V
9 5 V
10 4 V
9 5 V
9 4 V
10 4 V
9 5 V
10 4 V
9 4 V
10 5 V
9 4 V
10 5 V
9 4 V
9 4 V
10 5 V
9 4 V
10 5 V
9 4 V
10 4 V
9 5 V
10 4 V
9 5 V
9 4 V
10 4 V
9 5 V
10 4 V
9 5 V
10 4 V
9 4 V
10 5 V
9 4 V
9 4 V
10 5 V
1782 800 M
9 4 V
10 4 V
9 5 V
10 4 V
9 5 V
10 4 V
9 4 V
9 5 V
10 4 V
9 5 V
10 4 V
9 4 V
10 5 V
9 4 V
10 5 V
9 4 V
9 4 V
10 5 V
9 4 V
10 4 V
9 5 V
10 4 V
9 5 V
10 4 V
9 4 V
9 5 V
10 4 V
9 5 V
10 4 V
9 4 V
10 5 V
9 4 V
10 5 V
9 4 V
9 4 V
10 5 V
9 4 V
10 5 V
9 4 V
10 4 V
9 5 V
10 4 V
9 4 V
9 5 V
10 4 V
9 5 V
10 4 V
9 4 V
10 5 V
9 4 V
10 5 V
9 4 V
9 4 V
10 5 V
9 4 V
10 5 V
9 4 V
10 4 V
9 5 V
10 4 V
9 5 V
9 4 V
10 4 V
9 5 V
10 4 V
9 4 V
10 5 V
9 4 V
10 5 V
9 4 V
9 4 V
10 5 V
9 4 V
10 5 V
9 4 V
10 4 V
9 5 V
10 4 V
9 5 V
9 4 V
10 4 V
9 5 V
10 4 V
9 5 V
10 4 V
9 4 V
10 5 V
9 4 V
9 4 V
10 5 V
9 4 V
10 5 V
9 4 V
10 4 V
9 5 V
10 4 V
9 5 V
9 4 V
10 4 V
1962 732 M
9 4 V
10 5 V
9 4 V
10 5 V
9 4 V
9 4 V
10 5 V
9 4 V
10 5 V
9 4 V
10 4 V
9 5 V
10 4 V
9 5 V
9 4 V
10 4 V
9 5 V
10 4 V
9 4 V
10 5 V
9 4 V
10 5 V
9 4 V
9 4 V
10 5 V
9 4 V
10 5 V
9 4 V
10 4 V
9 5 V
10 4 V
9 5 V
9 4 V
10 4 V
9 5 V
10 4 V
9 5 V
10 4 V
9 4 V
10 5 V
9 4 V
9 4 V
10 5 V
9 4 V
10 5 V
9 4 V
10 4 V
9 5 V
10 4 V
9 5 V
9 4 V
10 4 V
9 5 V
10 4 V
9 5 V
10 4 V
9 4 V
10 5 V
9 4 V
9 5 V
10 4 V
9 4 V
10 5 V
9 4 V
10 4 V
9 5 V
10 4 V
9 5 V
9 4 V
10 4 V
9 5 V
10 4 V
9 5 V
10 4 V
9 4 V
10 5 V
9 4 V
9 5 V
10 4 V
9 4 V
10 5 V
9 4 V
10 5 V
9 4 V
10 4 V
9 5 V
9 4 V
10 4 V
9 5 V
10 4 V
9 5 V
10 4 V
9 4 V
10 5 V
9 4 V
9 5 V
10 4 V
9 4 V
10 5 V
2142 677 M
9 4 V
10 5 V
9 4 V
10 4 V
9 5 V
9 4 V
10 5 V
9 4 V
10 4 V
9 5 V
10 4 V
9 4 V
10 5 V
9 4 V
9 5 V
10 4 V
9 4 V
10 5 V
9 4 V
10 5 V
9 4 V
10 4 V
9 5 V
9 4 V
10 5 V
9 4 V
10 4 V
9 5 V
10 4 V
9 5 V
10 4 V
9 4 V
9 5 V
10 4 V
9 4 V
10 5 V
9 4 V
10 5 V
9 4 V
10 4 V
9 5 V
9 4 V
10 5 V
9 4 V
10 4 V
9 5 V
10 4 V
9 5 V
10 4 V
9 4 V
9 5 V
10 4 V
9 5 V
10 4 V
9 4 V
10 5 V
9 4 V
10 5 V
9 4 V
9 4 V
10 5 V
9 4 V
10 4 V
9 5 V
10 4 V
9 5 V
10 4 V
9 4 V
9 5 V
10 4 V
9 5 V
10 4 V
9 4 V
10 5 V
9 4 V
10 5 V
9 4 V
9 4 V
10 5 V
9 4 V
10 5 V
9 4 V
10 4 V
9 5 V
10 4 V
9 4 V
9 5 V
10 4 V
9 5 V
10 4 V
9 4 V
10 5 V
9 4 V
10 5 V
9 4 V
9 4 V
10 5 V
currentpoint stroke M
9 4 V
10 5 V
1.000 UL
LTb
3077 787 M
2142 352 L
522 603 M
2142 352 L
522 603 M
935 435 V
3077 787 M
1457 1038 L
522 603 M
0 869 V
935 -434 R
0 289 V
3077 787 M
0 325 V
2142 352 M
0 325 V
522 603 M
57 26 V
878 409 R
-58 -27 V
846 553 M
57 26 V
878 408 R
-58 -27 V
1170 502 M
57 26 V
878 409 R
-58 -27 V
1494 452 M
57 26 V
878 409 R
-58 -27 V
1818 402 M
57 26 V
878 409 R
-58 -27 V
2142 352 M
57 26 V
878 409 R
-58 -27 V
2142 352 M
-63 9 V
522 603 M
62 -10 V
2291 421 M
-63 9 V
671 672 M
62 -10 V
2439 490 M
-63 9 V
820 741 M
62 -10 V
2588 559 M
-63 9 V
969 810 M
62 -10 V
2737 629 M
-63 9 V
1118 880 M
62 -10 V
2886 698 M
-63 9 V
1266 949 M
62 -10 V
3035 767 M
-63 9 V
1415 1018 M
62 -10 V
522 893 M
63 0 V
-63 96 R
63 0 V
-63 97 R
63 0 V
-63 97 R
63 0 V
-63 96 R
63 0 V
-63 97 R
63 0 V
-63 96 R
63 0 V
stroke
grestore
end
showpage
}}%
\put(522,1690){\makebox(0,0){$a_0P(r,	\theta)$}}%
\put(396,1472){\makebox(0,0)[r]{0.6}}%
\put(396,1376){\makebox(0,0)[r]{0.5}}%
\put(396,1279){\makebox(0,0)[r]{0.4}}%
\put(396,1183){\makebox(0,0)[r]{0.3}}%
\put(396,1086){\makebox(0,0)[r]{0.2}}%
\put(396,989){\makebox(0,0)[r]{0.1}}%
\put(396,893){\makebox(0,0)[r]{0}}%
\put(3014,507){\makebox(0,0){$	\theta$}}%
\put(3084,751){\makebox(0,0)[l]{3}}%
\put(2935,682){\makebox(0,0)[l]{2.5}}%
\put(2786,613){\makebox(0,0)[l]{2}}%
\put(2637,543){\makebox(0,0)[l]{1.5}}%
\put(2488,474){\makebox(0,0)[l]{1}}%
\put(2340,405){\makebox(0,0)[l]{0.5}}%
\put(2191,336){\makebox(0,0)[l]{0}}%
\put(1098,369){\makebox(0,0){$r/a_0$}}%
\put(2096,309){\makebox(0,0)[r]{10}}%
\put(1772,359){\makebox(0,0)[r]{8}}%
\put(1448,409){\makebox(0,0)[r]{6}}%
\put(1124,459){\makebox(0,0)[r]{4}}%
\put(800,510){\makebox(0,0)[r]{2}}%
\put(476,560){\makebox(0,0)[r]{0}}%
\put(3074,1796){\makebox(0,0)[r]{$n=2$ $l=0$ $m_l=0$}}%
\end{picture}%
\endgroup
\endinput

\caption{Romlig fordeling av $a_0P(r,\theta)$ for den f\o rste eksiterte tilstand med $l=0$ i hydrogenatomet.\label{46}}
\end{center}
\end{figure}
For $l=0$ tilstander er det heller ikke noen avhengighet av vinkelen 
$\theta$. Det ser vi ogs\aa\ i figur 4.6 for $n=2$ og $l=0$.
\begin{figure}[h]
\begin{center}
% GNUPLOT: LaTeX picture with Postscript
\begingroup%
  \makeatletter%
  \newcommand{\GNUPLOTspecial}{%
    \@sanitize\catcode`\%=14\relax\special}%
  \setlength{\unitlength}{0.1bp}%
{\GNUPLOTspecial{!
%!PS-Adobe-2.0 EPSF-2.0
%%Title: wave5.tex
%%Creator: gnuplot 3.7 patchlevel 0.2
%%CreationDate: Fri Mar 17 11:53:33 2000
%%DocumentFonts: 
%%BoundingBox: 0 0 360 216
%%Orientation: Landscape
%%EndComments
/gnudict 256 dict def
gnudict begin
/Color false def
/Solid false def
/gnulinewidth 5.000 def
/userlinewidth gnulinewidth def
/vshift -33 def
/dl {10 mul} def
/hpt_ 31.5 def
/vpt_ 31.5 def
/hpt hpt_ def
/vpt vpt_ def
/M {moveto} bind def
/L {lineto} bind def
/R {rmoveto} bind def
/V {rlineto} bind def
/vpt2 vpt 2 mul def
/hpt2 hpt 2 mul def
/Lshow { currentpoint stroke M
  0 vshift R show } def
/Rshow { currentpoint stroke M
  dup stringwidth pop neg vshift R show } def
/Cshow { currentpoint stroke M
  dup stringwidth pop -2 div vshift R show } def
/UP { dup vpt_ mul /vpt exch def hpt_ mul /hpt exch def
  /hpt2 hpt 2 mul def /vpt2 vpt 2 mul def } def
/DL { Color {setrgbcolor Solid {pop []} if 0 setdash }
 {pop pop pop Solid {pop []} if 0 setdash} ifelse } def
/BL { stroke userlinewidth 2 mul setlinewidth } def
/AL { stroke userlinewidth 2 div setlinewidth } def
/UL { dup gnulinewidth mul /userlinewidth exch def
      10 mul /udl exch def } def
/PL { stroke userlinewidth setlinewidth } def
/LTb { BL [] 0 0 0 DL } def
/LTa { AL [1 udl mul 2 udl mul] 0 setdash 0 0 0 setrgbcolor } def
/LT0 { PL [] 1 0 0 DL } def
/LT1 { PL [4 dl 2 dl] 0 1 0 DL } def
/LT2 { PL [2 dl 3 dl] 0 0 1 DL } def
/LT3 { PL [1 dl 1.5 dl] 1 0 1 DL } def
/LT4 { PL [5 dl 2 dl 1 dl 2 dl] 0 1 1 DL } def
/LT5 { PL [4 dl 3 dl 1 dl 3 dl] 1 1 0 DL } def
/LT6 { PL [2 dl 2 dl 2 dl 4 dl] 0 0 0 DL } def
/LT7 { PL [2 dl 2 dl 2 dl 2 dl 2 dl 4 dl] 1 0.3 0 DL } def
/LT8 { PL [2 dl 2 dl 2 dl 2 dl 2 dl 2 dl 2 dl 4 dl] 0.5 0.5 0.5 DL } def
/Pnt { stroke [] 0 setdash
   gsave 1 setlinecap M 0 0 V stroke grestore } def
/Dia { stroke [] 0 setdash 2 copy vpt add M
  hpt neg vpt neg V hpt vpt neg V
  hpt vpt V hpt neg vpt V closepath stroke
  Pnt } def
/Pls { stroke [] 0 setdash vpt sub M 0 vpt2 V
  currentpoint stroke M
  hpt neg vpt neg R hpt2 0 V stroke
  } def
/Box { stroke [] 0 setdash 2 copy exch hpt sub exch vpt add M
  0 vpt2 neg V hpt2 0 V 0 vpt2 V
  hpt2 neg 0 V closepath stroke
  Pnt } def
/Crs { stroke [] 0 setdash exch hpt sub exch vpt add M
  hpt2 vpt2 neg V currentpoint stroke M
  hpt2 neg 0 R hpt2 vpt2 V stroke } def
/TriU { stroke [] 0 setdash 2 copy vpt 1.12 mul add M
  hpt neg vpt -1.62 mul V
  hpt 2 mul 0 V
  hpt neg vpt 1.62 mul V closepath stroke
  Pnt  } def
/Star { 2 copy Pls Crs } def
/BoxF { stroke [] 0 setdash exch hpt sub exch vpt add M
  0 vpt2 neg V  hpt2 0 V  0 vpt2 V
  hpt2 neg 0 V  closepath fill } def
/TriUF { stroke [] 0 setdash vpt 1.12 mul add M
  hpt neg vpt -1.62 mul V
  hpt 2 mul 0 V
  hpt neg vpt 1.62 mul V closepath fill } def
/TriD { stroke [] 0 setdash 2 copy vpt 1.12 mul sub M
  hpt neg vpt 1.62 mul V
  hpt 2 mul 0 V
  hpt neg vpt -1.62 mul V closepath stroke
  Pnt  } def
/TriDF { stroke [] 0 setdash vpt 1.12 mul sub M
  hpt neg vpt 1.62 mul V
  hpt 2 mul 0 V
  hpt neg vpt -1.62 mul V closepath fill} def
/DiaF { stroke [] 0 setdash vpt add M
  hpt neg vpt neg V hpt vpt neg V
  hpt vpt V hpt neg vpt V closepath fill } def
/Pent { stroke [] 0 setdash 2 copy gsave
  translate 0 hpt M 4 {72 rotate 0 hpt L} repeat
  closepath stroke grestore Pnt } def
/PentF { stroke [] 0 setdash gsave
  translate 0 hpt M 4 {72 rotate 0 hpt L} repeat
  closepath fill grestore } def
/Circle { stroke [] 0 setdash 2 copy
  hpt 0 360 arc stroke Pnt } def
/CircleF { stroke [] 0 setdash hpt 0 360 arc fill } def
/C0 { BL [] 0 setdash 2 copy moveto vpt 90 450  arc } bind def
/C1 { BL [] 0 setdash 2 copy        moveto
       2 copy  vpt 0 90 arc closepath fill
               vpt 0 360 arc closepath } bind def
/C2 { BL [] 0 setdash 2 copy moveto
       2 copy  vpt 90 180 arc closepath fill
               vpt 0 360 arc closepath } bind def
/C3 { BL [] 0 setdash 2 copy moveto
       2 copy  vpt 0 180 arc closepath fill
               vpt 0 360 arc closepath } bind def
/C4 { BL [] 0 setdash 2 copy moveto
       2 copy  vpt 180 270 arc closepath fill
               vpt 0 360 arc closepath } bind def
/C5 { BL [] 0 setdash 2 copy moveto
       2 copy  vpt 0 90 arc
       2 copy moveto
       2 copy  vpt 180 270 arc closepath fill
               vpt 0 360 arc } bind def
/C6 { BL [] 0 setdash 2 copy moveto
      2 copy  vpt 90 270 arc closepath fill
              vpt 0 360 arc closepath } bind def
/C7 { BL [] 0 setdash 2 copy moveto
      2 copy  vpt 0 270 arc closepath fill
              vpt 0 360 arc closepath } bind def
/C8 { BL [] 0 setdash 2 copy moveto
      2 copy vpt 270 360 arc closepath fill
              vpt 0 360 arc closepath } bind def
/C9 { BL [] 0 setdash 2 copy moveto
      2 copy  vpt 270 450 arc closepath fill
              vpt 0 360 arc closepath } bind def
/C10 { BL [] 0 setdash 2 copy 2 copy moveto vpt 270 360 arc closepath fill
       2 copy moveto
       2 copy vpt 90 180 arc closepath fill
               vpt 0 360 arc closepath } bind def
/C11 { BL [] 0 setdash 2 copy moveto
       2 copy  vpt 0 180 arc closepath fill
       2 copy moveto
       2 copy  vpt 270 360 arc closepath fill
               vpt 0 360 arc closepath } bind def
/C12 { BL [] 0 setdash 2 copy moveto
       2 copy  vpt 180 360 arc closepath fill
               vpt 0 360 arc closepath } bind def
/C13 { BL [] 0 setdash  2 copy moveto
       2 copy  vpt 0 90 arc closepath fill
       2 copy moveto
       2 copy  vpt 180 360 arc closepath fill
               vpt 0 360 arc closepath } bind def
/C14 { BL [] 0 setdash 2 copy moveto
       2 copy  vpt 90 360 arc closepath fill
               vpt 0 360 arc } bind def
/C15 { BL [] 0 setdash 2 copy vpt 0 360 arc closepath fill
               vpt 0 360 arc closepath } bind def
/Rec   { newpath 4 2 roll moveto 1 index 0 rlineto 0 exch rlineto
       neg 0 rlineto closepath } bind def
/Square { dup Rec } bind def
/Bsquare { vpt sub exch vpt sub exch vpt2 Square } bind def
/S0 { BL [] 0 setdash 2 copy moveto 0 vpt rlineto BL Bsquare } bind def
/S1 { BL [] 0 setdash 2 copy vpt Square fill Bsquare } bind def
/S2 { BL [] 0 setdash 2 copy exch vpt sub exch vpt Square fill Bsquare } bind def
/S3 { BL [] 0 setdash 2 copy exch vpt sub exch vpt2 vpt Rec fill Bsquare } bind def
/S4 { BL [] 0 setdash 2 copy exch vpt sub exch vpt sub vpt Square fill Bsquare } bind def
/S5 { BL [] 0 setdash 2 copy 2 copy vpt Square fill
       exch vpt sub exch vpt sub vpt Square fill Bsquare } bind def
/S6 { BL [] 0 setdash 2 copy exch vpt sub exch vpt sub vpt vpt2 Rec fill Bsquare } bind def
/S7 { BL [] 0 setdash 2 copy exch vpt sub exch vpt sub vpt vpt2 Rec fill
       2 copy vpt Square fill
       Bsquare } bind def
/S8 { BL [] 0 setdash 2 copy vpt sub vpt Square fill Bsquare } bind def
/S9 { BL [] 0 setdash 2 copy vpt sub vpt vpt2 Rec fill Bsquare } bind def
/S10 { BL [] 0 setdash 2 copy vpt sub vpt Square fill 2 copy exch vpt sub exch vpt Square fill
       Bsquare } bind def
/S11 { BL [] 0 setdash 2 copy vpt sub vpt Square fill 2 copy exch vpt sub exch vpt2 vpt Rec fill
       Bsquare } bind def
/S12 { BL [] 0 setdash 2 copy exch vpt sub exch vpt sub vpt2 vpt Rec fill Bsquare } bind def
/S13 { BL [] 0 setdash 2 copy exch vpt sub exch vpt sub vpt2 vpt Rec fill
       2 copy vpt Square fill Bsquare } bind def
/S14 { BL [] 0 setdash 2 copy exch vpt sub exch vpt sub vpt2 vpt Rec fill
       2 copy exch vpt sub exch vpt Square fill Bsquare } bind def
/S15 { BL [] 0 setdash 2 copy Bsquare fill Bsquare } bind def
/D0 { gsave translate 45 rotate 0 0 S0 stroke grestore } bind def
/D1 { gsave translate 45 rotate 0 0 S1 stroke grestore } bind def
/D2 { gsave translate 45 rotate 0 0 S2 stroke grestore } bind def
/D3 { gsave translate 45 rotate 0 0 S3 stroke grestore } bind def
/D4 { gsave translate 45 rotate 0 0 S4 stroke grestore } bind def
/D5 { gsave translate 45 rotate 0 0 S5 stroke grestore } bind def
/D6 { gsave translate 45 rotate 0 0 S6 stroke grestore } bind def
/D7 { gsave translate 45 rotate 0 0 S7 stroke grestore } bind def
/D8 { gsave translate 45 rotate 0 0 S8 stroke grestore } bind def
/D9 { gsave translate 45 rotate 0 0 S9 stroke grestore } bind def
/D10 { gsave translate 45 rotate 0 0 S10 stroke grestore } bind def
/D11 { gsave translate 45 rotate 0 0 S11 stroke grestore } bind def
/D12 { gsave translate 45 rotate 0 0 S12 stroke grestore } bind def
/D13 { gsave translate 45 rotate 0 0 S13 stroke grestore } bind def
/D14 { gsave translate 45 rotate 0 0 S14 stroke grestore } bind def
/D15 { gsave translate 45 rotate 0 0 S15 stroke grestore } bind def
/DiaE { stroke [] 0 setdash vpt add M
  hpt neg vpt neg V hpt vpt neg V
  hpt vpt V hpt neg vpt V closepath stroke } def
/BoxE { stroke [] 0 setdash exch hpt sub exch vpt add M
  0 vpt2 neg V hpt2 0 V 0 vpt2 V
  hpt2 neg 0 V closepath stroke } def
/TriUE { stroke [] 0 setdash vpt 1.12 mul add M
  hpt neg vpt -1.62 mul V
  hpt 2 mul 0 V
  hpt neg vpt 1.62 mul V closepath stroke } def
/TriDE { stroke [] 0 setdash vpt 1.12 mul sub M
  hpt neg vpt 1.62 mul V
  hpt 2 mul 0 V
  hpt neg vpt -1.62 mul V closepath stroke } def
/PentE { stroke [] 0 setdash gsave
  translate 0 hpt M 4 {72 rotate 0 hpt L} repeat
  closepath stroke grestore } def
/CircE { stroke [] 0 setdash 
  hpt 0 360 arc stroke } def
/Opaque { gsave closepath 1 setgray fill grestore 0 setgray closepath } def
/DiaW { stroke [] 0 setdash vpt add M
  hpt neg vpt neg V hpt vpt neg V
  hpt vpt V hpt neg vpt V Opaque stroke } def
/BoxW { stroke [] 0 setdash exch hpt sub exch vpt add M
  0 vpt2 neg V hpt2 0 V 0 vpt2 V
  hpt2 neg 0 V Opaque stroke } def
/TriUW { stroke [] 0 setdash vpt 1.12 mul add M
  hpt neg vpt -1.62 mul V
  hpt 2 mul 0 V
  hpt neg vpt 1.62 mul V Opaque stroke } def
/TriDW { stroke [] 0 setdash vpt 1.12 mul sub M
  hpt neg vpt 1.62 mul V
  hpt 2 mul 0 V
  hpt neg vpt -1.62 mul V Opaque stroke } def
/PentW { stroke [] 0 setdash gsave
  translate 0 hpt M 4 {72 rotate 0 hpt L} repeat
  Opaque stroke grestore } def
/CircW { stroke [] 0 setdash 
  hpt 0 360 arc Opaque stroke } def
/BoxFill { gsave Rec 1 setgray fill grestore } def
end
%%EndProlog
}}%
\begin{picture}(3600,2160)(0,0)%
{\GNUPLOTspecial{"
gnudict begin
gsave
0 0 translate
0.100 0.100 scale
0 setgray
newpath
1.000 UL
LTb
1.000 UL
LT0
3124 1796 M
263 0 V
522 893 M
17 -3 V
16 -2 V
16 -2 V
17 0 V
16 1 V
16 3 V
17 5 V
16 8 V
16 10 V
17 13 V
16 14 V
17 17 V
16 18 V
16 20 V
17 20 V
16 21 V
16 22 V
17 21 V
16 20 V
16 20 V
17 19 V
16 17 V
17 16 V
16 15 V
16 12 V
17 11 V
16 9 V
16 7 V
17 5 V
16 4 V
16 2 V
17 -1 V
16 -1 V
16 -4 V
17 -4 V
16 -6 V
17 -8 V
16 -8 V
16 -10 V
17 -10 V
16 -12 V
16 -11 V
17 -13 V
16 -13 V
16 -13 V
17 -14 V
16 -14 V
16 -14 V
17 -15 V
16 -14 V
17 -14 V
16 -14 V
16 -14 V
17 -14 V
16 -14 V
16 -14 V
17 -13 V
16 -13 V
16 -12 V
17 -13 V
16 -11 V
16 -12 V
17 -11 V
16 -11 V
17 -11 V
16 -10 V
16 -10 V
17 -10 V
16 -9 V
16 -9 V
17 -8 V
16 -9 V
16 -8 V
17 -7 V
16 -8 V
16 -7 V
17 -7 V
16 -6 V
17 -7 V
16 -6 V
16 -6 V
17 -6 V
16 -5 V
16 -5 V
17 -6 V
16 -5 V
16 -4 V
17 -5 V
16 -5 V
17 -4 V
16 -4 V
16 -4 V
17 -4 V
16 -4 V
16 -4 V
17 -4 V
16 -4 V
16 -3 V
17 -4 V
626 941 M
17 -3 V
16 -2 V
16 -2 V
17 0 V
16 0 V
16 3 V
17 4 V
16 7 V
16 8 V
17 11 V
16 13 V
16 14 V
17 16 V
16 17 V
17 18 V
16 18 V
16 19 V
17 18 V
16 18 V
16 17 V
17 16 V
16 15 V
16 14 V
17 12 V
16 11 V
16 10 V
17 7 V
16 6 V
17 5 V
16 2 V
16 2 V
17 -1 V
16 -2 V
16 -3 V
17 -4 V
16 -6 V
16 -7 V
17 -8 V
16 -8 V
16 -10 V
17 -10 V
16 -11 V
17 -11 V
16 -12 V
16 -12 V
17 -13 V
16 -12 V
16 -13 V
17 -13 V
16 -13 V
16 -13 V
17 -13 V
16 -12 V
17 -13 V
16 -12 V
16 -13 V
17 -11 V
16 -12 V
16 -12 V
17 -11 V
16 -11 V
16 -10 V
17 -10 V
16 -10 V
16 -10 V
17 -9 V
16 -9 V
17 -9 V
16 -9 V
16 -8 V
17 -8 V
16 -7 V
16 -8 V
17 -7 V
16 -7 V
16 -6 V
17 -7 V
16 -6 V
16 -6 V
17 -5 V
16 -6 V
17 -5 V
16 -5 V
16 -5 V
17 -5 V
16 -5 V
16 -5 V
17 -4 V
16 -4 V
16 -4 V
17 -5 V
16 -3 V
16 -4 V
17 -4 V
16 -4 V
17 -3 V
16 -4 V
16 -3 V
17 -4 V
730 989 M
16 -2 V
17 -3 V
16 -2 V
16 -1 V
17 0 V
16 0 V
17 2 V
16 4 V
16 5 V
17 6 V
16 8 V
16 8 V
17 10 V
16 11 V
16 11 V
17 11 V
16 11 V
16 12 V
17 11 V
16 10 V
17 10 V
16 9 V
16 9 V
17 7 V
16 6 V
16 6 V
17 4 V
16 3 V
16 2 V
17 1 V
16 0 V
17 -1 V
16 -2 V
16 -3 V
17 -4 V
16 -5 V
16 -5 V
17 -6 V
16 -7 V
16 -7 V
17 -7 V
16 -9 V
16 -8 V
17 -9 V
16 -9 V
17 -9 V
16 -9 V
16 -9 V
17 -10 V
16 -9 V
16 -10 V
17 -9 V
16 -9 V
16 -9 V
17 -10 V
16 -8 V
16 -9 V
17 -9 V
16 -8 V
17 -9 V
16 -8 V
16 -8 V
17 -7 V
16 -8 V
16 -7 V
17 -7 V
16 -7 V
16 -7 V
17 -6 V
16 -6 V
16 -6 V
17 -6 V
16 -6 V
17 -6 V
16 -5 V
16 -5 V
17 -5 V
16 -5 V
16 -5 V
17 -5 V
16 -4 V
16 -5 V
17 -4 V
16 -4 V
16 -4 V
17 -4 V
16 -4 V
17 -4 V
16 -3 V
16 -4 V
17 -4 V
16 -3 V
16 -4 V
17 -3 V
16 -3 V
16 -3 V
17 -4 V
16 -3 V
17 -3 V
834 1038 M
16 -3 V
17 -2 V
16 -3 V
16 -2 V
17 -1 V
16 -2 V
16 0 V
17 0 V
16 1 V
17 1 V
16 2 V
16 2 V
17 3 V
16 3 V
16 3 V
17 3 V
16 4 V
16 3 V
17 3 V
16 3 V
16 3 V
17 2 V
16 3 V
17 1 V
16 1 V
16 1 V
17 1 V
16 0 V
16 -1 V
17 -1 V
16 -2 V
16 -2 V
17 -2 V
16 -3 V
16 -3 V
17 -3 V
16 -4 V
17 -4 V
16 -4 V
16 -5 V
17 -4 V
16 -5 V
16 -5 V
17 -5 V
16 -6 V
16 -5 V
17 -5 V
16 -6 V
16 -5 V
17 -6 V
16 -5 V
17 -6 V
16 -5 V
16 -6 V
17 -5 V
16 -5 V
16 -5 V
17 -5 V
16 -6 V
16 -5 V
17 -4 V
16 -5 V
17 -5 V
16 -5 V
16 -4 V
17 -5 V
16 -4 V
16 -4 V
17 -4 V
16 -5 V
16 -4 V
17 -4 V
16 -4 V
16 -3 V
17 -4 V
16 -4 V
17 -3 V
16 -4 V
16 -3 V
17 -4 V
16 -3 V
16 -4 V
17 -3 V
16 -3 V
16 -3 V
17 -3 V
16 -4 V
16 -3 V
17 -3 V
16 -3 V
17 -3 V
16 -3 V
16 -3 V
17 -2 V
16 -3 V
16 -3 V
17 -3 V
currentpoint stroke M
16 -3 V
16 -2 V
938 1086 M
16 -3 V
17 -2 V
16 -3 V
16 -2 V
17 -3 V
16 -2 V
16 -2 V
17 -2 V
16 -3 V
16 -2 V
17 -2 V
16 -2 V
16 -2 V
17 -1 V
16 -2 V
17 -2 V
16 -2 V
16 -2 V
17 -2 V
16 -1 V
16 -2 V
17 -2 V
16 -2 V
16 -2 V
17 -2 V
16 -2 V
16 -3 V
17 -2 V
16 -2 V
17 -3 V
16 -2 V
16 -2 V
17 -3 V
16 -2 V
16 -3 V
17 -3 V
16 -2 V
16 -3 V
17 -3 V
16 -3 V
17 -2 V
16 -3 V
16 -3 V
17 -3 V
16 -3 V
16 -3 V
17 -3 V
16 -2 V
16 -3 V
17 -3 V
16 -3 V
16 -3 V
17 -3 V
16 -3 V
17 -3 V
16 -3 V
16 -2 V
17 -3 V
16 -3 V
16 -3 V
17 -3 V
16 -3 V
16 -2 V
17 -3 V
16 -3 V
16 -3 V
17 -2 V
16 -3 V
17 -3 V
16 -3 V
16 -2 V
17 -3 V
16 -3 V
16 -2 V
17 -3 V
16 -3 V
16 -2 V
17 -3 V
16 -3 V
16 -2 V
17 -3 V
16 -3 V
17 -2 V
16 -3 V
16 -3 V
17 -2 V
16 -3 V
16 -2 V
17 -3 V
16 -3 V
16 -2 V
17 -3 V
16 -2 V
16 -3 V
17 -2 V
16 -3 V
17 -3 V
16 -2 V
16 -3 V
1042 1134 M
16 -2 V
16 -3 V
17 -2 V
16 -3 V
17 -2 V
16 -3 V
16 -2 V
17 -2 V
16 -2 V
16 -2 V
17 -2 V
16 -2 V
16 -2 V
17 -2 V
16 -2 V
16 -2 V
17 -2 V
16 -1 V
17 -2 V
16 -2 V
16 -2 V
17 -2 V
16 -2 V
16 -2 V
17 -2 V
16 -2 V
16 -2 V
17 -3 V
16 -2 V
16 -2 V
17 -3 V
16 -2 V
17 -3 V
16 -2 V
16 -3 V
17 -2 V
16 -3 V
16 -3 V
17 -3 V
16 -2 V
16 -3 V
17 -3 V
16 -3 V
16 -3 V
17 -3 V
16 -2 V
17 -3 V
16 -3 V
16 -3 V
17 -3 V
16 -3 V
16 -3 V
17 -3 V
16 -2 V
16 -3 V
17 -3 V
16 -3 V
16 -3 V
17 -3 V
16 -3 V
17 -2 V
16 -3 V
16 -3 V
17 -3 V
16 -3 V
16 -2 V
17 -3 V
16 -3 V
16 -3 V
17 -2 V
16 -3 V
17 -3 V
16 -2 V
16 -3 V
17 -3 V
16 -2 V
16 -3 V
17 -3 V
16 -2 V
16 -3 V
17 -3 V
16 -2 V
16 -3 V
17 -3 V
16 -2 V
17 -3 V
16 -2 V
16 -3 V
17 -3 V
16 -2 V
16 -3 V
17 -2 V
16 -3 V
16 -3 V
17 -2 V
16 -3 V
16 -2 V
17 -3 V
16 -2 V
1146 1183 M
16 -3 V
16 -2 V
17 -3 V
16 -2 V
16 -1 V
17 -2 V
16 0 V
16 0 V
17 0 V
16 2 V
17 2 V
16 2 V
16 2 V
17 3 V
16 4 V
16 3 V
17 3 V
16 4 V
16 3 V
17 3 V
16 3 V
16 2 V
17 2 V
16 2 V
17 1 V
16 1 V
16 1 V
17 -1 V
16 0 V
16 -1 V
17 -2 V
16 -2 V
16 -2 V
17 -3 V
16 -3 V
16 -3 V
17 -4 V
16 -4 V
17 -4 V
16 -5 V
16 -4 V
17 -5 V
16 -5 V
16 -6 V
17 -5 V
16 -5 V
16 -6 V
17 -5 V
16 -5 V
17 -6 V
16 -5 V
16 -6 V
17 -5 V
16 -6 V
16 -5 V
17 -5 V
16 -5 V
16 -6 V
17 -5 V
16 -5 V
16 -4 V
17 -5 V
16 -5 V
17 -5 V
16 -4 V
16 -5 V
17 -4 V
16 -4 V
16 -5 V
17 -4 V
16 -4 V
16 -4 V
17 -4 V
16 -3 V
16 -4 V
17 -4 V
16 -3 V
17 -4 V
16 -4 V
16 -3 V
17 -3 V
16 -4 V
16 -3 V
17 -3 V
16 -3 V
16 -4 V
17 -3 V
16 -3 V
16 -3 V
17 -3 V
16 -3 V
17 -3 V
16 -3 V
16 -2 V
17 -3 V
16 -3 V
16 -3 V
17 -3 V
16 -3 V
1249 1231 M
17 -3 V
16 -2 V
17 -2 V
16 -1 V
16 -1 V
17 1 V
16 2 V
16 3 V
17 5 V
16 7 V
16 7 V
17 9 V
16 10 V
17 10 V
16 11 V
16 11 V
17 12 V
16 11 V
16 11 V
17 11 V
16 10 V
16 9 V
17 8 V
16 8 V
16 6 V
17 5 V
16 4 V
17 4 V
16 2 V
16 1 V
17 0 V
16 -2 V
16 -2 V
17 -3 V
16 -3 V
16 -5 V
17 -5 V
16 -6 V
16 -7 V
17 -7 V
16 -8 V
17 -8 V
16 -8 V
16 -9 V
17 -9 V
16 -9 V
16 -9 V
17 -10 V
16 -9 V
16 -10 V
17 -9 V
16 -9 V
16 -10 V
17 -9 V
16 -9 V
17 -9 V
16 -9 V
16 -8 V
17 -9 V
16 -8 V
16 -8 V
17 -8 V
16 -8 V
16 -7 V
17 -7 V
16 -7 V
16 -7 V
17 -7 V
16 -6 V
17 -7 V
16 -6 V
16 -6 V
17 -5 V
16 -6 V
16 -5 V
17 -6 V
16 -5 V
16 -5 V
17 -4 V
16 -5 V
17 -5 V
16 -4 V
16 -4 V
17 -5 V
16 -4 V
16 -4 V
17 -3 V
16 -4 V
16 -4 V
17 -4 V
16 -3 V
16 -4 V
17 -3 V
16 -3 V
17 -4 V
16 -3 V
16 -3 V
currentpoint stroke M
17 -3 V
16 -3 V
1353 1279 M
17 -2 V
16 -3 V
16 -1 V
17 -1 V
16 0 V
17 3 V
16 4 V
16 7 V
17 8 V
16 11 V
16 13 V
17 14 V
16 16 V
16 17 V
17 18 V
16 18 V
16 19 V
17 18 V
16 18 V
17 17 V
16 16 V
16 15 V
17 14 V
16 13 V
16 11 V
17 9 V
16 7 V
16 6 V
17 5 V
16 3 V
16 1 V
17 -1 V
16 -2 V
17 -3 V
16 -4 V
16 -6 V
17 -7 V
16 -8 V
16 -8 V
17 -10 V
16 -10 V
16 -11 V
17 -11 V
16 -12 V
16 -12 V
17 -13 V
16 -12 V
17 -13 V
16 -13 V
16 -13 V
17 -13 V
16 -13 V
16 -12 V
17 -13 V
16 -12 V
16 -12 V
17 -12 V
16 -12 V
17 -11 V
16 -12 V
16 -11 V
17 -10 V
16 -10 V
16 -10 V
17 -10 V
16 -9 V
16 -9 V
17 -9 V
16 -9 V
16 -8 V
17 -8 V
16 -7 V
17 -8 V
16 -7 V
16 -6 V
17 -7 V
16 -6 V
16 -7 V
17 -5 V
16 -6 V
16 -6 V
17 -5 V
16 -5 V
16 -5 V
17 -5 V
16 -5 V
17 -4 V
16 -5 V
16 -4 V
17 -4 V
16 -4 V
16 -4 V
17 -4 V
16 -4 V
16 -4 V
17 -3 V
16 -4 V
16 -3 V
17 -4 V
1457 1327 M
17 -2 V
16 -2 V
16 -2 V
17 -1 V
16 1 V
16 3 V
17 6 V
16 8 V
16 10 V
17 12 V
16 15 V
17 17 V
16 18 V
16 20 V
17 20 V
16 21 V
16 21 V
17 21 V
16 21 V
16 20 V
17 18 V
16 18 V
17 16 V
16 14 V
16 13 V
17 11 V
16 9 V
16 7 V
17 5 V
16 4 V
16 1 V
17 0 V
16 -2 V
16 -3 V
17 -5 V
16 -6 V
17 -7 V
16 -9 V
16 -9 V
17 -11 V
16 -11 V
16 -12 V
17 -12 V
16 -13 V
16 -14 V
17 -14 V
16 -14 V
16 -14 V
17 -14 V
16 -14 V
17 -14 V
16 -15 V
16 -14 V
17 -14 V
16 -13 V
16 -14 V
17 -13 V
16 -13 V
16 -12 V
17 -13 V
16 -12 V
16 -11 V
17 -12 V
16 -11 V
17 -10 V
16 -10 V
16 -10 V
17 -10 V
16 -9 V
16 -9 V
17 -9 V
16 -8 V
16 -8 V
17 -8 V
16 -7 V
16 -7 V
17 -7 V
16 -7 V
17 -6 V
16 -6 V
16 -6 V
17 -6 V
16 -5 V
16 -6 V
17 -5 V
16 -5 V
16 -5 V
17 -4 V
16 -5 V
17 -4 V
16 -5 V
16 -4 V
17 -4 V
16 -4 V
16 -4 V
17 -3 V
16 -4 V
16 -3 V
17 -4 V
522 893 M
10 4 V
9 4 V
10 5 V
9 4 V
9 5 V
10 4 V
9 4 V
10 5 V
9 4 V
10 5 V
9 4 V
10 4 V
9 5 V
9 4 V
10 5 V
9 4 V
10 4 V
9 5 V
10 4 V
9 5 V
10 4 V
9 4 V
9 5 V
10 4 V
9 4 V
10 5 V
9 4 V
10 5 V
9 4 V
10 4 V
9 5 V
9 4 V
10 5 V
9 4 V
10 4 V
9 5 V
10 4 V
9 5 V
10 4 V
9 4 V
9 5 V
10 4 V
9 5 V
10 4 V
9 4 V
10 5 V
9 4 V
10 4 V
9 5 V
9 4 V
10 5 V
9 4 V
10 4 V
9 5 V
10 4 V
9 5 V
10 4 V
9 4 V
9 5 V
10 4 V
9 5 V
10 4 V
9 4 V
10 5 V
9 4 V
10 5 V
9 4 V
9 4 V
10 5 V
9 4 V
10 4 V
9 5 V
10 4 V
9 5 V
10 4 V
9 4 V
9 5 V
10 4 V
9 5 V
10 4 V
9 4 V
10 5 V
9 4 V
10 5 V
9 4 V
9 4 V
10 5 V
9 4 V
10 5 V
9 4 V
10 4 V
9 5 V
10 4 V
9 5 V
9 4 V
10 4 V
9 5 V
10 4 V
9 4 V
702 940 M
10 5 V
9 4 V
10 4 V
9 4 V
9 3 V
10 4 V
9 3 V
10 4 V
9 3 V
10 3 V
9 3 V
10 3 V
9 2 V
9 3 V
10 2 V
9 3 V
10 2 V
9 2 V
10 2 V
9 3 V
10 2 V
9 2 V
9 2 V
10 2 V
9 2 V
10 2 V
9 2 V
10 2 V
9 2 V
10 2 V
9 2 V
9 2 V
10 3 V
9 2 V
10 2 V
9 3 V
10 3 V
9 2 V
10 3 V
9 3 V
9 3 V
10 3 V
9 4 V
10 3 V
9 4 V
10 4 V
9 4 V
10 4 V
9 4 V
9 4 V
10 5 V
9 5 V
10 4 V
9 5 V
10 6 V
9 5 V
10 5 V
9 6 V
9 6 V
10 5 V
9 6 V
10 6 V
9 7 V
10 6 V
9 6 V
10 7 V
9 6 V
9 7 V
10 6 V
9 7 V
10 7 V
9 6 V
10 7 V
9 7 V
10 7 V
9 6 V
9 7 V
10 7 V
9 7 V
10 6 V
9 7 V
10 6 V
9 7 V
10 6 V
9 7 V
9 6 V
10 6 V
9 6 V
10 6 V
9 6 V
10 5 V
9 6 V
10 5 V
9 5 V
9 5 V
10 5 V
9 5 V
currentpoint stroke M
10 5 V
9 4 V
882 1155 M
10 4 V
9 4 V
9 2 V
10 3 V
9 1 V
10 1 V
9 0 V
10 0 V
9 -1 V
10 -1 V
9 -2 V
9 -2 V
10 -3 V
9 -3 V
10 -4 V
9 -4 V
10 -4 V
9 -5 V
10 -5 V
9 -5 V
9 -5 V
10 -6 V
9 -5 V
10 -6 V
9 -6 V
10 -5 V
9 -6 V
10 -6 V
9 -5 V
9 -5 V
10 -5 V
9 -5 V
10 -5 V
9 -4 V
10 -4 V
9 -3 V
10 -3 V
9 -3 V
9 -2 V
10 -1 V
9 -1 V
10 -1 V
9 0 V
10 1 V
9 1 V
10 2 V
9 2 V
9 3 V
10 4 V
9 5 V
10 5 V
9 5 V
10 7 V
9 7 V
10 7 V
9 8 V
9 9 V
10 9 V
9 10 V
10 10 V
9 11 V
10 12 V
9 12 V
10 12 V
9 12 V
9 13 V
10 14 V
9 13 V
10 14 V
9 14 V
10 14 V
9 15 V
10 14 V
9 14 V
9 15 V
10 14 V
9 15 V
10 14 V
9 14 V
10 14 V
9 14 V
10 13 V
9 13 V
9 13 V
10 13 V
9 12 V
10 11 V
9 11 V
10 11 V
9 10 V
10 10 V
9 9 V
9 8 V
10 8 V
9 7 V
10 7 V
9 6 V
10 5 V
9 5 V
1062 1234 M
10 4 V
9 3 V
9 2 V
10 1 V
9 1 V
10 0 V
9 -1 V
10 -2 V
9 -2 V
10 -4 V
9 -4 V
9 -4 V
10 -5 V
9 -6 V
10 -7 V
9 -6 V
10 -8 V
9 -7 V
10 -8 V
9 -9 V
9 -8 V
10 -9 V
9 -9 V
10 -9 V
9 -9 V
10 -9 V
9 -9 V
10 -9 V
9 -9 V
9 -8 V
10 -9 V
9 -7 V
10 -8 V
9 -7 V
10 -7 V
9 -6 V
10 -5 V
9 -5 V
9 -4 V
10 -4 V
9 -3 V
10 -2 V
9 -1 V
10 -1 V
9 0 V
10 1 V
9 2 V
9 3 V
10 3 V
9 5 V
10 5 V
9 6 V
10 7 V
9 8 V
10 8 V
9 10 V
9 10 V
10 11 V
9 12 V
10 12 V
9 13 V
10 14 V
9 14 V
10 15 V
9 15 V
9 16 V
10 16 V
9 17 V
10 17 V
9 17 V
10 18 V
9 17 V
10 18 V
9 18 V
9 18 V
10 18 V
9 17 V
10 18 V
9 17 V
10 17 V
9 17 V
10 17 V
9 16 V
9 15 V
10 16 V
9 14 V
10 14 V
9 13 V
10 13 V
9 12 V
10 12 V
9 10 V
9 10 V
10 9 V
9 8 V
10 7 V
9 7 V
10 6 V
9 4 V
1242 1135 M
9 4 V
10 3 V
9 3 V
10 2 V
9 1 V
10 1 V
9 0 V
10 -1 V
9 -2 V
9 -2 V
10 -2 V
9 -3 V
10 -4 V
9 -4 V
10 -4 V
9 -5 V
10 -6 V
9 -5 V
9 -6 V
10 -7 V
9 -6 V
10 -7 V
9 -6 V
10 -7 V
9 -7 V
10 -7 V
9 -7 V
9 -6 V
10 -7 V
9 -6 V
10 -6 V
9 -6 V
10 -5 V
9 -6 V
10 -4 V
9 -5 V
9 -4 V
10 -3 V
9 -3 V
10 -2 V
9 -2 V
10 -1 V
9 0 V
10 0 V
9 1 V
9 2 V
10 2 V
9 3 V
10 3 V
9 5 V
10 5 V
9 6 V
10 6 V
9 7 V
9 8 V
10 9 V
9 9 V
10 10 V
9 10 V
10 11 V
9 12 V
10 12 V
9 13 V
9 13 V
10 14 V
9 13 V
10 15 V
9 14 V
10 15 V
9 15 V
10 16 V
9 15 V
9 16 V
10 15 V
9 16 V
10 15 V
9 16 V
10 15 V
9 15 V
10 15 V
9 15 V
9 15 V
10 14 V
9 14 V
10 13 V
9 13 V
10 12 V
9 12 V
10 11 V
9 11 V
9 10 V
10 10 V
9 9 V
10 8 V
9 7 V
10 7 V
9 6 V
10 6 V
9 5 V
1422 981 M
9 4 V
10 4 V
9 3 V
10 3 V
9 2 V
10 2 V
9 2 V
10 1 V
9 0 V
9 1 V
10 0 V
9 -1 V
10 -1 V
9 -1 V
10 -1 V
9 -2 V
10 -1 V
9 -3 V
9 -2 V
10 -2 V
9 -3 V
10 -3 V
9 -2 V
10 -3 V
9 -3 V
10 -3 V
9 -3 V
9 -2 V
10 -3 V
9 -2 V
10 -3 V
9 -2 V
10 -2 V
9 -2 V
10 -1 V
9 -1 V
9 -1 V
10 -1 V
9 0 V
10 0 V
9 0 V
10 1 V
9 2 V
10 1 V
9 2 V
9 3 V
10 3 V
9 4 V
10 4 V
9 4 V
10 5 V
9 5 V
10 6 V
9 6 V
9 7 V
10 7 V
9 7 V
10 8 V
9 8 V
10 9 V
9 9 V
10 10 V
9 9 V
9 10 V
10 10 V
9 11 V
10 11 V
9 11 V
10 11 V
9 11 V
10 12 V
9 11 V
9 12 V
10 11 V
9 12 V
10 11 V
9 12 V
10 11 V
9 12 V
10 11 V
9 11 V
9 11 V
10 11 V
9 10 V
10 10 V
9 10 V
10 10 V
9 9 V
10 9 V
9 8 V
9 8 V
10 8 V
9 7 V
10 7 V
9 6 V
10 6 V
9 6 V
currentpoint stroke M
10 5 V
9 5 V
1602 850 M
9 4 V
10 4 V
9 4 V
10 3 V
9 4 V
10 3 V
9 3 V
9 2 V
10 2 V
9 3 V
10 2 V
9 1 V
10 2 V
9 1 V
10 2 V
9 1 V
9 1 V
10 0 V
9 1 V
10 1 V
9 0 V
10 1 V
9 1 V
10 0 V
9 0 V
9 1 V
10 0 V
9 1 V
10 0 V
9 1 V
10 1 V
9 0 V
10 1 V
9 1 V
9 2 V
10 1 V
9 1 V
10 2 V
9 2 V
10 2 V
9 2 V
10 3 V
9 2 V
9 3 V
10 4 V
9 3 V
10 4 V
9 3 V
10 5 V
9 4 V
10 5 V
9 5 V
9 5 V
10 5 V
9 6 V
10 6 V
9 6 V
10 6 V
9 6 V
10 7 V
9 7 V
9 7 V
10 8 V
9 7 V
10 8 V
9 7 V
10 8 V
9 8 V
10 8 V
9 8 V
9 9 V
10 8 V
9 8 V
10 9 V
9 8 V
10 8 V
9 9 V
10 8 V
9 8 V
9 8 V
10 8 V
9 8 V
10 8 V
9 8 V
10 7 V
9 7 V
10 8 V
9 7 V
9 7 V
10 6 V
9 7 V
10 6 V
9 6 V
10 6 V
9 5 V
10 5 V
9 5 V
9 5 V
10 5 V
1782 758 M
9 5 V
10 4 V
9 4 V
10 4 V
9 4 V
10 3 V
9 4 V
9 4 V
10 3 V
9 3 V
10 3 V
9 4 V
10 3 V
9 3 V
10 2 V
9 3 V
9 3 V
10 3 V
9 2 V
10 3 V
9 2 V
10 3 V
9 2 V
10 3 V
9 2 V
9 3 V
10 2 V
9 3 V
10 2 V
9 3 V
10 2 V
9 3 V
10 3 V
9 2 V
9 3 V
10 3 V
9 3 V
10 3 V
9 3 V
10 4 V
9 3 V
10 3 V
9 4 V
9 4 V
10 3 V
9 4 V
10 4 V
9 4 V
10 5 V
9 4 V
10 5 V
9 4 V
9 5 V
10 5 V
9 5 V
10 5 V
9 5 V
10 6 V
9 5 V
10 5 V
9 6 V
9 6 V
10 6 V
9 6 V
10 5 V
9 7 V
10 6 V
9 6 V
10 6 V
9 6 V
9 6 V
10 7 V
9 6 V
10 6 V
9 7 V
10 6 V
9 6 V
10 7 V
9 6 V
9 6 V
10 6 V
9 6 V
10 6 V
9 6 V
10 6 V
9 6 V
10 6 V
9 6 V
9 5 V
10 6 V
9 5 V
10 5 V
9 5 V
10 6 V
9 4 V
10 5 V
9 5 V
9 5 V
10 4 V
1962 697 M
9 4 V
10 5 V
9 4 V
10 4 V
9 4 V
9 4 V
10 4 V
9 4 V
10 4 V
9 4 V
10 4 V
9 4 V
10 4 V
9 3 V
9 4 V
10 4 V
9 3 V
10 4 V
9 4 V
10 3 V
9 4 V
10 3 V
9 4 V
9 3 V
10 4 V
9 3 V
10 4 V
9 3 V
10 4 V
9 4 V
10 3 V
9 4 V
9 3 V
10 4 V
9 4 V
10 3 V
9 4 V
10 4 V
9 4 V
10 4 V
9 4 V
9 4 V
10 4 V
9 4 V
10 4 V
9 4 V
10 4 V
9 4 V
10 5 V
9 4 V
9 5 V
10 4 V
9 5 V
10 4 V
9 5 V
10 5 V
9 4 V
10 5 V
9 5 V
9 5 V
10 5 V
9 5 V
10 5 V
9 5 V
10 5 V
9 5 V
10 5 V
9 6 V
9 5 V
10 5 V
9 5 V
10 5 V
9 6 V
10 5 V
9 5 V
10 6 V
9 5 V
9 5 V
10 5 V
9 5 V
10 6 V
9 5 V
10 5 V
9 5 V
10 5 V
9 5 V
9 5 V
10 5 V
9 5 V
10 5 V
9 5 V
10 5 V
9 4 V
10 5 V
9 5 V
9 4 V
10 5 V
9 4 V
10 5 V
2142 653 M
9 5 V
10 4 V
9 4 V
10 5 V
9 4 V
9 4 V
10 4 V
9 5 V
10 4 V
9 4 V
10 4 V
9 4 V
10 4 V
9 5 V
9 4 V
10 4 V
9 4 V
10 4 V
9 4 V
10 4 V
9 4 V
10 4 V
9 4 V
9 4 V
10 4 V
9 4 V
10 4 V
9 4 V
10 4 V
9 4 V
10 4 V
9 4 V
9 5 V
10 4 V
9 4 V
10 4 V
9 4 V
10 4 V
9 4 V
10 4 V
9 5 V
9 4 V
10 4 V
9 4 V
10 5 V
9 4 V
10 4 V
9 5 V
10 4 V
9 4 V
9 5 V
10 4 V
9 5 V
10 4 V
9 5 V
10 4 V
9 5 V
10 4 V
9 5 V
9 4 V
10 5 V
9 5 V
10 4 V
9 5 V
10 5 V
9 4 V
10 5 V
9 5 V
9 5 V
10 4 V
9 5 V
10 5 V
9 5 V
10 4 V
9 5 V
10 5 V
9 5 V
9 4 V
10 5 V
9 5 V
10 5 V
9 4 V
10 5 V
9 5 V
10 4 V
9 5 V
9 5 V
10 4 V
9 5 V
10 5 V
9 4 V
10 5 V
9 4 V
10 5 V
9 4 V
9 5 V
10 4 V
currentpoint stroke M
9 5 V
10 4 V
1.000 UL
LTb
3077 787 M
2142 352 L
522 603 M
2142 352 L
522 603 M
935 435 V
3077 787 M
1457 1038 L
522 603 M
0 869 V
935 -434 R
0 289 V
3077 787 M
0 301 V
2142 352 M
0 301 V
522 603 M
57 26 V
878 409 R
-58 -27 V
792 561 M
57 26 V
878 409 R
-58 -27 V
1062 519 M
57 26 V
878 409 R
-58 -27 V
1332 477 M
57 26 V
878 409 R
-58 -27 V
1602 436 M
57 26 V
878 408 R
-58 -27 V
1872 394 M
57 26 V
878 408 R
-58 -27 V
2142 352 M
57 26 V
878 409 R
-58 -27 V
2142 352 M
-63 9 V
522 603 M
62 -10 V
2291 421 M
-63 9 V
671 672 M
62 -10 V
2439 490 M
-63 9 V
820 741 M
62 -10 V
2588 559 M
-63 9 V
969 810 M
62 -10 V
2737 629 M
-63 9 V
1118 880 M
62 -10 V
2886 698 M
-63 9 V
1266 949 M
62 -10 V
3035 767 M
-63 9 V
1415 1018 M
62 -10 V
522 893 M
63 0 V
-63 145 R
63 0 V
-63 145 R
63 0 V
-63 144 R
63 0 V
-63 145 R
63 0 V
stroke
grestore
end
showpage
}}%
\put(522,1690){\makebox(0,0){$a_0P(r,	\theta)$}}%
\put(396,1472){\makebox(0,0)[r]{0.2}}%
\put(396,1327){\makebox(0,0)[r]{0.15}}%
\put(396,1183){\makebox(0,0)[r]{0.1}}%
\put(396,1038){\makebox(0,0)[r]{0.05}}%
\put(396,893){\makebox(0,0)[r]{0}}%
\put(3014,507){\makebox(0,0){$	\theta$}}%
\put(3084,751){\makebox(0,0)[l]{3}}%
\put(2935,682){\makebox(0,0)[l]{2.5}}%
\put(2786,613){\makebox(0,0)[l]{2}}%
\put(2637,543){\makebox(0,0)[l]{1.5}}%
\put(2488,474){\makebox(0,0)[l]{1}}%
\put(2340,405){\makebox(0,0)[l]{0.5}}%
\put(2191,336){\makebox(0,0)[l]{0}}%
\put(1098,369){\makebox(0,0){$r/a_0$}}%
\put(2096,309){\makebox(0,0)[r]{12}}%
\put(1826,351){\makebox(0,0)[r]{10}}%
\put(1556,393){\makebox(0,0)[r]{8}}%
\put(1286,434){\makebox(0,0)[r]{6}}%
\put(1016,476){\makebox(0,0)[r]{4}}%
\put(746,518){\makebox(0,0)[r]{2}}%
\put(476,560){\makebox(0,0)[r]{0}}%
\put(3074,1796){\makebox(0,0)[r]{$n=2$ $l=1$ $m_l=0$}}%
\end{picture}%
\endgroup
\endinput

\caption{Romlig fordeling av $a_0P(r,\theta)$ for den f\o rste eksiterte tilstand med $l=1$ i hydrogenatomet med $m_l=0$.\label{47}}
\end{center}
\end{figure}

For tilstander med $l >0$ har vi derimot en avhengighet av $\theta$, 
noe figurene \ref{46}, \ref{47} og \ref{48} viser.
\begin{figure}
\begin{center}
% GNUPLOT: LaTeX picture with Postscript
\begingroup%
  \makeatletter%
  \newcommand{\GNUPLOTspecial}{%
    \@sanitize\catcode`\%=14\relax\special}%
  \setlength{\unitlength}{0.1bp}%
{\GNUPLOTspecial{!
%!PS-Adobe-2.0 EPSF-2.0
%%Title: wave6.tex
%%Creator: gnuplot 3.7 patchlevel 0.2
%%CreationDate: Fri Mar 17 11:53:50 2000
%%DocumentFonts: 
%%BoundingBox: 0 0 360 216
%%Orientation: Landscape
%%EndComments
/gnudict 256 dict def
gnudict begin
/Color false def
/Solid false def
/gnulinewidth 5.000 def
/userlinewidth gnulinewidth def
/vshift -33 def
/dl {10 mul} def
/hpt_ 31.5 def
/vpt_ 31.5 def
/hpt hpt_ def
/vpt vpt_ def
/M {moveto} bind def
/L {lineto} bind def
/R {rmoveto} bind def
/V {rlineto} bind def
/vpt2 vpt 2 mul def
/hpt2 hpt 2 mul def
/Lshow { currentpoint stroke M
  0 vshift R show } def
/Rshow { currentpoint stroke M
  dup stringwidth pop neg vshift R show } def
/Cshow { currentpoint stroke M
  dup stringwidth pop -2 div vshift R show } def
/UP { dup vpt_ mul /vpt exch def hpt_ mul /hpt exch def
  /hpt2 hpt 2 mul def /vpt2 vpt 2 mul def } def
/DL { Color {setrgbcolor Solid {pop []} if 0 setdash }
 {pop pop pop Solid {pop []} if 0 setdash} ifelse } def
/BL { stroke userlinewidth 2 mul setlinewidth } def
/AL { stroke userlinewidth 2 div setlinewidth } def
/UL { dup gnulinewidth mul /userlinewidth exch def
      10 mul /udl exch def } def
/PL { stroke userlinewidth setlinewidth } def
/LTb { BL [] 0 0 0 DL } def
/LTa { AL [1 udl mul 2 udl mul] 0 setdash 0 0 0 setrgbcolor } def
/LT0 { PL [] 1 0 0 DL } def
/LT1 { PL [4 dl 2 dl] 0 1 0 DL } def
/LT2 { PL [2 dl 3 dl] 0 0 1 DL } def
/LT3 { PL [1 dl 1.5 dl] 1 0 1 DL } def
/LT4 { PL [5 dl 2 dl 1 dl 2 dl] 0 1 1 DL } def
/LT5 { PL [4 dl 3 dl 1 dl 3 dl] 1 1 0 DL } def
/LT6 { PL [2 dl 2 dl 2 dl 4 dl] 0 0 0 DL } def
/LT7 { PL [2 dl 2 dl 2 dl 2 dl 2 dl 4 dl] 1 0.3 0 DL } def
/LT8 { PL [2 dl 2 dl 2 dl 2 dl 2 dl 2 dl 2 dl 4 dl] 0.5 0.5 0.5 DL } def
/Pnt { stroke [] 0 setdash
   gsave 1 setlinecap M 0 0 V stroke grestore } def
/Dia { stroke [] 0 setdash 2 copy vpt add M
  hpt neg vpt neg V hpt vpt neg V
  hpt vpt V hpt neg vpt V closepath stroke
  Pnt } def
/Pls { stroke [] 0 setdash vpt sub M 0 vpt2 V
  currentpoint stroke M
  hpt neg vpt neg R hpt2 0 V stroke
  } def
/Box { stroke [] 0 setdash 2 copy exch hpt sub exch vpt add M
  0 vpt2 neg V hpt2 0 V 0 vpt2 V
  hpt2 neg 0 V closepath stroke
  Pnt } def
/Crs { stroke [] 0 setdash exch hpt sub exch vpt add M
  hpt2 vpt2 neg V currentpoint stroke M
  hpt2 neg 0 R hpt2 vpt2 V stroke } def
/TriU { stroke [] 0 setdash 2 copy vpt 1.12 mul add M
  hpt neg vpt -1.62 mul V
  hpt 2 mul 0 V
  hpt neg vpt 1.62 mul V closepath stroke
  Pnt  } def
/Star { 2 copy Pls Crs } def
/BoxF { stroke [] 0 setdash exch hpt sub exch vpt add M
  0 vpt2 neg V  hpt2 0 V  0 vpt2 V
  hpt2 neg 0 V  closepath fill } def
/TriUF { stroke [] 0 setdash vpt 1.12 mul add M
  hpt neg vpt -1.62 mul V
  hpt 2 mul 0 V
  hpt neg vpt 1.62 mul V closepath fill } def
/TriD { stroke [] 0 setdash 2 copy vpt 1.12 mul sub M
  hpt neg vpt 1.62 mul V
  hpt 2 mul 0 V
  hpt neg vpt -1.62 mul V closepath stroke
  Pnt  } def
/TriDF { stroke [] 0 setdash vpt 1.12 mul sub M
  hpt neg vpt 1.62 mul V
  hpt 2 mul 0 V
  hpt neg vpt -1.62 mul V closepath fill} def
/DiaF { stroke [] 0 setdash vpt add M
  hpt neg vpt neg V hpt vpt neg V
  hpt vpt V hpt neg vpt V closepath fill } def
/Pent { stroke [] 0 setdash 2 copy gsave
  translate 0 hpt M 4 {72 rotate 0 hpt L} repeat
  closepath stroke grestore Pnt } def
/PentF { stroke [] 0 setdash gsave
  translate 0 hpt M 4 {72 rotate 0 hpt L} repeat
  closepath fill grestore } def
/Circle { stroke [] 0 setdash 2 copy
  hpt 0 360 arc stroke Pnt } def
/CircleF { stroke [] 0 setdash hpt 0 360 arc fill } def
/C0 { BL [] 0 setdash 2 copy moveto vpt 90 450  arc } bind def
/C1 { BL [] 0 setdash 2 copy        moveto
       2 copy  vpt 0 90 arc closepath fill
               vpt 0 360 arc closepath } bind def
/C2 { BL [] 0 setdash 2 copy moveto
       2 copy  vpt 90 180 arc closepath fill
               vpt 0 360 arc closepath } bind def
/C3 { BL [] 0 setdash 2 copy moveto
       2 copy  vpt 0 180 arc closepath fill
               vpt 0 360 arc closepath } bind def
/C4 { BL [] 0 setdash 2 copy moveto
       2 copy  vpt 180 270 arc closepath fill
               vpt 0 360 arc closepath } bind def
/C5 { BL [] 0 setdash 2 copy moveto
       2 copy  vpt 0 90 arc
       2 copy moveto
       2 copy  vpt 180 270 arc closepath fill
               vpt 0 360 arc } bind def
/C6 { BL [] 0 setdash 2 copy moveto
      2 copy  vpt 90 270 arc closepath fill
              vpt 0 360 arc closepath } bind def
/C7 { BL [] 0 setdash 2 copy moveto
      2 copy  vpt 0 270 arc closepath fill
              vpt 0 360 arc closepath } bind def
/C8 { BL [] 0 setdash 2 copy moveto
      2 copy vpt 270 360 arc closepath fill
              vpt 0 360 arc closepath } bind def
/C9 { BL [] 0 setdash 2 copy moveto
      2 copy  vpt 270 450 arc closepath fill
              vpt 0 360 arc closepath } bind def
/C10 { BL [] 0 setdash 2 copy 2 copy moveto vpt 270 360 arc closepath fill
       2 copy moveto
       2 copy vpt 90 180 arc closepath fill
               vpt 0 360 arc closepath } bind def
/C11 { BL [] 0 setdash 2 copy moveto
       2 copy  vpt 0 180 arc closepath fill
       2 copy moveto
       2 copy  vpt 270 360 arc closepath fill
               vpt 0 360 arc closepath } bind def
/C12 { BL [] 0 setdash 2 copy moveto
       2 copy  vpt 180 360 arc closepath fill
               vpt 0 360 arc closepath } bind def
/C13 { BL [] 0 setdash  2 copy moveto
       2 copy  vpt 0 90 arc closepath fill
       2 copy moveto
       2 copy  vpt 180 360 arc closepath fill
               vpt 0 360 arc closepath } bind def
/C14 { BL [] 0 setdash 2 copy moveto
       2 copy  vpt 90 360 arc closepath fill
               vpt 0 360 arc } bind def
/C15 { BL [] 0 setdash 2 copy vpt 0 360 arc closepath fill
               vpt 0 360 arc closepath } bind def
/Rec   { newpath 4 2 roll moveto 1 index 0 rlineto 0 exch rlineto
       neg 0 rlineto closepath } bind def
/Square { dup Rec } bind def
/Bsquare { vpt sub exch vpt sub exch vpt2 Square } bind def
/S0 { BL [] 0 setdash 2 copy moveto 0 vpt rlineto BL Bsquare } bind def
/S1 { BL [] 0 setdash 2 copy vpt Square fill Bsquare } bind def
/S2 { BL [] 0 setdash 2 copy exch vpt sub exch vpt Square fill Bsquare } bind def
/S3 { BL [] 0 setdash 2 copy exch vpt sub exch vpt2 vpt Rec fill Bsquare } bind def
/S4 { BL [] 0 setdash 2 copy exch vpt sub exch vpt sub vpt Square fill Bsquare } bind def
/S5 { BL [] 0 setdash 2 copy 2 copy vpt Square fill
       exch vpt sub exch vpt sub vpt Square fill Bsquare } bind def
/S6 { BL [] 0 setdash 2 copy exch vpt sub exch vpt sub vpt vpt2 Rec fill Bsquare } bind def
/S7 { BL [] 0 setdash 2 copy exch vpt sub exch vpt sub vpt vpt2 Rec fill
       2 copy vpt Square fill
       Bsquare } bind def
/S8 { BL [] 0 setdash 2 copy vpt sub vpt Square fill Bsquare } bind def
/S9 { BL [] 0 setdash 2 copy vpt sub vpt vpt2 Rec fill Bsquare } bind def
/S10 { BL [] 0 setdash 2 copy vpt sub vpt Square fill 2 copy exch vpt sub exch vpt Square fill
       Bsquare } bind def
/S11 { BL [] 0 setdash 2 copy vpt sub vpt Square fill 2 copy exch vpt sub exch vpt2 vpt Rec fill
       Bsquare } bind def
/S12 { BL [] 0 setdash 2 copy exch vpt sub exch vpt sub vpt2 vpt Rec fill Bsquare } bind def
/S13 { BL [] 0 setdash 2 copy exch vpt sub exch vpt sub vpt2 vpt Rec fill
       2 copy vpt Square fill Bsquare } bind def
/S14 { BL [] 0 setdash 2 copy exch vpt sub exch vpt sub vpt2 vpt Rec fill
       2 copy exch vpt sub exch vpt Square fill Bsquare } bind def
/S15 { BL [] 0 setdash 2 copy Bsquare fill Bsquare } bind def
/D0 { gsave translate 45 rotate 0 0 S0 stroke grestore } bind def
/D1 { gsave translate 45 rotate 0 0 S1 stroke grestore } bind def
/D2 { gsave translate 45 rotate 0 0 S2 stroke grestore } bind def
/D3 { gsave translate 45 rotate 0 0 S3 stroke grestore } bind def
/D4 { gsave translate 45 rotate 0 0 S4 stroke grestore } bind def
/D5 { gsave translate 45 rotate 0 0 S5 stroke grestore } bind def
/D6 { gsave translate 45 rotate 0 0 S6 stroke grestore } bind def
/D7 { gsave translate 45 rotate 0 0 S7 stroke grestore } bind def
/D8 { gsave translate 45 rotate 0 0 S8 stroke grestore } bind def
/D9 { gsave translate 45 rotate 0 0 S9 stroke grestore } bind def
/D10 { gsave translate 45 rotate 0 0 S10 stroke grestore } bind def
/D11 { gsave translate 45 rotate 0 0 S11 stroke grestore } bind def
/D12 { gsave translate 45 rotate 0 0 S12 stroke grestore } bind def
/D13 { gsave translate 45 rotate 0 0 S13 stroke grestore } bind def
/D14 { gsave translate 45 rotate 0 0 S14 stroke grestore } bind def
/D15 { gsave translate 45 rotate 0 0 S15 stroke grestore } bind def
/DiaE { stroke [] 0 setdash vpt add M
  hpt neg vpt neg V hpt vpt neg V
  hpt vpt V hpt neg vpt V closepath stroke } def
/BoxE { stroke [] 0 setdash exch hpt sub exch vpt add M
  0 vpt2 neg V hpt2 0 V 0 vpt2 V
  hpt2 neg 0 V closepath stroke } def
/TriUE { stroke [] 0 setdash vpt 1.12 mul add M
  hpt neg vpt -1.62 mul V
  hpt 2 mul 0 V
  hpt neg vpt 1.62 mul V closepath stroke } def
/TriDE { stroke [] 0 setdash vpt 1.12 mul sub M
  hpt neg vpt 1.62 mul V
  hpt 2 mul 0 V
  hpt neg vpt -1.62 mul V closepath stroke } def
/PentE { stroke [] 0 setdash gsave
  translate 0 hpt M 4 {72 rotate 0 hpt L} repeat
  closepath stroke grestore } def
/CircE { stroke [] 0 setdash 
  hpt 0 360 arc stroke } def
/Opaque { gsave closepath 1 setgray fill grestore 0 setgray closepath } def
/DiaW { stroke [] 0 setdash vpt add M
  hpt neg vpt neg V hpt vpt neg V
  hpt vpt V hpt neg vpt V Opaque stroke } def
/BoxW { stroke [] 0 setdash exch hpt sub exch vpt add M
  0 vpt2 neg V hpt2 0 V 0 vpt2 V
  hpt2 neg 0 V Opaque stroke } def
/TriUW { stroke [] 0 setdash vpt 1.12 mul add M
  hpt neg vpt -1.62 mul V
  hpt 2 mul 0 V
  hpt neg vpt 1.62 mul V Opaque stroke } def
/TriDW { stroke [] 0 setdash vpt 1.12 mul sub M
  hpt neg vpt 1.62 mul V
  hpt 2 mul 0 V
  hpt neg vpt -1.62 mul V Opaque stroke } def
/PentW { stroke [] 0 setdash gsave
  translate 0 hpt M 4 {72 rotate 0 hpt L} repeat
  Opaque stroke grestore } def
/CircW { stroke [] 0 setdash 
  hpt 0 360 arc Opaque stroke } def
/BoxFill { gsave Rec 1 setgray fill grestore } def
end
%%EndProlog
}}%
\begin{picture}(3600,2160)(0,0)%
{\GNUPLOTspecial{"
gnudict begin
gsave
0 0 translate
0.100 0.100 scale
0 setgray
newpath
1.000 UL
LTb
1.000 UL
LT0
3124 1796 M
263 0 V
522 893 M
17 -3 V
16 -1 V
16 3 V
17 7 V
16 10 V
16 12 V
17 13 V
16 10 V
16 7 V
17 3 V
16 -2 V
17 -7 V
16 -10 V
16 -13 V
17 -15 V
16 -15 V
16 -15 V
17 -13 V
16 -11 V
16 -9 V
17 -4 V
16 -2 V
17 2 V
16 4 V
16 8 V
17 10 V
16 12 V
16 13 V
17 14 V
16 14 V
16 15 V
17 14 V
16 13 V
16 12 V
17 11 V
16 9 V
17 7 V
16 6 V
16 4 V
17 2 V
16 0 V
16 -1 V
17 -2 V
16 -4 V
16 -6 V
17 -6 V
16 -8 V
16 -8 V
17 -9 V
16 -10 V
17 -10 V
16 -10 V
16 -11 V
17 -11 V
16 -11 V
16 -11 V
17 -10 V
16 -11 V
16 -11 V
17 -10 V
16 -10 V
16 -9 V
17 -10 V
16 -9 V
17 -8 V
16 -9 V
16 -8 V
17 -8 V
16 -7 V
16 -7 V
17 -7 V
16 -6 V
16 -6 V
17 -6 V
16 -6 V
16 -5 V
17 -5 V
16 -5 V
17 -5 V
16 -4 V
16 -4 V
17 -5 V
16 -4 V
16 -4 V
17 -3 V
16 -4 V
16 -3 V
17 -4 V
16 -3 V
17 -3 V
16 -4 V
16 -3 V
17 -3 V
16 -3 V
16 -3 V
17 -3 V
16 -3 V
16 -2 V
17 -3 V
626 941 M
17 -2 V
16 -1 V
16 2 V
17 5 V
16 9 V
16 10 V
17 11 V
16 9 V
16 6 V
17 2 V
16 -2 V
16 -6 V
17 -10 V
16 -11 V
17 -14 V
16 -14 V
16 -13 V
17 -12 V
16 -10 V
16 -8 V
17 -4 V
16 -2 V
16 1 V
17 4 V
16 7 V
16 8 V
17 10 V
16 12 V
17 12 V
16 12 V
16 13 V
17 12 V
16 11 V
16 10 V
17 9 V
16 8 V
16 6 V
17 5 V
16 3 V
16 2 V
17 0 V
16 -1 V
17 -3 V
16 -4 V
16 -5 V
17 -6 V
16 -7 V
16 -7 V
17 -9 V
16 -8 V
16 -10 V
17 -9 V
16 -10 V
17 -10 V
16 -10 V
16 -10 V
17 -9 V
16 -10 V
16 -10 V
17 -9 V
16 -9 V
16 -9 V
17 -9 V
16 -8 V
16 -8 V
17 -7 V
16 -8 V
17 -7 V
16 -7 V
16 -6 V
17 -6 V
16 -6 V
16 -6 V
17 -6 V
16 -5 V
16 -5 V
17 -5 V
16 -4 V
16 -5 V
17 -4 V
16 -4 V
17 -4 V
16 -4 V
16 -4 V
17 -3 V
16 -4 V
16 -3 V
17 -3 V
16 -4 V
16 -3 V
17 -3 V
16 -3 V
16 -3 V
17 -3 V
16 -3 V
17 -3 V
16 -3 V
16 -2 V
17 -3 V
730 989 M
16 -2 V
17 -2 V
16 1 V
16 3 V
17 5 V
16 6 V
17 6 V
16 5 V
16 3 V
17 1 V
16 -3 V
16 -4 V
17 -8 V
16 -8 V
16 -10 V
17 -10 V
16 -10 V
16 -9 V
17 -7 V
16 -6 V
17 -4 V
16 -2 V
16 0 V
17 2 V
16 3 V
16 5 V
17 6 V
16 7 V
16 7 V
17 7 V
16 8 V
17 7 V
16 6 V
16 6 V
17 6 V
16 4 V
16 3 V
17 3 V
16 1 V
16 0 V
17 -1 V
16 -1 V
16 -3 V
17 -3 V
16 -5 V
17 -4 V
16 -6 V
16 -6 V
17 -6 V
16 -7 V
16 -7 V
17 -7 V
16 -7 V
16 -8 V
17 -7 V
16 -8 V
16 -7 V
17 -7 V
16 -8 V
17 -7 V
16 -7 V
16 -6 V
17 -7 V
16 -6 V
16 -6 V
17 -6 V
16 -6 V
16 -6 V
17 -5 V
16 -5 V
16 -5 V
17 -5 V
16 -4 V
17 -5 V
16 -4 V
16 -4 V
17 -5 V
16 -3 V
16 -4 V
17 -4 V
16 -4 V
16 -3 V
17 -3 V
16 -4 V
16 -3 V
17 -3 V
16 -3 V
17 -3 V
16 -3 V
16 -3 V
17 -3 V
16 -3 V
16 -3 V
17 -3 V
16 -3 V
16 -2 V
17 -3 V
16 -3 V
17 -2 V
834 1038 M
16 -3 V
17 -2 V
16 -1 V
16 0 V
17 0 V
16 1 V
16 2 V
17 0 V
16 0 V
17 -1 V
16 -3 V
16 -3 V
17 -5 V
16 -5 V
16 -5 V
17 -6 V
16 -6 V
16 -5 V
17 -5 V
16 -3 V
16 -4 V
17 -2 V
16 -1 V
17 -1 V
16 0 V
16 0 V
17 2 V
16 1 V
16 2 V
17 1 V
16 2 V
16 2 V
17 1 V
16 1 V
16 1 V
17 0 V
16 0 V
17 0 V
16 -1 V
16 -2 V
17 -1 V
16 -3 V
16 -2 V
17 -3 V
16 -3 V
16 -4 V
17 -4 V
16 -4 V
16 -4 V
17 -4 V
16 -4 V
17 -5 V
16 -5 V
16 -4 V
17 -5 V
16 -4 V
16 -5 V
17 -5 V
16 -4 V
16 -5 V
17 -4 V
16 -4 V
17 -5 V
16 -4 V
16 -4 V
17 -4 V
16 -4 V
16 -3 V
17 -4 V
16 -4 V
16 -3 V
17 -4 V
16 -3 V
16 -4 V
17 -3 V
16 -3 V
17 -3 V
16 -4 V
16 -3 V
17 -3 V
16 -3 V
16 -3 V
17 -3 V
16 -2 V
16 -3 V
17 -3 V
16 -3 V
16 -3 V
17 -2 V
16 -3 V
17 -3 V
16 -2 V
16 -3 V
17 -3 V
16 -2 V
16 -3 V
17 -3 V
currentpoint stroke M
16 -2 V
16 -3 V
938 1086 M
16 -3 V
17 -2 V
16 -2 V
16 -3 V
17 -2 V
16 -2 V
16 -2 V
17 -2 V
16 -2 V
16 -3 V
17 -2 V
16 -3 V
16 -3 V
17 -3 V
16 -3 V
17 -2 V
16 -3 V
16 -3 V
17 -3 V
16 -3 V
16 -2 V
17 -3 V
16 -2 V
16 -3 V
17 -2 V
16 -2 V
16 -2 V
17 -2 V
16 -2 V
17 -2 V
16 -2 V
16 -2 V
17 -2 V
16 -2 V
16 -3 V
17 -2 V
16 -2 V
16 -2 V
17 -3 V
16 -2 V
17 -2 V
16 -3 V
16 -2 V
17 -3 V
16 -3 V
16 -2 V
17 -3 V
16 -3 V
16 -2 V
17 -3 V
16 -3 V
16 -3 V
17 -2 V
16 -3 V
17 -3 V
16 -3 V
16 -3 V
17 -2 V
16 -3 V
16 -3 V
17 -3 V
16 -2 V
16 -3 V
17 -3 V
16 -3 V
16 -2 V
17 -3 V
16 -3 V
17 -2 V
16 -3 V
16 -3 V
17 -2 V
16 -3 V
16 -3 V
17 -2 V
16 -3 V
16 -3 V
17 -2 V
16 -3 V
16 -2 V
17 -3 V
16 -3 V
17 -2 V
16 -3 V
16 -2 V
17 -3 V
16 -2 V
16 -3 V
17 -3 V
16 -2 V
16 -3 V
17 -2 V
16 -3 V
16 -2 V
17 -3 V
16 -2 V
17 -3 V
16 -2 V
16 -3 V
1042 1134 M
16 -2 V
16 -3 V
17 -2 V
16 -2 V
17 -3 V
16 -2 V
16 -2 V
17 -2 V
16 -2 V
16 -3 V
17 -2 V
16 -3 V
16 -2 V
17 -3 V
16 -3 V
16 -3 V
17 -3 V
16 -3 V
17 -3 V
16 -2 V
16 -3 V
17 -3 V
16 -2 V
16 -2 V
17 -2 V
16 -3 V
16 -2 V
17 -2 V
16 -2 V
16 -2 V
17 -2 V
16 -2 V
17 -2 V
16 -2 V
16 -2 V
17 -2 V
16 -3 V
16 -2 V
17 -2 V
16 -3 V
16 -2 V
17 -3 V
16 -2 V
16 -3 V
17 -2 V
16 -3 V
17 -3 V
16 -2 V
16 -3 V
17 -3 V
16 -3 V
16 -2 V
17 -3 V
16 -3 V
16 -3 V
17 -3 V
16 -2 V
16 -3 V
17 -3 V
16 -3 V
17 -2 V
16 -3 V
16 -3 V
17 -3 V
16 -2 V
16 -3 V
17 -3 V
16 -2 V
16 -3 V
17 -3 V
16 -2 V
17 -3 V
16 -3 V
16 -2 V
17 -3 V
16 -3 V
16 -2 V
17 -3 V
16 -2 V
16 -3 V
17 -3 V
16 -2 V
16 -3 V
17 -2 V
16 -3 V
17 -3 V
16 -2 V
16 -3 V
17 -2 V
16 -3 V
16 -2 V
17 -3 V
16 -2 V
16 -3 V
17 -3 V
16 -2 V
16 -3 V
17 -2 V
16 -3 V
1146 1183 M
16 -3 V
16 -2 V
17 -1 V
16 -1 V
16 1 V
17 1 V
16 1 V
16 1 V
17 0 V
16 -1 V
17 -3 V
16 -3 V
16 -5 V
17 -5 V
16 -5 V
16 -6 V
17 -6 V
16 -5 V
16 -5 V
17 -4 V
16 -3 V
16 -2 V
17 -1 V
16 -1 V
17 0 V
16 0 V
16 1 V
17 2 V
16 1 V
16 2 V
17 2 V
16 1 V
16 2 V
17 1 V
16 1 V
16 0 V
17 0 V
16 0 V
17 -1 V
16 -2 V
16 -1 V
17 -3 V
16 -2 V
16 -3 V
17 -3 V
16 -4 V
16 -4 V
17 -4 V
16 -4 V
17 -4 V
16 -5 V
16 -4 V
17 -5 V
16 -4 V
16 -5 V
17 -5 V
16 -4 V
16 -5 V
17 -4 V
16 -5 V
16 -4 V
17 -4 V
16 -5 V
17 -4 V
16 -4 V
16 -4 V
17 -4 V
16 -4 V
16 -3 V
17 -4 V
16 -4 V
16 -3 V
17 -3 V
16 -4 V
16 -3 V
17 -3 V
16 -4 V
17 -3 V
16 -3 V
16 -3 V
17 -3 V
16 -3 V
16 -3 V
17 -2 V
16 -3 V
16 -3 V
17 -3 V
16 -3 V
16 -2 V
17 -3 V
16 -3 V
17 -3 V
16 -2 V
16 -3 V
17 -2 V
16 -3 V
16 -3 V
17 -2 V
16 -3 V
1249 1231 M
17 -3 V
16 -1 V
17 0 V
16 3 V
16 5 V
17 6 V
16 6 V
16 6 V
17 3 V
16 0 V
16 -2 V
17 -5 V
16 -7 V
17 -9 V
16 -10 V
16 -10 V
17 -9 V
16 -9 V
16 -8 V
17 -5 V
16 -4 V
16 -2 V
17 0 V
16 1 V
16 4 V
17 5 V
16 5 V
17 7 V
16 7 V
16 8 V
17 7 V
16 7 V
16 7 V
17 6 V
16 5 V
16 4 V
17 4 V
16 2 V
16 1 V
17 1 V
16 -1 V
17 -2 V
16 -2 V
16 -4 V
17 -4 V
16 -5 V
16 -5 V
17 -6 V
16 -7 V
16 -6 V
17 -7 V
16 -8 V
16 -7 V
17 -7 V
16 -8 V
17 -7 V
16 -8 V
16 -7 V
17 -7 V
16 -7 V
16 -7 V
17 -7 V
16 -6 V
16 -7 V
17 -6 V
16 -6 V
16 -5 V
17 -6 V
16 -5 V
17 -6 V
16 -5 V
16 -4 V
17 -5 V
16 -4 V
16 -5 V
17 -4 V
16 -4 V
16 -4 V
17 -4 V
16 -3 V
17 -4 V
16 -4 V
16 -3 V
17 -3 V
16 -3 V
16 -4 V
17 -3 V
16 -3 V
16 -3 V
17 -3 V
16 -3 V
16 -3 V
17 -2 V
16 -3 V
17 -3 V
16 -3 V
16 -3 V
currentpoint stroke M
17 -2 V
16 -3 V
1353 1279 M
17 -2 V
16 -1 V
16 2 V
17 5 V
16 9 V
17 11 V
16 10 V
16 9 V
17 6 V
16 2 V
16 -2 V
17 -6 V
16 -9 V
16 -12 V
17 -14 V
16 -14 V
16 -13 V
17 -12 V
16 -10 V
17 -7 V
16 -5 V
16 -2 V
17 1 V
16 4 V
16 7 V
17 8 V
16 10 V
16 12 V
17 12 V
16 12 V
16 13 V
17 12 V
16 11 V
17 10 V
16 9 V
16 8 V
17 6 V
16 5 V
16 3 V
17 2 V
16 0 V
16 -1 V
17 -3 V
16 -4 V
16 -5 V
17 -6 V
16 -6 V
17 -8 V
16 -8 V
16 -9 V
17 -9 V
16 -10 V
16 -10 V
17 -10 V
16 -10 V
16 -9 V
17 -10 V
16 -10 V
17 -10 V
16 -9 V
16 -9 V
17 -9 V
16 -8 V
16 -9 V
17 -8 V
16 -7 V
16 -8 V
17 -7 V
16 -7 V
16 -6 V
17 -6 V
16 -6 V
17 -6 V
16 -5 V
16 -6 V
17 -5 V
16 -4 V
16 -5 V
17 -5 V
16 -4 V
16 -4 V
17 -4 V
16 -4 V
16 -3 V
17 -4 V
16 -4 V
17 -3 V
16 -3 V
16 -4 V
17 -3 V
16 -3 V
16 -3 V
17 -3 V
16 -3 V
16 -3 V
17 -3 V
16 -2 V
16 -3 V
17 -3 V
1457 1327 M
17 -2 V
16 -1 V
16 3 V
17 6 V
16 11 V
16 12 V
17 12 V
16 11 V
16 7 V
17 2 V
16 -2 V
17 -6 V
16 -10 V
16 -13 V
17 -15 V
16 -16 V
16 -14 V
17 -14 V
16 -11 V
16 -8 V
17 -5 V
16 -1 V
17 1 V
16 5 V
16 8 V
17 10 V
16 11 V
16 14 V
17 14 V
16 14 V
16 15 V
17 14 V
16 13 V
16 12 V
17 10 V
16 9 V
17 8 V
16 6 V
16 3 V
17 3 V
16 0 V
16 -1 V
17 -3 V
16 -4 V
16 -5 V
17 -6 V
16 -8 V
16 -8 V
17 -9 V
16 -10 V
17 -10 V
16 -10 V
16 -11 V
17 -11 V
16 -11 V
16 -11 V
17 -11 V
16 -10 V
16 -11 V
17 -10 V
16 -10 V
16 -10 V
17 -9 V
16 -9 V
17 -9 V
16 -8 V
16 -8 V
17 -8 V
16 -7 V
16 -7 V
17 -7 V
16 -6 V
16 -6 V
17 -6 V
16 -6 V
16 -5 V
17 -5 V
16 -5 V
17 -5 V
16 -4 V
16 -5 V
17 -4 V
16 -4 V
16 -4 V
17 -4 V
16 -3 V
16 -4 V
17 -3 V
16 -3 V
17 -4 V
16 -3 V
16 -3 V
17 -3 V
16 -3 V
16 -3 V
17 -3 V
16 -3 V
16 -3 V
17 -2 V
522 893 M
10 4 V
9 4 V
10 5 V
9 4 V
9 5 V
10 4 V
9 4 V
10 5 V
9 4 V
10 5 V
9 4 V
10 4 V
9 5 V
9 4 V
10 5 V
9 4 V
10 4 V
9 5 V
10 4 V
9 5 V
10 4 V
9 4 V
9 5 V
10 4 V
9 4 V
10 5 V
9 4 V
10 5 V
9 4 V
10 4 V
9 5 V
9 4 V
10 5 V
9 4 V
10 4 V
9 5 V
10 4 V
9 5 V
10 4 V
9 4 V
9 5 V
10 4 V
9 5 V
10 4 V
9 4 V
10 5 V
9 4 V
10 4 V
9 5 V
9 4 V
10 5 V
9 4 V
10 4 V
9 5 V
10 4 V
9 5 V
10 4 V
9 4 V
9 5 V
10 4 V
9 5 V
10 4 V
9 4 V
10 5 V
9 4 V
10 5 V
9 4 V
9 4 V
10 5 V
9 4 V
10 4 V
9 5 V
10 4 V
9 5 V
10 4 V
9 4 V
9 5 V
10 4 V
9 5 V
10 4 V
9 4 V
10 5 V
9 4 V
10 5 V
9 4 V
9 4 V
10 5 V
9 4 V
10 5 V
9 4 V
10 4 V
9 5 V
10 4 V
9 5 V
9 4 V
10 4 V
9 5 V
10 4 V
9 4 V
702 952 M
10 4 V
9 4 V
10 4 V
9 4 V
9 3 V
10 4 V
9 3 V
10 3 V
9 3 V
10 3 V
9 3 V
10 2 V
9 3 V
9 2 V
10 2 V
9 2 V
10 2 V
9 2 V
10 2 V
9 2 V
10 2 V
9 1 V
9 2 V
10 2 V
9 1 V
10 2 V
9 2 V
10 1 V
9 2 V
10 2 V
9 2 V
9 1 V
10 2 V
9 2 V
10 3 V
9 2 V
10 2 V
9 3 V
10 2 V
9 3 V
9 3 V
10 3 V
9 3 V
10 4 V
9 3 V
10 4 V
9 4 V
10 4 V
9 4 V
9 4 V
10 5 V
9 5 V
10 5 V
9 5 V
10 5 V
9 5 V
10 6 V
9 6 V
9 6 V
10 6 V
9 6 V
10 6 V
9 6 V
10 7 V
9 7 V
10 6 V
9 7 V
9 7 V
10 7 V
9 7 V
10 7 V
9 7 V
10 7 V
9 8 V
10 7 V
9 7 V
9 7 V
10 7 V
9 7 V
10 7 V
9 7 V
10 7 V
9 7 V
10 6 V
9 7 V
9 6 V
10 7 V
9 6 V
10 6 V
9 6 V
10 6 V
9 5 V
10 6 V
9 5 V
9 5 V
10 5 V
9 5 V
currentpoint stroke M
10 5 V
9 4 V
882 838 M
10 4 V
9 5 V
9 4 V
10 5 V
9 4 V
10 4 V
9 5 V
10 4 V
9 4 V
10 5 V
9 4 V
9 5 V
10 4 V
9 4 V
10 5 V
9 4 V
10 4 V
9 5 V
10 4 V
9 4 V
9 5 V
10 4 V
9 5 V
10 4 V
9 4 V
10 5 V
9 4 V
10 4 V
9 5 V
9 4 V
10 4 V
9 5 V
10 4 V
9 4 V
10 5 V
9 4 V
10 5 V
9 4 V
9 4 V
10 5 V
9 4 V
10 4 V
9 5 V
10 4 V
9 5 V
10 4 V
9 4 V
9 5 V
10 4 V
9 5 V
10 4 V
9 4 V
10 5 V
9 4 V
10 4 V
9 5 V
9 4 V
10 5 V
9 4 V
10 5 V
9 4 V
10 4 V
9 5 V
10 4 V
9 5 V
9 4 V
10 4 V
9 5 V
10 4 V
9 5 V
10 4 V
9 5 V
10 4 V
9 4 V
9 5 V
10 4 V
9 5 V
10 4 V
9 5 V
10 4 V
9 4 V
10 5 V
9 4 V
9 5 V
10 4 V
9 4 V
10 5 V
9 4 V
10 5 V
9 4 V
10 5 V
9 4 V
9 4 V
10 5 V
9 4 V
10 5 V
9 4 V
10 4 V
9 5 V
1062 957 M
10 4 V
9 4 V
9 4 V
10 3 V
9 3 V
10 3 V
9 2 V
10 3 V
9 2 V
10 1 V
9 2 V
9 1 V
10 1 V
9 1 V
10 1 V
9 0 V
10 1 V
9 0 V
10 0 V
9 0 V
9 0 V
10 -1 V
9 0 V
10 0 V
9 0 V
10 -1 V
9 0 V
10 0 V
9 0 V
9 0 V
10 0 V
9 0 V
10 0 V
9 0 V
10 1 V
9 1 V
10 1 V
9 1 V
9 1 V
10 2 V
9 2 V
10 2 V
9 2 V
10 3 V
9 3 V
10 3 V
9 3 V
9 4 V
10 4 V
9 5 V
10 4 V
9 5 V
10 6 V
9 5 V
10 6 V
9 6 V
9 7 V
10 6 V
9 7 V
10 7 V
9 8 V
10 7 V
9 8 V
10 8 V
9 8 V
9 9 V
10 8 V
9 9 V
10 9 V
9 9 V
10 9 V
9 9 V
10 9 V
9 9 V
9 9 V
10 9 V
9 9 V
10 9 V
9 9 V
10 9 V
9 8 V
10 9 V
9 8 V
9 9 V
10 8 V
9 8 V
10 8 V
9 7 V
10 7 V
9 7 V
10 7 V
9 7 V
9 6 V
10 6 V
9 6 V
10 5 V
9 5 V
10 5 V
9 5 V
1242 1001 M
9 4 V
10 4 V
9 3 V
10 3 V
9 2 V
10 2 V
9 2 V
10 1 V
9 1 V
9 0 V
10 0 V
9 0 V
10 -1 V
9 0 V
10 -2 V
9 -1 V
10 -2 V
9 -1 V
9 -2 V
10 -3 V
9 -2 V
10 -2 V
9 -3 V
10 -2 V
9 -3 V
10 -3 V
9 -2 V
9 -3 V
10 -2 V
9 -2 V
10 -2 V
9 -2 V
10 -2 V
9 -2 V
10 -1 V
9 -1 V
9 -1 V
10 0 V
9 0 V
10 0 V
9 1 V
10 1 V
9 1 V
10 2 V
9 2 V
9 3 V
10 3 V
9 3 V
10 4 V
9 5 V
10 5 V
9 5 V
10 6 V
9 6 V
9 6 V
10 7 V
9 8 V
10 7 V
9 9 V
10 8 V
9 9 V
10 9 V
9 10 V
9 9 V
10 11 V
9 10 V
10 10 V
9 11 V
10 11 V
9 11 V
10 11 V
9 11 V
9 12 V
10 11 V
9 11 V
10 12 V
9 11 V
10 11 V
9 11 V
10 11 V
9 11 V
9 11 V
10 10 V
9 10 V
10 10 V
9 10 V
10 9 V
9 9 V
10 9 V
9 8 V
9 8 V
10 8 V
9 7 V
10 7 V
9 6 V
10 6 V
9 6 V
10 5 V
9 4 V
1422 901 M
9 5 V
10 4 V
9 3 V
10 4 V
9 3 V
10 2 V
9 3 V
10 2 V
9 2 V
9 2 V
10 1 V
9 2 V
10 1 V
9 0 V
10 1 V
9 1 V
10 0 V
9 0 V
9 0 V
10 0 V
9 0 V
10 0 V
9 0 V
10 -1 V
9 0 V
10 0 V
9 -1 V
9 0 V
10 0 V
9 0 V
10 0 V
9 0 V
10 0 V
9 1 V
10 0 V
9 1 V
9 1 V
10 1 V
9 1 V
10 2 V
9 2 V
10 2 V
9 2 V
10 3 V
9 3 V
9 3 V
10 4 V
9 3 V
10 4 V
9 5 V
10 5 V
9 5 V
10 5 V
9 5 V
9 6 V
10 6 V
9 7 V
10 6 V
9 7 V
10 8 V
9 7 V
10 8 V
9 7 V
9 9 V
10 8 V
9 8 V
10 9 V
9 8 V
10 9 V
9 9 V
10 9 V
9 9 V
9 9 V
10 9 V
9 9 V
10 9 V
9 9 V
10 9 V
9 9 V
10 9 V
9 9 V
9 8 V
10 9 V
9 8 V
10 8 V
9 8 V
10 8 V
9 7 V
10 8 V
9 7 V
9 7 V
10 6 V
9 6 V
10 6 V
9 6 V
10 6 V
9 5 V
currentpoint stroke M
10 5 V
9 4 V
1602 793 M
9 5 V
10 4 V
9 4 V
10 4 V
9 3 V
10 4 V
9 4 V
9 3 V
10 3 V
9 3 V
10 4 V
9 3 V
10 2 V
9 3 V
10 3 V
9 2 V
9 3 V
10 2 V
9 3 V
10 2 V
9 2 V
10 3 V
9 2 V
10 2 V
9 3 V
9 2 V
10 2 V
9 2 V
10 3 V
9 2 V
10 2 V
9 3 V
10 2 V
9 3 V
9 2 V
10 3 V
9 3 V
10 3 V
9 3 V
10 3 V
9 3 V
10 4 V
9 3 V
9 4 V
10 3 V
9 4 V
10 4 V
9 4 V
10 5 V
9 4 V
10 4 V
9 5 V
9 5 V
10 5 V
9 5 V
10 5 V
9 5 V
10 6 V
9 5 V
10 6 V
9 6 V
9 6 V
10 6 V
9 6 V
10 6 V
9 6 V
10 6 V
9 7 V
10 6 V
9 7 V
9 6 V
10 7 V
9 6 V
10 7 V
9 6 V
10 7 V
9 6 V
10 7 V
9 6 V
9 7 V
10 6 V
9 6 V
10 7 V
9 6 V
10 6 V
9 6 V
10 6 V
9 6 V
9 5 V
10 6 V
9 6 V
10 5 V
9 5 V
10 5 V
9 5 V
10 5 V
9 5 V
9 4 V
10 5 V
1782 722 M
9 4 V
10 5 V
9 4 V
10 4 V
9 4 V
10 4 V
9 5 V
9 4 V
10 4 V
9 4 V
10 3 V
9 4 V
10 4 V
9 4 V
10 4 V
9 3 V
9 4 V
10 4 V
9 4 V
10 3 V
9 4 V
10 3 V
9 4 V
10 4 V
9 3 V
9 4 V
10 4 V
9 3 V
10 4 V
9 3 V
10 4 V
9 4 V
10 4 V
9 3 V
9 4 V
10 4 V
9 4 V
10 3 V
9 4 V
10 4 V
9 4 V
10 4 V
9 4 V
9 4 V
10 5 V
9 4 V
10 4 V
9 4 V
10 5 V
9 4 V
10 5 V
9 4 V
9 5 V
10 4 V
9 5 V
10 4 V
9 5 V
10 5 V
9 5 V
10 5 V
9 4 V
9 5 V
10 5 V
9 5 V
10 5 V
9 5 V
10 6 V
9 5 V
10 5 V
9 5 V
9 5 V
10 5 V
9 5 V
10 6 V
9 5 V
10 5 V
9 5 V
10 5 V
9 5 V
9 5 V
10 6 V
9 5 V
10 5 V
9 5 V
10 5 V
9 5 V
10 5 V
9 5 V
9 4 V
10 5 V
9 5 V
10 5 V
9 5 V
10 4 V
9 5 V
10 4 V
9 5 V
9 4 V
10 5 V
1962 677 M
9 5 V
10 4 V
9 4 V
10 5 V
9 4 V
9 4 V
10 5 V
9 4 V
10 4 V
9 4 V
10 5 V
9 4 V
10 4 V
9 4 V
9 4 V
10 5 V
9 4 V
10 4 V
9 4 V
10 4 V
9 4 V
10 5 V
9 4 V
9 4 V
10 4 V
9 4 V
10 4 V
9 5 V
10 4 V
9 4 V
10 4 V
9 4 V
9 4 V
10 5 V
9 4 V
10 4 V
9 4 V
10 4 V
9 5 V
10 4 V
9 4 V
9 4 V
10 5 V
9 4 V
10 4 V
9 5 V
10 4 V
9 4 V
10 5 V
9 4 V
9 5 V
10 4 V
9 4 V
10 5 V
9 4 V
10 5 V
9 4 V
10 5 V
9 4 V
9 5 V
10 4 V
9 5 V
10 5 V
9 4 V
10 5 V
9 4 V
10 5 V
9 5 V
9 4 V
10 5 V
9 4 V
10 5 V
9 5 V
10 4 V
9 5 V
10 5 V
9 4 V
9 5 V
10 4 V
9 5 V
10 5 V
9 4 V
10 5 V
9 4 V
10 5 V
9 5 V
9 4 V
10 5 V
9 4 V
10 5 V
9 4 V
10 5 V
9 4 V
10 5 V
9 4 V
9 5 V
10 4 V
9 4 V
10 5 V
2142 644 M
9 4 V
10 5 V
9 4 V
10 4 V
9 5 V
9 4 V
10 4 V
9 5 V
10 4 V
9 4 V
10 5 V
9 4 V
10 5 V
9 4 V
9 4 V
10 5 V
9 4 V
10 4 V
9 5 V
10 4 V
9 4 V
10 5 V
9 4 V
9 4 V
10 5 V
9 4 V
10 4 V
9 4 V
10 5 V
9 4 V
10 4 V
9 5 V
9 4 V
10 4 V
9 5 V
10 4 V
9 4 V
10 5 V
9 4 V
10 5 V
9 4 V
9 4 V
10 5 V
9 4 V
10 4 V
9 5 V
10 4 V
9 5 V
10 4 V
9 4 V
9 5 V
10 4 V
9 4 V
10 5 V
9 4 V
10 5 V
9 4 V
10 5 V
9 4 V
9 4 V
10 5 V
9 4 V
10 5 V
9 4 V
10 5 V
9 4 V
10 5 V
9 4 V
9 4 V
10 5 V
9 4 V
10 5 V
9 4 V
10 5 V
9 4 V
10 5 V
9 4 V
9 5 V
10 4 V
9 4 V
10 5 V
9 4 V
10 5 V
9 4 V
10 5 V
9 4 V
9 5 V
10 4 V
9 4 V
10 5 V
9 4 V
10 5 V
9 4 V
10 5 V
9 4 V
9 4 V
10 5 V
currentpoint stroke M
9 4 V
10 5 V
1.000 UL
LTb
3077 787 M
2142 352 L
522 603 M
2142 352 L
522 603 M
935 435 V
3077 787 M
1457 1038 L
522 603 M
0 869 V
935 -434 R
0 289 V
3077 787 M
0 292 V
2142 352 M
0 292 V
522 603 M
57 26 V
878 409 R
-58 -27 V
811 558 M
57 26 V
878 409 R
-58 -27 V
1101 513 M
57 26 V
878 409 R
-58 -27 V
1390 468 M
57 26 V
878 409 R
-58 -27 V
1679 424 M
57 26 V
878 408 R
-58 -27 V
1968 379 M
57 26 V
878 409 R
-58 -27 V
2142 352 M
-63 9 V
522 603 M
62 -10 V
2291 421 M
-63 9 V
671 672 M
62 -10 V
2439 490 M
-63 9 V
820 741 M
62 -10 V
2588 559 M
-63 9 V
969 810 M
62 -10 V
2737 629 M
-63 9 V
1118 880 M
62 -10 V
2886 698 M
-63 9 V
1266 949 M
62 -10 V
3035 767 M
-63 9 V
1415 1018 M
62 -10 V
522 893 M
63 0 V
-63 145 R
63 0 V
-63 145 R
63 0 V
-63 144 R
63 0 V
-63 145 R
63 0 V
stroke
grestore
end
showpage
}}%
\put(522,1690){\makebox(0,0){$a_0P(r,	\theta)$}}%
\put(396,1472){\makebox(0,0)[r]{0.2}}%
\put(396,1327){\makebox(0,0)[r]{0.15}}%
\put(396,1183){\makebox(0,0)[r]{0.1}}%
\put(396,1038){\makebox(0,0)[r]{0.05}}%
\put(396,893){\makebox(0,0)[r]{0}}%
\put(3014,507){\makebox(0,0){$	\theta$}}%
\put(3084,751){\makebox(0,0)[l]{3}}%
\put(2935,682){\makebox(0,0)[l]{2.5}}%
\put(2786,613){\makebox(0,0)[l]{2}}%
\put(2637,543){\makebox(0,0)[l]{1.5}}%
\put(2488,474){\makebox(0,0)[l]{1}}%
\put(2340,405){\makebox(0,0)[l]{0.5}}%
\put(2191,336){\makebox(0,0)[l]{0}}%
\put(1098,369){\makebox(0,0){$r/a_0$}}%
\put(1922,336){\makebox(0,0)[r]{25}}%
\put(1633,381){\makebox(0,0)[r]{20}}%
\put(1344,425){\makebox(0,0)[r]{15}}%
\put(1055,470){\makebox(0,0)[r]{10}}%
\put(765,515){\makebox(0,0)[r]{5}}%
\put(476,560){\makebox(0,0)[r]{0}}%
\put(3074,1796){\makebox(0,0)[r]{$n=3$ $l=1$ $m_l=0$}}%
\end{picture}%
\endgroup
\endinput

\caption{Romlig fordeling av $a_0P(r,\theta)$ for den andre eksiterte tilstand med $l=1$ i hydrogenatomet med $m_l=0$.}
\end{center}
\end{figure}
\begin{figure}[h]
   \setlength{\unitlength}{1mm}
   \begin{picture}(100,100)
   \put(-60,-120){\epsfxsize=25cm \epsfbox{wave7.ps}}
   \end{picture}
\caption{Romlig fordeling av $a_0P(r,\theta)$ for den andre eksiterte tilstand med $l=1$ i hydrogenatomet med $m_l=\pm 1$.\label{48}}
\end{figure}


%\begin{figure}[h]
%   \setlength{\unitlength}{1mm}
%   \begin{picture}(100,100)
%   \put(0,0){\epsfxsize=10cm \epsfbox{hydro2.ps}}
%   \end{picture}
%\end{figure}


\section{Kvantisering av banespinn i to dimensjoner}


Studier av to-dimensjonale systemer er sv\ae rt aktuelle
i faste stoffers fysikk. Nylig, se f.eks.~tidskriftet Physical Review Letters,
bind 84 (2000) s.~2223, klarte forskere ved University of Santa Barbara
og universitetet i M\"unchen \aa\ isolere enkeltelektroner (som et 
kunstig hydrogenatom)  i s\aa kalla kvantepunkter, eller Quantum dots,
som er to-dimensjonale systemer av liten utstrekning, i dette tilfellet
ca.~50 nm, som best\aa r av ett eller flere elektroner. Slike systemer er
bla.~tenkt som kandidater for \aa\ lage en kvantedatamaskin. 
Kvantepunkter er alts\aa\ mikroskopiske omr\aa der 
\begin{figure}[h]
\begin{center}
{\centering
\mbox
{\psfig{figure=st9f1.ps,height=8cm,width=8cm}}
}
\end{center}
\caption{Disse 50 nm i diameter ringene kan hver v\ae re vert for et
enkelt elektron som i g\aa r sirkel i en rein kvantetilstand som kan kontrolleres av et ytre p\aa satt magnetfelt. }
\end{figure}
Vi skal rekne p\aa \ en partikkel som beveger seg fritt p\aa \ en 
sirkel med radius $r$ i $xy$-planet.  Legg merke til at $r$ er konstant.  
Det betyr at hvis vi g\aa r over fra kartesiske koordinater $(x,y)$ til 
polarkoordinater $(r,\phi)$, s\aa \ kan partikkelens bevegelse beskrives 
kun ved hjelp av $\phi$.  Partikkelens hastighet er i polarkoordinater 
gitt ved ${\bf v}=r\frac{d\phi}{dt}{\bf e}_{\phi}\equiv 
r\omega{\bf e}_{\phi}$, der ${\bf e}_{\phi}$ er en enhetsvektor.  
Partikkelens banespinn er da gitt ved 
\[
{\bf L}={\bf r}\times m{\bf v}=mr^{2}\omega {\bf e}_{r}\times{\bf e}_{\phi}
=mr^{2}\omega {\bf e}_{z}
\]
Dermed ser vi at banespinnet er rettet langs $z$-aksen, og har lengden 
\[
L_{z}=mr^{2}\omega=mrv.
\]
Den kinetiske energien blir  
\[
E_{k}=\frac{1}{2}mv^{2}=\frac{1}{2}mr^{2}\omega^{2}.
\]
Fra $L_{z}=mr\omega$ f\aa s $\omega=\frac{L_{z}}{mr^{2}}$, og dermed 
\[
E_{k}=\frac{1}{2}mr^{2}\frac{L_{z}^{2}}{m^{2}r^{4}}=\frac{L_{z}^{2}}
{2mr^{2}}.
\]
Vi skal n\aa \ behandle partikkelen kvantemekanisk.  Det betyr at 
$L_{z}$ m\aa \ bli en operator.  I kartesiske koordinater er den gitt ved 
\[
\hat{L}_{z}=x\hat{p}_{y}-y\hat{p}_{x}=-i\hbar\left(x\frac{\partial}
{\partial y}-y\frac{\partial}{\partial x} \right) .
\]
I polarkoordinater har vi at 
$x=r\cos\phi$, $y=r\sin\phi$, slik at $\phi=\phi(x,y)=\arctan(y/x)$.
Da f\aa r vi at 
\begin{eqnarray}
\frac{\partial}{\partial \phi}&=&\frac{\partial}{\partial x}\frac{\partial x}
{\partial \phi}+\frac{\partial}{\partial y}\frac{\partial y}{\partial \phi}
\nonumber \\
&=&-r\sin\phi\frac{\partial}{\partial x}+r\cos\phi\frac{\partial}{\partial y}
\nonumber \\
&=&x\frac{\partial}{\partial y}-y\frac{\partial}{\partial x}. \nonumber
\end{eqnarray}
Dermed ser vi at i polarkoordinater er $\hat{L}_{z}=-i\hbar\frac{\partial}
{\partial \phi}$. 
Siden partikkelen kan bevege seg fritt, har den bare kinetisk energi.  
Da ser vi fra b) at Hamiltonoperatoren er gitt ved 
\[
\hat{H}=\frac{\hat{L}_{z}^{2}}{2mr^{2}}=-\frac{\hbar^{2}}{2mr^{2}}
\frac{\partial^{2}}{\partial \phi^{2}}.
\]

Et rimelig krav til b\o lgefunksjonen er at den skal v\ae re \'{e}ntydig, 
det vil si at den skal ha samme verdi i et punkt selv om vi beskriver det 
p\aa \ to forskjellige m\aa ter.  Et punkt p\aa \ sirkelen kan angis 
med vinkelen $\phi$, men det kan like gjerne angis ved $\phi+2\pi$.  
Siden $\psi(\phi)$ er sannsynlighetsamplituden for \aa \ finne partikkelen 
i et punkt p\aa \ sirkelen, ville det v\ae re litt rart om $\psi$ skulle 
forandre seg dersom vi byttet ut $\phi$ med $\phi+2\pi$.  Vi forlanger derfor 
at 
\[
\psi(\phi+2\pi)=\psi(\phi).
\]
Vi skal l\o se Schr\"{o}dingerlikningen $\hat{H}\psi=E\psi$, dvs. 
\[
-\frac{\hbar^{2}}{2mr^{2}}\frac{\partial^{2}}{\partial\phi^{2}}\psi(\phi)=
E\psi(\phi), 
\]
m.a.o 
\[
\psi''(\phi)=-k^{2}\psi{\phi}
\]
der $k\equiv\sqrt{2mr^{2}E/\hbar^{2}}$.
Denne likningen har generell l\o sning 
\[
\psi(\phi)=Ae^{ik\phi}+Be^{-ik\phi}.
\]
Krav om \'{e}ntydighet gir 
\[
Ae^{ik(\phi+2\pi)}+Be^{-ik(\phi+2\pi)}=Ae^{ik\phi}+Be^{-ik\phi}, 
\]
som gir at 
\[
e^{i2\pi k}=1\Rightarrow k=n,\;n=0,\pm 1,\pm 2,\pm 3,\ldots\;.
\]
Merk at vi kan ha $n=0$, da dette gir $\psi={\rm konstant}$, 
som er en akseptabel l\o sning av Schr\"{o}dingerlikningen.  
Siden $k=\sqrt{2mr^{2}E/\hbar^{2}}=n$, blir de kvantiserte energiniv\aa ene 
\[
E_{n}=\frac{\hbar^{2}n^{2}}{2mr^{2}}.
\]
N\aa \ kan vi imidlertid sjonglere litt med denne l\o sningen.  
Vi ser nemlig at b\aa de $e^{ik\phi}$ og $e^{-ik\phi}$ er l\o sninger, 
og dermed egenfunksjoner for $\hat{H}$.  Men vi ser ogs\aa \ at 
\[
\left[\hat{L}_{z},\hat{H}\right]=0,
\]
siden $\hat{H}$ er proporsjonal med $\hat{L}_{z}^{2}$.  Da vet vi at 
vi kan finne simultane egentilstander for $\hat{L}_{z}$ og $\hat{H}$.  
Men:
\begin{eqnarray}
\hat{L}_{z}(Ae^{ik\phi}+Be^{-ik\phi})&=&-i\hbar\frac{\partial}{\partial\phi}
(Ae^{ik\phi}+Be^{-ik\phi})\nonumber \\
&=&\hbar k (Ae^{ik\phi}-Be^{-ik\phi})\neq {\rm konst}\times \psi(\phi)
\nonumber
\end{eqnarray}
Dermed er $\psi$ p\aa \ den generelle formen ikke en egenfunksjon for 
$\hat{L}_{z}$.  Derimot ser vi at $Ae^{ik\phi}$ og $Be^{-ik\phi}$ 
hver for seg er egenfunksjoner for $\hat{L}_{z}$.  Det er da ingenting 
i hele verden som kan hindre oss i \aa \ velge som egenfunksjoner 
\[
\psi_{n}(\phi)=e^{ik\phi}=e^{in\phi},\;n=0,\pm 1, \pm 2, \ldots\;.
\]
Disse er da egenfunksjoner for b\aa de $\hat{H}$ og $\hat{L}_{z}$.  
Siden $\hat{L}_{z}\psi_{n}(\phi)=n\hbar\psi_{n}(\phi)$, ser vi 
at tilstandene $\psi_{n}$ og $\psi_{-n}$ har forskjellig angul\ae rmoment.  
Imidlertid er energien bare avhengig av $n^{2}$, s\aa \ $\psi_{n}$ og 
$\psi_{-n}$ er degenererte.  Konklusjon:  
\[
E_{n}=\frac{\hbar^{2}n^{2}}{2mr^{2}},\;\psi_{n}(\phi)=Ae^{in\phi},\;n=0,\pm 1, 
\pm 2,\ldots \;.
\]    


\section{Oppgaver}
\subsection{Analytiske oppgaver}
\subsubsection*{Oppgave 7.1, Eksamen H-1997}
\begin{itemize}
\item[a)] Gi en kort oversikt over de krav man stiller
til den kvantemekaniske b{\o}lgefunksjonen $\psi(x)$,
spesielt hva som kreves  av b{\o}lgefunksjonen
i en overgang
hvor den potensielle energien $V(x)$ g{\aa}r fra en
endelig verdi til  $V(x) = \infty$.
\end{itemize}
%
I det f{\o}lgende skal vi f{\o}rst se p{\aa} et \'{e}n--dimensjonalt
problem: En partikkel med masse $m$ beveger seg i et
potensialfelt  gitt ved
%
\[
V(x) = \left \{
\begin{array}{c}
   +\infty  \\
   -V_0\\
      0
\end{array}
%
\;\;\; \mbox{for} \;\;\;
%
\begin{array}{r}
 x < 0 \\
 0 \leq x < a \\
 x \geq  a
\end{array}
%
\right .
%
\]
%
med $V_0 > 0$.
%
\begin{itemize}
%
\item[b)] Tegn en skisse som viser de karakteristiske
trekk ved b{\o}lgefunksjonen $\psi_1(x)$ for grunn\-til\-standen
og b{\o}lgefunksjonen $\psi_2(x)$ for den f{\o}rste eksiterte tilstanden
for det tilfelle at begge tilstandene er bundne.
%
\item[c)] I det f{\o}lgende skal bredden $a$ av den potensielle
energien ha en slik verdi $a_1$ at energien  for grunntilstanden
blir $E_1 = -( 1 / 2 ) V_0$.\\
Vis at
\[
a_1 = \frac{3 \pi \hbar}{4 \sqrt{m V_0}} .
\]
%
\item[d)] Vis at det ikke finnes bundne eksiterte tilstander
n{\aa}r den potensielle energien har bredden $a_1$.
%
\end{itemize}

Vi skal n{\aa} behandle et tilsvarende problem i tre dimensjoner.
I dette tilfelle er den potensielle energien  gitt ved
%
\[
V(x) = \left \{
\begin{array}{c}
  - V_{0}\\
     0
\end{array}
%
\;\;\; \mbox{for} \;\;\;
%
\begin{array}{r}
 0 \leq x < a_1 \\
 x \geq  a_1
\end{array}
%
\right .
%
\]
%
hvor $a_1$ er den samme konstanten som ble funnet i punkt c).
Til hjelp i beregningen oppgir vi operatoren $\OP{\bigtriangledown}^2$
i polarkoordinater
%
\[
\OP{\bigtriangledown}^2 = \frac{1}{r} \frac{\partial^2}{\partial r^2} r
							 - \frac{1}{\hbar^2 r^2} \OP{L}^2.
\]
%
Her er $\OP{L}^2$ kvadratet av operatoren for banespinnet.
%
\begin{itemize}
%
\item[e)] Skriv opp Schr\"{o}dingerligningen for dette tilfelle.
Vis at energi egenfunksjonene kan skrives p{\aa} formen
$(1 / r) u(r) f(\theta, \phi)$
og forklar hvorfor de samtidig er egenfunksjoner
$\OP{L}^2$.
Vis at ligningen for $u(r)$ er gitt ved
%
\[
- \frac{\hbar^2}{2 m} \frac{d^2u(r)}{dr^2} + V_{eff}(r) u(r) = E u(r).
\]
%
Finn uttrykket for $V_{eff}(r)$.
%
\item[f)] Finn de mulige energi egenverdiene som har $l = 0$
(s--tilstandene).
%
\item[g)] Tegn kurven for $V_{eff}(r)$  for $l \neq 0$ og forklar
hvorfor det ikke finnes bundne tilstander n{\aa}r
kvantetallet $l$ for banespinnet
er st{\o}rre enn 2.
%
\end{itemize}
\subsubsection*{Kort fasit}
\begin{itemize}
%
\item[a)]  Se l{\ae}reboka {\sl Brehm and Mullin: Introduction to the
           structure of matter}, avsnitt 5.1, 5.2 
%
\item[b)] For $0 \leq x \leq a: \psi_1(x) = A \sin (kx), k =
\sqrt{mV_0/\hbar^2}$.\\
 For $a < x : \psi_2(x) = B \exp (-\beta x), \beta = k$
%
\item[c)] Svaret er oppgitt i oppgaven.
%
\item[d)] B�lgefunksjonen for en mulig eksitert bunden tilstand  med
energi E er $\psi(x) \sim \sin (k^{'}x), k^{'} = \sqrt{2m(E +
V_0)/\hbar^2}$.
Siden $E < 0$ m� $k^{'} <  \sqrt{2mV_0)/\hbar^2}$ og argumentet i
b�lgefunksjonen
i punktet $x = a_1$ blir $k^{'} a_1 = 3\pi/4 \sqrt{2} \approx 1,06
\pi$. Fra figuren fremg�r det at vi m� ha $k^{'}a_1 > 3\pi/2$. F�lgelig ingen
eksitert tilstand.
%
\item[e)] Siden den potensielle energi kun avhenger av $r$ kan alltid
b�lgefunksjonen skrives som et produkt som angitt i oppgaven.
Ligninger for $u(r)$ er som angitt i oppgaven med
%
\begin{eqnarray*}
V_{eff} &=& \frac{l(l+1)\hbar^2}{2m r^2} - V_0 \quad 0\leq r \leq a_1\\  
V_{eff} &=& \frac{l(l+1)\hbar^2}{2m r^2}\quad a_1 < r   
\end{eqnarray*}
%
\item[f)] $ l = 0$ gir samme energi egenverdiligning og med samme
randbetingelser som i det \'{e}n--dimensjonale tilfellet og med samme
egenverdi $-1/2 V_0$.
%
\item[g)] For at vi skal f� bundne tilstander m�
$V_{eff}(r = a_1) = \frac{l(l+1)\hbar^2}{2m a_1^2} - V_0 < 0$. For 
$ l < 2$ er dette oppfylt.
%
\end{itemize}

\subsubsection*{Oppgave 7.2}
Et elektron beveger seg i et to--dimensjonalt, uendelig dypt kassepotensial
$V(x, y)$ med utstrekning $L \times L$, dvs. gitt ved
%
\[
\begin{array}{rll}
V(x, y) =& 0 \hspace{1cm}  &for \; \mid x\mid < L/2 \;
\;\;\; og \;\;\; \mid y\mid < L / 2, \\
V(x, y) =& \infty \; &ellers.
\end{array}
\]
%
\begin{itemize}
%
\item[a)] Skriv ned Hamilton operatoren for systemet og finn
energi egenverdiene
og energi egenfunksjonene.

\item[b)] Norm\'{e}r egenfunksjonene.

\item[c)] Hva er degenerasjonen til de tre laveste energiniv\aa ene?

\item[d)] Beregn forventningsverdiene $\langle \OP{x}\rangle$,
$\langle \OP{p}_{x}\rangle$ og $\langle \OP{p}_{y}^{2}\rangle$
n\aa r elektronet befinner seg i grunn\-til\-standen.
%
\end{itemize}
\subsubsection*{Oppgave 7.3}
En partikkel med masse $m$ beveger seg fritt p{\aa} en sirkel
med radius $r$ i xy--planet.
%
\begin{itemize}
%
\item[a)] Hva er det klassiske uttrykket for z--komponenten
av banespinnet $L_{z}$?
(Banespinnet er parallelt med z--aksen  slik at $L = L_{z}$).
%
\item[b)] Hva er partikkelens kinetiske energi uttrykt ved $L_{z}$?
%
\item[c)] Bruk kvantiseringsregelen
  $L_{z} \longrightarrow -i\hbar \frac{\partial}{\partial \phi}$
  og vis at hamiltonoperatoren for  systemet er
\[
  \OP{H} = - \frac{\hbar^2}{2 m r^2} \frac{\partial^2}{\partial \phi^2}.
\]
%
\item[d)] Hvilke krav m{\aa} stilles til partikkelens b{\o}lgefunksjon
$\psi(\phi)$? Finn $\psi(\phi)$ og de kvantiserte energi egenverdiene
for systemet.
%
\item[e)] Vi antar n{\aa} at partikkelen er et elektron, men vi ser
bort fra egenspinnet. Et homogent magnetfelt parallelt med
z--aksen -- $ \vec{B} = B \vec{e}_z $ -- settes p{\aa}.
Hva blir n{\aa} $\psi(\phi)$ og de kvantiserte energi egenverdiene
for systemet?
%
\end{itemize}
%

Vi betrakter n{\aa} en partikkel p{\aa} den samme ringen, men med et
potensial $V(\phi)$ gitt ved
%
\[
V(\phi) = \left \{
\begin{array}{c}
     0\\
  + \infty
\end{array}
%
\;\;\; \mbox{for} \;\;\;
%
\begin{array}{r}
 0 < \phi < \pi  \\
\pi \leq \phi < 2 \pi 
\end{array}
%
\right .
%
\]
%
\begin{itemize}
%
\item[f)] Finn $\psi(\phi)$ og de kvantiserte energi egenverdiene
for systemet.
%
\item[g)] Energi egenverdiene for en partikkel i \'{e}n dimensjon
i en uendelig dyp br{\o}nn med bredde $L$ er gitt ved
\[
E_n = \frac{\hbar^2 \pi^2}{2 m L^2} n^2, \;\;\;\; n = 1,2,3, \cdots
\]
Sammenlign med resultatet fra f) og forklar.
\end{itemize}
%
\subsubsection*{L\o sning}
\begin{itemize}
% 
\item[a)] Det klassiske uttrykk for en partikkels banespinn
er gitt ved $\vec{L} = \vec{r} \times m\vec{v}$. Komponenten 
parallell med z-aksen blir da $L_z = rmv$ 
%
\item[b)] Partikkelen klassiske kinetiske 
energi er gitt ved 
%
\[
T =  \frac{1}{2}m v^2 = \frac{1}{2 m r^2} L_z^2
\]
%
\item[c)] Den kvantemekaniske Hamilton operatoren blir
%
\[ 
\OP{H}_0 = \OP{T} = \frac{1}{2 m r^2} \OP{L}_z 
        = -\frac{\hbar^2}{2 m r^2}\frac{\partial^2}{\partial \phi^2}
\]
% 
\item[d)] Egenverdi ligningen f�r formen
% 
\begin{equation}
-\frac{\hbar^2}{2 m r^2}\frac{\partial^2\psi(\phi)}{\partial \phi^2}
                 = E \psi(\phi)
\end{equation}
% 
B�lgefunksjonen $ \psi(\phi)$ m� v�re \`{e}n-tydig for alle $\phi$. 
Siden partikkelen 
beveger seg p� en sirkel, betyr dette at  
$\psi(\phi) = \psi(\phi + n \pi)$. Med dette kravet til  egenfunksjonene
blir l�sning av lign(1)
% 
\[
 \psi_n(\phi) = A e^{i n \phi} + B e^{-i n \phi}
\]
% 
hvor $n$ er et helt positivt tall. Dette gir energi egenverdiene 
%
\[ 
  E_n = \frac{\hbar^2}{2mr^2} n^2 
\]
%
\item[e)] N�r partikkelen plasseres i et homogent magnetfelt parallelt
med z-aksen, f�r Hamilton operatoren for systemet et ekstra energiledd
% 
\[
\OP{H} = \OP{H}_0 + \frac{e}{2m} B \OP{L}_z
\]
% 
Siden egenfunksjonene $\psi_n(\phi)$ ogs� er egenfunksjon for $\OP{L}_z$
blir de nye energi egenverdiene 
% 
\begin{eqnarray*}
E(B) &=& E_n + \frac{e}{2m} B n \hbar\\ 
     &=& \frac{\hbar^2}{2mr^2} n^2  + \frac{e}{2m} B n \hbar
\end{eqnarray*}
% 
\item[f)] P� grunn av potensialet $V(\phi)$ er b�lgefunksjonen
 $\psi(\phi) = 0$ for $\pi < \phi < 2 \pi$ og 
% 
\begin{eqnarray*}
  \psi_n(\phi = 0)    &=& 0 \; \longrightarrow \;B = -A\\
   \psi_n(\phi = \pi) &=& 0 \; \longrightarrow \;
                            \psi_n(\phi) = 2Ai \sin(n\phi)\\
\end{eqnarray*}
%
Energi egenverdiene blir 
%
\[ 
  E_n = \frac{\hbar^2}{2mr^2} n^2  
\]
% 
med kvantetallet $n$ som et positivt helt tall.
% 
\item[g)] Partikkelen beveger seg langs en halvsirkel $ 0 < \phi < \pi$
          Setter vi lengden av halvsirkelen lik$ L = \pi r$
blir energi egenverdiene
% 
\[
E_n = \frac{\hbar^2}{2mr^2} n^2 = \frac{\hbar^2 \pi^2}{2mL^2} n^2  
\]
%
som er  kvantemekanisk identisk med en partikkel i en uendelig
dypt potensialbr�nn med bredde $L$.
%
\end{itemize}

\subsubsection*{Oppgave 7.4}
De radielle b\o lgefunksjonene for elektronet i hydrog\'{e}n atomet med
hovedkvantetall $n$ = 2 er for s- og p-tilstanden gitt ved henholdsvis
\begin{eqnarray*}
R_{20}(r) &=& N_{20} \left( 2 - \frac{r}{a_{0}} \right)
exp \left( -\frac{r}{2a_{0}} \right)\\
R_{21}(r) &=& N_{21} \frac{r}{a_{0}}  exp \left( -\frac{r}{2a_{0}} \right).
\end{eqnarray*}

%
\begin{itemize}
%
\item[a)] Bestem normeringskonstantene $N_{20}$ og $N_{21}$.
%
\item[b)] For hvilken verdi av $r$ er sannsynligheten for
\aa ~finne elektronet
st\o rst n\aa r hydrog\'{e}n atom\-et antas \aa ~v\ae re i p-tilstanden?
Sammenlign
resultatet med hva som f\o lger fra Bohrs atommodell.

\item[c)] N\aa r atomet er i denne tilstanden, hva er sannsynligheten
for \aa~finne elektronet innenfor \'{e}n Bohr-radius?

\item[d)] Regn ut den den tilsvarende sannsynligheten n\aa r atomet er i
s-tilstanden og sammenlign med resultatet under punkt c). Hvordan kan man
forklare denne forskjellen?
%
\end{itemize}
%
\subsubsection*{L\o sning}
%
B�lgefunksjonen har formen 
$R_{nl}(r) Y_{lm}(\theta,\phi)$. De to komponentene normaliseres hver
for seg
%
\begin{eqnarray*}
\int_{\Omega} Y_{lm}^{\ast}(\theta, \phi) Y_{lm}(\theta, \phi)
\sin \theta d\theta d\phi &=& 1\\
\int_{0}^{\infty} \left(R_{nl}(r) \right )^2 r^2 dr &=& 1
\end{eqnarray*}
%
\begin{itemize}
% 
\item[a)] Normaliseringskonstanten for $n = 2, \; l = 0$ er bestemt av  
%
\begin{eqnarray*}
1 &=& \int_{0}^{\infty} R_{20}^2(r) r^2 dr = 
N_{20}^2 \int_{0}^{\infty} \left ( 2 -\frac{r}{a_0} \right ) ^2
e^{-r/a_0} r^2 dr\\
&=& N_{20}^2  a_0^3 \int_{0}^{\infty} \left ( 2 - t \right )^2 t^2 e^{-t}dt
 = N_{20}^2  a_0^2 \int_{0}^{\infty} \left ( 4 t^2 - 4 t^3 +
t^4 \right ) e^{-t} dt\\
&=& N_{20}^2  a_0^3 \left (4 \Gamma(3) - 4 \Gamma(4) + \Gamma(5)
\right ) = 8  N_{20}^2  a_0^3
\end{eqnarray*}
% 
og $N_{20} = 1/\left (2\sqrt{2 a_{0}^{3}}\right )$.
$\Gamma$--funksjonene er gitt i Rottman side 104.

Normaliseringskonstanten for $n = 2, \; l = 1$ er bestemt av  
%
\begin{eqnarray*}
1 &=& \int_{0}^{\infty} R_{21}^2(r) r^2 dr = 
N_{21}^2 \int_{0}^{\infty} \left (\frac{r}{a_0} \right )^2
e^{-r/a_0} r^2 dr\\
&=& N_{21}^2  a_0^3 \int_{0}^{\infty} t^4 e^{-t} dt
 = 24  N_{21}^2  a_0^3
\end{eqnarray*}
% 
og $N_{21} = 1/\left (2\sqrt{6 a_{0}^{3}} \right )$.
%
\item[b)] Sannsynligheten for � finne partikkelen i omr�det $\langle r,
r + dr \rangle$ er gitt ved 
% 
\[
P_{21}(r) dr = \left (R_{21}(r) \right )^2 r^2 dr = N_{21}^2 \left
( \frac{r}{a_0} \right )^2 e^{-r/a_0} r^2 dr
\]
% 
Maksimum sannsynlighet er gitt ved 
%
\[
\frac{d P_{21}(r)}{dr} = \frac{N_{21}^2}{a_0^2} \left ( 4 r^2-
\frac{r^4}{a_0} \right ) e^{-r/a_0} = 0
\]
%
som gir $r_{21}(maks) = 4 a_0$. Bohrs atommodell gir f�lgende
kvantiseringsbetingelse $r_n = n^2 a_0$ og med samme
resultat $r_{21}^{(Bohr)} = 4 a_0$.
%
\item[c)] Sannsynligheten for � finne partikkelen innenfor en Bohr
radius er gitt ved 
% 
\[ 
\int_0^{a_0} P_{21}(r) r^2 dr 
= N_{21}^2 \int_0^ {a_0}\left
( \frac{r}{a_0} \right )^2 e^{-r/a_0} r^2 dr
= \left ( N_{21}\right )^2 a_0^3
\int_0^1 t^4 e^{-t} dt
\]
%
Dette integralet kan l�ses ved delvis integrasjon 
%
\begin{eqnarray*}
\int_0^1 t^n e^{-t} dt &=& -e^{-1} + n  \int_0^1 t^{n-1} e^{-t} dt\\
&=& n! - \left ( 1 + n + n(n-1) + \dots + n! \right ) e^{-1}
\end{eqnarray*}
%
og
%
\[
\int_0^{a_0} P_{21}(r) r^2 dr = 1 - \frac{65}{24} e^{-1} = 0.0036
\]
%
\item[d)] Den tilsvarende beregning for $n=2, l = 0$ gir 
%
\begin{eqnarray*}
\int_0^{a_0} P_{20}(r) r^2 dr &=& 
N_{21}^2 \int_{0}^{a_0} \left (2 - \frac{r}{a_0} \right )^2
e^{-r/a_0} r^2 dr\\
  &=& \frac{1}{8} \int_0^{1} \left ( 4 t^2 - 4 t^3 + t^4 \right )e^{-t}dt
 = 1 - \frac{21}{8} e^{-1} = 0.034
\end{eqnarray*}
%
Begge tilstandene $n = 2, l= 0$ og $n = 2, l= 1$ har det meste av sin
sannsynlighets tetthet for $r > a_0$, men p--tilstanden, $n = 2, l= 1$
er den 
vesenlig mindre for sm� $r$ p� grunn av
sentrifugalbarrieren $ -l(l+1)/r^2$ i Hamilton operatoren.
%
\end{itemize}


\subsubsection*{Oppgave 7.5, Eksamen V-1993}

Energi egenverdiligningen for et elektron i  et hydrog\'{e}n--atom
er gitt ved
%
\[
\OP{H}_0 \psi_{n l m_{l}}= -E_0 \frac{1}{n^2} \psi_{n l m_{l}},
\]
%
hvor $E_0$ er en konstant.
\begin{itemize}
%
\item[a)] Sett opp Hamilton--operatoren $\OP{H}_0$ og skriv ned hvilke
betingelser kvantetallene $n$, $l$ og $m_l$ m{\aa} oppfylle.
\end{itemize}
%
Kvantetallet $m_l$ tilfredsstiller egenverdi--ligningen
%
\[
\left ( \frac{d^2}{d \phi^2} + m^2_l \right ) \Phi(\phi)=0.
\]
%
\begin{itemize}
%
\item[b)] Finn $\Phi(\phi)$ (normering er un{\o}dvendig)
og vis hvilke betingelser randkravene for $\Phi$
gir p{\aa} $m_l$.
\end{itemize}
%
B{\o}lgefunksjonen for $s$-tilstanden  med $n$ = 2  er gitt ved
%
\[
\psi _{200}= \sqrt{\frac{1}{8 \pi a_0^3}}
\left ( 1 - \frac{\rho}{2} \right ) \exp (- \frac{\rho}{2} ),
\;\;\;\; \rho = \frac{r}{a_0}
\]
%
\begin{itemize}
%
\item[c)] Vis at b{\o}lgefunksjonen er normert.
(Forslag: Bruk
$\int_{0}^{\infty}\rho ^{k}e^{-\rho} d\rho = k! $,
hvor $k$ er et heltall.)
Finn middelverdien (forventningsverdien) for radien
$< r_{2,0,0}>$ og den radius som  elektronet med st{\o}rst
sannsynlighet befinner
seg i. Forklar kort hvorfor det er forskjell i
disse to radiene.

\end{itemize}
%
Anta at Hamilton--operatoren har et ekstra ledd gitt ved
%
\[
\OP{H}_1 = \frac{E_0}{16 \hbar^2} \left (\OP{L}_x^2  +\OP{L}_y^2 \right ).
\]
%
\begin{itemize}
%
\item[d)] Finn egenverdiene for $\OP{H}_0 + \OP{H}_1$.
Tegn alle energiniv{\aa}ene med $ n = 1$ og 2
f{\o}r og etter $\OP{H}_1$ leddet er innf{\o}rt.
Hvilken degenerasjon har niv{\aa}ene i de
to tilfellene?
%
\end{itemize}
\subsubsection*{Kort fasit}
\begin{itemize}
%
\item[a)] Hamilton operatoren har formen
\[
\OP{H}_0 = -\frac{\hbar^2}{2m}\nabla^2 - \frac{e^2}{4\pi
			\epsilon_0}\frac{1}{r}
\]
hvor $n \geq 1$, $ (n=1,2,3,\ldots)$, $l\leq n-1$, $ (l=0,1,2,\ldots,n-1)$\\
og $ \left | m_l \right | \leq l$,
$ (m_l=-l,-l+1,\ldots,-1,0,1\ldots,l-1,l)$.
%
\item[b)] L{\o}sningen er $\Phi(\phi)=Ce^{im_l\phi}$
og randkravet $\Phi(\phi)=\Phi(\phi + 2 \pi)$ gir
$m_l =0, \pm1, \pm 2,\ldots$.
%
\item[c)] Normering
%
\[
N=\int{}{}\psi_{200}^*\psi_{200}
	 \rho^2 a_0 ^3 sin\theta d\theta d \phi d\rho = 1
\]
%
\[
<r>=\frac{a_0}{2}\int_0^{\infty}(\rho^3-\rho^4+\frac{1}{4 }
\rho^5)e^{-\rho} d\rho = 6a_0
\]
For {\aa} finne den radius elektronet med st{\o}rst
sannsynlighet befinner seg i, tar utgangspunkt i
$P=r^2e^{-r/a_0} (2-r/a_0)^2$ og krever $dP / dr = 0$. Det gir
%
\[
r ( 2 - \frac{r}{a_0} ) ( 4 - 6 \frac{r}{a_0} + \frac{r}{a_0}^2 ) = 0.
\]
%
med f{\o}lgende l{\o}sninger: $ r_1 + 0, \;\; r_2 = 2 a_0$,
og $ r_{3,4} = ( 3 \pm \sqrt{5} ) a_0$.
Innsatt i likningen for $P$, finner vi at den $r$ som gir maksimum er
%
\[
r = r_3 = a_0 (3+\sqrt{5})
\]
%
Forskjellen i $r$ skyldes at fordelingen $P(r)$ er skjev (usymmetrisk) med
hale for store $r$.
%
\item[d)] Vi kan omforme ekstraleddet til
$\OP{H}_1  = (E_0/16\hbar^2)(\OP{L}^2-\OP{L}_z^2)$.
Dette gir egenverdiligningen
%
\[
\left ( \OP{H}_0 + \OP{H}_1 \right ) \psi_{nlm_l}
= E_0 \left ( -\frac{1}{n^2} + \frac{1}{16} ( l (l+1) - m_l^2) \right )
	 \psi_{nlm_l}.
\]
%
Dette gir energiniv�er:
%
\begin{itemize}
% 
\item[a)] $n = 0,\; l = 0,\;, m_l = 0,\; \mbox{degenerasjon} = 1$
% 
\item[b)] $n = 1,\; l = 0,\;, m_l = 0,\; \mbox{degenerasjon} = 1$
% 
\item[c)]  $n = 1,\; l = 1,\;, m_l = 0,\; \mbox{degenerasjon} = 1$\\
            $n = 1,\; l = 1,\;, m_l = \pm 1,\; \mbox{degenerasjon} = 2$
\end{itemize}

\end{itemize}

\subsubsection*{Oppgave 7.6, Eksamen V-1994}
Vi skal i denne oppgaver f{\o}rst se p{\aa} et \'{e}n--dimensjonalt
kvantemekanisk problem. B{\o}l\-ge\-funk\-sjonen for grunntilstanden
for en partikkel med masse m  er gitt ved
%
\[
\psi_0(x) = A \exp \left (-\frac{1}{2} a^2 x^2 \right ).
\]
%
hvor $A$ er en normaliseringskonstant.
%
\begin{itemize}
%
\item[a)] Sett opp energi egenverdiligningen for partikkelen
og bestem den potensielle energien $V(x)$ som gir
$\psi_0(x)$ som l{\o}sning for grunntilstanden.
Finn grunntilstandsenergien $E_0$.
%
\item[b)] Finn normaliseringskonstanten $A$.
%
\item[c)] Vis at
\[
\psi_1(x) = \sqrt{2} a x \psi_0(x)
\]
%
ogs{\aa} er l{\o}sning av egenverdiligningen under a).
Finn den tilsvarende energien $E_1$.
%
\item[d)] Beregn $\langle \OP{x} \rangle$, $\langle \OP{p}\rangle$, 
$\langle \OP{x}^2\rangle$ og $\langle\OP{p}^2 \rangle$ for
grunntilstanden $\psi_0(x)$.
%
\item[e)] Uskarpheten for en operator i kvantemekanikken
er definert ved
%
\[
\Delta A = \sqrt{\langle \OP{A}^2 \rangle - \langle \OP{A}\rangle^2}.
\]
%
Bruk dette til {\aa} finne $\Delta x \cdot \Delta p$
for grunntilstanden $\psi_0(x)$. Komment\'{e}r resultatet.
%
\end{itemize}
%
Vi skal g{\aa} over til {\aa} studere problemet ovenfor i tre dimensjoner.
Den potensielle energien $V(x)$ overf{\o}res til $V(r)$, i.e. samme
funksjon, men n{\aa} avhengig av den radielle avstanden $r$.
%
\begin{itemize}
%
\item[f)] Sett opp egenverdiligningen i det tre--dimensjonale
tilfelle og vis at funksjonen
%
\[
\Phi_{n,p,q}(x,y,z) = \psi_n(x) \psi_p(y) \psi_q(z),
\]
er en l{\o}sning hvor $n, p, q = 0, 1, 2, \ldots$.
Angi degenerasjonen for grunntilstanden og den f{\o}rste
eksiterte tilstanden.
%
\item[g)] Egenfunksjonene i det tre--dimensjonale tilfellet vil
samtidig v{\ae}re egenfunksjoner for banespinn operatorene
$\OP{\vec{L}}^2$ og $\OP{L}_z$. Finn verdier for
kvantetallene $l$ og $m_l$ for operatorene
$\OP{\vec{L}}^2$ og $\OP{L}_z$ i grunntilstanden $\Phi_{0,0,0}$.
%
\end{itemize}
\subsubsection*{Kort fasit}
\begin{itemize}
%
\item[a)] $V(x) = (\hbar^2 a^4 / 2 m) x^2$ , $ E_0 = (\hbar^2 a^2 / 2 m)$.
%
\item[b)] $ A = ( a^2 / \pi)^{1/4}$.
%
\item[c)] $ E_1 = 3 E_0$.
%
\item[d)] $ <x> = 0, \;\, <p> = 0, \;\; <x^2> = (1 / 2a^2),\;\;
			  <p^2> = (1/2) \hbar^2 a^2$
%
\item[e)] $\Delta x \cdot \Delta p = (1 / 2) \hbar$
%
\item[f)] Degenerasjonen av grunntilstand er 1, av 1. eksiterte 3.
%
\item[g)] $l = 0, \;\; m_l = 0$.
%
\end{itemize}
\subsubsection*{Oppgave 7.7}
En partikkel med masse $m$ beveger seg i et tre--dimensjonalt
harmonisk oscillator potensial med potensiell energi
$V(r) = (1/2) k r^2$, hvor $k$ er en konstant.
Dette systemet skal vi f{\o}rst studere ved bruk av Bohrs atomteori
%
\begin{itemize}
%
\item[a)] Sett opp  Bohrs kvantiserings betingelse og bruk den til
{\aa} beregne energien og $r^{2}$ for alle tillatte sirkul{\ae}re
baner for partikkelen.
\end{itemize}
%
Vi skal deretter studere det samme problem ved bruk av
Schr\"{o}dingers kvantemekanikk.
%
\begin{itemize}
%
\item[b)] Sett opp systemets Hamilton operator og vis at
%
\[
\psi(\vec{r}) = N \exp (-\alpha r^2)
\]
er en egentilstand for systemet for en
bestemt verdi av $\alpha$ og finn energi egenverdien.
%
\item[c)] Beregn $\langle \OP{r}^2 \rangle$ for systemet 
i tilstanden $\psi(\vec{r})$
og sammenlign resultatet med punkt a).
%
\end{itemize}

\subsection{Numeriske oppgaver}

\subsubsection*{Oppgave 7.8}
Vi skal her studere
hydrogenatomet og vi skal bl.a. se p� den radielle
schr\"odingerlikningen. I den numeriske biten skal vi normere de
b�lgefunksjonene vi har funnet numerisk, og dette krever en del spennende
teknikker i f.eks. Maple. 

Vi skal se p� $1s$-, $2s$- og $2p$-tilstandene i atomet. I motsetning
til numerisk oppgave 2, skal vi ikke lete etter egenverdiene
for energien; de er gitt i oppaven.

\subsubsection*{Analytisk del}
Vi ser p� den radielle schr\"odingerlikningen. Det er vanlig � sette
$u(r)=rR(r)$, og l�se likningen m.h.p. denne:
\begin{equation}
-\frac{\hbar^2}{2m}\frac{\partial^2}{\partial r^2} - 
\left( \frac{ke^2}{r} - \frac{\hbar^2l(l+1)}{2mr^2}  \right)u(r) =
Eu(r).
\label{eq:l1}
\end{equation}
Vi ser at denne likningen er ekvivalent med den endimensjonale
schr\"odingerliknignen, men under et effektivt potensial:
\begin{equation}
V_\textrm{eff} = -\frac{ke^2}{r} + \frac{\hbar^2l(l+1)}{2mr^2}.
\label{eq:l1b}
\end{equation} 

\begin{itemize}
%
\item[a)]
Forklar kort hva symbolene $1s$, $2s$ og $2p$ st�r for, og hvorfor
energiene til $2s$ og $2p$ er like. 
\item[b)]
I den numeriske l�sningen innf�rer vi en dimensjonsl�s variabel
$\rho=r/\beta$, hvor $\beta$ er en eller annen konstant. Vis at vi da
kan omskrive (\ref{eq:l1}) til:
\begin{equation}
 -\frac{1}{2}\frac{\partial^2 u(\rho)}{\partial \rho^2} -
 \frac{mke^2\beta}{\hbar^2\rho}u(\rho) + \frac{l(l+1)}{2\rho^2}u(\rho)
 = \frac{m\beta�}{\hbar^2}Eu(\rho).
\label{eq:l3}
\end{equation}
Vi kan velge oss en gunstig $\beta$ ved � sette
\begin{equation}
\frac{mke^2\beta}{\hbar^2} = 1.
\label{eq:l4}
\end{equation}
Kjenner du igjen hva $\beta$ er for noe? (Hint: Tenk p� Bohr!) Vi
innf�rer deretter en \emph{skalert egenverdi} $\lambda$ ved:
\begin{equation} 
  \lambda = \frac{m\beta^2}{\hbar^2} E. \label{eq:l5} 
\end{equation}
Vis n�, ved innsetting av $\beta$ og den eksakte verdien $E=E_0/n^2$,
der $E_0 = (\frac{ke^2}{\hbar})^2/2$, at
\[ 
  \lambda = -\frac{1}{2n^2}. 
\]

\end{itemize}
\subsubsection*{Numerisk del}
\begin{itemize}
\item[c)]
Den likningen vi skal l�se numerisk er alts� gitt ved
\begin{equation}
-\frac{1}{2}\frac{\partial^2u(\rho)}{\partial \rho^2} -
\frac{u(\rho)}{\rho} + \frac{l(l+1)}{2\rho^2}u(\rho) - \lambda u(\rho)
= 0.
\label{eq:l6}
\end{equation}

Se p� tilstandene $1s$ og $2s$.
Her b�r du ikke velge $\rho=0$ for randbetingelsene, da dette vil gi
en numerisk divergens. (Merk at i likningen s� deler vi p�
$\rho$!) Velg heller en liten verdi, for eksempel $\epsilon=10^{-8}$,
hver gang du skal bruke ``null'' i randbetingelsene.

Fremstill de radielle b�lgefunksjonene grafisk. Forklar
kort hvorfor de er forskjellige. For hvilken verdi av $\rho$ vil
$R(\rho)$ g� mot null? 
\end{itemize}

Neste underoppgave g�r litt i dybden omkring egenverdier og tilh�rende
egenfunksjoner. Vi skal egentlig ``leke litt,'' og i en senere numerisk
oppgave skal vi anvende resultatene herfra direkte i den s�kalte 
\emph{shooting method} for � finne egenverdier numerisk.

Det kan v�re en fordel � studere kapittel 5-6 i l�reboka (Eigenfunctions
and Eigenvalues).

\begin{itemize}
\item[d)]
Vi skal n� unders�ke hva som skjer med b�lgefunksjonen dersom vi velger en
verdi for lambda som \emph{ikke} er p� formen $-1/n^2$. Velg henholdsvis
en verdi som er \emph{litt for h�y} og en som er \emph{litt for lav} i
forhold til en egenverdi, og fremstill de resulterende b�lgefunksjonene du
f�r ved l�sning av (\ref{eq:l6}) grafisk. Komment\'er resultatet. Pr�v
� forklare formen til de ulike b�lgefunksjonene.

\item[e)]
Finn b�lgefunksjonen for tilstanden $2p$. Her kan du f� bruk for at
for sm� verdier av $\rho$ er $u(\rho) \sim \rho^{l+1}$. Dette gir noe
ulike randbetingelser p� $u(\rho)$.

Fremstill sannsynlihetstettheten sammen med det effektive potensialet
grafisk, og forklar hvorfor sannsynligheten for � finne et elektron
n�r kjernen er mindre for $2p$-tilstanden enn for $2s$-tilstanden.
\end{itemize}



\clearemptydoublepage
\chapter{The hydrogen atom}

\begin{quotation}
You know, it would be sufficient to really understand the electron.
{\em Albert Einstein}
\end{quotation}






\section{Elektronets spinn}

I forrige kapittel s\aa\ vi at det frie hydrogenatomet utviste 
en degenerasjon i energien, dvs.~flere tilstander med ulike
kvantetall $m_l$ og $l$ hadde samme energi for samme
verdi av kvantetallet $n$. Denne degenerasjonen skyldes at 
Coulomb potensialet har en sf\ae risk symmetri. I dette
avsnittet skal vi se at n\aa r vi setter p\aa\ et magnetfelt
i en bestemt retning 
opphever vi denne symmetrien og dermed
ogs\aa\ degenerasjonen i energi. Konkret skal vi se p\aa\
den s\aa kalte Zeeman effekten, elektronets egenspinn, spinn-bane
vekselvirkningen og deres innvirkning p\aa\ energispekteret
til hydrogenatomet. Vi avslutter med \aa\ se p\aa\
regler for s\aa kalla elektriske dipoloverganger.

Egenskapene vi utleder her vil v\ae re av betydning for v\aa r forst\aa else
og studier av det periodiske system. 

Kvantetallene vi har for \aa\ beskrive et system er basert p\aa\
symmetrien til vekselvirkningen, og mye av den innsikt vi finner
ved \aa\ studere hydrogenatomet kan overf\o res til andre fagfelt, som 
f.eks.~molekylfysikk,
kjernefysikk og partikkelfysikk. 
  

\subsection{Zeeman effekten}\label{subsec:zeeman}
En ladd partikkel\footnote{Lesehenvisning er kap.~8-1, 8-2 og 8-3,
sidene 375-389.}  med ladning  $q$ 
som beveger i en sirkul\ae r sl\o yfe med radius $r$ og med en hastighet
$v$ setter opp en str\o m $i$ 
\be
    i=\frac{q}{T}=\frac{qv}{2\pi r}.
\ee
En slik str\o m i en sirkul\ae r bane setter igjen opp et magnetisk
dipolmoment ${\bf \mu}_L$, hvis absoluttverdi er gitt ved
\be
   |{\bf \mu}_L|=iA,
\ee
hvor $A$ er sl\o yfas areal. Vi kan uttrykke dette dipolmomentet vha.~banespinnet $L$. Absoluttverdien av banespinnet er gitt ved $|{\bf L}|=mvr$, som innsatt
i uttrykket for ${\bf \mu}_L$ gir
\be
       |{\bf \mu}_L|=iA=\frac{qv}{2\pi r}\pi^2r^2=\frac{qvr}{2},
\ee
eller
\be
   \frac{|{\bf \mu}_L|}{|{\bf L}|}=\frac{qvr}{2mvr}=\frac{q}{2m}.
\ee
Siden elektronet har negativ ladning $q=-e$, finner vi 
\be
   |{\bf \mu}_L|=-\frac{e}{2m}|{\bf L}|,
\ee
eller som vektorer
\be
   {\bf \mu}_L=-\frac{e}{2m}{\bf L}.
\ee

Dersom vi setter p\aa\ et magnetisk felt, f\aa r vi et kraftmoment 
p\aa\ str\o msl\o yfa gitt ved
\be
   {\bf \tau}={\bf \mu}_L\times {\bf B},
\ee
som igjen gir en potensiell energi
\be
   \Delta E=-{\bf \mu}_L{\bf B}.
\ee
Det er vanlig \aa\ introdusere en st\o rrelse som kalles
Bohrmagneton $\mu_B$ definert som
\be
   \mu_B=\frac{e\hbar}{2m},
\ee
hvis verdi er $\mu_B=9.274\times 10^{-24}$ J/T (T st\aa r for Tesla), eller
dersom vi \o nsker \aa\ bruke eV/G, hvor 1T$=10^4$G (hvor G st\aa r for Gauss)
har vi $\mu_B=5.733\times 10^{-9}$ eV/G. 
Vi kan da omskrive bidraget til den potensielle energien $\Delta E$ 
vha.~banespinnet $L$ til
\be
   \Delta E=\frac{e}{2m}{\bf L}{\bf B}=\frac{g_L\mu_B}{\hbar}{\bf L}{\bf B},
\ee
hvor $g_L=1$ og er en faktor som uttrykker forholdet mellom
ladningsfordelingen og massefordelingen. Vi har introdusert den her for \aa\
ha en viss symmetri i likningene n\aa r vi i neste underavsnitt
skal se p\aa\ elektronets egenspinn. 

Det siste uttrykket, med unntak av faktoren $\hbar$, er utleda fra klassisk
fysikk. Det vil si igjen at energien $\Delta E$ kan ta alle mulige verdier
som er tillatt av vinkelen mellom ${\bf L}$ og ${\bf B}$. Vi har da 
et kontinuerlig sett av verdier for $\Delta E$. Vi skal se straks at 
kvantemekanisk er ikke dette tilfelle, kun bestemte verdier for $\Delta E$ er
tillatte. 

Dersom vi setter p\aa\ et magnetisk felt i retning $z$-aksen, $B_z$, 
vil energien
$\Delta E$ ha sin maksimale eller minimale verdi n\aa r banespinnet er gitt
ved $L_z$, dvs.~at vinkelen mellom $B$-feltet og banespinnet er null.

Klassisk f\aa r vi da f\o lgende uttrykk for Hamiltonfunksjonen $H$ 
for et elektron som g\aa r i bane rundt et proton og med et p\aa satt
magnetfelt langs $z$-aksen $B_z$ 
\be
   H_{\mathrm{klassisk}}=\frac{{\bf p}^2}{2m}-\frac{ke^2}{r}+
                         \frac{e}{2m}L_zB_z=
       \frac{{\bf p}^2}{2m}-\frac{ke^2}{r}+\frac{g_L\mu_B}{\hbar}L_zB_z.
\ee
Dersom vi n\aa\ erstatter den klassiske bevegelsesmengde
med den tilsvarende kvantemekaniske operator 
$\OP{{\bf p}}=-i\hbar\nabla$ og den tilsvarende operator 
for banespinnets projeksjon
p\aa\ $z$-asken $\OP{L}_z=-i\hbar\partial/\partial\phi$, finner vi at den 
kvantemekaniske hamiltonfunksjonen er gitt ved
\be
   \OP{H}=-\frac{\hbar^2\nabla^2}{2m}-\frac{ke^2}{r}+
                         \frac{e}{2m}\OP{L}_zB_z.
\ee
Vi merker oss at den nye hamiltonfunksjonen kan skrives som den vi hadde
fra kapittel \ref{chap:banespinnkvantisering},
\be 
   \OP{H}_0=-\frac{\hbar^2\nabla^2}{2m}-\frac{ke^2}{r},
\ee
pluss et ledd som avhenger av banespinnets projeksjon langs
$z$-aksen, dvs.
\be 
   \OP{H}=\OP{H}_0+\frac{e}{2m}\OP{L}_zB_z.
\ee


Siden $B$-feltet er kun en konstant virker det ikke p\aa\
en b\o lgefunksjon.
Da vi har at
\be
   [\OP{H}_0,\OP{L}^2]=[\OP{H}_0,\OP{L}_z]= [\OP{L}^2,\OP{L}_z]=0,
\ee
m\aa\ vi ogs\aa\ ha
\be
      [\OP{H},\OP{H}_0]=[\OP{H},\OP{L}^2]=[\OP{H},\OP{L}_z]=0,
\ee
som igjen betyr at $\OP{H}$, $\OP{H}_0$, $\OP{L}^2$ og $\OP{L}_z$ 
har samme
egenfunksjon $\psi$ som vi fant for det frie hydrogenatomet.
$B$-feltet er konstant, slik at n\aa r vi opererer p\aa\ en b\o lgefunksjon
for hydrogenatomet $\psi(r,\theta,\phi)$
\be
   \psi(r,\theta,\phi)=\psi_{nlm_l}(r,\theta,\phi)=
    R_{nl}(r)Y_{lm_l}(\theta,\phi),
\ee
har vi 
\be
    \frac{e}{2m}B_z\OP{L}_z\psi_{nlm_l}=\frac{e}{2m}B_zm_l\hbar\psi_{nlm_l},
\ee
eller vha.~Bohrmagnetonen finner vi at
\be
    \frac{g_L\mu_B}{\hbar}B_z\OP{L}_z\psi_{nlm_l}=m_l\mu_BB_z\psi_{nlm_l}.
\ee    
Dette gir oss at 
\be 
   \OP{H}\psi_{nlm_l}=\OP{H}_0\psi_{nlm_l}+m_l\mu_BB_z\psi_{nlm_l}.
\ee
Fra  kapittel \ref{chap:banespinnkvantisering} har vi
\[
  \OP{H}_0\psi_{nlm_l}=E_{n}\psi_{nlm_l}
\]
hvor $E_{nl}=E_0/n^2$ med $E_0=-13.6$ eV. Den totale energien blir dermed
\be
   E_{nlm_l}=\frac{E_0}{n^2}+m_l\mu_BB_z
\ee
For $l\ne 0$ ser vi at vi opphever degenerasjonen i energi for de ulike
verdiene av $m_l$.

Ser vi konkret p\aa\ tilfellet for $n=2$ og $l=1$, dvs.~en s\aa kalt 
$2p$ tilstand finner vi f\o lgende tre energieegenverdier n\aa r vi setter
p\aa\ et magnetfelt langs $z$-aksen
\[
  E_{21+1}=\frac{E_0}{4}+\mu_BB_z,
\]
\[
  E_{21-1}=\frac{E_0}{4}-\mu_BB_z,
\]
og 
\[
  E_{210}=\frac{E_0}{4},
\]
som er lik energien for $E_{200}$ slik at for $m_l=0$ er 
$2s$ og $2p$ fremdeles degenererte.     
Denne oppsplittingen av niv\aa ene pga.~et p\aa satt magnetfelt
blei observert allerede i 1896 av den hollandske fysiker Zeeman.
Men forklaringen kom f\o rst med kvantemekanikken og kvantisering
av banespinn. Kun bestemte niv\aa er er tillatt, og ikke et kontinuerlig
sett slik klassisk fysikk gir. Effekten kalles den normale
Zeeman effekten.

Vi skal merke oss her at vi har opphevd degenerasjon i $m_l$ som 
vi s\aa\ i forrige avsnitt for det frie hydrogenatomet. 
I det tilfellet
ga den sf\ae riske symmetrien til Coulomb potensialet at alle $m_l$
for en gitt verdi av kvantetallet $n$ hadde samme energi. 

\subsection{Elektronets spinn og den anomale Zeeeman effekten}

Eksperiment\footnote{Lesehenvisning er kap.~84. og 8-5, sidene 390-397.} 
p\aa\ 20-tallet i forrige \aa rhundre leda til
postulering av elektronets egenspinn, med verdi $s=1/2$ og
kvantisering langs $z$-aksen p\aa\ $m_s\pm 1/2$, noe som svarer
til en degenerasjon p\aa\ $2s+1=2$ dersom vi trekker en parallell til
banespinnet $l$ med degenerasjon i $m_l$ p\aa\ $2l+1$. 

Et elektron har, i tillegg til et banespinn ${\bf L}$, ogs\aa\ et
egenspinn ${\bf S}$ som har en diskret verdi $1/2$. 
Klassisk finnes det ikke noen tilsvarende tilfeller, dog kan vi tenke
oss en analogi til jorda som g\aa r i bane rundt sola. Jorda har dermed et banespinn $L$ i sitt kretsl\o p rundt sola p\aa\ et \aa r. Men i tillegg roterer jorda om sin egen akse, som igjen gir opphav
til dag og natt.
Dette siste rotasjonsspinnet kan vi tenke p\aa\ som en analogi til
elektronets egenspinn. Goudsmit og Uhlenbeck, som postulerte at elektronet 
har et egenspinn, forestilte seg at elektronet kunne tenkes som en liten
kule som roterer om sin egen akse. Problemet med slike analogier burde v\ae re opplagt,
mens vi ikke har noen problemer med \aa\ forestille oss jorda
som tiln\ae rma sf\ae risk, viser eksperiment at 
elektronet har en utstrekning p\aa\
maks $10^{-16}$ m, slik at vi kan betrakte det som n\ae rmest
en punktpartikkel. En kjerne har f.eks~typiske radiuser p\aa\ $10^{-14}$ m,
mens hydrogenatomet, som vi s\aa\ i forrige avsnitt, har en midlere radius 
p\aa\ $3/2$ Bohrradier, ca. 0.08 nm.  

F\o r vi diskuterer eksperimentene som leda til elektronets spinn,
er det noen sider ved teorien vi har utvikla til n\aa\ som kan v\ae re
verd litt ettertanke. 
Dersom vi utelukker egenspinnet som et eget kvantetall, har vi sett at vi
trenger tre kvantetall for \aa\ beskrive elektronet i hydrogenatomet. 
Det skyldes
at vi har tre uavhengige koordinater som trengs for \aa\ beskrive 
elektronets posisjon. Hver frihetsgrad gir opphav til et kvantetall!
Fra et klassisk st\aa sted kan vi dermed si at det er bare rom for tre
kvantetall. Med elektronets egenspinn har vi et
nytt kvantetall. Men elektronet er ikke en klassisk partikkel. 

I Schr\"odingers likning har vi ikke noe problem med \aa\ innpasse
et nytt kvantetall som elektronets egenspinn, men teorien v\aa r
forutsier ikke denne egenskapen.
{\bf Vi m\aa\ derfor innafor ramma av Schr\"odingers likning
postulere egenspinnet til elektronet.} Ei heller gir den oss
massen til elektronet. 


Det faktum at spinnet ogs\aa\ kun tar verdien $1/2$ betyr at vi ikke
kan ta den klassiske grensa for store kvantetall, gitt ved
korrespondanseprinsippet. Vi kan sj\o lsagt gjenvinne den klassiske
grense ved \aa\ la $\hbar \rightarrow 0$, men siden 
elektronet ikke er en klassisk
partikkel, vi b\o r derfor v\ae re forsiktige n\aa r skal vi trekke
paralleller til klassiske systemer.


Eksperimentet som leda til postuleringen av elektronets egenspinn
kom fra studier av s\o lvatomer i 1921 av Stern og Gerlach.
De sendte s\o lvatomer gjennom et inhomogent magnetfelt for \aa\
studere kvantisering av banespinn. 
I et inhomogent magnetfelt, vil igjen en magnetisk dipol gi opphav
til en korreksjon til den potensielle energien gitt 
ved $U=-{\bf \mu}_L{\bf B}$, som svarer til en kraft som virker p\aa\
atomet gitt ved (vi velger for enkelthetsskyld ei kraft i $z$-retning)
\be
    F_z=-\frac{\partial U}{\partial z}=\mu_L\frac{\partial B_z}{\partial z},
\ee
som viser at krafta $F_z$ er proporsjonal med $\mu_L$. 
Kvantemekanisk og klassisk har vi sett at $\mu_L$ er proporsjonal
med banespinnet. Men kvantemekanikken gir oss kun diskrete verdier
for banespinnet. Fra foreg\aa ende underavsnitt har vi sett at
projeksjonen av $L$ p\aa\ $z$-aksen ga opphav til et sett med energier
som var proporsjonal med $m_l$, hvor $m_l$ kan anta $2l+1$ mulige verdier.
En skulle derfor i eksperimentet kunne observere $2l+1$ niv\aa\ .
For s\o lvatomer
i grunntilstanden med banespinn null skulle de ikke 
observere noen som helst avhengighet av magnetfeltet.
Eksperimentene viste derimot at str\aa len av s\o lvatomer blei
avb\o yd i to diskrete komponenter, en som var avb\o yd i positiv
$z$-retning og en i negativ $z$-retning.  
Phipps og Taylor gjentok et tilsvarende eksperiment men da for
hydrogenatomer i grunntilstanden i 1927, fremdeles med $l=0$. 
En tilsvarende oppsplitting blei observert.   

Enten er noe galt med teorien v\aa r ellers er det noe som mangler.
I tillegg skal vi merke oss at $2l+1$ gir et odde antall niv\aa er,
mens her har vi kun to. 
Goudsmit og Uhlenbeck foreslo i 1925 at denne
oppsplittingen skyldes at elektronet kan betraktes som en roterende
ladning, en spinnende ladning. Elektronet har alts\aa\ et innbygd
dipolmoment. F\o rst med Diracs formulering av en relativistisk
kvantemekanikk, var en istand til \aa\ lage en teori som inneholdt
denne frihetsgraden til elektronet. 
Innafor ramma av Schr\"odingers ikke-relativistiske kvantemekanikk
er vi n\o dt til \aa\ postulere denne egenskapen. Oppsummert har vi
f\o lgende.
\begin{itemize}
\item Oppsplittingen skyldes det magnetiske dipolmomentet til elektronet.
\item Elektronet har et egenspinn som vi kaller ${\bf S}$ som gir opphav 
      til et dipolmoment (p\aa\ lik linje med uttrykket for dipolmomentet 
      pga.~banespinnet) 
      \be
         {\bf \mu}_S=-g_S\frac{e}{2m}{\bf S},
      \ee
hvor $g_s$ kalles den gyromagnetiske faktor. Denne st\o rrelsen er forskjellig
fra 1 da elektronet ikke kan betraktes som et sf\ae risk legeme med konstant
forlhold mellom ladning og masse. Denne faktoren er m\aa lt til \aa\ v\ae re
\[
    g_S=2.00232.
\]
\item Siden vi har ei oppslitting i kun to niv\aa er for $l=0$ postulerer vi at
egenspinnet $S$ har degenerasjon (igjen p\aa\ lik linje med hva vi gjorde
for $l$)
\be  
    2s+1=2,
\ee
som betyr at 
\be
   s=\frac{1}{2},
\ee
dvs.~at elektronet har spinnverdi p\aa\ $1/2$. Den kvantemekaniske
spinnoperatorern $\OP{S}$ har som forventningsverdi
\be
   \OP{S}^2\chi= s(s+1)\hbar^2\chi,
\ee
og 
\be
   \OP{S}_z\chi=m_s\hbar\chi,
\ee
hvor $\chi$ er spinndelen av egenfunksjonen $\psi$. 
Kvantetallet $m_s$ tar verdien $m_s=\pm 1/2$ og
\be
   |\OP{S}|=\hbar\sqrt{s(s+1)}=\frac{\hbar}{2}\sqrt{3}.
\ee
\end{itemize}

Egenfunksjonen $\psi$ er n\aa\ gitt ved
\be
   \psi(r,\theta,\phi)=\psi_{nlm_lsm_s}(r,\theta,\phi)=\psi_{nlm_l1/2m_s}(r,\theta,\phi)=
    R_{nl}(r)Y_{lm_l}(\theta,\phi)\chi_{sm_l}(s),
\ee
med f\o lgende kvantetall og operatorer (vi dropper $(r,\theta,\phi)$ i 
uttrykket for egenfunksjonen $\psi$) 
\be
   \OP{L}^2\psi_{nlm_l1/2m_s}=\hbar^2l(l+1)\psi_{nlm_l1/2m_s},
\ee
\be
   \OP{L}_z\psi_{nlm_l1/2m_s}=m_l\hbar\psi_{nlm_l1/2m_s},
\ee
\be
   \OP{S}^2\psi_{nlm_l1/2m_s}=\hbar^2s(s+1)\psi_{nlm_l1/2m_s},
\ee
\be
   \OP{S}_z\psi_{nlm_l1/2m_s}=m_s\hbar\psi_{nlm_l1/2m_s}.
\ee
Hamiltonfunksjonen v\aa r tar n\aa\ f\o lgende form med 
magnetfelt p\aa satt i $z$-retning
\be
   \OP{H}=-\frac{\hbar^2\nabla^2}{2m}-\frac{ke^2}{r}+
                         \frac{e}{2m}(g_L\OP{L}_z+g_S\OP{S}_z)B_z,
\ee
med egenverdilikning
\be
   \OP{H}\psi_{nlm_l1/2m_s}=E_{nlm_l1/2m_s}\psi_{nlm_l1/2m_s},
\ee
med hamiltonoperatoren
\be
   \OP{H}=\left[-\frac{\hbar^2\nabla^2}{2m}-\frac{ke^2}{r}+
                         \frac{eB_z}{2m}(g_L\OP{L}_z+g_S\OP{S}_z)\right].
\ee
Egenverdien for energien er derved gitt 
\be
   E_{nlm_l1/2m_s}=\frac{E_0}{n^2}+\frac{eB_z}{2m}\hbar (g_Lm_l+g_Sm_s),
\ee
eller
\be
   E_{nlm_l1/2m_s}=\frac{E_0}{n^2}+\mu_BB_z(g_Lm_l+g_Sm_s),
\ee

Dersom vi ser tilbake p\aa\ $n=2$ og $l=1$ tilstanden fra forrige underavsnitt,
ser vi at vi har n\aa\ 2 mulige $m_s$ verdier, $m_s=\pm 1/2$ og tre mulige
$m_l$ verdier, $m_l=\pm 1$ og $m_l=0$. Det gir oss totalt 
seks egenverdier for energien, slik at vi f\aa r en ytterligere oppsplitting
av niv\aa ene i forhold til den normale Zeeman effekten. 
Denne tilleggsoppslittingen pga.~elektronetes egenspinn 
kalles for den anomale Zeeman effekten.

For $n=2$ og $l=1$ finner vi dermed f\o lgende egenverdier.
Siden vi n\aa\ har to mulige verdier for $m_s$

\[
  E_{21+1+1/2}=\frac{E_0}{4}+\mu_BB_z(g_L+\frac{g_S}{2}),
\]
\[
  E_{21+1-1/2}=\frac{E_0}{4}+\mu_BB_z(g_L-\frac{g_S}{2}),
\]
\[
  E_{210+1/2}=\frac{E_0}{4}+\frac{\mu_BB_z}{2}g_S,
\]
\[
  E_{210-1/2}=\frac{E_0}{4}-\frac{\mu_BB_z}{2}g_S,
\]
\[
  E_{21-1+1/2}=\frac{E_0}{4}+\mu_BB_z(\frac{g_S}{2}-g_L),
\]
og
\[
  E_{21-1-1/2}=\frac{E_0}{4}-\mu_BB_z(g_L+\frac{g_S}{2}).
\]
Vi merker oss at dersom $g_S$ hadde v\ae rt eksakt lik 2, ville
$E_{21+1-1/2}$ og $E_{21-1+1/2}$ ha samme verdi $E_0/4$.

Tilsvarende er n\aa\ grunntilstanden for $n=1$ og $l=0$ 
oppsplitta i to niv\aa er istedet for et, nemlig
\[
  E_{100+1/2}=E_0+\frac{\mu_BB_z}{2}g_S,
\]
\[
  E_{100-1/2}=E_0-\frac{\mu_BB_z}{2}g_S.
\]

Siden $\mu_B=5.733\times 10^{-9}$ eV/G, trenger vi verdier
for $B\sim 10^{10}$ G dersom korreksjonen pga.~det magnetiske
dipolmomentet fra banespinnet og egenspinnet til elektronet 
skal v\ae re p\aa\ st\o rrelse med $E_0$. Det beste vi kanskje 
f\aa r til idag er magnetfelt p\aa\ ca.~$B\sim 10^4$ G, eller noen f\aa\
Tesla. En antar at ved overflata p\aa\ en n\o ytronstjerne kan en
ha magnefelt p\aa\ st\o rrelse med $10^{12}$ G. Det vil igjen si at
niv\aa ene til et atom som hydrogenatomet vil v\ae re sv\ae rt forskjellige
p\aa\ ei n\o ytronstjerne sammenlikna  med jorda.

Sj\o l om feltet er p\aa\ st\o rrelse med den maksimale verdi av $B$
som er oppn\aa elig i labben, er korreksjonen veldig liten.
Men den er observerbar og beretter p\aa\ et vakkert vis om kvantisering
av spinn og dermed brudd med klassisk fysikk. Kun bestemte verdier av
energikorreksjonen  $\mu_L B$ og $\mu_S B$ er tillatte.



\subsection{Totalt spinn} 

I klassisk mekanikk spiller det totale banespinn en viktig
rolle, da dets tidsderiverte er lik kraftmomentet p\aa\
systemet under betraktning. P\aa\ tilsvarende vis kan vi skrive
det totale spinnet\footnote{Lesehenvisning er
kap 8-8, sidene 405-410.} i kvantemekanikken som summen av banespinn
og egenspinn gitt ved 
\be
   \OP{J}=\OP{L}+\OP{S}.
\ee
Siden denne formelen gjelder for b\aa de en-elektron
systemer som hydrogenatomet og for atomer
med flere elektroner, innf\o rer vi her f\o lgende notasjon.
{\bf Kvantetall som henviser til en-elektron
tilstander gis sm\aa\ bokstaver.
Kvantetall som representerer fler-elektron systemer 
gis store bokstaver. }
De siste tilstandene er satt sammen av kvantetallene
til et eller flere elektroner.
Det er viktig at dere merker dere denne form for notasjon.
Grunnen er at n\aa r vi ser p\aa\ det periodiske systemet
i neste kapittel, s\aa\ kan vi bygge opp kvantetallene til 
et fler-elektron system ved \aa\ legge sammen en-elektron
kvantetall. 

Generelt kommer vi til \aa\ bruke store bokstaver for
operatorene, dvs.~at f.eks.~det totale spinn er gitt 
$\OP{J}$. Siden sistnevnte er satt sammen av $\OP{L}$ og
$\OP{S}$ m\aa\ den ogs\aa\ utvise diskrete kvantetall
$J$ og $M_J$. Egenverdiene til det totale spinnet
er dermed gitt ved 
\be
   |\OP{J}|=\hbar\sqrt{J(J+1)},
\ee
og projeksjonen  langs $z$-aksen (det magnetiske 
kvantetallet) er
\be
   J_z=M_J\hbar,
\ee
med f\o lgende degenerasjon
\be
   M_J=-J, -J+1, \dots, J-1, J.
\label{eq:jvalues}
\ee

Med gitt verdi p\aa\ $L$ og $S$ kan vi bestemme
antallet mulige verdier av $J$ ved \aa\ studere $z$-komponenten
til $\OP{J}$. 
Den er gitt ved 
\be
\OP{J}_z=\OP{L}_z+\OP{S}_z.
\ee
Dersom vi ser bort ifra spinn-bane vekselvirkningen
som vi skal diskutere i neste underavsnitt,
vil b\aa de $\OP{L}_z$ og $\OP{S}_z$ ha kvantetall
$L_z=M_L\hbar$ og $S_z=M_S\hbar$ slik at 
\be
   M_J\hbar=M_L\hbar +M_S\hbar,
\ee
eller
\be
   M_J=M_L +M_S.
\ee
Siden maksimal verdi av $M_L$ er $L$ og $M_S$ er $S$
har vi
\be
   (M_J)_{\mathrm{maks}}=L+S.
\ee
N\aa\ skal vi utelukkende se p\aa\ hydrogenatomet,
da er $S=s=1/2$. I tr\aa d med notasjonen vi definerte ovenfor
bruker vi n\aa\ sm\aa\ bokstaver for kvantetallene, men ikke
operatorene, for en-elektroner systemer slik
som hydrogenatomet.
For hydrogenatomet har vi at maksimal verdi av $M_S=m_s=1/2$
slik at vi f\aa r
\be
   (m_j)_{\mathrm{maks}}=l+\frac{1}{2}.
\ee
Bruker vi  (\ref{eq:jvalues}) og det faktum at 
maksimal verdi av $M_J=m_j$ er $j$ har vi at
\be
   j=l+\frac{1}{2}, l-\frac{1}{2}, l-\frac{3}{2}, l-\frac{5}{2}, \dots 
\ee
For \aa\ bestemme hvor denne rekka slutter kan vi bruke
vektorulikheten
\be
   |\OP{L}+\OP{S}| \ge \left| |\OP{L}|-|\OP{S}|\right|.
\ee
Bruker vi $\OP{J}=\OP{L}+\OP{S}$ har vi 
\be
   |\OP{J}| \ge |\OP{L}|-|\OP{S}|,
\ee
eller
\be
   |\OP{J}|=\hbar\sqrt{J(J+1)}\ge |\hbar\sqrt{L(L+1)}-
   \hbar\sqrt{S(S+1)}|,
\ee
og spesialiser vi det hele til hydrogenatomet med $s=1/2$
finner vi 
\be
   |\OP{J}|=\hbar\sqrt{j(j+1)}\ge |\hbar\sqrt{l(l+1)}-
   \hbar\sqrt{\frac{1}{2}(\frac{1}{2}+1)}|.
\ee
Det er deretter mulig \aa\ vise at for hydrogenatomet
er det kun to mulige verdier av $j$ som tilfredsstiller 
denne ulikheten, nemlig 
\be
   j=l+\frac{1}{2}\hspace{0.1cm} \mathrm{og} \hspace{0.1cm}j=l-\frac{1}{2},
\ee
og dersom $l=0$ har vi
\be
   j=\frac{1}{2}.
\ee

Vi skal vente med eksempler og spektroskopisk 
notasjon til avsnitt \ref{sec:singelpartspekt}.
F\o r det skal vi f\o rst definere spinn-bane
vekselvirkningen.


\subsection{Spinn-bane vekselvirkningen}\label{subsec:spinnbane}

Spinn-bane vekselvirkningen\footnote{Lesehenvisning er
kap 8-9, sidene 410-416.} 
skyldes et internt magnetisk felt som settes opp
av protonet (kjernen) og som kopler til elektronets
spinn og gir opphav til korreksjon til 
bindingsenergien. For et atom slik som hydrogenatomet
er denne korreksjonen liten i forhold til
energien $E_n=E_0/n^2$, hvor $E_0=-13.6$. Typisk er den 4-5 
st\o rrelsesordener mindre enn bindingsenergien $E_n$ vi fant 
i avsnitt 7.4. Vi skal rekne p\aa\ dette nedenfor. 
Denne korreksjonen, som er proporsjonal med
finstrukturkonstanten $\alpha$
\[
  \alpha=\frac{e^2}{4\pi\epsilon_0\hbar c},
\]
gir opphav til en oppsplitting og finstruktur  i energispektret.
For tyngre atomer, er feltet som settes opp av kjernen st\o rre,
og dermed blir ogs\aa\ spinn-bane vekselvirkningen viktigere.

F\o r vi g\aa r videre er det viktige \aa\ skille to ting.
\begin{itemize}
\item For hydrogenatomet har vi sett at energien er degenerert,
   for en gitt verdi av hovedkvantetallet $n$ har vi
   $2n^2$ mulige tilstander. Alle disse tilstandene har samme energi.
   N\aa r vi setter p\aa\ et ytre magnetfelt kan vi oppheve denne
   degenerasjonen.
\item Finstrukturen pga.~spinn-bane vekselvirkningenskyldes
      et magnetisk felt satt opp av kjernen, og er derfor
      en korreksjon til energien som skyldes interne frihetsgrader,
      og ikke et ytre p\aa satt magnetisk felt.
\end{itemize}

La oss utlede uttrykket for spinn-bane vekselvirkningen.
Argumentasjonen her f\o lger et klassisk bilde\footnote{
Spinn-bane vekselvirkningen er en relativistisk korreksjon
og kan utledes formelt vha.~relativistisk kvantemekanikk
(FYS 303), men dette er utenfor pensum i FY102.}. 
Hittil har vi sett p\aa\ energien og andre st\o rrelser utifra
kjernen som referanse ramme. Dersom vi derimot velger 
elektronet som referanse ramme, vil kjernen ha en hastighet
$-{\bf v}$ og bevege seg i en sirkul\ae r bane om elektronet.
Det gir opphav til en str\o m ${\bf j}=-Ze{\bf v}$, som igjen,
i henhold til Ampere's lov, setter opp et magnetisk felt
\be
   {\bf B}=-\frac{Ze\mu_0}{4\pi}\frac{{\bf v}\times {\bf r}}{r^3},
 \ee
som ved bruk av banespinnet 
${\bf L}=m{\bf r}\times {\bf v}=-m{\bf v}\times {\bf r}$ kan omskrives
\be
   {\bf B}=\frac{Ze\mu_0}{4m\pi}\frac{{\bf L}}{r^3},
 \ee
og bruker vi at
\[
  \epsilon_0\mu_0=\frac{1}{c^2},
\]
har vi
\be
   {\bf B}=\frac{Ze}{4mc^2\pi\epsilon_0}\frac{{\bf L}}{r^3}.
\ee
Elektronets dipolmoment satt opp av egenspinnet vekselvirker
med det magnetiske feltet og gir et bidrag til energien.
I underavsnitt 4.5.2 viste vi at dette dipolmomentet var gitt
ved likning (4.163) og kunne skrives som
\[
  g_S\frac{e}{2m}{\bf S},
\]
slik at bidraget til energien kan skrives som
\be
 \Delta E=  g_S\frac{e}{2m}{\bf S}\frac{Ze}{4mc^2\pi\epsilon_0}\frac{{\bf L}}{r^3}.
\ee
Velger vi \aa\ sette $g_S=2$ har vi
\be
 \Delta E=\frac{Ze^2}{4m^2c^2\pi\epsilon_0r^3}{\bf S}{\bf L}.
\ee
Det finnes en ytterligere relativistisk korreksjon p\aa\ samme
form, den s\aa kalte Thomas presesjonen, slik at den totale  
korreksjonen blir
\be
 \Delta E=V_{SL }=\frac{Ze^2}{2m^2c^2\pi\epsilon_0r^3}{\bf S}{\bf L},
\ee
hvor vi har valgt \aa\ kalle denne korreksjonen til energien
for $V_{SL}$. 
Dette er et klassisk uttrykk, og velger vi \aa\
introdusere kvantemekaniske operatorer for banespinnet
og egenspinnet har vi
\be
   V_{SL }=Z\alpha\frac{\hbar}{2m^2c}\frac{\OP{S}\OP{L}}{r^3},
\ee
hvor vi har brukt definisjonen p\aa\ finstrukturkonstanten 
$\alpha$ for \aa\ omskrive siste ledd. 
\be
   \OP{J}^2=(\OP{L}+\OP{S})(\OP{L}+\OP{S})=
   \OP{L}^2+2\OP{L}\OP{S}+\OP{S}^2,
\ee
som gir
\be
   \OP{L}\OP{S}=\frac{\OP{J}^2-\OP{L}^2-\OP{S}^2}{2},
\ee
med den f\o lge at vi kan skrive spinn-bane vekselvirkningen
som
\be 
   V_{SL }=Z\alpha\frac{\hbar}{4m^2c}
          \frac{\OP{J}^2-\OP{L}^2-\OP{S}^2}{r^3}.
\ee
Dette er det generelle uttrykket, men siden vi legger
vekt p\aa\ hydrogenatomet i dette avsnittet, skal vi sette
$S=s=1/2$. 
N\aa r Denne operatoren legges til hamiltonfunksjonen
for hydrogenatomet, $\OP{H}_0$ i likning (4.153), s\aa\ 
kan ikke  $\OP{L}_z$ og $\OP{S}_z$ lenger bestemmes
skarpt. Vi velger derfor \aa\ klassifisere en tilstand
vha.~kvantetallene $n$, $l$, $j$ og $m_j$. 
Vi kan da rekne ut virkningen 
av spinn-bane vekselvirkningen p\aa\ en tilstand
$\psi_{nljm_j}$ gitt ved
\be
   V_{SL }\psi_{nljm_j}=Z\alpha\frac{\hbar}{4m^2c}
          \frac{\OP{J}^2-\OP{L}^2-\OP{S}^2}{r^3}\psi_{nljm_j},
\ee
som sammen med
\be 
   \OP{J}^2\psi_{nljm_j}=j(j+1)\hbar^2\psi_{nljm_j},
\ee
\be 
   \OP{L}^2\psi_{nljm_j}=l(l+1)\hbar^2\psi_{nljm_j},
\ee
og
\be 
   \OP{S}^2\psi_{nljm_j}=
    \frac{1}{2}(\frac{1}{2}+1)\hbar^2\psi_{nljm_j}=
    \frac{3}{4}\hbar^2\psi_{nljm_j},
\ee
kan vi skrive
\be
      V_{SL }\psi_{nljm_j}=Z\alpha\frac{\hbar^3}{4m^2c}
          \frac{j(j+1)-l(l+1)-3/4}{r^3}\psi_{nljm_j}.
\ee
N\aa\ er ikke dette forventningsverdien av operatoren
siden vi ogs\aa\ har en avhengighet av $r$, og egenfunksjonen
$\psi$ er ikke en egenfunksjon for $r$, akkurat som den 
heller ikke er en egenfunksjon av bevegelsesmengden.
N\aa r vi da skal rekne ut bidraget til energien,
m\aa\ vi finne middelverdien
\be
    \langle V_{SL }\rangle =
     \int\psi_{nljm_j}^*V_{SL }\psi_{nljm_j}d{\bf r},
\ee
som ved innsetting av hele operatoren er
\be
    \langle V_{SL }\rangle= 
     Z\alpha\frac{\hbar}{4m^2c}\int\psi_{nljm_j}^*
          \frac{\OP{J}^2-\OP{L}^2-\OP{S}^2}{r^3}\psi_{nljm_j}d{\bf r},
\ee
eller
\be
    \langle V_{SL }\rangle= 
     Z\alpha\frac{\hbar^3}{4m^2c}\int\psi_{nljm_j}^*
     \frac{(j(j+1)-l(l+1)-3/4)}{r^3}\psi_{nljm_j}d{\bf r},
\ee
som ved ytterligere pynting gir
\be
    \langle V_{SL }\rangle= 
     Z\alpha\frac{\hbar^3(j(j+1)-l(l+1)-3/4)}{4m^2c}
    \int\psi_{nljm_j}^*\frac{1}{r^3}\psi_{nljm_j}d{\bf r}.
\ee
Vi trenger \aa\ rekne ut 
forventningsverdien
\be
    \langle \frac{1}{r^3}\rangle= 
    \int\psi_{nljm_j}^*\frac{1}{r^3}\psi_{nljm_j}d{\bf r},
\ee
som gir
\be
   \langle \frac{1}{r^3}\rangle= 
   \frac{2}{a_0^3n^3l(l+1)(2l+1)},
\ee 
som vi igjen omskrives vha.~av definisjonen p\aa\
Bohrradien 
\[
  a_0=\frac{4\pi\epsilon_0\hbar^2}{Ze^2m}=
  \frac{\hbar}{Z\alpha mc},
\]
til
\be
   \langle \frac{1}{r^3}\rangle= 
   \left(\frac{Z\alpha mc}{\hbar n}\right)^3
   \frac{2}{l(l+1)(2l+1)}.
\ee 

Innsatt i uttrykket for forventningsverdien av spinn-bane
vekselvirkningen har vi da f\o lgende uttrykk
\be
    \langle V_{SL }\rangle=\frac{Z^4\alpha^4}{2n^3}mc^2 
     \left(\frac{(j(j+1)-l(l+1)-3/4)}{l(l+1)(2l+1)}\right).
   \label{eq:lsfinal1}
\ee
Vi kan forenkle dette uttrykket ytterligere dersom
vi bruker at energien for hydrogenatomet gitt ved 
\[
   |E_n|=\frac{|E_0|}{n^2},
\]
hvor $E_0=-13.6$ eV. Det siste leddet kan vi omskrive
vha.~av finstrukturkonstanten til
\[
   |E_0|=\frac{\alpha^2}{2}mc^2,
\]
som innsatt i likning (\ref{eq:lsfinal1})
gir
\be
    \langle V_{SL }\rangle=|E_n|\frac{Z^4\alpha^2}{n} 
     \left(\frac{(j(j+1)-l(l+1)-3/4)}{l(l+1)(2l+1)}\right).
   \label{eq:lsfinal2}
\ee
Vi merker oss at dette bidraget til energien 
er for sm\aa\ verdier av $Z$ mye mindre enn $E_n$,
da vi har finstrukturkonstanten kvadrert, dvs.
\[
   \alpha^2\sim \left(\frac{1}{137}\right)^2.
\]
Siden dette er en relativistisk korreksjon, ser vi 
at dersom vi \o ker $Z$ s\aa\ \o ker ogs\aa\ spinn-bane
vekselvirkningen. Dersom vi g\aa r tilbake til v\aa r diskusjon
av Bohratomet, s\aa\ vi i likning (1.132) at hastigheten
et elektron har i en bestemt Bohrbane $n$ var gitt
ved
\[ 
   v_n=\frac{Ze^2}{4\pi\epsilon_0 n\hbar}.
\]
For sm\aa\ verdier av $Z$, slik vi har i hydrogenatomet,
er dermed hastigheten bare noen f\aa\ prosent av
lyshastigheten, og vi kan trygt neglisjere relativistiske
korreksjoner. N\aa r vi \o ker ladningstallet derimot,
kan hastigheten n\aa\ st\o rrelser n\ae r lyshastigheten.
Dermed b\o r relativistike effekter komme inn, og spinn-bane
vekselvirkninger er en relativistisk korreksjon.

I en relativistisk formulering av kvantemekanikken
f\o lger 
spinn-bane vekselvirkningen automatisk. Her har vi brukt
et tiln\ae rmet klassisk bildet for \aa\ utlede denne 
korreksjonen. 

Siden hydrogenatomet kun har to mulige $j$-verdier
for en bestemt $l$\footnote{Det skyldes at 
egenspinnet er $s=1/2$ siden 
vi har kun et elektron. For flerelektron systemer, som vi skal studere i neste kapittel, kan vi ha flere verdier for $s$.}
, en for $j=l+1/2$ og en for $j=l-1/2$ trenger vi kun
\aa\ studere to tilfeller.

La oss rekne ut denne korreksjonen til energien $E_n$.
\begin{enumerate}
\item $j=l+1/2$ gir 
\be
    \langle V_{SL }\rangle=|E_n|
     \frac{Z^4\alpha^2}{n(l+1)(2l+1)}.
\ee
Den totale energien blir dermed 
\be 
   E_{nlj=l+1/2}=-|E_n|
           \left(1-\frac{Z^4\alpha^2}{n(l+1)(2l+1)}\right).
\ee
\item
 $j=l-1/2$ gir 
\be
    \langle V_{SL }\rangle=-|E_n|
     \frac{Z^4\alpha^2}{nl(2l+1)}.
\ee
Den totale energien blir dermed 
\be 
   E_{nlj=l-1/2}=-|E_n|
           \left(1+\frac{Z^4\alpha^2}{nl(2l+1)}\right).
\ee

\end{enumerate}

Vi kan deretter rekne ut hvor stor energiforskjellen
er mellom to niv\aa er med samme verdi for $l$ men ulik
total $j$. Tar vi differansen mellom tilstandene
med $j=l+1/2$ og $j=l-1/2$ finner vi 
\be
\delta_{SL }=|E_n|
     \frac{Z^4\alpha^2}{nl(l+1)}. 
\ee

Velger vi $n=2$ og $l=1$ og setter inn for finstrukturkonstanten,
ladning $Z=1$ og $|E_2|=3.4$ eV finner vi
\be
   \delta_{SL }=4.53\times 10^{-5} \hspace{0.1cm} \mathrm{eV},
\ee
som er fem st\o rrelsesordener mindre enn bindingsenergien
for denne tilstanden. Derav f\o lger ogs\aa\ navnet finstruktur
og det tilh\o rende navnet for $\alpha$, siden
energikorreksjoner er proporsjonal med finstrukturkonstanten.



\section{Energispekter for hydrogenatomet}
\label{sec:singelpartspekt}  

Vi skal  oppsummere resultatene fra de foreg\aa ende
avsnitt og diskutere energispekteret til hydrogenatomet.  

Vi skal ogs\aa\ innf\o re den klassiske spektroskopiske notasjonen
for tilstander klassifisert ved gitte kvantetall. 

I figur \ref{410} har vi spektret til hydrogenatomet opp
til $n=4$ for energien n\aa r den er gitt ved
hamiltonfunksjonen
\[
   \OP{H}_0=-\frac{\hbar^2\nabla^2}{2m}-\frac{ke^2}{r}.
\]
Energien er gitt ved
\[
   E_n=\frac{E_0}{n^2},
\]
med $E_0=-13.6$ eV, grunntilstanden, eller normaltistanden.
Energien er degenerert i dette tilfelle, og selvom
vi har trekvantetall, hovedkvantetallet $n$, banespinnet
$l$ og dets projeksjons p\aa\ $z$-aksen $m_l$, avhenger
energien kun av $n$. Vi har en degenerasjon n\aa r vi inkluderer
egenspinnet p\aa\ $2n^2$. Hver $l$-verdi gir $2l+1$ tilstander,
i tillegg har vi to muligheter til pga.~egenspinnet. 
Antall mulige tilstander er vist i parentes under niv\aa et. 
\begin{figure}[bp] 
\setlength{\unitlength}{1.0mm}
\begin{center}
\begin{picture}(140,140)(0,-30)
\thicklines
\put(5,-10){\line(0,1){140}}
\put(5,130){\line(1,0){115}}
\put(5,-10){\line(1,0){115}}
\put(120,-10){\line(0,1){140}}
%\multiput(5,0)(0,10){6}{\line(1,0){2}}
\thinlines
\put(-5,-2){-13.6}
\put(-5,2){$n=1$}
\put(-5,98){-3.4}
\put(-5,102){$n=2$}
\put(-5,116){-1.5}
\put(-5,120){$n=3$}
\put(-5,125){-0.85}
\put(-5,129){$n=4$}
\put(10,0){\nl{1s}}
\put(13,-5){$(2)$}
\put(10,100){\nl{2s}}
\put(13,95){$(2)$}
\put(40,100){\nl{2p}}
\put(43,95){$(6)$}
\put(10,118){\nl{3s}}
\put(13,113){$(2)$}
\put(40,118){\nl{3p}}
\put(43,113){$(6)$}
\put(70,118){\nl{3d}}
\put(73,113){$(10)$}
\put(13,122){$(2)$}
\put(10,127){\nl{4s}}
\put(40,127){\nl{4p}}
\put(43,122){$(6)$}
\put(73,122){$(10)$}
\put(103,122){$(14)$}
\put(70,127){\nl{4d}}
\put(100,127){\nl{4f}}
\put(15,-20){$l=0$}
\put(45,-20){$l=1$}
\put(75,-20){$l=2$}
\put(101,-20){$l=3$}
\end{picture}
\end{center}
\caption{Energitilstander for hydrogenatomet uten 
spinn-banevekselvirkning og ytre p\aa satt magnetisk felt $B$.
Energiene er i elektronvolt eV.
Tallene i parentes representerer antall mulige tilstander
gitt ved degenerasjonen $2l+1$ samt egenspinnets degenerasjon
$2s+1=2$. Som eksempel, for $l=2$ har vi en degenerasjon
i $m_l$ gitt ved $2l+1=5$ I tillegg har vi 2 mulige spinnprojeksjoner $m_s$, slik at det totale antall tilstander er 10.
N\aa r vi setter p\aa\ et ytre magnetisk felt er det mulig
\aa\ se denne degenerasjonen.\label{410}}
\end{figure}

For en gitt $n$ har vi at $l$ kan ta verdier
\[
  l=0,1,2,\dots n-1.
\]
Tilstander med ulike $l$-verdier gis den spektroskopiske
notasjonen som vist i tabellen nedenfor.
\begin{table}[h]
\begin{center}
\begin{tabular}{lccccccc} \hline \\
Verdi av $l$ &0 &1&2&3&4&5&$\dots$ \\  
Notasjon  &s &p&d&f&g&h&$\dots$  \\ 
Antall tilstander &2 &6&10&14&18&22&$\dots$ \\ \hline
\end{tabular}
\end{center}
\end{table} 

Vi skal utvide denne notasjonen i neste kapittel n\aa r vi
diskuterer det periodiske systemet.

Setter vi p\aa\ et ytre magnetfelt forandres hamiltonfunskjonen
for figur \ref{410} til 
\[
   \OP{H}=\left[-\frac{\hbar^2\nabla^2}{2m}-\frac{ke^2}{r}+
                         \frac{eB_z}{2m}(g_L\OP{L}_z+g_S\OP{S}_z)\right],
\]
med tilh\o rende energi
\[
   E_{nlm_l1/2m_s}=\frac{E_0}{n^2}+\mu_BB_z(g_Lm_l+g_Sm_l).
\]
Opphevingen av degenerasjonen pga.~et ytre magnetfelt er vist
i figur \ref{411} for $n=1$ og $n=2$.
\begin{figure}[bp] 
\setlength{\unitlength}{1.0mm}
\begin{center}
\begin{picture}(125,100)(0,-30)
\thicklines
\put(5,-10){\line(0,1){100}}
\put(5,90){\line(1,0){100}}
\put(5,-10){\line(1,0){100}}
\put(105,-10){\line(0,1){100}}
%\multiput(5,0)(0,10){6}{\line(1,0){2}}
\thinlines
\put(-5,0){$n=1$}
\put(-5,60){$n=2$}
\put(10,0){\nl{1s}}
\put(30,-5){\nl{$_{0,-1/2}$}}
\put(30,5){\nl{$_{0,1/2}$}}
\put(10,60){\nl{2s}}
\put(30,55){\nl{$_{0,-1/2}$}}
\put(30,65){\nl{$_{0,1/2}$}}
\put(60,60){\nl{2p}}
\put(80,45){\nl{$_{-1,-1/2}$}}
\put(80,50){\nl{$_{-1,1/2}$}}
\put(80,55){\nl{$_{0,-1/2}$}}
\put(80,65){\nl{$_{0,1/2}$}}
\put(80,70){\nl{$_{1,-1/2}$}}
\put(80,75){\nl{$_{1,1/2}$}}
\put(15,-20){$B_z=0$}
\put(35,-20){$B_z\ne 0$}
\put(65,-20){$B_z=0$}
\put(91,-20){$B_z\ne 0$}
\end{picture}
\end{center}
\caption{Energitilstander for hydrogenatomet uten 
spinn-banevekselvirkning men med og uten 
ytre p\aa satt magnetisk felt $B_z$.
Her ser vi kun p\aa\ $n=1$ og $n=2$ tilstander.
Kvantetallene $m_l$ og $m_s$ er inntegna ved hvert niv\aa\ .
Avstanden mellom niv\aa ene er kvalitativ. Den kvantitative
forskjellen er gitt i teksten.\label{411}
}
\end{figure}
I figur \ref{411} har vi inntegna verdien p\aa\ kvantetallene
$m_l$ og $m_s$.

Til slutt har vi spekteret for hydrogenatomet n\aa r vi tar
hensyn til spinn-bane vekselvirkningen. I dette tilfelle har
vi ikke noe ytre magnetfelt. Denne oppslittingen skyldes
kun indre frihetsgrader i hydrogenatomet.
For $l=0$ er spinn-bane vekselvirkninger null.
Spekteret opp til $n=3$ er vist i figur \ref{412}.
\begin{figure}[bp] 
\setlength{\unitlength}{1.0mm}
\begin{center}
\begin{picture}(120,130)(0,40)
\thicklines
\put(5,40){\line(0,1){90}}
\put(5,130){\line(1,0){95}}
\put(5,40){\line(1,0){95}}
\put(100,40){\line(0,1){90}}
\thinlines
\put(-5,50){$n=1$}
\put(-5,100){$n=2$}
\put(-5,120){$n=3$}
\put(10,50){\nl{$^2$S$_{1/2}$}}
\put(13,45){$(2)$}
\put(10,100){\nl{$^2$S$_{1/2}$}}
\put(13,95){$(2)$}
\put(40,95){\nl{$^2$P$_{1/2}$}}
\put(43,90){$(2)$}
\put(40,105){\nl{$^2$P$_{3/2}$}}
\put(43,100){$(4)$}
\put(10,120){\nl{$^2$S$_{1/2}$}}
\put(13,115){$(2)$}
\put(40,115){\nl{$^2$P$_{1/2}$}}
\put(43,110){$(2)$}
\put(40,125){\nl{$^2$P$_{3/2}$}}
\put(43,120){$(4)$}
\put(70,117){\nl{$^2$D$_{3/2}$}}
\put(73,112){$(4)$}
\put(70,123){\nl{$^2$D$_{5/2}$}}
\put(73,120){$(6)$}
\end{picture}
\end{center}
\caption{Energitilstander for hydrogenatomet med
spinn-banevekselvirkning men uten 
ytre p\aa satt magnetisk felt $B$.
Avstanden mellom niv\aa ene er kvalitativ.
Tallene i parentes representerer antall mulige tilstander
gitt ved degenerasjonen $2j+1$.
Som eksempel, for $j=3/2$ har vi en degenerasjon
i $m_j$ gitt ved $4$. 
N\aa r vi setter p\aa\ et ytre magnetisk felt er det mulig
\aa\ oppheve degenerasjonen i  $m_j$.\label{412} }
\end{figure}

I dette tilfellet er energien gitt ved
for $j=l+1/2$
\[ 
   E_{nlj=l+1/2}=-|E_n|
           \left(1-\frac{Z^4\alpha^2}{n(l+1)(2l+1)}\right).
\]
og for $j=l-1/2$
\[ 
   E_{nlj=l-1/2}=-|E_n|
           \left(1+\frac{Z^4\alpha^2}{nl(2l+1)}\right).
\]

N\aa r det gjelder spinn-bane vekselvirkningen og figur \ref{412} 
har vi innf\o rt en notasjon som er standard i spektroskopi.
Niv\aa ene klassifiseres utifra banespinn $L$, spinn $S$
og totalt spinn $J$. 
Notasjonen som brukes er
\be
    ^{2S+1}L_{J}.
\ee 
Istedet for sm\aa\ bokstaver for de ulike $l$-verdiene
brukes store bokstaver, som vist i tabellen nedenfor
\begin{table}[h]
\begin{center}
\begin{tabular}{lccccccc} \hline \\
Verdi av $L$ &0 &1&2&3&4&5&$\dots$ \\  
Notasjon  &S &P&D&F&G&H&$\dots$ \\ \hline
\end{tabular}
\end{center}
\end{table} 
En tilstand med $L=l=1$, $S=s=1/2$ og $J=j=3/2$, skrives
da som
\[
    ^{2}P_{3/2}.
\] 
Tilsvarende, med $L=l=2$, $S=s=1/2$ og $J=j=5/2$, skrives
\[
    ^{2}D_{5/2}.
\] 

Det er denne notasjonen vi skal ta med oss n\aa r vi beveger
oss over til det periodiske systemet.

Vi avslutter med \aa\ bemerke at dersom vi setter p\aa\
et ytre magnetfelt, vil vi f\aa\ en degenerasjon i $m_j$
ogs\aa\ . Det skal vi komme tilbake til i slutten av
neste kapittel. 




\section{Elektrisk dipol overgangsregler}
\label{sec:dipolregler}
La oss anta at vi har to tilstander\footnote{Lesehenvisning er
kap 7-5, sidene 366-373.}
\be
    \Psi_1({\bf r}, t)=\psi_1({\bf r})e^{-iE_1t/\hbar},
\ee
og
\be
    \Psi_2({\bf  r}, t)=\psi_2({\bf  r})e^{-iE_2t/\hbar}.
\ee
med veldefinerte energier og egenfunksjoner.

Dersom vi n\aa\ rekner ut
\be
   \Psi_1({\bf  r}, t)^*\Psi_1({\bf  r}, t)=
   \psi_1({\bf  r})^*\psi_1({\bf  r}) 
\ee
eller
\be
   \Psi_2({\bf  r}, t)^*\Psi_2({\bf  r}, t)=
   \psi_2({\bf  r})^*\psi_2({\bf  r}) 
\ee
ser vi at sannsynlighetstettheten er uavhengig av tiden $t$.
Slike tilstander kalles stasjon\ae re. Siden ladningsfordelingen
er proporsjonal med sannsynlighetstettheten, betyr det ogs\aa\
ar ladningsfordelingen til f.eks.~et elektron i grunntilstanden
til hydrogenatomet er uavhengig av tiden.
Det vil igjen si at vi har en statisk ladningsfordeling.
En statisk ladningsfordeling sender ikke ut elektromagnetisk
str\aa ling. Dersom vi g\aa r tilbake til Bohrs tredje postulat
om at elektronets energi i en gitt orbital er konstant,
ser vi at kvantemekanikken som teori har innbakt
Bohrs tredje postulat. 
Siden vi ikke kan ha emisjon
av e.m.~str\aa ling fra grunntilstanden, betyr det igjen
at elektronet i grunntilstanden ikke kan deaksellereres.
Teorien v\aa r gitt ved
Schr\"odingers likning  gir oss dermed muligheten for
\aa\ beskrive stasjon\ae re tilstander, dvs.~tilstander som
er stabile, uten at vi m\aa\ postulere dette slik Bohr gjorde. 

I tillegg kan vi beregne sannsynligheten for overganger
fra en tilstand til en annen.
Atomer som blir eksitert sender ut e.m.~str\aa ling og returnerer
f\o r eller siden til grunntilstanden.

La oss betrakte hydrogenatomet idet det er p\aa\ vei
fra en eksitert tilstand til en lavere liggende tilstand.
La oss kalle den eksiterte tilstanden for $2$ og den 
lavereliggende tilstanden for $1$, med henholdsvis energi
$E_2$ og $E_1$ og med de tilsvarende egenfunskjoner $\Psi_2$
og $\Psi_1$ definert ovenfor. 

Denne prossesen kan n\aa\ beskrives som en superposisjon
av tilstandene 1 og 2. Denne tilstanden er gitt ved
\be 
    \Psi({\bf  r}, t)=c_1\Psi_1({\bf  r}, t)+c_2\Psi_2({\bf  r}, t)
    =c_1\psi_1({\bf  r})e^{-iE_1t/\hbar}
     +c_2\psi_2({\bf  r})e^{-iE_2t/\hbar}.
\ee
Kvantemekanikken tillater at 
elektronet i hydrogenatomet kan v\ae re i begge tilstander 1 og 2,
med sannsynlighetstetthet
\begin{eqnarray} 
    \Psi({\bf  r}, t)^*\Psi({\bf  r}, t)= c_1^*c_1\psi_1^*\psi_1+c_2^*c_2\psi_2^*\psi_2 & & \nonumber \\ 
   + c_2^*c_1\psi_2^*\psi_1e^{-i(E_1-E_2)t/\hbar}+
    c_1^*c_2\psi_1^*\psi_2e^{i(E_1-E_2)t/\hbar}&&. 
\end{eqnarray}
Vi har ikke lenger en stasjon\ae r tilstand da 
sannsynlighetstettheten er en funksjon av tiden. Det betyr
igjen at ladningstettheten oscillerer i tid med en 
frekvens gitt ved
\be
    \nu=\frac{E_2-E_1}{h}.
\ee
Dersom vi tenker tilbake til FY101, s\aa\ vil en slik 
ladningsfordeling sende ut e.m.~str\aa ling ved samme 
frekvens. Men dette er n\o yaktig den samme frekvensen
til fotonet, i henhold til Bohr og Einstein 
(se Bohrs fjerde postulat), 
som blir emittert ved en spontan emisjon, siden energien
fotonet f\aa r er gitt ved $E_2-E_1$. 

Det enkleste bilde av et atoms ladningsfordeling som oscillerer
er det elektriske dipolmomentet. Fra FY101 vet vi at 
dette er produktet av elektronets ladning og 
forventningsverdien av ${\bf r}$, hvor $r$ er avstanden
fra elektronet til kjernen. Det elektriske dipolmomentet
er dermed et m\aa l for avtsanden til
elektronets ladningstetthet fra kjernen.
Dipolmomentet er gitt ved
\be
   {\bf D} = -e {\bf r} .
\ee
Ladningsfordelingen til et oscillerende elektrisk
dipolmoment sender ut e.m.~str\aa ling med frekvens
som svarer til oscillasjonsfrekvensen.

I v\aa rt tilfelle er denne oscillasjonsfrekvensen
gitt ved $\nu=E_2-E_1/h$.

For \aa\ lage oss et bilde av dette, anta n\aa\
at $\Psi_2$ svarer til $n=2$ og $l=1$, en s\aa kalt $2p$
tilstand. Tilstanden $\Psi_1$ svarer til grunntilstanden
i hydrogenatomet med $n=1$ og $l=0$. 
I figur \ref{45} har vi sannsynlighetstettheten (og dermed
ladningstettheten) for tilstanden $\Psi_1$.
Tilsvarende har vi i figur \ref{47} sannsynlighetstettheten (og dermed
ladningstettheten) for tilstanden $\Psi_2$.

Tilstanden $\Psi$, som er en superposisjon av
tilstandene $\Psi_1$ og $\Psi_2$, har n\aa\ en 
sannsynlighetstetthet
som oscillerer mellom figur \ref{45} og figur \ref{47}.
Denne oscillasjonen gir opphav til utsending av
e.m.~str\aa ling.

I dette kurset skal vi ikke rekne ut den totale
overgangsraten for elektrisk dipolovergang. Vi skal kun
vise at denne type overgang er tillatt dersom 
forskjellen i kvantetallene $m_l$, $l$ og $j$ er slik at
\be
  \Delta l=\pm 1,
\ee
\be
   \Delta m_l =0, \pm 1,
\ee
og
\be
   \Delta j=0, \pm 1.
\ee
 
Siden overgangssannsynsynligheten er proporsjonal
med dipolmomentet, trenger vi 
\aa\ rekne ut forventningsverdien av ${\bf r}$.

Denne forventningsverdien er gitt
ved 
\be
  \langle {\bf r}\rangle =\int_0^{\infty}
  \int_0^{\pi}\int_0^{2\pi}\Psi^*({\bf r}, t) {\bf r} 
  \Psi({\bf r},t)  r^2sin(\theta)drd\theta d\phi.
\ee
Setter vi inn $\Psi$ uttrykt ved tilstandene $\Psi_1$ 
og $\Psi_2$ har vi
\begin{eqnarray}
  \langle {\bf r}\rangle &=&\int_0^{\infty}
  \int_0^{\pi}\int_0^{2\pi}\left(c_1^*c_1\psi_1^*{\bf r}\psi_1+
    c_2^*c_2\psi_2^*{\bf r}\psi_2+\right. \nonumber \\ 
    & &\left.c_2^*c_1\psi_2^*{\bf r}\psi_1e^{-i(E_1-E_2)t/\hbar}+
    c_1^*c_2\psi_1^*{\bf r}\psi_2e^{i(E_1-E_2)t/\hbar}\right)r^2sin(\theta)drd\theta d\phi.
\label{eq:transbeast}
\end{eqnarray}
De f\o rste to leddene, som ikke involverer ledd
som oscillerer med tiden, blir null, som vi skal vise
nedenfor. Det elektriske dipolmomentet er dermed proporsjonalt
med de to siste ledd. 

Dersom vi dropper den tidsavhengige delen, kan vi skrive
integralet ovenfor p\aa\ f\o lgende generelle vis
\be
  \langle {\bf r}\rangle =\int_0^{\infty}
  \int_0^{\pi}\int_0^{2\pi}\psi_f^*{\bf r}\psi_i
   r^2sin(\theta)drd\theta d\phi,
   \label{eq:rforvent}  
\ee
hvor indeksene $i$ refererer til en begynnelsestilstand
$\psi_i$ og $f$ til en sluttilstand $\psi_f$.  

Integralet er tre-dimensjonalt, men den delen som er mest
interessant for oss er den som avhenger av vinkelen $\phi$.
Bruker vi  at
\be
   {\bf r}={\bf x}+{\bf y}+{\bf z}
\ee
med
\[
   x=rsin\theta cos\phi,
\] 
\[
   y=rsin\theta sin\phi,
\] 
og
\[
   z=rcos\theta
\] 
har vi at den delen av integralet som avhenger
av $\phi$ kan skilles ut i $x$, $y$ og $z$ ledd.
Leddet som avhenger av $x$ er gitt ved
\be
   I_x=rsin\theta \int_0^{2\pi}\psi_f^*r\psi_i
   cos\phi d\phi,
\ee
og bruker vi at
\[
   \psi_i(r,\theta,\phi)=\psi_{n_i,l_i,m_{l_i}}(r,\theta,\phi)
=R_{n_i,l_i}(r)P_{l_i,m_{l_i}}(\theta) e^{im_{l_i}\phi},
\]
(og tilsvarende for $\psi_f$) ser vi at det er kun 
eksponensialfunksjonen $e^{im_{l_i}\phi}$ som inng\aa r i den
$\phi$-avhengige delen av forventningsverdien.
Integralet $I_x$
kan skrives som
\be
   I_x=rsin\theta \int_0^{2\pi}e^{im_{l_f}\phi}e^{-im_{l_i}\phi}
   cos\phi d\phi,
\ee
eller
\be
   I_x=rsin\theta \int_0^{2\pi}e^{im_{l_f}\phi}e^{-im_{l_i}\phi}
   \frac{e^{i\phi}+e^{-i\phi}}{2} d\phi.
\ee
Tilsvarende har vi for $I_y$ at
\be
   I_y=rsin\theta \int_0^{2\pi}e^{im_{l_f}\phi}e^{-im_{l_i}\phi}
   \frac{e^{i\phi}-e^{-i\phi}}{2i} d\phi,
\ee
og for $I_z$ har vi
\be
   I_z=rsin\theta \int_0^{2\pi}
    e^{im_{l_f}\phi}e^{-im_{l_i}\phi}d\phi,
\ee
og siste ledd er forskjellig fra null bare dersom
\be
   m_{l_f}-m_{l_1}=\Delta m=0,
\ee
mens $I_x$ og $I_y$ gir
\be
   m_{l_f}-m_{l_1}=\Delta m=\pm 1,
\ee
og vi har dermed vist den f\o rste overgangsregelen.

N\aa r vi s\aa\ skal se p\aa\ forandringen i kvantetallet $l$,
trenger litt mer informasjon.

I oppgave 2.13 studerte vi et eksempel i en dimensjon hvor 
forventningsverdien av $x$ skulle reknes ut. Utifra symmetrien til b\o lgefunksjonen
\[
   \psi(-x)=(-)^l\psi(x),
\]
kunne vi vise at denne forventningsverdien var null.
I tre dimensjoner har vi at n\aa r vi ser p\aa\ symmetrien til 
egenfunksjonene m\aa\ vi studere oppf\o rselen n\aa r
$x\rightarrow -x$, $y\rightarrow -y$ og $z\rightarrow -z$.
Siden vi har valgt \aa\ jobbe med polarkoordinater, blir forandringen
$r\rightarrow r$ siden $r=\sqrt{x^2+y^2+z^2}$, 
$\theta\rightarrow \pi -\theta$ og $\phi\rightarrow \pi +\phi$.

Det betyr at det er kun vinkeldelen i v\aa r egenfunksjon, 
som er p\aa\ forma $R_{nl}(r)Y_{lm_l}(\theta,\phi)$, som p\aa virkes
av denne speilingen. Generelt har vi 
\be
   \psi(r, \pi -\theta, \pi +\phi)=(-)^l\psi(r, \theta, \pi),
\ee
hvor $l$ er banespinnet. 
Pariteten til egenfunksjonene  bestemmes dermed av verdien p\aa\ banespinnet.


Denne variabel forandringen leder til at forventningsverdien i
likning (\ref{eq:rforvent})  blir
\begin{eqnarray}
  \langle {\bf r}\rangle & = &\int_0^{\infty}
  \int_{\pi}^{0}\int_{0}^{2\pi}(-)^{l_f}\psi_f^*({\bf -r})(-)^{l_i}\psi_i
   r^2sin(\theta)dr(-d\theta) d\phi\nonumber \\ 
   &=&-(-)^{l_f+l_i}\int_0^{\infty}
  \int_{0}^{\pi}\int_{0}^{2\pi}\psi_f^*({\bf r})\psi_i
   r^2sin(\theta)dr(d\theta) d\phi.
\end{eqnarray}
Dersom vi ikke skal ha inkonsistens mellom siste uttrykk og
likning (\ref{eq:rforvent}) m\aa\ vi ha at 
\be
   -(-)^{l_f+l_i}=(-)^{l_f+l_i+1}=1,
\ee
som igjen impliserer at $l_f$ og $l_i$ ikke kan ha samme
paritet. Dersom $l_i$ og $l_f$ er begge odde eller like, m\aa\
forventningsverdien av ${\bf r}$ v\ae re null. 

Pariteten til ${\bf r}$ er odde, 
da ${\bf r}$ forandrer fortegn n\aa r
$x\rightarrow -x$, 
$y\rightarrow -y$ og $z\rightarrow -z$.
Dersom egenfunksjonene da har samme paritet, enten odde eller like,
blir integranden odde og integralet blir dermed null. 

Av dette f\o lger at 
\be
   \Delta l=\pm 1.
\ee

Det er en dypere sammeheng ogs\aa\ bak denne overgangsregelen.
Fotonet har totalt spinn lik $1$, og for at banespinnet skal v\ae re bevart,
m\aa\ vi ha at $l_f$ og $l_i$ har en forskjell p\aa\ $1$. 
Men, fotonet kan selvsagt ha  totalt spinn st\o rre enn 1 
ogs\aa\ , da vi kan legge til banespinn til egenspinnet.
Men da er det totale spinnet til fotonet st\o rre enn 1,
og vi har ikke lengre en elektrisk dipolovergang. Vi snakker
da heller om f.eks.~elektrisk kvadrupoloverganger osv.
La oss avslutte med \aa\ g\aa\ tilbake til likning (\ref{eq:transbeast}).

Leddene som inneholder den samme egenfunksjonen
\be
  \int_0^{\pi}\int_0^{2\pi}\left(c_1^*c_1\psi_1^*{\bf r}\psi_1+
    c_2^*c_2\psi_2^*{\bf r}\psi_2+\right)r^2sin(\theta)drd\theta d\phi,
\ee
blir null da egenfunksjonene har samme paritet. De eneste leddene som kan gi
noe er dermed kryssleddene hvor slutt og begynnelsestilstandene er
forskjellige
\begin{eqnarray}
  \langle {\bf r}\rangle=\int_0^{\infty}
  \int_0^{\pi}\int_0^{2\pi}
   \left(c_2^*c_1\psi_2^*{\bf r}\psi_1e^{-i(E_1-E_2)t/\hbar}\right.+ & &\nonumber \\
    \left. c_1^*c_2\psi_1^*{\bf r}\psi_2e^{i(E_1-E_2)t/\hbar}\right)r^2sin(\theta)drd\theta d\phi& &,
\end{eqnarray}
dog kan ikke  tilstandene $\psi_1$ og $\psi_2$ ha samme paritet.

Hva betyr det? Dersom $\psi_1$ representer $n=2$ og $l=0$ og $\psi_2$
representerer $n=1$ og $l=0$ er denne overgangen forbudt da vi har samme
paritet. Derimot kan vi ha en overgang 
dersom $\psi_1$ representer $n=2$ og $l=1$ og $\psi_2$
representerer $n=1$ og $l=0$.

Som tilleggsregel (som vi ikke skal bevise her) har vi ogs\aa\
\[
   \Delta j =0, \pm 1,
\]
men vi kan ikke ha en overgang fra $j=0$ til $j=0$. 

\section{Oppgaver}
\subsection{Analytiske oppgaver}
\subsubsection*{Oppgave 8.1}
I denne oppgaven ser vi bort fra elektronets egenspinn. H--atomet kan
da beskrives ved
tilstandsfunksjonene $\psi_{n l m_{l}}(\vec{r})$ som har f\o lgende
egenskaper:
%
\begin{eqnarray*}
\OP{H}_0 \psi_{n l m_{l}}(\vec{r}) &=&
E_{n} \psi_{n l m_{l}}(\vec{r}),                   \\
\OP{\vec{L}}^2 \psi_{n l m_{l}}(\vec{r}) &=&
l(l+1) \hbar^{2} \psi_{n l m_{l}}(\vec{r}),          \\
\OP{L}_{z} \psi_{n l m_{l}}(\vec{r}) &=&
m_{l} \hbar \psi_{n l m_{l}}(\vec{r}),                \\
\int \psi^{*}_{n l m_{l}}(\vec{r}) \psi_{n l m_{l}}(\vec{r})
d\vec{r} &=&
\delta_{n,n^{'}} \delta_{l,l^{'}} \delta_{m_{l},m_{l^{'}}}, \\
\delta_{k,k^{'}} &=& \left \{ \begin{array}{ll} 1& for \; \; k = k^{'}\\
															  0 & ellers.\\
										 \end{array}
							\right . \nonumber
\end{eqnarray*}
%
\begin{itemize}
%
\item[a)] Hva kaller vi ligninger av denne typen?
%
\end{itemize}
%
Operatoren $\OP{H}_{0}$ er tidsuavhengig.
Kvantetallet $n$ kan anta verdiene $1, 2, \ldots$.
For en gitt verdi av $n$ kan $l$ anta verdiene $0, 1, \ldots , n-1$,
og $m_{l}$ kan for en gitt verdi av $l$ anta verdiene $-l$, \mbox{$ -l+1, \dots , l$}.

I denne oppgaven trengs ingen andre opplysninger enn de som er gitt ovenfor.
Det er ikke n\o dvendig \aa ~kjenne de eksplisitte uttrykkene for operatorene
$\OP{H}_{0}$, $ \OP{\vec{L}}^{2}$ og
$\OP{L}_{z}$. Det skal ikke tas hensyn til elektronets egenspinn.

%
\begin{itemize}
%
\item[b)] Hvilke fysiske st\o rrelser er representert ved operatorene
$\OP{H}_{0}$, $\OP{\vec{L}}^{2}$ og
$\OP{L}_{z}$?
%
\item[c)] Hvilke fysiske st\o rrelser har skarpe verdier i tilstanden
$\psi_{n l m_{l}}(\vec{r})$?
%
\end{itemize}
%


Ved tiden $t = 0$ er H--atomets tilstand beskrevet ved tilstandsfunksjonen
$\psi_{n l m_{l}}(\vec{r})$. Tids- utviklingen av tilstandsfunksjonen er bestemt
av den tidsavhengige Schr\"{o}dingerligningen
\begin{eqnarray*}
\OP{H}_{0} \Psi (\vec{r}, t) =
i\hbar \frac{\partial }{\partial t} \Psi (\vec{r}, t).
\end{eqnarray*}

%
\begin{itemize}
%
\item[d)] Bestem tilstandsfunksjonen som beskriver H--atomets tilstand ved tiden
$t$.
%
\end{itemize}
%

La H--atomets tilstand ved tiden $t = 0$ n\aa ~v\ae re gitt ved
tilstandsfunksjonen
\begin{eqnarray}
\Phi (\vec{r}) = \frac{1}{\sqrt{2l+1}} \sum_{m_{l} = -l}^{l}
\psi_{n l m_{l}}(\vec{r}).
\label{a}
\end{eqnarray}

%
\begin{itemize}
%
\item[e)] Vis at $\Phi (\vec{r})$ er normert.

\item[f)] Vis at tilstandsfunksjonen ved tiden $t$ er
\begin{eqnarray}
\Psi (\vec{r}, t) = \Phi (\vec{r}) exp\left( -\frac{i}{\hbar } E_{n} t \right).
\label{b}
\end{eqnarray}

\item[g)] Bestem middelverdien for operatorene $\OP{H}_{0}$,
$\OP{\vec{L}}^{2}$ og $\OP{L}_{z}$ i tilstanden som
er beskrevet ved tilstandsfunksjonen $\Psi (\vec{r}, t)$ i lign.~(\ref{b}).
\end{itemize}

En st{\o}rrelse $A$ er representert ved operatoren $\OP{A}$ .
Fluktuasjoner $\Delta A$ i tilstanden
$\Psi$ er definert slik:
\[
\Delta A = \sqrt{\langle \OP{A}^2 \rangle - \langle \OP{A} \rangle^2}
\]
%
\begin{itemize}
%
\item[h)] Finn fluktuasjonen av st\o rrelsene representert ved
operatorene $\OP{H}_{0}$, $\OP{\vec{L}}^{2}$ og
$\OP{L}_{z}$ i tilstanden som er beskrevet ved tilstandsfunksjonen
$\Psi (\vec{r}, t)$ i lign.~(\ref{b}).
%
\end{itemize}
%

La H--atomets tilstand ved tiden $t = 0$ v\ae re gitt ved tilstandsfunksjonen
$\Phi (\vec{r})$ i lign.~(\ref{a}) og la oss tenke oss at vi foretar en ideell
m\aa ling av $L_{z}$.

%
\begin{itemize}
%
\item[i)] Hvor stor er sannsynligheten for \aa ~observere den bestemte verdien
$m_{l} \hbar$ for $L_{z}$ ved tiden $t = 0$?

\item[j)] Vil denne sannsynligheten v\ae re avhengig av ved hvilken tid $t > 0$
m\aa lingen utf\o res?
%
\end{itemize}
%

Vi lar n\aa ~H--atomet befinne seg i et homogent magnetfelt $B$
og velger z--aksen
langs magnetfeltet. Hamilton operatoren for systemet er da
\begin{eqnarray*}
\OP{H} = \OP{H}_{0} + \frac{e}{2m} B \OP{L}_{z},
\end{eqnarray*}
der $-e$ er elektronets ladning og $m$ er elektronets masse.

%
\begin{itemize}
%
\item[k)] Bestem H--atomets energi i tilstanden $\psi_{n l m_{l}}(\vec{r})$.

\item[l)] Er tilstanden $\Phi (\vec{r})$ i lign.~(1.1) en energi egentilstand
for $\OP{H}$? Begrunn svaret.

\item[l)] Bestem middelverdien av $\OP{H}$ i tilstanden
$\Phi (\vec{r})$.
%
\end{itemize}
%

\subsubsection*{Oppgave 8.2}
Ved overgangen fra en p-tilstand til en s-tilstand i et atom oppst\aa r en
spektrallinje med b\o lgelengde $\lambda$ = 6438,0 \AA \@. Atomet plasseres n\aa
~i et magnetfelt med styrke $B$ = 1,0 T\@. Vi skal i denne oppgaven ikke ta
hensyn til koblingen mellom elektronets spinn og dets banespinn.

%
\begin{itemize}
%
\item[a)] Hvordan splittes s- og p-niv\aa ene og hva blir deres
energiforskyvning?

\item[b)] Hvor mange nye spektrallinjer som tilfredsstiller kriteriene for
elektrisk dipolstr\aa ling oppst\aa r fra den gitte overgangen?

\item[c)] Hva blir b\o lgelengdene til disse nye spektrallinjene?
%
\end{itemize}
%
\subsubsection*{L\o sning}
%

%
\begin{itemize}
% 
\item[a)] Atomer i et ytre magnetfelt $\vec{B}$ langs z-aksen gir Zeeman splitting
% 
\begin{eqnarray}
\OP{H} &=& \OP{H}(B = 0) + \frac{e}{2m} \vec{B} \cdot \vec{L}\\
       &=& \OP{H}(B = 0) + \mu_B  B \frac{1}{\hbar} L_z
\end{eqnarray}
% 
hvor $\mu_B$ er Bohr-magneton. Dette gir ingen effekt p� s--niv�ene 
$(L = 0)$. For p--niv�ene f�r vi 
\begin{eqnarray}
E &= E_0 + \mu_B B \quad &M = +1\\
  &= E_0 \;\;\;\;\quad \quad \quad   &M = 0\\
  &= E_0 - \mu_B B \quad &M = -1
\end{eqnarray}
%
\item[b)] Overgangen $ p \longrightarrow s $ blir splittet i tre
komponenter. Utvalgsregler for dipolstr�ling er $\Delta L = \pm 1$
og $\Delta M = 0,\; \pm 1$. Dette gir 
% 
\begin{eqnarray}
p(M = +1)& \longrightarrow s(M = 0)\quad &\Delta M = +1\\
p(M = 0)& \longrightarrow s(M = 0)\quad &\Delta  M = 0\\
p(M = -1)& \longrightarrow s(M = 0)\quad &\Delta M = -1
\end{eqnarray}
% 
\item[c)] I beregningen bruker vi konstantene
% 
\begin{eqnarray}
h &=& 4,135669 \cdot 10^{-15}\; \mbox{eV s},\quad \quad
hc = 1239,84\; \mbox{eV nm}\\
\mu_B &=& 5,788383\cdot 10^{-5}\;\mbox{eV T}\\
\Delta E_0 &=& h \nu = \frac{hc}{\lambda} = 1,925819\;\mbox{eV}
\end{eqnarray}
%
Dette gir 
%
\begin{eqnarray}
E &=& E_0 \pm \mu_B \cdot 0,1\;T
  = (1,925819 \pm 5,8\cdot 10^{-5} )\;\mbox{eV}\\
\lambda_{M = +1} &=& 643,78\;\mbox{nm}\\
\lambda_{M = 0} &=& 643,80\;\mbox{nm}\\
\lambda_{M = -1} &=& 643,82\;\mbox{nm}
\end{eqnarray}
%
\end{itemize}


\subsubsection*{Oppgave 8.3}
Spinn--bane koblingen er en relativistisk effekt som skyldes at elektronet
har et egenspinn i tillegg til banespinnet i bevegelsen omkring
atomkjernen.
%
\begin{itemize}
%
\item[a)] Forklar hvorfor koblingen splitter spektrallinjen for
overgangen $2p
\rightarrow 1s$ i H--atomet i to linjer. Det er p\aa vist
eksperimentelt at disse
har en separasjon $\Delta \lambda = 5,3 \cdot 10^{-4}$ nm.
%
\item[b)] Spinn--bane koblingen kan beskrives ved operatoren
$\OP{H}_{LS} = A \OP{L} \cdot \OP{S}$.
Finn verdien p\aa ~konstanten $A$ for overgangen i a).
%
\end{itemize}
\subsubsection*{L\o sning}

\begin{itemize}
% 
\item[a)] Atomer med spinn--bane kobling
% 
\begin{eqnarray}
\OP{H}&=& \OP{H}_0 + A \vec{L} \cdot \vec{S}\\ 
      &=&  \OP{H}_0 + \frac{1}{2} A \left ( \OP{\vec{J}}^2 
                  -  \OP{\vec{L}}^2  - \OP{\vec{S}}^2 \right )
\end{eqnarray}
% 
N�r systemet er i tilstanden $\psi_{n l j m_j}$ i.e. en egentilstand
for operatorene $\OP{\vec{J}}^2, \OP{\vec{L}}^2, \OP{\vec{S}}^2$ og
$\OP{J}_z$ f�r vi 
% 
\[
\OP{H} \psi_{n l j m_j} = \left ( E_0 + \frac{\hbar^2}{2} A 
          \left (j(j+1) - l(l+1) -s(s+1) \right ) \right )
                 \psi_{n l j m_j}
\]
% 
For en s--tilstand har vi: $l = 0\; s = j = 1/2$ og  $E = E_0(s)$.\\
For en p--tilstand har vi: $l = 1\; s = 1/2$ som gir $ j = 1/2,  3/2 $ 
og energiene blir 
% 
\begin{eqnarray} 
E_{3/2} &=& E_0(p) + \frac{\hbar^2}{2} A \left ( \frac{15}{4} - 2- \frac{3}{4}
\right )  = E_0(p) + \frac{\hbar^2}{2} A\\
E_{1/2} &=& E_0(p) + \frac{\hbar^2}{2} A \left ( \frac{3}{4} - 2- \frac{3}{4}
\right )  = E_0(p) - \hbar^2 A
\end{eqnarray}
%
Overgangene
\begin{eqnarray} 
       p_{3/2} \longrightarrow s_{1/2} \quad \quad 
                         \Delta E_{3/2} = \Delta E +
                                       \frac{\hbar^2}{2}A \\
       p_{1/2} \longrightarrow s_{1/2} \quad \quad 
                         \Delta E_{1/2} = \Delta E -
                                         \hbar^2 A
\end{eqnarray}
%
hvor $\Delta E = E_0(p) - E_0(s)$.
%
\item[b)] Forskjell i b�lgelengde
% 
\[
\Delta \lambda = \frac{hc}{\Delta E_{1/2}} -  \frac{hc}{\Delta E_{3/2}}
               = hc \left ( \frac{1}{\Delta E - \hbar^2 A}
                           - \frac{1}{\Delta E + \frac{1}{2} \hbar^2 A}
                                                    \right )
\]
% 
som gir 
%
\[
\frac{\Delta \lambda}{hc} 
      = \frac{\frac{3}{2} \hbar^2 A}{(\Delta E -\hbar^2 A)
                                     (\Delta E +\frac{1}{2} \hbar^2 A)}
\approx \frac{3 \hbar^2 A}{2 \left (\Delta E\right )^2}
\]
%
siden $\Delta E = E(p) - E(s) = 10,2\;\mbox{eV}$ og $\hbar^2 A
<<\Delta E_0$.
Resultatet blir da 
%
\[
\hbar^2 A = \frac{2}{3}\frac{\Delta \lambda}{hc} (\Delta E)^2
\approx 3.0 \cdot 10^{-4}\;\mbox{eV}
\]
%
\end{itemize}

\subsubsection*{Oppgave 8.4, Eksamen V-1998}
I denne oppgaven skal vi til \aa \ begynne med se bort i fra 
elektronets egenspinn.  
Energi-egenverdiligningen for et elektron i hydrog\'{e}natomet er 
da gitt ved 
\[
\hat{H_0}\psi_{nlm_{l}}=-\frac{E_0}{n^2}\psi_{nlm_{l}}
\]
der $E_0$ er en konstant. 
\begin{itemize}
%
\item[a)] Sett opp Hamilton-operatoren $\hat{H_0}$ og skriv ned 
 hvilke betingelser kvantetallene $n$, $l$ og $m_{l}$ m\aa \ 
oppfylle.  Hvilken fysisk betydning har kvantetallene $l$ og 
$m_l$ ? 
Hva er degenerasjonsgraden for en gitt $n$?
%
\end{itemize}
%
Den laveste energi-egentilstanden er gitt ved 
\[
\psi_{100}(r,\theta,\phi)=A\exp\left(-\frac{r}{a_0}\right)
\]
der $A$ er en normeringskonstant og $a_0=4\pi\epsilon_0\hbar^2/me^2$.  
Til hjelp i det f\o lgende oppgir vi at i sf\ae riske koordinater er 
\[
\nabla^2=\frac{1}{r}\frac{\partial^2}{\partial r^2}r  
-\frac{1}{\hbar^2 r^2}\hat{{\bf L}}^2.
\] 
Videre kan du f\aa \ bruk for integralet $\int_{0}^{\infty}\rho^k\exp(-\rho) d\rho=k!$
der $k$ er et heltall.  
%
\begin{itemize}
%
\item[b)]  Vis ved innsetting i egenverdiligningen 
at $\psi_{100}$ er en egentilstand for $\hat{H_0}$ og bestem 
konstanten $E_0$ uttrykt ved naturkonstanter.  (Til kontroll 
oppgis at $E_0=13.6\;{\rm eV}$.)     
%
\item[c)]  Vis at normeringskonstanten $A$ har verdien 
$A=1/\sqrt{\pi}a_0^{3/2}$.  Beregn middelverdien
$\langle r_{100}\rangle$ for radien i denne tilstanden 
og den radius som elektronet med st\o rst 
sannsynlighet befinner seg i.  
%
\item[d)] 
Beregn b\o lgelengden for fotonet som sendes ut n\aa r 
hydrog\'{e}natomet g\aa r fra en $2p$-tilstand til $1s$-tilstanden 
(grunntilstanden). 
\end{itemize}

Vi skal n\aa \ ta hensyn til elektronets egenspinn.  P\aa \ 
grunn av spinn-banekoblingen f\aa r Hamilton--operatoren da 
et tilleggsledd 
\[
\hat{H}_{LS}=C\hat{{\bf L}}\cdot\hat{{\bf S}}
\]
der ${\hat{{\bf S}}}$ er operatoren for elektronets egenspinn 
og $C$ er en konstant.
\begin{itemize}  
\item[e)] Beregn de nye energiene for $2p$- og $1s$-tilstandene 
uttrykt ved $C$.   
Vis at spektrallinjen for overgangen fra $2p$ til  $1s$  
blir splittet i to linjer.  
Det er p\aa vist eksperimentelt at disse har en separasjon 
$\Delta \lambda = 5.3\cdot 10^{-4}\;{\rm nm}$.  
Bruk dette til \aa \ finne verdien p\aa \ konstanten $C$ for 
denne overgangen. 
\end{itemize}
%
\subsubsection*{Kort fasit}
\begin{itemize}
\item[a)] $\hat{H}_0=-\frac{\hbar^2}{2m}\nabla^2-\frac{e^2}{4\pi 
\epsilon_0 r}$, der $r$ er elektronets avstand fra kjernen.  
Kvantetallene m\aa \ oppfylle $n=1,2,\ldots$, $l=0,1,\ldots,n-1$ 
og $m_l=-l,-l+1,\ldots,l$.  Fysisk betydning: Kvantetalle $l$
bestemmer lengden av banespinnet og $m_l$ projeksjonen p�
kvantiserings--aksen. Degenerasjonsgrad for gitt $n$: 
\[
{\rm deg}(n)=\sum_{l=0}^{n-1}(2l+1)=n^2.
\]
\item[b)] Ved innsetting i egenverdiligningen f\aa s 
\[
\hat{H}_0\psi_{100}(r,\theta,\phi)=\left(\frac{\hbar^2}{ma_0}
-\frac{e^2}{4\pi\epsilon_0}\right)\frac{A}{r}e^{-r/a_0}
-\frac{\hbar^2}{2ma_0^2}Ae^{-r/a_0}, 
\]
som gir $a_0=\frac{4\pi\epsilon_0\hbar^2}{me^2}$ og 
$E_0=\frac{\hbar^2}{2ma_0^2}=\frac{m}{2\hbar^2}\left(\frac{e^2}
{4\pi\epsilon_0}\right)^2=13.6\;{\rm eV}$.  

\item[c)] Normalisering: 
\[
1=\int_0^{2\pi} d\phi\int_0^\pi \sin\theta d\theta\int_0^\infty r^2 dr 
A^2 e^{-2r/a_0}
\]
gir $A=\frac{1}{\sqrt{\pi}a_0^{3/2}}$. 
\[
\langle r_{100}\rangle=\int_0^{2\pi}d\phi\int_0^{\pi}\sin\theta d\theta 
\int_0^{\infty}r^2 dr \psi_{100}^*(r,\theta,\phi)r\psi_{100}(r,\theta,\phi) 
=\frac{3}{2}a_0. 
\]
Radiell sannsynlighetsfordeling: $P(r)=\frac{4}{a_0^3}r^2e^{-2r/a_0}$, 
$\frac{dP}{dr}=0$ gir\\
 $r(1-\frac{r}{a_0})=0$, dvs. $r=0$ eller $r=a_0$, 
der $r=a_0$ gir st\o rst sannsynlighet. 

\item[d)] $\frac{hc}{\lambda}=E_2-E_1$ gir 
$\lambda=\frac{4hc}{3E_0}=121.6\;{\rm nm}$.  

\item[e)] Har at 
\[
\langle \hat{H}_{LS}\rangle=\frac{C\hbar^2}{2}[j(j+1)-l(l+1)-s(s+1)].
\]
For $1s$ er $l=0$, $j=s=\frac{1}{2}$ som gir $\langle\hat{H}_{LS}\rangle
=0$, slik at $E_{1s}=-E_0$.  
For $2p$ er $l=1$, $s=\frac{1}{2}$ som gir $j=\frac{1}{2},\frac{3}{2}$ 
og $E_{1/2}=-\frac{E_0}{4}-C\hbar^2$, $E_{3/2}=-\frac{E_0}{4}+\frac{C
\hbar^2}{2}.$  Vi f\aa r da to linjer med b\o lgelengde hhv. 
$\lambda_{1/2}=\frac{hc}{E_{1/2}-E_{1s}}$ og $\lambda_{3/2}
=\frac{hc}{E_{3/2}-E_{1s}}$.  Ved \aa \ l\o se 
$\lambda_{1/2}-\lambda_{3/2}=\Delta\lambda=5.3\cdot 10^{-4}\;{\rm nm}$ 
med hensyn p\aa \ $C$\\
 f\aa s $C=6.9\cdot 10^{25}\;{\rm eV}^{-1}
{\rm s}^{-2}$.

\end{itemize}

\clearemptydoublepage
\part{Applications}
	

\chapter{The periodic system}



\section{Introduksjon}

Med utgangspunktet
i resultatene fra hydrogenatomet, skal
vi pr\o ve \aa\ forst\aa\ det periodiske 
systemet utifra kvantemekanikk.

Et av kjennetegnene er at vi har grunnstoffer med liknende
kjemiske egenskaper men h\o yst ulik verdi p\aa\ kjerneladningen
$Z$. Eksempler er atomene i edelgasserien, helium ($Z=2$), 
neon ($Z=10$), argon ($Z=18$),
krypton ($Z=38$) osv., eller alkalimetallene, 
med atomer som f.eks.~litium ($Z=3$),
natrium ($Z=11$), kalium ($Z=19$) osv. Felles for 
edelgassene er at ionisasjonsenergien, dvs.~energien som trengs
for \aa\ l\o srive et elektron er stor i forhold
til f.eks.~alkalimetallene. Figur \ref{51} viser ionisasjonsenergien
som funksjon av  kjerneladningen $Z$ opp til $Z=50$. 
Det er bla.~denne trenden kvantemekanikken med Schr\"odingers
likning var i stand til \aa\ forklare. 
For \aa\ forst\aa\ det periodiske systemet og dets oppbygging
skal vi f\o rst g\aa\ til det nest enkleste atomet, helium
med to elektroner i stedet for ett. 
I de etterf\o lgende avsnitt skal vi diskutere egenskapene
til andre elementer.
\begin{figure}[h]
\setlength{\unitlength}{1mm}
   \begin{picture}(100,100)
   \put(-60,-130){\epsfxsize=25cm \epsfbox{ioni.ps}}
   \end{picture}
%\begin{center}
%% GNUPLOT: LaTeX picture with Postscript
\setlength{\unitlength}{0.1bp}
\special{!
%!PS-Adobe-2.0
%%Creator: gnuplot
%%DocumentFonts: Helvetica
%%BoundingBox: 50 50 770 554
%%Pages: (atend)
%%EndComments
/gnudict 40 dict def
gnudict begin
/Color false def
/Solid false def
/gnulinewidth 5.000 def
/vshift -33 def
/dl {10 mul} def
/hpt 31.5 def
/vpt 31.5 def
/M {moveto} bind def
/L {lineto} bind def
/R {rmoveto} bind def
/V {rlineto} bind def
/vpt2 vpt 2 mul def
/hpt2 hpt 2 mul def
/Lshow { currentpoint stroke M
  0 vshift R show } def
/Rshow { currentpoint stroke M
  dup stringwidth pop neg vshift R show } def
/Cshow { currentpoint stroke M
  dup stringwidth pop -2 div vshift R show } def
/DL { Color {setrgbcolor Solid {pop []} if 0 setdash }
 {pop pop pop Solid {pop []} if 0 setdash} ifelse } def
/BL { stroke gnulinewidth 2 mul setlinewidth } def
/AL { stroke gnulinewidth 2 div setlinewidth } def
/PL { stroke gnulinewidth setlinewidth } def
/LTb { BL [] 0 0 0 DL } def
/LTa { AL [1 dl 2 dl] 0 setdash 0 0 0 setrgbcolor } def
/LT0 { PL [] 0 1 0 DL } def
/LT1 { PL [4 dl 2 dl] 0 0 1 DL } def
/LT2 { PL [2 dl 3 dl] 1 0 0 DL } def
/LT3 { PL [1 dl 1.5 dl] 1 0 1 DL } def
/LT4 { PL [5 dl 2 dl 1 dl 2 dl] 0 1 1 DL } def
/LT5 { PL [4 dl 3 dl 1 dl 3 dl] 1 1 0 DL } def
/LT6 { PL [2 dl 2 dl 2 dl 4 dl] 0 0 0 DL } def
/LT7 { PL [2 dl 2 dl 2 dl 2 dl 2 dl 4 dl] 1 0.3 0 DL } def
/LT8 { PL [2 dl 2 dl 2 dl 2 dl 2 dl 2 dl 2 dl 4 dl] 0.5 0.5 0.5 DL } def
/P { stroke [] 0 setdash
  currentlinewidth 2 div sub M
  0 currentlinewidth V stroke } def
/D { stroke [] 0 setdash 2 copy vpt add M
  hpt neg vpt neg V hpt vpt neg V
  hpt vpt V hpt neg vpt V closepath stroke
  P } def
/A { stroke [] 0 setdash vpt sub M 0 vpt2 V
  currentpoint stroke M
  hpt neg vpt neg R hpt2 0 V stroke
  } def
/B { stroke [] 0 setdash 2 copy exch hpt sub exch vpt add M
  0 vpt2 neg V hpt2 0 V 0 vpt2 V
  hpt2 neg 0 V closepath stroke
  P } def
/C { stroke [] 0 setdash exch hpt sub exch vpt add M
  hpt2 vpt2 neg V currentpoint stroke M
  hpt2 neg 0 R hpt2 vpt2 V stroke } def
/T { stroke [] 0 setdash 2 copy vpt 1.12 mul add M
  hpt neg vpt -1.62 mul V
  hpt 2 mul 0 V
  hpt neg vpt 1.62 mul V closepath stroke
  P  } def
/S { 2 copy A C} def
end
%%EndProlog
}
\begin{picture}(3600,2160)(0,0)
\special{"
%%Page: 1 1
gnudict begin
gsave
50 50 translate
0.100 0.100 scale
0 setgray
/Helvetica findfont 100 scalefont setfont
newpath
-500.000000 -500.000000 translate
LTa
600 251 M
2817 0 V
LTb
600 251 M
63 0 V
2754 0 R
-63 0 V
600 623 M
63 0 V
2754 0 R
-63 0 V
600 994 M
63 0 V
2754 0 R
-63 0 V
600 1366 M
63 0 V
2754 0 R
-63 0 V
600 1737 M
63 0 V
2754 0 R
-63 0 V
600 2109 M
63 0 V
2754 0 R
-63 0 V
1107 251 M
0 63 V
0 1795 R
0 -63 V
1670 251 M
0 63 V
0 1795 R
0 -63 V
2234 251 M
0 63 V
0 1795 R
0 -63 V
2797 251 M
0 63 V
0 1795 R
0 -63 V
3361 251 M
0 63 V
0 1795 R
0 -63 V
600 251 M
2817 0 V
0 1858 V
-2817 0 V
600 251 L
LT0
3114 1946 M
180 0 V
600 1262 M
56 817 V
713 652 L
56 292 V
56 -76 V
57 220 V
56 243 V
56 -68 V
57 283 V
56 307 V
1163 633 L
57 187 V
56 -124 V
56 161 V
57 174 V
56 -10 V
56 194 V
57 207 V
56 -848 V
56 131 V
57 32 V
56 21 V
56 -6 V
57 2 V
56 50 V
57 32 V
56 -1 V
56 -16 V
57 6 V
56 124 V
56 -252 V
57 141 V
56 142 V
56 -4 V
57 153 V
56 162 V
56 -729 V
57 113 V
112 50 V
57 34 V
56 3 V
56 17 V
57 13 V
56 13 V
56 66 V
57 -57 V
56 105 V
56 -238 V
57 116 V
3174 1946 D
600 1262 D
656 2079 D
713 652 D
769 944 D
825 868 D
882 1088 D
938 1331 D
994 1263 D
1051 1546 D
1107 1853 D
1163 633 D
1220 820 D
1276 696 D
1332 857 D
1389 1031 D
1445 1021 D
1501 1215 D
1558 1422 D
1614 574 D
1670 705 D
1727 737 D
1783 758 D
1839 752 D
1896 754 D
1952 804 D
2009 836 D
2065 835 D
2121 819 D
2178 825 D
2234 949 D
2290 697 D
2347 838 D
2403 980 D
2459 976 D
2516 1129 D
2572 1291 D
2628 562 D
2685 675 D
2797 725 D
2854 759 D
2910 762 D
2966 779 D
3023 792 D
3079 805 D
3135 871 D
3192 814 D
3248 919 D
3304 681 D
3361 797 D
stroke
grestore
end
showpage
}
\put(3054,1946){\makebox(0,0)[r]{Ionisasjonsenergi}}
\put(2008,51){\makebox(0,0){$Z$}}
\put(100,1180){%
\special{ps: gsave currentpoint currentpoint translate
270 rotate neg exch neg exch translate}%
\makebox(0,0)[b]{\shortstack{Ionisasjonsenergi (eV)}}%
\special{ps: currentpoint grestore moveto}%
}
\put(3361,151){\makebox(0,0){50}}
\put(2797,151){\makebox(0,0){40}}
\put(2234,151){\makebox(0,0){30}}
\put(1670,151){\makebox(0,0){20}}
\put(1107,151){\makebox(0,0){10}}
\put(540,2109){\makebox(0,0)[r]{25}}
\put(540,1737){\makebox(0,0)[r]{20}}
\put(540,1366){\makebox(0,0)[r]{15}}
\put(540,994){\makebox(0,0)[r]{10}}
\put(540,623){\makebox(0,0)[r]{5}}
\put(540,251){\makebox(0,0)[r]{0}}
\end{picture}

%\end{center}
\caption{Ionisasjonsenergi i eV som funksjon av ladningstallet $Z$
til kjernen.\label{51}}
\end{figure}

\section{Heliumatomet, en f\o rste tiln\ae rming}

Heliumatomet best\aa r av to elektroner pluss en kjerne.
Kjerneladningen er $Z=2$. 
N\aa r vi skal sette opp den potensielle energien til
systemet v\aa rt basert p\aa\ Coulombpotensialet,
m\aa\ vi ogs\aa\ ta hensyn til frast\o tingen mellom
elektronene.

Vi kaller avstanden mellom elektron 1 og kjernen
(som vi n\aa\ betrakter som et uendelig tungt massesenter)
for $r_1$, og avstanden mellom elektron 2 og kjernen for 
$r_2$ . 
Figur \ref{fig:heliumskisse} illustrerer dette\footnote{Vi ser bort ifra 
bla.~den kinetiske energien til partiklene i kjernen, da eksempelvis 
protonene er mye
tyngre enn elektronene. Dette kalles for Born-Oppenheimer approksimasjonen.}.
\begin{figure}[h]
\setlength{\unitlength}{1mm}
   \begin{picture}(100,50)
   \put(0,-70){\epsfxsize=16cm \epsfbox{heliumskisse.ps}}
   \end{picture}
\caption{Koordinatene brukt i beskrivelsen av heliumatomet.
\label{fig:heliumskisse}}
\end{figure}

Bidraget til den potensielle
energien pga.~tiltrekningen fra kjernen p\aa\ de to elektronene
blir dermed
\be
   -\frac{2ke^2}{r_1}-\frac{2ke^2}{r_2},
\ee 
og legger vi til frast\o tingen mellom de to elektronene
som er i en avstand $r_{12}=|{\bf r}_1-{\bf r}_2|$
har vi at den potensielle energien $V(r_1, r_2)$ er gitt
ved
\be
 V(r_1, r_2)=-\frac{2ke^2}{r_1}-\frac{2ke^2}{r_2}+
               \frac{ke^2}{r_{12}},
\ee
slik at den totale hamiltonfunksjonen for elektronene i heliumatomet
kan skrives
\be
   \OP{H}=-\frac{\hbar^2\nabla_1^2}{2m}-\frac{\hbar^2\nabla_2^2}{2m}
          -\frac{2ke^2}{r_1}-\frac{2ke^2}{r_2}+
               \frac{ke^2}{r_{12}},
\ee
og Schr\"odingers likning blir dermed 
\be
   \OP{H}\psi=E\psi.
\ee

Vi ser av uttrykket for den potensielle energien at vi ikke
lenger har et enkelt sentralsymmetrisk potensial.
For flere partikler blir det enda mer komplisert.

Dersom vi neglisjerer frast\o tingen mellom
elektroner, kan vi betrakte systemet som best\aa ende av 
ikke vekselvirkende elektroner, dvs.~elektronene f\o ler
kun tiltrekningen fra kjernen og vi kan addere deres bidrag til
energien. 

For heliumatomet betyr det at den potensielle energien
blir 
\be
    V(r_1, r_2)\approx -\frac{Zke^2}{r_1}-
                      \frac{Zke^2}{r_2}.
\ee
Fordelen med denne tiln\ae rmingen er at hvert elektron 
kan betraktes som uavhengig av det andre, den s\aa kalte
uavhengig-elektron model hvor hvert elektron kun ser
et sentralsymmetrisk potensial, eller sentralt felt.

La oss se om det kan gi oss noe meningsfylt.
F\o rst setter vi $Z=2$ og neglisjerer fullstendig
frast\o tingen mellom elektronene.

Vi kan n\aa\ bruke resultatene fra hydrogenatomet.
Elektron 1 har dermed en hamiltonoperator
\be
   \OP{h}_1=-\frac{\hbar^2\nabla_1^2}{2m}
          -\frac{2ke^2}{r_1},
\ee
med tilh\o rende egenfunksjon og egenverdilikning
\be
   \OP{h}_1\psi_a=E_a\psi_a,
\ee
hvor $a=\{ n_al_am_{l_a}\}$, kvantetallene fra hydrogenatomet.
Energien $E_a$ er dermed
\be
   E_a=\frac{Z^2E_0}{n_a^2},
\ee
med $E_0=-13.6$ eV, grunntilstanden i hydrogenatomet.
Helt tilsvarende har vi for elektron 2
\be
   \OP{h}_2=-\frac{\hbar^2\nabla_2^2}{2m}
          -\frac{2ke^2}{r_2},
\ee
med tilh\o rende egenfunksjon og egenverdilikning
\be
   \OP{h}_2\psi_b=E_b\psi_b,
\ee
hvor $b=\{ n_bl_bm_{l_b}\}$, og 
\be
   E_b=\frac{Z^2E_0}{n_b^2}.
\ee

Siden elektronene ikke vekselvirker kan vi anta at
grunntilstanden til heliumatomet er gitt ved produktet
\be
  \psi=\psi_a\psi_b,
\ee
slik at approksimasjonen til $\OP{H}$ gir f\o lgende Schr\"odingers likning
\be
   \left(\OP{h}_1+\OP{h}_2\right)\psi=
    \left(\OP{h}_1+\OP{h}_2\right)
    \psi_a({\bf r}_1)\psi_b({\bf r}_2)=
    E_{ab}\psi_a({\bf r}_1)\psi_b({\bf r}_2).
\ee
Energien blir dermed
\be
    \left(\OP{h}_1\psi_a({\bf r}_1)\right)\psi_b({\bf r}_2) +
    \left(\OP{h}_2\psi_b({\bf r}_2)\right)\psi_a({\bf r}_1) =
    \left(E_{a}+E_b\right)\psi_a({\bf r}_1)\psi_b({\bf r}_2),
\ee
dvs.
\be
   E_{ab}=Z^2E_0\left(\frac{1}{n_a^2}+\frac{1}{n_b^2}\right).
\ee
Setter vi inn $Z=2$ og antar at grunntilstanden er gitt ved
elektronene i laveste en-elektron tilstand med $n_a=n_b=1$
blir energien
\be
    E_{ab}=8E_0=-108.8\hspace{0.1cm} \mathrm{eV},
\ee
mens den eksperimentelle verdien er $-78.8$ eV.
{\bf Vi ser klart at \aa\ kutte det frast\o tende leddet
pga.~vekselvirkningen mellom elektronene gj\o r at vi
f\aa r for mye binding.}

Ionisasjonsenergien er gitt ved den energien som kreves for \aa\
l\o srive et elektron. Energien for \aa\ l\o srive det ene elektronet
er i den uavhengige-elektron modellen gitt ved
halve bindingsenergien, dvs.~$54.4$ eV, mens den eksperimentelle
verdien er $24.59$ eV.

Selv om vi har neglisjert frast\o tingen mellom elektronene, kommer
fremdeles mesteparten av bidraget til energien fra
tiltrekningen mellom kjernen og elektronene. 
La oss derfor gj\o re f\o lgende tankeeksperiment.

Anta at vi skal bygge det periodiske systemet vha.~den uavhengige
elektronmodellen. Vi antar ogs\aa\ at alle elektronene kan
plasseres i laveste en-elektron orbital gitt ved
$n=1$. 
For et system med $N$ elektroner og $Z$ protoner 
blir dermed bindingsenergien
n\aa r vi plasserer alle elektronene i laveste tilstand
\be
   E_N=NZ^2E_0.
\ee
Selv om den effektive ladningen skulle v\ae re liten, vil
fremdeles bindingsenergien \o ke som funksjon av elektrontallet
og ladningen. Ionisasjonsenergien vil v\ae re gitt ved
\be
   Z^2E_0,
\ee
som  er i strid med de eksperimentelle dataene vist i figur
\ref{51}.

Hvordan komme ut av dette uf\o ret? Svaret er Paulis prinsipp
som vi diskuterer i neste avsnitt. 

\section{Identiske partikler og Paulis eksklusjonsprinsipp}

Paulis eksklusjonsprinsipp (1925) 
er det siste postulatet om naturen som vi skal
befatte oss med. Som de tre foreg\aa ende, Einsteins og Plancks kvantiserings
postulat, de Broglies p\aa stand om materiens b\o lge og partikkel egenskaper
og Heisenbergs uskarphetsrelasjon, s\aa\ er dette p\aa stander om naturen som
springer utifra mange fors\o k p\aa\ \aa\ tolke og forst\aa\
eksperimentelle data. Pauli lanserte eksklusjonsprinsippet i et fors\o k
p\aa\ \aa\ forst\aa\ spektrene til ulike atomer, deres kjemiske egenskaper
og ionisasjonsenergier. 
Eksklusjonsprinsippet sier at den totale b\o lgefunksjonen for 
et system av elektroner skal v\ae re antisymmetrisk. Mer generelt, s\aa\
er den totale b\o lgefunksjonen for 
et system av {\bf fermioner}, som er partikler med halvtallig spinn, 
antisymmetrisk. 
Dette har som f\o lge at fermionene ikke kan ha samme sett kvantetall i ett
fler-fermion system.
Vi har sett i forrige kapittel at elektronet har et
egetspinn, med verdi $s=1/2$ og to mulige spinnprojeksjoner langs en valgt
akse $z$, $m_s=\pm 1/2$. 
B\o lgefunskjonen for et system av partikler med heltallig spinn, 
s\aa kalte {\bf Bosoner}, er symmetrisk. 
Tabell \ref{tab:partikkelstat} lister opp eksempler p\aa\ partikler
med heltallig og halvtallig spinn.
\begin{table}[h]
\caption{B\o lgefunksjonens symmetri for ulike partikler. \label{tab:partikkelstat}}
\begin{center}
\begin{tabular}{cccc} \hline \\
Partikkel   & Symmetri & Generisk navn & spinn \\ \hline
e$^-$& A & Fermion & $\frac{1}{2}$  \\    
proton& A & Fermion & $\frac{1}{2}$  \\    
n\o ytron& A & Fermion & $\frac{1}{2}$  \\    
$\mu^-$& A & Fermion & $\frac{1}{2}$  \\    
kvarker& A & Fermion & $\frac{1}{2}$  \\    
$\alpha$-partikkel& S & Boson & $0$  \\    
He-atom& S & Boson & $0$  \\    
$\pi$-meson& S & Boson & $0$  \\    
$\gamma$-foton& S & Boson & $1$  \\    
$\rho$-meson& S & Boson & $1$  \\  \hline 
\end{tabular}
\end{center}
\end{table} 
Merk at et helium-atom er satt sammen av elektroner. Men dersom vi betrakter
et system satt sammen av slike atomer, kan hver partikkel oppfattes som et 
boson.

Hva er det som ligger til grunn for denne inndelingen utifra partiklenes
egetspinn? 
Elektronene er identiske partikler. Klassisk kan vi dog tenke oss at vi kan 
skille mellom elektronene. Figur \ref{fig:identicalpart1}
viser et st\o t mellom f.eks. to elektroner. Dersom vi betrakter elektronene som klassiske partikler, kunne vi kanskje tenkt oss et eksperiment hvor vi er i stand til \aa\ f\o lge banene til begge elektronene, b\aa de f\o r og etter 
kollisjonen.  Dette eksperimentet kan tenkes gjennomf\o rt ved en serie av
fotobokser som sender ut fotoner  som vekselvirker med hvert sitt elektron.
I hvert tidsintervall bestemmer vi da n\o yaktig de to elektronenes posisjon.
Tilsynelatende b\o r vi derfor v\ae re i stand til \aa\ skille mellom
de to kollisjonene i Figur \ref{fig:identicalpart1}.

Problemet v\aa rt er vi har glemt Heisenbergs uskarphetsrelasjon. Vi kan ikke
bestemme n\o yaktig hvor elektron 1 og 2 er til enhver tid.
I praksis vil det si at et m\aa leapparat som skal f\o lge elektron 1 sin
bevegelse ikke vil v\ae re i stand til \aa\ skille mellom det to
prosessene vist i  Figur \ref{fig:identicalpart1}.
\begin{figure}[h]
\setlength{\unitlength}{1mm}
   \begin{picture}(100,50)
   \put(0,-60){\epsfxsize=14cm \epsfbox{identical_part1.ps}}
   \end{picture}
\caption{Kollisjon mellom to identiske partikler. Hendingene til venstre og h\o yre kan ikke skilles n\aa r vi har med identiske kvantemekaniske 
partikler \aa\ gj\o re.
\label{fig:identicalpart1}}
\end{figure}
Dette leder oss til begrepet om identiske og uskilbare partikler.
Dersom vi hadde sett p\aa\ et st\o t mellom protoner og elektroner
vil vi klart, pga.~forskjellig masse og ladning, kunne skille hvilke
partikler som kommer ut hvor.
Vi kan videref\o re problemet med identiske og uskilbare partikler 
til f.eks. to elektroner i heliumatomet. Dersom hvert elektron
er i hver sin $1s$ hydrogenorbital, er det en sterk grad av overlapp 
mellom b\o lgefunksjonene. Partiklene er identiske, og vi kan skille
mellom elektron 1 og 2.

Konsekvensen er en fundamental forskjell mellom identiske partikler
i kvantefysikk og klassisk fysikk, utrykt ved begrepet om uskilbarhet.
N\aa r vi derfor setter opp Schr\"odingers likning for et system
av partikler, m\aa\ formalismen v\aa r gjenspeile at identiske 
partikler er uskilbare.

La oss vise det ved et konkret eksempel.
I forrige avsnitt satte vi opp Schr\"odingers likning for to elektroner
\be
   \left(-\frac{\hbar^2\nabla_1^2}{2m}-\frac{\hbar^2\nabla_2^2}{2m}
          -\frac{2ke^2}{r_1}-\frac{2ke^2}{r_2}+
               \frac{ke^2}{r_{12}}\right) \Psi({\bf r}_1,{\bf r}_2)=
               E\Psi({\bf r}_1,{\bf r}_2).
\ee
Her har vi valgt \aa\ sette en merkelapp p\aa\ hvert elektron via
posisjonene ${\bf r}_1,{\bf r}_2$. Det er m\aa ten vi i praksis vil 
formulere matematisk problemet v\aa rt. 
Det er ikke noe i veien for at vi kan 
bytte om til $\Psi({\bf r}_2,{\bf r}_1)$. Men, 
en observabel st\o rrelse slik som
systemets total energi 
skal ikke avhenge av om hvilken merkelapp vi har valgt for henholdsvis
elektron 1 og 2. 

Problemet v\aa rt er dermed f\o lgende: hvordan skal vi bake inn begrepet
om uskilbarhet i b\o lgefunksjonen, som inneholder den fysiske informasjonen om systemet,  
og samtidig beholde merkelappene 
 ${\bf r}_1,{\bf r}_2$ osv.? 

La oss igjen glemme vekselvirkningen mellom elektronene i den totale
Hamiltonfunksjonen for heliumatomet.  
Schr\"odingers likning blir dermed
\be
   \left(-\frac{\hbar^2\nabla_1^2}{2m}-\frac{\hbar^2\nabla_2^2}{2m}
          -\frac{2ke^2}{r_1}-\frac{2ke^2}{r_2}\right) \Psi_T({\bf r}_1,{\bf r}_2)=
               E\Psi_T({\bf r}_1,{\bf r}_2),
\ee
hvor $\Psi_T({\bf r}_1,{\bf r}_2)$ er den totale egenfunksjonen.
Vi pr\o ver ansatzen
\be
   \Psi_T({\bf r}_1,{\bf r}_2)= 
    \psi_{\alpha}({\bf r}_1)\psi_{\beta}({\bf r}_2),
\ee
hvor $\psi$ er enpartikkel egenfunksjoner hentet fra f.eks.~hydrogenatomet.
Indeksene $\alpha$ og $\beta$ kan representere kvantetall som banespinn, egenspinn m.m.
Vi forenkler ytterligere og setter
\be
   \psi_{\alpha}({\bf r}_1)= \psi_{\alpha}(1),
\ee
og 
\be
   \psi_{\beta}({\bf r}_2)= \psi_{\beta}(2),
\ee
som betyr at det finnes f.eks. et elektron med kvantetall $\alpha$ i posisjon
${\bf r}_1$. 
Vi har dermed 
\be
   \Psi_T({\bf r}_1,{\bf r}_2)= 
    \psi_{\alpha}(1)\psi_{\beta}(2),
\ee
men det er ikke noe hinder for at vi har
\be
   \Psi_T({\bf r}_1,{\bf r}_2)= 
    \psi_{\beta}(1)\psi_{\alpha}(2),
\ee
da $\Psi_T$ ikke skal avhenge av v\aa rt valg av merking.

Hvilke konsekvenser f\aa r dette for en st\o rrelse 
som sannsynlighetsfordelingen? Sannsynligheten for \aa\ finne dette to-elektron
systemet i en bestemt kvantemekanisk tilstand skal v\ae re uavhengig av v\aa rt
valg av merkelapper. 

Sannsynlighetsfordelingen blir
\be
\Psi_T^*({\bf r}_1,{\bf r}_2)\Psi_T({\bf r}_1,{\bf r}_2)= 
\psi_{\alpha}^*(1)\psi^*_{\beta}(2)\psi_{\alpha}(1)\psi_{\beta}(2),
\ee
og bytter vi om $1\rightarrow 2$ og $2\rightarrow 1$ f\aa s
\be
\Psi_T^*({\bf r}_2,{\bf r}_1)\Psi_T({\bf r}_2,{\bf r}_1)= 
\psi_{\alpha}^*(2)\psi^*_{\beta}(1)\psi_{\alpha}(2)\psi_{\beta}(1).
\ee
Men disse sannsynlighetene er ikke like da f.eks.
\be
\psi_{\alpha}(2)\ne \psi_{\alpha}(1).
\ee

Men dersom vi erstatter
\[
   \Psi_T({\bf r}_1,{\bf r}_2)= 
    \psi_{\alpha}(1)\psi_{\beta}(2),
\]
med
\be
   \Psi_S({\bf r}_1,{\bf r}_2)= \frac{1}{\sqrt{2}}\left[
    \psi_{\alpha}(1)\psi_{\beta}(2)+\psi_{\beta}(1)\psi_{\alpha}(2)\right],
\ee
oppn\aa r vi ved ombyttet $1\rightarrow 2$ og $2\rightarrow 1$ at
\be
   \Psi_S({\bf r}_2,{\bf r}_1)= \frac{1}{\sqrt{2}}\left[
    \psi_{\alpha}(2)\psi_{\beta}(1)+\psi_{\beta}(2)\psi_{\alpha}(1)\right]=
\Psi_S({\bf r}_1,{\bf r}_2).
\ee
Indeksen $S$ st\aa r for symmetrisk, dvs.~at b\o lgefunksjonen vi har skrevet
er symmetrisk i ombyttet av posisjonene til elektronene. Vi har ogs\aa\
antatt at de respektive enpartikkel b\o lgefunksjonene er normerte.
Faktoren $1/\sqrt{2}$ er dermed en normeringskonstant for to-elektron funksjonen. 
Det er lett \aa\ overbevise seg selv om at 
sannsynligheten gitt ved den symmetriske b\o lgefunksjonen 
er identisk ved ombytte av partikkel 1 og 2.

Men vi kan ogs\aa\ skrive den totale egenfunksjonen p\aa\ et antisymmetrisk vis
\be
   \Psi_A({\bf r}_1,{\bf r}_2)= \frac{1}{\sqrt{2}}\left[
    \psi_{\alpha}(1)\psi_{\beta}(2)-\psi_{\beta}(1)\psi_{\alpha}(2)\right],
\ee
hvor indeksen $A$ st\aa r for antisymmetrisk.
Ved ombyttet $1\rightarrow 2$ og $2\rightarrow 1$ ser vi at
\be
   \Psi_A({\bf r}_2,{\bf r}_1)= \frac{1}{\sqrt{2}}\left[
    \psi_{\alpha}(2)\psi_{\beta}(1)-\psi_{\beta}(2)\psi_{\alpha}(1)\right]=
-\Psi_A({\bf r}_1,{\bf r}_2).
\ee
Ved ombyttet skifter funksjonen fortegn. Men sannsynlighetstettheten
forblir den samme da 
\be 
\Psi_A^*({\bf r}_2,{\bf r}_1)\Psi_A({\bf r}_2,{\bf r}_1)=
(-1)^2\Psi_A^*({\bf r}_1,{\bf r}_2)\Psi_A({\bf r}_1,{\bf r}_2).
\ee

Det er en viktig forskjell mellom en symmetrisk og en antisymmetrisk
funksjon. 
Dersom $\alpha=\beta$ og partiklene befinner seg i samme posisjon finner vi at
\be
   \Psi_S({\bf r}_1,{\bf r}_1)= \frac{2}{\sqrt{2}}\psi_{\alpha}(1)\psi_{\alpha}(1),
\ee
og at 
\be
   \Psi_A({\bf r}_1,{\bf r}_1)= 0!
\ee
For en antisymmetrisk funksjon kan ikke to partikler ha samme sett
kvantetall og v\ae re ved samme posisjon. Sannsynligheten for dette er lik 
null.
Det er bla.~denne egenskapen Pauli brukte ved den antisymmetriske 
egenfunksjonene
for \aa\ kunne tolke oppbyggingen av det periodiske systemet.

For et system av elektroner betyr dette at vi ikke kan plassere mer enn 
to elektroner i den laveste hydrogenorbitalen $1s$. Dette kommer vi tilbake
til i de neste avsnittene.
Da Pauli postulerte eksklusjonsprinsippet var dette svaret p\aa\ et 
fundamentalt problem som flere, deriblant Bohr hadde s\o kt etter en generell
forklaring for. At ikke mer enn to elektroner kan v\ae re i den innerste
$1s$ orbitalen pga.~eksklusjonsprinsippet, har store konsekvenser for
oppbyggingen av atomene, molekyler og de fleste materialer vi kjenner i dag.
Dersom alle elektronene kunne plasseres i den innerste orbitalen, ville alle
atomer ha oppf\o rt seg som edelgasser. Atomene ville da v\ae rt tiln\ae rmet
inerte og ville f.eks.~ikke kunne danne molekyler, og uten molekyler
ikke noe liv!

Den antisymmetriske funksjonen kan ogs\aa\ skrives som en $2\times 2$ 
determinant
\be
   \Psi_A({\bf r}_1,{\bf r}_2)=\frac{1}{\sqrt{2}}
\left| \begin{array}{cc} \psi_{\alpha}(1)& \psi_{\alpha}(2)\\\psi_{\beta}(1)&\psi_{\beta}(2)\end{array} \right|.
\ee 
For tre partikler har vi
\be
   \Psi_A({\bf r}_1,{\bf r}_2,{\bf r}_3)=\frac{1}{\sqrt{3!}}
\left| \begin{array}{ccc} \psi_{\alpha}(1)& \psi_{\alpha}(2)& \psi_{\alpha}(3)\\\psi_{\beta}(1)&\psi_{\beta}(2)&\psi_{\beta}(3)\\\psi_{\gamma}(1)&\psi_{\gamma}(2)&\psi_{\gamma}(3)\end{array} \right|.
\ee 
En slik determinant kalles for en Slater determinant.
Skriver vi ut denne $3\times 3$ determinanten f\aa r vi  6 ledd
\begin{eqnarray}
\Psi_A({\bf r}_1,{\bf r}_2,{\bf r}_3)&=
\frac{1}{\sqrt{3!}}\left[
\psi_{\alpha}(1)\psi_{\beta}(2)\psi_{\gamma}(3)+
\psi_{\beta}(1)\psi_{\gamma}(2)\psi_{\alpha}(3)+
\psi_{\gamma}(1)\psi_{\alpha}(2)\psi_{\beta}(3)-\right. \nonumber \\
&\left.\psi_{\gamma}(1)\psi_{\beta}(2)\psi_{\alpha}(3)-
\psi_{\beta}(1)\psi_{\alpha}(2)\psi_{\gamma}(3)-
\psi_{\alpha}(1)\psi_{\gamma}(2)\psi_{\beta}(3)
\right].
\end{eqnarray}
Hver av de lin\ae re kombinasjonene er en l\o sning av Schr\"odingers likning
for den samme totalenergien. Vi ser ogs\aa\ at funksjonen er antisymmetrisk
ved ombytte av to partikler og at den blir null dersom to partikler er ved
samme posisjon og har samme sett kvantetall.
En fler-elektron eller fler-fermion determinant basert p\aa\ en basis
av en-partikkel egenfunksjoner burde dermed v\ae re enkel \aa\ sette
opp.

\section{Heliumatomet}
%
\noindent
\begin{minipage}{0.45\textwidth}
%
Vi skal i det f{\o}lgende gi en kort oversikt over
hvordan vi bestemmer elektron konfigurasjonen
og de karakteristiske egenskaper for
grunntilstanden (tilstanden med st{\o}rst bindingsenergi)
i et atom innenfor det som heter {\bf sentralfelt  modellen}.

\hspace*{0.5cm}  Et atom karakteriseres ved et atomnummer Z.
Det angir samtidig antall elektroner i systemet.  I sentralfelt modellen
antar vi at den elektriske tiltrekningen mellom elektronene og
den indre atomkjernen samt  frast{\o}tningen elektronene imellom
gir opphav til et modifisert Coulombfelt. I dette feltet
beveger elektronene seg tiln{\ae}rmet uavhengig av hverandre
p{\aa} tilsvarende m{\aa}te som det ene elektronet i hydrog\'{e}n
atomet. Hvert enkelt elektron er karakterisert ved et sett kvantetall
$n, l,m_l, m_s$ med samme fysiske betydningen
som i hydrog\'{e}n atomet.

%
\end{minipage}
%
\hspace{0.05\textwidth}
%
\begin{minipage}{0.50\textwidth}
%
\begin{center}
\setlength{\unitlength}{1.0mm}
\begin{picture}(50,85)(0,-15)
\thicklines
%1s
\put(0,0){K}
\put(10,0){1s}
\put(20,0){\line(1,0){10}}
\put(35,0){2}
\put(45,0){2}
%2s2p
\put(0,12.5){L}
\put(10,10){2s}
\put(20,10){\line(1,0){10}}
\put(35,10){2}
\put(10,15){2p}
\put(20,15){\line(1,0){10}}
\put(35,15){6}
\put(45,15){10}
%3s3p
\put(0,27.5){M}
\put(10,25){3s}
\put(20,25){\line(1,0){10}}
\put(35,25){2}
\put(10,30){3p}
\put(20,30){\line(1,0){10}}
\put(35,30){6}
\put(45,30){18}

%4s3d4p
\put(0,45){N}
\put(10,40){4s}
\put(20,40){\line(1,0){10}}
\put(35,40){2}
\put(10,45){3d}
\put(20,45){\line(1,0){10}}
\put(35,45){10}
\put(10,50){4p}
\put(20,50){\line(1,0){10}}
\put(35,50){6}
\put(45,50){36}
%5s4d5p
\put(0,65){O}
\put(10,60){5s}
\put(20,60){\line(1,0){10}}
\put(35,60){2}
\put(10,65){3d}
\put(20,65){\line(1,0){10}}
\put(35,65){10}
\put(10,70){4p}
\put(20,70){\line(1,0){10}}
\put(35,70){6}
\put(45,70){54}
\put(0,-10){Skall}
\put(10,-10){Niv\aa\ }
\put(35,-10){$N_{e^-}$}
\put(45,-10){$\sum e^-$}
\end{picture}
\end{center}
%\end{center}
\end{minipage}
%
\vspace{0.1mm}

De mulige elektrontilstandene i et atom ut fra sentralfelt modellen
er vist i figuren ovenfor. Der vises
spektroskopisk notasjon, maksimalt  antall elektroner
($N_{e^-}$) i et skall og totalt antall elektroner opp til og med 
et gitt skall, ($\sum e^-$).
Vi har i figuren  en vertikal skala som angir energiforskjellen mellom
tilstandene karakterisert med kvantetallene  $(nl) $, og med 
$ (nl) = 1s$ som den laveste -  den sterkest
bundne tilstand.  Rekkef{\o}lgen av elektron tilstandene er noe
forskjellig fra det vi hadde for hydrog\'{e}n atomet
og skyldes at elektron frast{\o}tninger er inkludert.
Det vil v{\ae}re n{\o}dvendig med visse sm{\aa} modifikasjoner
i rekkef{\o}lgen for enkelte atomer, men i alminnelighet vil elektron
tilstandene v{\ae}re som vist p{\aa} figuren.

Legg merke til visse energi gap, omr{\aa}der hvor
avstanden  mellom to nabo tilstander er klart st{\o}rre enn gjennomsnittet.
Dette gir opphav til en skallstruktur
i atomene. Energi gap finner vi mellom $1s$ og $2s$, mellom $2p$
og $3s$ og mellom $3p$ og $4s$ osv. Energi tilstandene  mellom
to energi gap danner et {\sl \bf skall}. Hvert sett av $nl$
tilstander innen et skall kalles et {\bf \sl underskall}.
Antall elektroner som et underskall if{\o}lge
Pauliprinsippet kan inneholde, er vist p{\aa} figuren i kolonnen
$N_{e^-}$. Det er ogs{\aa} summert opp for hele skallet.

He atomet har 2 elektroner og fyller opp det innerste
$1s$ skallet. Den relativt store avstand til de neste
elektron tilstandene forklarer at He  er et spesielt  stabilt
grunnstoff. Samme fenomen opptrer ved de andre energi
gapene - Ne ved $Z = 10$, Ar ved $Z = 18$ osv..

Egenskapene til de forskjellige atomene er bestemt av elektron
konfigurasjonen og de {\bf \sl spektroskopiske kvantetallene}.
Vi skal n{\aa} definere hva det er og sette opp regler til {\aa} bestemme
dem for det enkelte atom.

La oss starte med et system av to elektroner.  Hvert
elektron har et banespinn $\vec{l}$ med tilh{\o}rende kvantetall $l$
og $m_l$.
For to-elektron systemet har vi et total banespinn,
gitt ved
%
\begin{equation}
\vec{L} = \vec{l_1} + \vec{l_2}.
\label{b1}
\end{equation}
Mer generelt har vi
\begin{equation}
\vec{L} = \sum_{i=1}^{N_{e^-}}\vec{l_i},
\end{equation}
hvor summen l\o per over antall elektroner $N_{e^-}$.
%
Det tilh{\o}rende kvantetallet $L$ er bestemt ut fra regelen
%
\begin{equation}
L = |l_1 - l_2|, |l_1 - l_2|+1, |l_1 - l_2| +2,\ldots ,l_1 + l_2,
\label{b2}
\end{equation}
%
og med $M_L = -L,\ldots ,+L$ 
Legg merke til at dette er  samme regel som ble brukt til {\aa}
sette sammen $\vec{L}$ og $\vec{S}$ til $\vec{J}$ i forbindelse
med spinn-bane koblingen  i hydrog\'{e}n atomet.\\[2ex]
%
{\sl Eksempel:} For $l_1 = 1$ og $l_2 = 2$, f{\aa}r vi fra regelen
ovenfor $L = 1, 2, 3$.
Videre har hvert elektron et egenspinn $\vec{s}$ med kvantetall $s = 1/2$.
Dette gir et totalt egenspinn
%
\begin{equation}
\vec{S} = \vec{s}_1 + \vec{s}_2,
\label{b3}
\end{equation}
%
for to-elektron systemet med kvantetall $ S = 0$ eller $1$
etter samme regel som i likning~(\ref{b2}).
Mer generelt har vi igjen
\begin{equation}
\vec{S} = \sum_{i=1}^{N_{e^-}}\vec{s_i},
\end{equation}
hvor summen igjen l\o per over antall elektroner $N_{e^-}$.


For He med begge elektronene i $(n,l) = (1,0)$ betyr dette
at $l_1 = l_2 = 0$, og likning (\ref{b2}) gir $L = 0$.
For egenspinnet har vi derimot to muligheter, $S = 0$ eller $S = 1$.
Hvilken skal vi velge?  Her kommer Pauliprinsippet til hjelp og krever
at den totale b{\o}lgefunksjonen m{\aa} v{\ae}re antisymmetrisk. La oss se
n{\ae}rmere p{\aa} hva dette betyr.

B{\o}lgefunksjonen for ett elektron kan skrives som $\psi_{l,m_l,m_s}$.
Kvantetallet $m_s$ har verdiene $\pm 1/2$. En alternativ  skrivem{\aa}te for
$m_s = + 1/2$ er $+$ og for $m_s = - 1/2$ $-$. 
Den totale b{\o}lgefunksjonen for to elektroner i He f{\aa} r formen
%
\begin{equation}
\psi(1, 2) = \sqrt{\frac{1}{2}}
	 \left (\psi_{1, 0, 0, +}(1) \psi_{1, 0, 0, -}(2)
			- \psi_{1, 0, 0, +}(2) \psi_{1, 0, 0, -}(1) \right ).
\label{b4}
\end{equation}
%
Romdelen av b{\o}lgefunksjonen for de to elektronene er symmetrisk,
mens egenspinndelen er antisymmetrisk. Det ser vi ved {\aa} omforme likning (\ref{b4})
til
%
\begin{equation}
\psi(1, 2) = \psi_{1, 0, 0}(1) \psi_{1, 0, 0}(2)
	         \sqrt{\frac{1}{2}} \left (| +, - > - | -, + > \right ).
\label{b5}
\end{equation}
For egenspinnet bruker vi betegnelsen $ | +, - >$ til {\aa} angi
at elektron nr.~1 har $m_s = + 1 / 2 $ (opp), og elektron nr.~2
har $m_s = - 1 / 2$ (ned). I tilstanden $ | -, + >$ er de byttet om.
Symmetri egenskapen til b{\o}lgefunksjonen $\psi( 1, 2 )$ kan vi
formulere slik
%
\begin{equation}
\psi( 1, 2 ) = \left ( \begin{array}{c}
					  romtilstand\\
					  symmetrisk
					  \end{array}
					  \right )
					  \times
					  \left ( \begin{array}{c}
					  spinntilstand\\
					  antisymmetrisk
					  \end{array}
					  \right ),
\label{b6}
\end{equation}
%
og den totale b{\o}lgefunksjonen er antisymmetrisk og tilfredsstiller
Pauliprinsippet.
%

To-elektron b{\o}lgefunksjonen for He i likning (\ref{b5}) tilfredsstiller
Pauliprinsippet og har totalt banespinn $ L = 0$. Neste sp{\o}rsm{\aa}l
blir da hvilket totalt egenspinn $S$  tilstanden har.

For en to-elektron egenspinn funksjon har vi
%
\begin{equation}
\chi_{S = 0}(1,2) = 	\sqrt{\frac{1}{2}} \left (| +, - > - | -, + > \right ).
\label{b7}
\end{equation}
%
Denne tilstanden har $S = 0$ og er antisymmetrisk. Det siste er lett
{\aa} se av formen. En annen mulighet er en egenspinn tilstand med
$S = 1$. Denne kommer i tre varianter avhengig av $M_S$.
%
\begin{equation}
\chi_{S = 1}(1,2) = \left \{
	\begin{array}{rll}
	&|+, + >           &M_S = +1\\
	\sqrt{\frac{1}{2}} & \left (| +, - > + | -, + > \right ).
							&M_S = \;\; 0\\
	&| -, - >          &M_S = -1
	\end{array}
	\right .
\label{b8}
\end{equation}
%
Disse tre tilstandene har $S = 1$ og er alle symmetriske,
hvilket ogs{\aa} tydelig fremg{\aa}r av formen.
Figur \ref{fig:heliumspinn} viser forksjellen i b\o lgefunksjoner.
Egenspinn tilstanden i likning (\ref{b7}) brukes i b{\o}lgefunksjonen i likning (\ref{b5}).
Men vi har ogs{\aa} situasjoner hvor romfunksjonen kan v{\ae}re antisymmetrisk.
Da bruker vi egenspinn funksjonen i likning (\ref{b8}).
\begin{figure}[hbtp]
\setlength{\unitlength}{1mm}
   \begin{picture}(100,160)
   \put(20,0){\epsfxsize=12cm \epsfbox{heliums.ps}}
   \end{picture}
\caption{Grafisk fremstilling av spinn $S=0$ 
og $S=1$ tilstander som summen av to spinn $s_1$ og $s_2$ med magnitude
$\hbar\sqrt{3}/4$. Hvert av spinnene kan bli funnet med lik sannsynlighet p\aa\ en kjegle som er symmetrisk om den vertikale $z$-aksen. Men deres orientering er korrelert slik at dersom et av spinnene peker i en bestemt retning s\aa\ vil
ogs\aa\ det andre peke i den samme generelle retning. Dersom deres $z$-komponenter er begge positive $s_{1_z}=s_{2_z}=+1/2\hbar$ eller begge er negative
$s_{1_z}=s_{2_z}=-1/2\hbar$, blir den totale $z$-komponenten henholdsvis 
$S_z=+\hbar$ eller $S_z=-\hbar$. Magnituden til totalspinnet er 
$S=\sqrt{2}\hbar$. Dersom enkeltspinnene har projeksjoner med ulike verdier, men peker i samme generelle retning, adderes spinn-projeksjonene
til $S_z=0$ men magnituden til totalspinnet er fremdeles $S=\sqrt{2}\hbar$, da 
vektoren $S$ befinner seg i planet vinkelrett $z$-aksen. Del a) av figuren
illustrerer dette. Da vi har tre muligheter for spinn-projeksjon kalles dette
for en triplett. For singlett tilstanden, peker spinnene i motsatt retning
og adderes til totalt spinn $S=0$ med projeksjon $S_z=0$. Vi kan si at de
to spinnene ikke er i fase med hverandre, som gjenspeiler det faktum at
vi har et minus tegn i spinnb\o lgefunksjonen  i likning (\ref{b7}).
For triplett tilstanden med $S_z=0$ er de to spinnene i fase, gjenspeilt 
i et pluss tegn i spinnb\o lgefunksjonen  i likning (\ref{b8}).
\label{fig:heliumspinn}}
\end{figure}


For He atomet betyr dette at grunntilstanden har $L = 0$ og $S = 0$.
Men vi trenger ett kvantetall
til, det totale spinn som  i hydrog\'{e}n ble betegnet med
$J$. Den generelle definisjonen er
%
\begin{equation}
\vec{J} = \vec{L} + \vec{S}.
\label{b9}
\end{equation}
%
I tilfelle He betyr dette $J = 0$.

Et samlet symbol for disse kvantetallene er
%
\begin{equation}
^{2 S + 1}L_J,
\label{b10}
\end{equation}
%
og vi f{\aa}r for He: $^{2 S + 1}L_J$ = $ ^{1}S_0$.

La oss n{\aa} anvende denne teorien til {\aa} lage en eksitert tilstand
i He, dvs. en tilstand som har h{\o}yere energi - mindre
bindingsenergi enn He i grunntilstanden. Dette kan vi gj{\o}re
ved {\aa} flytte ett av elektronene fra $1s$ til
$2s$ tilstanden. B{\o}lgefunksjonen f{\aa} r da formen
%
\begin{equation}
\psi(1, 2) = \sqrt{\frac{1}{2}}
\left ( R_{1s}(1) R_{2s}(2) + R_{1s}(2) R_{2s}(1) \right )
\chi_{S = 0}(1,2).
\label{b11}
\end{equation}
%
Romdelen av b{\o}lgefunksjonen er symmetrisk og egenspinndelen
er antisymmetrisk med $ S = 0$, se  likning (\ref{b7}). Spektroskopisk
betegnelse blir $^{1}S_0$. En annen mulighet er
 %
\begin{equation}
\psi(1, 2) = \sqrt{\frac{1}{2}}
\left ( R_{1s}(1) R_{2s}(2) - R_{1s}(2) R_{2s}(1) \right )
\chi_{S = 1}(1,2),
\label{b12}
\end{equation}
%
hvor romdelen er antisymmetrisk og egenspinndelen symmetrisk
med $S = 1$. Det gir spektroskopisk betegnelse
$^{3}S_1$. 
Dette kan vi oppsummere enten ved \aa\ sette
\begin{equation}
\psi( 1, 2 ) = \left ( \begin{array}{c}
					  romtilstand\\
					  symmetrisk
					  \end{array}
					  \right )
					  \times
					  \left ( \begin{array}{c}
					  spinntilstand\\
					  antisymmetrisk
					  \end{array}
					  \right ),
\end{equation}
eller
\begin{equation}
\psi( 1, 2 ) = \left ( \begin{array}{c}
					  romtilstand\\
					  antisymmetrisk
					  \end{array}
					  \right )
					  \times
					  \left ( \begin{array}{c}
					  spinntilstand\\
					  symmetrisk
					  \end{array}
					  \right ),
\end{equation}


Eksperimentelt har He-tilstanden i likning (\ref{b12}) med $S = 1$
st{\o}rre bindingsenergi enn tilstanden i likning (\ref{b11}) med $S = 0$.
Dette kan forklares ved at i den antisymmetriske romdelen av
b{\o}lgefunksjonen i likning (\ref{b12}) har de to elektronene i middel
st{\o}rre avstand enn i tilstand (\ref{b11}). Derved blir elektron
frast{\o}tningen mindre.
Legg merke til at for en symmetrisk romfunksjon kan de to elektronene
v{\ae}re p{\aa} samme sted i rommet, mens for den antisymmetriske
romfunksjonen er sannsynligheten for dette lik null.
Spektret for noen av de lavereliggende tilstandene i heliumatomet er
vist i Figur \ref{heliumspektrum}.
\begin{figure}[bp] 
\setlength{\unitlength}{1.0mm}
\begin{center}
\begin{picture}(140,140)(0,-30)
\thicklines
\put(5,-10){\line(0,1){140}}
\put(5,130){\line(1,0){115}}
\put(5,-10){\line(1,0){115}}
\put(120,-10){\line(0,1){140}}
%\multiput(5,0)(0,10){6}{\line(1,0){2}}
\thinlines
\put(-5,0){--110}
\put(-5,20){--100}
\put(-4,40){--90}
\put(-4,60){--80}
\put(-4,80){--70}
\put(-4,100){--60}
\put(-4,120){--50}
\put(10,2){\nl}
\put(10,-3){$n_1=1; n_2=1$}
\put(40,62){\nl}
\put(40,57){$n_1=1, l_1=0; n_2=1, l_2=0$, singlett}
\put(10,85){\nl}
\put(10,80){$n_1=1; n_2=2$}
\put(40,101){\nl}
\put(40,96){$n_1=1, l_1=0; n_2=2, l_2=0$}
\put(40,110){$n_1=1, l_1=0; n_2=2, l_2=1$}
\put(40,105){\nl}
\put(70,106){\nl}
\put(100,108){singlett}
\put(100,104){triplett}
\put(100,101){singlett}
\put(100,97){triplett}
\put(70,104){\nl}
\put(70,102){\nl}
\put(70,100){\nl}

\end{picture}
\end{center}
\caption{Energitilstander for heliumatomet. 
Energiene er i elektronvolt eV. Kolonnen til venstre angir resultatene for en uavhengig partikkel modell uten Coulomb vekselvirkningen mellom elektronene.
Den midterste kolonnen angir resultatet med Coulomb vekselvirkningen men
uten den fulle antisymmetriske b\o lgefunksjonen. Den siste kolonnen angir
de eksperimentelle verdiene for de lavest liggende tilstandene og deres 
teoretiske tolkninger.\label{heliumspektrum}}
\end{figure}

\section{System med mer enn to elektroner}


%
\begin{figure}
%
\begin{center}
%
\begin{pspicture}(10,6)

%%%%%%%%%%

%%%     linje 1   %%%

\rput(0,4.5){
             \rput(0,0){
                \psframe(0,0)(0.5,0.5)
              }
              \multiput(0,0.5)(0.5,0){4}{ 
                 \psframe(0,0)(0.5,0.5)
              }
              \rput(0.2,1.1){s}
              \rput(1.2,1.1){p}
              \rput(-0.3,.2){K}
              \rput(-0.3,.7){L}
              \rput(1.2,.2){H}

              \psline{->}(0.25,0.05)(0.25,0.45)

} %%* end rput linje 1

%%%  linje 2   %%%

\rput(0,3){
          \multiput(0,0)(2.5,0){2}  {
             \rput(0,0){
                \psframe(0,0)(0.5,0.5)   
                 \psline{->}(0.15,0.05)(0.15,0.45)
                 \psline{<-}(0.35,0.05)(0.35,0.45)

                \rput(0.2,1.1){s}
                \rput(1.2,1.1){p}
             }
             \multiput(0,0.5)(0.5,0){4}{ 
                 \psframe(0,0)(0.5,0.5)
             }
          }
          \rput(-0.3,.2){K}
          \rput(-0.3,.7){L}
          \rput(1.2,.2){He}
          \rput(3.7,.2){Li}
          \psline{->}(2.75,0.55)(2.75,0.95)

}  %* end rput linje 2

%%%   linje 3  %%%%

\rput(0,1.5)  {
             \multiput(0,0)(2.5,0){4}  {
%%%%%%%%%%
                \rput(0,0){
                   \psframe(0,0)(0.5,0.5)
                   \psline{->}(0.15,0.05)(0.15,0.45)
                   \psline{<-}(0.35,0.05)(0.35,0.45) 
                }  
                \multiput(0,0.5)(0.5,0){4}  {
                   \psframe(0,0)(0.5,0.5)
                }
                \rput(0.2,1.1){s}
                \rput(1.2,1.1){p}
                \psline{->}(0.15,0.55)(0.15,0.95)
                \psline{<-}(0.35,0.55)(0.35,0.95)
             }
             \rput(-0.3,0.2){K}  
             \rput(-0.3,0.7){L}
             \rput(1.2,0.2){Be}     
             \rput(3.7,0.2){B}   
             \rput(6.2,0.2){C}
             \rput(8.7,0.2){N}
             \multiput(2.5,0.55)(2.5,0){3} {
               \psline{->}(0.75,0.05)(0.75,0.45)
             }
             \multiput(5,0.55)(2.5,0){2} {
                 \psline{->}(1.25,0.05)(1.25,0.45)
              }
              \psline{->}(9.25,0.55)(9.25,0.95)

}    %% end linje 3 

%%%   linje 4   %%%
\rput(0,0)  {
             \multiput(0,0)(2.5,0){3} {
%%%%%%%%%%
                \rput(0,0){
                   \psframe(0,0)(0.5,0.5)
                   \psline{->}(0.15,0.05)(0.15,0.45)
                   \psline{<-}(0.35,0.05)(0.35,0.45)
                }
                \multiput(0,0.5)(0.5,0){4}  {
                   \psframe(0,0)(0.5,0.5)
                }
                \psline{->}(0.15,0.55)(0.15,0.95)
                \psline{<-}(0.35,0.55)(0.35,0.95)
                \psline{->}(0.65,0.55)(0.65,0.95)
                \psline{<-}(0.85,0.55)(0.85,0.95)

                \rput(0.2,1.1){s}
                \rput(1.2,1.1){p}
             }

             \psline{->}(1.25,0.55)(1.25,0.95)
             \psline{->}(1.75,0.55)(1.75,0.95)

             \multiput(2.5,0.55)(2.5,0){2}  {
                \psline{->}(1.15,0)(1.15,0.45)
                \psline{<-}(1.35,0)(1.35,0.45)
             }
             \psline{->}(4.25,0.55)(4.25,0.95)
             \psline{->}(6.65,0.55)(6.65,0.95)
             \psline{<-}(6.85,0.55)(6.85,0.95)

             \rput(-0.3,0.2){K}
             \rput(-0.3,0.7){L}
             \rput(1.2,0.2){O}
             \rput(3.7,0.2){F}
             \rput(6.2,0.2){Ne}
 }   %% end linje 4 




\end{pspicture}
%
\end{center}
\caption{Elektron konfigurasjonene for de 10 f{\o}rste grunnstoffene.\label{elek}}
\end{figure}
%

Vi skal n{\aa} bruke teorien p{\aa} systemer med flere elektroner. Med
utgangspunkt i de regler vi hittil har presentert kan vi sette opp
elektron konfigurasjonen til de 10 f{\o}rste grunnstoffene.
Dette er vist i figur \ref{elek}.
Symbolet $\uparrow \downarrow$ i figuren betyr at de to elektronene
er i en antisymmetrisk spinn tilstand med $S = 0$.
For de fem f{\o}rste grunnstoffen er
de spektroskopiske kvantetallene:
%
\begin{center}
H: $^2S_{1/2}$\hspace{0.5cm} He: $^1S_0$\hspace{0.5cm} Li: $^2S_{1/2}$
\hspace{0.5cm} Be: $^1S_0$\hspace{0.5cm} B: $^2P_J,
J = 1/2 \;\;eller\;\; 3/2$
\end{center}
%
St{\o}rre problemer blir det med de fem siste grunnstoffene.
Her trenger vi noen flere regler:
%
\begin{enumerate}
%
\item Hvis alle tilstandene i et underskall inneholder elektroner,
$(nl)^{2 (2 l + 1)}$, danner  det et {\sl lukket skall}.
En slik konfigurasjon har totalt $L = 0$ og $S = 0$.\\
Eksempel: Underskallet $2p$ i Ne inneholder
$2 \cdot (2 \cdot 1 + 1 ) = 6$ elektroner if{\o}lge figur~\ref{elek}.
Dette er det maksimalt tillatte  antall etter
Pauliprinsippet, og  konfigurasjonen har $L = S = 0$.
Det samme argumentet kan vi bruke p{\aa} skallet $1s$
og underskallet $2s$. Til sammen vil derfor de 10 elektronene i Ne
ha $L = S = J = 0$, dvs. $^1S_0$.
%
\item Grunnstoffet N har if{\o}lge figur~\ref{elek} 3 elektroner i $2p$
og spinnfunksjonen er $|+, +, +>$. Alle spinnvektorene er rettet
samme vei, og vi f{\aa}r $M_S = 1/2 + 1/2 + 1/2 = 3/2$ med $S = 3/2$
siden $ S \geq M_S$. Dette er et eksempel p{\aa}
{\sl \bf Hunds regel}: {\sl Elektronene fyller et underskall slik at
$S$ blir st{\o}rst mulig n{\aa}r vi tar hensyn til at
Pauliprinsippet skal v{\ae}re oppfylt.}
Forklaringen er f{\o}lgende:
Vi vet fra He atomet at vi f{\aa}r st{\o}rst
bindingsenergi  ved {\aa} minimalisere  elektron
frast{\o}tningen. Det oppn{\aa}r vi ved {\aa} antisymmetrisere
romdel av b{\o}lgefunksjonen, og etter Pauliprinsippet
betyr det {\aa} gj{\o}re  spinnfunksjonen mest mulig symmetrisk.
Dette er Hund's regel og vi ser den anvendt i figur~\ref{elek} b{\aa}de
for  C, $Z = 6$, N, $Z = 7$, O, $Z=8$ og F, $Z = 9$.
%
\item En regel til er viktig og kommer til anvendelse i tilfelle
F med $Z = 9$. Fra figur~\ref{elek} ser vi at $2p$ underskallet
inneholder 5 elektroner. Adderer vi til ett elektron, f\aa r
vi et lukket $2p$ underskall. Det gir $L = S = 0$.
Vi kan da skrive
%
\begin{equation}
\vec{L} = \vec{L}_1 + \vec{l},
\label{b13}
\end{equation}
%
hvor $L$ er banespinnet for $2p^6$, $L_1$ er banespinnet
for $2p^5$ og $l$ er banespinnet for et elektron i $2p$
skallet. De tilh{\o}rende kvantetallene blir $L = 0$, $l = 1$
og $L_1$. Bruker vi regelen i likning (\ref{b2}), f\aa r vi som eneste
mulighet $L_1 = 1$. Tilsvarende analyse kan vi gjennomf{\o}re
for egenspinnet med svar $S_1 = 1/2$.\\
Konklusjon: Har vi et underskall $nl$,
vil elektron konfigurasjonen
$(nl)^{2 (2 l + 1) - 1}$ ha samme banespinn og egenspinn som et enkelt
elektron i samme underskall.\\
Denne symmetrien betyr at $F, Z = 9$ og $B, Z = 5$ har samme
spektroskopiske kvantetall - $^2P_J$ med mulighetene
$J = 1/2 \;\;eller\;\; 3/2$.
%
\end{enumerate}
%
For en fullstendig beskrivelse av de 10 f{\o}rste grunnstoffene
i figur~\ref{elek} trenges enn{\aa} litt mer teori. Vi f{\aa}r ikke fastlagt
banespinnet for C, N og O n{\o}yaktig. Men det vil vi overlate til
videre studium i kvantefysikken. Vi kan likevel bruke teorien v{\aa}r
til {\aa} gi en god beskrivelse av mange grunnstoffer
i det periodiske system, ikke bare de som er angitt i figur~\ref{elek}.


\section{Alkalimetallene}
I tolkningen av atom\ae re spektra tar vi utgangspunkt 
i den enkle skallmodellen (sentralfelt modellen) for fler--elektron atomer.
Her 
plasseres elek\-tronene i
tilstander som har de samme kvantetallene som energi egentilstandene for
Hy\-dro\-gen atomet: 
\[
1s; 2s, 2p; 3s, 3p, 3d;\ldots 
\]
Et hovedskall karakteriseres ved kvantetallet $n$. Et sp\o rsm\aa l
som melder seg er hvor mange 
elek\-troner
kan plasseres i et hovedskall som funksjon av kvantetallet $n$.
I sentralfelt modellen sier Pauli prinsippet at to eller
flere elektroner ikke kan ha samme sett kvantetall.
Det f\o lger av det generelle kravet om at den totale
b\o lgefunksjonen for fermioner skal v\ae re antisymmetrisk. 
Dermed er
antall elektroner som vi kan ha for hver verdi av $l$,
som danner et underskall,  gitt
ved $2(2l+1)$, hvor to-tallet foran $2l+1$ kommer fra de
to mulige spinn 
projeksjoner et elektron kan ha. Et skall i sentralfelt
modellen er bestemt av kvantetallet $n$, og for hver 
$n$ verdi har vi $n-1$ mulige $l$ verdier. Totalt f\aa r
vi dermed $2n^2$ tilstander i hvert skall. 

Alkalimetallene er karakterisert ved at det har et s\aa kalt valenselektron
utenfor et lukka skall, slik som f.eks.~Litium atomet som har to
elektroner i $1s$ skallet og ett elektron  i $2s$ skallet. 
For Natriumatomet, som har elleve elektroner, har vi fylt opp
de tre indre skallene, $1s$, $2s$ og $2p$, slik at det siste elektronet 
plasseres i $3s$ skallet. Innenfor ramma av sentralfelt modellen vil 
da $2p$ elektronet i Natrium ha mindre bindingsenergi enn $2s$ elektronet.
Dette er forskjellig fra det vi har sett i hydrogenatomet.
\AA rsaken til at bindingsenergien til $2p$ orbitalen er 
litt mindre enn $2s$ orbitalen, er at et elektron fra denne
orbitalen f\o ler en frast\o ting fra elektronene i
$1s$ og $2s$ orbitalene, som er fylt. Vi kan dermed si at
frast\o tingen et elektron i $2p$ orbitalen f\o ler fra 
de fire elektronene
minsker dets energi i forhold til $2s$ orbitalen.
I Hydrogenatomet er ikke $1s$ og $2s$ orbitalene fylt og $2s$
og $2p$ tilstandene har samme energi. 
 

Litium og Natrium er eksempler p\aa\ metaller som kalles alkalimetallene,
som kjennetegnes ved at ionisasjonsenergien, se tilbake p\aa\ figur
\ref{51}, er liten i forhold til edelgass atomene som Helium og Neon. 
Det har som f\o lge at alkalimetallene er sv\ae rt reaktive. 

Alkalimetallene er og har v\ae rt ekstremt popul\ae re, og var de f\o rste
som blei brukt til \aa\ lage et Bose-Einstein kondensat via laser kj\o lning
og innfangning i magnetiske feller i sm\aa\ omr\aa der p\aa\ st\o rrelse med
noen f\aa\ $\mu$m. En viktig \aa rsak bak denne populariteten skyldes
at eksitasjonsspektret har frekvenser som er i omr\aa det for synlig lys,
noe som det gj\o r det enklere \aa\ generere ulike overganger. 
En grunn til for denne omsverminingen av alkaliene er at det er lett \aa\
lage atom\ae re str\aa ler. Atomene har et stort gasstrykk ved lave temperaturer, og en oppvarming av alkalimetaller i en ovn med en liten \aa pning
produserer en str\aa le av atomer som lett kan manipuleres med laserlys. 

Andre alkalimetaller er Kalium, Rubidium, Cesium og Francium.
Pr\o v \aa\ sette opp deres respektive grunntilstandsfunksjoner. 
Her skal vi begrense til \aa\ se p\aa\ Litium og Natrium.
%
\begin{figure}[h]
\begin{center}
{\centering
\mbox
{\psfig{figure=sixpack.ps,height=8cm,width=8cm}}
}
\end{center}
\caption{Atom\ae r sixpack av Cesium atomer. Fluorescerende lys sendt ut av seks Cesium atomer avsl\o rer et tett cluster p\aa\ ca.~10 $\mu$m av slike atomer,
avgrensa i et lite omr\aa de vha.~magnetiske felt og lasere. Tyske fysikere
demonstrerte nylig hvordan slike atomer kan fanges inn kun vha.~lasere,
et interessant utgangspunkt for manipulering av kvantemekaniske tilstander. 
For mer informasjon, se  Phys.~Rev.~Lett.~{\bf 85} (2000) 3777.
}
\end{figure}
\subsection{Litium atomet}

Litium atomet har i grunntilstanden konfigurasjonen $1s^22s^1$.
Her skal vi studere noen tilstander hvor
valens elektronet er i en eksitert tilstand. Figur \ref{litium} viser
energi niv{\aa}ene for
noen slike tilstander. Vi angir konfigurasjonen for valens
elektronet, og for enkelte niv{\aa}er ogs{\aa} termsymbolet.
For $2p$ og $3s$ niv{\aa}ene er eksitasjonsenergien
over $2s$ niv{\aa}et oppgitt i eV. Siden $2p$ orbitalen kommer rett
over $2s$ orbitalen, vil den f\o rste eksiterte tilstanden for Litium v\ae re
gitt ved konfigurasjonen $1s^22p^1$. Vi kan jo selvsagt ogs\aa\ eksitere
et elektron til h\o yere liggende orbitaler slik som f.eks.~$3s$ elller
$3d$ orbitalen. 
%
\addtocounter{figure}{1}
%
\begin{figure}[h]
%
\begin{picture}(465.26,210.83)%$li.pit 0
\thicklines
\put(22.0,25.45){\line(1,0){6.0}}
\put(24.10,25.45){\vector(0,1){190.59}}
\put(30.10,212.05){\makebox(0,0)[tl]{E (eV)}}
\put(48.21,30.45){\makebox(0,0)[tl]{2s}}
\put(59.89,25.45){\line(1,0){37.98}}
\put(69.74,36.0){\makebox(0,0)[tl]{\small 0,0 eV}}
\put(59.89,121.58){\line(1,0){37.25}}
\put(108.10,59.70){\makebox(0,0)[tl]{2p}}
\put(126.36,47.70){\line(1,0){35.79}}
\put(126.36,59.95){\line(1,0){35.79}}
\put(127.20,70.60){\makebox(0,0)[tl]{\small 1,8511 eV}}
\put(127.20,43.45){\makebox(0,0)[tl]{\small 1,8509 eV}}
\put(126.36,146.08){\line(1,0){36.79}}
\put(126.36,157.39){\line(1,0){36.79}}
\put(108.10,157.73){\makebox(0,0)[tl]{3p}}
\put(172.99,70.60){\makebox(0,0)[tl]{$^2P_{3/2}$}}
\put(172.99,51.04){\makebox(0,0)[tl]{$^2P_{1/2}$}}
\put(210.35,188.24){\makebox(0,0)[tl]{3d}}
\put(227.88,176.24){\line(1,0){38.71}}
\put(227.88,194.14){\line(1,0){37.98}}
\put(276.09,197.45){\makebox(0,0)[tl]{$^2D_{5/2}$}}
\put(276.09,181.64){\makebox(0,0)[tl]{$^2D_{3/2}$}}
\put(48.21,124.58){\makebox(0,0)[tl]{3s}}
\put(60.74,132.17){\makebox(0,0)[tl]{\small 3,3790 eV}}
%
\end{picture}
%
\caption{Energitilstander for Litium atomet. \label{litium}}
\end{figure}
%
Det er viktig \aa\ merke seg er at $p$ og $d$ niv{\aa}ene
er splittet i to, mens $s$ niv{\aa}ene ikke er splittet. I tillegg 
har f.eks.~niv{\aa}et $^2P_{1/2}$ lavere energi enn $^2P_{3/2}$.

Denne energisplittingen for $p$ og $d$ niv{\aa}ene
skyldes spinn-bane koblingen, som er proporsjonal
med $\vec{L} \cdot \vec{S}$, se tilbake p\aa\ avsnitt \ref{subsec:spinnbane}.
Bruker vi 
\[
   \OP{J}^2=(\OP{L}+\OP{S})(\OP{L}+\OP{S})=
   \OP{L}^2+2\OP{L}\OP{S}+\OP{S}^2,
\]
kan vi omskrive spinn-bane vekselvirkningen som
\[
  a \OP{L}\OP{S}=a\frac{\OP{J}^2-\OP{L}^2-\OP{S}^2}{2}.
\]
Bruker vi deretter at 
\[
  \OP{J}^2\psi = \hbar^2 J(J+1)\psi, 
\]
\[
  \OP{L}^2\psi = \hbar^2 L(L+1)\psi, 
\]
og 
\[
  \OP{S}^2\psi = \hbar^2 S(S+1)\psi, 
\]
og 
setter vi inn de respektive $J$, $S$ og $L$ verdiene for
de eksiterte tilstandene i dubletten finner vi at differansen
pga.~spinn-bane vekselvirkningen mellom $J=3/2$ og $J=1/2$ blir
\[
  \Delta E=\frac{3}{2}a\hbar^2.
\]
Spinn-bane vekselvirkningen forklarer ogs\aa\ hvorfor 
$^2P_{1/2}$ niv\aa et har en lavere energi enn $^2P_{3/2}$ niv\aa et.


\subsection{Na atomet}

Natrium (Na) atomet har 11 elektroner. Vi kan igjen bruke 
den enkle skallmodellen 
for elek\-tronene
til � bestemme elektronenes grunntilstandskonfigurasjon.
Grunn\-til\-stand\-en karakteriseres 
ved kvantetallene $ L,\; S,\; J$ satt sammen 
i termsymbolet
$^{2S + 1}L_J$. 
De 10 innerste elektronene danner  3 lukkede underskall
$1s^2$, $2s^2$ og $2p^6$ hvor kvantetallene $L = S = J =0$.
Spinnkvantetallene for hele systemet bestemmes derfor
av det siste elektronet i orbitalen
$3s$ til $L=0,\;\;S = 1 / 2,\;\; J = 1 / 2$. Det gir en spektroskopisk
notasjon gitt ved $^{2}S_{1/2}$ og en grunntilstandskonfigurasjon gitt ved 
$1s^22s^22p^63s^1$. 






I figur \ref{natatom} viser vi 
energiniv{\aa}ene for noen slike tilstander.
For $3p$ og $4s$ niv{\aa}ene er eksitasjonsenergien
over $3s$ niv{\aa}et oppgitt i eV.
%
\begin{figure}[h]
%
\begin{picture}(465.26,210.83)
\thicklines
\put(22.0,25.45){\line(1,0){6.0}}
\put(24.10,25.45){\vector(0,1){190.59}}
\put(30.10,212.05){\makebox(0,0)[tl]{E (eV)}}
\put(48.21,30.45){\makebox(0,0)[tl]{3s}}
\put(59.89,25.45){\line(1,0){37.98}}
\put(69.74,36.0){\makebox(0,0)[tl]{\small 0,0 eV}}
\put(59.89,121.58){\line(1,0){37.25}}

\put(108.10,59.70){\makebox(0,0)[tl]{3p}}

\put(126.36,47.70){\line(1,0){35.79}}
\put(126.36,59.95){\line(1,0){35.79}}
\put(127.20,70.60){\makebox(0,0)[tl]{\small 2.10436 eV}} 
\put(127.20,43.45){\makebox(0,0)[tl]{\small 2.10223 eV}}

\put(126.36,176.24){\line(1,0){36.79}}
\put(126.36,184.14){\line(1,0){36.79}}
\put(108.10,187.24){\makebox(0,0)[tl]{4p}}

\put(210.35,158.73){\makebox(0,0)[tl]{3d}}
\put(227.88,155.08){\line(1,0){38.71}}
\put(227.88,157.39){\line(1,0){37.98}}

\put(48.21,124.58){\makebox(0,0)[tl]{4s}}
\put(60.74,132.17){\makebox(0,0)[tl]{\small 3.19123 eV}}
%
\end{picture}
%

\caption{Energitilstander for Na atomet. \label{natatom}}

\end{figure}
%
Bruk resultatene fra foreg\aa ende avsnitt og pr\o v \aa\ klassifisere
konfigurasjonene for $3p$ og $3d$ orbitalene samt deres respektive
spektroskopiske faktorer $^{2S+1}L_J$. 



\section{Jordalkalimetaller}
Denne klassen av metaller er kjennetegna ved at vi har to valenselektroner
utenfor lukka skall, se igjen figur \ref{elek}. 
Eksempler er Beryllium, Magnesium og Kalsium. For Beryllium har vi
fylt opp $2s$ skallet ogs\aa\ . For Magnesium har vi to elektroner i
$3s$ skallet. Disse atomene har dermed en st\o rre ionisasjonsenergi
enn alkalimetallene, som vist ogs\aa\ i figur \ref{51}. 
 
For disse atomene  m\aa\ vi huske at det totale spinn, banespinn og
egenspinn er summene av de enkelte elektronenes ulike spinn. 

For egenspinnet betyr det at vi kan ha verdiene $0$ og $1$, siden vi legger
sammen egenspinnene til to elektroner med halvtallig spinn. 

For Beryllium vil, siden de to valenselektronene befinner seg i $2s$
orbitalen,  det totale banespinnet v\ae re gitt ved 
$L=0$, da $l_1=0$ og $l_2=0$, banespinnene
til henholdsvis elektron 1 og 2. Totalt spinn $J=L+S$ blir dermed 
enten  $0$ eller $1$. Konfigurasjonen for grunntilstanden er gitt
ved $1s^22s^2$.


For disse metallene er $2s�$ skallet ogs\aa\ fylt, det betyr
at vi m\aa\ ha 
$S=0$ og $L=0$ og dermed en spektroskopisk term gitt ved $^1S_0$.
Den romlige b\o lgefunksjonen er symmetrisk mens spinnb\o lgefunksjonen
m\aa\ v\ae re antisymmetrisk for at den totale b\o lgefunksjonen
skal v\ae re antisymmetrisk.

\section{Zeeman effekten for atomer}
Vi avslutter dette kapitlet med \aa\ se p\aa\ Zeeman effekten for atomer
og det som kalles for den anomale Zeeman effekten.

M\aa let v\aa rt er \aa\ beregne virkningen av det magnetiske
momentet n\aa r vi tar hensyn til spinn-bane vekselvirkningen
$V_{LS}\propto \OP{S}\OP{L}$.
Hvorfor? 
G\aa r vi tilbake til uttrykket for spinn-bane vekselvirkningen,
ser vi fra likning (\ref{eq:lsfinal2}) at spinn-bane vekselvirkningen
er proporsjonal med $Z^4$, hvor $Z$ er kjernens ladninsgtall.
Det betyr at for tyngre kjerner kan vi ikke neglisjere denne korreksjonen.

Som dere vil se av gangen nedenfor, skaper dette leddet noen komplikasjoner
i forhold til resultatene fra avsnitt \ref{subsec:zeeman}.
 
Tidligere, se tilbake p\aa\ avsnitt \ref{subsec:zeeman}, har vi sett 
fra v\aa r diskusjon om Zeeman effekten, at 
banespinnet gir opphav til et magnetisk moment
\[
   \OP{M}_L=-\frac{e}{2m}\OP{L},
\]
og 
dersom vi setter p\aa\ et magnetisk felt, f\aa r vi et kraftmoment 
som igjen gir et tillegg til potensiell energi gitt ved
\[
   \Delta E=-\OP{M}_L{\bf B}.
\]
Tilsvarende fant vi et magnetisk dipolmoment for egenspinnet gitt ved
\[
         \OP{M}_S=-g_S\frac{e}{2m}\OP{S},
      \]
med $g_S\approx 2$. 
Det totale magnetiske momentet $\OP{M}$ er dermed gitt ved 
\be
   \OP{M}=\OP{M}_L+\OP{M}_S= -\frac{e}{2m}\left(\OP{L}+g_S\OP{S}\right),
\ee
som resulterer i et energibidrag
\be
   \Delta E=-\OP{M}{\bf B}.
\ee
Merk at b\aa de $\OP{L}$ og $\OP{S}$ er gitt ved
\be
  \OP{L}=\sum_{i=1}^{N_e}\OP{l}_i,
\ee
og
\be
  \OP{S}=\sum_{i=1}^{N_e}\OP{s}_i,
\ee
hvor $\OP{l}_i$ er banespinnet til hvert enkeltelektron og $\OP{s}_i$
er det tilsvarende egenspinnet. 
Absoluttverdiene til $\OP{L}$ og $\OP{S}$ er gitt ved henholdsvis
\be
   |\OP{L}|=\hbar\sqrt{L(L+1)},
\ee
og
\be
   |\OP{S}|=\hbar\sqrt{S(S+1)},
\ee
da $\OP{L}^2\psi=\hbar^2L(L+1)\psi$ og $\OP{S}^2\psi=\hbar^2S(S+1)\psi$.
Det betyr ogs\aa\ at 
\be
   |\OP{M}|=-\frac{e\hbar}{2m}\left(\sqrt{L(L+1)}+g_S\sqrt{S(S+1)}\right). 
\ee
Men, siden vi har spinn-bane vekselvirkningen til stede har vi g\aa tt over
til \aa\ bruke det totale spinnet $J$, gitt ved
\[
\OP{J}=\OP{L}+\OP{S}.
\]
Vi skal merke oss er at $\OP{M}$ og $\OP{J}$ 
ikke er parallelle!
Banespinnet og egenspinnet setter opp et kraftmoment, uttrykt ved
det magnetiske momentet $\OP{M}$. Uten et ytre p\aa satt kraftmoment,
er det totale spinnet $J$ bevart og vi kan visualisere effekten 
av banespinn vekselvirkningen som om $L$ og $S$ dreier om
det resulterende spinnet. 
Siden det magnetiske momentet $\OP{M}$ heller ikke er 
parallellt med $\OP{J}$ kan vi forestille oss at ogs\aa\
det magnetiske momentet dreier om $\OP{J}$. 
Den gjennomsnittlige verdien av $\OP{M}$, dvs.~$\langle \OP{M} \rangle$
er gitt ved komponenten til $\OP{M}$ parallellt med $\OP{J}$, dvs.
\be
   \langle \OP{M} \rangle =(\OP{M}\frac{\OP{J}}{|\OP{J}|})
                           \frac{\OP{J}}{|\OP{J}|},
   \label{eq:opm1}
\ee
hvor 
\[
    \frac{\OP{J}}{|\OP{J}|},  
\]
er enhetsvektoren i $\OP{J}$ retningen. 
Setter vi inn for $\OP{M}$ og bruker at $\OP{J}=\OP{L}+\OP{S}$ kan vi skrive om
likning (\ref{eq:opm1}) som 
\be
   \langle \OP{M} \rangle =-\frac{e}{2m_e}
          \frac{\OP{J}^2+\OP{J}\OP{S}}{|\OP{J}|^2}\OP{J},
   \label{eq:opm2}
\ee
eller
\be
   \langle \OP{M} \rangle =-\frac{e}{2m_e}\left(1+
          \frac{\OP{J}\OP{S}}{|\OP{J}|^2}\right)\OP{J},
   \label{eq:opm3}
\ee
og med 
\be
\OP{L}\OP{L}=(\OP{J}-\OP{S})(\OP{J}-\OP{S})=\OP{J}^2-
                     2\OP{J}\OP{S}+\OP{S}^2
\ee
finner vi at 
\be
   \OP{J}\OP{S}=\frac{1}{2}\left(\OP{J}^2+\OP{L}^2+\OP{S}^2\right)   
\ee
som innsatt gir, n\aa r vi bruker 
\be
   \OP{J}^2\psi=\hbar^2J(J+1)\psi,
\ee
\be
   \OP{L}^2\psi=\hbar^2L(L+1)\psi,
\ee
og 
\be
   \OP{S}^2\psi=\hbar^2S(S+1)\psi,
\ee
at 
\be
   \langle \OP{M} \rangle =-\frac{e}{2m_e}g\OP{J}.
\ee
Her har vi introdusert 
Landes $g$-faktor gitt ved 
\be
g=1+\frac{J(J+1)+S(S+1)-L(L+1)}{2J(J+1)},
\ee
som for partikler med halvtallig spinn gir
\[
  g=1\pm\frac{1}{2l+1},
\]
da $j=l\pm 1/2$.

Dette uttrykket kan s\aa\ brukes til \aa\ rekne ut oppsplittingen
i energispekteret for et atom i et svakt magnetfelt i forhold til 
spinnbane vekselvirkningen. Dersom magnetfeltet er sterkt, vil resultatene
fra avsnitt \ref{subsec:zeeman} gjelde.

Velger vi $z$-aksen til \aa\ sammenfalle med magnetfeltets retningen,
finner vi at forandringen i energi kan skrives
\be
   \Delta E=-\langle \OP{M}\rangle {\bf B}=\frac{eg}{2m_e}\OP{J}_zB_z,
\ee
hvis virkning p\aa\ en egentilstand med kvantetall $J, M_J$ gir
\be
   \Delta E=\mu_BgM_JB_z.
\ee
hvor vi har brukt at Bohr magnetonen er definert som $\mu_B=e\hbar/2m_e$. 

Tabellen her viser oppsplittingen for Hydrogen for tilstandene
$^2S_{1/2}$ og $^2P_{3/2}$ med et ytre magnetfelt $B_Z=0.05$ T.
Energibidraget blir dermed
\[
   \Delta E=gM_J
     \left(5.79\times 10^{-5}\hspace{0.1cm}\mathrm{eV/T}\right)
     (0.05\hspace{0.1cm}\mathrm{T}).
\]
\begin{table}[h]
\caption{Resultater for den anomale Zeeeman effekten for Hydrogenatomet.}
\begin{center}
\begin{tabular}{ccccccc} \hline \\
   &L &S&J&g&$M_J$&$\Delta E \times 10^{-5}$ eV \\ 
   &  & & &  & & \\ \hline
&  & & &  & & \\    
$^2S_{1/2}$&0 &$\frac{1}{2}$ &$\frac{1}{2}$ & 2 &$\pm\frac{1}{2}$ & $\pm 0.290$    \\
&  & & &  & & \\
$^2P_{3/2}$&1 &$\frac{1}{2}$ &$\frac{3}{2}$ & $\frac{4}{3}$ &$\pm\frac{3}{2}$ & $\pm 0.579$    \\
&  & & &  & & \\
$^2P_{1/2}$&1 &$\frac{1}{2}$ &$\frac{1}{2}$ & $\frac{2}{3}$ &$\pm\frac{1}{2}$ & $\pm 0.193$ \\
&  & & &  & & \\ \hline 
\end{tabular}
\end{center}
\end{table} 



Vi avslutter med \aa\ studere spekteret til Kaliumatomet (med spinn-bane
vekselvirkningen)  
som plasseres i et svakt og homogent ytre magnetfelt
$\vec{B}$ (fra eksamensoppgaven i FYS 113 v\aa r 2000). 
Da vil 
vekselvirkningen mellom atomets magnetiske dipolmoment $\OP{M}$
og magnetfeltet
$\vec{B}$ gi opphav til et ekstra ledd $\OP{H}^{'}$ i
Hamiltonoperatoren
\be
\OP{H}^{'} = - \OP{M} \cdot \vec{B}.
\ee
$\OP{H}^{'}$ vil gi opphav til en splitting av energiniv\aa ene.

Vi skal beregne, uttrykt ved $\mu_{B} = e \hbar / 2 m$ og feltstyrken B,
energi oppsplittingen av grunntilstanden og de to f�rste eksiterte
tilstandene og deres degenerasjonsgrad.
%
Utvalgsreglene for elektromagnetiske dipoloverganger er $\Delta j = 0, \pm 1$;
$\Delta m_{j} = 0, \pm 1$; \mbox{$\Delta l = \pm 1$}.

Kaliumatomet har 19 elektroner, noe som betyr at grunntilstanden er gitt ved
konfigurasjon $1s^22s^22p^63s^23p^64s^1$ og med $J=1/2$, $L=0$ og
$S=1/2$. Den f\o rste eksiterte tilstanden har konfigurasjonen 
$1s^22s^22p^63s^23p^64p^1$ med $J=1/2$ eller $J=3/2$, $L=1$ og
$S=1/2$. 
Energidifferansen mellom grunntilstanden og den f\o rste eksiterte
tilstanden er gitt ved $1.617$ eV. 


Tilstanden med $J=3/2$ har en eksitasjonsenergi,
n\aa r vi tar med spinn-bane koplingen gitt ved
\[
  E_{J=3/2}=1.617 + \frac{1}{2}a\hbar^2=1.621 \hspace{0.1cm} \mathrm{eV},
\]
mens tilstanden med $J=1/2$ har
\[
  E_{J=1/2}=1.617 - a\hbar^2=1.610 \hspace{0.1cm} \mathrm{eV}.
\] 
B\o lgelengden for overgangen til grunntilstanden er gitt
for tilstanden med $J=1/2$ ved
\[
  \lambda_{J=1/2}=\frac{hc}{1.610\hspace{0.05cm} \mathrm{eV}}=
  \frac{1240\hspace{0.05cm} \mathrm{eVnm}}{1.610\hspace{0.05cm} \mathrm{eV}}=770.2 \hspace{0.1cm} \mathrm{nm},
\]
mens b\o lgelengden for overgangen til grunntilstanden er gitt
for tilstanden med $J=3/2$ ved
\[
  \lambda_{J=3/2}=\frac{hc}{1.621\hspace{0.05cm} \mathrm{eV}}=
  \frac{1240\hspace{0.05cm} \mathrm{eVnm}}{1.621\hspace{0.05cm} \mathrm{eV}}=765.0 \hspace{0.1cm} \mathrm{nm}.
\]
Et eventuelt spektrometer som skal m\aa le disse energidifferansene
m\aa\ derfor minst ha en oppl\o sning
p\aa\ $5.2$ nm. 

De ulike tilstandene gir ogs\aa\ ulike Landes faktorer $g$.
For tilstanden $^2S_{1/2}$ er $L = 0,\;\; S = 1 / 2,\;\;
J = 1 / 2$ som gir $ g = 2$.
For tilstanden $^2P_{1/2}$ er $L = 1,\;\; S = 1 / 2,\;\;
J = 1 / 2$ som gir $ g = \frac{2}{3}$.
For tilstanden $^2P_{1/2}$ er $L = 1,\;\; S = 1 / 2,\;\;
J = 3 / 2$ som gir $ g = \frac{4}{3}$.
%
Den generelle formelen for energisplittingen blir
$\Delta E(^{2S+1}L_J) = \mu_B g B M_J$. Dette gir
%
\[
\Delta E(^{2S+1}L_J) =
\left \{
\begin{array}{rcl}
2\mu_B B M_J \;\;\;\; & for\;\; & ^{2}S_{1/2}\\
\frac{2}{3}\mu_B B M_J \;\;\;\; & for & ^{2}P_{1/2}\\
\frac{4}{3}\mu_B B M_J \;\;\;\; & for & ^{2}P_{3/2}
\end{array}
\right .
\]
%
Ingen tilstander har n{\aa} samme energi slik at
degenerasjonsgraden blir lik 1.
%
Uten magnetfelt f{\aa}r vi 2 spektrallinjer:
$^{2}P_{1/2} \longrightarrow ^{2}S_{1/2}$ og
$^{2}P_{3/2} \longrightarrow ^{2}S_{1/2}$.
Med magnetfelt:
$^{2}P_{1/2} \longrightarrow ^{2}S_{1/2}$
splittes opp i 4 linjer. Energy forskyvningen p{\aa} grunn av magnetfeltet
blir $\{+4/3, +2/3, -2/3, -4/3\} \times \mu_B B$.
Og $^{2}P_{3/2} \longrightarrow ^{2}S_{1/2}$ gir 6 linjer
med en energiforskyvning
$\{+5/3, +1, +1/3, -1/3, -1, -5/3\}\times \mu_B B$.
\section{Oppgaver}
\subsection{Analytiske oppgaver}
\subsubsection*{Oppgave 9.1, Eksamen V-2001}
%
Vi skal studere grunntilstanden i helium atomet. F{\o}lgende opplysninger
kan v{\ae}re til nytte:
Energien til et atom (eller ion) som best{\aa}r av ett elektron
og en kjerne med ladning $Z e$, er i grunntilstanden
%
\[
E_1 = - Z^2 e^2 / ( 8 \pi \epsilon_0 a_0),
\;\;\; hvor\;\;\;
a_0 = 4 \pi \epsilon_0 \hbar^2 / (m_e e^2).
\]
%
B{\o}lgefunksjonen er da
%
\[
\psi_{1,0,0}(\vec{r}) = \frac{(Z / a_0)^{3/2}}{\sqrt{\pi}}
\exp(-Zr/a_0).
\]
%
Energien av hydrog\'{e}n atomet i grunntilstanden $= -13,6$~eV.
%
\begin{itemize}
%
\item[a)]	Anta ( i dette ene sp{\o}rsm{\aa}let) at det overhodet
ikke er noen vekselvirkning mellom elektronene. Hvilken energi har i
s{\aa} fall helium atomet i grunntilstanden?
\end{itemize}
%
Helium atomet har i grunntilstanden konfigurasjonen $1 s^2$.
For $1s$--orbitalen bruker vi en
b{\o}lgefunksjon av typen
%
\[
g(r) = \frac{\beta^{3/2}}{\sqrt{\pi}} \exp( -\beta r);\;\;\;
\beta = konstant.
\]
%
Vi oppgir rom--integralene
%
\[
\int \left \{ g(r)\right \}^2 d\vec{r} = 1; \;\;\;
\int \left \{ g(r)\right \}^2 \frac{1}{r} d\vec{r} = \beta; \;\;\;
\int g(r) \nabla^2 g(r) d\vec{r} = -\beta^2,
\]
%
og det dobbelte rom--integralet
%
\[
\int \int \left \{ g(r_1) \right \}^2 \left \{ g(r_2) \right \}^2
				\frac{1}{r_{12}} dV_1 dV_2 = \frac{5}{8} \beta
					 \;\;\; hvor \;\;\; r_{12} = | \vec{r_1} - \vec{r_2} |.
\]
%
\begin{itemize}
%
\item[b)] Skriv opp Hamilton operatoren $\OP{H}$  for helium atomet.
(Vi regner at kjernen er i ro).
%
\item[c)] Vis at forventningsverdien (middelverdien) 
$\langle \OP{H}\rangle$
av energien i den gitte konfigurasjonen er
%
\[
\langle\OP{H} \rangle = \frac{\hbar^2}{m_e} \beta^2
		  - \frac{27}{32} \frac{e^2}{\pi \epsilon_0} \beta.
\]
%
\end{itemize}
%
Vi kan n{\aa} finne energien i grunntilstanden ganske n{\o}yaktig ved
et variasjonsprinsipp. Det sier at b{\o}lgefunksjoner som ikke er
lik den eksakte energi egenfunksjonen for
grunntilstanden, alltid gir en forventningsverdi for energien  som er
st{\o}rre enn den virkelige egenverdien.
Metoden g{\aa}r ut p{\aa} {\aa} finne den b{\o}lgefunksjonen
(av typen $g(r)$ ), som gir den minste
forventningsverdien for energien.
%
\begin{itemize}
%
\item[d)] Hvilken (tiln{\ae}rmet) verdi for energien i grunntilstanden
finner vi etter den skisserte metoden?\\
Til kontroll: Eksperimentelt er energien $-78,98 $~eV.
\end{itemize}
%
Den modellen som vi etter dette har av helium atomet i grunntilstanden,
er karakterisert ved at virkningen av det ene elektronet p{\aa} det
andre skjer ved en avskjerming av
kjerneladningen, slik at hvert elektron er i en stasjon{\ae}r tilstand
i feltet fra en viss effektiv kjerneladning.

\begin{itemize}
%
\item[e)] Hvor stor er den effektive kjerneladningen?
\end{itemize}
%
\subsubsection*{Oppgave 9.2}
%
I den enkle skallmodellen for fler--elektron atomer plasseres elektronene i
tilstander som har de samme kvantetallene som energi egentilstandene for
H--atomet:\\
$1s; 2s, 2p; 3s, 3p, 3d;\ldots ;nl = n0, nl = n1, \ldots, nl = n,n-1$.
%
\begin{itemize}
%
\item[a)] Hvor mange elektroner kan det plasseres i hver av tilstandene
$1s; 2s, 2p; 3s, 3p, 3d;\ldots ;$ $ nl = n0, \ldots , nl = n,n-1$?
Formul\'{e}r det prinsippet som bestemmer hvor mange elektroner
det kan v\ae re i hver tilstand.
%
\end{itemize}
%
I H--atomet har tilstandene $nl = n0, nl = n1\ldots , nl = n,n-1$ alle
samme energi. De er degenererte. I et fler--elektron
atom, er dette ikke lenger tilfelle.
%
\begin{itemize}
%
\item[b)] Diskut\'{e}r \aa rsaken til dette.
%
\end{itemize}
%
Atomene He, Be og Ne har henholdsvis 2, 4 og 10 elektroner.
Grunntilstanden for
disse tre atomer betegnes med symbolet
$^{1}S_{0} \left( ^{2S+1}L_{J}\right)$.

%
\begin{itemize}
%
\item[c)] Skriv ned verdien av kvantetallet $S$ for det
totale egenspinn, $L$
for det totale banespinn og $J$ for det totale spinn for He, Be og Ne i
grunntilstanden.
%
\end{itemize}
%
Na--atomet har 11 elektroner. Grunntilstanden er gitt
ved symbolet $^{2}S_{1/2}$.
%
\begin{itemize}
%
\item[d)] Begrunn at verdiene av grunntilstandens spinn kvantetall er
bestemt av valenselektronets kvantetall.
%
\end{itemize}
%
De to laveste eksiterte tilstandene i Na--atomet er $^{2}P_{1/2}$ og
$^{2}P_{3/2}$.

%
\begin{itemize}
%
\item[e)] Angi hvordan spinn kvantetallene for disse tilstandene bestemmes
av valenselektronets kvantetall.
%
\end{itemize}
%
Tilstandene $^{2}P_{1/2}$ og $^{2}P_{3/2}$ ligger henholdsvis 2,1022 eV og
2,1044 eV over grunntilstanden.
%
\begin{itemize}
%
\item[f)] Diskut\'{e}r kort og kvalitativt \aa rsaken til energiforskjellen
mellom tilstandene $^{2}P_{1/2}$ og $^{2}P_{3/2}$. Hvor stor er
degenerasjonsgraden for hvert av niv\aa ene og hva er \aa rsakene til denne
degenerasjonen?
%
\end{itemize}
%
Dersom Na atomet er i en av tilstandene
$^{2}P_{1/2}$ og $^{2}P_{3/2}$ vil
det g\aa ~over til grunntilstanden og emittere et foton.
%
\begin{itemize}
%
\item[g)] Hvor stor b\o lgelengdeoppl\o sning
m\aa ~en minst ha for \aa ~skille
spektrallinjene fra overgangene $^{2}P_{1/2}$ $\rightarrow$ $^{2}S_{1/2}$ og
$^{2}P_{3/2}$ $\rightarrow$ $ ^{2}S_{1/2}$ i et spektrometer?
%
\end{itemize}
%
N\aa r vi tar hensyn til egenspinn--banespinn kobling (spin--banekobling) er
operatoren $\OP{\vec{M}}$ for det magnetiske dipol
momentet gitt ved
%
\[
\OP{\vec{M}} = -\frac{e}{2m} g \OP{\vec{J}},
\]
%
der $m$ er elektronets hvilemasse,
$\OP{\vec{J}}$ er operatoren for det
totale spinnet og
%
\[
g = 1 + \frac{j(j+1) + s(s+1) -l(l+1)}{2j(j+1)}.
\]
%
\begin{itemize}
%
\item[h)] Vis at $g = 2$ for tilstanden $^{2}S_{1/2}$, $g = \frac{4}{3}$ for
tilstanden $^{2}P_{3/2}$ og $g = \frac{2}{3}$ for tilstanden $^{2}P_{1/2}$.
%
\end{itemize}
%
Dersom Na atomet plasseres i et svakt og homogent ytre magnetfelt
$\vec{B}$ vil
vekselvirkningen mellom atomets magnetiske dipolmoment $\vec{M}$
og magnetfeltet
$\vec{B}$ gi opphav til et ekstra ledd $\OP{H}^{'}$ i
Hamiltonoperatoren
%
\[
\OP{H}^{'} = - \OP{\vec{M}} \cdot \vec{B}.
\]
%
$\OP{H^{'}}$ vil gi opphav til en splitting av energiniv\aa ene.
%
\begin{itemize}
%
\item[i)] Beregn, uttrykt ved $\mu_{B} = e \hbar / 2 m$ og feltstyrken B,
oppsplittingen av niv\aa ene $^{2}S_{1/2}$, $^{2}P_{1/2}$ og $^{2}P_{3/2}$ og
angi de nye energiniv\aa enes kvantetall og deres degenerasjonsgrad.
%
\end{itemize}
%
Utvalgsreglene for elektromagnetiske dipoloverganger er $\Delta j = 0, \pm 1$;
$\Delta m_{j} = 0, \pm 1$; \mbox{$\Delta l = \pm 1$}.
%
\begin{itemize}
%
\item[j)] Tegn figurer som viser hvordan overgangene
$^{2}P_{1/2}$ $\rightarrow$
 $^{2}S_{1/2}$ og $^{2}P_{3/2}$ $\rightarrow$ $ ^{2}S_{1/2}$ og de tilsvarende
spektrallinjene blir modifisert av magnetfeltet $\vec{B}$.
%
\end{itemize}
%
%
\subsubsection*{L\o sning}
%
\begin{itemize}
% 
\item[a)]I sentralfeltmodellen er kvantetallene bestemt etter f�lgende
regler
%
\begin{eqnarray}
n &=& 1, 2, 3, \ldots\\
l &=& 0, 1, 2, \ldots, n-1\\
m_l &=& -l, -l + 1, -l + 2, \ldots, +l\\
m_s &=& -1/2, +1/2
\end{eqnarray}
%
Pauliprinsippet krever at b�lgefunksjonen for elektroner
\underline{m�} v�re antisymmetrisk. I sentralfeltmodellen betyr det at
ingen elektroner i et mange-elektronsystem kan ha samme sett
kvantetall.

I en tilstand $(nl)$ kan vi derfor ha $2 \cdot (2 l + 1)$ elektroner,
alle med forskjellig verdier av kvantetallene $m_l, m_s$.
% 
\item[b)] I et fler-elektron system inneholder den potensielle
energien bidrag fra Columbfrast�tning mellom elektronene  som ikke forekommer
i et Hydrog\'{e}n atom. Dette har som konsekvens at tilstander $(nl)$
med samme $n$, men forskjellig $l$, ikke har samme energi.
%
\item[c)]Atomene He, Be og Ne har f�lgende elektron konfigurasjon
med tilh�rende kvantetall L, S, J
% 
\begin{eqnarray}
 \mbox{He :}& \quad 1s^2 \quad &L = 0,\;\; S = 0\;\; J = 0 
\quad \mbox{Spektroskopisk kvantetall :}\; ^1S_0\\
\mbox{Be :}& \quad 1s^22s^2 \quad &L = 0,\;\; S = 0\;\; J = 0 
\quad \mbox{Spektroskopisk kvantetall :}\; ^1S_0\\
\mbox{Ne :}& \quad 1s^22s^22p^6  \quad &L = 0,\;\; S = 0\;\; J = 0 
\quad \mbox{Spektroskopisk kvantetall :} \;^1S_0
\end{eqnarray}
% 
Alle elektronkonfigurasjonene danner lukkede skall og har f�lgelig
$L = S = J = 0$.
%
\item[d)] Na har elektronkonfigurasjonen $1s^22s^22p^63s$. De f�rste 10
elektronene danner lukkede skall med $L = S = J = 0$, Kvantetallene
$L,S,J$ for Na er derfor bestemt av det siste elektronet til $L = 0,
\; S = 1/2,\; J = 1/2$.
%
\item[e)] De to f�rste eksiterte tilstandene i Na har
elektronkonfigurasjon $1s^22s^22p^63p$ og kvantetallene $L,S,J$
bestemmes igjen av kvantetallene for det siste elektronet, n� i en
$3p$ tilstand. Dette gir $ L = 1,\; S = 1/2$ og med to muligheter
$J =  1/2,\;  3/2$.
%
\item[f)] Energiforskjellen mellom tilstandene $^2P_{1/2}$ og 
$^2P_{3/2}$ skyldes spinn-bane leddet i Hamiltonfunksjonene
% 
\[
\OP{H} = \OP{H}_0 + A \OP{\vec{L}}\cdot \OP{\vec{S}}
\]
%
Med de kvantetallene som er angitt ovenfor gir dette energier
% 
\begin{eqnarray}
E_{3/2} 
&=& E(3p) + \frac{1}{2} A \left ( J(J + 1) - L(L +1) - S(S+1) \right )
 = E(3p) + \frac{1}{2} A \hbar^2 \\
E_{1/2} 
&=& E(3p) + \frac{1}{2} A \left ( J(J + 1) - L(L +1) - S(S+1) \right )
 = E(3p) - A \hbar^2 \\
\end{eqnarray}
%

I tillegg til kvantetallene $L, S, J$ har tilstandene kvantetallet
$M_J$ med mulige verdier $ M_J = -J, -J + 1, \ldots ,+J $. Dette gir
fire muligheter (degenerasjon) for $J= 3/2$ og  
 to muligheter (degenerasjon) for $J= 1/2$. Alle disse tilstandene har
samme energi s� lenge systemet ikke er p�virket av et ytre magnetfelt.
%
\item[g)] Dipoloverganger i Na
%
\begin{eqnarray}
^2P_{3/2} &\longrightarrow &^2S_{1/2}\quad 
\lambda_1=  \frac{hc}{2,1044\mbox{eV}} 
 \approx 5.9081 \cdot 10^{-7}\mbox{m}\\
^2P_{1/2} &\longrightarrow &^2S_{1/2}\quad 
\lambda_2 =
\frac{hc}{2,1022\mbox{eV}}  \approx 5.9019 \cdot 10^{-7}\mbox{m} 
\end{eqnarray}
%
som gir et krav til en b�lgelengde oppl�sning p� 
 $\Delta \lambda = \lambda_2 - \lambda_1 \approx 0,6 \mbox{nm}$
%
\item[h)]
%
\begin{eqnarray}
^2S_{1/2}& \quad &L = 0\; S = 1/2\; J = 1/2 \quad 
\mbox{gir}\;\; g = 2\\
^2P_{3/2}& \quad &L = 1\; S = 1/2\; J = 3/2 \quad 
\mbox{gir}\;\; g = 4/3\\
^2P_{1/2}& \quad &L = 1\; S = 1/2\; J = 1/2 \quad 
\mbox{gir}\;\; g = 2/3
\end{eqnarray}
%
\item[i)] Ved bruk av Land\'{e}'s g-faktor for atomer i et ytre
magnetfelt f�r vi et ekstra bidrag til energien (Hamiltonfunksjonen)
p� formen 
%
\[
\OP{H}_B = -\OP{\vec{M}} \cdot \vec{B}
         = \mu_B g \frac{1}{\hbar} \OP{\vec{J}} \cdot \vec{B}
         = \mu_B g B \frac{1}{\hbar} \OP{J}_z
\]
% 
Energibidraget for de tre tilstandene blir da 
% 
\[
\langle \OP{H}_B \rangle
= \mu_B B \left \{ \begin{array}{ll}
                       2 M_j    \quad & ^2S_{1/2}\\
                       2/3 M_j  \quad & ^2P_{1/2}\\
                       4/3 M_j  \quad & ^2P_{3/2}
                 \end{array} \right .
\]
%
  
\end{itemize}



\subsubsection*{Oppgave 9.3}
%
Det n\o ytrale karbon atomet har 6 elektroner i skall rundt kjernen.
%
\begin{itemize}
%
\item[a)] Angi elektronkonfigurasjonen for atomet n\aa r det er i
grunntilstanden.

\item[b)] Hvilke mulige verdier finnes for den totale orbitale dreieimpulsen $L$ og
det totale spinnet $S$?

\item[c)] Bruk Hunds regel til \aa ~bestemme $S$.
Hvilke mulige verdier kan vi ha for $L$ i grunntilstanden og
hva gir dette i spektroskopisk notasjon.
%
\end{itemize}
%

\subsubsection*{Oppgave 9.4}
%
I det opprinnelige Stern-Gerlach eksperimentet ble en str\aa le av Ag--atomer
sendt gjennom et inhomogent magnetfelt $B$ som splittet den i to.
Elektronkonfigurasjonen for s\o lv i grunntilstanden er
$1s^{2}2s^{2}2p^{6}3s^{2}3p^{6}3d^{10}4s^{2}4p^{6}4d^{10}5s$.

%
\begin{itemize}
%
\item[a)] Hva er spinnkvantetallene $L$ og $S$ for dette atomet n\aa r det er i
grunntilstanden?

\item[b)] Hvis et elektron fra $4d$-skallet eksiteres opp til $5p$-skallet,
hvilke muligheter kan n\aa ~kvantetallene $L$ og $S$ ta?

\item[c)] Beregn det magnetiske momentet til et s\o lvatom i grunntilstanden
uttrykt i enheten $J/T$.

\item[d)] Hva blir energiforskjellen m\aa lt i $eV$ mellom to atomer i hver sin
splittede str\aa le hvis magnetfeltets st\o rrelse er $B$ = 0,1 $T$?

\item[e)] Hvilke verdier kan z-komponenten av det magnetiske momentet for
s\o lv atomet ta hvis $5s$-elektronet eksiteres opp til $5p$-orbitalen? Se her
bort fra $LS$-koblingen.
%
\end{itemize}
%

%
\subsubsection*{Oppgave 9.5, Eksamen V-1992}
%
Litium atomet har i grunntilstanden konfigurasjonen $1s^2 2s$.
I denne oppgaven skal vi studere noen tilstander hvor
valens elektronet er i en eksitert tilstand. Figur~\ref{fig3.1} viser
energi niv{\aa}ene for
noen slike tilstander. Vi angir konfigurasjonen for valens
elektronet, og for enkelte niv{\aa}er ogs{\aa} termsymbolet.
For $2p$ og $3s$ niv{\aa}ene er eksitasjonsenergien
over $2s$ niv{\aa}et oppgitt i $eV$.
%
\begin{figure}[h]
%
\begin{picture}(465.26,210.83)
\thicklines
\put(22.0,25.45){\line(1,0){6.0}}
\put(24.10,25.45){\vector(0,1){190.59}}
\put(30.10,212.05){\makebox(0,0)[tl]{E (eV)}}
\put(48.21,30.45){\makebox(0,0)[tl]{2s}}
\put(59.89,25.45){\line(1,0){37.98}}
\put(69.74,36.0){\makebox(0,0)[tl]{\small 0,0 eV}}
\put(59.89,121.58){\line(1,0){37.25}}
\put(108.10,59.70){\makebox(0,0)[tl]{2p}}
\put(126.36,47.70){\line(1,0){35.79}}
\put(126.36,59.95){\line(1,0){35.79}}
\put(127.20,70.60){\makebox(0,0)[tl]{\small 1,8511 eV}}
\put(127.20,43.45){\makebox(0,0)[tl]{\small 1,8509 eV}}
\put(126.36,146.08){\line(1,0){36.79}}
\put(126.36,157.39){\line(1,0){36.79}}
\put(108.10,157.73){\makebox(0,0)[tl]{3p}}
\put(172.99,70.60){\makebox(0,0)[tl]{$^2P_{3/2}$}}
\put(172.99,51.04){\makebox(0,0)[tl]{$^2P_{1/2}$}}
\put(210.35,188.24){\makebox(0,0)[tl]{3d}}
\put(227.88,176.24){\line(1,0){38.71}}
\put(227.88,194.14){\line(1,0){37.98}}
\put(276.09,197.45){\makebox(0,0)[tl]{$^2D_{5/2}$}}
\put(276.09,181.64){\makebox(0,0)[tl]{$^2D_{3/2}$}}
\put(48.21,124.58){\makebox(0,0)[tl]{3s}}
\put(60.74,132.17){\makebox(0,0)[tl]{\small 3,3790 eV}}
%
\end{picture}
%
\caption{\label{fig3.1}Energitilstander for Litium atomet.}
\end{figure}
%
\begin{itemize}
%
\item[a)] Forklar hva et termsymbol som $^2D_{5/2}$ forteller
om tilstandens kvantetall. Hvilken degenerasjonsgrad har energi
niv{\aa}et $^{2}D_{5/2}$ (dvs. antall forskjellige tilstander
med samme energi)?
Angi ogs{\aa} degenerasjonsgraden for $3s$ niv{\aa}et.
%
\item[b)] Gi en kort forklaring p{\aa} hvorfor $p$ og $d$ niv{\aa}ene
er splittet i to, mens $s$ niv{\aa}ene ikke er splittet.
Hvorfor har niv{\aa}et $^2P_{1/2}$ lavere energi enn $^2P_{3/2}$?
%
\end{itemize}
%
Vi tenker oss n{\aa} at litium atomet er i en
$3s$ tilstand, og at det deretter ikke er p{\aa}virket
av omgivelsene.
%
\begin{itemize}
%
\item[c)]  Gj{\o}r rede for de forskjellige elektriske
dipol overganger som kan foreg{\aa} f{\o}r atomet er
blitt stabilt. Hvilke utvalgsregler gjelder for slike
prosesser.
%
\item[d)] Beregn b{\o}lgelengdene til de utsendte fotonene
i de elektriske dipol overgangene.
%
\end{itemize}
%
\subsubsection*{Kort fasit}
\begin{itemize}
%
\item[a)] Termsymbolet er definert ved $^{2 S + 1}L_J$ hvor
$L$ er kvantetallet for banespinnet,
$S$ kvantetallet for egenspinnet og $J$ kvantetallet for totalspinnet
-- $\vec{J} = \vec{L} + \vec{S}$.
For en gitt $J$ har vi $2 J + 1$ tilstander svarende til forskjellige
verdier av kvantetallet $M_J$ som alle har samme energi.
For $2s$ niv{\aa}et er $J= 1 / 2$. Dette gir en degenerasjonsgrad p{\aa} 2.
%
\item[b)] $p$ niv{\aa}ene har $S = 1 /2$ og $L = 1$. Dette gir $ J = 1 / 2,
\;\; 3 / 2$. Spinn--bane koblingen gir $+a/2$ for $J = 3/2$ og
 $-a$ for $J = 1/2$
P{\aa} tilsvarende m{\aa}te f{\aa}r vi for $d$ niv{\aa}ene
to tilstander med h.h.v.
$J = 5 / 2, \;\; 3 / 2$.
For $s$ niv{\aa}ene har vi $L = 0$ og vi f{\aa}r bare en tilstand med
$J = S = 1 / 2$.
%
\item[c)] og d) \hspace*{0.2cm} Utvalgsregler for elektriske dipol
overganger
er $ \Delta L = \pm 1, \; \Delta M_l = 0, \pm 1$ og $\Delta S = 0$.
Dette gir f{\o}lgende overganger
%
\begin{eqnarray*}
1.\;\; 3s \longrightarrow \; ^2\!P_{3/2} \; \;
       \Delta E = 1,5279 eV \;\;
       \lambda_1 = 8,1147 \times 10^{-7} m\\
2.\;\; ^2P_{3 / 2} \longrightarrow 2s \; \;
   \Delta E = 1,8511 eV \;\;
   \lambda_2 = 6,6979 \times 10^{-7} m\\
3.\;\; 3s \longrightarrow  \; ^2\!P_{1/2} \; \;
   \Delta E = 1,5281 eV \;\;
   \lambda_3 = 8,1136 \times 10^{-7} m\\
4.\;\; ^2P_{1 / 2} \longrightarrow 2s \; \;
   \Delta E = \frac{h c}{\lambda_4} = 1,8509 eV \;\;
   \lambda_4 = 6,6986 \times 10^{-7} m
\end{eqnarray*}
%
\end{itemize}
%

\subsubsection*{Oppgave 9.6}
%
I denne oppgaven skal vi se bort fra elektronets
egenspinn.
Energi egenverdiligningen for et elektron i  et hydrog\'{e}n--atom
er da gitt ved
%
\[
\OP{H}_0 \psi_{n l m_{l}}= -E_0 \frac{1}{n^2} \psi_{n l m_{l}},
\]
%
hvor $E_0$ er en konstant.
\begin{itemize}
%
\item[a)] Sett opp Hamilton--operatoren $\OP{H}_0$ og skriv ned hvilke
betingelser kvantetallene $n$, $l$ og $m_l$ m{\aa} oppfylle.
\end{itemize}
%
Kvantetallet $m_l$ tilfredsstiller egenverdi--ligningen
%
\[
\left ( \frac{d^2}{d \phi^2} + m^2_l \right ) \Phi(\phi)=0.
\]
%
\begin{itemize}
%
\item[b)] Finn $\Phi(\phi)$ (normering er un{\o}dvendig)
og vis hvilke betingelser randkravene for $\Phi$
gir p{\aa} $m_l$.
\end{itemize}
%
B{\o}lgefunksjonen for $s$-tilstanden  med $n$ = 2  er gitt ved
%
\[
\psi _{200}= \sqrt{\frac{1}{8 \pi a_0^3}}
\left ( 1 - \frac{\rho}{2} \right ) \exp (- \frac{\rho}{2} ),
\;\;\;\; \rho = \frac{r}{a_0}
\]
%
\begin{itemize}
%
\item[c)] Vis at b{\o}lgefunksjonen er normert.
(Forslag: Bruk
$\int_{0}^{\infty}\rho ^{k}e^{-\rho} d\rho = k! $,
hvor $k$ er et heltall.)
Finn middelverdien (forventningsverdien) for radien
$< r_{2,0,0}>$ og den radius som  elektronet med st{\o}rst
sannsynlighet befinner
seg i. Forklar kort hvorfor det er forskjell i
disse to radiene.

\end{itemize}
%
Anta at Hamilton--operatoren har et ekstra ledd gitt ved
%
\[
\OP{H}_1 = \frac{E_0}{16 \hbar^2} \left (\OP{L}_x^2  +\OP{L}_y^2 \right ).
\]
%
\begin{itemize}
%
\item[d)] Finn egenverdiene for $\OP{H}_0 + \OP{H}_1$.
Tegn alle energiniv{\aa}ene med $ n = 1$ og 2
f{\o}r og etter $\OP{H}_1$ leddet er innf{\o}rt.
Hvilken degenerasjon har niv{\aa}ene i de
to tilfellene?
%
\end{itemize}
%
\subsubsection*{L\o sning}
\begin{itemize}
% 
\item[a)] Det klassiske uttrykk for energien til et elektron i
hydrog\'{e}n atomet er gitt ved 
% 
\begin{eqnarray*}
H_{kl} &=& \frac{1}{2} m v^2 + \frac{e^2}{4 \pi \varepsilon_0 r}\\
&=& \frac{p^2}{2m} + \frac{k}{r}
\end{eqnarray*}
% 
hvor $k =  \frac{e^2}{4 \pi \varepsilon_0}$.\\
Overgang til kvantemekanikk gir 
%
\begin{eqnarray*}
\vec{r} &\longrightarrow& \OP{\vec{r}} = \vec{r}\\
\vec{p} &\longrightarrow& \OP{\vec{p}} = -i\hbar \vec{\bigtriangledown}
\end{eqnarray*}
%
og den kvantemekaniske Hamilton operatoren (uten egenspinn) blir
%
\[
\OP{H}_0 = \frac{\OP{\vec{p}}^2}{2m} + \frac{k}{r} =
-\frac{\hbar^2}{2m}\left (\vec{\bigtriangledown}\right )^2 - \frac{k}{r}
\]
%
Betingelsene for de tre kvantetallene $n,l,m_l$ er 
%
\begin{eqnarray*}
n &=& 1, 2, 3, \ldots\\
l &=& 0, 1, 2, \ldots ,n - 1\\
m_l &=& -l, -l + 1, \ldots, +l
\end{eqnarray*}
%
\item[b)] L�sningen av egenverdiligningen for $\Phi(\phi)$ blir 
%
\[
 \Phi(\phi) = A e^{im_l \phi}
\]
%
Funksjonen m� v�re \'{e}ntydig. Det krever at $\Phi(\phi) =
\Phi(\phi + \mu \cdot 2\pi)$. Dette betyr at $m_l$ m� v�re heltallig,
$m_l = 0, \pm 1, \pm 2, \cdots$.\\
Normering bestemmer $A$ til 
%
\[
\int_0^{2 \pi} \Phi^{\ast}(\phi) \Phi(\phi) d \phi 
= A^{\ast} A\int_0^{2 \pi} d \phi = 1
\]
% 
som gir $A = \sqrt{1/2\pi}$.
% 
\item[c)] B�lgefunksjonen er normert. Vi f�r 
%
\begin{eqnarray*}
\frac{1}{8\pi a_0^3} \int_0^{\infty} \int_{\Omega} \left ( 1
-\frac{\rho}{2} \right )^2 e^{-\rho} a_0^3 \rho^2 d \rho \sin \theta
d\theta d \phi
&=& \frac{1}{2} \int_0^{\infty} \left ( 1
-\frac{\rho}{2} \right )^2 e^{-\rho} \rho^2 d \rho\\
&=&\frac{1}{2} \int_0^{\infty} \left ( \rho^2 - \rho^3 + \frac{1}{4}
\rho^4 \right ) e^{-\rho} d \rho\\ 
 &=& \frac{1}{2} \left ( 2! - 3! + \frac{1}{4} 4! \right ) = 1
\end{eqnarray*}
%
Middelverdien 
%
\begin{eqnarray*}
\langle r \rangle &=& \int \psi_{200}^{\ast} r  \psi_{200} d\vec{r}  
= \frac{1}{2 a_0^3} \int_0^{\infty} \left ( 1 - \frac{\rho}{2} \right
)^2 e^{-\rho} a_0^4 \rho^3 d \rho\\
&=&\frac{1}{2}a_0 \left ( 3! -4! + \frac{1}{4} 5! \right ) = 6 a_0
\end{eqnarray*}
% 
Den radielle sannsynlighetstetthet er gitt ved 
% 
\begin{equation}
P(r) =  \psi_{200}^{\ast} \psi_{200} \cdot 4 \pi \rho^2
     = \frac{1}{2a_0} \left ( 1 -\frac{\rho}{2} \right )^2 r^2
                    e^{-\rho}
     =  \frac{1}{2a_0} \left ( \rho^2 - \rho^3 + \frac{1}{4} \rho^4
     \right ) e^{-\rho}
\end{equation}
%
Maksimum/minimum's punktene er gitt ved 
%
\begin{eqnarray}
\frac{dP(\rho)}{d\rho}  
&=& \frac{1}{2a_0^2} \left [ \left ( 2 \rho - 3 \rho^2 + \rho^3 \right ) 
- \left ( \rho^2 -\rho^3 + \frac{1}{4} \rho^4 \right )
 \right ]e^{-\rho}      \nonumber\\
&=& \rho \left ( 2 - 4 \rho + 2 \rho^2 - \frac{1}{4} \rho^3 \right ) e^{-\rho} = 0
\end{eqnarray}
% 
Fra $P(\rho)$ i lign.(1) ser vi at $P(\rho) \ge 0$ for alle
$\rho$. Den har to minimumspunkter $P(\rho = 0) = 0$ og $P(\rho) = 2)
= 0$. Betingelsen i lign.(2) kan da skrives p� formen
% 
\[
\rho \left ( \rho - 2 \right )\left ( a \rho^2 + b \rho +c \right ) =0
\]
%
med verdiene $ a = -1/4$, $b = 3/2$ og $c = -1$. L�sningene av 2.grads
ligningen 
% 
\[
\rho^2 - 6\rho +4 = 0 \longrightarrow
\rho = 3 \pm \sqrt{5} = \left \{ \begin{array}{l}
                                            5.24\\
                                            0.76
                                 \end{array}
                        \right .
\]
%
og de fire l�sningene gir
% 
\[
P(\rho = 0) = 0, \quad P(\rho = 0.76) = 0.052, \quad
 P(\rho =2 ) = 0 \quad \mbox{og}  \quad P(\rho = 5.24) = 0.19
\]
%
De to maksimunspunktene svarer to de to Bohr radien $r_{1s} =
1\cdot a_0$ og $r_{2s} = 4 \cdot a_0$. 
%
\item[d)] Hamilton operatorene har n� formen 
% 
\begin{eqnarray*}
\OP{H} &=& \OP{H}_0 + \frac{E_0}{16 \hbar^2} 
      \left \{ \OP{L}_x^2 +  \OP{L}_y^2
\right \}\\
&=& \OP{H}_0 + \frac{E_0}{16 \hbar^2} 
      \left \{ \OP{\vec{L}}^2 - \OP{L}_z^2
\right \}
\end{eqnarray*}
%
Funksjonene $\psi_{nlm_l}$ er ogs� egenfunksjoner for operatorene
$\OP{\vec{L}}^2$ og $\OP{L}_z^2$.
% 
\begin{eqnarray*}
 \OP{\vec{L}}^2 \psi_{nlm_l} &=& l(l+1) \hbar^2
 \psi_{nlm_l}\\
 \OP{L}_z  \psi_{nlm_l} &=& m_l\hbar \psi_{nlm_l}
\end{eqnarray*}
%
Dette gir 
% 
\[
\left [ \OP{H}_0 + \frac{E_0}{16 \hbar^2} 
 \left \{ \OP{\vec{L}}^2 
	 - \OP{L}_z^2 \right \} \right ]   \psi_{nlm_l}
	= -E_0 \left [ \frac{1}{n^2} - \frac{l(l+1) -m_l^2}{16} \right ]
	      \psi_{nlm_l}
\]
%
\end{itemize}



\subsubsection*{Oppgave 9.7}
%
Vi skal i denne oppgave studere helium atomet.
%
\begin{itemize}
%
\item[a)] Gi en kort formulering av Pauliprinsippet.
\end{itemize}
%
I den laveste energi egentilstanden (grunntilstanden) for He--atomet
har begge elektronene kvantetallene $ n = 1, \;\; l = 0$.
%
\begin{itemize}
%
\item[b)] Bruk Pauliprinsippet og forklar hvilke verdier vi
har p{\aa} $L, S$ og $J$ for grunntilstanden i He--atomet.
Sett opp den spektroskopiske betegnelsen for grunntilstanden.
\end{itemize}
%
Grunntilstandsenergien for He--atomet er E$_g$(He) $= -78,98$~eV og for
He$^+$--ionet er energi egenverdiene gitt ved
%
\[
E_n(He^+) = -\frac{54,44 \,eV}{n^2} \;\;\; n = 1, 2, \ldots
\]
Vi forutsetter at ved ionisering (ett elektron fjernes) av
et He--atom etterlates He$^+$--ionet i en egentilstand hvor
kvantetallene $n$ og $l$ har de samme verdiene som er tilskrevet
ett av elek\-tro\-nene i He--atomet.

En \underline{samling} (gass) av He--atomer i grunntilstanden utsettes
for elektromagnetisk str{\aa}ling med frekvens $\nu$.
He--atomene forutsettes {\aa} v{\ae}re i ro. Ved bestr{\aa}lingen vil
et He--atom kunne absorbere et energikvant $h \nu$ og spaltes
til et He$^+$--ion og ett fritt elektron.
%
\begin{itemize}
%
\item[c)] Sett opp de ikke--relativistiske uttrykkene som
f{\o}lger av at energi og bevegelsesmengde bevares ved prosessen
%
\begin{center}
foton + He $\longrightarrow$  He$^+$ + elektron
\end{center}
%
\end{itemize}
I det f{\o}lgende ser vi bort fra He$^+$--ionets rekylbevegelsen.
%
\begin{itemize}
%
\item[d)]  Hvilken verdi m{\aa} $\nu$ minst ha om en skal f{\aa} ionisasjon
av He--atomet?
\end{itemize}
%
Den kinetiske energien $E_k$ av de frigjorte elektronene kan
m{\aa}les og viser $E_k = 25,0$~eV n{\aa}r den elektromagnetiske str{\aa}ling
har en frekvens $\nu$ gitt ved $h\nu = 49,50$~eV.
%
\begin{itemize}
%
\item[e)] Bestem ut fra dette grunntilstandsenergien for
He--atomet.
\end{itemize}
%
He--atomene plasseres n{\aa} i et homogent og konstant
magnetfelt $\vec{B}$.
%
\begin{itemize}
%
\item[f)] Vil magnetfeltet ha noen innflytelse p{\aa} He--atomets
energi i grunntilstanden? Begrunn svaret.
\end{itemize}
%
He--atomene utsettes s{\aa} for elektromagnetisk str{\aa}ling med frekvens
$\nu$ gitt ved $h \nu = 49,50$~eV. Den kinetiske energien av de
frigjorte elektronene m{\aa}les.
%
\begin{itemize}
%
\item[g)] Hva vil en forvente som resultat av en
slik m{\aa}ling med hensyn til den kinetiske energi av de frigjorte
elektronene? Begrunn svaret.
%
\item[h)] Beregn elektronenes energi n{\aa}r magnetfeltet har
verdi $|\vec{B}| = 0,1\;T$
%
\end{itemize}
%

\subsubsection*{Oppgave 9.8}
%

Vi skal i denne oppgave f{\o}rst se p{\aa} et \'{e}n--dimensjonalt
kvantemekanisk problem. B{\o}l\-ge\-funk\-sjonen for grunntilstanden
for en partikkel med masse m  er gitt ved
%
\[
\psi_0(x) = A \exp \left (-\frac{1}{2} a^2 x^2 \right ).
\]
%
hvor $A$ er en normaliseringskonstant.
%
\begin{itemize}
%
\item[a)] Sett opp energi egenverdiligningen for partikkelen
og bestem den potensielle energien $V(x)$ som gir
$\psi_0(x)$ som l{\o}sning for grunntilstanden.
Finn grunntilstandsenergien $E_0$.
%
\item[b)] Finn normaliseringskonstanten $A$.
%
\item[c)] Vis at
\[
\psi_1(x) = \sqrt{2} a x \psi_0(x)
\]
%
ogs{\aa} er l{\o}sning av egenverdiligningen under a).
Finn den tilsvarende energien $E_1$.
%
\item[d)] Beregn $\langle \OP{x} \rangle$, $\langle \OP{p}\rangle$, 
$\langle \OP{x}^2\rangle$ og $\langle\OP{p}^2 \rangle$ for
grunntilstanden $\psi_0(x)$.
%
\item[e)] Uskarpheten for en operator i kvantemekanikken
er definert ved
%
\[
\Delta A = \sqrt{\langle \OP{A}^2 \rangle - \langle \OP{A}\rangle^2}.
\]
%
Bruk dette til {\aa} finne $\Delta x \cdot \Delta p$
for grunntilstanden $\psi_0(x)$. Komment\'{e}r resultatet.
%
\end{itemize}
%
Vi skal g{\aa} over til {\aa} studere problemet ovenfor i tre dimensjoner.
Den potensielle energien $V(x)$ overf{\o}res til $V(r)$, i.e. samme
funksjon, men n{\aa} avhengig av den radielle avstanden $r$.
%
\begin{itemize}
%
\item[f)] Sett opp egenverdiligningen i det tre--dimensjonale
tilfelle og vis at funksjonen
%
\[
\Phi_{n,p,q}(x,y,z) = \psi_n(x) \psi_p(y) \psi_q(z),
\]
er en l{\o}sning hvor $n, p, q = 0, 1, 2, \ldots$.
Angi degenerasjonen for grunntilstanden og den f{\o}rste
eksiterte tilstanden.
%
\item[g)] Egenfunksjonene i det tre--dimensjonale tilfellet vil
samtidig v{\ae}re egenfunksjoner for banespinn operatorene
$\OP{\vec{L}}^2$ og $\OP{L}_z$. Finn verdier for
kvantetallene $l$ og $m_l$ for operatorene
$\OP{\vec{L}}^2$ og $\OP{L}_z$ i grunntilstanden $\Phi_{0,0,0}$.
%
\end{itemize}
\subsubsection*{L\o sning}

%
\begin{itemize}
%
\item[a)] Schr\"{o}dingers egenverdiligning er 
%
\begin{equation}
\left ( -\frac{\hbar^2}{2m} \frac{d^2}{dx^2} 
       + V(x) \right ) \psi_0(x) =  E \psi_0(x)
\end{equation}
%
For $\psi_0(x)$ f�r vi  
% 
\[
\frac{d}{dx} \psi_0(x) = (-a^2 x) \psi_0(x),\;\; \mbox{og}\;\;
\frac{d^2}{dx^2} \psi_0(x) = (-a^2 + a^4 x^2) \psi_0(x)
% 
\]
%
Innsatt i lign~(1) gir dette
%
\[
 \left ( -\frac{\hbar^2}{2m} ( -a^2 + a^4 x^2)
                  + V(x) \right )  \psi_0(x) =  E \psi_0(x)
\]
%
For at venstre side skal v�re lik h�yre side for alle $x$ f�r vi
betingelsene
% 
\[
V(x) = \frac{1}{2} m \omega^2 x^2 
       \;\mbox{og}\;
E_0 = \frac{1}{2}\hbar \omega
     \; \mbox{hvor}\; \omega = \frac{\hbar a^2}{m}
\]
%
\item[b)] Normaliseringen gir 
% 
\[
|A|^2 = \int_{-\infty}^{\infty} e^{-a^2 x^2} dx = \frac{a}{\sqrt{\pi}}
\; \longrightarrow \; A = \left( \frac{\pi}{a^2} \right )^{1/4}
\]
\item[c)] For  $\psi_1(x)$ f�r vi 
% 
\begin{eqnarray*}
\frac{d}{dx} \psi_1(x) &=& \sqrt{2}a \left (1 - a^2 x^2 \right ) \psi_0(x)\\
\frac{d^2}{dx^2} \psi_1(x) &=& (a^4 x^2 - 3 a^2) \psi_1(x)
% 
\end{eqnarray*}
% 
Dette gir samme potensial som under a) 
og en energi egenverdi
%
\[
E_1 = \frac{3}{2} \hbar \omega = 3 E_0
\]
%
i overenstemmelse energiformelen $ E_n =(n + 1)\hbar \omega$ for den harmoniske oscillator
%
\item[d)]  Vi har f�lgende uttrykk
%
\[
\langle x\rangle_0 = \int_{-\infty}^{\infty} \psi_0^{\ast}(x) x \psi_{n}(x)_0 \;dx
 \int_{-\infty}^{\infty} |\psi_0(x)|^2 x  \;dx = 0
\]
%
som er null p� grunn av minus paritet for integranden

Uttrykket for $\langle p\rangle_{n} $ blir 
%
\[
\langle p\rangle_0 = \int_{-\infty}^{\infty} \psi_0^\ast(x) 
                  (-i\hbar\frac{d}{dx}) \psi_0(x)\; dx = 0
\]
%
igjen p� grunn av minus paritet for integranden.
%
\item[e)] For � finne $\Delta x \cdot \Delta p$ m� vi f�rst 
beregne $\langle x^2\rangle $ og $\langle p^2\rangle $.
%
\begin{eqnarray*}
\langle x^2\rangle_0 &=& \int_{-\infty}^{\infty} \psi_0^\ast(x) x^2 \psi_0(x)\;dx
                    = \frac{a}{\sqrt{\pi}}
             \int_{-\infty}^{\infty}x^2 \exp (-a^2 x^2) \;dx\\ 
              &=& \frac{a}{\sqrt{\pi}} 2 \frac{1}{2 a^3}
                   \Gamma(3/2) = \frac{1}{2 a^2}\\
%
\langle p^2\rangle_0 &=& \int_{-\infty}^{\infty} \psi_0^\ast(x)
                      (-\hbar^2 \frac{d^2}{dx^2}) \psi_0(x) \;dx
             = -\hbar^2 \int_{-\infty}^{\infty} \psi_0^\ast(x)
                      (-a^2 + a^4 x^2) \psi_0(x) dx\\
           &=& \hbar^2 a^2 
              - \hbar^2 a^4  \frac{a}{\sqrt{\pi}}
             \int_{-\infty}^{\infty}x^2 \exp (-a^2 x^2) \;dx\\ 
           &=& \hbar^2 a^2 
              - \hbar^2 a^4 \frac{1}{2a^2} 
           = \frac{1}{2} \hbar^2 a^2\\
%
\end{eqnarray*}
%
som gir 
% 
\begin{eqnarray*}
\Delta x_0 \Delta p_0 = \sqrt{\langle x^2\rangle_0 \langle p^2\rangle_0} 
      = \sqrt{\frac{1}{2 a^2} \frac{1}{2} \hbar^2 a^2}
      = \frac{1}{2} \hbar\\
%
\end{eqnarray*}
% 
Dette er i overensstemmelse med Heisenbergs uskarphets relasjon.

%
\item[f)] I det tre--dimensjonale tilfelle blir Schr\"{o}dingers egenverdiligning 
for den harmoniske oscillator  
%
\begin{equation}
\left \{ -\frac{\hbar^2}{2m} \left ( \frac{d^2}{dx^2} + \frac{d^2}{dy^2}
                                 +  \frac{d^2}{dz^2} \right )
       + \frac{1}{2} m \omega \left ( x^2 + y^2 + z^2 \right ) \right \} \psi(x,y,z)
             =  E \psi(x,y,z)
\end{equation}
%
For $\psi(x,y,z) = \Phi_{n,p,q}(x,y,z)$ f�r vi l�sningen med egenverdi
% 
\[
E_{n,p,q} = \left ( \frac{3}{2} + n + p + q \right )\hbar \omega
\]   
%
For grunntilstanden og den f�rste eksiterte tilstand f�r vi 
% 
\begin{eqnarray*}
\mbox{Grunntilstand} :& \quad n = 0,\; p = 0, \; q = 0 \quad &E_{npq} = \frac{3}{2}\hbar \omega
              \quad \mbox{degenerasjon} = 1 \\
 \mbox{1. eksiterte tilstand} :& \quad
     \left \{ \begin{array}{c}
                 n = 1,\; p = 0, \; q = 0\\
                 n = 0,\; p = 1, \; q = 0\\
                 n = 0,\; p = 0, \; q = 1
              \end{array} \right .
              \quad
               &E_{npq} = \frac{5}{2}\hbar \omega
                \quad \mbox{degenerasjon} = 3
 \end{eqnarray*}
%
\item[g)] Egenverdiligningene for operatorene $\OP{L}^2$ 0g $\OP{L}_z$ blir 
i dette tilfelle 
% 
\begin{eqnarray*}
\OP{L}^2 \Phi(x,y,z) = \OP{L}^2 \Phi(r) = 0\\
\OP{L}_z \Phi(x,y,z) = \OP{L}_z \Phi(r) = 0\\
\end{eqnarray*}
%
og kvantetallene blir $ l = 0,\; m_l = 0$ siden funksjonen kun avhenger av koordinaten $r$,
mens begge operatorene bare  inneholder derivasjon  med hensyn p� 
koordinatene $\theta,\; \phi$.
 
\end{itemize}
 
\subsubsection*{Oppgave 9.9}
%
En partikkel med masse $m$ beveger seg i et tre--dimensjonalt
harmonisk oscillator potensial med potensiell energi
$V(r) = (1/2) k r^2$, hvor $k$ er en konstant.
Dette systemet skal vi f{\o}rst studere ved bruk av Bohrs atomteori
%
\begin{itemize}
%
\item[a)] Sett opp  Bohrs kvantiserings betingelse og bruk den til
{\aa} beregne energien og $r^{2}$ for alle tillatte sirkul{\ae}re
baner for partikkelen.
\end{itemize}
%
Vi skal deretter studere det samme problem ved bruk av
Schr\"{o}dingers kvantemekanikk.
%
\begin{itemize}
%
\item[b)] Sett opp systemets Hamilton operator og vis at
%
\[
\psi(\vec{r}) = N \exp (-\alpha r^2)
\]
er en egentilstand for systemet for en
bestemt verdi av $\alpha$ og finn energi egenverdien.
%
\item[c)] Beregn $\langle \OP{r}^2 \rangle$ for systemet 
i tilstanden $\psi(\vec{r})$
og sammenlign resultatet med punkt a).
%
\end{itemize}

\subsubsection*{Oppgave 9.10, Eksamen H-1996}
%
I den enkle skallmodellen (sentralfelt modellen) for fler--elektron atomer 
plasseres elek\-tronene i
tilstander som har de samme kvantetallene som energi egentilstandene for
Hy\-dro\-gen atomet: 
\[
1s; 2s, 2p; 3s, 3p, 3d;\ldots 
\]
%
\begin{itemize}
%
\item[a)] Hvor mange elektroner kan det plasseres i hver av tilstandene
$nl = 1s; 2s, 2p$;  $3s, 3p$ og $3d;$?
Et hovedskall karakteriseres ved kvantetallet $n$. Finn hvor mange 
elek\-troner
kan plasseres i et hovedskall som funksjon av kvantetallet $n$.
Formul\'{e}r det prinsipp som bestemmer hvor mange elektroner
det kan v\ae re i hver tilstand.
%
\item[b)] Natrium (Na) atomet har 11 elektroner. Bruk den enkle skallmodellen 
for elek\-tronene
ti � bestemme elektronenes grunntilstandskonfigurasjon.
Grunn\-til\-stand\-en karakteriseres ogs�  ved kvantetallene $ L,\; S,\; J$ satt sammen 
i termsymbolet
$^{2S + 1}L_J$. Forklar hva kvantetallene betyr og finn termsymbolet for 
grunn\-til\-stand\-en.
%
\end{itemize}
%

I resten av denne oppgaven skal vi studere noen tilstander hvor
det siste (valens) elektronet i Na er i en eksitert tilstand. 
I figur~\ref{fig3.2} er det vist 
energi niv{\aa}ene for noen slike tilstander.
For $3p$ og $4s$ niv{\aa}ene er eksitasjonsenergien
over $3s$ niv{\aa}et oppgitt i $eV$.
%
\begin{figure}[h]
%
\begin{picture}(465.26,210.83)
\thicklines
\put(22.0,25.45){\line(1,0){6.0}}
\put(24.10,25.45){\vector(0,1){190.59}}
\put(30.10,212.05){\makebox(0,0)[tl]{E (eV)}}
\put(48.21,30.45){\makebox(0,0)[tl]{3s}}
\put(59.89,25.45){\line(1,0){37.98}}
\put(69.74,36.0){\makebox(0,0)[tl]{\small 0,0 eV}}
\put(59.89,121.58){\line(1,0){37.25}}

\put(108.10,59.70){\makebox(0,0)[tl]{3p}}

\put(126.36,47.70){\line(1,0){35.79}}
\put(126.36,59.95){\line(1,0){35.79}}
\put(127.20,70.60){\makebox(0,0)[tl]{\small 2.10436 eV}} 
\put(127.20,43.45){\makebox(0,0)[tl]{\small 2.10223 eV}}

\put(126.36,176.24){\line(1,0){36.79}}
\put(126.36,184.14){\line(1,0){36.79}}
\put(108.10,187.24){\makebox(0,0)[tl]{4p}}

\put(210.35,158.73){\makebox(0,0)[tl]{3d}}
\put(227.88,155.08){\line(1,0){38.71}}
\put(227.88,157.39){\line(1,0){37.98}}

\put(48.21,124.58){\makebox(0,0)[tl]{4s}}
\put(60.74,132.17){\makebox(0,0)[tl]{\small 3.19123 eV}}
%
\end{picture}
%
\caption{\label{fig3.2}Energitilstander for Na atomet.}
\end{figure}
%
\begin{itemize}
%
\item[c)] Gi en kort forklaring p{\aa} hvorfor $p$ og $d$ niv{\aa}ene
er splittet i to, mens $s$ niv{\aa}ene ikke er splittet.
%
\item[d)] Finn termsymbolet og beregn degenerasjonsgraden for hver av de 
eksiterte tilstandene. 
\end{itemize}
%
Vi tenker oss n{\aa} at Na atomet har valens elektronet  i en
$4s$ tilstand, og at det deretter ikke er p{\aa}virket
av omgivelsene.
%
\begin{itemize}
%
\item[e)]  Gj{\o}r rede for de forskjellige elektriske
dipol overganger som kan foreg{\aa} f{\o}r atomet er
blitt stabilt. Hvilke utvalgsregler gjelder for slike
prosesser?
%
\item[f)] Beregn b{\o}lgelengdene til de utsendte fotonene
i de elektriske dipol overgangene.
%
\end{itemize}
%
\subsubsection*{Kort fasit}
\begin{itemize}
% 
\item[a)]Den generelle formel for antall tilstander er $N(n,l) = 2 n^2$\\
Pauliprinsippet: B�lgefunksjonen for N elektroner m� v�re antisymmetrisk
ved ombytning av to vilk�rlige elektroner.\\ 
Pauliprinsippet i den enkle skallmodell: Ingen elektroner kan ha samme sett
kvantetall.
%
\item[b)] Grunntilstanden for Na er: $1s^22s^22p^63s$. Kvantetallet
$L$ er det totale banespinn, $S$ det totale egenspinn og $J$ det totale spinn
for systemet. For Na er $L = 0$, $S = J = 1/2$ som gir termsymbolet $^{2}S_{1/2}$.
%
\item[c)]  Se l{\ae}reboka {\sl Brehm and Mullin: Introduction to the
           structure of matter}, avsnitt 8.9, Oppsplittingen skyldes
            spinn--bane kobling.
%
\item[d)] Tilstanden $3s$ har termsymbol $^{2}S_{1/2}$ med degenerasjonsgrad $ = 2$.\\
 Tilstanden $3p$(2.10223~eV) har termsymbol $^{2}P_{1/2}$ med degenerasjonsgrad $ = 2$.\\
 Tilstanden $3p$(2.10436~eV) har termsymbol $^{2}P_{3/2}$ med degenerasjonsgrad $ = 4$.\\
 Tilstanden $4s$ har termsymbol $^{2}S_{1/2}$ med degenerasjonsgrad $ = 2$.\\
 Tilstanden $3d$(laveste tilstand) har termsymbol $^{2}D_{3/2}$ med\\
 degenerasjonsgrad $ = 4$.\\
 Tilstanden $3d$(h�yeste tilstand) har termsymbol $^{2}D_{5/2}$ med\\
 degenerasjonsgrad $ = 6$.\\
 Tilstanden $4p$(laveste tilstand) har termsymbol $^{2}P_{1/2}$ med\\
 degenerasjonsgrad $ = 2$.\\
 Tilstanden $4p$(h�yeste tilstand) har termsymbol $^{2}P_{3/2}$ med\\
degenerasjonsgrad $ = 4.$\\
%
\item[e)] Utvalgsregler for dipoloverganger:
          Se l{\ae}reboka {\sl Brehm and Mullin: Introduction to the
           structure of matter}, avsnitt 7.5
Dette gir
\[
4s_{1/2} \rightarrow 3p_{3/2} \rightarrow 3s_{1/2}\;\;\; \mbox{og}\;\;\; 
4s_{1/2} \rightarrow 3p_{1/2}\rightarrow 3s_{1/2}\;\;\;
\]
%
\item[f)] $4s_{1/2} \rightarrow 3p_{3/2}(2.10436~eV)$ gir $\lambda = 1140.9~nm$\\
$4s_{1/2} \rightarrow 3p_{1/2}(2.10223~eV)$ gir $\lambda = 1138,7~nm$\\
$3p_{3/2}(2.10436~eV) \rightarrow 3s_{1/2}$ gir $\lambda = 589,2~nm$\\
$3p_{1/2}(2.10223~eV) \rightarrow 3s_{1/2}$ gir $\lambda = 589,8~nm$
%
\end{itemize}
%

\subsubsection*{Oppgave 9.11, Eksamen H-1994}
Vi skal i denne oppgaven studere en partikkel med masse $m$
i tre--dimensjonal bevegelse. Oppgaven vil kreve bruk av
$\bigtriangledown^2$ som i polarkoordinater kan skrives p{\aa}
formen
%
\[
\bigtriangledown^2 = \frac{1}{r} \frac{\partial^2}{\partial r^2} r
+ \frac{1}{r^2 \sin \theta} \frac{\partial}{\partial \theta}
\left ( \sin \theta \frac{\partial}{\partial \theta} \right )
+ \frac{1}{r^2 \sin^2 \theta}\; \frac{\partial^2}{\partial \phi^2}
\]
Vi antar f{\o}rst at partikkelen er fri, ingen potensiell energi.
%
\begin{itemize}
%
\item[a)] Vis at $\psi = (N / r) \sin (kr)$ er en energi egenfunksjon,
og finn energien $E$. $N$ og $k$ er konstanter.
%
\item[b)] Forklar hvorfor $\psi$ er en s--tilstand, det vil si
at banespinnet $L = 0$ (om origo).
%
\item[c)] Vis at det ikke finnes andre akseptable energi egenfunksjoner
med  $L= 0$ enn de som er definert i punkt a) med
vilk{\aa}rlig, reelle verdier p{\aa} konstanten $k$.
%
\end{itemize}
%
Vi antar n{\aa} at partikkelens potensielle energi
$V(r)$  er en ren funksjon av $r$ definert ved
%
\[
V(r) = \left \{
\begin{array}{crl}
 0        & r < r_0,\\
 \infty   & r > r_0.
\end{array}
\right .
\]
%
\begin{itemize}
%
\item[d)] Finn n{\aa} partikkelens energi egentilstander med $L =0$.
			 Beregn normaliseringskonstanten.
%
\end{itemize}
%
I det f{\o}lgende skal vi begrense oss til $L = 0$ tilstander.
Systemet vi skal studere  best{\aa}r n{\aa} av to n{\o}ytroner
i tre--dimensjonal bevegelse og med potensiell energi  $V(r)$
som angitt ovenfor. Vi antar ingen ekstra vekselvirkning
mellom n{\o}ytronene.
%
\begin{itemize}
%
\item[e)] Finn de to laveste to--partikkel energi egentilstandene
og beregn energi egenverdien. Finn det totale egenspinn S
for de to tilstandene.
%
\end{itemize}
%
Vi innf{\o}rer n{\aa} en ekstra vekselvirkning mellom de
to n{\o}ytronene gitt ved $V_S = a \OP{\vec{S}}_1 \cdot \OP{\vec{S}}_2,$
hvor $\OP{\vec{S}}_1$ og $\OP{\vec{S}}_2$ er operatorene for
n{\o}ytronenes egenspinn og $a$ en positiv konstant.
%
\begin{itemize}
%
\item[f)] Finn endringen i energi egenverdiene som denne
vekselvirkningen gir for de to laveste energi egentilstandene
funnet under punkt e).
%
\end{itemize}

\subsection{Numeriske oppgaver}
\subsubsection*{Prosjektoppgave FY102 V-2003}

I denne oppgaven skal vi studere en enkel modell for litium-atomet. Vi
f�r bruk for den radielle Schr\"odingerlikningen, og kjennskap til den
analytiske l�sningen for hydrogen-atomet er ogs� en fordel. 
Dere vil ogs\aa\ trenge resultater fra oblig 9, oppgave 2 (spesielt den frivillige delen). L\o sningen til den
oppgaven finner dere p\aa\ kursets nettsted.

Oppgaven er mer en modifisert hydrogenatom-problemstilling enn et
mangepartikkelproblem slik en uavhengig partikkel-modell for
litium ville v�re. Vi skal basere oss p\aa\ sentralfeltmodellen (uavhengig 
partikkel modell) for atomer.

\subsubsection*{Oppgave 1, En enkel modell for litium-atomet}


I den enkle skallmodellen for fler--elektron atomer plasseres elektronene i
tilstander som har de samme kvantetallene som energi egentilstandene for
H--atomet:\\
$1s; 2s, 2p; 3s, 3p, 3d;\ldots ;nl = n0, nl = n1, \ldots, nl = n,n-1$.
%
\begin{itemize}
%
\item[1a)] Hvor mange elektroner kan det plasseres i hver av tilstandene
$1s; 2s, 2p; 3s, 3p, 3d;\ldots ;$ $ nl = n0, \ldots , nl = n,n-1$?
Formul\'{e}r det prinsippet som bestemmer hvor mange elektroner
det kan v\ae re i hver tilstand. Forklar kort bakgrunnen for sentrafeltmodellen
i atomfysikk.
%
\end{itemize}
%
I H--atomet har tilstandene $nl = n0, nl = n1\ldots , nl = n,n-1$ alle
samme energi. De er degenererte. I et fler--elektron
atom, er dette ikke lenger tilfelle.
%
\begin{itemize}
%
\item[1b)] Diskut\'{e}r \aa rsaken til dette.
%
\end{itemize}
%
Li--atomet har 3 elektroner. Grunntilstanden er gitt
ved symbolet $^{2}S_{1/2}$.
%
\begin{itemize}
%
\item[1c)] Begrunn at verdiene av grunntilstandens spinn kvantetall er
bestemt av valenselektronets kvantetall.
%
\end{itemize}
%
De to laveste eksiterte tilstandene i Li--atomet er $^{2}P_{1/2}$ og
$^{2}P_{3/2}$.
Tilstandene $^{2}P_{1/2}$ og $^{2}P_{3/2}$ ligger henholdsvis 1.8509 eV og
1,8511 eV over grunntilstanden.
%
\begin{itemize}
%
\item[1d)] Angi hvordan spinn kvantetallene for disse tilstandene bestemmes
av valenselektronets kvantetall.
Diskut\'{e}r kort og kvalitativt \aa rsaken til energiforskjellen
mellom tilstandene $^{2}P_{1/2}$ og $^{2}P_{3/2}$. Hvor stor er
degenerasjonsgraden for hvert av niv\aa ene og hva er \aa rsakene til denne
degenerasjonen?
%
\end{itemize}


Litium-atomet har tre elektroner i bane rundt en kjerne med ladning tre.
Vi skal i resten av oppgaven
fors�ke � bestemme de to laveste energiniv�ene til valenseletronet, og
vi skal se at vi har brutt degenerasjon mhp. spinnet $l$. Vi
antar derfor at de to innerste elektronene befinner seg i
hydrogenliknende orbitaler i $1s$-tilstanden. Dette
vil gi opphav til en ladningstetthet $\rho(r)$ i en avstand $r$ fra
kjernen gitt ved
\be
\rho(r) = -\frac{e}{4\pi} \alpha^3 \exp(-\alpha r),
\label{eq:rho}
\ee
der $\alpha \approx 5.38 a_0^{-1}$ og $a_0$ er Bohrradien.

\begin{itemize}
%
\item[1e)]
Vis at ladningstettheten representerer en total ladning p� to
elektroner, det vil si at
\[
\underset{\mathbb{R}^3}\int \rho(r) d\tau = -2e.
\]
\end{itemize}

\begin{itemize}
%
\item[1f)]
Fra elektromagnetismen har vi Gauss' lov som kan skrives p�
differensialform, og med relasjonen $\vec{E}=-\nabla\phi$, der $\phi$
er det elektriske potensialet, f�r vi:
\be
\label{eq:poisson}
\nabla^2 \phi = -\frac{\rho}{\epsilon_0}.
\ee
Denne relasjonen kalles ogs� Poissons likning.

Vis at det elektriske potensialet
\be
\label{eq:vpot}
\phi(r)=-\frac{2ke}{r}\left[1 - (1+\frac{\alpha r}{2})\exp(-\alpha r)  \right]
\ee
tilfredsstiller Poissons likning for ladningstettheten (\ref{eq:rho})
for $1s$-elektronene. Merk at $k=1/4\pi\epsilon_0$.

Finn den totale potensielle energien for litium-atomet, der posisjonen $r$
til valenselektronet er den eneste frihetsgraden. Fremstill denne grafisk
sammen med den asymptotiske oppf�rselen for sm� $r$ og
for store $r$. Tolk dette i lys av sentralfelt-modellen for
atomer.
\end{itemize}

\subsection*{Oppgave 2, Schr\"odingers likning  p� dimensjonsl�s form}

Vi vil ogs� i denne oppgaven skrive om Schr\"odingers likning  
p� dimensjonsl�s form slik
at den egner seg bedre 
for numerisk analyse. Vi starter med � sette $s=r/a$, der $a$ er en
konstant med dimensjon lengde, slik at $s$ blir uten dimensjon.

\begin{itemize}
%
\item[2a)]
Skriv ned hamiltonoperatoren $\OP{H}$ for valenselektronet og den
tidsuavhengige radielle Schr\"odingerlikningen med hensyn p�
$u(r)=rR(r)$.
\end{itemize}

\begin{itemize}
%
\item[2b)]
Vis at vi n� kan skrive den tidsuavhengige Schr\"odingerlikningen p�
denne m�ten:
\be
\label{eq:se1}
-\frac{1}{2}\frac{\partial^2}{\partial s^2}u(s) -
\frac{3mke^2a}{\hbar^2s}u(s) - \frac{mea^2}{\hbar^2}\phi(s)u(s) + \frac{1}{2}\frac{l(l+1)}{s^2}u(s)= \frac{ma^2}{\hbar^2}Eu(s)
\ee
\end{itemize}

Vi vet n� at alle leddene i likningen m� ha samme dimensjon for at
uttrykket skal gi mening, nemlig
dimensjonen til $u(s)$. (Denne kan du pr�ve � finne p� egenh�nd.) Da
m� spesielt koeffisienten foran det andre leddet v�re dimensjonsl�s.
Vi kan n� bestemme en gunstig verdi for $a$ ved hjelp av:
\be
\label{eq:a}
\frac{mke^2a}{\hbar^2} = 1.
\ee

\begin{itemize}
%
\item[2c)]
Hva kjenner du igjen konstanten $a$ som? Sett inn uttrykket for 
$\phi(s)$ og skriv ut hele likningen. 
Bruk ogs� at konstanten $\alpha=5.38
\enhet{bohrradier}^{-1}$. Defin\'er ogs� $E =
\frac{\hbar^2}{2ma^2} \SC{E}$. Hva kjenner du faktoren foran $\SC{E}$ som?
\end{itemize}

\subsection*{Oppgave 3, Numerisk l�sning av den radielle Schr\"odingers likning}

Det som er nytt i den numeriske l�sningen av SL i forhold til
tidligere oppgaver, er at vi ikke har noen energi-egenverdier gitt.
Vi vil bruke en variant av \emph{todelings-metoden} (eller intervallhalveringsmetoden) for l�sning av
likninger numerisk. Hvis vi kan finne et intervall $[a,b]$ der vi vet
egenverdien vil ligge, kan vi teste med $\SC{E}=c=(a+b)/2$ og oppn� et
nytt intervall, enten $[a,c]$ eller $[c,b]$, der den l�sningen vil
ligge. Ved � gjenta prosessen $N$ ganger, vil vi ha et intervall med
lengde $(b-a)/2^N$ der vi vet egenverdien ligger, og det er klart at
det skal ikke mange todelinger til f�r vi har ganske god presisjon p�
egenverdien.

Vi nevner ogs� her muligheten for � konstruere Maple-funksjoner som
returnerer for eksempel et \texttt{dsolve}-funksjonskall. Dette kan
v�re sv�rt nyttig \aa\ forenkle prosessen med � repetere den numeriske
l�sningen betraktelig. Her er et vilk�rlig eksempel:
\begin{center}
\texttt{ NumeriskL�sning := (a, b) -> dsolve(\{diff(f(x),x\$2) = a*f(x),
f(0)=1, D(f)(0)=b\}, f(x), numeric  );    }
\end{center}
Her f�r vi en ny numerisk l�sning som vi kan sende til
\texttt{odeplot} for hver \texttt{a} og \texttt{b} vi setter inn:
\begin{center}
\texttt{ odeplot(NumeriskL�sning(2.0, -0.5),[x,f(x)],-5..5); }
\end{center}

\begin{itemize}
%
\item[3a)]
L�s den radielle SL for $2s$- og $2p$-tilstandene for valenselektronet,
og finn de tilh�rende energi-egenverdiene $\SC{E}$ ved hjelp av 
\emph{todelings-metoden}. her b\o r du lage en egen funksjon som gj\o r dette.
Velg integrasjonsvariabel $s \in [0:20a_0]$ 
n\aa r du integrer opp
difflikningene. 
Beregn de tilsvarende verdiene $E_{2s}$ og
$E_{2p}$. Hvordan 
stemmer dine verdier med de eksperimentelle verdiene $E_{2s} = -5,392
\enhet{eV}$ og $E_{2p} = -3,552 \enhet{eV}$? Den siste verdien er uten spinn-bane vekselvirkningen.

Husk at for sm� $s$ er $u(s)\sim s^{l+1}$. Den modifiserte b�lgefunksjonen 
for $2s$-orbitalen 
$u(r)$ m� ha \'en node i 
tillegg til $u(0)=0$; hvorfor det?
\end{itemize}

\begin{itemize}
%
\item[3b)]
Finn eksitasjonsenergien mellom de to tilstandene til
valenselektronet. Sammenlikn med verdiene oppgitt i oppgave 1, og 
komment\'er resultatet. 
\end{itemize}

\subsection*{Prosjektoppgave FYS2140 V-2004}

I denne prosjektoppgaven skal vi studere energi egentilstandene 
i Helium atomet. I motsetning til Hydrogen kan Schr\"{o}dingers 
egenverdiligning for Helium ikke l�ses eksakt, og vi m� finne
 approksimative metoder.

Vi vil begrense oss til � studere de tre laveste energi egentilstandene.
De eksperimentelle energiene til de tre laveste tilstandene
er funnet til � v�re 
% 
\[
 E(0) = -78,98\;eV, \quad E(1) = -59,09\;eV, \quad E(2) = -58,37\;eV,
\]
%
I modellberegningen v�r antar vi ren Coulomb vekselvirkning mellom atomets kjerne, de to 
elektronene og elektronene imellom, Vi ser ogs� bort fra rekylenergien til kjernen som 
vi antar ligger i ro. B�lgefunksjonene skal omfatte b�de rom-- og spinn-del og 
tilfredsstille Pauliprinsippet.

%
\begin{itemize}
% 
\item[a)] Vi starter med det enklere atomet He$^{+}$ med ett elektron
i tillegg til en He--kjerne.   
Sett opp Schr\"{o}dingers egenverdiligning 
for dette systemet. Omform ligningen slik at den kan l�ses numerisk.
Bruk Matlab. Lag en m-funksjon og beregn og plot funksjonene
$\Psi_{1s}(\vec{r})$ og $\Psi_{2s}(\vec{r})$.
% 

\item[b)] Modellen i a) utvides til � beskrive det n�ytrale He atomet 
ved �  inkludere elektron nr.2. Vi ser fremdeles bort fra 
Coulombfrast�tningen elektronene immellom.
 P� dette grunnlag konstru\'{e}r de formelle utrykkene til de totale 
egenfunksjonene for systemets tre laveste tilstander. Bruk symbolene
$\Psi_{1s}(\vec{r})$ og $\Psi_{2s}(\vec{r})$.
 Bestem energien i denne forenklede modellen  og kvantetallene $L$ $S$ og $J$. 
Hva blir de spektroskopiske faktorene for disse tre tilstandene?
Husk at b�de rom- og spinn-funksjonene skal v�re med, og de skal tilfredsstille 
Pauliprinsippet.
%
\item[c)] Vi skal n�  forbedre modellen i b). Som et f�rste skritt beregner vi 
energikorreksjonen
% 
\begin{equation}
\Delta E = \int \left (\Psi_{1s}(1,2) \right )^{*} 
                          \frac{ke^2}{r_{12}}
                     \Psi_{1s}(1,2)d\tau
\end{equation}
%
hvor $k = 1/(4\pi\varepsilon_0)$ og 
 $r_{12} = \sqrt{r_1^2 + r_2^2 - 2 \vec{r}_1\cdot \vec{r}_2}$
Det nye bidraget  gir  et 6-dimensjonalt integral, angitt ved volumelementet $d\tau$. 
Det kan l�ses.
Nedenfor i avsnittet Hjelpemidler angir vi en formel, lign~(6) for � hjelpe dere p� vei. 
Vi  har her integrert over koordinaten $\vec{r}_2$. 
Da gjenst�r integrasjon over $\vec{r}_1$. For � tilpasse lign~(6) til 
problemet, m� dere sette parametrene $s = 0,\quad Z = 2$

Finn den nye energien for grunntilstanden og sammenlign med resultatet i b).

% 


\end{itemize}
%
Resultatet i c) er fremdeles langt fra den eksperimentelle energien for
grunntilstanden. Et neste skritt kan da v�re f�lgende: 
Vi modifiserer vekselvirkningen mellom elektronene og atomkjernen ved 
� innf�re en effektiv elektronladning {(\sl screening)} i vekselvirkningen 
mellom elektronene og kjernen. $e_{eff} = (2 - s)e$ hvor $s$ er 
en parameter. Den  totale potensielle energien kan da skrives p� formen 
%
\begin{equation}
V = V_1 + V_2
\end{equation}
%
hvor 
%
\begin{eqnarray}
  V_1 &=& -\frac{(2-s)e^2k}{r_1}  - \frac{(2-s)e^2k}{r_2} \nonumber\\
  V_2 &=&  \frac{ke^2}{r_{12}} - \frac{ske^2}{r_1} - \frac{ske^2}{r_2 }
\label{pot}
\end{eqnarray}
%
hvor igjen $k = 1/(4\pi\varepsilon_0)$ 
og $r_{12} = \sqrt{r_1^2 + r_2^2 - 2 \vec{r}_1\cdot \vec{r}_2}$
% 
\begin{itemize}
% 
\item[d)] Vi ser i f�rste omgang bort fra vekselvirkningen $V_2$.
 Da kan energien for de tre tilstandene uttrykkes som en funksjon  av s,
dvs.~$E_\nu^0(s)$. 
Finn  $E_\nu^0(s)$ og den tilsvarende  funksjonen 
p� analytisk form.  $\Psi_{1s}(\vec{r})$ og $\Psi_{2s}(\vec{r})$, 
som n\aa\ er avhengige av parameteren $s$.
Tips: Se Hjelpemidler og tenk gjennom hvilken verdi ladningsparameteren skal ha 
i dette tilfelle.
%
\item[e)] S� inkluderes vekselvirkningen $V_2$,  og vi beregner bidraget fra formelen
%
\begin{equation}
E_\nu^1(s) = E_\nu^0(s) + \int \left (\Psi_\nu^{(c)}(1,2) \right )^* V_2\Psi_\nu^{(c)}(1,2)d\tau
\end{equation}
%
hvor $E_\nu^0(s)$ er energien funnet i d) og $\Psi_{\nu}^{(c)}(1,2)$ de
tilsvarende egenfunksjonene.
Det nye bidraget  gir  igjen et 6-dimensjonalt integral, angitt ved volumelementet $d\tau$. 
Det kan l�ses.
Nedenfor i avsnittet Hjelpemidler angir vi formlene, lign~(6 -8) 
hvor vi har integrert over koordinaten $\vec{r}_2$.
og integrasjon over $\vec{r}_1$ gjenst�r.
I dette tilfelle m� vi beholde parameteren $s$
og sette $Z=2 - s$ 
 
Beregn formelen for energiene $E_\nu^1(s)$ for alle tre tilstandene som en funksjon av s.
Plot (Matlab) grunntilstandsenergien $E_0^1(s)$ og finn den verdien av s som 
gir lavest verdi p� $E_0^1(s)$. 

Bruk denne verdien av s og finn energiene til de to eksiterte tilstandene.
Komment\'{e}r resultatet i lys av de tidligere beregningene.


\item[f)] \'{E}n--partikkel b�lgefunksjonene $\Psi_{1s}(\vec{r})$ og $\Psi_{2s}(\vec{r})$
     er avhengige av den effektive ladningsparameteren  s. Lag et plot som viser forskjellen
mellom funksjonene  og de tilsvarende sannsynlighetsfordelingene
for resultatet i a) og med det tilsvarende resultat fra e) hvor verdien  av s som ble funnet.
Kan forskjellen forklares fysisk?
 

\end{itemize} 

\subsubsection*{Hjelpemidler til FYS2140 prosjekt V-2004}
%
Oppgaven ovenfor omfatter to Hydrogen type \'{e}npartikkel b�lgefunksjoner
%
\begin{eqnarray}
\Psi_{1s}(\vec{r}) &=& \frac{(Z/a_0)^{3/2}}{\sqrt{\pi}} \exp(-(Z/a_0) r ) \nonumber\\
\Psi_{2s}(\vec{r}) &=& \frac{(Z/a_0)^{3/2}}{4 \sqrt{2\pi}} (2-Zr/a_0)\exp(-Zr/(2a_0)
\end{eqnarray}
hvor Z er en ladnings parameter og $a_0$ er Bohr-radien.  
%
F�lgende integraler vil hjelpe p� beregningen. 
%
\begin{eqnarray}
\int \Psi_{1s}(r_1)\Psi_{1s}(r_2)\frac{V_2}{ke^2} \Psi_{1s}(r_1)\Psi_{1s}(r_2)  d^3r_2
 = \int \Psi_{1s}(r_1))^2 \frac{V_2}{ke^2} (\Psi_{1s}(r_2))^2 d^3r_2 \nonumber\\
 = (\Psi_{1s}(r_1))^2 \frac{1}{r_1} \left (1 - s - s\frac{Z}{a_0}r_1 -  \exp(-\frac{2Z}{a_0} r_1)
                                     - \frac{Z}{a_0}r_1 \exp(-\frac{2Z}{a_0}r_1) \right )
\end{eqnarray}

%
\begin{eqnarray}
\int \Psi_{1s}(r_1)\Psi_{2s}(r_2) \frac{V_2}{ke^2} \Psi_{1s}(r_1)\Psi_{2s}(r_2) d^3r_2
=\int (\Psi_{1s}(r_1))^2 \frac{V_2}{ke^2} (\Psi_{2s}(r_2))^2  d^3r_2 \nonumber\\
= \frac{(\Psi_{1s}(r_1))^2}{32} \frac{1}{r_1} \left ( 1 - s - \frac{3}{2}s\frac{Z}{a_0}r_1
        - \exp(-\frac{2Zr_1}{a_0}) - \frac{1}{2}\frac{Z}{a_0}r_1 \exp(-\frac{2Zr_1}{a_0}) 
                                                    \right . \nonumber\\
 \left . + \frac{Z^2}{a_0^2} r_1^2 \exp(-\frac{2Zr_1}{a_0}) 
                - \frac{Z^3}{a_0^3} r_1^3 \exp(-\frac{2Zr_1}{a_0}) \right )
\end{eqnarray}

%
\begin{eqnarray}
\int \Psi_{1s}(r_1)\Psi_{2s}(r_2) \frac{V_2}{ke^2} \Psi_{1s}(r_2)\Psi_{2s}(r_1) d^3r_2
=\int \Psi_{1s}(r_1)\Psi_{2s}(r_1) \frac{V_2}{ke^2} \Psi_{1s}(r_2)\Psi_{2s}(r_2) d^3r_2 \nonumber \\
= \frac{1}{64\pi}\left ( \frac{Z}{a_0} \right )^3 \frac{1}{r_1}
\left [ \left ( -3s\frac{Z}{a_0} r_1 + 2s\frac{Z^2}{a_0^2} r_1^2 - 2s - \frac{Z}{a_0} r_1 +2
                                     \right ) \exp(- \frac{2Zr_1}{a_0})\right . \nonumber\\
+ \left . \left ( -2 + \frac{Z}{a_0} r_1 + 4 \frac{Z^2}{a_0^2} r_1^2 
                          - 2 \frac{Z^3}{a_0^3} r_1^3 \right )
                          \exp(- \frac{4Zr_1}{a_0}) \right ]
\end{eqnarray}
Til slutt et nyttig integral
\begin{equation}
\int_0^{\infty} x^n \exp(- x\alpha)dx = \frac{n!}{\alpha^{n+1}}
\end{equation}

  
\clearemptydoublepage
\chapter{Molecules}



\section{Introduksjon}
I v\aa r diskusjon av molekyler skal vi i all hovedsak se p\aa\
to atomer som danner et molekyl, som f.eks.~koksalt, NaCl. 
Bindings mekanismen mellom atomer i dannelsen av et molekyl skyldes
i all hovedsak elektrostatiske krefter mellom atomene (eller ionene).

N\aa r to atomer er langt borte fra hverandre er kreftene som virker mellom
dem tiln\ae rma lik null. Bringes atomene n\ae rmere kan en 
ha enten tiltrekkende
krefter eller frast\o tende krefter, slik at den potensielle energien
mellom atomene kan v\ae re positiv eller negativ, helt avhengig 
av avstanden $R$ mellom atomene. Den potensielle energien mellom atomene
kan approksimeres med  
\be
   V(R)=-\frac{A}{R^m}   -\frac{B}{R^n}   ,
\ee
hvor $m,n$ er heltall og $A,B$ konstanter med dimensjon energi-lengde,
dvs.~med enhet eVnm. Konstantene $A,B$ og heltallene  $m,n$ m\aa\
bestemmes for hvert molekyl utifra kravet om at vi \o nsker
\aa\ reprodusere f.eks.~bindingsenergi, spektra osv. Det er 
alts\aa\ et eksempel p\aa\ en parametrisert vekselvirkning, og
ikke en fundamental vekselvirkning slik som
Coulomb vekselvirkninger er.

Et velkjent potensial fra litteraturen er det s\aa kalla Lennard-Jones 
potensialet  gitt ved
\be
   V_{LJ}(R)=4\epsilon\left\{\left(\frac{\sigma}{R}\right)^{12}-
                             \left(\frac{\sigma}{R}\right)^6\right\},
\label{eq:lejo}
\ee
hvor $\epsilon$ og $\sigma$ er parametre som bestemmes ved \aa\ tilpasse
egenskaper til ulike molekyler. For vekselvirkningen mellom
He-atomer er $\epsilon=8.79\times 10^{-4}$ eV og $\sigma=0.256$ nm.
\begin{figure}
\begin{center}
% GNUPLOT: LaTeX picture with Postscript
\begingroup%
  \makeatletter%
  \newcommand{\GNUPLOTspecial}{%
    \@sanitize\catcode`\%=14\relax\special}%
  \setlength{\unitlength}{0.1bp}%
{\GNUPLOTspecial{!
%!PS-Adobe-2.0 EPSF-2.0
%%Title: lennard.tex
%%Creator: gnuplot 3.7 patchlevel 0.2
%%CreationDate: Wed Apr 26 12:50:27 2000
%%DocumentFonts: 
%%BoundingBox: 0 0 360 216
%%Orientation: Landscape
%%EndComments
/gnudict 256 dict def
gnudict begin
/Color false def
/Solid false def
/gnulinewidth 5.000 def
/userlinewidth gnulinewidth def
/vshift -33 def
/dl {10 mul} def
/hpt_ 31.5 def
/vpt_ 31.5 def
/hpt hpt_ def
/vpt vpt_ def
/M {moveto} bind def
/L {lineto} bind def
/R {rmoveto} bind def
/V {rlineto} bind def
/vpt2 vpt 2 mul def
/hpt2 hpt 2 mul def
/Lshow { currentpoint stroke M
  0 vshift R show } def
/Rshow { currentpoint stroke M
  dup stringwidth pop neg vshift R show } def
/Cshow { currentpoint stroke M
  dup stringwidth pop -2 div vshift R show } def
/UP { dup vpt_ mul /vpt exch def hpt_ mul /hpt exch def
  /hpt2 hpt 2 mul def /vpt2 vpt 2 mul def } def
/DL { Color {setrgbcolor Solid {pop []} if 0 setdash }
 {pop pop pop Solid {pop []} if 0 setdash} ifelse } def
/BL { stroke userlinewidth 2 mul setlinewidth } def
/AL { stroke userlinewidth 2 div setlinewidth } def
/UL { dup gnulinewidth mul /userlinewidth exch def
      10 mul /udl exch def } def
/PL { stroke userlinewidth setlinewidth } def
/LTb { BL [] 0 0 0 DL } def
/LTa { AL [1 udl mul 2 udl mul] 0 setdash 0 0 0 setrgbcolor } def
/LT0 { PL [] 1 0 0 DL } def
/LT1 { PL [4 dl 2 dl] 0 1 0 DL } def
/LT2 { PL [2 dl 3 dl] 0 0 1 DL } def
/LT3 { PL [1 dl 1.5 dl] 1 0 1 DL } def
/LT4 { PL [5 dl 2 dl 1 dl 2 dl] 0 1 1 DL } def
/LT5 { PL [4 dl 3 dl 1 dl 3 dl] 1 1 0 DL } def
/LT6 { PL [2 dl 2 dl 2 dl 4 dl] 0 0 0 DL } def
/LT7 { PL [2 dl 2 dl 2 dl 2 dl 2 dl 4 dl] 1 0.3 0 DL } def
/LT8 { PL [2 dl 2 dl 2 dl 2 dl 2 dl 2 dl 2 dl 4 dl] 0.5 0.5 0.5 DL } def
/Pnt { stroke [] 0 setdash
   gsave 1 setlinecap M 0 0 V stroke grestore } def
/Dia { stroke [] 0 setdash 2 copy vpt add M
  hpt neg vpt neg V hpt vpt neg V
  hpt vpt V hpt neg vpt V closepath stroke
  Pnt } def
/Pls { stroke [] 0 setdash vpt sub M 0 vpt2 V
  currentpoint stroke M
  hpt neg vpt neg R hpt2 0 V stroke
  } def
/Box { stroke [] 0 setdash 2 copy exch hpt sub exch vpt add M
  0 vpt2 neg V hpt2 0 V 0 vpt2 V
  hpt2 neg 0 V closepath stroke
  Pnt } def
/Crs { stroke [] 0 setdash exch hpt sub exch vpt add M
  hpt2 vpt2 neg V currentpoint stroke M
  hpt2 neg 0 R hpt2 vpt2 V stroke } def
/TriU { stroke [] 0 setdash 2 copy vpt 1.12 mul add M
  hpt neg vpt -1.62 mul V
  hpt 2 mul 0 V
  hpt neg vpt 1.62 mul V closepath stroke
  Pnt  } def
/Star { 2 copy Pls Crs } def
/BoxF { stroke [] 0 setdash exch hpt sub exch vpt add M
  0 vpt2 neg V  hpt2 0 V  0 vpt2 V
  hpt2 neg 0 V  closepath fill } def
/TriUF { stroke [] 0 setdash vpt 1.12 mul add M
  hpt neg vpt -1.62 mul V
  hpt 2 mul 0 V
  hpt neg vpt 1.62 mul V closepath fill } def
/TriD { stroke [] 0 setdash 2 copy vpt 1.12 mul sub M
  hpt neg vpt 1.62 mul V
  hpt 2 mul 0 V
  hpt neg vpt -1.62 mul V closepath stroke
  Pnt  } def
/TriDF { stroke [] 0 setdash vpt 1.12 mul sub M
  hpt neg vpt 1.62 mul V
  hpt 2 mul 0 V
  hpt neg vpt -1.62 mul V closepath fill} def
/DiaF { stroke [] 0 setdash vpt add M
  hpt neg vpt neg V hpt vpt neg V
  hpt vpt V hpt neg vpt V closepath fill } def
/Pent { stroke [] 0 setdash 2 copy gsave
  translate 0 hpt M 4 {72 rotate 0 hpt L} repeat
  closepath stroke grestore Pnt } def
/PentF { stroke [] 0 setdash gsave
  translate 0 hpt M 4 {72 rotate 0 hpt L} repeat
  closepath fill grestore } def
/Circle { stroke [] 0 setdash 2 copy
  hpt 0 360 arc stroke Pnt } def
/CircleF { stroke [] 0 setdash hpt 0 360 arc fill } def
/C0 { BL [] 0 setdash 2 copy moveto vpt 90 450  arc } bind def
/C1 { BL [] 0 setdash 2 copy        moveto
       2 copy  vpt 0 90 arc closepath fill
               vpt 0 360 arc closepath } bind def
/C2 { BL [] 0 setdash 2 copy moveto
       2 copy  vpt 90 180 arc closepath fill
               vpt 0 360 arc closepath } bind def
/C3 { BL [] 0 setdash 2 copy moveto
       2 copy  vpt 0 180 arc closepath fill
               vpt 0 360 arc closepath } bind def
/C4 { BL [] 0 setdash 2 copy moveto
       2 copy  vpt 180 270 arc closepath fill
               vpt 0 360 arc closepath } bind def
/C5 { BL [] 0 setdash 2 copy moveto
       2 copy  vpt 0 90 arc
       2 copy moveto
       2 copy  vpt 180 270 arc closepath fill
               vpt 0 360 arc } bind def
/C6 { BL [] 0 setdash 2 copy moveto
      2 copy  vpt 90 270 arc closepath fill
              vpt 0 360 arc closepath } bind def
/C7 { BL [] 0 setdash 2 copy moveto
      2 copy  vpt 0 270 arc closepath fill
              vpt 0 360 arc closepath } bind def
/C8 { BL [] 0 setdash 2 copy moveto
      2 copy vpt 270 360 arc closepath fill
              vpt 0 360 arc closepath } bind def
/C9 { BL [] 0 setdash 2 copy moveto
      2 copy  vpt 270 450 arc closepath fill
              vpt 0 360 arc closepath } bind def
/C10 { BL [] 0 setdash 2 copy 2 copy moveto vpt 270 360 arc closepath fill
       2 copy moveto
       2 copy vpt 90 180 arc closepath fill
               vpt 0 360 arc closepath } bind def
/C11 { BL [] 0 setdash 2 copy moveto
       2 copy  vpt 0 180 arc closepath fill
       2 copy moveto
       2 copy  vpt 270 360 arc closepath fill
               vpt 0 360 arc closepath } bind def
/C12 { BL [] 0 setdash 2 copy moveto
       2 copy  vpt 180 360 arc closepath fill
               vpt 0 360 arc closepath } bind def
/C13 { BL [] 0 setdash  2 copy moveto
       2 copy  vpt 0 90 arc closepath fill
       2 copy moveto
       2 copy  vpt 180 360 arc closepath fill
               vpt 0 360 arc closepath } bind def
/C14 { BL [] 0 setdash 2 copy moveto
       2 copy  vpt 90 360 arc closepath fill
               vpt 0 360 arc } bind def
/C15 { BL [] 0 setdash 2 copy vpt 0 360 arc closepath fill
               vpt 0 360 arc closepath } bind def
/Rec   { newpath 4 2 roll moveto 1 index 0 rlineto 0 exch rlineto
       neg 0 rlineto closepath } bind def
/Square { dup Rec } bind def
/Bsquare { vpt sub exch vpt sub exch vpt2 Square } bind def
/S0 { BL [] 0 setdash 2 copy moveto 0 vpt rlineto BL Bsquare } bind def
/S1 { BL [] 0 setdash 2 copy vpt Square fill Bsquare } bind def
/S2 { BL [] 0 setdash 2 copy exch vpt sub exch vpt Square fill Bsquare } bind def
/S3 { BL [] 0 setdash 2 copy exch vpt sub exch vpt2 vpt Rec fill Bsquare } bind def
/S4 { BL [] 0 setdash 2 copy exch vpt sub exch vpt sub vpt Square fill Bsquare } bind def
/S5 { BL [] 0 setdash 2 copy 2 copy vpt Square fill
       exch vpt sub exch vpt sub vpt Square fill Bsquare } bind def
/S6 { BL [] 0 setdash 2 copy exch vpt sub exch vpt sub vpt vpt2 Rec fill Bsquare } bind def
/S7 { BL [] 0 setdash 2 copy exch vpt sub exch vpt sub vpt vpt2 Rec fill
       2 copy vpt Square fill
       Bsquare } bind def
/S8 { BL [] 0 setdash 2 copy vpt sub vpt Square fill Bsquare } bind def
/S9 { BL [] 0 setdash 2 copy vpt sub vpt vpt2 Rec fill Bsquare } bind def
/S10 { BL [] 0 setdash 2 copy vpt sub vpt Square fill 2 copy exch vpt sub exch vpt Square fill
       Bsquare } bind def
/S11 { BL [] 0 setdash 2 copy vpt sub vpt Square fill 2 copy exch vpt sub exch vpt2 vpt Rec fill
       Bsquare } bind def
/S12 { BL [] 0 setdash 2 copy exch vpt sub exch vpt sub vpt2 vpt Rec fill Bsquare } bind def
/S13 { BL [] 0 setdash 2 copy exch vpt sub exch vpt sub vpt2 vpt Rec fill
       2 copy vpt Square fill Bsquare } bind def
/S14 { BL [] 0 setdash 2 copy exch vpt sub exch vpt sub vpt2 vpt Rec fill
       2 copy exch vpt sub exch vpt Square fill Bsquare } bind def
/S15 { BL [] 0 setdash 2 copy Bsquare fill Bsquare } bind def
/D0 { gsave translate 45 rotate 0 0 S0 stroke grestore } bind def
/D1 { gsave translate 45 rotate 0 0 S1 stroke grestore } bind def
/D2 { gsave translate 45 rotate 0 0 S2 stroke grestore } bind def
/D3 { gsave translate 45 rotate 0 0 S3 stroke grestore } bind def
/D4 { gsave translate 45 rotate 0 0 S4 stroke grestore } bind def
/D5 { gsave translate 45 rotate 0 0 S5 stroke grestore } bind def
/D6 { gsave translate 45 rotate 0 0 S6 stroke grestore } bind def
/D7 { gsave translate 45 rotate 0 0 S7 stroke grestore } bind def
/D8 { gsave translate 45 rotate 0 0 S8 stroke grestore } bind def
/D9 { gsave translate 45 rotate 0 0 S9 stroke grestore } bind def
/D10 { gsave translate 45 rotate 0 0 S10 stroke grestore } bind def
/D11 { gsave translate 45 rotate 0 0 S11 stroke grestore } bind def
/D12 { gsave translate 45 rotate 0 0 S12 stroke grestore } bind def
/D13 { gsave translate 45 rotate 0 0 S13 stroke grestore } bind def
/D14 { gsave translate 45 rotate 0 0 S14 stroke grestore } bind def
/D15 { gsave translate 45 rotate 0 0 S15 stroke grestore } bind def
/DiaE { stroke [] 0 setdash vpt add M
  hpt neg vpt neg V hpt vpt neg V
  hpt vpt V hpt neg vpt V closepath stroke } def
/BoxE { stroke [] 0 setdash exch hpt sub exch vpt add M
  0 vpt2 neg V hpt2 0 V 0 vpt2 V
  hpt2 neg 0 V closepath stroke } def
/TriUE { stroke [] 0 setdash vpt 1.12 mul add M
  hpt neg vpt -1.62 mul V
  hpt 2 mul 0 V
  hpt neg vpt 1.62 mul V closepath stroke } def
/TriDE { stroke [] 0 setdash vpt 1.12 mul sub M
  hpt neg vpt 1.62 mul V
  hpt 2 mul 0 V
  hpt neg vpt -1.62 mul V closepath stroke } def
/PentE { stroke [] 0 setdash gsave
  translate 0 hpt M 4 {72 rotate 0 hpt L} repeat
  closepath stroke grestore } def
/CircE { stroke [] 0 setdash 
  hpt 0 360 arc stroke } def
/Opaque { gsave closepath 1 setgray fill grestore 0 setgray closepath } def
/DiaW { stroke [] 0 setdash vpt add M
  hpt neg vpt neg V hpt vpt neg V
  hpt vpt V hpt neg vpt V Opaque stroke } def
/BoxW { stroke [] 0 setdash exch hpt sub exch vpt add M
  0 vpt2 neg V hpt2 0 V 0 vpt2 V
  hpt2 neg 0 V Opaque stroke } def
/TriUW { stroke [] 0 setdash vpt 1.12 mul add M
  hpt neg vpt -1.62 mul V
  hpt 2 mul 0 V
  hpt neg vpt 1.62 mul V Opaque stroke } def
/TriDW { stroke [] 0 setdash vpt 1.12 mul sub M
  hpt neg vpt 1.62 mul V
  hpt 2 mul 0 V
  hpt neg vpt -1.62 mul V Opaque stroke } def
/PentW { stroke [] 0 setdash gsave
  translate 0 hpt M 4 {72 rotate 0 hpt L} repeat
  Opaque stroke grestore } def
/CircW { stroke [] 0 setdash 
  hpt 0 360 arc Opaque stroke } def
/BoxFill { gsave Rec 1 setgray fill grestore } def
end
%%EndProlog
}}%
\begin{picture}(3600,2160)(0,0)%
{\GNUPLOTspecial{"
gnudict begin
gsave
0 0 translate
0.100 0.100 scale
0 setgray
newpath
1.000 UL
LTb
550 300 M
63 0 V
2837 0 R
-63 0 V
550 520 M
63 0 V
2837 0 R
-63 0 V
550 740 M
63 0 V
2837 0 R
-63 0 V
550 960 M
63 0 V
2837 0 R
-63 0 V
550 1180 M
63 0 V
2837 0 R
-63 0 V
550 1400 M
63 0 V
2837 0 R
-63 0 V
550 1620 M
63 0 V
2837 0 R
-63 0 V
550 1840 M
63 0 V
2837 0 R
-63 0 V
550 2060 M
63 0 V
2837 0 R
-63 0 V
765 300 M
0 63 V
0 1697 R
0 -63 V
1302 300 M
0 63 V
0 1697 R
0 -63 V
1839 300 M
0 63 V
0 1697 R
0 -63 V
2376 300 M
0 63 V
0 1697 R
0 -63 V
2913 300 M
0 63 V
0 1697 R
0 -63 V
3450 300 M
0 63 V
0 1697 R
0 -63 V
1.000 UL
LTb
550 300 M
2900 0 V
0 1760 V
-2900 0 V
550 300 L
1.000 UL
LT0
3087 1947 M
263 0 V
550 1846 M
29 -269 V
30 -224 V
29 -187 V
29 -156 V
696 880 L
726 772 L
29 -89 V
29 -74 V
30 -61 V
29 -50 V
29 -41 V
30 -32 V
29 -27 V
29 -20 V
29 -17 V
30 -12 V
29 -9 V
29 -6 V
30 -4 V
29 -3 V
29 0 V
29 0 V
30 2 V
29 2 V
29 3 V
30 4 V
29 4 V
29 4 V
29 5 V
30 5 V
29 5 V
29 4 V
30 5 V
29 5 V
29 5 V
30 5 V
29 5 V
29 4 V
29 5 V
30 4 V
29 5 V
29 4 V
30 4 V
29 4 V
29 3 V
29 4 V
30 4 V
29 3 V
29 3 V
30 3 V
29 3 V
29 3 V
30 3 V
29 3 V
29 2 V
29 3 V
30 2 V
29 3 V
29 2 V
30 2 V
29 2 V
29 2 V
29 2 V
30 2 V
29 1 V
29 2 V
30 1 V
29 2 V
29 1 V
30 2 V
29 1 V
29 1 V
29 2 V
30 1 V
29 1 V
29 1 V
30 1 V
29 1 V
29 1 V
29 1 V
30 1 V
29 1 V
29 1 V
30 0 V
29 1 V
29 1 V
29 1 V
30 0 V
29 1 V
29 1 V
30 0 V
29 1 V
29 0 V
30 1 V
29 0 V
29 1 V
29 0 V
30 1 V
29 0 V
stroke
grestore
end
showpage
}}%
\put(3037,1947){\makebox(0,0)[r]{$V(R)$}}%
\put(2000,50){\makebox(0,0){$R$ [nm]}}%
\put(100,1180){%
\special{ps: gsave currentpoint currentpoint translate
270 rotate neg exch neg exch translate}%
\makebox(0,0)[b]{\shortstack{$V(R)$}}%
\special{ps: currentpoint grestore moveto}%
}%
\put(3450,200){\makebox(0,0){0.5}}%
\put(2913,200){\makebox(0,0){0.45}}%
\put(2376,200){\makebox(0,0){0.4}}%
\put(1839,200){\makebox(0,0){0.35}}%
\put(1302,200){\makebox(0,0){0.3}}%
\put(765,200){\makebox(0,0){0.25}}%
\put(500,2060){\makebox(0,0)[r]{0.007}}%
\put(500,1840){\makebox(0,0)[r]{0.006}}%
\put(500,1620){\makebox(0,0)[r]{0.005}}%
\put(500,1400){\makebox(0,0)[r]{0.004}}%
\put(500,1180){\makebox(0,0)[r]{0.003}}%
\put(500,960){\makebox(0,0)[r]{0.002}}%
\put(500,740){\makebox(0,0)[r]{0.001}}%
\put(500,520){\makebox(0,0)[r]{0}}%
\put(500,300){\makebox(0,0)[r]{-0.001}}%
\end{picture}%
\endgroup
\endinput

\end{center}
\caption{Plot av potensialet i likning (\ref{eq:lejo}) for Van der Waals vekselvirkning
mellom He-atomer. Likevektsposisjonen $r_0=0.287$ nm.\label{fig:61}}
\end{figure} 
Figur \ref{fig:61} viser en skisse av dette potensialet for disse parametrene.
De kvalitative egenskapene i denne figuren gjelder for alle
molekyler. 
Potensialet er b\aa de frast\o tende og tiltrekkende og vi merker
oss at det utviser et minimum for en bestemt verdi $R$, som vi skal kalle
$r_0$. 
Dette minimum definerer  grunntilstanden (st\o rst bindingsenergi) 
for molekylet og avstanden $r_0$ er da likevektsavstanden mellom
atomene n\aa r molekylet er i sin grunntilstand. Grunnen til at vi har 
frast\o ting ved sm\aa\ verdier av $R$ skyldes like ladninger som
frast\o ter hverandre samt Pauli prinsippet, partikler med 
halvtallig spinn kan ikke ha samme sett kvantetall. 

Typisk for molekyler er at dissosiasjonsenergien\footnote{Vi kommer til 
\aa\ veksle mellom \aa\ bruke ordet bindingsenergi og dissosiasjonsenergi. 
Men n\aa r vi bruker bindingsenergi i molekylsammenheng
s\aa\ tenker vi alt\aa\ p\aa\ den energien som er n\o dvendig for
\aa\ binde to atomer sammen. Denne energien er selvsagt forskjellig
fra bindingsenergien til det enkelte atom.}, energien 
som trengs for \aa\ separere et molekyl i dets enkeltatomer,
er mindre enn 
bindingsnergien for atomene som inng\aa r. 
For \aa\ ta eksemplet
som vi skal diskutere i avsnitt 10.3, s\aa\ er dissosiasjonsenergien
for H$_2^+$ molekylet som best\aa r av to protoner og et elektron
gitt ved $E_B=-2.8$ eV. Ser vi tilbake p\aa\ hydrogenatomet, s\aa\
er bindingsenergien gitt ved $E_B=E_0=-13.6$ eV.

Likevektsavstanden er ogs\aa\ st\o rre enn den typiske
st\o rrelsen til hydrogenatomet. Vi husker tilbake fra kapittel 7 at
midlere radius for hydrogenatomet er ca.~en Bohrradius, dvs.~0.0529 nm.
Avstanden mellom de to protonene i H$_2^+$ molekylet er $r_0=0.106$ nm.

I v\aa r diskusjon om binding, skal vi i all hovedsak se p\aa\
ionisk og kovalent binding. 
Ionebinding er en av de mest vanlige bindingsformer, 
med koksalt som det klassiske eksemplet. For NaCl er
dissosiasjonsenergien  $4.2$ eV.  Likevektsavstanden $r_0=0.24$ nm. Andre typer
binding er s\aa kalla kovalent binding\footnote{Avsnitt 10.2 i boka.} med 
H$_2^+$, H$_2$, CO, H$_2$O, CO$_2$ og CH$_4$ molekylene 
som velkjente eksempler.
Hydrogen-binding og Van der Waals binding\footnote{Avsnitt 10.4 i boka.} er
andre eksempler.


I dette kapittelet skal vi se p\aa\ egenskaper ved molekyler
som er ulike fra det vi har sett ved atomer. I tillegg til 
et eksitasjonsspekter som skyldes elektrostatiske krefter, har vi 
en tilleggsfinstruktur som har opphav i rotasjons og vibrasjons
bidrag til energien. Det skal vi se p\aa\ i neste avsnitt.

Vi avslutter med en diskusjon av H$_2^+$ og H$_2$ molekylene
og kovalent binding.

\begin{figure}[h]
\begin{center}
{\centering
\mbox
{\psfig{figure=st10f1.ps,height=8cm,width=8cm}}
}
\end{center}
\caption{Et mer eksotisk eksempel utgj\o r disse to mulige tistandene
for en berylliumkjerne, som i dette tilfelle best\aa r av 12 partikler,
4 protoner og 8 n\o ytroner. Vi kan anta den settes sammen av
to $\alpha$-partikler (2 protoner og to n\o ytroner) samt fire ekstra
n\o ytroner. For mer informasjon, se  Phys.~Rev.~Lett.~{\bf 82} (1999) 1383.
}
\end{figure}
\begin{figure}[h]
\begin{center}
{\centering
\mbox
{\psfig{figure=st11f1.ps,height=6cm,width=8cm}}
}
\end{center}
\caption{En Molekyl\ae r gjeterhund: flytende krystallmolekyler
(ellipser) innrettes i et magnetfelt og tvinger organiske molekyler til \aa\
rettes opp langs ei linje. For mer informasjon, se  
Phys.~Rev.~Lett.~{\bf 84} (2000) 2742.}
\end{figure}
\begin{figure}[h]
\begin{center}
{\centering
\mbox
{\psfig{figure=st14f1.ps,height=6cm,width=8cm}}
}
\end{center}
\caption{Et lysmolekyl: Fotoner kan fanges inn 
i halvledere p\aa\ et vis som minner om elektroner i et atom.
I  Phys.~Rev.~Lett.~{\bf 81} (1998) 2582 klarte en \aa\ kople slike fotoniske
atomer sammen, med mange av de samme egenskapene som elektroner utviser i 
et molekyl. }
\end{figure}
\begin{figure}[h]
\begin{center}
{\centering
\mbox
{\psfig{figure=st32f1.ps,height=8cm,width=8cm}}
}
\end{center}
\caption{Krystallinsk sprettert: H$_2$ molekyler ( i r\o dt) med ulike kjernespinn p\aa virkes ulikt av overflata til en litium-fluorid krystall.
Resultatene av en slik p\aa virkning kan fortelle om elektriske
krefter ved overflata til en krystall. 
For mer informasjon, se 
Phys.~Rev.~Lett.~{\bf 81} (1998) 5608. }
\end{figure}



\section{Rotasjon og vibrasjon}

Vi skal begrense oss til to-atomige molekyler. Energien til et slikt
system kan beskrives ved 
\be
    E=E_{\mathrm{atom}}+E_{\mathrm{rot}}+E_{\mathrm{vib}}=
      \mathrm{konstant}+E_{\mathrm{rot}}+E_{\mathrm{vib}},
\ee
siden vi er interessert i 
bidragene fra rotasjon og vibrasjon\footnote{Lesehenvisning er avsnittene 
10-6, 10-7 og 10-8.}. Disse  er mye mindre enn den atom\ae re
delen $E_{\mathrm{atom}}$, som igjen har sitt opphav i Coulomb 
vekselvirkningen mellom elektroner og mellom elektroner og kjernen. 

Dersom dere tenker tilbake til hydrogenatomet eller andre atomer i det 
periodiske systemet, s\aa\ har vi at det er flere elektronvolt i forskjell
mellom f.eks.~grunntilstanden til hydrogenatomet og dets f\o rste
eksiterte tilstand. Bidraget fra rotasjonsdelen er typisk p\aa\
st\o rrelsesorden med $10^{-4}$ eV, mens vibrasjonsbidraget er p\aa\
st\o rrelse med $0.1-1$ eV. Vi kan da tenke oss at i tillegg til det vanlige
eksitasjonsspekteret som skyldes Coulombvekselvirkningen  mellom
molekylets ulike bestandeler, s\aa\ har vi ogs\aa\ en finstruktur i energien
som skyldes rotasjon og vibrasjon. Det er disse to frihetsgradene
som bla.~skiller et atom fra et molekyl. 

I dette avsnittet skal vi se bort ifra det atom\ae re bidraget til
energien og kun se p\aa\ finstrukturen som skyldes rotasjon og vibrasjon. 
Til slutt skal vi diskutere molekylspekteret en f\aa r pga.~rotasjon
og vibrasjon.

\subsection{Rotasjon i molekyler}

Anta n\aa\ at vi har to atomer med masse $m_1$ og $m_2$ som danner
et molekyl. Schr\"odingers likning i dette tilfelle blir 
\be
    \left\{-\frac{\hbar^2\nabla_1^2}{2m_1}-\frac{\hbar^2\nabla_2^2}{2m_2}
   +V(r_1,r_2)\right\}\Psi(r_1,r_2)=E\Psi(r_1,r_2),
\ee
hvor $V(r_1,r_2)$ er vekselvirkningen mellom de to molekylene.
Radiene $r_1$ og $r_2$ er avstanden fra et valgt origo til de to atomene.

Vi foretar s\aa\ en transformasjon til tyngdepunktssystemet og definerer
relativ avstand mellom atomene som
\be
   {\bf r}={\bf r}_1-{\bf r}_2,
\ee
redusert masse
\be
  \mu=\frac{m_1m_2}{m_1+m_2},
\ee
og massesenter koordinaten $R$ 
\be
    {\bf R}=\frac{1}{m_1+m_2}\left(m_1{\bf r}_1+m_2{\bf r}_2\right).
\ee

Vi skal som vanlig fors\o ke \aa\ gj\o re livet s\aa\ enkelt som
mulig for oss selv (uten \aa\ tape viktig fysikk) 
og velger derfor 
\aa\ se p\aa\ rotasjon om tyngdepunktet med origo
i tyngdepunktet, dvs.~at vi setter ${\bf R}=0$. Dermed faller den delen av
Schr\"odingers likning som avhenger av ${\bf R}$ bort og vi sitter igjen med en likning
for den relative avstanden $r$
\be
    \left\{-\frac{\hbar^2\nabla_r^2}{2\mu}
   +V(r)\right\}\Psi({\bf r})=E\Psi({\bf r}).
\ee
Siden potensialet avhenger kun av absoluttverdien av ${\bf r} $ kan vi pr\o ve
oss p\aa\ tilsvarende ansats for b\o lgefunksjonen $\Psi$ som vi gjorde
for hydrogenatomet; vi skiller ut en vinkeldel og en radiell del
\be
   \Psi({\bf r}) = R(r)Y_{LM_L}(\theta,\phi),
\ee
slik at den radielle delen av Schr\"odingers likning blir n\aa\
\be
    \left\{-\frac{\hbar^2}{2\mu r^2}\frac{d}{dr}\left(r^2\frac{d}{dr}\right)
    +\frac{\hbar^2L(L+1)}{2\mu r^2}
   +V(r)\right\}R(r)=ER(r),
\ee
hvor $L$ er banespinnet til systemet med verdi $L=0,1,2,\dots$
Vi ser kun p\aa\ rotasjonsbevegelse om massesenteret samtidig som vi antar
at avstanden mellom atomene er konstant. Vi setter derfor $r=r_0$.
Den potensielle energien er $V(r=r_0)$ som er en konstant. Schr\"odingers likning reduserer
seg i dette tilfellet til
\be
    \left(\frac{\hbar^2L(L+1)}{2\mu r_0^2}
   +V(r_0)\right\}R(r_0)=ER(r_0),
\ee
siden $r_0$ er en konstant. Det betyr at energien er gitt ved
\be
    \frac{\hbar^2L(L+1)}{2\mu r_0^2}+V(r_0)=E=E_{\mathrm{rot}}+V(r_0),
\ee
med rotasjonsleddet gitt ved
\be
  E_{\mathrm{rot}}= \frac{\hbar^2L(L+1)}{2\mu r_0^2}.
\ee

Hvordan skal  vi tolke nevneren $1/\mu r_0^2$ 
i uttrykket for rotasjonsenergien?
Her f\o lger en klassisk betraktning. For klassisk rotasjon med konstant
vinkelfrekvens $\omega$ og med 
to gjenstander hvis hastigheter er henholdsvis
\be
   v_1=\omega r_1,
\ee
og 
\be
   v_2=\omega r_2,
\ee
har vi at det totale banespinnet er gitt ved
\be
   L=m_1v_1r_1+m_2v_2r_2=\left(m_1r_1^2+m_2r_2^2\right)\omega=I\omega,
\ee
hvor $I=m_1r_1^2+m_2r_2^2$ er treghetsmomentet. Den kinetiske
energien kan derved uttrykkes vha.~treghetsmomentet og banespinnet via
\be
   E=\frac{1}{2}m_1v_1^2+\frac{1}{2}m_2v_2^2=
  \frac{1}{2}I\omega^2=\frac{L^2}{2I},
\ee
og siden vi har valgt \aa\ se p\aa\ rotasjon om tyngdepunktet med origo
i tyngdepunktet, dvs.~${\bf R}=0$, 
har vi at
\[
   {\bf R}={\bf 0}=m_1{\bf r}_1+m_2{\bf r}_2,
\]
som igjen betyr at ${\bf r}={\bf r}_1+{\bf r}_2$
slik at vi kan uttrykke ${\bf r}_1$ og ${\bf r}_2$
vha.~${\bf r}$ som
\[
   {\bf r}_1=\frac{m_2}{m_1+m_2}{\bf r}, 
\]
og 
\[
   {\bf r}_2=\frac{m_1}{m_1+m_2}{\bf r} .
\]
Treghetsmomentet i tyngdepunktssystemet blir dermed
\be 
    I=m_1r_1^2+m_2r_2^2=I=m_1\left(\frac{m_2}{m_1+m_2}{\bf r}\right)^2
                          +m_2\left(\frac{m_1}{m_1+m_2}{\bf r}\right)^2
    =\mu{\bf r}^2=I_{\mathrm{cm}}.
\ee
Vi kan da omskrive uttrykket for rotasjonsenergien vha.~treghetsmomentet
som
\be
  E_{\mathrm{rot}}= \frac{\hbar^2L(L+1)}{2I_{\mathrm{cm}}}=\frac{\hbar^2L(L+1)}{2\mu r_0^2}
\ee

Vi skal diskutere spekteret som framkommer pga.~rotasjonsfrihetsgrader
i slutten av dette avsnittet. Vi merker oss her at vi vil f\aa\ en herskare
av rotasjonsniv\aa er pga.~alle ulike verdier som $L$ kan anta. 


\subsection{Vibrasjon i molekyler}

Vi setter $L=0$, slik at den radielle delen av Schr\"odingers likning reduserer seg til
\be
    \left\{-\frac{\hbar^2}{2\mu r^2}\frac{d}{dr}\left(r^2\frac{d}{dr}\right)
   +V(r)\right\}R(r)=ER(r),
   \label{eq:vibramol}
\ee
og velger vi 
\[
   R(r)=\frac{1}{r}u(r),
\]
kan vi omskrive, slik vi gjorde for hydrogenatomet, likning
(\ref{eq:vibramol})  
som
\be
    \left\{-\frac{\hbar^2}{2\mu}\frac{d^2}{dr^2}+V(r)\right\}u(r)=Eu(r).
\ee
Generelt s\aa\ har den potensielle energien for et molekyl en komplisert
form som vist i Figur \ref{fig:61}. 

Vi skal anta at vi ser p\aa\ sm\aa\ forstyrrelser rundt likevektavstanden
$r_0$ mellom molekylene. Rundt likevektsavstanden approksimerer vi deretter
potensialet med et harmonisk oscillator potensial som vist i 
Figur \ref{fig:vibmolekylapproks}.
\begin{figure}[h]
\setlength{\unitlength}{1mm}
   \begin{picture}(100,100)
   \put(0,-50){\epsfxsize=16cm \epsfbox{molekylapproks.ps}}
   \end{picture}
\caption{En ilustrasjon av et molekyl\ae rt potensial og ulike
approksimasjoner som harmonisk oscillator og en uendelig potensialbr\o nn. 
\label{fig:vibmolekylapproks}}
\end{figure}

Vi velger dermed
\be
   V(r)\approx \frac{1}{2}k(r-r_0)^2,
\ee
og med et variabel bytte $\eta  =r-r_0$ f\aa r vi tilbake
Schr\"odingers likning for den harmoniske oscillator, dvs.
\be
    \left\{-\frac{\hbar^2}{2\mu}\frac{d^2}{d\eta^2}+
    \frac{1}{2}k\eta^2\right\}u(\eta)=Eu(\eta).
\ee
Bildet vi lager er av atomer som vibrerer rundt en likevektsstilling,
derav benevningen vibrasjonsenergi. Vi kan tenke oss atomene som holdt
sammen av ei 'fj\ae r' med lengde $r_0$ og at vi har en kraftkonstant
$k$. Vi tiln\ae rmer disse vibrasjoner som om de skulle foreg\aa\
i en dimensjon, og kan derfor redusere Schr\"odingers likning til en en-dimensjonal
likning, i dette tilfelle approksimert ved et harmonisk
oscillator potensial. 

Til denne likningen kjenner vi uttrykket for energien, nemlig
\be
   E=\hbar \omega \left(n+\frac{1}{2}\right) \hspace{0.5cm} n=0,1,\dots ,
\ee
med 
\be
  \omega=\sqrt{\frac{k}{\mu}}.
\ee

Energien skal svare til vibrasjonsenergien 
\be
   E_{\mathrm{vib}}=\hbar \omega \left(n+\frac{1}{2}\right) \hspace{0.5cm} n=0,1,\dots.
\ee

Legg merke til at tenkningen v\aa r b\aa de i forhold til rotasjon
og vibrasjon er basert p\aa\ antagelser med basis i klassisk
mekanikk. Forskjellen fra klassisk fysikk er at b\aa de vibrasjons
og rotasjons energiene er kvantiserte. 

I tillegg skal vi merke oss at approksimasjonen vi har valgt
for vibrasjonsenergien kun gjelder for sm\aa\ forskyvninger utifra
likevektsposisjonen $r_0$. Dersom $\eta=r-r_0$ blir stor gjelder
ikke lenger harmonisk oscillator approksimasjonen til den potensielle
energien $V(r)$. 

\subsection{Molekylspektra} 

Den totale energien er n\aa\ gitt ved
\be
    E=E_{\mathrm{atom}}+E_{\mathrm{rot}}+E_{\mathrm{vib}}=
      \mathrm{konstant}+\frac{\hbar^2L(L+1)}{2I_{\mathrm{cm}}}+
      \hbar \omega \left(n+\frac{1}{2}\right),
\ee
med $L=0,1,2,\dots$ og $n=0,1,2,\dots$. 

Ser vi p\aa\ elektrisk dipoloverganger har vi f\o lgende utvalgsregler
\begin{enumerate}
 \item $\Delta L =\pm 1$ mellom rotasjonsniv\aa er.
 \item $\Delta n=0, \pm 1$ mellom vibrasjonsniv\aa er.
\end{enumerate}
Det betyr at energiforskjellen pga.~overganger 
mellom rotasjonsniv\aa er avhenger av hvorvidt vi har
$\Delta L =\pm 1$. For $\Delta L =1$ finner vi n\aa r vi g\aa r fra
$L$ til $L+1$ 
\be
   \Delta E_{\mathrm{rot}}= \frac{\hbar^2(L+1)}{I_{\mathrm{cm}}},
\ee
mens for overgangen $L\rightarrow L-1$ finner vi
\be
   \Delta E_{\mathrm{rot}}= -\frac{\hbar^2L}{I_{\mathrm{cm}}}.
\ee
Legger vi sammen rotasjons og vibrasjonsforandringene kan vi oppsummere
energidifferansene for elektriske dipoloverganger ved
\begin{enumerate}
\item $L\rightarrow L+1$ og $\Delta n=1$ 
\be
   \Delta E_{\mathrm{rot}}+\Delta E_{\mathrm{vib}}=\hbar\omega+ \frac{\hbar^2(L+1)}{I_{\mathrm{cm}}},
\ee
 mens for 
\item $L\rightarrow L-1$ og $\Delta n=1$ har vi 
\be
   \Delta E_{\mathrm{rot}}+\Delta E_{\mathrm{vib}}=\hbar\omega-\frac{\hbar^2L}{I_{\mathrm{cm}}}.
\ee
\end{enumerate}

La oss avslutte dette avsnittet med \aa\ se p\aa\ et utvalgt eksempel,
nemlig CO molekylet. 
Den f\o rste tabellen gir oss eksperimentelle verdier for ulike absorbsjonsoverganger (absorbsjon av e.m.~str\aa ling) for CO molekylet.
\begin{table}[h]
\caption{Absorbsjons linjer for rotasjonsoverganger i CO molekylet.}
\begin{center}
\begin{tabular}{lll}\hline
{\bf Rotasjons overganger} & {\bf B\o lgelengde} & {\bf Frekvens}\\ \hline
$L=0\rightarrow L=1$ & $2.60\times 10^{-3}$ m  & $1.15\times 10^{11}$ Hz \\ 
$L=1\rightarrow L=2$ & $1.30\times 10^{-3}$ m  & $2.30\times 10^{11}$ Hz\\ 
$L=2\rightarrow L=3$ & $8.77\times 10^{-4}$ m  & $3.46\times 10^{11}$ Hz\\ 
$L=3\rightarrow L=4$ & $6.50\times 10^{-5}$ m  & $4.61\times 10^{11}$ Hz\\ \hline
\end{tabular}
\end{center}
\end{table}

Ser vi p\aa\  $L=0\rightarrow L=1$  overgangen kan vi bruke denne til \aa\
rekne ut treghetsmomentet. Det absorberte fotonet har overf\o rt en
energi $h\nu=\hbar\omega$ slik at 
\be
   \Delta E_{\mathrm{rot}}= \frac{\hbar^2(L+1)}{2I_{\mathrm{cm}}}= \frac{\hbar^2}{I_{\mathrm{cm}}}=\hbar\omega,
\ee
som gir at 
\be
    I_{\mathrm{cm}}=\frac{\hbar}{\omega}.
\ee
Legg ogs\aa\ merke til at energien $\Delta E_{\mathrm{rot}}=\hbar\omega$ er
p\aa\ $7.57\times 10^{-5}$ eV, mye mindre enn bindingsenergien som er p\aa\
noen f\aa\ elektronvolt. 
Henter vi den observerte frekvensen fra tabellen v\aa r har vi
\be
   \omega=2\pi \nu=7.23\times 10^{11} \hspace{0.1cm} \mathrm{rad/s},
\ee
som innsatt i uttrykket for treghetsmomentet gir oss at 
$I_{\mathrm{cm}}=1.46\times 10^{-46}$ kgm$^2$.

Vi kan s\aa\ bruke treghetsmomentet til \aa\ rekne ut 
likevektsposisjonen $r_0$. Men da trenger vi verdien for den reduserte
massen $\mu$, siden 
\be
    \mu r_0^2=I_{\mathrm{cm}}.
\ee

Karbonatomet har atom\ae r masse 12 u mens oksygenatomet har masse
16 u. Da finner vi
\be
    \mu =\frac{(12u)(16u)}{12u+16u}=1.14\times 10^{-26}
    \hspace{0.1cm}\mathrm{kg}, 
\ee
som gir
\be
   r_0=\sqrt{\frac{I_{\mathrm{cm}}}{\mu}}=1.13\times 10^{-10}
    \hspace{0.1cm}\mathrm{m},
\ee
eller 0.113 nm. Dette eksemplet viser hvordan spektroskopiske
data kan brukes til \aa\ bestemme egenskaper ved molekyler.

Hva vibrasjon ang\aa r, s\aa\ er frekvensen for en overgang
fra $n=0$ til $n=1$ for CO molekylet gitt ved $6.42 \times 10^{13}$ Hz.
Denne frekvensen er to st\o rrelsesordener st\o rre enn den for 
rotasjonsovergangen. Det vil si at energien for vibrasjonsovergangen
er mye st\o rre enn den for en rein rotasjonsovergang. 

Vi kan bruke denne eksperimentelle verdien til \aa\ rekne fj\ae rkonstanten
$k$ da 
\be
  \omega^2\mu=k.
\ee
Vinkelfrekvensen er $\omega= 2\pi \nu=4.03\times 10^{11}$ rad/s og bruker vi 
verdien vi fant for den reduserte masse $\mu$ har vi
\be
   k=1.86\times 10^3 \hspace{0.1cm}\mathrm{N/m}.
\ee


\section{Ionisk og kovalent binding}

I dette avsnittet skal vi se p\aa\ forskjellene mellom
ionisk og kovalent binding vha.~noen utvalgte eksempler.

\subsection{Ionisk binding}
Hva ionisk binding ang\aa r, kan vi i all hovedsak si at bindingen
skyldes tiltrekningen mellom ioner av ulik ladning, selv om vi ogs\aa\
har et visst innslag av kovalent binding, dvs.~at ett eller flere elektroner
deles mellom de involverte atomene. Vi skal til slutt i dette avsnittet
vise et enkelt m\aa l for graden av ionisk og kovalent binding.

Som eksempel skal vi bruke kaliumklorid, KCl. 
For at dette molekylet skal v\ae re stabilt, m\aa\ dets bindingsenergi
v\ae re st\o rre (i absoluttverdi) enn bindingsenergien til de adskilte
kalium og klor atomene, dvs.
\be
    E(KCl) < E(K)+E(Cl).
\ee
Kalium har i grunntilstanden 
den elektroniske konfigurasjonen $1s^22s^22p^63s^23p^64s^1$
mens klor har konfigurasjonen  $1s^22s^22p^63s^23p^5$.

For at vi skal f\aa\ et kalium ion, m\aa\ vi tilf\o re atomet energi
for \aa\ l\o srive det svakest bundne elektronet. Denne energien svarer til
ionisasjonsenergien
for kalium, som  er $4.34$ eV. Vi f\aa r da et kalium ion K$^+$, med
elektron konfigurasjon $1s^22s^22p^63s^23p^6$, dvs.~at vi har fylt alle
skall opp til og med $3p$. Et kloratom kan tiltrekke seg dette l\o srevne
elektronet. I denne prosessen frigj\o res det energi pga.~elektronaffiniteten
til kloratomet. Energien som frigj\o res er $3.62$ eV, slik at vi sitter 
igjen med et energioverskudd p\aa\ $0.72$ eV, og klorionets konfigurasjon er
den samme som kaliumionet. 
Denne energien er positiv i en skala hvor vi har satt energien til kalium
og klor atomene til null n\aa r de er langt fra hverandre. 
Vi kaller dette energioverskuddet for $E_{\mathrm{ion}}$.
For at vi skal f\aa\ binding m\aa\ vi legge til tiltrekningen
som ionene f\o ler gitt ved 
\[
   -\frac{ke^2}{r},
\]
hvor  $r$ er avstanden mellom ionene. Men dersom vi bare har et tiltrekkende
ledd, vil ikke v\ae re i stand til \aa\ finne et energiminimum og en tilsvarende
avstand $r_0$ mellom atomene. N\aa r avstanden mellom 
ionene minsker, vil de svakest bundne elektronene i $3p$ skallene
i klor og kalium begynne \aa\ overlappe hverandre. Siden to elektroner
ikke kan ha samme kvantetall pga.~Paulis 
eksklusjonsprinsipp f\aa r vi et
frast\o tende ledd til bindingsenergien for molekylet.

Vi kan dermed modellere den totale bindingsenergien for et  
molekyl som best\aa r av to ioner som
\be
  E(r)=-\frac{ke^2}{r}+E_{\mathrm{ion}}+E_{\mathrm{Pauli}},
\ee
hvor det siste leddet kan skrives som
\be
E_{\mathrm{Pauli}}=\frac{A}{r^n},
\ee
hvor $A$ er en konstant og $n$ et heltall som kan bestemmes utifra
eksperimentelle data, se eksemplet nedenfor.

KCl molekylet har en bindingsenergi p\aa\ 4.4 eV og $r_0=0.27$ nm, mens
koksalt har en bindingsenergi p\aa\ 4.26 eV og $r_0=0.24$ nm.
Kan du, utifra elektronkonfigurasjonene til kalium, natrium og klor
ionene forklare hvorfor $r_0$ og bindingsenergien til KCl er st\o rre
enn de tilsvarende for NaCl?

\subsubsection{Parametrisering av bindingsenergien til NaF}

Ved hjelp av eksperimentelle data som ionisasjonsenergien 
til Na, elektronaffiniteten til fluor, $r_0$ og bindingsenergien
til molekylet, kan vi lage oss en enkel parametrisering av
bindingsenergien $E(r)$. 

Ionisasjonsenergien for Na er 5.14 eV mens elektronaffiniteten til fluor
er 3.40 eV. Det gir oss $E_{\mathrm{ion}}=1.74$ eV. Bindingsenergien
er 4.99 eV mens $r_0=0.193$ nm. 

Der energien har sitt minimum finner vi at 
\[
   -\frac{ke^2}{r_0}=-\frac{1.44\hspace{0.1cm}\mathrm{eVnm}}
          {0.193\hspace{0.1cm}\mathrm{nm}}=-7.46\hspace{0.1cm}\mathrm{eV},
\]
som gir
\be
  E_{\mathrm{Pauli}}=E(r_0)+\frac{ke^2}{r_0}-E_{\mathrm{ion}}=0.72
                            \hspace{0.1cm}\mathrm{eV},
\ee
eller
\be
E_{\mathrm{Pauli}}=\frac{A}{r^n}=0.72
                            \hspace{0.1cm}\mathrm{eV}.
\ee

Vi kan deretter bestemme konstantene $n$ og $A$. 
Ved $r=r_0$ m\aa\ den deriverte av energien v\ae re null. Det betyr
at kreftene satt opp av Coulombvekselvirkningen mellom ionene og 
Pauliprinsippet balanserer hverandre. 
Vi f\aa r dermed
\be
   \frac{dE(r)}{dr}\left|_{r=r_0}\right.=0,
\ee
eller
\be
   \frac{d(-\frac{ke^2}{r})}{dr}\left|_{r=r_0}\right.=
   \frac{d(-\frac{A}{r^n})}{dr}\left|_{r=r_0}\right. ,
\ee
eller
\be 
   \frac{ke^2}{r_0^2}=\frac{nA}{r_0^{n+1}}.
\ee
Setter vi inn $A/r_0^n=0.72$ eV og $r_0=0.193$ nm, finner vi at
$n=10.4\approx 10$ og $A=9\times 10^{-8}$ eVnm$^{10}$, dvs.~
\[
   E_{\mathrm{Pauli}}=\frac{9\times 10^{-8}
   \hspace{0.1cm}\mathrm{eVnm}^{10}}{r^{10}}.
\]
  
\subsection{H$_2^+$ molekylet, kovalent binding og tunneling}

I dette avsnittet skal vi i all hovedsak konsentrere
oss om H$_2^+$ molekylet. Framstillinga her g\aa r utover det som
er diskutert i tekstboka, avsnitt 10-1, men er tatt med for \aa\ vise
at man faktisk kan lage brukbare estimater om bindingsenergien
vha.~enkle approksimasjoner. 

H$_2^+$ molekylet best\aa r av to protoner og et elektron
med bindingsenergi  $E_B=-2.8$ eV
og likevektsposisjon $r_0=0.106$ nm.


Vi definerer systemet v\aa rt, to protoner og et elektron
vha.~f\o lgende variable, se Figur 10-1 i boka. 
Elektronet  plasseres i en avstand
${\bf r}$ fra et valgt origo, det ene protonet i avstanden $-{\bf R}/2$ 
og det andre i avstanden ${\bf R}/2$, slik at avstanden til elektronet
blir henholdsvis ${\bf r}- {\bf R}/2$ og ${\bf r}+ {\bf R}/2$ slik som vist 
i Figur 10-1 i boka. 

N\aa r vi skal sette opp Schr\"odingers likning for dette systemet 
skal vi se bort ifra bidraget fra protonene til den kinetiske energien,
da massen til protonene er ca.~2000 ganger st\o rre enn elektronets masse.
Vi kan dermed anta at hastigheten til protonene er veldig liten i forhold
til elektronets og betrakter protonene, slik vi ogs\aa\ gjorde
med hydrogenatomet, som om de er i ro i forhold til elektronet.

Det betyr at vi kan skrive Schr\"odingers likning p\aa\ f\o lgende form
\be
    \left\{-\frac{\hbar^2\nabla_r^2}{2m_e}
     -\frac{ke^2}{|{\bf r}- {\bf R}/2|}-\frac{ke^2}{|{\bf r}+ {\bf R}/2|}
     +\frac{ke^2}{R}
     \right\}\psi({\bf r},{\bf R})=E\psi({\bf r},{\bf R}),
\ee
hvor det f\o rste leddet er den kinetiske energien til
elektronet, det andre leddet er den potensielle energien elektronet 
f\o ler i forhold til protonet i $-{\bf R}/2$  mens det tredje leddet
er den tilsvarende potensielle energien fra protonet i 
${\bf R}/2$. Det fjerde og siste leddet p\aa\ venstre side skyldes
frast\o tingen mellom de to protonene. Vi har alst\aa\ neglisjert 
protonenes kinetiske energi. 
I figur \ref{62} har vi plotta den potensielle energien
\be  V({\bf r}, {\bf R})=
     -\frac{ke^2}{|{\bf r}- {\bf R}/2|}-\frac{ke^2}{|{\bf r}+ {\bf R}/2|}
     +\frac{ke^2}{R}.
\ee
I denne figuren har vi fiksert verdien av 
$|{\bf R}|=2a_0$ og $|{\bf R}|=8a_0$, 
som er henholdsvis 2 og 8 Bohrradier. Legg merke til at i 
omr\aa det mellom $|{\bf r}|=-|{\bf R}|/2$ 
( i figuren er enhetene gitt ved $r/a_0$, med
$a_0=0.0529$) og $|{\bf r}|=|{\bf R}|/2$  
kan elektronet tunnelere gjennom 
potensialbarrieren. 
Husk at  $-{\bf R}/2$ og ${\bf R}/2$
svarer til posisjonene til de to protonene i molekylet.
Vi legger ogs\aa\ merke til at ettersom $R$ \o ker s\aa\ blir potensialet
svakere. Dette f\aa r konsekvenser for bindingsenergien til 
molekylet, da den blir mindre n\aa r avstanden mellom atomene
\o ker. 
Potensialbarrieren som oppst\aa r 
betyr at elektronet ikke vil v\ae re ved det ene
protonet hele tida.
\begin{figure}[h]
\begin{center}
% GNUPLOT: LaTeX picture with Postscript
\begingroup%
  \makeatletter%
  \newcommand{\GNUPLOTspecial}{%
    \@sanitize\catcode`\%=14\relax\special}%
  \setlength{\unitlength}{0.1bp}%
{\GNUPLOTspecial{!
%!PS-Adobe-2.0 EPSF-2.0
%%Title: coulomb.tex
%%Creator: gnuplot 3.7 patchlevel 0.2
%%CreationDate: Fri Apr 28 12:34:58 2000
%%DocumentFonts: 
%%BoundingBox: 0 0 360 216
%%Orientation: Landscape
%%EndComments
/gnudict 256 dict def
gnudict begin
/Color false def
/Solid false def
/gnulinewidth 5.000 def
/userlinewidth gnulinewidth def
/vshift -33 def
/dl {10 mul} def
/hpt_ 31.5 def
/vpt_ 31.5 def
/hpt hpt_ def
/vpt vpt_ def
/M {moveto} bind def
/L {lineto} bind def
/R {rmoveto} bind def
/V {rlineto} bind def
/vpt2 vpt 2 mul def
/hpt2 hpt 2 mul def
/Lshow { currentpoint stroke M
  0 vshift R show } def
/Rshow { currentpoint stroke M
  dup stringwidth pop neg vshift R show } def
/Cshow { currentpoint stroke M
  dup stringwidth pop -2 div vshift R show } def
/UP { dup vpt_ mul /vpt exch def hpt_ mul /hpt exch def
  /hpt2 hpt 2 mul def /vpt2 vpt 2 mul def } def
/DL { Color {setrgbcolor Solid {pop []} if 0 setdash }
 {pop pop pop Solid {pop []} if 0 setdash} ifelse } def
/BL { stroke userlinewidth 2 mul setlinewidth } def
/AL { stroke userlinewidth 2 div setlinewidth } def
/UL { dup gnulinewidth mul /userlinewidth exch def
      10 mul /udl exch def } def
/PL { stroke userlinewidth setlinewidth } def
/LTb { BL [] 0 0 0 DL } def
/LTa { AL [1 udl mul 2 udl mul] 0 setdash 0 0 0 setrgbcolor } def
/LT0 { PL [] 1 0 0 DL } def
/LT1 { PL [4 dl 2 dl] 0 1 0 DL } def
/LT2 { PL [2 dl 3 dl] 0 0 1 DL } def
/LT3 { PL [1 dl 1.5 dl] 1 0 1 DL } def
/LT4 { PL [5 dl 2 dl 1 dl 2 dl] 0 1 1 DL } def
/LT5 { PL [4 dl 3 dl 1 dl 3 dl] 1 1 0 DL } def
/LT6 { PL [2 dl 2 dl 2 dl 4 dl] 0 0 0 DL } def
/LT7 { PL [2 dl 2 dl 2 dl 2 dl 2 dl 4 dl] 1 0.3 0 DL } def
/LT8 { PL [2 dl 2 dl 2 dl 2 dl 2 dl 2 dl 2 dl 4 dl] 0.5 0.5 0.5 DL } def
/Pnt { stroke [] 0 setdash
   gsave 1 setlinecap M 0 0 V stroke grestore } def
/Dia { stroke [] 0 setdash 2 copy vpt add M
  hpt neg vpt neg V hpt vpt neg V
  hpt vpt V hpt neg vpt V closepath stroke
  Pnt } def
/Pls { stroke [] 0 setdash vpt sub M 0 vpt2 V
  currentpoint stroke M
  hpt neg vpt neg R hpt2 0 V stroke
  } def
/Box { stroke [] 0 setdash 2 copy exch hpt sub exch vpt add M
  0 vpt2 neg V hpt2 0 V 0 vpt2 V
  hpt2 neg 0 V closepath stroke
  Pnt } def
/Crs { stroke [] 0 setdash exch hpt sub exch vpt add M
  hpt2 vpt2 neg V currentpoint stroke M
  hpt2 neg 0 R hpt2 vpt2 V stroke } def
/TriU { stroke [] 0 setdash 2 copy vpt 1.12 mul add M
  hpt neg vpt -1.62 mul V
  hpt 2 mul 0 V
  hpt neg vpt 1.62 mul V closepath stroke
  Pnt  } def
/Star { 2 copy Pls Crs } def
/BoxF { stroke [] 0 setdash exch hpt sub exch vpt add M
  0 vpt2 neg V  hpt2 0 V  0 vpt2 V
  hpt2 neg 0 V  closepath fill } def
/TriUF { stroke [] 0 setdash vpt 1.12 mul add M
  hpt neg vpt -1.62 mul V
  hpt 2 mul 0 V
  hpt neg vpt 1.62 mul V closepath fill } def
/TriD { stroke [] 0 setdash 2 copy vpt 1.12 mul sub M
  hpt neg vpt 1.62 mul V
  hpt 2 mul 0 V
  hpt neg vpt -1.62 mul V closepath stroke
  Pnt  } def
/TriDF { stroke [] 0 setdash vpt 1.12 mul sub M
  hpt neg vpt 1.62 mul V
  hpt 2 mul 0 V
  hpt neg vpt -1.62 mul V closepath fill} def
/DiaF { stroke [] 0 setdash vpt add M
  hpt neg vpt neg V hpt vpt neg V
  hpt vpt V hpt neg vpt V closepath fill } def
/Pent { stroke [] 0 setdash 2 copy gsave
  translate 0 hpt M 4 {72 rotate 0 hpt L} repeat
  closepath stroke grestore Pnt } def
/PentF { stroke [] 0 setdash gsave
  translate 0 hpt M 4 {72 rotate 0 hpt L} repeat
  closepath fill grestore } def
/Circle { stroke [] 0 setdash 2 copy
  hpt 0 360 arc stroke Pnt } def
/CircleF { stroke [] 0 setdash hpt 0 360 arc fill } def
/C0 { BL [] 0 setdash 2 copy moveto vpt 90 450  arc } bind def
/C1 { BL [] 0 setdash 2 copy        moveto
       2 copy  vpt 0 90 arc closepath fill
               vpt 0 360 arc closepath } bind def
/C2 { BL [] 0 setdash 2 copy moveto
       2 copy  vpt 90 180 arc closepath fill
               vpt 0 360 arc closepath } bind def
/C3 { BL [] 0 setdash 2 copy moveto
       2 copy  vpt 0 180 arc closepath fill
               vpt 0 360 arc closepath } bind def
/C4 { BL [] 0 setdash 2 copy moveto
       2 copy  vpt 180 270 arc closepath fill
               vpt 0 360 arc closepath } bind def
/C5 { BL [] 0 setdash 2 copy moveto
       2 copy  vpt 0 90 arc
       2 copy moveto
       2 copy  vpt 180 270 arc closepath fill
               vpt 0 360 arc } bind def
/C6 { BL [] 0 setdash 2 copy moveto
      2 copy  vpt 90 270 arc closepath fill
              vpt 0 360 arc closepath } bind def
/C7 { BL [] 0 setdash 2 copy moveto
      2 copy  vpt 0 270 arc closepath fill
              vpt 0 360 arc closepath } bind def
/C8 { BL [] 0 setdash 2 copy moveto
      2 copy vpt 270 360 arc closepath fill
              vpt 0 360 arc closepath } bind def
/C9 { BL [] 0 setdash 2 copy moveto
      2 copy  vpt 270 450 arc closepath fill
              vpt 0 360 arc closepath } bind def
/C10 { BL [] 0 setdash 2 copy 2 copy moveto vpt 270 360 arc closepath fill
       2 copy moveto
       2 copy vpt 90 180 arc closepath fill
               vpt 0 360 arc closepath } bind def
/C11 { BL [] 0 setdash 2 copy moveto
       2 copy  vpt 0 180 arc closepath fill
       2 copy moveto
       2 copy  vpt 270 360 arc closepath fill
               vpt 0 360 arc closepath } bind def
/C12 { BL [] 0 setdash 2 copy moveto
       2 copy  vpt 180 360 arc closepath fill
               vpt 0 360 arc closepath } bind def
/C13 { BL [] 0 setdash  2 copy moveto
       2 copy  vpt 0 90 arc closepath fill
       2 copy moveto
       2 copy  vpt 180 360 arc closepath fill
               vpt 0 360 arc closepath } bind def
/C14 { BL [] 0 setdash 2 copy moveto
       2 copy  vpt 90 360 arc closepath fill
               vpt 0 360 arc } bind def
/C15 { BL [] 0 setdash 2 copy vpt 0 360 arc closepath fill
               vpt 0 360 arc closepath } bind def
/Rec   { newpath 4 2 roll moveto 1 index 0 rlineto 0 exch rlineto
       neg 0 rlineto closepath } bind def
/Square { dup Rec } bind def
/Bsquare { vpt sub exch vpt sub exch vpt2 Square } bind def
/S0 { BL [] 0 setdash 2 copy moveto 0 vpt rlineto BL Bsquare } bind def
/S1 { BL [] 0 setdash 2 copy vpt Square fill Bsquare } bind def
/S2 { BL [] 0 setdash 2 copy exch vpt sub exch vpt Square fill Bsquare } bind def
/S3 { BL [] 0 setdash 2 copy exch vpt sub exch vpt2 vpt Rec fill Bsquare } bind def
/S4 { BL [] 0 setdash 2 copy exch vpt sub exch vpt sub vpt Square fill Bsquare } bind def
/S5 { BL [] 0 setdash 2 copy 2 copy vpt Square fill
       exch vpt sub exch vpt sub vpt Square fill Bsquare } bind def
/S6 { BL [] 0 setdash 2 copy exch vpt sub exch vpt sub vpt vpt2 Rec fill Bsquare } bind def
/S7 { BL [] 0 setdash 2 copy exch vpt sub exch vpt sub vpt vpt2 Rec fill
       2 copy vpt Square fill
       Bsquare } bind def
/S8 { BL [] 0 setdash 2 copy vpt sub vpt Square fill Bsquare } bind def
/S9 { BL [] 0 setdash 2 copy vpt sub vpt vpt2 Rec fill Bsquare } bind def
/S10 { BL [] 0 setdash 2 copy vpt sub vpt Square fill 2 copy exch vpt sub exch vpt Square fill
       Bsquare } bind def
/S11 { BL [] 0 setdash 2 copy vpt sub vpt Square fill 2 copy exch vpt sub exch vpt2 vpt Rec fill
       Bsquare } bind def
/S12 { BL [] 0 setdash 2 copy exch vpt sub exch vpt sub vpt2 vpt Rec fill Bsquare } bind def
/S13 { BL [] 0 setdash 2 copy exch vpt sub exch vpt sub vpt2 vpt Rec fill
       2 copy vpt Square fill Bsquare } bind def
/S14 { BL [] 0 setdash 2 copy exch vpt sub exch vpt sub vpt2 vpt Rec fill
       2 copy exch vpt sub exch vpt Square fill Bsquare } bind def
/S15 { BL [] 0 setdash 2 copy Bsquare fill Bsquare } bind def
/D0 { gsave translate 45 rotate 0 0 S0 stroke grestore } bind def
/D1 { gsave translate 45 rotate 0 0 S1 stroke grestore } bind def
/D2 { gsave translate 45 rotate 0 0 S2 stroke grestore } bind def
/D3 { gsave translate 45 rotate 0 0 S3 stroke grestore } bind def
/D4 { gsave translate 45 rotate 0 0 S4 stroke grestore } bind def
/D5 { gsave translate 45 rotate 0 0 S5 stroke grestore } bind def
/D6 { gsave translate 45 rotate 0 0 S6 stroke grestore } bind def
/D7 { gsave translate 45 rotate 0 0 S7 stroke grestore } bind def
/D8 { gsave translate 45 rotate 0 0 S8 stroke grestore } bind def
/D9 { gsave translate 45 rotate 0 0 S9 stroke grestore } bind def
/D10 { gsave translate 45 rotate 0 0 S10 stroke grestore } bind def
/D11 { gsave translate 45 rotate 0 0 S11 stroke grestore } bind def
/D12 { gsave translate 45 rotate 0 0 S12 stroke grestore } bind def
/D13 { gsave translate 45 rotate 0 0 S13 stroke grestore } bind def
/D14 { gsave translate 45 rotate 0 0 S14 stroke grestore } bind def
/D15 { gsave translate 45 rotate 0 0 S15 stroke grestore } bind def
/DiaE { stroke [] 0 setdash vpt add M
  hpt neg vpt neg V hpt vpt neg V
  hpt vpt V hpt neg vpt V closepath stroke } def
/BoxE { stroke [] 0 setdash exch hpt sub exch vpt add M
  0 vpt2 neg V hpt2 0 V 0 vpt2 V
  hpt2 neg 0 V closepath stroke } def
/TriUE { stroke [] 0 setdash vpt 1.12 mul add M
  hpt neg vpt -1.62 mul V
  hpt 2 mul 0 V
  hpt neg vpt 1.62 mul V closepath stroke } def
/TriDE { stroke [] 0 setdash vpt 1.12 mul sub M
  hpt neg vpt 1.62 mul V
  hpt 2 mul 0 V
  hpt neg vpt -1.62 mul V closepath stroke } def
/PentE { stroke [] 0 setdash gsave
  translate 0 hpt M 4 {72 rotate 0 hpt L} repeat
  closepath stroke grestore } def
/CircE { stroke [] 0 setdash 
  hpt 0 360 arc stroke } def
/Opaque { gsave closepath 1 setgray fill grestore 0 setgray closepath } def
/DiaW { stroke [] 0 setdash vpt add M
  hpt neg vpt neg V hpt vpt neg V
  hpt vpt V hpt neg vpt V Opaque stroke } def
/BoxW { stroke [] 0 setdash exch hpt sub exch vpt add M
  0 vpt2 neg V hpt2 0 V 0 vpt2 V
  hpt2 neg 0 V Opaque stroke } def
/TriUW { stroke [] 0 setdash vpt 1.12 mul add M
  hpt neg vpt -1.62 mul V
  hpt 2 mul 0 V
  hpt neg vpt 1.62 mul V Opaque stroke } def
/TriDW { stroke [] 0 setdash vpt 1.12 mul sub M
  hpt neg vpt 1.62 mul V
  hpt 2 mul 0 V
  hpt neg vpt -1.62 mul V Opaque stroke } def
/PentW { stroke [] 0 setdash gsave
  translate 0 hpt M 4 {72 rotate 0 hpt L} repeat
  Opaque stroke grestore } def
/CircW { stroke [] 0 setdash 
  hpt 0 360 arc Opaque stroke } def
/BoxFill { gsave Rec 1 setgray fill grestore } def
end
%%EndProlog
}}%
\begin{picture}(3600,2160)(0,0)%
{\GNUPLOTspecial{"
gnudict begin
gsave
0 0 translate
0.100 0.100 scale
0 setgray
newpath
1.000 UL
LTb
400 300 M
63 0 V
2987 0 R
-63 0 V
400 593 M
63 0 V
2987 0 R
-63 0 V
400 887 M
63 0 V
2987 0 R
-63 0 V
400 1180 M
63 0 V
2987 0 R
-63 0 V
400 1473 M
63 0 V
2987 0 R
-63 0 V
400 1767 M
63 0 V
2987 0 R
-63 0 V
400 2060 M
63 0 V
2987 0 R
-63 0 V
400 300 M
0 63 V
0 1697 R
0 -63 V
781 300 M
0 63 V
0 1697 R
0 -63 V
1163 300 M
0 63 V
0 1697 R
0 -63 V
1544 300 M
0 63 V
0 1697 R
0 -63 V
1925 300 M
0 63 V
0 1697 R
0 -63 V
2306 300 M
0 63 V
0 1697 R
0 -63 V
2688 300 M
0 63 V
0 1697 R
0 -63 V
3069 300 M
0 63 V
0 1697 R
0 -63 V
3450 300 M
0 63 V
0 1697 R
0 -63 V
1.000 UL
LTb
400 300 M
3050 0 V
0 1760 V
-3050 0 V
400 300 L
1.000 UL
LT0
3087 1947 M
263 0 V
1117 2060 M
22 -13 V
31 -20 V
31 -23 V
31 -24 V
31 -28 V
30 -31 V
31 -35 V
31 -40 V
31 -46 V
31 -55 V
30 -65 V
31 -79 V
31 -99 V
31 -127 V
31 -172 V
31 -246 V
30 -386 V
12 -271 V
193 0 R
11 163 V
31 224 V
31 95 V
30 0 V
31 -95 V
31 -224 V
11 -163 V
193 0 R
12 271 V
30 386 V
31 246 V
31 172 V
31 127 V
31 99 V
31 79 V
30 65 V
31 55 V
31 46 V
31 40 V
31 35 V
30 31 V
31 28 V
31 24 V
31 23 V
31 20 V
22 13 V
1.000 UL
LT1
3087 1847 M
263 0 V
400 1661 M
31 0 V
31 0 V
30 0 V
31 0 V
31 0 V
31 0 V
31 0 V
30 0 V
31 0 V
31 0 V
31 0 V
31 0 V
31 0 V
30 0 V
31 0 V
31 0 V
31 0 V
31 0 V
30 0 V
31 0 V
31 0 V
31 0 V
31 0 V
30 0 V
31 0 V
31 0 V
31 0 V
31 0 V
30 0 V
31 0 V
31 0 V
31 0 V
31 0 V
30 0 V
31 0 V
31 0 V
31 0 V
31 0 V
31 0 V
30 0 V
31 0 V
31 0 V
31 0 V
31 0 V
30 0 V
31 0 V
31 0 V
31 0 V
31 0 V
30 0 V
31 0 V
31 0 V
31 0 V
31 0 V
30 0 V
31 0 V
31 0 V
31 0 V
31 0 V
30 0 V
31 0 V
31 0 V
31 0 V
31 0 V
31 0 V
30 0 V
31 0 V
31 0 V
31 0 V
31 0 V
30 0 V
31 0 V
31 0 V
31 0 V
31 0 V
30 0 V
31 0 V
31 0 V
31 0 V
31 0 V
30 0 V
31 0 V
31 0 V
31 0 V
31 0 V
30 0 V
31 0 V
31 0 V
31 0 V
31 0 V
31 0 V
30 0 V
31 0 V
31 0 V
31 0 V
31 0 V
30 0 V
31 0 V
31 0 V
1.000 UL
LT2
3087 1747 M
263 0 V
400 1884 M
31 -10 V
31 -11 V
30 -11 V
31 -13 V
31 -14 V
31 -15 V
31 -17 V
30 -18 V
31 -21 V
31 -24 V
31 -27 V
31 -31 V
31 -37 V
30 -42 V
31 -52 V
31 -63 V
31 -78 V
31 -102 V
30 -136 V
31 -193 V
31 -295 V
23 -374 V
186 0 R
7 144 V
30 376 V
31 232 V
31 156 V
31 113 V
31 84 V
30 66 V
31 51 V
31 42 V
31 35 V
31 28 V
31 23 V
30 20 V
31 17 V
31 13 V
31 12 V
31 9 V
30 8 V
31 6 V
31 4 V
31 3 V
31 1 V
30 0 V
31 -1 V
31 -3 V
31 -4 V
31 -6 V
30 -8 V
31 -9 V
31 -12 V
31 -13 V
31 -17 V
30 -20 V
31 -23 V
31 -28 V
31 -35 V
31 -42 V
31 -51 V
30 -66 V
31 -84 V
31 -113 V
31 -156 V
31 -232 V
30 -376 V
7 -144 V
186 0 R
23 374 V
31 295 V
31 193 V
30 136 V
31 102 V
31 78 V
31 63 V
31 52 V
30 42 V
31 37 V
31 31 V
31 27 V
31 24 V
31 21 V
30 18 V
31 17 V
31 15 V
31 14 V
31 13 V
30 11 V
31 11 V
31 10 V
stroke
grestore
end
showpage
}}%
\put(3037,1747){\makebox(0,0)[r]{$R=0.4$ nm}}%
\put(3037,1847){\makebox(0,0)[r]{$\varepsilon=-13.6$ eV}}%
\put(3037,1947){\makebox(0,0)[r]{$R=0.1$ nm}}%
\put(1925,50){\makebox(0,0){$r/a_0$}}%
\put(100,1180){%
\special{ps: gsave currentpoint currentpoint translate
270 rotate neg exch neg exch translate}%
\makebox(0,0)[b]{\shortstack{$V({\bf r},{\bf R})$ [eV]}}%
\special{ps: currentpoint grestore moveto}%
}%
\put(3450,200){\makebox(0,0){8}}%
\put(3069,200){\makebox(0,0){6}}%
\put(2688,200){\makebox(0,0){4}}%
\put(2306,200){\makebox(0,0){2}}%
\put(1925,200){\makebox(0,0){0}}%
\put(1544,200){\makebox(0,0){-2}}%
\put(1163,200){\makebox(0,0){-4}}%
\put(781,200){\makebox(0,0){-6}}%
\put(400,200){\makebox(0,0){-8}}%
\put(350,2060){\makebox(0,0)[r]{0}}%
\put(350,1767){\makebox(0,0)[r]{-10}}%
\put(350,1473){\makebox(0,0)[r]{-20}}%
\put(350,1180){\makebox(0,0)[r]{-30}}%
\put(350,887){\makebox(0,0)[r]{-40}}%
\put(350,593){\makebox(0,0)[r]{-50}}%
\put(350,300){\makebox(0,0)[r]{-60}}%
\end{picture}%
\endgroup
\endinput

\end{center}
\caption{Plot av  $V(r,R)$ for
$|{\bf R}|$=0.1 og 0.4 nm. Enhet langs $x$-aksen er $r/a_0$ .
Legg merke til at vi ogs\aa\ har plottet
energien (den rette linja) som svarer til bindingsenergien for
hydrogenatomet, 
dvs.~$\varepsilon=-13.6$ eV. \label{62} }
\end{figure} 
Siden potensialet v\aa rt er symmetrisk med tanke p\aa\
ombytte av 
${\bf R}\rightarrow -{\bf R}$
og ${\bf r}\rightarrow -{\bf r}$
betyr det at sannsyligheten for \aa\ bevege seg fra et proton
til et annet m\aa\ v\ae re lik i begge retninger. 
Vi kan da tenke oss at elektronet deler sin tid mellom
de to protonene.

Med dette i baktanke skal vi lage oss en 
modell for beskrivelse av molekylet.
Siden vi har kun et elektron kan vi forestille oss at i grensa $R\rightarrow
\infty$, dvs.~stor avstand mellom atomene (som i v\aa rt tilfelle
bare er to protoner) s\aa\ er elektronet bare bundet til et 
av protonene, og vi har et hydrogenatom. 
Som ansats for elektronets b\o lgefunksjon skal vi derfor ta utgangspunkt
i b\o lgefunksjonen for elektronet i grunntilstanden for hydrogenatomet, nemlig
\be
    \psi_{100}(r)=\left(\frac{1}{\pi a_0^3}\right)^{1/2} e^{-r/a_0},
\ee
som vi har diskutert i kapittel 6. 

Vi tenker oss deretter at elektronet kan koples enten til 
det ene protonet eller det andre. 
Vi definerer derfor to
hydrogenb\o lgefunksjoner
\be
   \psi_1({\bf r},{\bf R})=\left(\frac{1}{\pi a_0^3}\right)^{1/2} e^{-|{\bf r}- {\bf R}/2|/a_0},
\ee
og 
\be
   \psi_2({\bf r},{\bf R})=\left(\frac{1}{\pi a_0^3}\right)^{1/2} e^{-|{\bf r}+ {\bf R}/2|/a_0}.
\ee

Basert p\aa\ disse to b\o lgefunksjonene, som representerer hvor elektronet
kan v\ae re, {\bf lager vi oss en ansats for b\o lgefunskjonen for molekylet ved
\aa\ pr\o ve med f\o lgende form}
\be
   \psi_+({\bf r},{\bf R})=C_+\left(\psi_1({\bf r},{\bf R})+\psi_2({\bf r},{\bf R})\right).
\ee
Konstanten $C_+$ er en normeringskonstant.
Denne ansatsen baseres seg p\aa\ diskusjonen v\aa r ovenfor
om potensialets form og at elektronet har like stor sannsynlighet
for \aa\ v\ae re n\ae r det ene eller andre protonet. 
I siste likning valgte vi en symmetrisk form for b\o lgefunksjonen, men det er
ikke noe som forbyr oss \aa\ velge en antisymmetrisk b\o lgefunksjon
\be
   \psi_-({\bf r},{\bf R})=C_-\left(\psi_1({\bf r},{\bf R})-\psi_2({\bf r},{\bf R})\right).
\ee
Det vi har gjort er \aa\ bruke hydrogenatomet som modell og korrigere
for tilstedev\ae relsen av et tillegsproton. Som vi skal se nedenfor
gir denne ansatsen et kvalitativt riktig bilde av molekylet.
Det vi ogs\aa\ skal vise er at den symmetriske ansatsen gir en bundet tilstand,
mens den antisymmetriske gir en ubunden tilstand.

Men f\o r vi g\aa r videre kan det v\ae re nyttig \aa\ studere
disse to b\o lgefunskjonene for ulike verdier av $R=|{\bf R}|$.
Figur \ref{63} viser $\psi_+({\bf r},{\bf R})$ for ulike verdier av
$|{\bf R}|$. Vi legger merke til at n\aa r $R$ \o ker, dvs.~ at protonene
er lengre i fra hverandre, s\aa\ begynner de to hydrogen b\o lgefunksjonene
\aa\ skille lag. 
I diskusjonen v\aa r er det sv\ae rt nyttig \aa\ 
se p\aa\ sannsynlighetstettheten gitt ved
\be
   \psi_{\pm}^*\psi_{\pm}=\psi_1^*\psi_1+\psi_2^*\psi_2\pm 2\psi_2^*\psi_1.
\ee
\begin{figure}
\begin{center}
% GNUPLOT: LaTeX picture with Postscript
\begingroup%
  \makeatletter%
  \newcommand{\GNUPLOTspecial}{%
    \@sanitize\catcode`\%=14\relax\special}%
  \setlength{\unitlength}{0.1bp}%
{\GNUPLOTspecial{!
%!PS-Adobe-2.0 EPSF-2.0
%%Title: h2pluss.tex
%%Creator: gnuplot 3.7 patchlevel 0.2
%%CreationDate: Wed Apr 26 15:09:14 2000
%%DocumentFonts: 
%%BoundingBox: 0 0 360 216
%%Orientation: Landscape
%%EndComments
/gnudict 256 dict def
gnudict begin
/Color false def
/Solid false def
/gnulinewidth 5.000 def
/userlinewidth gnulinewidth def
/vshift -33 def
/dl {10 mul} def
/hpt_ 31.5 def
/vpt_ 31.5 def
/hpt hpt_ def
/vpt vpt_ def
/M {moveto} bind def
/L {lineto} bind def
/R {rmoveto} bind def
/V {rlineto} bind def
/vpt2 vpt 2 mul def
/hpt2 hpt 2 mul def
/Lshow { currentpoint stroke M
  0 vshift R show } def
/Rshow { currentpoint stroke M
  dup stringwidth pop neg vshift R show } def
/Cshow { currentpoint stroke M
  dup stringwidth pop -2 div vshift R show } def
/UP { dup vpt_ mul /vpt exch def hpt_ mul /hpt exch def
  /hpt2 hpt 2 mul def /vpt2 vpt 2 mul def } def
/DL { Color {setrgbcolor Solid {pop []} if 0 setdash }
 {pop pop pop Solid {pop []} if 0 setdash} ifelse } def
/BL { stroke userlinewidth 2 mul setlinewidth } def
/AL { stroke userlinewidth 2 div setlinewidth } def
/UL { dup gnulinewidth mul /userlinewidth exch def
      10 mul /udl exch def } def
/PL { stroke userlinewidth setlinewidth } def
/LTb { BL [] 0 0 0 DL } def
/LTa { AL [1 udl mul 2 udl mul] 0 setdash 0 0 0 setrgbcolor } def
/LT0 { PL [] 1 0 0 DL } def
/LT1 { PL [4 dl 2 dl] 0 1 0 DL } def
/LT2 { PL [2 dl 3 dl] 0 0 1 DL } def
/LT3 { PL [1 dl 1.5 dl] 1 0 1 DL } def
/LT4 { PL [5 dl 2 dl 1 dl 2 dl] 0 1 1 DL } def
/LT5 { PL [4 dl 3 dl 1 dl 3 dl] 1 1 0 DL } def
/LT6 { PL [2 dl 2 dl 2 dl 4 dl] 0 0 0 DL } def
/LT7 { PL [2 dl 2 dl 2 dl 2 dl 2 dl 4 dl] 1 0.3 0 DL } def
/LT8 { PL [2 dl 2 dl 2 dl 2 dl 2 dl 2 dl 2 dl 4 dl] 0.5 0.5 0.5 DL } def
/Pnt { stroke [] 0 setdash
   gsave 1 setlinecap M 0 0 V stroke grestore } def
/Dia { stroke [] 0 setdash 2 copy vpt add M
  hpt neg vpt neg V hpt vpt neg V
  hpt vpt V hpt neg vpt V closepath stroke
  Pnt } def
/Pls { stroke [] 0 setdash vpt sub M 0 vpt2 V
  currentpoint stroke M
  hpt neg vpt neg R hpt2 0 V stroke
  } def
/Box { stroke [] 0 setdash 2 copy exch hpt sub exch vpt add M
  0 vpt2 neg V hpt2 0 V 0 vpt2 V
  hpt2 neg 0 V closepath stroke
  Pnt } def
/Crs { stroke [] 0 setdash exch hpt sub exch vpt add M
  hpt2 vpt2 neg V currentpoint stroke M
  hpt2 neg 0 R hpt2 vpt2 V stroke } def
/TriU { stroke [] 0 setdash 2 copy vpt 1.12 mul add M
  hpt neg vpt -1.62 mul V
  hpt 2 mul 0 V
  hpt neg vpt 1.62 mul V closepath stroke
  Pnt  } def
/Star { 2 copy Pls Crs } def
/BoxF { stroke [] 0 setdash exch hpt sub exch vpt add M
  0 vpt2 neg V  hpt2 0 V  0 vpt2 V
  hpt2 neg 0 V  closepath fill } def
/TriUF { stroke [] 0 setdash vpt 1.12 mul add M
  hpt neg vpt -1.62 mul V
  hpt 2 mul 0 V
  hpt neg vpt 1.62 mul V closepath fill } def
/TriD { stroke [] 0 setdash 2 copy vpt 1.12 mul sub M
  hpt neg vpt 1.62 mul V
  hpt 2 mul 0 V
  hpt neg vpt -1.62 mul V closepath stroke
  Pnt  } def
/TriDF { stroke [] 0 setdash vpt 1.12 mul sub M
  hpt neg vpt 1.62 mul V
  hpt 2 mul 0 V
  hpt neg vpt -1.62 mul V closepath fill} def
/DiaF { stroke [] 0 setdash vpt add M
  hpt neg vpt neg V hpt vpt neg V
  hpt vpt V hpt neg vpt V closepath fill } def
/Pent { stroke [] 0 setdash 2 copy gsave
  translate 0 hpt M 4 {72 rotate 0 hpt L} repeat
  closepath stroke grestore Pnt } def
/PentF { stroke [] 0 setdash gsave
  translate 0 hpt M 4 {72 rotate 0 hpt L} repeat
  closepath fill grestore } def
/Circle { stroke [] 0 setdash 2 copy
  hpt 0 360 arc stroke Pnt } def
/CircleF { stroke [] 0 setdash hpt 0 360 arc fill } def
/C0 { BL [] 0 setdash 2 copy moveto vpt 90 450  arc } bind def
/C1 { BL [] 0 setdash 2 copy        moveto
       2 copy  vpt 0 90 arc closepath fill
               vpt 0 360 arc closepath } bind def
/C2 { BL [] 0 setdash 2 copy moveto
       2 copy  vpt 90 180 arc closepath fill
               vpt 0 360 arc closepath } bind def
/C3 { BL [] 0 setdash 2 copy moveto
       2 copy  vpt 0 180 arc closepath fill
               vpt 0 360 arc closepath } bind def
/C4 { BL [] 0 setdash 2 copy moveto
       2 copy  vpt 180 270 arc closepath fill
               vpt 0 360 arc closepath } bind def
/C5 { BL [] 0 setdash 2 copy moveto
       2 copy  vpt 0 90 arc
       2 copy moveto
       2 copy  vpt 180 270 arc closepath fill
               vpt 0 360 arc } bind def
/C6 { BL [] 0 setdash 2 copy moveto
      2 copy  vpt 90 270 arc closepath fill
              vpt 0 360 arc closepath } bind def
/C7 { BL [] 0 setdash 2 copy moveto
      2 copy  vpt 0 270 arc closepath fill
              vpt 0 360 arc closepath } bind def
/C8 { BL [] 0 setdash 2 copy moveto
      2 copy vpt 270 360 arc closepath fill
              vpt 0 360 arc closepath } bind def
/C9 { BL [] 0 setdash 2 copy moveto
      2 copy  vpt 270 450 arc closepath fill
              vpt 0 360 arc closepath } bind def
/C10 { BL [] 0 setdash 2 copy 2 copy moveto vpt 270 360 arc closepath fill
       2 copy moveto
       2 copy vpt 90 180 arc closepath fill
               vpt 0 360 arc closepath } bind def
/C11 { BL [] 0 setdash 2 copy moveto
       2 copy  vpt 0 180 arc closepath fill
       2 copy moveto
       2 copy  vpt 270 360 arc closepath fill
               vpt 0 360 arc closepath } bind def
/C12 { BL [] 0 setdash 2 copy moveto
       2 copy  vpt 180 360 arc closepath fill
               vpt 0 360 arc closepath } bind def
/C13 { BL [] 0 setdash  2 copy moveto
       2 copy  vpt 0 90 arc closepath fill
       2 copy moveto
       2 copy  vpt 180 360 arc closepath fill
               vpt 0 360 arc closepath } bind def
/C14 { BL [] 0 setdash 2 copy moveto
       2 copy  vpt 90 360 arc closepath fill
               vpt 0 360 arc } bind def
/C15 { BL [] 0 setdash 2 copy vpt 0 360 arc closepath fill
               vpt 0 360 arc closepath } bind def
/Rec   { newpath 4 2 roll moveto 1 index 0 rlineto 0 exch rlineto
       neg 0 rlineto closepath } bind def
/Square { dup Rec } bind def
/Bsquare { vpt sub exch vpt sub exch vpt2 Square } bind def
/S0 { BL [] 0 setdash 2 copy moveto 0 vpt rlineto BL Bsquare } bind def
/S1 { BL [] 0 setdash 2 copy vpt Square fill Bsquare } bind def
/S2 { BL [] 0 setdash 2 copy exch vpt sub exch vpt Square fill Bsquare } bind def
/S3 { BL [] 0 setdash 2 copy exch vpt sub exch vpt2 vpt Rec fill Bsquare } bind def
/S4 { BL [] 0 setdash 2 copy exch vpt sub exch vpt sub vpt Square fill Bsquare } bind def
/S5 { BL [] 0 setdash 2 copy 2 copy vpt Square fill
       exch vpt sub exch vpt sub vpt Square fill Bsquare } bind def
/S6 { BL [] 0 setdash 2 copy exch vpt sub exch vpt sub vpt vpt2 Rec fill Bsquare } bind def
/S7 { BL [] 0 setdash 2 copy exch vpt sub exch vpt sub vpt vpt2 Rec fill
       2 copy vpt Square fill
       Bsquare } bind def
/S8 { BL [] 0 setdash 2 copy vpt sub vpt Square fill Bsquare } bind def
/S9 { BL [] 0 setdash 2 copy vpt sub vpt vpt2 Rec fill Bsquare } bind def
/S10 { BL [] 0 setdash 2 copy vpt sub vpt Square fill 2 copy exch vpt sub exch vpt Square fill
       Bsquare } bind def
/S11 { BL [] 0 setdash 2 copy vpt sub vpt Square fill 2 copy exch vpt sub exch vpt2 vpt Rec fill
       Bsquare } bind def
/S12 { BL [] 0 setdash 2 copy exch vpt sub exch vpt sub vpt2 vpt Rec fill Bsquare } bind def
/S13 { BL [] 0 setdash 2 copy exch vpt sub exch vpt sub vpt2 vpt Rec fill
       2 copy vpt Square fill Bsquare } bind def
/S14 { BL [] 0 setdash 2 copy exch vpt sub exch vpt sub vpt2 vpt Rec fill
       2 copy exch vpt sub exch vpt Square fill Bsquare } bind def
/S15 { BL [] 0 setdash 2 copy Bsquare fill Bsquare } bind def
/D0 { gsave translate 45 rotate 0 0 S0 stroke grestore } bind def
/D1 { gsave translate 45 rotate 0 0 S1 stroke grestore } bind def
/D2 { gsave translate 45 rotate 0 0 S2 stroke grestore } bind def
/D3 { gsave translate 45 rotate 0 0 S3 stroke grestore } bind def
/D4 { gsave translate 45 rotate 0 0 S4 stroke grestore } bind def
/D5 { gsave translate 45 rotate 0 0 S5 stroke grestore } bind def
/D6 { gsave translate 45 rotate 0 0 S6 stroke grestore } bind def
/D7 { gsave translate 45 rotate 0 0 S7 stroke grestore } bind def
/D8 { gsave translate 45 rotate 0 0 S8 stroke grestore } bind def
/D9 { gsave translate 45 rotate 0 0 S9 stroke grestore } bind def
/D10 { gsave translate 45 rotate 0 0 S10 stroke grestore } bind def
/D11 { gsave translate 45 rotate 0 0 S11 stroke grestore } bind def
/D12 { gsave translate 45 rotate 0 0 S12 stroke grestore } bind def
/D13 { gsave translate 45 rotate 0 0 S13 stroke grestore } bind def
/D14 { gsave translate 45 rotate 0 0 S14 stroke grestore } bind def
/D15 { gsave translate 45 rotate 0 0 S15 stroke grestore } bind def
/DiaE { stroke [] 0 setdash vpt add M
  hpt neg vpt neg V hpt vpt neg V
  hpt vpt V hpt neg vpt V closepath stroke } def
/BoxE { stroke [] 0 setdash exch hpt sub exch vpt add M
  0 vpt2 neg V hpt2 0 V 0 vpt2 V
  hpt2 neg 0 V closepath stroke } def
/TriUE { stroke [] 0 setdash vpt 1.12 mul add M
  hpt neg vpt -1.62 mul V
  hpt 2 mul 0 V
  hpt neg vpt 1.62 mul V closepath stroke } def
/TriDE { stroke [] 0 setdash vpt 1.12 mul sub M
  hpt neg vpt 1.62 mul V
  hpt 2 mul 0 V
  hpt neg vpt -1.62 mul V closepath stroke } def
/PentE { stroke [] 0 setdash gsave
  translate 0 hpt M 4 {72 rotate 0 hpt L} repeat
  closepath stroke grestore } def
/CircE { stroke [] 0 setdash 
  hpt 0 360 arc stroke } def
/Opaque { gsave closepath 1 setgray fill grestore 0 setgray closepath } def
/DiaW { stroke [] 0 setdash vpt add M
  hpt neg vpt neg V hpt vpt neg V
  hpt vpt V hpt neg vpt V Opaque stroke } def
/BoxW { stroke [] 0 setdash exch hpt sub exch vpt add M
  0 vpt2 neg V hpt2 0 V 0 vpt2 V
  hpt2 neg 0 V Opaque stroke } def
/TriUW { stroke [] 0 setdash vpt 1.12 mul add M
  hpt neg vpt -1.62 mul V
  hpt 2 mul 0 V
  hpt neg vpt 1.62 mul V Opaque stroke } def
/TriDW { stroke [] 0 setdash vpt 1.12 mul sub M
  hpt neg vpt 1.62 mul V
  hpt 2 mul 0 V
  hpt neg vpt -1.62 mul V Opaque stroke } def
/PentW { stroke [] 0 setdash gsave
  translate 0 hpt M 4 {72 rotate 0 hpt L} repeat
  Opaque stroke grestore } def
/CircW { stroke [] 0 setdash 
  hpt 0 360 arc Opaque stroke } def
/BoxFill { gsave Rec 1 setgray fill grestore } def
end
%%EndProlog
}}%
\begin{picture}(3600,2160)(0,0)%
{\GNUPLOTspecial{"
gnudict begin
gsave
0 0 translate
0.100 0.100 scale
0 setgray
newpath
1.000 UL
LTb
350 300 M
63 0 V
3037 0 R
-63 0 V
350 593 M
63 0 V
3037 0 R
-63 0 V
350 887 M
63 0 V
3037 0 R
-63 0 V
350 1180 M
63 0 V
3037 0 R
-63 0 V
350 1473 M
63 0 V
3037 0 R
-63 0 V
350 1767 M
63 0 V
3037 0 R
-63 0 V
350 2060 M
63 0 V
3037 0 R
-63 0 V
660 300 M
0 63 V
0 1697 R
0 -63 V
1280 300 M
0 63 V
0 1697 R
0 -63 V
1900 300 M
0 63 V
0 1697 R
0 -63 V
2520 300 M
0 63 V
0 1697 R
0 -63 V
3140 300 M
0 63 V
0 1697 R
0 -63 V
1.000 UL
LTb
350 300 M
3100 0 V
0 1760 V
-3100 0 V
350 300 L
1.000 UL
LT0
3087 1947 M
263 0 V
350 312 M
31 3 V
32 3 V
31 3 V
31 5 V
32 5 V
31 7 V
31 8 V
32 10 V
31 11 V
31 14 V
31 17 V
32 21 V
31 25 V
31 31 V
32 36 V
31 45 V
31 54 V
32 65 V
31 79 V
31 95 V
32 116 V
31 140 V
31 169 V
32 205 V
31 118 V
31 -225 V
31 -187 V
32 -154 V
31 -127 V
31 -105 V
32 -86 V
31 -72 V
31 -59 V
32 -49 V
31 -40 V
31 -33 V
32 -28 V
31 -22 V
31 -19 V
32 -15 V
31 -13 V
31 -10 V
31 -8 V
32 -6 V
31 -5 V
31 -4 V
32 -3 V
31 -2 V
31 -1 V
32 0 V
31 1 V
31 2 V
32 3 V
31 4 V
31 5 V
32 6 V
31 8 V
31 10 V
31 13 V
32 15 V
31 19 V
31 22 V
32 28 V
31 33 V
31 40 V
32 49 V
31 59 V
31 72 V
32 86 V
31 105 V
31 127 V
32 154 V
31 187 V
31 225 V
31 -118 V
32 -205 V
31 -169 V
31 -140 V
32 -116 V
31 -95 V
31 -79 V
32 -65 V
31 -54 V
31 -45 V
32 -36 V
31 -31 V
31 -25 V
32 -21 V
31 -17 V
31 -14 V
31 -11 V
32 -10 V
31 -8 V
31 -7 V
32 -5 V
31 -5 V
31 -3 V
32 -3 V
31 -3 V
1.000 UL
LT1
3087 1847 M
263 0 V
350 300 M
31 0 V
32 0 V
31 1 V
31 0 V
32 0 V
31 0 V
31 0 V
32 0 V
31 1 V
31 0 V
31 1 V
32 0 V
31 1 V
31 1 V
32 1 V
31 1 V
31 1 V
32 2 V
31 2 V
31 2 V
32 3 V
31 4 V
31 5 V
32 5 V
31 6 V
31 8 V
31 10 V
32 11 V
31 14 V
31 17 V
32 21 V
31 25 V
31 30 V
32 36 V
31 44 V
31 53 V
32 64 V
31 78 V
31 95 V
32 114 V
31 138 V
31 167 V
31 203 V
32 245 V
31 62 V
31 -170 V
32 -122 V
31 -79 V
31 -39 V
32 0 V
31 39 V
31 79 V
32 122 V
31 170 V
31 -62 V
32 -245 V
31 -203 V
31 -167 V
31 -138 V
32 -114 V
31 -95 V
31 -78 V
32 -64 V
31 -53 V
31 -44 V
32 -36 V
31 -30 V
31 -25 V
32 -21 V
31 -17 V
31 -14 V
32 -11 V
31 -10 V
31 -8 V
31 -6 V
32 -5 V
31 -5 V
31 -4 V
32 -3 V
31 -2 V
31 -2 V
32 -2 V
31 -1 V
31 -1 V
32 -1 V
31 -1 V
31 -1 V
32 0 V
31 -1 V
31 0 V
31 -1 V
32 0 V
31 0 V
31 0 V
32 0 V
31 0 V
31 -1 V
32 0 V
31 0 V
stroke
grestore
end
showpage
}}%
\put(3037,1847){\makebox(0,0)[r]{$R=0.1$ nm}}%
\put(3037,1947){\makebox(0,0)[r]{$R=0.5$ nm}}%
\put(1900,50){\makebox(0,0){$r$ [nm]}}%
\put(100,1180){%
\special{ps: gsave currentpoint currentpoint translate
270 rotate neg exch neg exch translate}%
\makebox(0,0)[b]{\shortstack{$\psi_+({\bf r},{\bf R})$}}%
\special{ps: currentpoint grestore moveto}%
}%
\put(3140,200){\makebox(0,0){0.4}}%
\put(2520,200){\makebox(0,0){0.2}}%
\put(1900,200){\makebox(0,0){0}}%
\put(1280,200){\makebox(0,0){-0.2}}%
\put(660,200){\makebox(0,0){-0.4}}%
\put(300,2060){\makebox(0,0)[r]{60}}%
\put(300,1767){\makebox(0,0)[r]{50}}%
\put(300,1473){\makebox(0,0)[r]{40}}%
\put(300,1180){\makebox(0,0)[r]{30}}%
\put(300,887){\makebox(0,0)[r]{20}}%
\put(300,593){\makebox(0,0)[r]{10}}%
\put(300,300){\makebox(0,0)[r]{0}}%
\end{picture}%
\endgroup
\endinput

\end{center}
\caption{Plott av  $\psi_+({\bf r},{\bf R})$ for ulike verdier av
$|{\bf R}|$. Merk at funksjonene ikke er normaliserte. Enheten p\aa\
$y$-aksen er vilk\aa rlig.\label{63}}
\end{figure} 
\begin{figure}
\begin{center}
% GNUPLOT: LaTeX picture with Postscript
\begingroup%
  \makeatletter%
  \newcommand{\GNUPLOTspecial}{%
    \@sanitize\catcode`\%=14\relax\special}%
  \setlength{\unitlength}{0.1bp}%
{\GNUPLOTspecial{!
%!PS-Adobe-2.0 EPSF-2.0
%%Title: p2pluss.tex
%%Creator: gnuplot 3.7 patchlevel 0.2
%%CreationDate: Wed Apr 26 15:13:43 2000
%%DocumentFonts: 
%%BoundingBox: 0 0 360 216
%%Orientation: Landscape
%%EndComments
/gnudict 256 dict def
gnudict begin
/Color false def
/Solid false def
/gnulinewidth 5.000 def
/userlinewidth gnulinewidth def
/vshift -33 def
/dl {10 mul} def
/hpt_ 31.5 def
/vpt_ 31.5 def
/hpt hpt_ def
/vpt vpt_ def
/M {moveto} bind def
/L {lineto} bind def
/R {rmoveto} bind def
/V {rlineto} bind def
/vpt2 vpt 2 mul def
/hpt2 hpt 2 mul def
/Lshow { currentpoint stroke M
  0 vshift R show } def
/Rshow { currentpoint stroke M
  dup stringwidth pop neg vshift R show } def
/Cshow { currentpoint stroke M
  dup stringwidth pop -2 div vshift R show } def
/UP { dup vpt_ mul /vpt exch def hpt_ mul /hpt exch def
  /hpt2 hpt 2 mul def /vpt2 vpt 2 mul def } def
/DL { Color {setrgbcolor Solid {pop []} if 0 setdash }
 {pop pop pop Solid {pop []} if 0 setdash} ifelse } def
/BL { stroke userlinewidth 2 mul setlinewidth } def
/AL { stroke userlinewidth 2 div setlinewidth } def
/UL { dup gnulinewidth mul /userlinewidth exch def
      10 mul /udl exch def } def
/PL { stroke userlinewidth setlinewidth } def
/LTb { BL [] 0 0 0 DL } def
/LTa { AL [1 udl mul 2 udl mul] 0 setdash 0 0 0 setrgbcolor } def
/LT0 { PL [] 1 0 0 DL } def
/LT1 { PL [4 dl 2 dl] 0 1 0 DL } def
/LT2 { PL [2 dl 3 dl] 0 0 1 DL } def
/LT3 { PL [1 dl 1.5 dl] 1 0 1 DL } def
/LT4 { PL [5 dl 2 dl 1 dl 2 dl] 0 1 1 DL } def
/LT5 { PL [4 dl 3 dl 1 dl 3 dl] 1 1 0 DL } def
/LT6 { PL [2 dl 2 dl 2 dl 4 dl] 0 0 0 DL } def
/LT7 { PL [2 dl 2 dl 2 dl 2 dl 2 dl 4 dl] 1 0.3 0 DL } def
/LT8 { PL [2 dl 2 dl 2 dl 2 dl 2 dl 2 dl 2 dl 4 dl] 0.5 0.5 0.5 DL } def
/Pnt { stroke [] 0 setdash
   gsave 1 setlinecap M 0 0 V stroke grestore } def
/Dia { stroke [] 0 setdash 2 copy vpt add M
  hpt neg vpt neg V hpt vpt neg V
  hpt vpt V hpt neg vpt V closepath stroke
  Pnt } def
/Pls { stroke [] 0 setdash vpt sub M 0 vpt2 V
  currentpoint stroke M
  hpt neg vpt neg R hpt2 0 V stroke
  } def
/Box { stroke [] 0 setdash 2 copy exch hpt sub exch vpt add M
  0 vpt2 neg V hpt2 0 V 0 vpt2 V
  hpt2 neg 0 V closepath stroke
  Pnt } def
/Crs { stroke [] 0 setdash exch hpt sub exch vpt add M
  hpt2 vpt2 neg V currentpoint stroke M
  hpt2 neg 0 R hpt2 vpt2 V stroke } def
/TriU { stroke [] 0 setdash 2 copy vpt 1.12 mul add M
  hpt neg vpt -1.62 mul V
  hpt 2 mul 0 V
  hpt neg vpt 1.62 mul V closepath stroke
  Pnt  } def
/Star { 2 copy Pls Crs } def
/BoxF { stroke [] 0 setdash exch hpt sub exch vpt add M
  0 vpt2 neg V  hpt2 0 V  0 vpt2 V
  hpt2 neg 0 V  closepath fill } def
/TriUF { stroke [] 0 setdash vpt 1.12 mul add M
  hpt neg vpt -1.62 mul V
  hpt 2 mul 0 V
  hpt neg vpt 1.62 mul V closepath fill } def
/TriD { stroke [] 0 setdash 2 copy vpt 1.12 mul sub M
  hpt neg vpt 1.62 mul V
  hpt 2 mul 0 V
  hpt neg vpt -1.62 mul V closepath stroke
  Pnt  } def
/TriDF { stroke [] 0 setdash vpt 1.12 mul sub M
  hpt neg vpt 1.62 mul V
  hpt 2 mul 0 V
  hpt neg vpt -1.62 mul V closepath fill} def
/DiaF { stroke [] 0 setdash vpt add M
  hpt neg vpt neg V hpt vpt neg V
  hpt vpt V hpt neg vpt V closepath fill } def
/Pent { stroke [] 0 setdash 2 copy gsave
  translate 0 hpt M 4 {72 rotate 0 hpt L} repeat
  closepath stroke grestore Pnt } def
/PentF { stroke [] 0 setdash gsave
  translate 0 hpt M 4 {72 rotate 0 hpt L} repeat
  closepath fill grestore } def
/Circle { stroke [] 0 setdash 2 copy
  hpt 0 360 arc stroke Pnt } def
/CircleF { stroke [] 0 setdash hpt 0 360 arc fill } def
/C0 { BL [] 0 setdash 2 copy moveto vpt 90 450  arc } bind def
/C1 { BL [] 0 setdash 2 copy        moveto
       2 copy  vpt 0 90 arc closepath fill
               vpt 0 360 arc closepath } bind def
/C2 { BL [] 0 setdash 2 copy moveto
       2 copy  vpt 90 180 arc closepath fill
               vpt 0 360 arc closepath } bind def
/C3 { BL [] 0 setdash 2 copy moveto
       2 copy  vpt 0 180 arc closepath fill
               vpt 0 360 arc closepath } bind def
/C4 { BL [] 0 setdash 2 copy moveto
       2 copy  vpt 180 270 arc closepath fill
               vpt 0 360 arc closepath } bind def
/C5 { BL [] 0 setdash 2 copy moveto
       2 copy  vpt 0 90 arc
       2 copy moveto
       2 copy  vpt 180 270 arc closepath fill
               vpt 0 360 arc } bind def
/C6 { BL [] 0 setdash 2 copy moveto
      2 copy  vpt 90 270 arc closepath fill
              vpt 0 360 arc closepath } bind def
/C7 { BL [] 0 setdash 2 copy moveto
      2 copy  vpt 0 270 arc closepath fill
              vpt 0 360 arc closepath } bind def
/C8 { BL [] 0 setdash 2 copy moveto
      2 copy vpt 270 360 arc closepath fill
              vpt 0 360 arc closepath } bind def
/C9 { BL [] 0 setdash 2 copy moveto
      2 copy  vpt 270 450 arc closepath fill
              vpt 0 360 arc closepath } bind def
/C10 { BL [] 0 setdash 2 copy 2 copy moveto vpt 270 360 arc closepath fill
       2 copy moveto
       2 copy vpt 90 180 arc closepath fill
               vpt 0 360 arc closepath } bind def
/C11 { BL [] 0 setdash 2 copy moveto
       2 copy  vpt 0 180 arc closepath fill
       2 copy moveto
       2 copy  vpt 270 360 arc closepath fill
               vpt 0 360 arc closepath } bind def
/C12 { BL [] 0 setdash 2 copy moveto
       2 copy  vpt 180 360 arc closepath fill
               vpt 0 360 arc closepath } bind def
/C13 { BL [] 0 setdash  2 copy moveto
       2 copy  vpt 0 90 arc closepath fill
       2 copy moveto
       2 copy  vpt 180 360 arc closepath fill
               vpt 0 360 arc closepath } bind def
/C14 { BL [] 0 setdash 2 copy moveto
       2 copy  vpt 90 360 arc closepath fill
               vpt 0 360 arc } bind def
/C15 { BL [] 0 setdash 2 copy vpt 0 360 arc closepath fill
               vpt 0 360 arc closepath } bind def
/Rec   { newpath 4 2 roll moveto 1 index 0 rlineto 0 exch rlineto
       neg 0 rlineto closepath } bind def
/Square { dup Rec } bind def
/Bsquare { vpt sub exch vpt sub exch vpt2 Square } bind def
/S0 { BL [] 0 setdash 2 copy moveto 0 vpt rlineto BL Bsquare } bind def
/S1 { BL [] 0 setdash 2 copy vpt Square fill Bsquare } bind def
/S2 { BL [] 0 setdash 2 copy exch vpt sub exch vpt Square fill Bsquare } bind def
/S3 { BL [] 0 setdash 2 copy exch vpt sub exch vpt2 vpt Rec fill Bsquare } bind def
/S4 { BL [] 0 setdash 2 copy exch vpt sub exch vpt sub vpt Square fill Bsquare } bind def
/S5 { BL [] 0 setdash 2 copy 2 copy vpt Square fill
       exch vpt sub exch vpt sub vpt Square fill Bsquare } bind def
/S6 { BL [] 0 setdash 2 copy exch vpt sub exch vpt sub vpt vpt2 Rec fill Bsquare } bind def
/S7 { BL [] 0 setdash 2 copy exch vpt sub exch vpt sub vpt vpt2 Rec fill
       2 copy vpt Square fill
       Bsquare } bind def
/S8 { BL [] 0 setdash 2 copy vpt sub vpt Square fill Bsquare } bind def
/S9 { BL [] 0 setdash 2 copy vpt sub vpt vpt2 Rec fill Bsquare } bind def
/S10 { BL [] 0 setdash 2 copy vpt sub vpt Square fill 2 copy exch vpt sub exch vpt Square fill
       Bsquare } bind def
/S11 { BL [] 0 setdash 2 copy vpt sub vpt Square fill 2 copy exch vpt sub exch vpt2 vpt Rec fill
       Bsquare } bind def
/S12 { BL [] 0 setdash 2 copy exch vpt sub exch vpt sub vpt2 vpt Rec fill Bsquare } bind def
/S13 { BL [] 0 setdash 2 copy exch vpt sub exch vpt sub vpt2 vpt Rec fill
       2 copy vpt Square fill Bsquare } bind def
/S14 { BL [] 0 setdash 2 copy exch vpt sub exch vpt sub vpt2 vpt Rec fill
       2 copy exch vpt sub exch vpt Square fill Bsquare } bind def
/S15 { BL [] 0 setdash 2 copy Bsquare fill Bsquare } bind def
/D0 { gsave translate 45 rotate 0 0 S0 stroke grestore } bind def
/D1 { gsave translate 45 rotate 0 0 S1 stroke grestore } bind def
/D2 { gsave translate 45 rotate 0 0 S2 stroke grestore } bind def
/D3 { gsave translate 45 rotate 0 0 S3 stroke grestore } bind def
/D4 { gsave translate 45 rotate 0 0 S4 stroke grestore } bind def
/D5 { gsave translate 45 rotate 0 0 S5 stroke grestore } bind def
/D6 { gsave translate 45 rotate 0 0 S6 stroke grestore } bind def
/D7 { gsave translate 45 rotate 0 0 S7 stroke grestore } bind def
/D8 { gsave translate 45 rotate 0 0 S8 stroke grestore } bind def
/D9 { gsave translate 45 rotate 0 0 S9 stroke grestore } bind def
/D10 { gsave translate 45 rotate 0 0 S10 stroke grestore } bind def
/D11 { gsave translate 45 rotate 0 0 S11 stroke grestore } bind def
/D12 { gsave translate 45 rotate 0 0 S12 stroke grestore } bind def
/D13 { gsave translate 45 rotate 0 0 S13 stroke grestore } bind def
/D14 { gsave translate 45 rotate 0 0 S14 stroke grestore } bind def
/D15 { gsave translate 45 rotate 0 0 S15 stroke grestore } bind def
/DiaE { stroke [] 0 setdash vpt add M
  hpt neg vpt neg V hpt vpt neg V
  hpt vpt V hpt neg vpt V closepath stroke } def
/BoxE { stroke [] 0 setdash exch hpt sub exch vpt add M
  0 vpt2 neg V hpt2 0 V 0 vpt2 V
  hpt2 neg 0 V closepath stroke } def
/TriUE { stroke [] 0 setdash vpt 1.12 mul add M
  hpt neg vpt -1.62 mul V
  hpt 2 mul 0 V
  hpt neg vpt 1.62 mul V closepath stroke } def
/TriDE { stroke [] 0 setdash vpt 1.12 mul sub M
  hpt neg vpt 1.62 mul V
  hpt 2 mul 0 V
  hpt neg vpt -1.62 mul V closepath stroke } def
/PentE { stroke [] 0 setdash gsave
  translate 0 hpt M 4 {72 rotate 0 hpt L} repeat
  closepath stroke grestore } def
/CircE { stroke [] 0 setdash 
  hpt 0 360 arc stroke } def
/Opaque { gsave closepath 1 setgray fill grestore 0 setgray closepath } def
/DiaW { stroke [] 0 setdash vpt add M
  hpt neg vpt neg V hpt vpt neg V
  hpt vpt V hpt neg vpt V Opaque stroke } def
/BoxW { stroke [] 0 setdash exch hpt sub exch vpt add M
  0 vpt2 neg V hpt2 0 V 0 vpt2 V
  hpt2 neg 0 V Opaque stroke } def
/TriUW { stroke [] 0 setdash vpt 1.12 mul add M
  hpt neg vpt -1.62 mul V
  hpt 2 mul 0 V
  hpt neg vpt 1.62 mul V Opaque stroke } def
/TriDW { stroke [] 0 setdash vpt 1.12 mul sub M
  hpt neg vpt 1.62 mul V
  hpt 2 mul 0 V
  hpt neg vpt -1.62 mul V Opaque stroke } def
/PentW { stroke [] 0 setdash gsave
  translate 0 hpt M 4 {72 rotate 0 hpt L} repeat
  Opaque stroke grestore } def
/CircW { stroke [] 0 setdash 
  hpt 0 360 arc Opaque stroke } def
/BoxFill { gsave Rec 1 setgray fill grestore } def
end
%%EndProlog
}}%
\begin{picture}(3600,2160)(0,0)%
{\GNUPLOTspecial{"
gnudict begin
gsave
0 0 translate
0.100 0.100 scale
0 setgray
newpath
1.000 UL
LTb
450 300 M
63 0 V
2937 0 R
-63 0 V
450 593 M
63 0 V
2937 0 R
-63 0 V
450 887 M
63 0 V
2937 0 R
-63 0 V
450 1180 M
63 0 V
2937 0 R
-63 0 V
450 1473 M
63 0 V
2937 0 R
-63 0 V
450 1767 M
63 0 V
2937 0 R
-63 0 V
450 2060 M
63 0 V
2937 0 R
-63 0 V
750 300 M
0 63 V
0 1697 R
0 -63 V
1350 300 M
0 63 V
0 1697 R
0 -63 V
1950 300 M
0 63 V
0 1697 R
0 -63 V
2550 300 M
0 63 V
0 1697 R
0 -63 V
3150 300 M
0 63 V
0 1697 R
0 -63 V
1.000 UL
LTb
450 300 M
3000 0 V
0 1760 V
-3000 0 V
450 300 L
1.000 UL
LT0
3087 1947 M
263 0 V
450 300 M
30 0 V
31 0 V
30 0 V
30 0 V
31 1 V
30 0 V
30 0 V
30 1 V
31 1 V
30 2 V
30 2 V
31 3 V
30 4 V
30 7 V
31 9 V
30 15 V
30 20 V
30 31 V
31 44 V
30 66 V
30 95 V
31 140 V
30 206 V
30 300 V
31 200 V
30 -364 V
30 -249 V
30 -169 V
31 -116 V
30 -79 V
30 -54 V
31 -37 V
30 -25 V
30 -17 V
31 -12 V
30 -8 V
30 -5 V
31 -4 V
30 -2 V
30 -2 V
30 -1 V
31 -1 V
30 -1 V
30 0 V
31 0 V
30 0 V
30 -1 V
31 0 V
30 0 V
30 0 V
30 0 V
31 0 V
30 1 V
30 0 V
31 0 V
30 0 V
30 1 V
31 1 V
30 1 V
30 2 V
30 2 V
31 4 V
30 5 V
30 8 V
31 12 V
30 17 V
30 25 V
31 37 V
30 54 V
30 79 V
31 116 V
30 169 V
30 249 V
30 364 V
31 -200 V
30 -300 V
30 -206 V
31 -140 V
30 -95 V
30 -66 V
31 -44 V
30 -31 V
30 -20 V
30 -15 V
31 -9 V
30 -7 V
30 -4 V
31 -3 V
30 -2 V
30 -2 V
31 -1 V
30 -1 V
30 0 V
30 0 V
31 -1 V
30 0 V
30 0 V
31 0 V
30 0 V
1.000 UL
LT1
3087 1847 M
263 0 V
450 300 M
30 0 V
31 0 V
30 0 V
30 0 V
31 0 V
30 0 V
30 0 V
30 0 V
31 0 V
30 0 V
30 0 V
31 0 V
30 0 V
30 0 V
31 0 V
30 0 V
30 0 V
30 0 V
31 0 V
30 0 V
30 0 V
31 0 V
30 0 V
30 1 V
31 0 V
30 0 V
30 1 V
30 1 V
31 1 V
30 2 V
30 3 V
31 5 V
30 6 V
30 10 V
31 14 V
30 20 V
30 30 V
31 43 V
30 64 V
30 93 V
30 137 V
31 200 V
30 294 V
30 430 V
31 122 V
30 -321 V
30 -207 V
31 -123 V
30 -57 V
30 0 V
30 57 V
31 123 V
30 207 V
30 321 V
31 -122 V
30 -430 V
30 -294 V
31 -200 V
30 -137 V
30 -93 V
30 -64 V
31 -43 V
30 -30 V
30 -20 V
31 -14 V
30 -10 V
30 -6 V
31 -5 V
30 -3 V
30 -2 V
31 -1 V
30 -1 V
30 -1 V
30 0 V
31 0 V
30 -1 V
30 0 V
31 0 V
30 0 V
30 0 V
31 0 V
30 0 V
30 0 V
30 0 V
31 0 V
30 0 V
30 0 V
31 0 V
30 0 V
30 0 V
31 0 V
30 0 V
30 0 V
30 0 V
31 0 V
30 0 V
30 0 V
31 0 V
30 0 V
stroke
grestore
end
showpage
}}%
\put(3037,1847){\makebox(0,0)[r]{$R=0.1$ nm}}%
\put(3037,1947){\makebox(0,0)[r]{$R=0.5$ nm}}%
\put(1950,50){\makebox(0,0){$r$ [nm]}}%
\put(100,1180){%
\special{ps: gsave currentpoint currentpoint translate
270 rotate neg exch neg exch translate}%
\makebox(0,0)[b]{\shortstack{$|\psi_+({\bf r},{\bf R})|^2$}}%
\special{ps: currentpoint grestore moveto}%
}%
\put(3150,200){\makebox(0,0){0.4}}%
\put(2550,200){\makebox(0,0){0.2}}%
\put(1950,200){\makebox(0,0){0}}%
\put(1350,200){\makebox(0,0){-0.2}}%
\put(750,200){\makebox(0,0){-0.4}}%
\put(400,2060){\makebox(0,0)[r]{3000}}%
\put(400,1767){\makebox(0,0)[r]{2500}}%
\put(400,1473){\makebox(0,0)[r]{2000}}%
\put(400,1180){\makebox(0,0)[r]{1500}}%
\put(400,887){\makebox(0,0)[r]{1000}}%
\put(400,593){\makebox(0,0)[r]{500}}%
\put(400,300){\makebox(0,0)[r]{0}}%
\end{picture}%
\endgroup
\endinput

\end{center}
\caption{Plott av  $|\psi_+({\bf r},{\bf R})|^2$ for ulike verdier av
$|{\bf R}|$. Merk at funksjonene ikke er normaliserte. Enheten p\aa\
$y$-aksen er vilk\aa rlig.\label{64}}
\end{figure} 
I Figur \ref{64} har vi plotta sannsynlighetstettheten 
for $\psi_+$. Vi ser at vi har et omr\aa de mellom de to
protonene hvor sannsynlighetstettheten er forskjellig fra null
n\aa r $R=0.1$ nm, mens sannsynlighetstettheten minsker n\aa r av
$R$ \o ker. Det betyr at elektronet kan v\ae re mellom de to protonene.

Fysisk betyr det at vi har en negativ ladningsfordeling mellom
protonene. Denne negative ladningsfordelingen tiltrekker protonene
og binder systemet! Det skal vi vise nedenfor i v\aa r beregning
av energien. At et elektrons (eller flere) ladningsfordeling deles mellom
to atomer, er et eksempel p\aa\ det som kalles {\bf kovalent binding}.

Hva s\aa\ med den andre ansatsen for b\o lgefunksjonen gitt ved
den antisymmetriske formen?

I Figur \ref{65} viser vi et plott av  $\psi_-({\bf r},{\bf R})$
for $R=0.1$ og $R=0.5$. For denne b\o lgefunksjonen har vi et
omr\aa de hvor de to bidragene kansellerer hverandre. For $R=0.1$ nm,
som er n\ae r likevektsavstanden for molekylet, er funskjonen
null ved $r=0$. 
Ser vi deretter p\aa\ sannsynlighetstettheten for denne ansatsen,
ser vi at for $R=0.1$ har vi null sannsynlighet for \aa\
finne elektronet. Det er vist i Figur \ref{66}.
\begin{figure}
\begin{center}
% GNUPLOT: LaTeX picture with Postscript
\begingroup%
  \makeatletter%
  \newcommand{\GNUPLOTspecial}{%
    \@sanitize\catcode`\%=14\relax\special}%
  \setlength{\unitlength}{0.1bp}%
{\GNUPLOTspecial{!
%!PS-Adobe-2.0 EPSF-2.0
%%Title: h2minus.tex
%%Creator: gnuplot 3.7 patchlevel 0.2
%%CreationDate: Wed Apr 26 15:09:15 2000
%%DocumentFonts: 
%%BoundingBox: 0 0 360 216
%%Orientation: Landscape
%%EndComments
/gnudict 256 dict def
gnudict begin
/Color false def
/Solid false def
/gnulinewidth 5.000 def
/userlinewidth gnulinewidth def
/vshift -33 def
/dl {10 mul} def
/hpt_ 31.5 def
/vpt_ 31.5 def
/hpt hpt_ def
/vpt vpt_ def
/M {moveto} bind def
/L {lineto} bind def
/R {rmoveto} bind def
/V {rlineto} bind def
/vpt2 vpt 2 mul def
/hpt2 hpt 2 mul def
/Lshow { currentpoint stroke M
  0 vshift R show } def
/Rshow { currentpoint stroke M
  dup stringwidth pop neg vshift R show } def
/Cshow { currentpoint stroke M
  dup stringwidth pop -2 div vshift R show } def
/UP { dup vpt_ mul /vpt exch def hpt_ mul /hpt exch def
  /hpt2 hpt 2 mul def /vpt2 vpt 2 mul def } def
/DL { Color {setrgbcolor Solid {pop []} if 0 setdash }
 {pop pop pop Solid {pop []} if 0 setdash} ifelse } def
/BL { stroke userlinewidth 2 mul setlinewidth } def
/AL { stroke userlinewidth 2 div setlinewidth } def
/UL { dup gnulinewidth mul /userlinewidth exch def
      10 mul /udl exch def } def
/PL { stroke userlinewidth setlinewidth } def
/LTb { BL [] 0 0 0 DL } def
/LTa { AL [1 udl mul 2 udl mul] 0 setdash 0 0 0 setrgbcolor } def
/LT0 { PL [] 1 0 0 DL } def
/LT1 { PL [4 dl 2 dl] 0 1 0 DL } def
/LT2 { PL [2 dl 3 dl] 0 0 1 DL } def
/LT3 { PL [1 dl 1.5 dl] 1 0 1 DL } def
/LT4 { PL [5 dl 2 dl 1 dl 2 dl] 0 1 1 DL } def
/LT5 { PL [4 dl 3 dl 1 dl 3 dl] 1 1 0 DL } def
/LT6 { PL [2 dl 2 dl 2 dl 4 dl] 0 0 0 DL } def
/LT7 { PL [2 dl 2 dl 2 dl 2 dl 2 dl 4 dl] 1 0.3 0 DL } def
/LT8 { PL [2 dl 2 dl 2 dl 2 dl 2 dl 2 dl 2 dl 4 dl] 0.5 0.5 0.5 DL } def
/Pnt { stroke [] 0 setdash
   gsave 1 setlinecap M 0 0 V stroke grestore } def
/Dia { stroke [] 0 setdash 2 copy vpt add M
  hpt neg vpt neg V hpt vpt neg V
  hpt vpt V hpt neg vpt V closepath stroke
  Pnt } def
/Pls { stroke [] 0 setdash vpt sub M 0 vpt2 V
  currentpoint stroke M
  hpt neg vpt neg R hpt2 0 V stroke
  } def
/Box { stroke [] 0 setdash 2 copy exch hpt sub exch vpt add M
  0 vpt2 neg V hpt2 0 V 0 vpt2 V
  hpt2 neg 0 V closepath stroke
  Pnt } def
/Crs { stroke [] 0 setdash exch hpt sub exch vpt add M
  hpt2 vpt2 neg V currentpoint stroke M
  hpt2 neg 0 R hpt2 vpt2 V stroke } def
/TriU { stroke [] 0 setdash 2 copy vpt 1.12 mul add M
  hpt neg vpt -1.62 mul V
  hpt 2 mul 0 V
  hpt neg vpt 1.62 mul V closepath stroke
  Pnt  } def
/Star { 2 copy Pls Crs } def
/BoxF { stroke [] 0 setdash exch hpt sub exch vpt add M
  0 vpt2 neg V  hpt2 0 V  0 vpt2 V
  hpt2 neg 0 V  closepath fill } def
/TriUF { stroke [] 0 setdash vpt 1.12 mul add M
  hpt neg vpt -1.62 mul V
  hpt 2 mul 0 V
  hpt neg vpt 1.62 mul V closepath fill } def
/TriD { stroke [] 0 setdash 2 copy vpt 1.12 mul sub M
  hpt neg vpt 1.62 mul V
  hpt 2 mul 0 V
  hpt neg vpt -1.62 mul V closepath stroke
  Pnt  } def
/TriDF { stroke [] 0 setdash vpt 1.12 mul sub M
  hpt neg vpt 1.62 mul V
  hpt 2 mul 0 V
  hpt neg vpt -1.62 mul V closepath fill} def
/DiaF { stroke [] 0 setdash vpt add M
  hpt neg vpt neg V hpt vpt neg V
  hpt vpt V hpt neg vpt V closepath fill } def
/Pent { stroke [] 0 setdash 2 copy gsave
  translate 0 hpt M 4 {72 rotate 0 hpt L} repeat
  closepath stroke grestore Pnt } def
/PentF { stroke [] 0 setdash gsave
  translate 0 hpt M 4 {72 rotate 0 hpt L} repeat
  closepath fill grestore } def
/Circle { stroke [] 0 setdash 2 copy
  hpt 0 360 arc stroke Pnt } def
/CircleF { stroke [] 0 setdash hpt 0 360 arc fill } def
/C0 { BL [] 0 setdash 2 copy moveto vpt 90 450  arc } bind def
/C1 { BL [] 0 setdash 2 copy        moveto
       2 copy  vpt 0 90 arc closepath fill
               vpt 0 360 arc closepath } bind def
/C2 { BL [] 0 setdash 2 copy moveto
       2 copy  vpt 90 180 arc closepath fill
               vpt 0 360 arc closepath } bind def
/C3 { BL [] 0 setdash 2 copy moveto
       2 copy  vpt 0 180 arc closepath fill
               vpt 0 360 arc closepath } bind def
/C4 { BL [] 0 setdash 2 copy moveto
       2 copy  vpt 180 270 arc closepath fill
               vpt 0 360 arc closepath } bind def
/C5 { BL [] 0 setdash 2 copy moveto
       2 copy  vpt 0 90 arc
       2 copy moveto
       2 copy  vpt 180 270 arc closepath fill
               vpt 0 360 arc } bind def
/C6 { BL [] 0 setdash 2 copy moveto
      2 copy  vpt 90 270 arc closepath fill
              vpt 0 360 arc closepath } bind def
/C7 { BL [] 0 setdash 2 copy moveto
      2 copy  vpt 0 270 arc closepath fill
              vpt 0 360 arc closepath } bind def
/C8 { BL [] 0 setdash 2 copy moveto
      2 copy vpt 270 360 arc closepath fill
              vpt 0 360 arc closepath } bind def
/C9 { BL [] 0 setdash 2 copy moveto
      2 copy  vpt 270 450 arc closepath fill
              vpt 0 360 arc closepath } bind def
/C10 { BL [] 0 setdash 2 copy 2 copy moveto vpt 270 360 arc closepath fill
       2 copy moveto
       2 copy vpt 90 180 arc closepath fill
               vpt 0 360 arc closepath } bind def
/C11 { BL [] 0 setdash 2 copy moveto
       2 copy  vpt 0 180 arc closepath fill
       2 copy moveto
       2 copy  vpt 270 360 arc closepath fill
               vpt 0 360 arc closepath } bind def
/C12 { BL [] 0 setdash 2 copy moveto
       2 copy  vpt 180 360 arc closepath fill
               vpt 0 360 arc closepath } bind def
/C13 { BL [] 0 setdash  2 copy moveto
       2 copy  vpt 0 90 arc closepath fill
       2 copy moveto
       2 copy  vpt 180 360 arc closepath fill
               vpt 0 360 arc closepath } bind def
/C14 { BL [] 0 setdash 2 copy moveto
       2 copy  vpt 90 360 arc closepath fill
               vpt 0 360 arc } bind def
/C15 { BL [] 0 setdash 2 copy vpt 0 360 arc closepath fill
               vpt 0 360 arc closepath } bind def
/Rec   { newpath 4 2 roll moveto 1 index 0 rlineto 0 exch rlineto
       neg 0 rlineto closepath } bind def
/Square { dup Rec } bind def
/Bsquare { vpt sub exch vpt sub exch vpt2 Square } bind def
/S0 { BL [] 0 setdash 2 copy moveto 0 vpt rlineto BL Bsquare } bind def
/S1 { BL [] 0 setdash 2 copy vpt Square fill Bsquare } bind def
/S2 { BL [] 0 setdash 2 copy exch vpt sub exch vpt Square fill Bsquare } bind def
/S3 { BL [] 0 setdash 2 copy exch vpt sub exch vpt2 vpt Rec fill Bsquare } bind def
/S4 { BL [] 0 setdash 2 copy exch vpt sub exch vpt sub vpt Square fill Bsquare } bind def
/S5 { BL [] 0 setdash 2 copy 2 copy vpt Square fill
       exch vpt sub exch vpt sub vpt Square fill Bsquare } bind def
/S6 { BL [] 0 setdash 2 copy exch vpt sub exch vpt sub vpt vpt2 Rec fill Bsquare } bind def
/S7 { BL [] 0 setdash 2 copy exch vpt sub exch vpt sub vpt vpt2 Rec fill
       2 copy vpt Square fill
       Bsquare } bind def
/S8 { BL [] 0 setdash 2 copy vpt sub vpt Square fill Bsquare } bind def
/S9 { BL [] 0 setdash 2 copy vpt sub vpt vpt2 Rec fill Bsquare } bind def
/S10 { BL [] 0 setdash 2 copy vpt sub vpt Square fill 2 copy exch vpt sub exch vpt Square fill
       Bsquare } bind def
/S11 { BL [] 0 setdash 2 copy vpt sub vpt Square fill 2 copy exch vpt sub exch vpt2 vpt Rec fill
       Bsquare } bind def
/S12 { BL [] 0 setdash 2 copy exch vpt sub exch vpt sub vpt2 vpt Rec fill Bsquare } bind def
/S13 { BL [] 0 setdash 2 copy exch vpt sub exch vpt sub vpt2 vpt Rec fill
       2 copy vpt Square fill Bsquare } bind def
/S14 { BL [] 0 setdash 2 copy exch vpt sub exch vpt sub vpt2 vpt Rec fill
       2 copy exch vpt sub exch vpt Square fill Bsquare } bind def
/S15 { BL [] 0 setdash 2 copy Bsquare fill Bsquare } bind def
/D0 { gsave translate 45 rotate 0 0 S0 stroke grestore } bind def
/D1 { gsave translate 45 rotate 0 0 S1 stroke grestore } bind def
/D2 { gsave translate 45 rotate 0 0 S2 stroke grestore } bind def
/D3 { gsave translate 45 rotate 0 0 S3 stroke grestore } bind def
/D4 { gsave translate 45 rotate 0 0 S4 stroke grestore } bind def
/D5 { gsave translate 45 rotate 0 0 S5 stroke grestore } bind def
/D6 { gsave translate 45 rotate 0 0 S6 stroke grestore } bind def
/D7 { gsave translate 45 rotate 0 0 S7 stroke grestore } bind def
/D8 { gsave translate 45 rotate 0 0 S8 stroke grestore } bind def
/D9 { gsave translate 45 rotate 0 0 S9 stroke grestore } bind def
/D10 { gsave translate 45 rotate 0 0 S10 stroke grestore } bind def
/D11 { gsave translate 45 rotate 0 0 S11 stroke grestore } bind def
/D12 { gsave translate 45 rotate 0 0 S12 stroke grestore } bind def
/D13 { gsave translate 45 rotate 0 0 S13 stroke grestore } bind def
/D14 { gsave translate 45 rotate 0 0 S14 stroke grestore } bind def
/D15 { gsave translate 45 rotate 0 0 S15 stroke grestore } bind def
/DiaE { stroke [] 0 setdash vpt add M
  hpt neg vpt neg V hpt vpt neg V
  hpt vpt V hpt neg vpt V closepath stroke } def
/BoxE { stroke [] 0 setdash exch hpt sub exch vpt add M
  0 vpt2 neg V hpt2 0 V 0 vpt2 V
  hpt2 neg 0 V closepath stroke } def
/TriUE { stroke [] 0 setdash vpt 1.12 mul add M
  hpt neg vpt -1.62 mul V
  hpt 2 mul 0 V
  hpt neg vpt 1.62 mul V closepath stroke } def
/TriDE { stroke [] 0 setdash vpt 1.12 mul sub M
  hpt neg vpt 1.62 mul V
  hpt 2 mul 0 V
  hpt neg vpt -1.62 mul V closepath stroke } def
/PentE { stroke [] 0 setdash gsave
  translate 0 hpt M 4 {72 rotate 0 hpt L} repeat
  closepath stroke grestore } def
/CircE { stroke [] 0 setdash 
  hpt 0 360 arc stroke } def
/Opaque { gsave closepath 1 setgray fill grestore 0 setgray closepath } def
/DiaW { stroke [] 0 setdash vpt add M
  hpt neg vpt neg V hpt vpt neg V
  hpt vpt V hpt neg vpt V Opaque stroke } def
/BoxW { stroke [] 0 setdash exch hpt sub exch vpt add M
  0 vpt2 neg V hpt2 0 V 0 vpt2 V
  hpt2 neg 0 V Opaque stroke } def
/TriUW { stroke [] 0 setdash vpt 1.12 mul add M
  hpt neg vpt -1.62 mul V
  hpt 2 mul 0 V
  hpt neg vpt 1.62 mul V Opaque stroke } def
/TriDW { stroke [] 0 setdash vpt 1.12 mul sub M
  hpt neg vpt 1.62 mul V
  hpt 2 mul 0 V
  hpt neg vpt -1.62 mul V Opaque stroke } def
/PentW { stroke [] 0 setdash gsave
  translate 0 hpt M 4 {72 rotate 0 hpt L} repeat
  Opaque stroke grestore } def
/CircW { stroke [] 0 setdash 
  hpt 0 360 arc Opaque stroke } def
/BoxFill { gsave Rec 1 setgray fill grestore } def
end
%%EndProlog
}}%
\begin{picture}(3600,2160)(0,0)%
{\GNUPLOTspecial{"
gnudict begin
gsave
0 0 translate
0.100 0.100 scale
0 setgray
newpath
1.000 UL
LTb
400 300 M
63 0 V
2987 0 R
-63 0 V
400 476 M
63 0 V
2987 0 R
-63 0 V
400 652 M
63 0 V
2987 0 R
-63 0 V
400 828 M
63 0 V
2987 0 R
-63 0 V
400 1004 M
63 0 V
2987 0 R
-63 0 V
400 1180 M
63 0 V
2987 0 R
-63 0 V
400 1356 M
63 0 V
2987 0 R
-63 0 V
400 1532 M
63 0 V
2987 0 R
-63 0 V
400 1708 M
63 0 V
2987 0 R
-63 0 V
400 1884 M
63 0 V
2987 0 R
-63 0 V
400 2060 M
63 0 V
2987 0 R
-63 0 V
705 300 M
0 63 V
0 1697 R
0 -63 V
1315 300 M
0 63 V
0 1697 R
0 -63 V
1925 300 M
0 63 V
0 1697 R
0 -63 V
2535 300 M
0 63 V
0 1697 R
0 -63 V
3145 300 M
0 63 V
0 1697 R
0 -63 V
1.000 UL
LTb
400 300 M
3050 0 V
0 1760 V
-3050 0 V
400 300 L
1.000 UL
LT0
3087 1947 M
263 0 V
400 1173 M
31 -2 V
31 -2 V
30 -2 V
31 -3 V
31 -3 V
31 -4 V
31 -5 V
30 -5 V
31 -7 V
31 -9 V
31 -10 V
31 -13 V
31 -15 V
30 -18 V
31 -22 V
31 -27 V
31 -32 V
31 -39 V
30 -47 V
31 -57 V
31 -70 V
31 -84 V
31 -101 V
30 -123 V
31 -71 V
31 135 V
31 112 V
31 92 V
30 77 V
31 63 V
31 52 V
31 43 V
31 35 V
30 30 V
31 24 V
31 20 V
31 17 V
31 14 V
31 11 V
30 10 V
31 8 V
31 6 V
31 6 V
31 5 V
30 4 V
31 4 V
31 3 V
31 3 V
31 3 V
30 2 V
31 3 V
31 3 V
31 3 V
31 4 V
30 4 V
31 5 V
31 6 V
31 6 V
31 8 V
30 10 V
31 11 V
31 14 V
31 17 V
31 20 V
31 24 V
30 30 V
31 35 V
31 43 V
31 52 V
31 63 V
30 77 V
31 92 V
31 112 V
31 135 V
31 -71 V
30 -123 V
31 -101 V
31 -84 V
31 -70 V
31 -57 V
30 -47 V
31 -39 V
31 -32 V
31 -27 V
31 -22 V
30 -18 V
31 -15 V
31 -13 V
31 -10 V
31 -9 V
31 -7 V
30 -5 V
31 -5 V
31 -4 V
31 -3 V
31 -3 V
30 -2 V
31 -2 V
31 -2 V
1.000 UL
LT1
3087 1847 M
263 0 V
400 1180 M
31 0 V
31 0 V
30 0 V
31 0 V
31 0 V
31 0 V
31 -1 V
30 0 V
31 0 V
31 0 V
31 0 V
31 0 V
31 -1 V
30 0 V
31 0 V
31 -1 V
31 -1 V
31 0 V
30 -1 V
31 -1 V
31 -2 V
31 -1 V
31 -2 V
30 -3 V
31 -3 V
31 -3 V
31 -4 V
31 -5 V
30 -7 V
31 -7 V
31 -9 V
31 -11 V
31 -13 V
30 -16 V
31 -20 V
31 -23 V
31 -29 V
31 -34 V
31 -42 V
30 -51 V
31 -61 V
31 -74 V
31 -89 V
31 -109 V
30 9 V
31 159 V
31 142 V
31 130 V
31 123 V
30 122 V
31 123 V
31 130 V
31 142 V
31 159 V
30 9 V
31 -109 V
31 -89 V
31 -74 V
31 -61 V
30 -51 V
31 -42 V
31 -34 V
31 -29 V
31 -23 V
31 -20 V
30 -16 V
31 -13 V
31 -11 V
31 -9 V
31 -7 V
30 -7 V
31 -5 V
31 -4 V
31 -3 V
31 -3 V
30 -3 V
31 -2 V
31 -1 V
31 -2 V
31 -1 V
30 -1 V
31 0 V
31 -1 V
31 -1 V
31 0 V
30 0 V
31 -1 V
31 0 V
31 0 V
31 0 V
31 0 V
30 0 V
31 -1 V
31 0 V
31 0 V
31 0 V
30 0 V
31 0 V
31 0 V
stroke
grestore
end
showpage
}}%
\put(3037,1847){\makebox(0,0)[r]{$R=0.1$ nm}}%
\put(3037,1947){\makebox(0,0)[r]{$R=0.5$ nm}}%
\put(1925,50){\makebox(0,0){$r$ [nm]}}%
\put(100,1180){%
\special{ps: gsave currentpoint currentpoint translate
270 rotate neg exch neg exch translate}%
\makebox(0,0)[b]{\shortstack{$\psi_{-}({\bf r},{\bf R})$}}%
\special{ps: currentpoint grestore moveto}%
}%
\put(3145,200){\makebox(0,0){0.4}}%
\put(2535,200){\makebox(0,0){0.2}}%
\put(1925,200){\makebox(0,0){0}}%
\put(1315,200){\makebox(0,0){-0.2}}%
\put(705,200){\makebox(0,0){-0.4}}%
\put(350,2060){\makebox(0,0)[r]{50}}%
\put(350,1884){\makebox(0,0)[r]{40}}%
\put(350,1708){\makebox(0,0)[r]{30}}%
\put(350,1532){\makebox(0,0)[r]{20}}%
\put(350,1356){\makebox(0,0)[r]{10}}%
\put(350,1180){\makebox(0,0)[r]{0}}%
\put(350,1004){\makebox(0,0)[r]{-10}}%
\put(350,828){\makebox(0,0)[r]{-20}}%
\put(350,652){\makebox(0,0)[r]{-30}}%
\put(350,476){\makebox(0,0)[r]{-40}}%
\put(350,300){\makebox(0,0)[r]{-50}}%
\end{picture}%
\endgroup
\endinput

\end{center}
\caption{Plott av  $\psi_-({\bf r},{\bf R})$ for ulike verdier av
$|{\bf R}|$. Merk at funksjonene ikke er normaliserte. Enheten p\aa\
$y$-aksen er vilk\aa rlig.\label{65}}
\end{figure} 
\begin{figure}
\begin{center}
% GNUPLOT: LaTeX picture with Postscript
\begingroup%
  \makeatletter%
  \newcommand{\GNUPLOTspecial}{%
    \@sanitize\catcode`\%=14\relax\special}%
  \setlength{\unitlength}{0.1bp}%
{\GNUPLOTspecial{!
%!PS-Adobe-2.0 EPSF-2.0
%%Title: p2minus.tex
%%Creator: gnuplot 3.7 patchlevel 0.2
%%CreationDate: Wed Apr 26 15:13:39 2000
%%DocumentFonts: 
%%BoundingBox: 0 0 360 216
%%Orientation: Landscape
%%EndComments
/gnudict 256 dict def
gnudict begin
/Color false def
/Solid false def
/gnulinewidth 5.000 def
/userlinewidth gnulinewidth def
/vshift -33 def
/dl {10 mul} def
/hpt_ 31.5 def
/vpt_ 31.5 def
/hpt hpt_ def
/vpt vpt_ def
/M {moveto} bind def
/L {lineto} bind def
/R {rmoveto} bind def
/V {rlineto} bind def
/vpt2 vpt 2 mul def
/hpt2 hpt 2 mul def
/Lshow { currentpoint stroke M
  0 vshift R show } def
/Rshow { currentpoint stroke M
  dup stringwidth pop neg vshift R show } def
/Cshow { currentpoint stroke M
  dup stringwidth pop -2 div vshift R show } def
/UP { dup vpt_ mul /vpt exch def hpt_ mul /hpt exch def
  /hpt2 hpt 2 mul def /vpt2 vpt 2 mul def } def
/DL { Color {setrgbcolor Solid {pop []} if 0 setdash }
 {pop pop pop Solid {pop []} if 0 setdash} ifelse } def
/BL { stroke userlinewidth 2 mul setlinewidth } def
/AL { stroke userlinewidth 2 div setlinewidth } def
/UL { dup gnulinewidth mul /userlinewidth exch def
      10 mul /udl exch def } def
/PL { stroke userlinewidth setlinewidth } def
/LTb { BL [] 0 0 0 DL } def
/LTa { AL [1 udl mul 2 udl mul] 0 setdash 0 0 0 setrgbcolor } def
/LT0 { PL [] 1 0 0 DL } def
/LT1 { PL [4 dl 2 dl] 0 1 0 DL } def
/LT2 { PL [2 dl 3 dl] 0 0 1 DL } def
/LT3 { PL [1 dl 1.5 dl] 1 0 1 DL } def
/LT4 { PL [5 dl 2 dl 1 dl 2 dl] 0 1 1 DL } def
/LT5 { PL [4 dl 3 dl 1 dl 3 dl] 1 1 0 DL } def
/LT6 { PL [2 dl 2 dl 2 dl 4 dl] 0 0 0 DL } def
/LT7 { PL [2 dl 2 dl 2 dl 2 dl 2 dl 4 dl] 1 0.3 0 DL } def
/LT8 { PL [2 dl 2 dl 2 dl 2 dl 2 dl 2 dl 2 dl 4 dl] 0.5 0.5 0.5 DL } def
/Pnt { stroke [] 0 setdash
   gsave 1 setlinecap M 0 0 V stroke grestore } def
/Dia { stroke [] 0 setdash 2 copy vpt add M
  hpt neg vpt neg V hpt vpt neg V
  hpt vpt V hpt neg vpt V closepath stroke
  Pnt } def
/Pls { stroke [] 0 setdash vpt sub M 0 vpt2 V
  currentpoint stroke M
  hpt neg vpt neg R hpt2 0 V stroke
  } def
/Box { stroke [] 0 setdash 2 copy exch hpt sub exch vpt add M
  0 vpt2 neg V hpt2 0 V 0 vpt2 V
  hpt2 neg 0 V closepath stroke
  Pnt } def
/Crs { stroke [] 0 setdash exch hpt sub exch vpt add M
  hpt2 vpt2 neg V currentpoint stroke M
  hpt2 neg 0 R hpt2 vpt2 V stroke } def
/TriU { stroke [] 0 setdash 2 copy vpt 1.12 mul add M
  hpt neg vpt -1.62 mul V
  hpt 2 mul 0 V
  hpt neg vpt 1.62 mul V closepath stroke
  Pnt  } def
/Star { 2 copy Pls Crs } def
/BoxF { stroke [] 0 setdash exch hpt sub exch vpt add M
  0 vpt2 neg V  hpt2 0 V  0 vpt2 V
  hpt2 neg 0 V  closepath fill } def
/TriUF { stroke [] 0 setdash vpt 1.12 mul add M
  hpt neg vpt -1.62 mul V
  hpt 2 mul 0 V
  hpt neg vpt 1.62 mul V closepath fill } def
/TriD { stroke [] 0 setdash 2 copy vpt 1.12 mul sub M
  hpt neg vpt 1.62 mul V
  hpt 2 mul 0 V
  hpt neg vpt -1.62 mul V closepath stroke
  Pnt  } def
/TriDF { stroke [] 0 setdash vpt 1.12 mul sub M
  hpt neg vpt 1.62 mul V
  hpt 2 mul 0 V
  hpt neg vpt -1.62 mul V closepath fill} def
/DiaF { stroke [] 0 setdash vpt add M
  hpt neg vpt neg V hpt vpt neg V
  hpt vpt V hpt neg vpt V closepath fill } def
/Pent { stroke [] 0 setdash 2 copy gsave
  translate 0 hpt M 4 {72 rotate 0 hpt L} repeat
  closepath stroke grestore Pnt } def
/PentF { stroke [] 0 setdash gsave
  translate 0 hpt M 4 {72 rotate 0 hpt L} repeat
  closepath fill grestore } def
/Circle { stroke [] 0 setdash 2 copy
  hpt 0 360 arc stroke Pnt } def
/CircleF { stroke [] 0 setdash hpt 0 360 arc fill } def
/C0 { BL [] 0 setdash 2 copy moveto vpt 90 450  arc } bind def
/C1 { BL [] 0 setdash 2 copy        moveto
       2 copy  vpt 0 90 arc closepath fill
               vpt 0 360 arc closepath } bind def
/C2 { BL [] 0 setdash 2 copy moveto
       2 copy  vpt 90 180 arc closepath fill
               vpt 0 360 arc closepath } bind def
/C3 { BL [] 0 setdash 2 copy moveto
       2 copy  vpt 0 180 arc closepath fill
               vpt 0 360 arc closepath } bind def
/C4 { BL [] 0 setdash 2 copy moveto
       2 copy  vpt 180 270 arc closepath fill
               vpt 0 360 arc closepath } bind def
/C5 { BL [] 0 setdash 2 copy moveto
       2 copy  vpt 0 90 arc
       2 copy moveto
       2 copy  vpt 180 270 arc closepath fill
               vpt 0 360 arc } bind def
/C6 { BL [] 0 setdash 2 copy moveto
      2 copy  vpt 90 270 arc closepath fill
              vpt 0 360 arc closepath } bind def
/C7 { BL [] 0 setdash 2 copy moveto
      2 copy  vpt 0 270 arc closepath fill
              vpt 0 360 arc closepath } bind def
/C8 { BL [] 0 setdash 2 copy moveto
      2 copy vpt 270 360 arc closepath fill
              vpt 0 360 arc closepath } bind def
/C9 { BL [] 0 setdash 2 copy moveto
      2 copy  vpt 270 450 arc closepath fill
              vpt 0 360 arc closepath } bind def
/C10 { BL [] 0 setdash 2 copy 2 copy moveto vpt 270 360 arc closepath fill
       2 copy moveto
       2 copy vpt 90 180 arc closepath fill
               vpt 0 360 arc closepath } bind def
/C11 { BL [] 0 setdash 2 copy moveto
       2 copy  vpt 0 180 arc closepath fill
       2 copy moveto
       2 copy  vpt 270 360 arc closepath fill
               vpt 0 360 arc closepath } bind def
/C12 { BL [] 0 setdash 2 copy moveto
       2 copy  vpt 180 360 arc closepath fill
               vpt 0 360 arc closepath } bind def
/C13 { BL [] 0 setdash  2 copy moveto
       2 copy  vpt 0 90 arc closepath fill
       2 copy moveto
       2 copy  vpt 180 360 arc closepath fill
               vpt 0 360 arc closepath } bind def
/C14 { BL [] 0 setdash 2 copy moveto
       2 copy  vpt 90 360 arc closepath fill
               vpt 0 360 arc } bind def
/C15 { BL [] 0 setdash 2 copy vpt 0 360 arc closepath fill
               vpt 0 360 arc closepath } bind def
/Rec   { newpath 4 2 roll moveto 1 index 0 rlineto 0 exch rlineto
       neg 0 rlineto closepath } bind def
/Square { dup Rec } bind def
/Bsquare { vpt sub exch vpt sub exch vpt2 Square } bind def
/S0 { BL [] 0 setdash 2 copy moveto 0 vpt rlineto BL Bsquare } bind def
/S1 { BL [] 0 setdash 2 copy vpt Square fill Bsquare } bind def
/S2 { BL [] 0 setdash 2 copy exch vpt sub exch vpt Square fill Bsquare } bind def
/S3 { BL [] 0 setdash 2 copy exch vpt sub exch vpt2 vpt Rec fill Bsquare } bind def
/S4 { BL [] 0 setdash 2 copy exch vpt sub exch vpt sub vpt Square fill Bsquare } bind def
/S5 { BL [] 0 setdash 2 copy 2 copy vpt Square fill
       exch vpt sub exch vpt sub vpt Square fill Bsquare } bind def
/S6 { BL [] 0 setdash 2 copy exch vpt sub exch vpt sub vpt vpt2 Rec fill Bsquare } bind def
/S7 { BL [] 0 setdash 2 copy exch vpt sub exch vpt sub vpt vpt2 Rec fill
       2 copy vpt Square fill
       Bsquare } bind def
/S8 { BL [] 0 setdash 2 copy vpt sub vpt Square fill Bsquare } bind def
/S9 { BL [] 0 setdash 2 copy vpt sub vpt vpt2 Rec fill Bsquare } bind def
/S10 { BL [] 0 setdash 2 copy vpt sub vpt Square fill 2 copy exch vpt sub exch vpt Square fill
       Bsquare } bind def
/S11 { BL [] 0 setdash 2 copy vpt sub vpt Square fill 2 copy exch vpt sub exch vpt2 vpt Rec fill
       Bsquare } bind def
/S12 { BL [] 0 setdash 2 copy exch vpt sub exch vpt sub vpt2 vpt Rec fill Bsquare } bind def
/S13 { BL [] 0 setdash 2 copy exch vpt sub exch vpt sub vpt2 vpt Rec fill
       2 copy vpt Square fill Bsquare } bind def
/S14 { BL [] 0 setdash 2 copy exch vpt sub exch vpt sub vpt2 vpt Rec fill
       2 copy exch vpt sub exch vpt Square fill Bsquare } bind def
/S15 { BL [] 0 setdash 2 copy Bsquare fill Bsquare } bind def
/D0 { gsave translate 45 rotate 0 0 S0 stroke grestore } bind def
/D1 { gsave translate 45 rotate 0 0 S1 stroke grestore } bind def
/D2 { gsave translate 45 rotate 0 0 S2 stroke grestore } bind def
/D3 { gsave translate 45 rotate 0 0 S3 stroke grestore } bind def
/D4 { gsave translate 45 rotate 0 0 S4 stroke grestore } bind def
/D5 { gsave translate 45 rotate 0 0 S5 stroke grestore } bind def
/D6 { gsave translate 45 rotate 0 0 S6 stroke grestore } bind def
/D7 { gsave translate 45 rotate 0 0 S7 stroke grestore } bind def
/D8 { gsave translate 45 rotate 0 0 S8 stroke grestore } bind def
/D9 { gsave translate 45 rotate 0 0 S9 stroke grestore } bind def
/D10 { gsave translate 45 rotate 0 0 S10 stroke grestore } bind def
/D11 { gsave translate 45 rotate 0 0 S11 stroke grestore } bind def
/D12 { gsave translate 45 rotate 0 0 S12 stroke grestore } bind def
/D13 { gsave translate 45 rotate 0 0 S13 stroke grestore } bind def
/D14 { gsave translate 45 rotate 0 0 S14 stroke grestore } bind def
/D15 { gsave translate 45 rotate 0 0 S15 stroke grestore } bind def
/DiaE { stroke [] 0 setdash vpt add M
  hpt neg vpt neg V hpt vpt neg V
  hpt vpt V hpt neg vpt V closepath stroke } def
/BoxE { stroke [] 0 setdash exch hpt sub exch vpt add M
  0 vpt2 neg V hpt2 0 V 0 vpt2 V
  hpt2 neg 0 V closepath stroke } def
/TriUE { stroke [] 0 setdash vpt 1.12 mul add M
  hpt neg vpt -1.62 mul V
  hpt 2 mul 0 V
  hpt neg vpt 1.62 mul V closepath stroke } def
/TriDE { stroke [] 0 setdash vpt 1.12 mul sub M
  hpt neg vpt 1.62 mul V
  hpt 2 mul 0 V
  hpt neg vpt -1.62 mul V closepath stroke } def
/PentE { stroke [] 0 setdash gsave
  translate 0 hpt M 4 {72 rotate 0 hpt L} repeat
  closepath stroke grestore } def
/CircE { stroke [] 0 setdash 
  hpt 0 360 arc stroke } def
/Opaque { gsave closepath 1 setgray fill grestore 0 setgray closepath } def
/DiaW { stroke [] 0 setdash vpt add M
  hpt neg vpt neg V hpt vpt neg V
  hpt vpt V hpt neg vpt V Opaque stroke } def
/BoxW { stroke [] 0 setdash exch hpt sub exch vpt add M
  0 vpt2 neg V hpt2 0 V 0 vpt2 V
  hpt2 neg 0 V Opaque stroke } def
/TriUW { stroke [] 0 setdash vpt 1.12 mul add M
  hpt neg vpt -1.62 mul V
  hpt 2 mul 0 V
  hpt neg vpt 1.62 mul V Opaque stroke } def
/TriDW { stroke [] 0 setdash vpt 1.12 mul sub M
  hpt neg vpt 1.62 mul V
  hpt 2 mul 0 V
  hpt neg vpt -1.62 mul V Opaque stroke } def
/PentW { stroke [] 0 setdash gsave
  translate 0 hpt M 4 {72 rotate 0 hpt L} repeat
  Opaque stroke grestore } def
/CircW { stroke [] 0 setdash 
  hpt 0 360 arc Opaque stroke } def
/BoxFill { gsave Rec 1 setgray fill grestore } def
end
%%EndProlog
}}%
\begin{picture}(3600,2160)(0,0)%
{\GNUPLOTspecial{"
gnudict begin
gsave
0 0 translate
0.100 0.100 scale
0 setgray
newpath
1.000 UL
LTb
450 300 M
63 0 V
2937 0 R
-63 0 V
450 476 M
63 0 V
2937 0 R
-63 0 V
450 652 M
63 0 V
2937 0 R
-63 0 V
450 828 M
63 0 V
2937 0 R
-63 0 V
450 1004 M
63 0 V
2937 0 R
-63 0 V
450 1180 M
63 0 V
2937 0 R
-63 0 V
450 1356 M
63 0 V
2937 0 R
-63 0 V
450 1532 M
63 0 V
2937 0 R
-63 0 V
450 1708 M
63 0 V
2937 0 R
-63 0 V
450 1884 M
63 0 V
2937 0 R
-63 0 V
450 2060 M
63 0 V
2937 0 R
-63 0 V
750 300 M
0 63 V
0 1697 R
0 -63 V
1350 300 M
0 63 V
0 1697 R
0 -63 V
1950 300 M
0 63 V
0 1697 R
0 -63 V
2550 300 M
0 63 V
0 1697 R
0 -63 V
3150 300 M
0 63 V
0 1697 R
0 -63 V
1.000 UL
LTb
450 300 M
3000 0 V
0 1760 V
-3000 0 V
450 300 L
1.000 UL
LT0
3087 1947 M
263 0 V
450 300 M
30 0 V
31 0 V
30 0 V
30 1 V
31 0 V
30 0 V
30 1 V
30 1 V
31 2 V
30 2 V
30 3 V
31 5 V
30 6 V
30 10 V
31 15 V
30 21 V
30 31 V
30 46 V
31 66 V
30 98 V
30 144 V
31 210 V
30 308 V
30 451 V
31 299 V
30 -546 V
30 -373 V
30 -254 V
31 -174 V
30 -118 V
30 -81 V
31 -56 V
30 -37 V
30 -26 V
31 -18 V
30 -12 V
30 -8 V
31 -5 V
30 -4 V
30 -3 V
30 -1 V
31 -2 V
30 -1 V
30 0 V
31 0 V
30 -1 V
30 0 V
31 0 V
30 0 V
30 0 V
30 0 V
31 0 V
30 0 V
30 1 V
31 0 V
30 0 V
30 1 V
31 2 V
30 1 V
30 3 V
30 4 V
31 5 V
30 8 V
30 12 V
31 18 V
30 26 V
30 37 V
31 56 V
30 81 V
30 118 V
31 174 V
30 254 V
30 373 V
30 546 V
31 -299 V
30 -451 V
30 -308 V
31 -210 V
30 -144 V
30 -98 V
31 -66 V
30 -46 V
30 -31 V
30 -21 V
31 -15 V
30 -10 V
30 -6 V
31 -5 V
30 -3 V
30 -2 V
31 -2 V
30 -1 V
30 -1 V
30 0 V
31 0 V
30 -1 V
30 0 V
31 0 V
30 0 V
1.000 UL
LT1
3087 1847 M
263 0 V
450 300 M
30 0 V
31 0 V
30 0 V
30 0 V
31 0 V
30 0 V
30 0 V
30 0 V
31 0 V
30 0 V
30 0 V
31 0 V
30 0 V
30 0 V
31 0 V
30 0 V
30 0 V
30 0 V
31 0 V
30 0 V
30 0 V
31 0 V
30 0 V
30 1 V
31 0 V
30 0 V
30 1 V
30 0 V
31 2 V
30 1 V
30 3 V
31 3 V
30 6 V
30 7 V
31 12 V
30 16 V
30 24 V
31 36 V
30 52 V
30 76 V
30 112 V
31 163 V
30 240 V
30 350 V
31 -32 V
30 -482 V
30 -310 V
31 -185 V
30 -86 V
30 0 V
30 86 V
31 185 V
30 310 V
30 482 V
31 32 V
30 -350 V
30 -240 V
31 -163 V
30 -112 V
30 -76 V
30 -52 V
31 -36 V
30 -24 V
30 -16 V
31 -12 V
30 -7 V
30 -6 V
31 -3 V
30 -3 V
30 -1 V
31 -2 V
30 0 V
30 -1 V
30 0 V
31 0 V
30 -1 V
30 0 V
31 0 V
30 0 V
30 0 V
31 0 V
30 0 V
30 0 V
30 0 V
31 0 V
30 0 V
30 0 V
31 0 V
30 0 V
30 0 V
31 0 V
30 0 V
30 0 V
30 0 V
31 0 V
30 0 V
30 0 V
31 0 V
30 0 V
stroke
grestore
end
showpage
}}%
\put(3037,1847){\makebox(0,0)[r]{$R=0.1$ nm}}%
\put(3037,1947){\makebox(0,0)[r]{$R=0.5$ nm}}%
\put(1950,50){\makebox(0,0){$r$ [nm]}}%
\put(100,1180){%
\special{ps: gsave currentpoint currentpoint translate
270 rotate neg exch neg exch translate}%
\makebox(0,0)[b]{\shortstack{$|\psi_-({\bf r},{\bf R})|^2$}}%
\special{ps: currentpoint grestore moveto}%
}%
\put(3150,200){\makebox(0,0){0.4}}%
\put(2550,200){\makebox(0,0){0.2}}%
\put(1950,200){\makebox(0,0){0}}%
\put(1350,200){\makebox(0,0){-0.2}}%
\put(750,200){\makebox(0,0){-0.4}}%
\put(400,2060){\makebox(0,0)[r]{2000}}%
\put(400,1884){\makebox(0,0)[r]{1800}}%
\put(400,1708){\makebox(0,0)[r]{1600}}%
\put(400,1532){\makebox(0,0)[r]{1400}}%
\put(400,1356){\makebox(0,0)[r]{1200}}%
\put(400,1180){\makebox(0,0)[r]{1000}}%
\put(400,1004){\makebox(0,0)[r]{800}}%
\put(400,828){\makebox(0,0)[r]{600}}%
\put(400,652){\makebox(0,0)[r]{400}}%
\put(400,476){\makebox(0,0)[r]{200}}%
\put(400,300){\makebox(0,0)[r]{0}}%
\end{picture}%
\endgroup
\endinput

\end{center}
\caption{Plott av  $|\psi_-({\bf r},{\bf R})|^2$ for ulike verdier av
$|{\bf R}|$. Merk at funksjonene ikke er normaliserte. Enheten p\aa\
$y$-aksen er vilk\aa rlig.\label{66}}
\end{figure} 

Det betyr at vi har et omr\aa de mellom protonene hvor det ikke
er noen sannsynlighet for \aa\ finne elektronet. Fysisk vil det si 
at protonene ser kun deres egne ladninger i dette omr\aa det,
noe som igjen forklarer hvorfor den antisymmetriske ansatsen
ikke gir binding. 

La oss n\aa\ bruke disse to b\o lgefunksjonene og vise at
vi faktisk kan rekne ut bindingsenergien.

F\o rst trenger vi \aa\  finne normeringsfaktorene $C_{\pm}$ ved \aa\ rekne ut
\be
   \int C_{\pm}^2\left(\psi_1({\bf r},{\bf R})\pm\psi_2({\bf r},{\bf R})\right)^*\left(\psi_1({\bf r},{\bf R})\pm\psi_2({\bf r},{\bf R})\right)d^3r =1.
\ee
Det g\aa r an \aa\ vise at\footnote{Her gir vi kun resultatet!} at
\be
   C_{\pm}^2=\frac{1}{2(1\pm \Delta (R))},
\ee
med 
\be
   \Delta (R)=\int \psi_1({\bf r},{\bf R})^*\psi_2({\bf r},{\bf R})d^3r=\left(1+\frac{R}{a_0}
   +\frac{R^2}{3a_0^2}\right)e^{-R/a_0}.
\ee

Vi kan dermed rekne ut forventningsverdien av energien gitt ved
\be
   \langle \OP{H} \rangle_{\pm} = 
    \frac{\int \psi_{\pm}^*({\bf r},{\bf R})\OP{H}\psi_{\pm}({\bf r},{\bf R})d^3r}{2(1\pm \Delta (R))},
\ee
med 
\be
   \OP{H}=-\frac{\hbar^2\nabla_r^2}{2m_e}
     -\frac{ke^2}{|{\bf r}- {\bf R}/2|}-\frac{ke^2}{|{\bf r}+ {\bf R}/2|}
     +\frac{ke^2}{R}.
\ee
Skriver vi ut alle bidragene f\aa r vi fire ledd, nemlig
\begin{eqnarray}
   \langle \OP{H} \rangle_{\pm}& =& 
    \frac{1}{2(1\pm \Delta (R))}
    \left\{\int \psi_{1}^*({\bf r},{\bf R})\OP{H}\psi_{1}({\bf r},{\bf R})d^3r
    +\int \psi_{2}^*({\bf r},{\bf R})\OP{H}\psi_{2}({\bf r},{\bf R})d^3r\right. \nonumber \\
  & & \left. 
   \pm \int \psi_{1}^*({\bf r},{\bf R})\OP{H}\psi_{2}({\bf r},{\bf R})d^3r
   \pm \int \psi_{2}^*({\bf r},{\bf R})\OP{H}\psi_{1}({\bf r},{\bf R})d^3r
    \right\},
\end{eqnarray}
som igjen kan forenkles dersom vi bruker at $\OP{H}$ og b\o lgenfunksjonene
$\psi_{\pm}$ forandres ikke ved 
ombyttene ${\bf R}\rightarrow -{\bf R}$
og ${\bf r}\rightarrow -{\bf r}$, til
\be
   \langle \OP{H} \rangle_{\pm} = 
    \frac{1}{(1\pm \Delta (R))}
    \left\{\int \psi_{1}^*({\bf r},{\bf R})\OP{H}\psi_{1}({\bf r},{\bf R})d^3r
   \pm \int \psi_{2}^*({\bf r},{\bf R})\OP{H}\psi_{1}({\bf r},{\bf R})d^3r
    \right\}.
\ee

La oss droppe normeringsfaktoren i f\o rste omgang og se f\o rst p\aa\ 
\begin{eqnarray}
  \int \psi_{1}^*({\bf r},{\bf R})\OP{H}\psi_{1}({\bf r},{\bf R})d^3r
=&& \nonumber \\
 \int \psi_{1}^*({\bf r},{\bf R})\left\{-\frac{\hbar^2\nabla_r^2}{2m_e}
     -\frac{ke^2}{|{\bf r}- {\bf R}/2|}+\frac{ke^2}{R}-\frac{ke^2}{|{\bf r}+ {\bf R}/2|}
     \right\}\psi_{1}({\bf r},{\bf R})d^3r. & &
\end{eqnarray}
De to f\o rste leddene gir oss bindingsenergien til hydrogenatomet.
Det tredje leddet er uavhengig av integrasjonsvariabelen $r$ og gir
kun $ke^2/R$. Det siste leddet skyldes tiltrekningen fra det andre
protonet slik at dette bidraget til energien gir oss
\be
  \int \psi_{1}^*({\bf r},{\bf R})\OP{H}\psi_{1}({\bf r},{\bf R})d^3r=
   E_0+\frac{ke^2}{R}-\int \psi_{1}^*({\bf r},{\bf R})\frac{ke^2}{|{\bf r}+ {\bf R}/2|}\psi_{1}({\bf r},{\bf R})d^3r.
\ee
Det siste leddet kan ogs\aa\ reknes ut slik at vi f\aa r
\be
     \int \psi_{1}^*({\bf r},{\bf R})\OP{H}\psi_{1}({\bf r},{\bf R})d^3r=
   E_0+\frac{ke^2}{R}\left(1+\frac{R}{a_0}\right)e^{-2R/a_0}.
\ee
Helt tilsvarende kan vi rekne ut 
\begin{eqnarray}
  \int \psi_{2}^*({\bf r},{\bf R})\OP{H}\psi_{1}({\bf r},{\bf R})d^3r
=S(R)=& & \nonumber \\
 \int \psi_{2}^*({\bf r},{\bf R})\left\{-\frac{\hbar^2\nabla_r^2}{2m_e}
     -\frac{ke^2}{|{\bf r}- {\bf R}/2|}+\frac{ke^2}{R}-\frac{ke^2}{|{\bf r}+ {\bf R}/2|}
     \right\}\psi_{1}({\bf r},{\bf R})d^3r, & &
\end{eqnarray}
som gir
\be
     S(R)
=\left(E_0+\frac{ke^2}{R}\right)\Delta (R)-\int \psi_{2}^*({\bf r},{\bf R})\frac{ke^2}{|{\bf r}+ {\bf R}/2|}\psi_{1}({\bf r},{\bf R})d^3r. 
\ee

Det siste leddet kan ogs\aa\ reknes ut. Utrekning gir
\be
\int \psi_{2}^*({\bf r},{\bf R})\frac{ke^2}{|{\bf r}+ {\bf R}/2|}\psi_{1}({\bf r},{\bf R})d^3r=\frac{ke^2}{a_0}\left(1+\frac{R}{a_0}\right)e^{-R/a_0},
\ee
slik at
funksjonen $S(R)$ er gitt ved
\be
  S(R)=   \left(E_0+\frac{ke^2}{R}\right)\Delta (R)-\frac{ke^2}{a_0}\left(1+\frac{R}{a_0}\right)e^{-R/a_0}.
\ee
Merk at $S(R)$ er tiltrekkende, negativt fortegn.
 
Legger vi alle bitene sammen finner vi at energien er gitt ved
\be
   \langle \OP{H} \rangle_{+}=    \frac{1}{(1+ \Delta (R))}
    \left\{E_0+\frac{ke^2}{R}\left(1+\frac{R}{a_0}\right)e^{-2R/a_0}+S(R)\right\},
\ee
og 
\be
   \langle \OP{H} \rangle_{-}= \frac{1}{(1-\Delta (R))}
    \left\{E_0+\frac{ke^2}{R}\left(1+\frac{R}{a_0}\right)e^{-2R/a_0}-S(R)\right\}.
\ee
Hvordan tolker vi leddet $S(R)$? Dette leddet er et kryssledd da det inneholder
b\aa de $\psi_{2}({\bf r},{\bf R})$ og  $\psi_{1}({\bf r},{\bf R})$, og 
uttrykker muligheten for at elektronet kan v\ae re b\aa de ved det 
ene og det 
andre protonet. Det er dette leddet som gir opphav til binding eller ikke.
Siden $S(R)$ er tiltrekkende (negativt fortegn), f\aa r vi mer binding
for den symmetriske b\o lgenfunksjonen. For den antisymmetriske 
b\o lgefunksjonen blir bidraget fra $S(R)$ repulsivt, og gir ikke
binding.  N\aa r vi tenker oss at elektronet kan v\ae re begge steder,
er det b\o lgebildet vi bruker.
Dette kan vi igjen forst\aa\ ved \aa\ g\aa\ tilbake til Figurene \ref{64}
og \ref{66}
for sannsynlighetstetthetene. For den positive b\o lgefunksjonen
har vi en negativ ladningsfordeling mellom protonene. Denne ladningsfordelingen
binder protonene til elektronet. For den antisymmetriske b\o lgefunksjonen
har vi ingen negativ ladningsfordeling mellom protonene, og protonene f\o ler
dermed ei frast\o ting med det resultat at molekylet ikke er bundet.
Matematisk gjenspeiles dette i energien via  st\o rrelsen $S(R)$.
\begin{figure}
\begin{center}
% GNUPLOT: LaTeX picture with Postscript
\begingroup%
  \makeatletter%
  \newcommand{\GNUPLOTspecial}{%
    \@sanitize\catcode`\%=14\relax\special}%
  \setlength{\unitlength}{0.1bp}%
{\GNUPLOTspecial{!
%!PS-Adobe-2.0 EPSF-2.0
%%Title: molbinding.tex
%%Creator: gnuplot 3.7 patchlevel 0.2
%%CreationDate: Thu Apr 27 15:55:25 2000
%%DocumentFonts: 
%%BoundingBox: 0 0 360 216
%%Orientation: Landscape
%%EndComments
/gnudict 256 dict def
gnudict begin
/Color false def
/Solid false def
/gnulinewidth 5.000 def
/userlinewidth gnulinewidth def
/vshift -33 def
/dl {10 mul} def
/hpt_ 31.5 def
/vpt_ 31.5 def
/hpt hpt_ def
/vpt vpt_ def
/M {moveto} bind def
/L {lineto} bind def
/R {rmoveto} bind def
/V {rlineto} bind def
/vpt2 vpt 2 mul def
/hpt2 hpt 2 mul def
/Lshow { currentpoint stroke M
  0 vshift R show } def
/Rshow { currentpoint stroke M
  dup stringwidth pop neg vshift R show } def
/Cshow { currentpoint stroke M
  dup stringwidth pop -2 div vshift R show } def
/UP { dup vpt_ mul /vpt exch def hpt_ mul /hpt exch def
  /hpt2 hpt 2 mul def /vpt2 vpt 2 mul def } def
/DL { Color {setrgbcolor Solid {pop []} if 0 setdash }
 {pop pop pop Solid {pop []} if 0 setdash} ifelse } def
/BL { stroke userlinewidth 2 mul setlinewidth } def
/AL { stroke userlinewidth 2 div setlinewidth } def
/UL { dup gnulinewidth mul /userlinewidth exch def
      10 mul /udl exch def } def
/PL { stroke userlinewidth setlinewidth } def
/LTb { BL [] 0 0 0 DL } def
/LTa { AL [1 udl mul 2 udl mul] 0 setdash 0 0 0 setrgbcolor } def
/LT0 { PL [] 1 0 0 DL } def
/LT1 { PL [4 dl 2 dl] 0 1 0 DL } def
/LT2 { PL [2 dl 3 dl] 0 0 1 DL } def
/LT3 { PL [1 dl 1.5 dl] 1 0 1 DL } def
/LT4 { PL [5 dl 2 dl 1 dl 2 dl] 0 1 1 DL } def
/LT5 { PL [4 dl 3 dl 1 dl 3 dl] 1 1 0 DL } def
/LT6 { PL [2 dl 2 dl 2 dl 4 dl] 0 0 0 DL } def
/LT7 { PL [2 dl 2 dl 2 dl 2 dl 2 dl 4 dl] 1 0.3 0 DL } def
/LT8 { PL [2 dl 2 dl 2 dl 2 dl 2 dl 2 dl 2 dl 4 dl] 0.5 0.5 0.5 DL } def
/Pnt { stroke [] 0 setdash
   gsave 1 setlinecap M 0 0 V stroke grestore } def
/Dia { stroke [] 0 setdash 2 copy vpt add M
  hpt neg vpt neg V hpt vpt neg V
  hpt vpt V hpt neg vpt V closepath stroke
  Pnt } def
/Pls { stroke [] 0 setdash vpt sub M 0 vpt2 V
  currentpoint stroke M
  hpt neg vpt neg R hpt2 0 V stroke
  } def
/Box { stroke [] 0 setdash 2 copy exch hpt sub exch vpt add M
  0 vpt2 neg V hpt2 0 V 0 vpt2 V
  hpt2 neg 0 V closepath stroke
  Pnt } def
/Crs { stroke [] 0 setdash exch hpt sub exch vpt add M
  hpt2 vpt2 neg V currentpoint stroke M
  hpt2 neg 0 R hpt2 vpt2 V stroke } def
/TriU { stroke [] 0 setdash 2 copy vpt 1.12 mul add M
  hpt neg vpt -1.62 mul V
  hpt 2 mul 0 V
  hpt neg vpt 1.62 mul V closepath stroke
  Pnt  } def
/Star { 2 copy Pls Crs } def
/BoxF { stroke [] 0 setdash exch hpt sub exch vpt add M
  0 vpt2 neg V  hpt2 0 V  0 vpt2 V
  hpt2 neg 0 V  closepath fill } def
/TriUF { stroke [] 0 setdash vpt 1.12 mul add M
  hpt neg vpt -1.62 mul V
  hpt 2 mul 0 V
  hpt neg vpt 1.62 mul V closepath fill } def
/TriD { stroke [] 0 setdash 2 copy vpt 1.12 mul sub M
  hpt neg vpt 1.62 mul V
  hpt 2 mul 0 V
  hpt neg vpt -1.62 mul V closepath stroke
  Pnt  } def
/TriDF { stroke [] 0 setdash vpt 1.12 mul sub M
  hpt neg vpt 1.62 mul V
  hpt 2 mul 0 V
  hpt neg vpt -1.62 mul V closepath fill} def
/DiaF { stroke [] 0 setdash vpt add M
  hpt neg vpt neg V hpt vpt neg V
  hpt vpt V hpt neg vpt V closepath fill } def
/Pent { stroke [] 0 setdash 2 copy gsave
  translate 0 hpt M 4 {72 rotate 0 hpt L} repeat
  closepath stroke grestore Pnt } def
/PentF { stroke [] 0 setdash gsave
  translate 0 hpt M 4 {72 rotate 0 hpt L} repeat
  closepath fill grestore } def
/Circle { stroke [] 0 setdash 2 copy
  hpt 0 360 arc stroke Pnt } def
/CircleF { stroke [] 0 setdash hpt 0 360 arc fill } def
/C0 { BL [] 0 setdash 2 copy moveto vpt 90 450  arc } bind def
/C1 { BL [] 0 setdash 2 copy        moveto
       2 copy  vpt 0 90 arc closepath fill
               vpt 0 360 arc closepath } bind def
/C2 { BL [] 0 setdash 2 copy moveto
       2 copy  vpt 90 180 arc closepath fill
               vpt 0 360 arc closepath } bind def
/C3 { BL [] 0 setdash 2 copy moveto
       2 copy  vpt 0 180 arc closepath fill
               vpt 0 360 arc closepath } bind def
/C4 { BL [] 0 setdash 2 copy moveto
       2 copy  vpt 180 270 arc closepath fill
               vpt 0 360 arc closepath } bind def
/C5 { BL [] 0 setdash 2 copy moveto
       2 copy  vpt 0 90 arc
       2 copy moveto
       2 copy  vpt 180 270 arc closepath fill
               vpt 0 360 arc } bind def
/C6 { BL [] 0 setdash 2 copy moveto
      2 copy  vpt 90 270 arc closepath fill
              vpt 0 360 arc closepath } bind def
/C7 { BL [] 0 setdash 2 copy moveto
      2 copy  vpt 0 270 arc closepath fill
              vpt 0 360 arc closepath } bind def
/C8 { BL [] 0 setdash 2 copy moveto
      2 copy vpt 270 360 arc closepath fill
              vpt 0 360 arc closepath } bind def
/C9 { BL [] 0 setdash 2 copy moveto
      2 copy  vpt 270 450 arc closepath fill
              vpt 0 360 arc closepath } bind def
/C10 { BL [] 0 setdash 2 copy 2 copy moveto vpt 270 360 arc closepath fill
       2 copy moveto
       2 copy vpt 90 180 arc closepath fill
               vpt 0 360 arc closepath } bind def
/C11 { BL [] 0 setdash 2 copy moveto
       2 copy  vpt 0 180 arc closepath fill
       2 copy moveto
       2 copy  vpt 270 360 arc closepath fill
               vpt 0 360 arc closepath } bind def
/C12 { BL [] 0 setdash 2 copy moveto
       2 copy  vpt 180 360 arc closepath fill
               vpt 0 360 arc closepath } bind def
/C13 { BL [] 0 setdash  2 copy moveto
       2 copy  vpt 0 90 arc closepath fill
       2 copy moveto
       2 copy  vpt 180 360 arc closepath fill
               vpt 0 360 arc closepath } bind def
/C14 { BL [] 0 setdash 2 copy moveto
       2 copy  vpt 90 360 arc closepath fill
               vpt 0 360 arc } bind def
/C15 { BL [] 0 setdash 2 copy vpt 0 360 arc closepath fill
               vpt 0 360 arc closepath } bind def
/Rec   { newpath 4 2 roll moveto 1 index 0 rlineto 0 exch rlineto
       neg 0 rlineto closepath } bind def
/Square { dup Rec } bind def
/Bsquare { vpt sub exch vpt sub exch vpt2 Square } bind def
/S0 { BL [] 0 setdash 2 copy moveto 0 vpt rlineto BL Bsquare } bind def
/S1 { BL [] 0 setdash 2 copy vpt Square fill Bsquare } bind def
/S2 { BL [] 0 setdash 2 copy exch vpt sub exch vpt Square fill Bsquare } bind def
/S3 { BL [] 0 setdash 2 copy exch vpt sub exch vpt2 vpt Rec fill Bsquare } bind def
/S4 { BL [] 0 setdash 2 copy exch vpt sub exch vpt sub vpt Square fill Bsquare } bind def
/S5 { BL [] 0 setdash 2 copy 2 copy vpt Square fill
       exch vpt sub exch vpt sub vpt Square fill Bsquare } bind def
/S6 { BL [] 0 setdash 2 copy exch vpt sub exch vpt sub vpt vpt2 Rec fill Bsquare } bind def
/S7 { BL [] 0 setdash 2 copy exch vpt sub exch vpt sub vpt vpt2 Rec fill
       2 copy vpt Square fill
       Bsquare } bind def
/S8 { BL [] 0 setdash 2 copy vpt sub vpt Square fill Bsquare } bind def
/S9 { BL [] 0 setdash 2 copy vpt sub vpt vpt2 Rec fill Bsquare } bind def
/S10 { BL [] 0 setdash 2 copy vpt sub vpt Square fill 2 copy exch vpt sub exch vpt Square fill
       Bsquare } bind def
/S11 { BL [] 0 setdash 2 copy vpt sub vpt Square fill 2 copy exch vpt sub exch vpt2 vpt Rec fill
       Bsquare } bind def
/S12 { BL [] 0 setdash 2 copy exch vpt sub exch vpt sub vpt2 vpt Rec fill Bsquare } bind def
/S13 { BL [] 0 setdash 2 copy exch vpt sub exch vpt sub vpt2 vpt Rec fill
       2 copy vpt Square fill Bsquare } bind def
/S14 { BL [] 0 setdash 2 copy exch vpt sub exch vpt sub vpt2 vpt Rec fill
       2 copy exch vpt sub exch vpt Square fill Bsquare } bind def
/S15 { BL [] 0 setdash 2 copy Bsquare fill Bsquare } bind def
/D0 { gsave translate 45 rotate 0 0 S0 stroke grestore } bind def
/D1 { gsave translate 45 rotate 0 0 S1 stroke grestore } bind def
/D2 { gsave translate 45 rotate 0 0 S2 stroke grestore } bind def
/D3 { gsave translate 45 rotate 0 0 S3 stroke grestore } bind def
/D4 { gsave translate 45 rotate 0 0 S4 stroke grestore } bind def
/D5 { gsave translate 45 rotate 0 0 S5 stroke grestore } bind def
/D6 { gsave translate 45 rotate 0 0 S6 stroke grestore } bind def
/D7 { gsave translate 45 rotate 0 0 S7 stroke grestore } bind def
/D8 { gsave translate 45 rotate 0 0 S8 stroke grestore } bind def
/D9 { gsave translate 45 rotate 0 0 S9 stroke grestore } bind def
/D10 { gsave translate 45 rotate 0 0 S10 stroke grestore } bind def
/D11 { gsave translate 45 rotate 0 0 S11 stroke grestore } bind def
/D12 { gsave translate 45 rotate 0 0 S12 stroke grestore } bind def
/D13 { gsave translate 45 rotate 0 0 S13 stroke grestore } bind def
/D14 { gsave translate 45 rotate 0 0 S14 stroke grestore } bind def
/D15 { gsave translate 45 rotate 0 0 S15 stroke grestore } bind def
/DiaE { stroke [] 0 setdash vpt add M
  hpt neg vpt neg V hpt vpt neg V
  hpt vpt V hpt neg vpt V closepath stroke } def
/BoxE { stroke [] 0 setdash exch hpt sub exch vpt add M
  0 vpt2 neg V hpt2 0 V 0 vpt2 V
  hpt2 neg 0 V closepath stroke } def
/TriUE { stroke [] 0 setdash vpt 1.12 mul add M
  hpt neg vpt -1.62 mul V
  hpt 2 mul 0 V
  hpt neg vpt 1.62 mul V closepath stroke } def
/TriDE { stroke [] 0 setdash vpt 1.12 mul sub M
  hpt neg vpt 1.62 mul V
  hpt 2 mul 0 V
  hpt neg vpt -1.62 mul V closepath stroke } def
/PentE { stroke [] 0 setdash gsave
  translate 0 hpt M 4 {72 rotate 0 hpt L} repeat
  closepath stroke grestore } def
/CircE { stroke [] 0 setdash 
  hpt 0 360 arc stroke } def
/Opaque { gsave closepath 1 setgray fill grestore 0 setgray closepath } def
/DiaW { stroke [] 0 setdash vpt add M
  hpt neg vpt neg V hpt vpt neg V
  hpt vpt V hpt neg vpt V Opaque stroke } def
/BoxW { stroke [] 0 setdash exch hpt sub exch vpt add M
  0 vpt2 neg V hpt2 0 V 0 vpt2 V
  hpt2 neg 0 V Opaque stroke } def
/TriUW { stroke [] 0 setdash vpt 1.12 mul add M
  hpt neg vpt -1.62 mul V
  hpt 2 mul 0 V
  hpt neg vpt 1.62 mul V Opaque stroke } def
/TriDW { stroke [] 0 setdash vpt 1.12 mul sub M
  hpt neg vpt 1.62 mul V
  hpt 2 mul 0 V
  hpt neg vpt -1.62 mul V Opaque stroke } def
/PentW { stroke [] 0 setdash gsave
  translate 0 hpt M 4 {72 rotate 0 hpt L} repeat
  Opaque stroke grestore } def
/CircW { stroke [] 0 setdash 
  hpt 0 360 arc Opaque stroke } def
/BoxFill { gsave Rec 1 setgray fill grestore } def
end
%%EndProlog
}}%
\begin{picture}(3600,2160)(0,0)%
{\GNUPLOTspecial{"
gnudict begin
gsave
0 0 translate
0.100 0.100 scale
0 setgray
newpath
1.000 UL
LTb
400 300 M
63 0 V
2987 0 R
-63 0 V
400 520 M
63 0 V
2987 0 R
-63 0 V
400 740 M
63 0 V
2987 0 R
-63 0 V
400 960 M
63 0 V
2987 0 R
-63 0 V
400 1180 M
63 0 V
2987 0 R
-63 0 V
400 1400 M
63 0 V
2987 0 R
-63 0 V
400 1620 M
63 0 V
2987 0 R
-63 0 V
400 1840 M
63 0 V
2987 0 R
-63 0 V
400 2060 M
63 0 V
2987 0 R
-63 0 V
552 300 M
0 63 V
0 1697 R
0 -63 V
857 300 M
0 63 V
0 1697 R
0 -63 V
1163 300 M
0 63 V
0 1697 R
0 -63 V
1468 300 M
0 63 V
0 1697 R
0 -63 V
1773 300 M
0 63 V
0 1697 R
0 -63 V
2078 300 M
0 63 V
0 1697 R
0 -63 V
2382 300 M
0 63 V
0 1697 R
0 -63 V
2687 300 M
0 63 V
0 1697 R
0 -63 V
2992 300 M
0 63 V
0 1697 R
0 -63 V
3297 300 M
0 63 V
0 1697 R
0 -63 V
1.000 UL
LTb
400 300 M
3050 0 V
0 1760 V
-3050 0 V
400 300 L
1.000 UL
LT0
3087 1947 M
263 0 V
400 887 M
31 -37 V
31 -33 V
30 -30 V
31 -28 V
31 -24 V
31 -23 V
31 -20 V
30 -18 V
31 -17 V
31 -16 V
31 -13 V
31 -13 V
31 -12 V
30 -10 V
31 -10 V
31 -9 V
31 -7 V
31 -8 V
30 -6 V
31 -6 V
31 -6 V
31 -4 V
31 -5 V
30 -4 V
31 -3 V
31 -3 V
31 -3 V
31 -3 V
30 -2 V
31 -2 V
31 -2 V
31 -1 V
31 -1 V
30 -1 V
31 -1 V
31 -1 V
31 0 V
31 -1 V
31 0 V
30 0 V
31 0 V
31 0 V
31 0 V
31 1 V
30 0 V
31 0 V
31 1 V
31 0 V
31 1 V
30 1 V
31 1 V
31 0 V
31 1 V
31 1 V
30 1 V
31 1 V
31 1 V
31 1 V
31 1 V
30 1 V
31 1 V
31 1 V
31 1 V
31 1 V
31 1 V
30 1 V
31 1 V
31 1 V
31 1 V
31 1 V
30 1 V
31 1 V
31 1 V
31 1 V
31 1 V
30 1 V
31 1 V
31 1 V
31 1 V
31 1 V
30 1 V
31 1 V
31 1 V
31 1 V
31 1 V
30 1 V
31 1 V
31 1 V
31 1 V
31 1 V
31 1 V
30 0 V
31 1 V
31 1 V
31 1 V
31 1 V
30 0 V
31 1 V
31 1 V
1.000 UL
LT1
3087 1847 M
263 0 V
400 1930 M
31 -68 V
31 -64 V
30 -59 V
31 -55 V
31 -52 V
31 -49 V
31 -45 V
30 -43 V
31 -41 V
31 -38 V
31 -36 V
31 -34 V
31 -32 V
30 -31 V
31 -29 V
31 -28 V
31 -26 V
31 -25 V
30 -24 V
31 -22 V
31 -22 V
31 -20 V
31 -20 V
30 -18 V
31 -18 V
31 -17 V
31 -16 V
31 -16 V
30 -15 V
31 -14 V
31 -14 V
31 -13 V
31 -12 V
30 -13 V
31 -11 V
31 -11 V
31 -11 V
31 -10 V
31 -10 V
30 -9 V
31 -9 V
31 -9 V
31 -8 V
31 -8 V
30 -8 V
31 -7 V
31 -8 V
31 -7 V
31 -6 V
30 -7 V
31 -6 V
31 -6 V
31 -5 V
31 -6 V
30 -5 V
31 -5 V
31 -5 V
31 -5 V
31 -5 V
30 -4 V
31 -4 V
31 -4 V
31 -4 V
31 -4 V
31 -4 V
30 -3 V
31 -4 V
31 -3 V
31 -3 V
31 -4 V
30 -3 V
31 -2 V
31 -3 V
31 -3 V
31 -3 V
30 -2 V
31 -3 V
31 -2 V
31 -2 V
31 -2 V
30 -3 V
31 -2 V
31 -2 V
31 -2 V
31 -1 V
30 -2 V
31 -2 V
31 -2 V
31 -1 V
31 -2 V
31 -1 V
30 -2 V
31 -1 V
31 -2 V
31 -1 V
31 -1 V
30 -2 V
31 -1 V
31 -1 V
1.000 UL
LT2
3087 1747 M
263 0 V
400 582 M
31 0 V
31 0 V
30 0 V
31 0 V
31 0 V
31 0 V
31 0 V
30 0 V
31 0 V
31 0 V
31 0 V
31 0 V
31 0 V
30 0 V
31 0 V
31 0 V
31 0 V
31 0 V
30 0 V
31 0 V
31 0 V
31 0 V
31 0 V
30 0 V
31 0 V
31 0 V
31 0 V
31 0 V
30 0 V
31 0 V
31 0 V
31 0 V
31 0 V
30 0 V
31 0 V
31 0 V
31 0 V
31 0 V
31 0 V
30 0 V
31 0 V
31 0 V
31 0 V
31 0 V
30 0 V
31 0 V
31 0 V
31 0 V
31 0 V
30 0 V
31 0 V
31 0 V
31 0 V
31 0 V
30 0 V
31 0 V
31 0 V
31 0 V
31 0 V
30 0 V
31 0 V
31 0 V
31 0 V
31 0 V
31 0 V
30 0 V
31 0 V
31 0 V
31 0 V
31 0 V
30 0 V
31 0 V
31 0 V
31 0 V
31 0 V
30 0 V
31 0 V
31 0 V
31 0 V
31 0 V
30 0 V
31 0 V
31 0 V
31 0 V
31 0 V
30 0 V
31 0 V
31 0 V
31 0 V
31 0 V
31 0 V
30 0 V
31 0 V
31 0 V
31 0 V
31 0 V
30 0 V
31 0 V
31 0 V
stroke
grestore
end
showpage
}}%
\put(3037,1747){\makebox(0,0)[r]{$E_0=-13.6$ eV}}%
\put(3037,1847){\makebox(0,0)[r]{$\langle \OP{H} \rangle_{-}$}}%
\put(3037,1947){\makebox(0,0)[r]{$\langle \OP{H} \rangle_{+}$}}%
\put(1925,50){\makebox(0,0){$R$ [nm]}}%
\put(100,1180){%
\special{ps: gsave currentpoint currentpoint translate
270 rotate neg exch neg exch translate}%
\makebox(0,0)[b]{\shortstack{$E(R)$ [eV]}}%
\special{ps: currentpoint grestore moveto}%
}%
\put(3297,200){\makebox(0,0){0.24}}%
\put(2992,200){\makebox(0,0){0.22}}%
\put(2687,200){\makebox(0,0){0.2}}%
\put(2382,200){\makebox(0,0){0.18}}%
\put(2078,200){\makebox(0,0){0.16}}%
\put(1773,200){\makebox(0,0){0.14}}%
\put(1468,200){\makebox(0,0){0.12}}%
\put(1163,200){\makebox(0,0){0.1}}%
\put(857,200){\makebox(0,0){0.08}}%
\put(552,200){\makebox(0,0){0.06}}%
\put(350,2060){\makebox(0,0)[r]{20}}%
\put(350,1840){\makebox(0,0)[r]{15}}%
\put(350,1620){\makebox(0,0)[r]{10}}%
\put(350,1400){\makebox(0,0)[r]{5}}%
\put(350,1180){\makebox(0,0)[r]{0}}%
\put(350,960){\makebox(0,0)[r]{-5}}%
\put(350,740){\makebox(0,0)[r]{-10}}%
\put(350,520){\makebox(0,0)[r]{-15}}%
\put(350,300){\makebox(0,0)[r]{-20}}%
\end{picture}%
\endgroup
\endinput

\end{center}
\caption{Plott av  $\langle \OP{H} \rangle_{\pm}$ for ulike verdier av
$|{\bf R}|$.\label{67}}
\end{figure} 
De to forventningsverdiene er plotta i Figur \ref{67}, og vi ser at den symmetriske
b\o lgefunskjonen gir binding, mens den antisymmetriske ikke gir noen binding.
Grunnen til at vi sier at den antisymmetriske energien ikke gir binding,
skyldes at kurven for $\langle \OP{H} \rangle_{-}$ ikke utviser
noe minimum som funksjon av $R$.

I Figur \ref{67} har vi ogs\aa\ plotta bindingsenergien til hydrogenatomet,
$E_0=-13.6$ eV. 
I grensa $R\rightarrow \infty$ vil elektronet bare v\ae re bundet
til et proton. Vi ser at begge forventningsverdiene 
$\langle \OP{H} \rangle_{\pm}$ 
n\ae rmer seg denne grensa n\aa r $R$ blir stor.

For den symmetriske l\o sningen har vi et minimumspunkt ved $R=0.13$ nm,
og en energi p\aa\ $-15.37$ eV. Det svarer til et tillegg p\aa\ 
$1.77$ eV i forhold til $-13.6$, som er resultatet ved $R\rightarrow \infty$.
Dette tillegget er dissosiasjonsenergien, eller bindingsenergien for molekylet.
Den eksperimentelle verdien er $2.8$ eV, og likevektsavstanden
er $r_0=0.106$ nm. V\aa r enkle approksimasjon gir alts\aa\
resultater som ikke er s\aa\ langt vekke fra virkeligheten.

Vi f\aa r et tillegg i binding fordi elektronet n\aa\ f\o ler
tiltrekningen til to protoner. 
Men det er elektronet som binder atomene sammen, det er dette som kalles for kovalent binding. 
Dersom vi istedet for H$_2^+$ molekylet
hadde sett p\aa\ Heliumionet He$^+$, som ogs\aa\ har to protoner og et
elektroner, er bindingsenergien $E_B=-54.4$ eV. 

I v\aa rt tilfelle er de to atomene (to protoner) bundet sammen pga.~en
elektronladningsfordeling mellom seg. For heliumionet, er det derimot
to protoner som binder et elektron til seg.
\begin{figure}
\begin{center}
% GNUPLOT: LaTeX picture with Postscript
\begingroup%
  \makeatletter%
  \newcommand{\GNUPLOTspecial}{%
    \@sanitize\catcode`\%=14\relax\special}%
  \setlength{\unitlength}{0.1bp}%
{\GNUPLOTspecial{!
%!PS-Adobe-2.0 EPSF-2.0
%%Title: molbinding1.tex
%%Creator: gnuplot 3.7 patchlevel 0.2
%%CreationDate: Fri Apr 28 12:52:10 2000
%%DocumentFonts: 
%%BoundingBox: 0 0 360 216
%%Orientation: Landscape
%%EndComments
/gnudict 256 dict def
gnudict begin
/Color false def
/Solid false def
/gnulinewidth 5.000 def
/userlinewidth gnulinewidth def
/vshift -33 def
/dl {10 mul} def
/hpt_ 31.5 def
/vpt_ 31.5 def
/hpt hpt_ def
/vpt vpt_ def
/M {moveto} bind def
/L {lineto} bind def
/R {rmoveto} bind def
/V {rlineto} bind def
/vpt2 vpt 2 mul def
/hpt2 hpt 2 mul def
/Lshow { currentpoint stroke M
  0 vshift R show } def
/Rshow { currentpoint stroke M
  dup stringwidth pop neg vshift R show } def
/Cshow { currentpoint stroke M
  dup stringwidth pop -2 div vshift R show } def
/UP { dup vpt_ mul /vpt exch def hpt_ mul /hpt exch def
  /hpt2 hpt 2 mul def /vpt2 vpt 2 mul def } def
/DL { Color {setrgbcolor Solid {pop []} if 0 setdash }
 {pop pop pop Solid {pop []} if 0 setdash} ifelse } def
/BL { stroke userlinewidth 2 mul setlinewidth } def
/AL { stroke userlinewidth 2 div setlinewidth } def
/UL { dup gnulinewidth mul /userlinewidth exch def
      10 mul /udl exch def } def
/PL { stroke userlinewidth setlinewidth } def
/LTb { BL [] 0 0 0 DL } def
/LTa { AL [1 udl mul 2 udl mul] 0 setdash 0 0 0 setrgbcolor } def
/LT0 { PL [] 1 0 0 DL } def
/LT1 { PL [4 dl 2 dl] 0 1 0 DL } def
/LT2 { PL [2 dl 3 dl] 0 0 1 DL } def
/LT3 { PL [1 dl 1.5 dl] 1 0 1 DL } def
/LT4 { PL [5 dl 2 dl 1 dl 2 dl] 0 1 1 DL } def
/LT5 { PL [4 dl 3 dl 1 dl 3 dl] 1 1 0 DL } def
/LT6 { PL [2 dl 2 dl 2 dl 4 dl] 0 0 0 DL } def
/LT7 { PL [2 dl 2 dl 2 dl 2 dl 2 dl 4 dl] 1 0.3 0 DL } def
/LT8 { PL [2 dl 2 dl 2 dl 2 dl 2 dl 2 dl 2 dl 4 dl] 0.5 0.5 0.5 DL } def
/Pnt { stroke [] 0 setdash
   gsave 1 setlinecap M 0 0 V stroke grestore } def
/Dia { stroke [] 0 setdash 2 copy vpt add M
  hpt neg vpt neg V hpt vpt neg V
  hpt vpt V hpt neg vpt V closepath stroke
  Pnt } def
/Pls { stroke [] 0 setdash vpt sub M 0 vpt2 V
  currentpoint stroke M
  hpt neg vpt neg R hpt2 0 V stroke
  } def
/Box { stroke [] 0 setdash 2 copy exch hpt sub exch vpt add M
  0 vpt2 neg V hpt2 0 V 0 vpt2 V
  hpt2 neg 0 V closepath stroke
  Pnt } def
/Crs { stroke [] 0 setdash exch hpt sub exch vpt add M
  hpt2 vpt2 neg V currentpoint stroke M
  hpt2 neg 0 R hpt2 vpt2 V stroke } def
/TriU { stroke [] 0 setdash 2 copy vpt 1.12 mul add M
  hpt neg vpt -1.62 mul V
  hpt 2 mul 0 V
  hpt neg vpt 1.62 mul V closepath stroke
  Pnt  } def
/Star { 2 copy Pls Crs } def
/BoxF { stroke [] 0 setdash exch hpt sub exch vpt add M
  0 vpt2 neg V  hpt2 0 V  0 vpt2 V
  hpt2 neg 0 V  closepath fill } def
/TriUF { stroke [] 0 setdash vpt 1.12 mul add M
  hpt neg vpt -1.62 mul V
  hpt 2 mul 0 V
  hpt neg vpt 1.62 mul V closepath fill } def
/TriD { stroke [] 0 setdash 2 copy vpt 1.12 mul sub M
  hpt neg vpt 1.62 mul V
  hpt 2 mul 0 V
  hpt neg vpt -1.62 mul V closepath stroke
  Pnt  } def
/TriDF { stroke [] 0 setdash vpt 1.12 mul sub M
  hpt neg vpt 1.62 mul V
  hpt 2 mul 0 V
  hpt neg vpt -1.62 mul V closepath fill} def
/DiaF { stroke [] 0 setdash vpt add M
  hpt neg vpt neg V hpt vpt neg V
  hpt vpt V hpt neg vpt V closepath fill } def
/Pent { stroke [] 0 setdash 2 copy gsave
  translate 0 hpt M 4 {72 rotate 0 hpt L} repeat
  closepath stroke grestore Pnt } def
/PentF { stroke [] 0 setdash gsave
  translate 0 hpt M 4 {72 rotate 0 hpt L} repeat
  closepath fill grestore } def
/Circle { stroke [] 0 setdash 2 copy
  hpt 0 360 arc stroke Pnt } def
/CircleF { stroke [] 0 setdash hpt 0 360 arc fill } def
/C0 { BL [] 0 setdash 2 copy moveto vpt 90 450  arc } bind def
/C1 { BL [] 0 setdash 2 copy        moveto
       2 copy  vpt 0 90 arc closepath fill
               vpt 0 360 arc closepath } bind def
/C2 { BL [] 0 setdash 2 copy moveto
       2 copy  vpt 90 180 arc closepath fill
               vpt 0 360 arc closepath } bind def
/C3 { BL [] 0 setdash 2 copy moveto
       2 copy  vpt 0 180 arc closepath fill
               vpt 0 360 arc closepath } bind def
/C4 { BL [] 0 setdash 2 copy moveto
       2 copy  vpt 180 270 arc closepath fill
               vpt 0 360 arc closepath } bind def
/C5 { BL [] 0 setdash 2 copy moveto
       2 copy  vpt 0 90 arc
       2 copy moveto
       2 copy  vpt 180 270 arc closepath fill
               vpt 0 360 arc } bind def
/C6 { BL [] 0 setdash 2 copy moveto
      2 copy  vpt 90 270 arc closepath fill
              vpt 0 360 arc closepath } bind def
/C7 { BL [] 0 setdash 2 copy moveto
      2 copy  vpt 0 270 arc closepath fill
              vpt 0 360 arc closepath } bind def
/C8 { BL [] 0 setdash 2 copy moveto
      2 copy vpt 270 360 arc closepath fill
              vpt 0 360 arc closepath } bind def
/C9 { BL [] 0 setdash 2 copy moveto
      2 copy  vpt 270 450 arc closepath fill
              vpt 0 360 arc closepath } bind def
/C10 { BL [] 0 setdash 2 copy 2 copy moveto vpt 270 360 arc closepath fill
       2 copy moveto
       2 copy vpt 90 180 arc closepath fill
               vpt 0 360 arc closepath } bind def
/C11 { BL [] 0 setdash 2 copy moveto
       2 copy  vpt 0 180 arc closepath fill
       2 copy moveto
       2 copy  vpt 270 360 arc closepath fill
               vpt 0 360 arc closepath } bind def
/C12 { BL [] 0 setdash 2 copy moveto
       2 copy  vpt 180 360 arc closepath fill
               vpt 0 360 arc closepath } bind def
/C13 { BL [] 0 setdash  2 copy moveto
       2 copy  vpt 0 90 arc closepath fill
       2 copy moveto
       2 copy  vpt 180 360 arc closepath fill
               vpt 0 360 arc closepath } bind def
/C14 { BL [] 0 setdash 2 copy moveto
       2 copy  vpt 90 360 arc closepath fill
               vpt 0 360 arc } bind def
/C15 { BL [] 0 setdash 2 copy vpt 0 360 arc closepath fill
               vpt 0 360 arc closepath } bind def
/Rec   { newpath 4 2 roll moveto 1 index 0 rlineto 0 exch rlineto
       neg 0 rlineto closepath } bind def
/Square { dup Rec } bind def
/Bsquare { vpt sub exch vpt sub exch vpt2 Square } bind def
/S0 { BL [] 0 setdash 2 copy moveto 0 vpt rlineto BL Bsquare } bind def
/S1 { BL [] 0 setdash 2 copy vpt Square fill Bsquare } bind def
/S2 { BL [] 0 setdash 2 copy exch vpt sub exch vpt Square fill Bsquare } bind def
/S3 { BL [] 0 setdash 2 copy exch vpt sub exch vpt2 vpt Rec fill Bsquare } bind def
/S4 { BL [] 0 setdash 2 copy exch vpt sub exch vpt sub vpt Square fill Bsquare } bind def
/S5 { BL [] 0 setdash 2 copy 2 copy vpt Square fill
       exch vpt sub exch vpt sub vpt Square fill Bsquare } bind def
/S6 { BL [] 0 setdash 2 copy exch vpt sub exch vpt sub vpt vpt2 Rec fill Bsquare } bind def
/S7 { BL [] 0 setdash 2 copy exch vpt sub exch vpt sub vpt vpt2 Rec fill
       2 copy vpt Square fill
       Bsquare } bind def
/S8 { BL [] 0 setdash 2 copy vpt sub vpt Square fill Bsquare } bind def
/S9 { BL [] 0 setdash 2 copy vpt sub vpt vpt2 Rec fill Bsquare } bind def
/S10 { BL [] 0 setdash 2 copy vpt sub vpt Square fill 2 copy exch vpt sub exch vpt Square fill
       Bsquare } bind def
/S11 { BL [] 0 setdash 2 copy vpt sub vpt Square fill 2 copy exch vpt sub exch vpt2 vpt Rec fill
       Bsquare } bind def
/S12 { BL [] 0 setdash 2 copy exch vpt sub exch vpt sub vpt2 vpt Rec fill Bsquare } bind def
/S13 { BL [] 0 setdash 2 copy exch vpt sub exch vpt sub vpt2 vpt Rec fill
       2 copy vpt Square fill Bsquare } bind def
/S14 { BL [] 0 setdash 2 copy exch vpt sub exch vpt sub vpt2 vpt Rec fill
       2 copy exch vpt sub exch vpt Square fill Bsquare } bind def
/S15 { BL [] 0 setdash 2 copy Bsquare fill Bsquare } bind def
/D0 { gsave translate 45 rotate 0 0 S0 stroke grestore } bind def
/D1 { gsave translate 45 rotate 0 0 S1 stroke grestore } bind def
/D2 { gsave translate 45 rotate 0 0 S2 stroke grestore } bind def
/D3 { gsave translate 45 rotate 0 0 S3 stroke grestore } bind def
/D4 { gsave translate 45 rotate 0 0 S4 stroke grestore } bind def
/D5 { gsave translate 45 rotate 0 0 S5 stroke grestore } bind def
/D6 { gsave translate 45 rotate 0 0 S6 stroke grestore } bind def
/D7 { gsave translate 45 rotate 0 0 S7 stroke grestore } bind def
/D8 { gsave translate 45 rotate 0 0 S8 stroke grestore } bind def
/D9 { gsave translate 45 rotate 0 0 S9 stroke grestore } bind def
/D10 { gsave translate 45 rotate 0 0 S10 stroke grestore } bind def
/D11 { gsave translate 45 rotate 0 0 S11 stroke grestore } bind def
/D12 { gsave translate 45 rotate 0 0 S12 stroke grestore } bind def
/D13 { gsave translate 45 rotate 0 0 S13 stroke grestore } bind def
/D14 { gsave translate 45 rotate 0 0 S14 stroke grestore } bind def
/D15 { gsave translate 45 rotate 0 0 S15 stroke grestore } bind def
/DiaE { stroke [] 0 setdash vpt add M
  hpt neg vpt neg V hpt vpt neg V
  hpt vpt V hpt neg vpt V closepath stroke } def
/BoxE { stroke [] 0 setdash exch hpt sub exch vpt add M
  0 vpt2 neg V hpt2 0 V 0 vpt2 V
  hpt2 neg 0 V closepath stroke } def
/TriUE { stroke [] 0 setdash vpt 1.12 mul add M
  hpt neg vpt -1.62 mul V
  hpt 2 mul 0 V
  hpt neg vpt 1.62 mul V closepath stroke } def
/TriDE { stroke [] 0 setdash vpt 1.12 mul sub M
  hpt neg vpt 1.62 mul V
  hpt 2 mul 0 V
  hpt neg vpt -1.62 mul V closepath stroke } def
/PentE { stroke [] 0 setdash gsave
  translate 0 hpt M 4 {72 rotate 0 hpt L} repeat
  closepath stroke grestore } def
/CircE { stroke [] 0 setdash 
  hpt 0 360 arc stroke } def
/Opaque { gsave closepath 1 setgray fill grestore 0 setgray closepath } def
/DiaW { stroke [] 0 setdash vpt add M
  hpt neg vpt neg V hpt vpt neg V
  hpt vpt V hpt neg vpt V Opaque stroke } def
/BoxW { stroke [] 0 setdash exch hpt sub exch vpt add M
  0 vpt2 neg V hpt2 0 V 0 vpt2 V
  hpt2 neg 0 V Opaque stroke } def
/TriUW { stroke [] 0 setdash vpt 1.12 mul add M
  hpt neg vpt -1.62 mul V
  hpt 2 mul 0 V
  hpt neg vpt 1.62 mul V Opaque stroke } def
/TriDW { stroke [] 0 setdash vpt 1.12 mul sub M
  hpt neg vpt 1.62 mul V
  hpt 2 mul 0 V
  hpt neg vpt -1.62 mul V Opaque stroke } def
/PentW { stroke [] 0 setdash gsave
  translate 0 hpt M 4 {72 rotate 0 hpt L} repeat
  Opaque stroke grestore } def
/CircW { stroke [] 0 setdash 
  hpt 0 360 arc Opaque stroke } def
/BoxFill { gsave Rec 1 setgray fill grestore } def
end
%%EndProlog
}}%
\begin{picture}(3600,2160)(0,0)%
{\GNUPLOTspecial{"
gnudict begin
gsave
0 0 translate
0.100 0.100 scale
0 setgray
newpath
1.000 UL
LTb
400 300 M
63 0 V
2987 0 R
-63 0 V
400 476 M
63 0 V
2987 0 R
-63 0 V
400 652 M
63 0 V
2987 0 R
-63 0 V
400 828 M
63 0 V
2987 0 R
-63 0 V
400 1004 M
63 0 V
2987 0 R
-63 0 V
400 1180 M
63 0 V
2987 0 R
-63 0 V
400 1356 M
63 0 V
2987 0 R
-63 0 V
400 1532 M
63 0 V
2987 0 R
-63 0 V
400 1708 M
63 0 V
2987 0 R
-63 0 V
400 1884 M
63 0 V
2987 0 R
-63 0 V
400 2060 M
63 0 V
2987 0 R
-63 0 V
677 300 M
0 63 V
0 1697 R
0 -63 V
1232 300 M
0 63 V
0 1697 R
0 -63 V
1786 300 M
0 63 V
0 1697 R
0 -63 V
2341 300 M
0 63 V
0 1697 R
0 -63 V
2895 300 M
0 63 V
0 1697 R
0 -63 V
3450 300 M
0 63 V
0 1697 R
0 -63 V
1.000 UL
LTb
400 300 M
3050 0 V
0 1760 V
-3050 0 V
400 300 L
1.000 UL
LT0
3087 1947 M
263 0 V
400 1944 M
31 -373 V
31 -282 V
30 -216 V
523 906 L
554 777 L
31 -99 V
31 -77 V
30 -59 V
31 -44 V
31 -32 V
31 -23 V
31 -16 V
31 -9 V
30 -5 V
31 -1 V
31 2 V
31 5 V
31 7 V
30 8 V
31 10 V
31 10 V
31 11 V
31 11 V
30 12 V
31 11 V
31 12 V
31 12 V
31 11 V
30 11 V
31 10 V
31 11 V
31 10 V
31 9 V
30 9 V
31 9 V
31 8 V
31 8 V
31 7 V
31 7 V
30 7 V
31 6 V
31 5 V
31 6 V
31 5 V
30 4 V
31 5 V
31 4 V
31 3 V
31 4 V
30 3 V
31 3 V
31 3 V
31 3 V
31 2 V
30 2 V
31 2 V
31 2 V
31 2 V
31 1 V
30 2 V
31 1 V
31 1 V
31 1 V
31 1 V
31 1 V
30 1 V
31 1 V
31 1 V
31 0 V
31 1 V
30 1 V
31 0 V
31 1 V
31 0 V
31 0 V
30 1 V
31 0 V
31 0 V
31 1 V
31 0 V
30 0 V
31 0 V
31 0 V
31 1 V
31 0 V
30 0 V
31 0 V
31 0 V
31 0 V
31 0 V
31 0 V
30 1 V
31 0 V
31 0 V
31 0 V
31 0 V
30 0 V
31 0 V
31 0 V
1.000 UL
LT1
3087 1847 M
263 0 V
400 722 M
31 0 V
31 0 V
30 0 V
31 0 V
31 0 V
31 0 V
31 0 V
30 0 V
31 0 V
31 0 V
31 0 V
31 0 V
31 0 V
30 0 V
31 0 V
31 0 V
31 0 V
31 0 V
30 0 V
31 0 V
31 0 V
31 0 V
31 0 V
30 0 V
31 0 V
31 0 V
31 0 V
31 0 V
30 0 V
31 0 V
31 0 V
31 0 V
31 0 V
30 0 V
31 0 V
31 0 V
31 0 V
31 0 V
31 0 V
30 0 V
31 0 V
31 0 V
31 0 V
31 0 V
30 0 V
31 0 V
31 0 V
31 0 V
31 0 V
30 0 V
31 0 V
31 0 V
31 0 V
31 0 V
30 0 V
31 0 V
31 0 V
31 0 V
31 0 V
30 0 V
31 0 V
31 0 V
31 0 V
31 0 V
31 0 V
30 0 V
31 0 V
31 0 V
31 0 V
31 0 V
30 0 V
31 0 V
31 0 V
31 0 V
31 0 V
30 0 V
31 0 V
31 0 V
31 0 V
31 0 V
30 0 V
31 0 V
31 0 V
31 0 V
31 0 V
30 0 V
31 0 V
31 0 V
31 0 V
31 0 V
31 0 V
30 0 V
31 0 V
31 0 V
31 0 V
31 0 V
30 0 V
31 0 V
31 0 V
stroke
grestore
end
showpage
}}%
\put(3037,1847){\makebox(0,0)[r]{$E_0=-13.6$}}%
\put(3037,1947){\makebox(0,0)[r]{$\langle \OP{H} \rangle_{+}$}}%
\put(1925,50){\makebox(0,0){$R$ [nm]}}%
\put(100,1180){%
\special{ps: gsave currentpoint currentpoint translate
270 rotate neg exch neg exch translate}%
\makebox(0,0)[b]{\shortstack{$E(R)$ [eV]}}%
\special{ps: currentpoint grestore moveto}%
}%
\put(3450,200){\makebox(0,0){0.6}}%
\put(2895,200){\makebox(0,0){0.5}}%
\put(2341,200){\makebox(0,0){0.4}}%
\put(1786,200){\makebox(0,0){0.3}}%
\put(1232,200){\makebox(0,0){0.2}}%
\put(677,200){\makebox(0,0){0.1}}%
\put(350,2060){\makebox(0,0)[r]{-6}}%
\put(350,1884){\makebox(0,0)[r]{-7}}%
\put(350,1708){\makebox(0,0)[r]{-8}}%
\put(350,1532){\makebox(0,0)[r]{-9}}%
\put(350,1356){\makebox(0,0)[r]{-10}}%
\put(350,1180){\makebox(0,0)[r]{-11}}%
\put(350,1004){\makebox(0,0)[r]{-12}}%
\put(350,828){\makebox(0,0)[r]{-13}}%
\put(350,652){\makebox(0,0)[r]{-14}}%
\put(350,476){\makebox(0,0)[r]{-15}}%
\put(350,300){\makebox(0,0)[r]{-16}}%
\end{picture}%
\endgroup
\endinput

\end{center}
\caption{Plott av  $\langle \OP{H} \rangle_{+}$ for ulike verdier av
$|{\bf R}|$.\label{68}}
\end{figure} 
Til slutt,
legg ogs\aa\ merke til at den totale energien er p\aa\ 
$-15.37$ eV ved minimumspunktet, st\o rre i absoluttverdi 
enn $E_0=-13.6$ eV. 
F\o rst ved store verdier av $R$ begynner bindingsenergien \aa\ n\ae rme
seg $-13.6$ eV. Figur \ref{68} viser bindingsenergien for den symmetriske
b\o lgefunksjonen for store $R$. 
For en verdi av $R=0.4$ nm, betyr det energien er st\o rre
i absolutt verdi en hydrogenatomets bindingsenergi.   
Dersom  vi ser tilbake p\aa\ Figur
\ref{62} hvor vi har tegna inn ei rett linje for $E_0=-13.6$ eV,  
ser vi at elektronet m\aa\ tunnelere gjennom potensialbarrieren.
Alts\aa\, kovalent binding ved tunneling. 
I tekstboka, se avsnitt 10-1, blir H$_2^+$ molekylet framstilt
som binding pga.~tunneling. Det er ikke korrekt dersom dere
ser p\aa\ plottet av den potensielle energien for $R=0.1$ nm,
som svarer til omtrent likevektsavstanden. I det tilfelle er
den potensielle energien p\aa\ nesten -50 eV, mens den 
totale eksperimentelle er p\aa\ ca.~-16.4 eV. Da er det ikke 
noen tunneling. Derimot, dersom vi ser p\aa\
verdier for $R\sim 0.4$ nm, kan vi ha tunneling. Men da har ikke systemet
lavest energi. 

For \aa\ repetere, H$_2^+$ molekylet er et eksempel
p\aa\ kovalent binding, hvor et elektron (eller flere) setter 
opp en ladningsfordeling som binder atomene. 



\subsection{H$_2$ molekylet}

Vi avslutter denne diskusjonen  med \aa\ nevne at ogs\aa\
H$_2$ molekylet er et eksempel p\aa\ kovalent binding.
Systemet\footnote{Diskutert i avsnitt 10-2 i boka.} 
er litt mer komplisert enn H$_2^+$ molekylet da vi n\aa\
har to elektroner. Men igjen er det ladningsfordelingen
som elektronene setter opp som binder atomene, i dette
tilfelle to protoner. Mekanismen er den samme som for
H$_2^+$ molekylet. Men tilleggselektronet bidrar med mer
binding, slik at dissosiasjonsenergein for dette 
molekylet er $4.5$ eV, mot 2.8 eV for  
H$_2^+$. mer binding betyr ogs\aa\ at likevektsavstanden
minsker, her er den 0.074 nm, kontra 0.106 i H$_2^+$. 

Siden vi n\aa\ har to elektroner, m\aa\ vi ogs\aa\ huske,
se tilbake p\aa\ Helium i forrige kapittel, at vi har 
en frast\o ting mellom elektronene som kommer i tillegg. 

Det vi ogs\aa\ skal legge merke til er at siden 
elektronene er fermioner, kan de ikke
ha samme sett kvantetall. Det betyr at dersom begge elektroner
er i $1s$ orbitalen, s\aa\ er den romlige delen
av b\o lgefunksjonen symmetrisk. Det betyr at
spinndelen m\aa\ v\ae re antisymmetrisk.
Lager vi en ansats for b\o lgefunksjonen basert p\aa\ en symmetrisk
romdel og en antisymmetrisk spinndel, s\aa\ g\aa r det an \aa\
vise at vi f\aa r binding n\aa r vi plasserer begge
elektronene i $1s$ orbitalen . Velger vi symmetrisk spinndel
f\aa r vi ikke binding, men da m\aa\ vi ogs\aa\ plassere elektronene
i ulike orbitaler. 

Klarer du \aa\ forklare hvorfor H$_3$ ikke er et stabilt molekyl?
Eller hva med et molekyl av heliumatomer? Sistnevnte utviser dog 
molekyl\ae r binding
ved sv\ae rt h\o ye trykk eller lave temperaturer via van der Waals
krefter pga.~ei indusert dipol-dipol kraft. 

\subsection{Ionisk eller kovalent binding?}

Vi har tidligere sett at f.eks.~koksalt, NaCl, utviser en form
for kovalent binding ogs\aa\, hvor det ene elektronet 
som avgis av natriumatomet til kloratomet, kan p\aa\ et vis
tolkes som om det deles mellom de to atomene.

Et m\aa l for graden av ionisk binding kan v\ae re \aa\ sammenlikne
det eksperimentelle dipolmomentet et molekyl har og dipolmomentet
som framkommer dersom vi forestiller oss at molekylet er satt sammen
av to ioner, som punktladninger,  med midlere avstand $r_0$ i grunntilstanden.
Dipolmomentet $\OP{D}_{\mathrm{ion}}$ for koksalt blir da
\be
  \OP{D}_{\mathrm{ion}}=er_0=(1.6\times 10^{-19}\hspace{0.1cm}\mathrm{C})
  (2.4\times 10^{-10}\hspace{0.1cm}\mathrm{m})=3.8\times 10^{-29}\hspace{0.1cm}\mathrm{Cm}.
\ee
Den eksperimentelle verdien er
\be
  \OP{D}_{\mathrm{eksp}}=
 3.0\times 10^{-29}\hspace{0.1cm}\mathrm{Cm}).
\ee

Forholdet mellom disse to st\o rrelsene blir dermed
\be
   \OP{D}_{\mathrm{eksp}}/\OP{D}_{\mathrm{ion}}=0.8.
\ee
Det betyr at vi har ca.~80\% ionisk binding og 20\% kovalent
binding i vanlig koksalt.


\section{Oppgaver}
\subsection{Analytiske oppgaver}
\subsubsection*{Oppgave 10.1}

%
Energien til grunntilstanden av molekyl ionet $H_{2}^{+}$ er 2,65 eV
mindre enn
energien til et system med et n\o ytralt hydrog\'{e}n atom
i sin grunntilstand og et
proton $H^{+}$ i stor avstand fra hydrog\'{e}n atomet.

%
\begin{itemize}
%
\item[a)] Hvor stor er forskjellen mellom energien til $H_{2}^{+}$--ionet og
energien til et system best\aa ende av to protoner og et elektron med stor
gjensidig separasjon?

\item[b)] Hva blir forskjellen i energi mellom et system av et
$H_{2}^{+}$--ion
pluss et fritt elektron og et system med to hydrog\'{e}n atomer i sine
grunntilstander?

\item[c)] Finn ionisasjons energien for hydrog\'{e}n molekylet
$H_{2}$ n\aa r det
trengs 4,48 eV for \aa ~spalte dette molekylet til to
hydrog\'{e}n atomer som begge
er i grunntilstanden?
%
\end{itemize}
%

\subsubsection*{Oppgave 10.2}
%
To--atomige molekyler som holdes sammen av en kovalent binding har en potensiell
energi som funksjon av den interatom\ae re avstanden $R$ beskrevet ved Morse
potensialet
\begin{eqnarray*}
V(R) = D \left( \exp [-2a(R-R_{0})] - 2 \exp [-a(R-R_{0})] \right),
\end{eqnarray*}
der $D$, $a$ og $R_{0}$ er konstanter. For oksyg\'{e}n molekylet
$O_{2}$ finner man
eksperimentelt verdiene $D = 5,2$~eV, $R_{0} = 0,12$~nm og $a = 27$~nm$^{-1}$.

%
\begin{itemize}
%
\item[a)] Plott potensialet og vis at det har et minimum for $R = R_{0}$.

\item[b)] Hvor stor energi trengs for \aa ~spalte et molekyl beskrevet ved dette
potensialet?

\item[c)] Vis at sm\aa ~vibrasjoner av molekylet rundt likevektsavstanden
$R_{0}$ kan beskrives ved et harmonisk potensial p\aa ~formen $V = \frac{1}{2}
k (R-R_{0})^{2}$, og bestem konstanten $k$.

\item[d)] Beregn frekvensen og de kvantiserte energi niv\aa ene for denne
vibrasjonen. Massen til et oksyg\'{e}n atom er 16 u hvor den atom\ae re masseenheten
u er gitt som \mbox{1u = $1,66\cdot 10^{-27}$ kg}.

\item[e)] Hva blir n\aa ~spaltningsenergien for molekylet n\aa r det befinner
seg i grunntilstanden?
%
\end{itemize}
%

\subsubsection*{Oppgave 10.3}
%
\begin{itemize}
%
\item[a)] Vis at hamiltonoperatoren for bevegelsen omkring massesenteret til et system av
to partikler med masse $m_{1}$ og $m_{2}$ som har en relativ avstand $r$ og
gjensidig potensiell energi $V(r)$ er
\begin{eqnarray*}
\OP{H} = -\frac{\hbar^{2}}{2\mu } \nabla^{2} + V(r),
\end{eqnarray*}
der $\mu$ er systemets reduserte masse.
%
\end{itemize}
%

I sf\ae riske koordinater (romlige polarkoordinater) er
\begin{eqnarray*}
\nabla^{2} = \frac{1}{r^{2}} \frac{\partial }{\partial r} r^{2} \frac{\partial
}{\partial r} -\frac{1}{\hbar^{2} r^{2}} \OP{L}^{2},
\end{eqnarray*}
der $\OP{L}$ er operatoren for banespinnet. Vi antar forel\o pig
at systemet utvikler seg slik at den relativ avstanden er konstant lik $r_{0}$.

%
\begin{itemize}
%
\item[b)] Begrunn at hamiltonoperatoren for systemet n\aa r vi ser bort fra
massesenterbevegelsen p\aa ~en konstant energi n\ae r er gitt ved
\[
\OP{H} = \frac{\OP{L}^{2}}{2 \mu r_{0}^{2}}.
\]
%
\item[c)] Hvilken type bevegelse utf\o rer systemet?

\item[d)] Skriv ned systemets energi egenverdier E og angi de tilh\o rende
energi egenfunksjonene $\psi_{E}$ (Det kreves dog ikke eksplisitte uttrykk for
disse). Hvilke andre fysiske st\o rrelser kan samtidig ha skarpe verdier?

\item[e)] Hva forteller uttrykket $\mid \psi_{E} \mid^{2} sin\theta d\theta
d\phi$ om systemet?
%
\end{itemize}
%

Hvis systemet ovenfor er et toatomig molekyl, vil den kovalente bindingen ikke
v\ae re s\aa ~sterk at antagelsen om en konstant interatom\ae r avstand kan
opprettholdes. Sentrifugalkreftene som skyldes systemets rotasjon vil f\o re til
at bindingen strekkes, og dette vil gi opphav til en korreksjon til
hamilton operatoren. Den nye hamilton operatoren er gitt ved
\[
\OP{H} = \frac{\OP{L}^{2}}{2 \mu r_{0}^{2}} -
D\OP{L}^{4},
\]
der D er en konstant og $D\hbar^{2} << 1/2\mu r_{0}^{2}$.

%
\begin{itemize}
%
\item[f)] Finn energi egenverdiene for slike molekyler.

\item[g)] Finn frekvensene for de fotonene som kan emitteres (eller absorberes)
via elektrisk dipolstr\aa ling.
%
\end{itemize}
%
\subsubsection*{Oppgave 10.4, Eksamen V-1992}
%
Et diatomisk molekyl roterer med vinkelfart $\omega_r$ om en akse gjennom
massemiddelpunktet og vinkelrett p{\aa} forbindelseslinjen mellom
atomkjernene. Atomene har masse $m_1$ og $m_2$, og avstanden mellom
dem er  $R_0$.
%
\begin{itemize}
%
\item[a)] Finn treghetsmomentet $I_{cm}$ for molekylet om rotasjonsaksen,
uttrykt ved $R_0,\; m_1,$ og $m_2$.
Beregn det klassiske uttrykk for banespinnet $L$ om samme akse
og vis at rotasjons energien til molekylet blir
%
\[
E_{rot} = \frac{L^2}{2 I_{cm}}
\]
%
\end{itemize}
%
Vi skal n{\aa} behandle molekylet kvantemekanisk.
%
\begin{itemize}
%
\item[b)]  Sett opp
Hamilton operatoren for systemet og finn energi egenverdiene.
Finn ogs{\aa} et uttrykk for energi forskjellen mellom to nabo niv{\aa}er.
%
\item[c)] Et CO molekyl er observert til {\aa} absorbere et foton
med  frekvens $1,15  \times 10^{11}$~Hz. Dette svarer til en overgang
fra grunntilstanden i CO molekylet til den f{\o}rste eksiterte
rotasjons tilstand. Finn av dette molekylets treghetsmoment.
%
\end{itemize}
%
Et diatomisk molekyl kan ogs{\aa} vibrere om likevektstilstanden  $R_0$
langs forbindelseslinjen mellom atomkjernene.
Avviket fra likevekt for hver atomkjerne kaller vi $\xi_1$  og
$\xi_2$.
%
\begin{itemize}
%
\item[d)] Skiss\'{e}r den potensielle energien til molekylet som
funksjon av avstanden mellom atomkjernene.
Vis at dette for sm{\aa} avvik fra likevekt tiln{\ae}rmet
f{\o}rer til Schr\"{o}dingerligningen
%
\[
-\frac{\hbar^2}{2 \mu} \frac{d^2}{d\xi^2} \psi(\xi)
	+ \frac{1}{2} K \xi^2 \psi(\xi) = E_{vib} \psi(\xi),
\]
%
Forklar st{\o}rrelsene som inng{\aa}r.
%
\item[e)] Skriv opp energi egenverdiene $E_{vib}$ i
vibrasjonsbevegelsen uttrykt ved
st{\o}rrelsene i ligningen ovenfor. Sett ogs{\aa} opp uttrykket
for vinkelfrekvensen $\omega_v$ for vibrasjonen.
Hva blir energi forskjellen mellom to nabo niv{\aa}er? Hvorfor er laveste
vibrasjonsenergi st{\o}rre enn null?
%
\item[f)] Still opp et uttrykk for molekylets totale energi
egenverdier for en samtidig rotasjon og vibrasjons bevegelse.
%
\end{itemize}
%
Eksperimentelt er vinkelfrekvensen for et CO molekyl m{\aa}lt
til $\omega_v = 4,03 \times 10^{14}\; s^{-1}$.
En gass med CO molekyler blir utsatt for elektromagnetisk
str{\aa}ling med frekvenser i omr{\aa}det
$\nu = (1,0 \times 10^{12} - 1.0 \times 10^{14}) \; s^{-1}$.
%
\begin{itemize}
%
\item[g)] Hvilke utvalgsregler m{\aa} v{\ae}re oppfylt
n{\aa}r molekylene ved en elektrisk dipol overgang forandrer
rotasjons/vibrasjons	tilstand?
%
\item[h)] Gi eksempler p{\aa} tillatte absorbsjons
overganger og beregn den tilh{\o}rende absorbsjons energien.
%

\end{itemize}
%
%
\subsubsection*{L\o sning}
%
\begin{itemize}
% 
\item[a)] Vi velger origo i systemets tyngdepunkt. Tyngdepunkt- og
relativ- koordinatene er da bestemt av ligningene
% 
\[
-m_1 r_1 + m_2 r_2 = 0, \quad
r_1 + r_2 = R_0
\]
% 
som gir 
% 
\[ 
r_1 = \frac{m_2}{m_1 + m_2} R_0,  \quad
r_2 = \frac{m_1}{m_1 + m_2} R_0
\]
%
og systemetes treghetsmoment blir 
% 
\[
I_T = m_1 r_1^2  + m_2 r_2^2 = \frac{m_1 m_2}{m_1 + m_2} R_0^2 
    = \mu R_0^2
\]
% 
hvor $\mu$ er den reduserte masse for de to partiklene.

Det klassiske uttrykket for banespinnet er gitt ved 
$\vec{L} = \vec{r} \times \vec{p}$. I rotasjon om tyngdepunktet blir
% 
\begin{eqnarray}
L &=& r_1 m_1 v_1 + r_2 m_2 v_2 
= r_1 m_1 r_1 \omega + r_2 m_2 r_2\omega\\
 &=& ( m_1 r_1^2 + m_2 r_2^2 ) \omega
  = \frac{m_1 m_2}{m_1 + m_2} R_0^2 \omega\\
 &=& \mu R_0^2 \omega = I_T \omega
\end{eqnarray}
% 
Det klassiske uttrykket for systemets rotasjonsenergi blir 
% 
\begin{eqnarray} 
H_{kl} &=& \frac{1}{2} m_1 v_1^2 + \frac{1}{2} m_2 v_2^2
        =  \frac{1}{2} \mu R_o^2 \omega^2\\
       &=& \frac{1}{2\mu R_0^2} L^2 = \frac{1}{2I_T} L^2
\end{eqnarray}
%
\item[b)] Kvantemekanisk f�r vi 
% 
\[
 \OP{H} = \frac{1}{2I_T} ( \OP{\vec{L}})^2
\] 
% 
med  egenverdiligning
% 
\[
\frac{1}{2 I_T} (\OP{\vec{L}})^2 \psi
       = E_l \psi = \frac{1}{2 I_T} l ( l + 1) \hbar^2 \psi
\]
% 
med kvantetallet $ l = 0,1,2,\ldots$. Energiavstanden mellom to
nabo niv�er blir 
% 
\[
\Delta E_l = E_l - E_{l - 1} = \frac{\hbar^2}{2 I} [ l(l+1) - (l-1)l]
                            = \frac{\hbar^2}{I} l
\]
% 
\item[c)] Overgangen fra grunntilstanden til det f�rste 
eksiterte $l =1$  tilstanden gir 
% 
\[ 
E_f = h f = E_1 - E_0 = \frac{\hbar^2}{I_T}
\]
% 
sombestemmer systemets treghetsmoment til 
%
\[
I_T = \frac{h}{4 \pi^2 f} = 1,46 \times 10^{-46} \; \mbox{kg m}^2
%%	I = \frac{h}{4 \pi^2 f} = 1,46 \times 10^{-46}\; \mbox{kg m^2}
\]
%
\item[d)] Det klassiske likevektspunktet for molekylet bestemmes av 
% 
\[
\left ( \frac{d V(R)}{dR} \right )_{R = R_0} 
     = - \alpha \beta e^{-\beta R_0} + \frac{k}{R_0^2} = 0
\]
% 
Vi velger $V(R_0) = 0$ og rekkeutvikler omkring  $R = R_0$ (det svarer
til $\xi = 0$)
%
\[
V(\xi) \approx \frac{1}{2} \xi^2 \left (\frac{dV}{d\xi^2} \right
)_{\xi= 0} = \frac{1}{2} k \xi^2
\]
% 
og Schr\"{o}dinger ligningen f�r formen 
% 
\[
- \frac{\hbar^2}{2 \mu} \frac{d^2}{dr^2} \psi(r) + \frac{1}{2} k \xi^2 \psi(r)
= E_{vib} \psi(r),
\]
%
hvor $\mu$ er den reduserte masse og $E_{vib}$ energi egenverdien.
%
\item[e)]  L�sningen f�lger fremgangsm�ten for en harmonisk oscillator 
% 
\[
E_{vib}( n ) = \left ( n + \frac{1}{2} \right ) \hbar \omega
\]
%
med $\omega = \sqrt{\frac{k}{\mu}}$.
Energi avstanden mellom to nabo niv{\aa} er
% 
\[
E_{vib}(n) - E_{vib}(n - 1) = \hbar \omega
\]
%
Systemet f{\aa}r en nullpunkts energi
%
\[ 
E_{vib}(\mbox{grunntilstand}) = \frac{1}{2} \hbar \omega > 0.
\]
%
\item[f)] Det totale energien  blir 
% 
\[
E = E_{rot} + E_{vib} = \frac{\hbar^2}{2 I_T} l ( l + 1 )
		       + ( n + 1 / 2 ) \hbar \omega
\]
%
\item[g)]  Utvalgsreglene for elektrisk dipol overganger
er $ \Delta n = \pm 1, \;\; \Delta l = \pm 1$  og
$\Delta m_l = 0, \pm 1$.
%
\item[h)]Eksempel p� overgang
%
\[
( n = 0, l = 0 ) \longrightarrow( n = 1, l = 1 ):
\;\; \Delta E = 0,2657\; eV
\]
%
\end{itemize}


\subsubsection*{Oppgave 10.5, Eksamen H-1993}
%
Vi skal i denne oppgaven studere rotasjonstilstander i
et to--atomig molekyl.
%
\begin{itemize}
%
\item[a)] Gi en fysikalsk beskrivelse av et
slikt roterende system og sett opp uttrykket for
energi egenverdiene. Forklar de forskjellige st{\o}rrelsene
som inng{\aa}r. Skiss\'{e}r energispektret for rotasjonstilstandene
med tilh{\o}rende kvantetall.
%
\end{itemize}
%
Hvis et slikt to--atomig molekyl er i en  eksitert tilstand
gitt ved kvantetallet $L$ for det totale banespinn,
vil systemet deeksitere og g{\aa} over til en lavere tilstand med
totalt banespinn $L - 1$ ved utsendelse av et foton.
%
\begin{itemize}
%
\item[b)] Vis at frekvensspektret som oppst{\aa}r
n{\aa}r et slikt roterende to--atomig molekyl de--eksiterer
ned til grunntilstanden, er ekvidistant, dvs. at spektral
linjene har konstant avstand.
%
\item[c)] For molekylet HCl tilsvarer denne 
frekvensdifferansen en  energi $2.62 \times 10^{-3}$~eV.
Hva er treghetsmomentet $I$ for dette molekylet?
%
\item[d)] Anta at masse for et H--atom er $ 1\, u$ og massen
for Cl er $35\, u$ og beregn avstanden mellom disse
to atomene i molekylet.
%
\end{itemize}
%
\subsubsection*{Kort fasit}
\begin{itemize}
%
\item[a,b)] Se l{\ae}reboka {\sl Brehm and Mullin: Introduction to the
           structure of matter}, avsnitt 10.6 og 10.8 fig~10.16
%
\item[c)] For overganger mellom den f{\o}rste eksiterte tilstand
med $L =1$ og grunntilstanden kan vi bestemme treghetmoment parameteren
ved
%
\[ \Delta E = \frac{2 \hbar^2}{2 I} \; \longrightarrow \;
I = \frac{\hbar^2}{\Delta E} = \underline{2,65 \times 10^{-47}\;kg\;m^2}.
\]
%
\item[d)] Avstanden mellom H og CL atomene bestemmes ved
%
\[
I = \mu r_0^2 = \frac{M_H M_{Cl}}{M_H + M_{Cl}} r_0^2
= \frac{1 u 35 u}{36 u} r_0^2\; \longrightarrow \;r_0
= \underline{1,26 \times 10^{-10}\;m}.
\]
%
\end{itemize}
%
\subsubsection*{Oppgave 10.6, Eksamen H-1994}
%

Vi har et  system av to protoner og
et elektron. Dette svarer til partiklene i molekyl--ionet  H$_2^{+}$.
Protonene regner vi som klassiske partikler som er i ro,
avstanden mellom dem kaller vi  $R$. Elektronets
avstand fra protonene kaller vi $r_a$ og $r_b$.
I oppgaven f{\aa}r vi bruk for {\aa} vite at ionisasjonsenergien
for et hydrog\'{e}n atom i grunntilstanden er 13.6~eV.
%
\begin{itemize}

%
\item[a)] Lag en skisse som beskriver systemet og hvor $R$, $r_a$
og $r_b$ inng{\aa}r. Skriv opp  den kvantemekaniske Hamilton
operatoren $\OP{H}$ for systemet. Vi ser her bort fra den
kinetiske energien for protonene.
%
\end{itemize}
%
Vi antar n{\aa} at protonene er i ro i en innbyrdes avstand $R$
og lar $E$ betegne den \underline{laveste} energi egenverdien av $\OP{H}$
for denne avstanden. $E$ blir da en funksjon av $R$, $E = E(R)$.
%
\begin{itemize}
%
\item[b)] Skiss\'{e}r kurven $E(R)$ som funksjon av $R$.
Hvilken verdi har $E$ n{\aa}r $R \longrightarrow \infty$ ?
N{\aa}r $R$ velges meget liten (i forhold til elektronets
bohr--radius) er $E$ tiln{\ae}rmet gitt ved
\[
E = \frac{e^2}{4 \pi \varepsilon_0 R} - 54,4\;\mbox{eV}
\]
Forklar dette.
%
\item[c)] Elektronets b{\o}lgefunksjon  i $H_2^{+}$ ionet
vil enten ha like eller odde paritet.
Forklar hva dette betyr. Tiln{\ae}rmet kan vi beskrive
elektronets b{\o}lgefunksjon ved {\aa} bruke hydrog\'{e}n
atomets elektron b{\o}lgefunksjoner $\psi_{nlm}(\vec{r})$.
Vis hvordan dette gj{\o}res for de to tilfellene
med henholdsvis like og odde paritet.
%
\item[d)] Diskut\'{e}r hvilken av de to mulighetene som svarer
til grunntilstanden og tegn en skisse som viser
sannsynlighetstettheten for elektronet langs molekylaksen
for hver av de to tilfellene.
%
\end{itemize}
%
\subsubsection*{Kort fasit}
\begin{itemize}
%
\item[a)] Se l{\ae}reboka {\sl Brehm and Mullin: Introduction to the
           structure of matter}, avsnitt 10.1 og fig~10.2
%
\item[b)] $E(R) < 0$ og vokser eksponentielt mot grensen
-13.6~eV for $R \rightarrow \infty$. Dette svarer til at elektronet er
bundet til et av protonene som et hydrog\'{e}natom.
$E(R) \rightarrow -\infty$ for  $R \rightarrow 0$ pga. coulomb frast{\o}tning
mellom protonene.

For $R$ liten og endelig f{\aa}r vi coulomb frast{\o}tning
$(e^2 /4 \pi \varepsilon_0 R)$, mens elektronets
bindingsenergi til protonene svarer til 4 ganger energien i et hydrog\'{e}n
atom $4 \cdot (-13,6)$~eV.
%
\item[c)d)] Se l{\ae}reboka {\sl Brehm and Mullin: Introduction to the
           structure of matter}, avsnitt 10.1
%
\end{itemize}
%
\subsubsection*{Oppgave 10.7}
%
En model for Natriumklorid (NaCl) er at molekylet best�r av to deler -- et positivt NA$^+$ 
og et negativt Cl$^-$ ion. Anta at det tiltrekkende potensialet mellom ionene er elektrostatisk: 
$-e^2 / r$, hvor $r$ er avstanden mellom ionene. Anta videre at det er et repulsivt ledd i den potensielle
energien $V(r)$ som �ker sterk n�r elektronstrukturene kommer i kontakt med hverandre og er av formen:
$+A / r^n$. Den totale potensielle energien blir  da $V(r) = -e^2 / r + A /r^n$.
Konstantene $A$ og $n$ skal bestemmes.
%
\begin{itemize}
%
\item[a)] Finn likevektsavstanden $r_0$ mellom de to ionene i NaCl molekylet (minimum potensiell 
energi). Finn $V_{min}$  for $r = r_0$.
%
\item[b)] Beregn den annenderiverte av $V(r)$ for $r = r_0$. Dette kan betraktes som fj�rkonstanten i 
en harmonisk oscillator approksimasjon for den potensielle energien i n�rheten av likevektspunktet.
%
\item[c)] Bruk de reduserte masser for isotopene Na$^{23}$ og Cl$^{35}$ og beregn energiavstandene av 
vibrasjonsniv�ene i den harmoniske approksimasjonen. Finn ogs� nullpunktsenergien for 
vibrasjonsbevegelsen.
%
\item[d)] Likevektsavstanden i Na$^{23}$Cl$^{35}$   er m�lt til 2,5~�, og frekvensen av 
str�lingen fra en overgang mellom to nabo vibrasjonsniv�er er $1,14 \times 10^{13}$Hz.
Bruk disse verdiene til � bestemme konstantene $A$ og $n$.
%
\item[e)] Bruk resultatet ovenfor til � beregne frigj�ringsenergien (i eV) for NaCl molekylet i grunntilstanden
og til et fritt Na$^+$ ion og Cl$^-$ ion.
%
\end{itemize}


%%%%%%%%%%%%%%%%%%%%%%%%%%%%%%%%%%%%%%%%%%%%%%%%

\clearemptydoublepage

\chapter{Quantum statistics}\label{chap:lasere}

\section{Uskilbarhet og kvantestatistikk}

Dersom vi \o nsker \aa\ beskrive systemer som best\aa r av mange partikler,
f.eks.~en gass av atomer med $10^{23}$ partikler, er det klart at vi ikke 
kan l\o se Schr\"odingers likning for et slikt system. Bare det \aa\ skrive ned
begynnelsesposisjonene til hver partikkel utifra et tre-dimensjonalt koordinat
system vil v\ae re en tiln\ae rmet h\aa pl\o s opgave. Heldigvis er det ikke 
n\o dvendig med en detaljert kunnskap om systemet for \aa\ kunne bestemme
st\o rrelser som midlere energi og posisjon. Grunnen er at det er sammenheng 
mellom observerbare st\o rrelser og den mest sannsynlige oppf\o rselen til
et system slik at en statistisk beskrivelse gir en meget bra approksimasjon.
Betingelsen er at systemet har nok partikler og frihetsgrader til at en statistisk beskrivelse er gyldig.
Mange av disse begrepene gjennomg\aa s i mer detalj i kurset FYS 114.
Her skal vil n\o ye oss med \aa\ betrakte f\o rst et system av klassiske
partikler i termisk likevekt med omgivelsene ved en gitt temperatur
$T$. Det rekker langt p\aa\ vei n\aa r vi skal pr\o ve \aa\ forst\aa\
mekanismene bak stimulert emisjon og eksemplet v\aa rt med Helium-Neon laseren.

Et slikt system kan beskrives med teorier fra statistisk fysikk. For en gass
av klassiske partikler, er gitte tilstander med energi $E$ beskrevet av 
en sannsynlighetsfordelingsfunksjon som kalles Boltzmannfordelingen. Vi har 
et eksempel tidligere i dette kurset, nemlig Maxwells hastighetsfordeling.
Kjenner vi 
sannsynligheten partiklene har for en bestemt energi kan vi beregne 
st\o rrelser som f.eks.~systemets midlere energi.

Vi ser f\o rst p\aa\ et system av klassiske partikler, som i praksis
betyr at vi ikke skiller mellom bosoner og fermioner og betrakter 
partiklene som skilbare. Vi avgrenser oss til to energier,
$E_1$ og $E_2$ og antar at overgangssannlygheten for at partiklene kan g\aa\
fra tilstand 1 til tilstand 2 er gitt ved $T_{1\rightarrow 2}$. Tilsvarende
har vi en overgangssannsynlighet fra tilstand 2 til tilstand $1$ gitt ved
$T_{2\rightarrow 1}$. Dersom det er $n_1$ partikler i tilstand 1 og $n_2$ 
i tilstand 2 er de totale overgangssannsynlighetene gitt ved 
$n_1T_{1\rightarrow 2}$ og $n_2T_{2\rightarrow 1}$.
Dersom vi antar at det er en balanse ved en gitt temperatur mellom de
to overgangssannsynligheten, det s\aa kalte prinsippet om detaljert balanse,
kan vi sette
\be
n_1T_{1\rightarrow 2}=n_2T_{2\rightarrow 1},
\ee
eller
\be
\frac{n_1}{n_2}=\frac{T_{2\rightarrow 1}}{T_{1\rightarrow 2}}.
\ee
Boltzmannfordelingen er gitt ved
\be
     n_{1,2} = Ae^{-E_{1,2}/k_BT},
\ee
hvor $A=e^{-\alpha(T)}$ 
er en normeringskonstant som avhenger av temperaturen 
og $k_B$ er Boltzmanns konstant. 
Innsatt i forholdet mellom sannsynlighetene finner vi
\be 
   \frac{n_1}{n_2}=\frac{e^{-E_{1}/k_BT}}{e^{-E_{2}/k_BT}}=
                   \frac{T_{2\rightarrow 1}}{T_{1\rightarrow 2}}.
\ee
Dette uttrykket gjelder for klassiske partikler.
Dersom partiklene antas \aa\ ikke vekselvirke (uavhengige partikler)
kan vi anta at sannsynligheten $P$ for \aa\ finne et system av $n$ partikler 
med energi $E$ er gitt ved produktet av alle en-partikkel sansynnlighetene
$P_1$ slik at vi finner
\be
    P_n=(P_1)^n.
\ee

Dersom systemet v\aa rt best\aa r av $n$ identiske 
bosoner i samme tilstand m\aa\ vi passe p\aa\
at b\o lgefunksjonen skal v\ae re symmetrisk. 
Det gir en faktor $n!$  i tillegg dersom
alle $n$ partiklene er i samme tilstand slik at
\be
    P^{\mathrm{boson}}_n=n!P_n=n!(P_1)^n.
\ee
Legger vi til et boson har vi
\be
    P^{\mathrm{boson}}_{n+1}=(n+1)!P_{n+1}=(n+1)n!(P_1)^nP_1,
\ee
eller
\be
    P^{\mathrm{boson}}_{n+1}=(n+1)!P_{n+1}=(n+1)P_1P^{\mathrm{boson}}_n.
\ee
For \aa\ se dette, kan det l\o nne seg \aa\ g\aa\ tilbake til definisjonen
av en symmetrisk b\o lgefunksjon for identiske men uskilbare partikler.
For to bosoner har vi f\o lgende b\o lgefunksjon
\be
   \Psi_S({\bf r}_1,{\bf r}_2)=\frac{1}{\sqrt{2}}
\left(\psi_{\alpha}(1)\psi_{\alpha}(2)+\psi_{\beta}\psi_{\beta}(2)\right),
\ee 
som gir dersom $\alpha=\beta$
\be
   \Psi_S({\bf r}_1,{\bf r}_2)=
\frac{2}{\sqrt{2}}\psi_{\alpha}(1)\psi_{\alpha}(2),
\ee 
slik at sannsynligheten for \aa\ finne dette to-partikkelsystemet 
i en slik tilstand er
\be
   \Psi_S^*({\bf r}_1,{\bf r}_2)\Psi_S({\bf r}_1,{\bf r}_2)=
2\psi^*_{\alpha}(1)\psi^*_{\alpha}(2)\psi_{\alpha}(1)\psi_{\alpha}(2),
\ee 
eller to ganger det klassiske resultatet.
For tre partikler har vi seks ledd
\begin{eqnarray}
\Psi_S({\bf r}_1,{\bf r}_2,{\bf r}_3)&=
\frac{1}{\sqrt{3!}}\left[
\psi_{\alpha}(1)\psi_{\beta}(2)\psi_{\gamma}(3)+
\psi_{\beta}(1)\psi_{\gamma}(2)\psi_{\alpha}(3)+
\psi_{\gamma}(1)\psi_{\alpha}(2)\psi_{\beta}(3)+\right. \nonumber \\
&\left.\psi_{\gamma}(1)\psi_{\beta}(2)\psi_{\alpha}(3)+
\psi_{\beta}(1)\psi_{\alpha}(2)\psi_{\gamma}(3)+
\psi_{\alpha}(1)\psi_{\gamma}(2)\psi_{\beta}(3)
\right],
\end{eqnarray}
og settes $\alpha=\beta=\gamma$ har vi 
\be
\Psi_S({\bf r}_1,{\bf r}_2,{\bf r}_3)=
\frac{6}{\sqrt{3!}}
\psi_{\alpha}(1)\psi_{\alpha}(2)\psi_{\alpha}(3),
\ee
og den tilsvarende sannsynligheten blir dermed 
\be
\Psi_S^*({\bf r}_1,{\bf r}_2,{\bf r}_3)\Psi_S({\bf r}_1,{\bf r}_2,{\bf r}_3)=
3!
\psi_{\alpha}^*(1)\psi_{\alpha}^*(2)\psi_{\alpha}^*(3)
\psi_{\alpha}(1)\psi_{\alpha}(2)\psi_{\alpha}(3),
\ee
eller $3!=6$ ganger det klassiske resultatet.
Tilstedev\ae relsen av et boson i en gitt kvantetilstand forsterker 
dermed sannsynligheten for at flere bosoner vil finnes i samme tilstand.
Formen p\aa\ b\o lgefunksjonen, dvs.~v\aa rt krav om at den skal v\ae re
symmetrisk, f\aa r dermed fundamentale konsekvenser for materiens
oppf\o rsel. Denne f\o ringen
gir bla.~opphav til en av naturens mest spektalu\ae re faseoverganger,
Bose-Einstein kondensasjon (BEC), 
se f.eks.~http://www.Colorado.EDU/physics/2000/bec/
for mer detaljer vedr\o rende  BEC med atomer.
 
Vi kan bruke disse resultatene n\aa r vi skal studere forholdet 
\be
n_1T^{\mathrm{boson}}_{1\rightarrow 2}=n_2T^{\mathrm{boson}}_{2\rightarrow 1}.
\ee
Her er $n_1$ og $n_2$ midlere antall bosoner i tilstandene 1 og 2 og 
$T^{\mathrm{boson}}_{1\rightarrow 2}$ og 
$T^{\mathrm{boson}}_{2\rightarrow 1}$ deres respektive 
overgangsrater.
Disse overgangsratene kan vi uttrykke vha.~de klassiske ratene ved simpelthen
\aa\ multiplisere det klassiske resultatet med en faktor $(1+n)$.
Dvs.~at siden det i snitt er $n_2$ bosoner i  tilstand 2 gitt overgangen
$1\rightarrow 2$ s\aa\ er 
$T^{\mathrm{boson}}_{1\rightarrow 2}$ en faktor $(1+n_2)$ st\o rre enn
$T_{1\rightarrow 2}$.
Vi har dermed 
\be
    T^{\mathrm{boson}}_{1\rightarrow 2}=(1+n_2)T_{1\rightarrow 2},
\ee
og
\be
    T^{\mathrm{boson}}_{2\rightarrow 1}=(1+n_1)T_{2\rightarrow 1}.
\ee
Bruker vi prinsippet om detaljert balanse finner vi
\be
   n_1(1+n_2)T_{1\rightarrow 2}=n_2(1+n_1)T_{2\rightarrow 1},
\ee
som kan omskrives
\be 
    \frac{n_1(1+n_2)}{n_2(1+n_1)}=\frac{e^{-E_{1}/k_BT}}{e^{-E_{2}/k_BT}}.
\ee
Siste uttrykk kan skrives som
\be 
    \frac{n_1}{1+n_1}e^{E_{1}/k_BT}=\frac{n_2}{1+n_2}e^{E_{2}/k_BT},
\ee
og vi ser at venstre side er uavhengig av tilstand 2 og h\o yre side er uavhengig av tilstand 1. Den felles verdien for begge sider kan derfor ikke involvere
egenskaper som er spesielle for enten den eller den andre tilstanden. 
Det er kun en st\o rrelse som er felles, og det er temperaturen $T$.
Vi setter derfor begge uttrykk lik en felles normeringskonstant
$e^{-\alpha(T)}=e^{-\alpha}$ slik at
\be 
    \frac{n_1}{1+n_1}e^{E_{1}/k_BT}=e^{-\alpha},
\ee
som gir
\be 
    \frac{n_1}{1+n_1}=e^{-(\alpha+E_{1}/k_BT)},
\ee
eller
\be 
   n_1\left[1-e^{-(\alpha+E_{1}/k_BT)}\right]=e^{-(\alpha+E_{1}/k_BT)},
\ee
som resulteter i 
\be 
   n_1 =
   \frac{e^{-(\alpha+E_{1}/k_BT)}}{1-e^{-(\alpha+E_{1}/k_BT)} }.
\ee
Det siste uttrykket kan omskrives til
\be 
   n_{\mathrm{boson}}(E,T) =
   \frac{1}{e^{(\alpha+E/k_BT)}-1 },
\ee
som er fordelingsfunksjone for bosoner som gir det gjennomsnittlige antall
bosoner i en gitt tilstand med energi $E$ ved temperatur $T$.

Dersom vi ser p\aa\ fermioner, m\aa\ vi passe p\aa\ at den totale b\o lgefunksjonen skal v\ae re antisymmetrisk med tanke p\aa\ ombytte av to partikler.
Det medf\o rer at 
\be
    T^{\mathrm{fermion}}_{1\rightarrow 2}=(1-n_2)T_{1\rightarrow 2},
\ee
og
\be
    T^{\mathrm{fermion}}_{2\rightarrow 1}=(1-n_1)T_{2\rightarrow 1}.
\ee
Gjentar vi samme beregning som vi foretok for bosoner finner vi
fordelingsfunksjonen for fermioner gitt ved
\be 
   n_{\mathrm{fermion}}(E,T) =
   \frac{1}{e^{(\alpha+E/k_BT)}+1 },
\ee
som gir midlere antall fermioner i en tilstand $E$ i likevekt ved en temperatur
$T$.    

Det er en egenskap ved disse tre fordelingsfunksjonene som er verdt \aa\
huske. Dersom energien for f.eks.~en eksitert tilstand er mye st\o rre enn
$k_BT$, dvs.~$E \gg k_BT$ ser vi at
\be 
    n_{\mathrm{fermion}}(E,T)\approx n_{\mathrm{boson}}(E,T) \approx 
    n_{\mathrm{boltzmann}}(E,T) \ll 1,
\ee
som betyr ogs\aa\ antall partikler per kvantetilstand $E$ mer mye mindre
enn 1.

Som eksempel kan vi tenke oss en gass hydrogenatomer ved romtemperatur,
$T=300$ K. Setter vi inn Boltzmanns konstant 
$k_B=8.617\times 10^{-5}$ eVK$^{-1}$ finner vi at 
$k_BT=0.026$ eV. Dersom hydrogenatomene er i sin f\o rste
eksiterte tilstand med energi $E_2=-3.4$ eV, finner vi at forholdet
mellom grunntilstanden med $E_1=-13.6$ eV 
og den f\o rste eksistert tilstanden $E_2$ er gitt ved
n\aa r vi bruker Boltzmanns fordeling
\be 
   \frac{n_2}{n_1}=\frac{e^{3.4/0.026}}{e^{13.6/0.026}}\approx 0!
\ee
Skal vi ha et tall som er lite men forskjellig fra 
null b\o r vi finne overgangen med mindre energiforskjeller dersom vi 
\o nsker \aa\ studere overganger ved romtemperatur.
Vi skal se n\ae rmere p\aa\ det i siste avsnitt i eksemplet med
Helium-Neon laseren.

\section{Emisjon, absorpsjon og stimulert emisjon}

Vi s\aa\ i forrige avsnitt at det relative antallet partikler
per kvantetilstand ved to forskjellige temperaturer for et system
i termisk likevekt  kan approksimeres ved Boltzmann fordelingen dersom vi
se p\aa\ en fortynnet gass av enten bosoner eller fermioner.
Vi skal benytte oss av dette i v\aa r forst\aa else av laseren, som 
er en forkortelse for {\bf l}ight {\bf a}mplification by {\bf s}timulated
{\bf e}mission of {\bf r}adiation, lysforsterkning ved stimulert emisjon
av str\aa ling. En maser er det tilsvarende systemet, men da med 
str\aa ling i mikrob\o lgeomr\aa det til det elektromagnetiske
spektrum.

For \aa\ forst\aa\ laseren trenger vi \aa\ skille mellom tre typer
prosesser, spontan emisjon, stimulert absorpsjon og stimulert emisjon
av eletromagnetisk str\aa ling. Vi har sett p\aa\ den f\o rste prosessen
tidligere. 
Figur \ref{fig:emisjonab}
\begin{figure}[h]
\setlength{\unitlength}{1mm}
   \begin{picture}(100,100)
   \put(-10,-50){\epsfxsize=16cm \epsfbox{emisjonab.ps}}
   \end{picture}
\caption{Illustrasjon av (a) spontan emisjon, (b) stimulert
absorpsjon og (c) stimulert emisjon av elektromagnetisk
str\aa ling. \label{fig:emisjonab}}
\end{figure}
viser disse prosessene.
Ved spontan emisjon er atomet i den \o vre tilstanden $E_2$ og henfaller
til tilstanden med energi $E_1$ med utsending av et foton med
frekvens $\nu =(E_2-E_1)/h$. Den typiske levetid for en eksitert tilstand
er ganske liten, typisk mindre enn $10^{-8}$ s. Dersom vi skal
lage en laser er vi avhengig av en tilstand med lengre levetid,
s\aa kalte metastabile tilstander, her kan levetida v\ae re opp til
$10^{-3}$ s. 

I stimulert absorpsjon stimuler et innkommende foton med frekvens
$\nu =(E_2-E_1)/h$ atomet til \aa\ gj\o re en overgang fra tilstanden
$E_1$ til tilstanden $E_2$.
I stimulert emisjon derimot, er atomet i tilstanden $E_2$.
Et innkommende foton med frekvens $\nu =(E_2-E_1)/h$ stimulerer atomet 
i tilstanden $E_2$ til \aa\ sende ut et foton med samme frekvens
samtidig som det henfaller til tilstand $E_1$. Vi f\aa r da 
to utkommende fotoner med samme frekvens, en forsterkning dermed.

Disse tre prosessene vil forekomme med en viss sannsynlighet
og forholdet mellom de respektive sannsynlighetene er avgj\o rende
for virkem\aa ten til en laser. Vi antar at vi har en gass av atomer
ved gitt temperatur og gitt antall partikler.
Spektret til den elektromagnetiske str\aa lingen er gitt ved
$\rho(\nu)$.
Vi begrenser oss til en to-niv\aa\ model med energier $E_1$ og 
$E_2$, hvor $E_2 > E_1$. Vi antar ogs\aa\ at det finnes 
$n_1$ atomer med energi $E_1$ og $n_2$ atomer med energi $E_2$.

For spontan emisjon har vi ikke tilstede elektromagnetisk
str\aa ling f\o r emisjon. Vi antar at denne overgangen kan skrives
vha.~en overgangssannsynlighet $A_{21}$ som avhenger av tilstandene
1 og 2.
For stimulert absorpsjon finnes det en bestemt energifordeling for
fotonene (som er bosoner). Vi skal vise at regningen v\aa r gir oss
Plancks fordeling for frekvensspekretet til fotonene, dvs.~frekvensefordelingen for et svart legeme.
Vi setter prosessen for stimulert absorpsjon lik
\be
   T_{1\rightarrow 2} = B_{12}\rho(\nu).
\ee
Faktoren $B_{12}$ er som $A_{12}$ en faktor som avhenger av tilstandene
1 og 2. 

For stimulert emisjon m\aa\ vi ogs\aa\ ta hensyn til at det kan 
foreg\aa\ spontan emisjon, det betyr at overgangssannsynligheten for
stimulert emisjon er gitt ved 
\be
   T_{2\rightarrow 1} = A_{21}+B_{21}\rho(\nu).
\ee

Setter vi opp likningene for detaljert balanse mellom absorpsjon
og
dermed overgangen $�\rightarrow 2$ og total emisjon fra $2\rightarrow 1$ 
har vi
\be
n_1T_{1\rightarrow 2}=n_2T_{2\rightarrow 1},
\ee
som gir
\be
 n_1B_{12}\rho(\nu)=n_2(A_{21}+B_{21}\rho(\nu)),
\ee
og l\o ser vi med tanke p\aa\ $\rho(\nu)$ finner vi
\be
\rho(\nu)= \frac{A_{21}/B_{21}}{(n_1/n_2)(B_{12}/B_{21})-1},
\ee
og med Boltzmann fordelingen 
\be
    \frac{n_1}{n_2}=e^{(E_2-E_1)/k_BT}=e^{h\nu/k_BT},
\ee
finner vi 
\be
   \rho(\nu)= \frac{A_{21}/B_{21}}{e^{h\nu/k_BT}(B_{12}/B_{21})-1}.
\ee
Denne frekvensfordelingensfunksjonen m\aa\ v\ae re konsistent med
Plancks str\aa lingslov for et svart legeme som er
gitt ved
\be
   \rho(\nu)= \frac{8\pi h\nu^3}{c^3}\frac{1}{e^{h\nu/k_BT}-1}.
\ee
Det gir
\be
    \frac{B_{12}}{B_{21}}=1,
\ee
og 
\be
   \frac{A_{21}}{B_{21}}=\frac{8\pi h\nu^3}{c^3}.
\ee
Det var Einstein som utledet disse forholdene i 1917. Derfor kalles
koeffisientene $A$ og $B$ for Einstein koeffisientene..
Vi finner kun forholdet mellom koeffisientene. For \aa\ beregne
de eksakte verdiene m\aa\ vi foreta en kvantemekanisk beregning
hvor vi benytter b\o lgefunksjonen for \aa\ beregne en kvantemekanisk
forventningsverdi.

Det er mye interessant fysikk i de to siste likningene. Vi ser at 
koeffisientene for stimulert emisjon og absorpsjon er like og at forholdet
mellom spontan emisjon og stimulert emisjon varierer som funksjon av 
$\nu^3$. Det betyr at jo st\o rre eenrgiforskjellen er mellom to
tilstander  desto st\o rre sannsynlighet har vi for spontan emisjon.
Vi kan alternativt uttrykket dette forholdet vha.
\be
 \frac{A_{21}}{B_{21}\rho(\nu)}=e^{h\nu/k_BT}-1.
\ee
Dersom $h\nu \gg k_BT$ for atomer i termisk likevekt ved en temperatur
$T$, s\aa\ vil spontan emisjon dominere over stimulert emisjon.
Dette er tilfelle for de fleste atomer og molekyler. 
Stimulert emisjon blir viktig dersom $h\nu \approx k_BT$
eller $h\nu \ll k_BT$.

Vi kan n\aa\ uttrykke forholdet mellom total emisjon og absorpsjon som
\be
  \frac{\mathrm{emisjonsrate}}{\mathrm{absorpsjonsrate}}=
  \frac{n_2(A_{21}+B_{21}\rho(\nu))}{n_1B_{12}\rho(\nu)},
\ee
som resulterer i 
\be
  \frac{\mathrm{emisjonsrate}}{\mathrm{absorpsjonsrate}}=
  \left[\frac{A_{21}}{B_{21}\rho(\nu)}+1\right]\frac{n_2}{n_1}.
\ee
Dersom $h\nu \approx k_BT$
eller $h\nu \ll k_BT$ finner vi 
\be
  \frac{\mathrm{emisjonsrate}}{\mathrm{absorpsjonsrate}}\approx
  \frac{n_2}{n_1}.
\ee

Dersom vi har termisk likevekt m\aa\ vi kunne anta at $n_1 > n_2$.
Kan vi ha flere atomer i tilstandene $n_2$ vil emisjon dominere 
over absorpsjon. Det betyr at den elektromagnetiske str\aa lingen
med frekvens $\nu=(E_2-E_1)/h$ vil forsterkes. mer kommer ut enn det som 
sendes inn. Men, antallet atomer i den eksiterte tilstanden vil reduseres 
til likevekt igjen etableres. For at emisjon skal dominere over absorpsjon, 
m\aa\ vi opprettholde et antall
atomer i tistand 2 som er st\o rre enn i tilstand 1. 
Dette kalles ogs\aa\ populasjonsinversjon.
Lasere og masere
er anretninger som gj\o r dette, typisk ved det som kalles optisk pumping 
hvis output er en intens koherent og monokromatisk str\aa le. Et eksempel
p\aa\ en laser vom virker etter denne m\aa ten er Ribidiumlaseren, 
den f\o rste laseren som blei laget, i 1960. Dens skjematiske virkem\aa ten
er vist i Figur \ref{fig:rubylaser}
\begin{figure}[h]
\setlength{\unitlength}{1mm}
   \begin{picture}(100,120)
   \put(-10,-50){\epsfxsize=18cm \epsfbox{rubylaser.ps}}
   \end{picture}
\caption{Illustrasjon av virkem\aa ten til en rubidiumlaser
som sender ut monokromatisk str\aa ling med b\o lgelengde
$\lambda=694.3$ nm. \label{fig:rubylaser}}
\end{figure}



\section{Helium-Neon laseren}
I 1961 blei
helium-neon laseren presentert. Dette er en gasslaser hvor
mekanismen bak populasjonsinversjon er forskjellig fra
rubidiumlaseren. 
Figur \ref{fig:heneonlaser} viser de ulike eksiterte tilstandene
som er viktige for virkningen av helium-neon laseren. V\ae r obs p\aa\
at eksitasjonsspektret til b\aa de helium og neon er mye mer 
komplisert enn vist p\aa\ figuren. De eksiterte tilstandene i helium som
er av interesse er de f\o rste triplett og singlett tilstandene,
med spektroskopisk notasjon henholdsvis $^3S_1$ og $^1S_0$ med eksitasjonsenergier 19.72 eV og 20.61 eV. 
Her kan vi tenke oss at et av elektronene er eksistert til $2s$ orbitalen,
slik at elektronkonfigurasjonen er $1s^12s^1$. Begge tilstandene er metastabile
da elektriske dipoloverganger til grunntilstanden er forbudte. Neon har to
grupper eksiterte tilstander med energi, med liten energiforksjell, 
n\ae r de to tilstandene i helium, med energier p\aa\ henholdsvis 
19.83 eV og 20.66 eV. I grunntilstanden har neon konfigurasjonen
$1s^22s^22p^6$. Vi kan tenke oss at det er et av $2p$ elektronene
som eksiteres til enten $4s$ eller $5s$ orbitalen, som vist i figuren.
\begin{figure}[h]
\setlength{\unitlength}{1mm}
   \begin{picture}(100,120)
   \put(-10,-30){\epsfxsize=18cm \epsfbox{heneonlaser.ps}}
   \end{picture}
\caption{Illustrasjon av helium-neon laseren.\label{fig:heneonlaser}}
\end{figure}
Neon atomene i gassblandingen eksiteres til disse tilstandene via
kollisjoner med eksiterte heliumatomer. Den kinetiske energien til 
heliumatomene s\o rger for den ekstra energiforskjellen, ca.~0.05 eV,
som trengs for \aa\ eksitere neonatomene. Det finnes ytterligere en eksitert
tilstand for neon med energi 18.70 eV over grunntilstanden og 1.96 eV under tilstanden med 20.66 eV. Denne tilstanden har elektronkonfigurasjonen 
$1s^22s^22p^53p^1$. Denne tilstanden er normalt ikke okkupert og populasjonsinversjon mellom disse eksiterte tilstandene forekommer umiddelbart.
Den stimulerte emisjonen som foreg\aa r mellom disse tilstandene 
resulteter i utsendte fotoner  med energi 1.96 eV og b\o lgelengde
$\lambda=632.84$ nm, som produser r\o dt lys. Etter stimulert emisjon, 
henfaller atomene spontant til den lavereliggende tilstanden med
elektronkonfigurasjon  $1s^22s^22p^53s^1$
og fotoner med utsendt b\o lgelengde $\lambda \approx 600$ nm.
Dette henfallet etterf\o lges av ikke-str\aa lings deeksitasjoner (typisk
via kollisjoner) tilbake til grunntilstanden. 

I helium-neon laseren har vi 4 tilstander som er involvert i produksjonen av 
str\aa ling med $\lambda=632.84$ nm. For rubidiumlaseren har vi kun 
tre tilstander. For en treniv\aa laser er populasjonsinversjon vanskeligere
\aa\ f\aa\ til da mer enn halvparten av atomene som er i grunntilstanden
m\aa\ eksiteres. I helium-neon laseren er det lettere \aa\ f\aa\ til populasjonsinversjon da tilstanden som n\aa s etter stimulert emisjon ikke er grunntilstanden, men en eksitert tilstand slik at antall atomer i 
grunntilstanden kan holdes lavt.



\clearemptydoublepage
\chapter{Nuclear and particle physics}\label{chap:kjernepartikkel}

\section{Kjernefysikk}

\section{Nukleosyntese}

\section{Partikkelfysikk}

\section{Kosmologi}
\section{Oppgaver}
\subsection{Analytiske oppgaver}
\subsubsection*{Oppgave 12.1}
%
Vi skal i denne oppgave studere f{\o}lgende radioaktive
desintegrasjons prosess
%
\begin{center}
{\Large \bf A} $\longrightarrow$
{\Large \bf B} $\longrightarrow$
{\Large \bf C}
\end{center}
%
hvor substansen A har en desintegrasjonskonstant $\lambda_A$,
B en desintegrasjonskonstant $\lambda_B$, mens substansen
C er stabil. Ved tiden $t = 0$ har vi $N_0$ atomkjerner av
type A og ingen av typen B eller C.
%
\begin{itemize}
%
\item[a)] Finn antall kjerner av type A -- $N_A$ som funksjon
av tiden $t$.
%
\item[b)] Finn antall kjerner av type B -- $N_B$ som funksjon
av $t$.
%
\item[c)] Finn antall kjerner av type C -- $N_C$ som funksjon
av $t$.
\end{itemize}
%
\subsubsection*{Oppgave 12.2}
%
Vi skal i denne oppgaven studere kjernereaksjonen
$ a + A \longrightarrow  B + b$. Prosjektilet $a$ med masse $m_a$
har kinetisk energi $E_k$, mens m{\aa}lsystemet $A$
med masse $M_A$ ligger i ro. Vi forutsetter at vi kan
behandle problemet ikke--relativistisk.
%
\begin{itemize}
%
\item[a)] Vis at den totale kinetiske energien i tyngdepunkt systemet
er
%
\[
\frac{M_A}{M_A + m_a} E_k
\]
%
\item[b)] Sett opp den formelle definisjonen for Q--verdien
for reaksjonen.
%
\item[c)] Vis at den totale energien tilgjengelig for
kjernereaksjonen $A(a,b)B$ er
%
\[
Q + \frac{M_A}{M_A + m_a} E_k
\]
%
\item[d)]  Finn den minste kinetiske energien $E_k$ for prosjektilet
$a$ i lab--systemet som gj{\o}r kjernereaksjonen mulig i
det tilfelle at $Q < 0 $
\end{itemize}
%
\subsubsection*{Oppgave 12.3}
%
En berylliumkjerne $_4^9$Be som ligger i ro, utsettes for
str{\aa}ling av $\alpha$--partikler med kinetisk energi
$K_{\alpha} = 5,3$~MeV. I den resulterende kjernereaksjon
produseres et foton $\gamma$ og en ukjent kjerne $X$.
%
\begin{itemize}
%
\item[a)] Identifis\'{e}r den ukjente kjernen $X$ ved bruk av
de bevaringslover man kjenner for slike kjernereaksjoner.
%
\item[b)] Hva blir Q--verdien for denne reaksjonen i MeV?
%
\item[c)] Beregn energien til fotonet hvis det beveger
seg fremover i samme retning som den innkommende
$\alpha$--partikkelen.
%
\end{itemize}
%
%
\begin{figure}[htbp]
\begin{center}
%
\begin{picture}(150.26,160.83)
\thicklines
\put(18.26,25.45){\line(1,0){38.71}}
\put(18.26,49.95){\line(1,0){38.71}}
\put(18.26,91.42){\line(1,0){38.71}}
\put(102.25,146.08){\line(1,0){37.98}}
\put(186.25,73.51){\line(1,0){38.71}}
\put(113.94,146.08){\vector(-3,-2){78.97}}
\put(113.94,146.08){\vector(-3,-4){70.71}}
\put(113.94,146.08){\vector(-1,-2){60.39}}
\put(126.36,146.08){\vector(1,-1){70.24}}
\put(24.10,91.42){\vector(0,-1){40.78}}
\put(35.79,91.42){\vector(0,-1){64.09}}
\put(48.21,49.95){\vector(0,-1){23.56}}
\put(113.94,160.39){\makebox(0,0)[tl]{$^{212}_{83}Bi$}}
\put(192.09,68.20){\makebox(0,0)[tl]{$^{212}_{84}Po$}}
\put(24.10,20.14){\makebox(0,0)[tl]{$^{208}_{81}Tl$}}
\put(-5.0,15.14){\makebox(0,0)[tl]{keV}}
\put(-5.0,94.42){\makebox(0,0)[tl]{\small 6,1}}
\put(-5.0,52.95){\makebox(0,0)[tl]{\small 5,2}}
\put(-5.0,28.45){\makebox(0,0)[tl]{\small 0,0}}
\end{picture}
%
\end{center}
\caption{\label{fig4.1}Desintegrasjonskanaler for $^{212}Bi$.}
%
\end{figure}
%
\subsubsection*{Oppgave 12.4}
%

Atomkjernen $^{212}Bi$ er radioaktiv. I figur~\ref{fig4.1} er vist noen av de
desintegrasjonskanalene som er mulige for $^{212}Bi$.
%
\begin{itemize}
%
\item[a)] Gi en kort forklaring p{\aa} hvilke prosesser som
er vist i figur~\ref{fig4.1} og sett opp reaksjons ligningene.
Hvilke generelle prinsipper
bestemmer prosessenes forl{\o}p.
%
\end{itemize}
%
Vi skal n{\aa} tenke oss at den radioaktive str{\aa}lingen blir analysert
for {\aa} bestemme energi fordelingen.
%
\begin{itemize}
%
\item[b)] Hvordan ser energi fordelingen ut for de forskjellige
typer str{\aa}ling vist i figur~\ref{fig4.1}?
Bruk massetabellen i begynnelsen av oppgavesettet
til {\aa} beregne energien av str{\aa}lingen. Vi ser bort fra
mulig rekyl energi.
%
\item[c)] Vis at baryontall og leptontall er bevart i reaksjonen
$^{212}Bi \longrightarrow ^{212}\!Po$ i figur~\ref{fig4.1}.
\end{itemize}
%
\subsubsection*{Oppgave 12.5}
%
\begin{itemize}
%
\item[a)]   Baryonet $\Lambda$ er ustabilt. Ved tiden  $t = 0$  har vi
$N_0$   $\Lambda$ partikler.  Utled formelen
$N(t) = N_0 \exp (-\lambda t)$
som bestemmer antall $\Lambda$ partikler som funksjon av tiden $t$.
%
\item[ b)]  Defin\*{e}r st{\o}rrelsene midlere levetid $\tau$
og halveringstid $ t_{1 / 2}$. Finn sammenhengen mellom
disse st{\o}rrelser og  desintegrasjonskonstanten  $\lambda$.
%
\item[ c)]  Vi ser p{\aa} reaksjonen
$  p + p \longrightarrow p + \Lambda + X$.
Hvilke kvantetall m{\aa} v{\ae}re bevart i reaksjonen?
Hvilken partikkel m{\aa} $X$ v{\ae}re?
%
\item[ d)]  Vi ser p{\aa} reaksjonen i punkt c)  i massesentersystemet
	  der de to protonene med masse  $m_p$  kolliderer
	  med like store
	  og motsatt rettede bevegelsesmengder $p_0$.
	  Finn et uttrykk for den minste
	  verdien $p_0$ kan ha for at reaksjonen skal kunne skje.
%
\item[ e)]  Vi ser p{\aa} reaksjonen i punkt c) i laboratoriesystemet
	  der det ene protonet har bevegelsesmengden  $p_0$,
	  mens det andre  ligger i ro.
	  Finn et uttrykk for den minste verdien  $p_0$  kan ha for at
	  reaksjonen skal kunne skje.

\end{itemize}
%
\subsubsection*{Oppgave 12.6}
%

Aktiviten $R = |dN / dt|$ av Carbon i levende materie er
$0,007 \mu$~Ci pr.~kg. Dette skyldes isotopen C$^{14}$.
Trekull funnet  blant b{\aa}lrester i en indiansk teltleir
viste en aktivitet p{\aa} $0,0048 \mu$~Ci pr.~kg. Anta at
halveringstiden for C$^{14}$ er 5730 {\aa}r. Beregn hvor mange
{\aa}r det er siden denne teltleiren har v{\ae}rt i bruk.

%
\subsubsection*{Oppgave 12.7}
%

Finn et uttrykk for $p_{min}$ for reaksjonen
%
\[
p_1 + p_2 \longrightarrow p_1 + p_2 +  \pi^0
\]
der $p_1$ har bevegelsesmengden $p_0$, og $p_2$ ligger
i ro i laboratoriesystemet.
%

\subsubsection*{Oppgave 12.8}

Hvilke konserveringsregler gj{\o}r f{\o}lgende reaksjoner
(u)mulige?
%
\begin{eqnarray*}
p + p &\longrightarrow&  p + \gamma\\
p + p &\longrightarrow&  p + p + e^-\\
p + p &\longrightarrow&  \pi + p + p\\
p + p &\longrightarrow&  p + p + K^+\\
p + p &\longrightarrow&  p + p + K^+ + K^0\\
p + p &\longrightarrow&  e^+ + e^-\\
p + p &\longrightarrow&  e^+ + e^- + \mu^+\\
p + p &\longrightarrow&  e^+ + e^- +\nu
\end{eqnarray*}
%

\subsubsection*{Oppgave 12.9}
%
\begin{itemize}
%
\item[a)] Forklar hvordan mesoner og baryoner settes sammen av kvarker.
Hva er kvarkinnholdet i protoner, n{\o}ytroner, $\pi$--mesoner
($\pi^+, \pi^0, \pi^-$) og $\Lambda^0$ partiklene?
%
\item[b)] Har det noen gang blitt observert en fri kvark? Gi
forslag til forklaring.
%
\item[c)] $\Lambda^0$ desintegrerer ofte til $p + \pi^-$.
 Hvilke kvantetall er ikke bevart i denne reaksjonen?
%
\item[d)] Sett opp de ligningene som er n{\o}dvendige
til {\aa} bestemme den kinetiske energien til protonet
og $\pi^-$--mesonet
i prosessen diskutert i c)
n{\aa}r vi antar at $\Lambda$ er i ro f{\o}r desintegrasjonen.
%
\end{itemize}
%

\subsubsection*{Oppgave 12.10}
%

Vi skal i denne oppgaven studere radioaktive atomkjerner.
%
\begin{itemize}
%
\item[a)] Hvis antallet radioaktive atomkjerner med
desintegrasjonskonstant $\lambda$ ved tiden $t = 0$ er $N_0$,
utled formelen for antallet radioaktive atomkjerner $N(t)$
ved tiden $t > 0$.
%
\item[b)] Finn sammenhengen mellom $\lambda$, den midlere
levetid $\overline{t}$ og halveringstiden $t_{1/2}$.
%
\item[c)] Defin\'{e}r aktivitetet $R$ og
finn $R(t)$.
%
\item[d)] Et radioaktivt element med desintegrasjonskonstant
$\lambda$ blir produsert i en reaktor i et konstant antall A
pr.~sec.. Vis at etter en tid $t$ er antallet radioaktive
kjerner $N_A(t)$ gitt ved ligningen
%
\[
N_A(t) = \frac{A}{\lambda} \left ( 1 - \exp (-\lambda t) \right ).
\]
%
\item[e)] Det vil oppst{\aa} likevekt mellom produksjon og desintegrasjon
av det radioaktive elementet. Finn denne likevektsverdi.
Hvor lang tid $t$ vil det g{\aa} f{\o}r antallet radioaktive
elementer er halvdelen av likevektsverdien?
%
\end{itemize}
%


\subsubsection*{Oppgave 12.11}
%
Mange atomkjerner er ustabile. Etter en viss tid vil de desintegrere.
I fig.~\ref{4.2} er vist tre eksempler p� slike prosesser. I 
fig.~\ref{4.2}~(a)
er to tilstander i $_8^{17}$O angitt, grunntilstanden og \'{e}n eksitert tilstand,
0,871~MeV over grunntilstanden. 
Samme tilfelle har vi i  fig.~\ref{4.2}~(c) for $_3^7$Li hvor den 
eksiterte tilstand har energi 0,476~MeV.
Tallene i parentes er de atom�re massene i enheten $u = 931,478$~MeV.
%
\begin{figure}[h]
\begin{center}
%
\setlength{\unitlength}{0.7cm}
%
\begin{picture}(21,7)
\thicklines

\put(0,0){\makebox(0,0)[bl]{
		\put(3,1){\line(1,0){2.0}}
		\put(3,3){\line(1,0){2.0}}
		\put(0,5){\line(1,0){2.0}}

		\put(1,5){\vector(3,-4){2.8}}
		\put(1,5){\vector(3,-2){2.8}}
		\put(4,3){\vector(0,-1){1.8}}

                \put(3.4,-0.5){(a)}
		\put(0,5.2){\makebox(0,0)[bl]{\small $^{17}_{7}$N(17,0084)}}
		\put(4,3.2){\makebox(0,0)[bl]{\small 0.871~MeV}}
		\put(3,0.8){\makebox(0,0)[tl]{\small $^{17}_{8}$O(16,9991)}}
         }}        
%
\put(14,0){\makebox(0,0)[bl]{
		\put(0,1){\line(1,0){2.0}}
		\put(0,3){\line(1,0){2.0}}
		\put(3,5){\line(1,0){2.0}}

		\put(4,5){\vector(-3,-4){2.8}}
		\put(4,5){\vector(-3,-2){2.8}}
		\put(1,3){\vector(0,-1){1.8}}

                \put(0.4,-0.5){(c)}
		\put(0,0.8){\makebox(0,0)[tl]{\small $^{7}_{3}$Li(7,0160)}}
		\put(-1,3.2){\makebox(0,0)[bl]{\small 0.476~MeV}}
		\put(3,5.2){\makebox(0,0)[bl]{\small $^{7}_{4}$Be(7,0169)}}
         }}    

 %
\put(7,0){\makebox(0,0)[bl]{
		\put(0,1){\line(1,0){2.0}}
		\put(3,5){\line(1,0){2.0}}

		\put(4,5){\vector(-3,-4){2.8}}

                \put(0.4,-0.5){(b)}
		\put(0,0.8){\makebox(0,0)[tl]{\small $^{11}_{5}$B(11,0093)}}
		\put(3,5.2){\makebox(0,0)[bl]{\small $^{11}_{6}$C(11,0114)}}
         }}          
\end{picture}
%
\end{center}
\caption{\label{4.2} Radioaktiv desintegrasjon}
%
\end{figure}
%
\begin{itemize}
%
\item[a)] Gi en kort forklaring p� hvilke prosesser som er vist i 
fig.~\ref{4.2}~(a)
og forklar hvilke generelle prinsipper som bestemmer prosessenes forl�p.
Sett opp reaksjonsligningene og beregn energi av den 
utg�ende str�ling i de tre tilfellene.
%
\item[b)] Gjenta det samme for prosessen beskrevet i 
fig.~\ref{4.2}~(b).
%
\item[c)] Gjenta det samme for prosessen beskrevet i 
fig.~\ref{4.2}~(c).
%
\end{itemize}
%
\subsubsection*{Oppgave 12.12}
%
Radioaktive kjerner er ustabile og sender ut bl.~a. $\alpha$ partikler
($\alpha = _2^4$He). Ett eksempel er $_{92}^{232}$U 
som g�r over til Thorium med en halveringstid p� 72~�r.
%
\begin{itemize}
%
\item[a)] Sett opp reaksjonsligningen i det generelle tilfelle og spesielt
for  $_{92}^{232}$U  ved utsendelse av en $\alpha$ partikkel.
%
\item[b)] Utled loven for radioaktiv desintegrasjon og finn sammenhengen 
med halveringstiden T$_{1/2}$.
%
\end{itemize}
%
\begin{figure}[ht]
\begin{center}
%
\setlength{\unitlength}{1cm}
%
\begin{picture}(10,6)
\thicklines

\put(0,0){\makebox(0,0)[bl]{
		\put(0,0){\line(1,0){2}}
		\put(2,0){\line(0,1){4}}
		\put(0,2){\line(1,0){8}}
                \put(-0.1,4){\line(1,0){0.2}}

                \put(2,2){\dashbox{0.2}(2,2){}}

		\put(0,0){\vector(0,1){5}}

                \qbezier(2,4)(4,2)(8,2)

		\put(-1,0){\makebox(0,0)[cl]{$-V_0$}}
		\put(-1,0){\makebox(0,8)[cl]{$-V_b$}}


                \put(-0.5,2){\makebox(0,0)[cl]{0}}
                \put(2.1,1.9){\makebox(0,0)[tl]{\small $R$}}
                \put(3.5,1.8){\makebox(0,0)[cl]{\small $R + b$}}
                \put(8.2,2){\makebox(0,0)[cl]{\small $r$}}
                \put(0.2,4.5){\makebox(0,0)[bl]{ E}}
                \put(1,1){I}
                \put(3,1){II}
                \put(6,1){III}
         }}
%
\end{picture}
%
\end{center}
\caption{\label{4.3} Kvantemekanisk model for utsendelse av en $\alpha$
partikkel}
%
\end{figure}

I teorien for $\alpha$ desintegrasjon antar man at slike partikler 
eksisterer inne i atomkjernen. For � slippe ut m� de g� gjennom en 
potensialbarri\'{e}re. Vi kan forestille oss dette som et \'{e}n--dimensjonalt
kvantemekanisk problem hvor den potensielle energien har formen som vis i 
fig.~\ref{4.3}.
%
\begin{itemize}
%
\item[c)] Hvordan forklarer vi potensialbarri\'{e}rens h�yde V$_b$?
%
\end{itemize}
%
Vi skal n� diskutere det kvantemekaniske problemet for en $\alpha$--partikkel
i  potensialfeltet gitt i fig.~\ref{4.3}. For � forenkle beregningen bruker vi
imidlertid det stiplede potensialet.
%
\begin{itemize}
%
\item[d)] Sett opp Schr\"{o}dingerligningen for hvert av de tre omr�dene
%
\[
\mbox{I:} \;\;0 < r < R, \;\;\,\, \mbox{II:}\;\; R < r < R + b, \;\;\,\,
                \mbox{III:}\;\; R + b < r < \infty,
\]
%
og finn l�sningen med integrasjonskonstanter.
%
\item[e)] Sett opp de betingelser som bestemmer relasjonen mellom
integrasjonskonstantene. Konstantene skal ikke finnes eksplisitt.
%
\item[f)] Nevn rent kvalitativt hva som bestemmer halveringstiden.
%
\end{itemize}
%
\subsubsection*{Oppgave 12.13}
%
\begin{itemize}
%
\item[a)] En partikkel med hvilemasse $m > 0$ og bevegelsesmengde
     $p > 0$ st�ter mot en annen partikkel som har samme 
     hvilemasse og som ligger i ro. Er det kinematisk mulig at 
     de to partikler fortsetter etter st�tet sammen som \underline{ett}
     system med hvilemasse $M = 2m$?  Begrunn svaret ved �
     regne b�de klassisk og relativistisk. 
%
\item[b)]  Hvordan desintegrerer et fritt n�ytron? 
           Hvilken type vekselvirkning for�rsaker denne 
           desintegrasjonen?  Hvorfor kan ikke et  
           fritt proton desintegrere p� lignende m�te?
%
\item[c)] Hva er isospinnet for protonet, n�ytronet, $\pi$--mesonene og
     $\Lambda^{0}$?  Hvilke vekselvirkninger for�rsaker reaksjonene 
%
          \begin{eqnarray*}
              \pi^{0} + p &\rightarrow &\pi^{+} + n\\
              \Lambda^{0}&\rightarrow &p + \pi^{-}
          \end{eqnarray*}
%
     Vis at isospinnets tredjekomponent er bevart i den ene av
     de to reaksjoner, men ikke i den andre. Er resultatet i
     overensstemmelse med de to typer vekselvirkning? Hva er
     kvarkinnholdet av et proton, et n�ytron, $\pi$--mesonene 
     og $\Lambda^{0}$? 
%
\item[d)]   Hvilke av f�lgende reaksjoner er forbudt, og i tilfelle
            hvorfor?
%
           \begin{eqnarray*}                   
              p + p &\rightarrow &p + p + n,\\
              p + p &\rightarrow &p + n,\\
              p + p &\rightarrow &p + p + \pi^{0},\\
              p + p &\rightarrow &e^{+} + e^{+},\\
              e^{-} + p &\rightarrow  &e^{-} + p + \gamma,\\
              e^{-} + p &\rightarrow &\nu_{\mu} + n
          \end{eqnarray*}          
%
\end{itemize}
%


\subsubsection*{Oppgave 12.14}
%
\begin{itemize}
%
\item[a)] Forklar kort hva som menes med bindingsenergi til en atomkjerne
   med massetall $A$ og ladning $Z$. 
   Gitt Weizs\"acker semi--empiriske masseformel for bindingsenergi 
   $B_W(Z,A)$ til en kjerne med massetall $A$ og ladning $Z$
%
\[
                B_W(Z,A)=  C_1 A - C_2 A^{2/3} 
                          - C_3 \frac{Z^2}{A^{1/3}} - C_4 \frac{(A-2Z)^2}{A} .
               \]
%
Gi en kort fysisk begrunnelse for de enkelte ledd.
%
\end{itemize}
%
Vi velger verdiene for de enkelte konstantene til � v�re

\begin{center}
$C_1 = 15,56$~MeV,  $C_2 = 17,23$~MeV, $C_3 = 0,70$~MeV og $C_4 = 23,29$~MeV.
\end{center}

\begin{itemize}
%
\item[b)] Forklar kort hva som menes med $Q$-verdien for en kjernereaksjon.
   Sett opp $Q_{\alpha}$ verdien for en spontan $\alpha$-partikkel desintegrasjon.
%
\end{itemize}
%
Vi skal n� bruke Weizs\"ackers semi--empiriske masseformel for
    bindingsenergien $B_W(Z,A)$ fra a) til � beregne  $Q_{\alpha}$.
%
\begin{itemize}
%
\item[c)] For store verdier av $A$ og $Z$ , vis at vi kan approksimere
   den frigjorte energien n�r en kjerne med gitt
   $A,Z$ spontant emitterer en $\alpha$-partikkel, til 
%
\begin{eqnarray}
 Q_{\alpha} \approx& -4 C_1 + \frac{8}{3} C_2 \frac{1}{A^{1/3}}
              + 4 C_3 \frac{Z(1-Z/3A)}{A^{1/3}}\nonumber \\
             & - 4 C_4 \frac{(N-Z)^2}{A^2} + BE(_2^4\mbox{He}), \nonumber
\end{eqnarray}
%
hvor $BE(_2^4\mbox{He}) = 28,3$~MeV er  bindingsenergien til
 He. Hjelp: Bruk binomial utviklingen for ledd av typen
%
\[
 (A-x)^{n/m} \approx \left ( 1 - x \frac{n}{m A} \right ) A^{n/m}.
\]

%
\item[d)] To naturlig forekommende isotoper av s\o lv og
                gull er $^{107}_{47}$Ag og $^{197}_{79}$Au.
                Bruk resultatene fra c) og 
                diskuter stabiliteten til disse kjernene med
                tanke p� $\alpha$-partikkel desintegrasjon.
%
\item[e)] $^{235}_{92}$U kan fisjonere spontant. Beregn
          ved hjelp av formelen i a)  den frigjorte energien 
          i reaksjonen  $^{235}_{92}$U$\rightarrow$$^{87}_{35}$Br
          $+$$^{145}_{57}$La $+3n$. 
          Hvilket ledd i den semi-empiriske masseformelen
                gir st�rst bidrag til forandringen i 
                bindingsenergien?
%
\end{itemize}

%%%%%%%%%%%%%%%%%%%%%%%%%%%%%%%%%%%%%%%%%%%%%%%%

\clearemptydoublepage
\chapter{Solid state physics}\label{chap:faststoff}

\section{Halvledere og transistorer}

\section{Superledning og superfluiditet}
\section{Oppgaver}
\subsection{Analytiske oppgaver}
\subsubsection*{Oppgave 13.1}
%
Kobber er en god elektrisk leder. Ved romtemperatur er
resistiviteten m\aa lt til $\rho = 1,7 \cdot 10^{-8} \Omega m$.
Egenvekten er 8,96 g/$cm^{3}$. Hvert kobberatom bidrar med ett
ledningselektron.

%
\begin{itemize}
%
\item[a)] Beregn tettheten av ledningselektroner i metallisk
kobber.

\item[b)] Finn midlere termisk hastighet $\overline{v}$ og
midlere kollisjonslengde $L$ fra den klassiske Drudemodellen
for elektronene i metallet.

\item[c)] Fermienergien for elektronene i kobber er 7,0 eV.
Hva gir n\aa ~den kvantemekaniske elektronmodellen for de to
st\o rrelsene i punkt b)?
%
\end{itemize}


\clearemptydoublepage
\chapter{Quantum computers}\label{chap:kvantedatamaskiner}
\begin{quotation}
It would appear that we have reached the limits of what it is possible
to achieve with computer technology, although one should be careful
with such statements, as they tend to sound pretty silly in five
years. {\em John von Neumann}
\end{quotation}



For bare et knapt ti\aa r siden, inngikk mange av de sidene ved 
kvantemekanikken
som har v\ae rt med \aa\ definere det n\aa\ allerede omfangsrike
forskningsfeltet om kvantedatamaskiner og kvanteinformasjonsteori,
som en del av filosofiske diskusjoner om kvantemekanikkens 
begrensninger, paradokser, implikasjoner om v\aa r forst\aa else
av naturen. 
I de siste fem \aa r, til manges overraskelse, har grunnleggende sider
av kvantemekanikken slik som superposisjonsprinsippet\footnote{Husk
at dette har sitt opphav i de Broglie sitt postulat om materiens b\o lge
og partikkel natur.}, Schr\"odingers katt tilstander, lokal virkelighets
beskrivelse kontra 'spooky action at distance' 
og kvante entaglement\footnote{Som s\aa\ ofte, henfaller jeg til
anglosaksiske ord eller deres latiniserte varianter. 
Entanglement er en fri oversettelse av
det tyske ord Verschr\"ankung, \aa\ legge i kryss, f\o rste gang introdusert
av Schr\"odinger i 1935 for \aa\ beskrive sammensatte kvantetilstander.}  
danna grunnlaget for den rivende utviklinga vi ser i 
f.eks.~kvante informasjonsteori, fra felt som kvantekryptering, kvante
teleportasjon, helt nye algoritmer for rask s\o king i databaser,
til kvantenettverk til vakre eksperimentelle realiseringer av kvantekretser.

Dette er et felt i rivende utvikling, og ingen kan med sikkerhet si
hvordan feltet vil se ut om bare noen f\aa\ \aa r, men at det representerer
et hav av fantastiske muligheter er neppe \aa\ ta munnen for full. 
Framtida tilh\o rer dere!

Her vil dere bare f\aa\ en sterkt kondensert oversikt, mer informasjon
kan dere finne p\aa\ f.eks.~www.qubit.org, 
www.quantum.at eller www.iu.hio.no/data/quantum.html 
med flere
linker til liknende steder. 

Det er to viktige sp\o rsm\aa l som reiser seg i tilknytting enhver
ny teknologi. Det f\o rste er: 
Hva kan en kvantedatamaskin gj\o re i tillegg til en vanlig datamaskin?
Hittil har f\o lgende omr\aa der pekt seg
\begin{itemize}
 \item Kvanteinformasjonsteori, ny m\aa te \aa\ tenke informasjonsteori p\aa\ .
 \item Kvantealgoritmer til kryptering og s\o king i store databaser.
 \item Simulering av fysiske systemer i langt st\o rre skala enn dagens maskiner.  
 \item Raskere s\o king i store databaser.
 \item Bra utgangspunkt for parallellisering pga.~superponering av kvantemekaniske tilstander. 
  \item Operere med mye st\o rre datamengder
 \item Generere virkelige vilk\aa rlige tall.
  \item Teleportasjon?
\item ...
\end{itemize}
Det andre sp\o rsm\aa let retter s\o kelyset mot om vi overhodet 
kan lage en kvantedatamaskin. Per dags dato finnes det flere 
metoder til \aa\ lage kvantemekaniske kretser, hvor f\o lgende er blant
de mest lovende
\begin{itemize}
 \item Ionefelle baserte kretser
 \item Quantum dots, et eller flere elektroner som er fanga inn i sm\aa\
       omr\aa der mellom halvlederlag
 \item Hulroms kvanteeletrodynamikk (Cavity QED)
 \item Kjernemagnetisk resonans (NMR)
 \item Supraledende Qubits
\end{itemize}
\begin{figure}[h]
\begin{center}
{\centering
\mbox
{\psfig{figure=st29f1.ps,height=8cm,width=8cm}}
}
\end{center}
\caption{Denne stabile CO$_2$ laseren tillater forskere \aa\
fange enkeltatomer for mer enn 5 minutter, for deretter \aa\ kunne
studere atomenes ulike tilstander, 
se Phys.~Rev.~Lett.~{\bf 82} (1999) 4204.}
\end{figure}

Hittil har en v\ae rt i stand til \aa\ lage kvantemekaniske  
kretser vha.~ionefelle
baserte kretser og kjernemagnetisk resonans. 
Vi skal diskutere noen av disse kretsene i dette kapitlet. Men f\o rst
et aldri s\aa\ lite historisk tilbakeblikk.


\section{Historisk tilbakeblikk og Moores lov}

I 1946 framsto
ENIAC (Electronic Numerical Integrator And Computer), som kunne addere
5000 tall per sekund, som det store teknologiske
gjennombruddet. Dens reknekraft svarte alts\aa\ til noen tusen flytende talls 
operasjoner 
per sekund (FLOPS). Idag har vi maskiner som kan utf\o re trillioner av FLOPS.
ENIAC besto av 
\begin{enumerate}
\item ca.~19000 vakuumr\o r
\item veide ca.~30 tonn
\item brukte ca 174 kW, eller 233 hk
\item trengte ca.~150 m$^2$ med plass (30 ft $\times$ 50 ft) 
\end{enumerate}
\begin{figure}
\begin{center}
{\centering\mbox{\psfig {figure=eniac2.ps,height=8cm,width=10cm}
}}
\end{center}
\caption{Bilde av ENIAC som viser noe av dens omfang.}  
\end{figure}    
Et ekspertpanel i 1949 uttrykte forh\aa pningsfullt f\o lgende
``.... {\em en eller annen dag kan vi utvikle en like kraftig datamaskin med
bare 1500 vakuumr\o r, med vekt p\aa\ kanskje 1500 kg og et forbruk p\aa\
ca.~10 kW......}
Resten er vel historie...? Transistoren som blei introdusert p\aa\
begynnelsen av 50-tallet revolusjonerte fullstendig feltet, og innen f\aa\
\aa r var vakuumr\o r teknologien akterutseilt.

Idag representeres bit 0 og 1 vha.~spenningsforksjeller.
Med dagens teknologi brukes ca.~100000 elektroner for \aa\
lagre en bit med informasjon og en chip har en utstrekning p\aa\ noen
f\aa\ micrometer. Transistoren, som er arbeidshesten i enhver datamaskin,
best\aa r i dag av noen  f\aa\ hundre elektroner. Skal miniatyriseringen
av elektroniske kretser fortsette med uforminska styrke, vil vi,
dersom vi ekstrapolerer trenden i forminskning fra 
1960 til \aa r 2010 
knapt trenge et elektron for \aa\ lagre en bit med informasjon.
Det sier seg sj\o l at f\o r eller siden vil kvantemekaniske
effekter begynne \aa\ spille en viktig rolle, og dagens    
teknologi vil m\o te veggen dersom ikke nye m\aa ter \aa\ bygge kretser
utvikles. 
Denne ekstrapolasjonen kalles ogs\aa\ Moore sin lov, etter Gordon Moore
ved Intel,
kjent for sin observasjon i 1965, kun fire \aa r etter at den f\o rst
integrerte kretsen kom p\aa\ markedet, at antall transistorer per 
integrert krets ville doble hver 18 m\aa ned. Tabellen her viser
antall transistorer per krets fra 1971 til Pentium 4 prosessorern
fra 2000.
\begin{table}
\caption{Moore sin lov, antall transistorer per krets fra 1971 til
2000. For mer informasjon se www.intel.com/research/silicon/mooreslaw.htm.}
\begin{center}
\begin{tabular}{lrr}\hline\\
Prosessor & \aa r &  antall transistorer per krets\\ \hline \\
         4004 &1971 &  2250 \\
          8008& 1972 &  2500 \\
          8080& 1974 & 5000 \\
           8086 & 1978 & 29000\\
            286 &  1982&  120000\\
        386[tm] processor &  1985 &275000\\
        486[tm] DX processor & 1989 & 1180000\\
        Pentium 256 processor & 1993 & 3100000\\
        Pentium II  processor & 1997 & 7500000\\
        Pentium III processor & 1999 & 24000000\\
        Pentium 4   processor & 2000 & 42000000\\ \hline 
\end{tabular}
\end{center}
\end{table}

Det er her kvantemekanikken kommer inn. 
Kvantemekanikk tilbyr en enkel og naturlig representasjon av bits:
      vi kan f.eks.~tenke p\aa\ tilstander i et atom, hvor
bit 0 er gitt ved normaltilstanden (grunntilstanden) mens bit 1 er gitt
ved en elller annen eksitert tilstand. 

En enda enklere tiln\ae rming  er \aa\ se p\aa\ enkeltelektroner. 
Kan vi isolere
et enkelt elektron, kan vi vha.~et ytre p\aa satt magnetfelt ha spinn
egenverdier $+1/2$ eller $-1/2$. Den f\o rste kan da tilsvare bit 0 mens
den andre spinnegenveriden svarer til bit 1.


En slik kvantemekanisk representasjon av en bit kalles {\bf QUBIT} og det leder
oss til neste avsnitt!



\section{Superposisjon og qubits}

\subsection{Superposisjonsprinsippet}

Det kvantemekaniske superposisjonsprinsippet spiller en sentral
rolle i alle betraktninger om kvante informasjonsteori, de fleste 
s\aa kalla 'gedanken' eksperiment og paradokser i kvantemekanikk.

Vi har i kapittel 2 allerede stifta bekjentskap med dobbelspalt eksperimentene,
som i f\o lge Feynman har i seg 'hjerte av kvantemekanikken'. 
De viktigste bestandelene i dette eksperimentet er en partikkel kilde,
en dobbeltspalt innretning og en skjerm hvor vi kan observere eventuelle
interferens m\o nster. Interferens m\o nstrene kan kun forst\aa s dersom
vi antar at materien utviser en b\o lgenatur. Denne type eksperiment har blitt
gjort med flere partikkel typer, fra fotoner, via elektroner, til
n\o ytroner og atomer. 
Kvantemekanisk er tilstanden vi observerer gitt ved en koherent
superposisjon 
\be
   \Psi(x,t)=\Psi(x,t)_a+\Psi(x,t)_b,
\ee
hvor indeks $a$ svarer til en tilstand med bare spalt $a$ mens 
indeks $b$ er den tilsvarende tilstanden for spalt $b$.
En slik {\bf superposisjon} av kvantemekaniske tilstander kalles for koherente
tilstander (kvante koherens) og f\o lger fra postulatet om materiens
b\o lge og partikkel natur.
Det finnes per dags ingen eksperiment som tillater oss \aa\ si
bestemt hvilken spalt f.eks.~et enkelt elektron g\aa r gjennom. 
\O nsker vi \aa\ gj\o re det, vil en eventuell m\aa ling
kreve at vi vekselvirker med partikkelen, noe som leder til
{\bf dekoherens}, dvs.~tap av interferens.
Kun n\aa r vi ikke har noen kunnskap om hvilken spalt partikkelen
passerte kan vi observe interferens! Det er klart at dette strider med
v\aa r oppfatning av en partikkel som en lokalisert st\o rrelse. 



\subsection{Qubits}

I informasjonsteori er den fundamentale enheten en bit. 
Den utgj\o r et system med to verdier, '0' eller '1'. 
I sin klassiske realisering, kan vi tenke oss en bit som en mekanisk
bryter, et system  med to forskjellige tilstander. Vi kunne tenke
oss ogs\aa\ at energi, eller potensialforksjellen mellom de to
tilstandene er s\aa pass stor at en ikke kan ha spontane overganger
fra f.eks.~bit '0' til bit '1'. 

Den kvantemekaniske analog til den klassiske bit er den s\aa kalla
{\em qubit}, og p\aa\ lik linje med sin klassiske partner, m\aa\ den
ha minst to tilstander, som vi heretter kaller for $|0\rangle$ og
$|1\rangle$. I prinsippet kan ethvert kvantemekanisk system som har
minst to tilstander tjene som en basis for en qubit, tenk bare p\aa\ et 
elektron i et magnetfelt. Avhengig av magnetfeltets retning, kan vi ha 
kvantetallene $m_s=\pm 1/2$.  Disse to kvantetallene kan tjene som
basis for en qubit.
I slutten av dette kapitlet skal vi se p\aa\ et eksperimentelt
oppsett som manipulerer to qubits. 

Alt dette h\o res kanskje ikke s\aa\ banebrytende. Men kopler vi superposisjons
prinsippet til en slik qubit, kan vi lage oss en generell qubit tilstand
\be
    |q\rangle = \alpha |0\rangle + \beta |1\rangle,
\ee
med den egenskap at $|\alpha|^2+|\beta|^2=1$. Det betyr ikke at qubiten 
har en verdi et sted mellom '0' og '1', men heller at qubiten er i en
kvantemekanisk superposisjon av begge tilstander, og dersom vi foretar
en m\aa ling p\aa\ denne tilstanden, finner vi en sannsynlighet
$|\alpha|^2$ for at den er i tilstand '0' og $|\beta|^2$
for at den er i tilstand '1'. Qubiten er i en koherent superposisjon
av to kvantemekaniske basis tilstander.  
Et enkelt eksempel er 
\be
    |q\rangle = \frac{1}{\sqrt{2}}(|0\rangle + |1\rangle),
\ee
som betyr at vi har 50\% sannsynlighet for at qubiten er i tilstand
'0' og 50\% sannsynlighet for at den er i tilstand '1'. 

Hvorfor er anvendelsen av det kvantemekaniske superposisjonsprinsippet
av interesse? Og hva er nytt i forhold til en klassisk representasjon?

La oss se p\aa\ et tilfelle hvor vi har en kvantemekanisk
tilstand med \aa tte komponenter, dvs.~vi har superposisjonen 
\be
    |\psi\rangle = a_0|\psi\rangle_0+a_1|\psi\rangle_1+a_2|\psi\rangle_2+a_3|\psi\rangle_3+a_4|\psi\rangle_4+a_5|\psi\rangle_5+a_6|\psi\rangle_6+a_7|\psi\rangle_7,
\label{eq:dataord}
\ee
hvor hver komponent $|\psi\rangle_i$ er enten i en tilstand 
'0' eller '1'. Vi kunne f.eks.~tenke oss at hver $|\psi\rangle_i$ 
var et elektron med enten $m_s=-1/2$ eller $m_s=+1/2$. Koeffisientene
$a_i$ er imagin\ae re.  

Dersom vi \o nsker \aa\ representere denne tilstanden i en klassisk datamaskin
for en 
eventuell kvantemekanisk beregning, m\aa\ vi lagre hver av koeffisientene
$a_i$ et sted i minnet. 
Vi trenger da et ord med 128 bits for 
\aa\ representere hver $a_i$ som er imagin\ae r i en konvensjonell datamaskin.
Et alternativ er alltid \aa\ skrive dataene ut p\aa\
disk. Men dersom vi ender opp med  \aa\ lese og skrive store datamengder  
til og fra ei fil, 
senker dette hastigheten p\aa\ programmet v\aa rt, da lesing 
til/fra ei fil tar lengre tid enn \aa\ aksessere data i internminnet
til en datamaskin.
I v\aa rt tilfelle trenger vi \aa tte adresser
i minnet for \aa\ lagre en bestemt kombinasjon
av koeffisientene $a_i$. Det burde ikke v\ae re vanskelig \aa\
overbevise seg sj\o l at dersom antall elektroner \o ker 
(systemet v\aa rt blir st\o rre), \o ker
ogs\aa\ v\aa rt behov for minne.
Med dagens teknologi, kan vi lagre informasjon som svarer til
ca.~$\sim 2^{30}-2^{35}$ i minnet p\aa\ de beste datamaskinene vi har
tilgjengelig.  Det vil da svare til et sted mellom 30 og 35 elektroner
som kan ha spinn opp eller ned. Generelt har vi et totalt antall
tilstander gitt ved $2^n$, hvor $n$ kan v\ae re antall elektroner 
i to tilstander, spinn opp eller ned. 

Hvordan omg\aa\ dette problemet?  
Det kvantemekaniske superposisjonsprinsippet kommer oss her til
unnsetting. En tilstand som den beskrevet i likning (\ref{eq:dataord}),
gitt ved en bestemt kombinasjon av koeffisientene $a_i$, kan da tenkes
lagret som et enkelt ord i et enkelt kvanteregister, 
dvs.~kun en adresse,
i motsetning til de \aa tte vi trenger for en klassisk datamaskin.
Dersom vi f.eks.~kan manipulere
$n=500$ qubits,  har vi $2^{500}$ mulige tilstander, som er mye st\o rre
enn det estimerte antallet atomer i verdensrommet.
Kvantemekanikk gir oss dermed et potensiale for informasjonsbehandling 
langt utover det en klassiske datamaskin kan  gj\o re. 

Hvordan en skal lage slike kvanteregistre er dog ikke klart. 
Det en har klart eksperimentelt hittil er \aa\ lage 
enkle kvantemekaniske kretser hvor en manipulerer noen f\aa\
qubits. 

\section{Operasjoner p\aa\ en qubit}\label{sec:qubitoper}

Innsikt i noen av de mer grunnleggende operasjonene som danner
grunnlaget for en kvantedatamaskin og kvanteinformasjons teori, finnes
ved \aa\ studere et tenkt oppsett for en enkel str\aa lesplitter,
f.eks.~gitt ved endringen av polarisasjonsretningen til en innkommende
lysstr\aa le.


Vi kan tenke oss at vi har en innkommende str\aa le som kan v\ae re
i to tilstander, med lik sannsynlighet. I v\aa rt tilfelle ser vi for
oss en partikkel som enten kommer inn ovenifra (tilstand  $|0\rangle$)
eller nedenfra (tilstand  $|1\rangle$) mot str\aa lesplitteren. 
Str\aa len p\aa virkes deretter av en str\aa le splitter. 
Str\aa len splittes i to, med lik sannsynlighet for at partikkelen
kommer ut ovenfor eller nedenfor. En enkel matematisk beskrivelse av
denne str\aa lesplitteren er gitt ved den s\aa kalte Hadamard
transformasjonen $\OP{H}$, hvis virkning p\aa\ en tilstand
$|0\rangle$ eller $|1\rangle$ er
\be
    \OP{H}|0\rangle=
    \frac{1}{\sqrt{2}}(|0\rangle +|1\rangle),
\ee
og 
\be
    \OP{H}|1\rangle=
    \frac{1}{\sqrt{2}}(|0\rangle -|1\rangle).
\ee
Vi merker oss ogs\aa\ at 
\be
    \OP{H}\OP{H}|0\rangle=
    \OP{H}\frac{1}{\sqrt{2}}(|0\rangle +|1\rangle)
    =|0\rangle,
\ee
og 
\be
    \OP{H}\OP{H}|1\rangle=
    \OP{H}\frac{1}{\sqrt{2}}(|0\rangle -|1\rangle)
    =|1\rangle.
\ee

Virkningen av denne kvantemekaniske transformasjonen er alts\aa\ 
\aa\ lage
en midlertidig tilstand som best\aa r av en superposisjon av
bit '0' og bit '1'. Klassisk er ikke det mulig. 
En alternativ beskrivelse er gitt ved en matriserepresentasjon
av tilstandene $|0\rangle$ og $|1\rangle$ og transformasjonen
$\OP{H}$, gitt ved
\be
   |0\rangle=\left(\begin{array}{c} 1 \\ 0\end{array}   \right),
\ee
\be
   |1\rangle=\left(\begin{array}{c} 0 \\ 1\end{array}   \right),
\ee
og 
\be
   \OP{H}=\frac{1}{\sqrt{2}}
   \left(\begin{array}{cc} 1 & 1 \\ 1& -1\end{array}   \right).
\ee
Overbevis deg selv om at dette stemmer!

Det finnes flere slike operasjoner p\aa\ qubits. Den s\aa kalte 
NOT-kretsen, hvor bit '0' skifter til bit '1', eller omvendt,
kan skrives som
\be
   \OP{H}_{\mathrm{NOT}}=   
   \left(\begin{array}{cc} 0 & 1 \\ 1& 0\end{array}   \right).
\ee
Vi ser da at
\be
   \OP{H}_{\mathrm{NOT}}|0\rangle=   
   \left(\begin{array}{cc} 0 & 1 \\ 1& 0\end{array}\right)
   \left(\begin{array}{c} 1 \\ 0\end{array}   \right)=
   \left(\begin{array}{c} 0 \\ 1\end{array}   \right)=|1\rangle,
\ee
og
\be
   \OP{H}_{\mathrm{NOT}}|1\rangle=   
   \left(\begin{array}{cc} 0 & 1 \\ 1& 0\end{array}\right)
   \left(\begin{array}{c} 0 \\ 1\end{array}   \right)=
   \left(\begin{array}{c} 1 \\ 0\end{array}   \right)=|0\rangle.
\ee
Det siste er ikke noe annet enn en alternativ m\aa te \aa\
uttrykke virkningen av en standard NOT-krets i elektronikk.

En annen viktig operasjon er den s\aa kalte faseskiftsoperasjonen, hvor vi 
kan skifte fasen til en av amplitudene ved hjelp av f.eks.~laserlys
med en bestemt frekvens. Matematisk kan vi uttrykke denne operasjonen
som
\be
   \OP{H}_{\Phi}|0\rangle=e^{i\phi}|0\rangle,
\ee
og
\be
   \OP{H}_{\Phi}|1\rangle=|1\rangle,
\ee
eller med operatoren p\aa\ f\o lgende form 
\be
   \OP{H}_{\Phi}=   
   \left(\begin{array}{cc} e^{i\phi} & 0 \\ 0& 1\end{array}   \right).
\ee

N\aa r vi skal lage kvantemekaniske kretser, kan vi tenke oss
at flere slike operasjoner p\aa\ en qubit tilstander utgj\o r
en bestemt endelig krets.


Hittil har vi sett p\aa\ en begynnelsestilstand som bare best\aa r av
en tilstand. 
Vi kan ogs\aa\ lage en 
begynnelsetilstand  gitt ved en superposisjon, f.eks.
\be
    |q\rangle_{in} = \alpha |0\rangle_{in} + \beta |1\rangle_{in},
\ee
hvor $\alpha$ og $\beta$ er to imagin\ae re konstanter. 
Virkningen av v\aa r str\aa lesplitter er gitt ved
\be
    |q\rangle_{ut}=\OP{H}|q\rangle_{in} = 
    \frac{1}{\sqrt{2}}((\alpha+\beta)|0\rangle_{ut} + 
    (\alpha-\beta)|1\rangle_{ut}),
\ee
hvor $\alpha+\beta$ er sannsynlighetsamplituden for \aa\ finne
partikkelen i den \o vre utg\aa ende str\aa len mens $\alpha-\beta$
er den tilsvarende amplituden for \aa\ finne den i nedre utg\aa ende
str\aa len. Velger vi enten $\alpha=0$ eller $\beta=0$, ser vi at
partikkelen har like stor sannsynlighet for \aa\ v\ae re i den nedre
utg\aa ende str\aa len som den \o vre. Velger vi $\alpha=\beta$ vil partikkelen
definitivt komme ut i den \o vre str\aa len. 
Virker vi en gang til p\aa\ den resulterende tilstanden f\aa r vi tilbake
begynnelsestilstanden,
\be 
   |q\rangle_{ut}=\OP{H}\OP{H}|q\rangle_{in}=|q\rangle_{in}.
\ee
Som et siste eksempel kan vi tenke oss en krets representert vha.~f\o lgende
operasjoner
\be 
   \OP{H}\OP{\Phi}\OP{H}|0\rangle,
\ee
hvis resultat er
\be
    \frac{1}{\sqrt{2}}((e^{i\phi}+1)|0\rangle + 
    (e^{i\phi}-1)|1\rangle).
\ee
Dersom $\phi=0$ f\aa r vi bit '0' som resultat mens for $\phi=\pi$
finner bit '1' som sluttresultat. Det betyr at faseskiftet $\phi$ 
gj\o r en i stand til \aa\ veksle mellom bit '0' eller bit '1'. 



\section{Entanglement og to-qubit tilstander}

Forestill deg n\aa\ en partikkelkilde som sender ut et par med partikler,
slik at en partikkel dukker opp til venstre mens den andre kommer ut til
h\o yre for kilden. Vi kan tenke oss at kilden er slik innretta at
partiklene som sendes ut kommer med motsatt retta bevegelsesmengder, eller
spinn, eller polarisasjonsretning for \aa\ nevne noen muligheter.
Vi gir partikkelen til venstre merkelappen partikkel '1' mens partikkelen
til h\o yre merkes som partikkel '2'. Dersom vi beholder eksemplet 
med str\aa lesplitteren, er det slik at dersom partikkel '1' kommer ut
i den \o vre str\aa len, vil alltid partikkel '2' komme ut i den
nedre str\aa len til h\o yre. I v\aa r bit-sjargong, betyr det at dersom
partikkel '1' kommer ut med bit '0', m\aa\ partikkel '2' komme ut med
bit '1', eller motsatt. Kvantemekanisk kan vi tenke oss denne tilstanden
av to partikler, eller to qubits om vi vil, gitt ved
to qubits kombinasjonen
\be
   \frac{|0\rangle_1 |1\rangle_2 + |1\rangle_1|0\rangle_2}{\sqrt{2}},
   \label{eq:superent}
\ee
hvor indeksene '1' og '2' henspeiler p\aa\ henholdsvis partikkel 1 og 2. 
Denne likningen beskriver det som kalles for en entangled tilstand (kryssa
tilstand) med den interessante egenskap at ingen av de to qubiten har en bestemt verdi. Men det vi finner ut om denne kvantetilstanden ved m\aa ling
p\aa\ en av qubitene (partiklene), en m\aa ling hvis resultat ikke er kjent
a priori, er at den andre qubiten har motsatt verdi av f.eks.~spinn
eller bevegelsesmengde.   
Det ser ut som om det kan v\ae re kvantekorrelasjoner, sj\o l om
m\aa lingen p\aa\ partiklene foretas n\aa r partiklene er langt
borte. Denne kvantemekaniske ikke-lokaliteten ga opphav til
den store disputten mellom Einstein og Bohr om paradokser
i kvantemekanikken. Tiltsanden i siste likning kalles ogs\aa\ for en 
Bell tilstanden (etter den irske fysikeren John Bell), eller EPR par 
(etter Einstein, Podolsky
og Rosen) og 
er en viktig byggestein i kvanteinformasjonsteori, teleportasjon og
kryptering.
Korrelasjonene som framkommer i slike tilstander
 har v\ae rt heftig debattert siden artikkelen til  
Einstein, Podolsky
og Rosen i 1935. 
John Bell viste at m\aa lingskorrelasjonene mellom slike kvantemekaniske
tilstander er mye sterkere enn de som finnes blant klassiske
systemer. 

I v\aa r diskusjon skal vi n\o ye oss med \aa\ sl\aa\ fast, p\aa\ lik
linje med dobbelspalt eksperiment hvor vi ikke kan fastsl\aa\ hvorvidt
partikkelen g\aa r gjennom spalt 1 eller to, at den kvantemekaniske
superposisjonen slik vi ser den i likning (\ref{eq:superent}) ikke 
tillater oss \aa\ si hvilke av de to en-qubit mulighetene 
g\aa r inn i tilstanden. Vi kan ikke si om qubit '1', partikkel '1',
er i  bit '0' eller '1', likes\aa\ om qubit '2', partikkel '2', er i  
bit '0' eller '1'. Men, dersom vi m\aa ler p\aa\ qubit '1' kan vi umiddelbart
si noe om hva slags tilstand qubit '2' er i. Disse betraktningene leder oss
til sp\o rsm\aa let om hvordan vi kan lage og observere s\aa kalte 'entangled'
tilstander. 

En mulighet er ved henfall av partikler med spinn 0 til to partikler
med halvtallig spinn. Det totale spinnet m\aa\ v\ae re bevart,
noe som betyr at de to utg\aa ende partiklene m\aa\ ha motsatt retta
spinn. Et slikt eksempel er et n\o ytralt pion, et boson (meson) med
masse 134 MeV/c$^2$, som kan henfalle til et elektron og et
positron, en partikkel med elektronets masse men motsatt ladning, et
s\aa kalt antielektron. Dersom det kun er spinnets verdi som tillater
oss \aa\ skille mellom mulige utkommer ved kilden, vil den resulterende 
to qubit kvantetilstanden se ut som f\o lger
\be
   \frac{|+\rangle_1 |-\rangle_2 - |-\rangle_1|+\rangle_2}{\sqrt{2}},
   \label{eq:superent1}
\ee
hvor $+$ og $-$ henspeiler til henholdsvis spinn opp og spinn ned, mens
indeksene 1 og 2 refererer til partikkel '1' og '2'. 
En m\aa ling p\aa\ partikkel '1' vil umiddelbart ogs\aa\ fastlegge
spinnet til partikkel '2'. Det er slik virkning 'uten vekselvirkning'
samt det kvantemekaniske superposisjonsprinsippet som danner grunnlaget
for det nye forskningsfeltet
om kvantedatamaskiner og kvante informasjonsteori. 


Vi avslutter dette avsnittet med en kort beskrivelse av to-qubit tilstander.
I neste avsnitt skal vi se hvordan vi kan bruke slike to-qubit tilstander
til \aa\ simulere kvantemekaniske kretser.

En en-qubit tilstand er gitt ved 
\be
    |q\rangle_{1} = \alpha_1 |0\rangle_{1} + \beta_1 |1\rangle_{1},
\ee
hvor indeksen '1' henviser til qubit '1'. Setter vi sammen en slik
en-qubit tilstand sammen med en annen en-qubit tilstand har vi
\be
     |q\rangle_{12}= |q\rangle_{1} |q\rangle_{2}
\ee
eller 
\be
   |q\rangle_{12}=\alpha_1 \alpha_2|0\rangle_1|0\rangle_2 +
                  \alpha_1 \beta_2|0\rangle_1|1\rangle_2 +
                  \beta_1 \alpha_2|1\rangle_1|0\rangle_2 +
                  \beta_1 \beta_2|1\rangle_1|1\rangle_2,
\ee
slik at v\aa re nye basistilstander blir
$|0\rangle_1|0\rangle_2$, $|0\rangle_1|1\rangle_2$, $|1\rangle_1|0\rangle_2$ 
og $|1\rangle_1|1\rangle_2$. 

Dersom alle koeffisientene er forskjellige fra null, kan vi lage en 
tilstand som n\aa\ er en superposisjon av fire ulike tilstander. 

Slik kan vi fortsette med flere qubits og lage superposisjoner 
av flere og mer kompliserte basistilstander. 


\section{Kvantemekaniske kretser}

I elektronikk er det slik at vi kan bygge alle mulige  type kretser vha.~kun
to basiskretser, en s\aa kalt NOT krets og en AND krets. Disse kan sl\aa s
sammen til en NAND krets, slik at  i grunnen trenger vi kun en basis krets.

En NOT krets er reversibel, dvs.~at vi kan fortelle utifra utgangsbiten
hvilken verdi inngangs-biten har. Dersom utgangs-biten har verdi '1', m\aa\
inngangs-biten ha verdi '0'. Kretsen er vist i Figur \ref{notgates}.
\begin{figure}[h]
\setlength{\unitlength}{4mm}
\begin{picture}(40,10)(-2,0)
\linethickness{0.4pt}
\put(5.00,3.00){\line(1,0){2.00}}
\put(2.00,0.00){\line(1,1){3.00}}
\put(2.00,6.00){\line(1,-1){3.00}}
\put(0.00,3.00){\line(1,0){2.00}}
\put(2.00,0.00){\line(0,1){6.00}}
\put(0.00,3,30){A}
\put(6.00,3.30){B}
\put(15.00,3.00){\begin{tabular}{cc}\hline A& B\\ \hline 0&1\\ 1& 0\\ \hline\end{tabular}}
\end{picture}
\caption{NOT krets med tilh\o rende 
verdiskjemaer. En NOT krets er s\aa kalt reversibel da 
vi kan lese av utgangsbiten B hvilken verdi A har.\label{notgates}}
\end{figure}


Problemet med klassiske kretser som en AND og en NAND krets 
er at de er irreversible.
Det betyr at vi ikke kan fortelle utifra utgangs-biten hva slags verdi
de to inng\aa ende bitene har. Figur \ref{classicgates} 
\begin{figure}[h]
\setlength{\unitlength}{4mm}
\begin{picture}(40,10)(-2,0)
\linethickness{0.4pt}
\qbezier(2.00,6.00)(7.00,6.00)(9.00,3.00)
\qbezier(2.00,0.00)(7.00,0.00)(9.00,3.00)
\put(9.00,3.00){\line(1,0){1.50}}
\put(0.00,5.00){\line(1,0){2.00}}
\put(0.00,1.00){\line(1,0){2.00}}
\put(2.00,0.00){\line(0,1){6.00}}
\put(0.00,5.30){A}
\put(0.00,1.30){B}
\put(10.70,2.90){C}
\put(15.00,3.00){\begin{tabular}{ccc}\hline A& B & C\\ \hline 0&0&0\\ 0&1&0\\ 1& 0& 0\\ 1& 1& 1\\ \hline\end{tabular}}
\end{picture}
\caption{AND krets med tilh\o rende 
verdiskjemaer. En AND krets er s\aa kalt irreversibel da 
vi kan ikke avlede av utgangsbiten C hvilken verdi A og B har,
unntatt for tilfellet hvor C$=1$. \label{classicgates}}
\end{figure}
viser en AND krets med tilh\o rende bitverdier. 
Dersom vi \o nsker \aa\ lage kvantekretser derimot m\aa\ vi ha kretser
som er reversible. Grunnen til det ligger i kravet om hermitiske
operatorer i kvantemekanikken. 
Enhver kvantekrets m\aa\ kunne formuleres som virkningen
av en eller annen kvantemekanisk operator $\OP{H}$. Generelt har vi da at
\be 
   |q\rangle_{ut}=\OP{H}|q\rangle_{in},
\ee
men siden $\OP{H}$ m\aa\  v\ae re hermitesk, dvs.
$\OP{H}^{\dagger}=\OP{H}$ og $\OP{H}^{\dagger}\OP{H}=1$, har vi
\be 
   \OP{H}^{\dagger}|q\rangle_{ut}=\OP{H}^{\dagger}\OP{H}|q\rangle_{in}=
   |q\rangle_{in}.
\ee
Vi ser av siste likning at vi kan f\aa\ ut 
begynnelsestilstanden utifra den 
inverse transformasjonen $\OP{H}^{\dagger}|q\rangle_{ut}$ p\aa\ sluttilstanden.
Det betyr igjen at vi m\aa\ kunne dedusere fra utgangs-biten(e)
hvilken verdi inngangs-biten(e) hadde. En slik krets kalles
reversibel og skiller seg fra en klassisk irreversibel  
AND krets. 

Det er mulig \aa\ vise innenfor informasjonsteori at en kan lage en
reversibel Turingmaskin vha.~kun to basis kretser, p\aa\ lik linje med
en klassisk irreversibel Turingmaskin basert p\aa\ kun en NAND krets.
Dvs.~at vi kan lage oss alle mulige typer reversible kretser vha.~kun
to basis kretser. Disse er en NOT krets og en CNOT krets. 

En CNOT krets, eller kontrollert NOT som den kalles, har blitt demonstrert
vha.~b\aa de ionefelle teknikker, hvor enkeltioners kvantemekaniske
tilstander representerer bit '0' og '1', og kjernemagnetisk resonans 
teknikker. 

Vi skal diskutere disse to kretsene i neste underavsnitt og avslutte
med en faktisk realisering av en CNOT krets. 
   
\subsection{CNOT kretser}
Vi har allerede stifta bekjentskap med en NOT gate, se tilbake
til avsnitt \ref{sec:qubitoper}. En NOT krets virker
kun p\aa\ en qubit. En CNOT krets er vist i 
Figur \ref{cnotgate} med tilh\o rende verdiskjema. 
\begin{figure}[h]
\setlength{\unitlength}{4mm}
\begin{picture}(40,10)(-2,0)
\linethickness{0.4pt}
\put(5.00,1.00){\circle*{3}}
\put(5.00,6.00){\circle{1.5}}
\put(0.00,6.00){\line(1,0){10.00}}
\put(0.00,1.00){\line(1,0){10.00}}
\put(5.00,1.00){\line(0,1){4.25}}
\put(0.00,6.30){C}
\put(0.00,1.30){T}
\put(9.50,6.30){C'}
\put(9.50,1.30){T'}
\put(15.00,3.00){\begin{tabular}{cccc}\hline C& T & C'& T'\\ \hline 0&0&0&0\\ 0&1&0&1\\ 1& 0& 1& 1\\ 1& 1& 1& 0\\ \hline\end{tabular}}
\end{picture}
\caption{CNOT krets med tilh\o rende 
verdiskjemaer. Bokstaven C st\aa r for kontroll-biten mens
T st\aa r for target-biten.\label{cnotgate}}
\end{figure}
En CNOT krets beskriver vekselvirkningen mellom to qubits.
Den ene qubiten kalles for kontroll-biten mens den andre kalles
target-biten. Vi bruker indeks $c$ for den f\o rste og indeks $t$ for den
siste. Dens virkning er slik at n\aa r kontrollbiten har verdi bit '0',
forblir targetbiten ogs\aa\ uforandra. Dersom kontroll-biten har verdien
bit '1', forandres target-biten fra f.eks.~bit '0'  til bit '1', eller
motsatt, dvs.~virkningen p\aa\ target-biten er som en NOT krets
dersom kontroll-biten har verdi bit '1'. 
Vi kan derfor tenke oss en kvantemekanisk begynnelsestilstand
gitt ved en qubit for kontroll-biten og en qubit for target-biten.
Virkningen kan da oppsummeres som f\o lgende
\be
   |0\rangle_{c}  |0\rangle_{t}\rightarrow
   |0\rangle_{c}  |0\rangle_{t},
\ee
\be
   |0\rangle_{c}  |1\rangle_{t}\rightarrow
   |0\rangle_{c}  |1\rangle_{t},
\ee
\be
   |1\rangle_{c}  |0\rangle_{t}\rightarrow
   |1\rangle_{c}  |1\rangle_{t},
\ee
og
\be
   |1\rangle_{c}  |1\rangle_{t}\rightarrow
   |1\rangle_{c}  |0\rangle_{t}.
\ee
Matematisk kan vi uttrykke en CNOT krets som en
$4\times 4$ matrise
\be
   \OP{H}_{\mathrm{CNOT}}=
    \left(\begin{array}{cccc} 1 & 0 & 0 &0  \\ 
                              0 & 1 & 0 &0  \\
                              0& 0 & 0 &1  \\
                              0 & 0 & 1 &0\end{array}\right),
\ee
som virker p\aa\ basistilstandene
$|0\rangle_c|0\rangle_t$, $|0\rangle_c|1\rangle_t$, $|1\rangle_c|0\rangle_t$ 
og $|1\rangle_c|1\rangle_t$. Som vektorer kan vi skrive disse tilstandene
p\aa\ f\o lgende vis
\be
   |0\rangle_c|0\rangle_t=\left(\begin{array}{c} 1  \\ 
                                                 0  \\
                                                 0  \\
                                                 0  \end{array}\right),
\ee
\be
   |0\rangle_c|1\rangle_t=\left(\begin{array}{c} 0  \\ 
                                                 1  \\
                                                 0  \\
                                                 0  \end{array}\right),
\ee
\be
   |1\rangle_c|0\rangle_t=\left(\begin{array}{c} 0  \\ 
                                                 0  \\
                                                 1  \\
                                                 0  \end{array}\right),
\ee
og 
\be
   |1\rangle_c|1\rangle_t=\left(\begin{array}{c} 0  \\ 
                                                 0  \\
                                                 0  \\
                                                 1  \end{array}\right).
\ee

Som et eksempel kan vi rekne
\be
   \OP{H}_{\mathrm{CNOT}}|1\rangle_c|1\rangle_t=
    \left(\begin{array}{cccc} 1 & 0 & 0 &0  \\ 
                              0 & 1 & 0 &0  \\
                              0& 0 & 0 &1  \\
                              0 & 0 & 1 &0\end{array}\right)
                              \left(\begin{array}{c} 0  \\ 
                                                 0  \\
                                                 0  \\
                                                 1  \end{array}\right)=
\left(\begin{array}{c} 0  \\ 
                                                 0  \\
                                                 1  \\
                                                 0  \end{array}\right),
\ee
som er 
$|1\rangle_c|0\rangle_t$, dvs.~target-biten forandrer verdi fra bit '1'
til bit '0' n\aa r kontroll-biten har verdi bit '1'.


\section{CNOT kretser og Schr\"odingers katt tilstander}
Vi  avslutter dette kapitlet med en diskusjon av det som kalles
hyperfinsplitting, som har sitt opphav i koplingen mellom kjernespinnet
og det totale spinnet til elektronene, og hvordan denne effekten 
kan nyttes til \aa\ lage b\aa de en CNOT krets og en  virkelig 
Schr\"odingers katt tilstand.

Ser vi tilbake p\aa\ alkalimetallene og tar for oss Natrium,
s\aa\ husker vi at grunntilstanden er gitt ved den spektroskopiske
termen $^2S_{1/2}$ mens den f\o rste eksiterte tilstanden er gitt
ved eksitasjonen av et elektron i $3s$ orbitalen til $3p$ orbitalen.
Pga.~spinnbane vekselvirkningen splittes det f\o rste eksiterte
niv\aa et i to, ett med totalt spinn $J=1/2$ og et med $J=3/2$, noe som
gir henholdsvis $^2P_{1/2}$ og $^2P_{3/2}$ som spektroskopiske
termer. Finstrukturoppslittingen mellom disse to niv\aa ene er gitt
ved en energidifferanse p\aa\ ca $0.0022$ eV, eller en 
frekvens $\nu\approx 515$ GHz. Energioppsplittingen
blir litt mer komplisert dersom vi tar hensyn til vekselvirkningen
mellom det totale kjernespinnet $I$ og det totale spinnet til 
valenselektronet $J$. Siden kjernen best\aa r av fermioner,
dvs.~partikler med halvtallig spinn, kan kjernespinnet ta verdiene
$I=0,1/2,1,3/2,2,\dots$.

Kjernespinnet og elektronspinnet kan igjen koples til et
totalt spinn $\OP{F}=\OP{I}+\OP{J}$. En og samme verdi for $F$ 
kan oppn\aa s ved ulike $I$ og $J$, p\aa\ lik linje med hva vi fant
for det totale spinn $J$ pga.~ulike verdier av $L$ og $S$.
Energitilstanden som skyldes ulike $F$ verdier er dermed splitta via
det som kalles for en hyperfin oppslitting gitt ved
$A\OP{I}\OP{J}$, p\aa\ lik linje med spinnbane vekselvirkningen, med
$A$ en konstant.
Energiforksjellen mellom to
hyperfin niv\aa er er gitt ved
\be
\Delta E_{\mathrm{hfs}}(F)-\Delta E_{\mathrm{hfs}}(F-1)=
\hbar AF+3\hbar BF\frac{F^2-I(I+1)-J(J+1)+1/2}{2I(2I-1)J(J-1)},
\ee
med $F$ den st\o rste verdien og med 
$A$ og $B$ to konstanter som kan bestemmes ved bruk av
 eksperimentelle spektra. 

For Natrium, har kjernespinnet verdien $I=3/2$, som betyr
at grunntilstanden har $F=1$ og $F=2$. Energiforskjellen er gitt
ved en frekvens p\aa\ $\nu\approx 1.77$ GHz. For den eksiterte
$^2P_{3/2}$ tilstanden har vi $F=0,1,2$ og $3$ med en oppslitting
p\aa\ ca.~100 MHz. Med et ytre p\aa satt magnetfelt kompliseres
dette bildet ytterligere.

Slike hyperfin tilstander har blitt brukt nylig, sammen 
med teknikker for \aa\ fange inn enkeltioner i avgrensa omr\aa der
til \aa\ lage b\aa de Schr\"odingers katt tilstander
og simulere en CNOT kretser. 

Schr\"odingers katt paradosket blei presentert av Schr\"odinger i 1935
og har sitt utgangspunkt i det kvantemekaniske superposisjonsprinsippet
og det faktum at kvantemekanikken som teori har som m\aa lsetting
\aa\ beskrive b\aa de makroskopiske og mikroskopiske frihetsgrader.

Kort fortalt g\aa r paradokset ut p\aa\ f\o lgende. En katt 
befinner seg i et lukka rom sammen med en beholder med en radioaktiv
kjerne hvis sannsynlighet for \aa\ henfalle innen en time er
50\%. N\aa r kjernen henfaller knuses en beholder med en giftig
gass som deretter tar livet av katten. 
Kvantemekanisk kan vi uttrykke systemet som best\aa r av b\aa de
kjernen og katten som en superposisjon av to muligheter
\be
   |\Psi\rangle=\frac{1}{\sqrt{2}}
             \left(|\mathrm{katt}\rangle_{\mathrm{ikke-levende}}
             |\mathrm{kjerne}\rangle_{\mathrm{henfalt}}+
             |\mathrm{katt}\rangle_{\mathrm{levende}}
             |\mathrm{kjerne}\rangle_{\mathrm{ikke-henfalt}}\right).
             \label{eq:cateq}
\ee 
Klassisk er det neppe trivielt \aa\ tenke seg en slik tilstand. 
Dersom vi foretar en m\aa ling, f.eks.~ved \aa\ kikke inn i beholderen,  
av systemet best\aa ende av b\aa de katten og kjernen
etter en time, vil b\o lgefunksjonen klappe sammen til en av
l\o sningene med en sannsynlighet p\aa\ 50\%. Det er egentlig alt
hva vi kan si om systemet. 

Vi legger merke til at tilstanden vi har 
beskrevet ovenfor likner p\aa\ to-qubit tilstandene vi diskuterte
i de to foreg\aa ende  avsnitt. 
Sp\o rsm\aa let vi da stiller oss er om det overhodet er mulig
\aa\ lage slike Schr\"odingers katt tilstander. 

Vha.~ionefelle teknikker og laserteknologi, 
klarte forskere ved National Institute
of Standards i Boulder, Colorado, \aa\ lage 
b\aa de Schr\"odingers katt tilstander og \aa\ simulere en CNOT krets.

Enkeltioner kan idag fanges inn i sm\aa\ omr\aa der,
noen f\aa\ nanometere, vha.~s\aa kalte ionefeller. Ladde partikler, som f.eks.~et ion, kan 'fanges' vha.~elektromagnetiske felter, enten
ved \aa\ bruke en kombinasjon av enten statiske elektriske
og magnetiske felt (s\aa kalt Penning trap) eller 
et tidsavhengig inhomogent elektrisk felt (s\aa kalt Paul trap).

For \aa\ kunne 'fange' inn et ion, trengs ei kraft $\bf{F}$,
f.eks.~$\bf{F}\propto \bf{r}$ hvor $\bf{r}$ er avstanden fra et valgt origo
i ionefella. Slike krefter kan oppn\aa s vha.~et kvadropol potensial
\be
   \Phi=\Phi_0(\alpha x^2+\beta y^2+\gamma y^2)/r_0^2,
\ee
hvor $\Phi_0$ er et anvendt spenningsfall for en kvadropol elektrode,
$r_0$ er den typiske st\o rrelsen for fella mens konstantene $\alpha$,
$\beta$ og $\gamma$ bestemmer forma p\aa\ potensialet. 

Vi skal ikke g\aa\ inn i detalj her (forskning i dette  feltet resulterte i
Nobelprisen i Fysikk i 1989 til Dehmelt og Paul) men vi legger
merke til at potensialet v\aa rt likner et harmonisk
oscillator potensial.
Avgrenser vi oss til et system i en dimensjon kan vi n\aa\ forestille
oss et ion som er fanga inn i et harmonisk oscillator potensial
$1/2kx^2$, med fj\ae rkonstanten  $k=\Phi_0\alpha/r_0^2$, som vist i Figur
\ref{iontrap}.
\begin{figure}[h]
\setlength{\unitlength}{1mm}
\begin{picture}(140,60)(0,0)
\linethickness{0.4pt}
\qbezier(60,50)(80,-20)(100, 50) 
\put(50,20){\line(1,0){60}}
\put(50,22){$n=0$}
\put(50,32){$n=1$}
\put(50,42){$n=2$}
\put(50,30){\line(1,0){60}}
\put(80,20){\circle*{5}}
\put(50,40){\line(1,0){60}}
\end{picture}
\caption{Skjematisk framstilling av et ion (svart sirkel) 
som er fanga inn i et
harmonisk oscillator potensial. Energiniv\aa ene som er
inntegna er ulike harmonisk oscillator tilstander.
I tillegg har vi ogs\aa\ elektroniske eksitasjonsfrihetsgrader
i ionet ogs\aa\ .\label{iontrap}}
\end{figure}
I denne figuren har vi inntegna harmonisk oscillator eksitasjonsenergiene.
I tillegg utviser ogs\aa\ ionet indre elektroniske eksitasjoner.
Det er kombinasjonen av disse to frihetsgradene som tillot forskere ved
National Institute
of Standards i Boulder, Colorado, \aa\ lage 
b\aa de Schr\"odingers katt tilstander og \aa\ simulere en CNOT krets.

Vi kan tenke oss at eksitasjonsfrihetsgradene fra den harmoniske
oscillator representerer katten i likning (\ref{eq:cateq}). Vi kan
la grunntilstanden i den harmoniske oscillator br\o nnen erstatte
tilstanden hvor katten er i live mens den f\o rste eksiterte
harmonisk oscillator energien erstatter tilstanden hvor katten
er d\o d. 

Forskerne i Colorado brukte Be$^+$ som ion i eksperimentet sitt.
Be$^+$ har som grunntilstand konfigurasjonen $1s^22s^1$ med en hyperfin
tilstand gitt ved $F=1$. Den f\o rste eksiterte
tilstanden som forskerne brukte hadde hyperfin kvantetallet 
$F=2$. I fors\o ket p\aa\ lage en Schr\"odingers katt tilstand,
kan vi la disse to tilstandene til ionet representere den henfallende
kjernen i likning (\ref{eq:cateq}). V\aa r nye 
likning (\ref{eq:cateq}) blir f.eks.~
\be
   |\Psi\rangle=\frac{1}{\sqrt{2}}
             \left(|0\rangle_{\mathrm{HO}}
             |g\rangle_{\mathrm{Be}^+}+
             |1\rangle_{\mathrm{HO}}
             |e\rangle_{\mathrm{Be}^+}\right),
             \label{eq:cateq1}
\ee 
hvor $|0\rangle_{\mathrm{HO}}$ og $|1\rangle_{\mathrm{HO}}$
er henholdsvis den laveste og f\o rste eksiterte harmonisk oscillator
tilstandene mens $|g\rangle_{\mathrm{Be}^+}$ og 
$|e\rangle_{\mathrm{Be}^+}$ er henholdsvis grunntilstanden og den
f\o rste eksiterte tilstanden i Be-ionet.
Begynnelsestilstanden er gitt ved 
$|0\rangle_{\mathrm{HO}}|g\rangle_{\mathrm{Be}^+}$. En laserpuls 
lager tilstandene $|0\rangle_{\mathrm{HO}}|g\rangle_{\mathrm{Be}^+}$
og $|0\rangle_{\mathrm{HO}}|e\rangle_{\mathrm{Be}^+}$. 
Deretter manipuleres harmonisk oscillator eksitasjonsfrihetsgradene
slik at vi til slutt ender opp med en line\ae r kombinasjon av
tilstandene  $|0\rangle_{\mathrm{HO}}|g\rangle_{\mathrm{Be}^+}$
og  $|1\rangle_{\mathrm{HO}}|e\rangle_{\mathrm{Be}^+}$. 

Dette oppsette blei ogs\aa\ brukt til \aa\ demonstrere hvordan
en kan bruke kvantemekaniske tilstander til \aa\ lage
en CNOT krets. 
For \aa\ simulere en slik CNOT krets kan vi la de to
harmonisk oscillator tilstandene representere en kontroll-qubit
mens de to hyperfintilstandene i Be-ionet kan reserveres som bit '0'
og bit '1' for target-biten, dvs.
\be
   |0\rangle_t=|g\rangle_{\mathrm{Be}^+},
\ee
\be
   |1\rangle_t=|e\rangle_{\mathrm{Be}^+},
\ee
\be
   |0\rangle_c=|0\rangle_{\mathrm{HO}},
\ee
og 
\be
   |1\rangle_c=|1\rangle_{\mathrm{HO}}.
\ee
Vha.~ulike laserpulsers virkning p\aa\ disse fire basistilstandene
 kunne dermed forskningsgruppen ved Boulder i Colorado implementere
og demonstrere en CNOT krets, se www.boulder.nist.gov for mer informasjon.











\clearemptydoublepage

\end{document}

