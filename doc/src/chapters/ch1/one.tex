\chapter{BRUDDET MED KLASSISK FYSIKK}
\begin{quotation}
Quantum mechanics: Real black magic calculus. {\em Albert Einstein}
\end{quotation}
\begin{quotation}
Anybody who is not shocked by quantum theory has not understood it. {\em Niels Bohr}
\end{quotation}

Kursets f\o rste del har  som hensikt \aa\ gi dere 
en viss oversikt over den historiske utviklingen som f\o rte 
til formuleringen av kvantemekanikken i 1925.
Det er p\aa\ ingen m\aa te en fullstendig historisk oversikt. Vi vil kun ta for
oss det vi anser som de viktigste bitene p\aa\ denne vei, deriblant en 
beskrivelse av svart legeme str\aa ling vha.\ Plancks kvantiseringshypotese,
fotoelektrisk effekt, Compton spredning, R\"ontgen str\aa ling, Bohrs
atommodell, materiens partikkel og b\o lgeegenskaper samt om Heisenbergs
uskarphetsrelasjon. I tillegg, vil vi ogs\aa\ ta for oss litt b\o lgel\ae re
som er relevant for dette kurset. Men f\o rst en generell introduksjon.

\section{Introduksjon}

Ved starten av forrige \aa rhundre kan en kanskje beskrive situasjonen 
i Fysikk som Pandoras
boks av eksperimentelle observasjoner som, med basis i veletablert klassisk
fysikk, ikke lot seg forklare.  
V\aa r historie begynner dermed ved slutten og 188-tallet og 
starten av forrige \aa rhundre,
n\ae rmere bestemt 14 desember 1900, da Planck foreleste om 
sin teori for svart legeme str\aa lning til det Tyske Fysiske
selskap. En rekke fysiske fenomen, fotoelektrisk effekt, diskrete
spektra fra ulike gasser, R\"ontgen str\aa lning m.m., kunne ikke
forklares vha.~det vi idag kaller for klassisk fysikk, dvs.~den 
tids bevegelseslover for naturen, utrykt vha.~Newtons og Maxwells likninger.
Teorien resulterte i flere inkonsekvenser i forhold til eksperiment, 
slik som den s\aa kalte 'ultrafiolette 
katastrofe', eller at elektroner kunne klappe sammen med kjernen.  

I f\o rste omgang ble disse problemene l\o st vha.~{\em ad hoc} hypoteser,
slik som Plancks kvantiserings hypotese eller Bohrs atommodell. Den historiske
gangen viser oss ogs\aa\ klart at fysikk er et eksperimentelt fag, {\em teorier
om naturen utvikles h\aa nd i h\aa nd med eksperiment.}

Etterhvert som en fikk st\o rre eksperimentell innsikt om b\aa de atomer og
str\aa ling, viste slike {\em ad hoc} forklaringer seg som utilstrekkelige.
Krisen i klassisk fysikk kom til sin ende i 1925 med formuleringen
av kvantemekanikken.
Her f\o lger noen viktige oppdagelser og teorier som var 
med \aa\ forme den f\o rste tida. \newline\newline
\begin{tabular}{lll}
1898 & Madam Curie & Radioaktiv polonium og radium\\
1900 & Planck & Plancks kvantiserings hypotese og svart legeme str\aa ling\\
1905 & Einstein & Fotoelektrisk effekt \\
1911 & Rutherford & Atommodell\\
1913 & Bohr & Kvanteteori for atomspektra\\
1922 & Compton & Spredning av fotoner p\aa\ elektroner\\
1923 & Goudsmit og Uhlenbeck & Elektronets egenspinn\\
1924 & Pauli & Paulis ekslusjonsprinsipp \\
1925 & De Broglie & Materieb\o lger\\
1926 & Schr\"odinger & B\o lgelikning og ny naturlov\\
1927 & Heisenberg & Uskarphetsrelasjonen \\
1927 & Davisson og Germer & Eksperiment som p\aa viste materiens b\o lgeegenskaper\\
1927 & Born & Tolkningen av b\o lgefunksjonen \\
1928 & Dirac & Relativistisk kvantemekanikk og prediksjon av positronet 
\end{tabular}\newline\newline
Kvantemekanikken har v\ae rt et unnv\ae rlig verkt\o y i v\aa r s\o ken etter
\aa\ beskrive naturen, fra \aa\ forklare hvordan sola skinner, til studier
av atomer og deres struktur og understrukturer, supraledning, kjernefusjon
i stjerner, n\o ytronstjerner, strukturen til DNA molekylet,  
element\ae rpartiklene i naturen og m.m. Og kvantemekanikken ligger til grunn
for store deler av v\aa r n\aa v\ae rende og framtidige teknologiske utvikling.

Fra et mer filosofisk st\aa sted kan vi si at 
dialektikken mellom eksperiment og teori var med \aa\ 
forme en ny naturvitenskapelig
filosofi. Som basis for v\aa r forst\aa else av naturen har vi erstatta
den objektive determinismen gitt ved f.eks.~Newtons lover med en 
subjektiv og sannsynlighetsbestemt determinisme. 

\subsection{Hva er kvantemekanikk?} 

Kvantemekanikk er et matematisk byggverk, et sett av regler for \aa\
lage fysiske teorier om naturen, en matematisk m\aa te 
\aa\ uttrykke naturlover.
Schr\"odingers likning er v\aa r naturlov, b\aa de for mikroskopiske
og makroskopiske systemer. Settet med regler inneholder ogs\aa\ tolkninger av 
teorien samt ulike postulater, f.eks.~hvordan en m\aa ling av en fysisk
st\o rrelsen skal defineres. 
Reglene er enkle men h\o yst ikke-trivielle i sine tolkninger, noe som har leda
og leder til interessante kontroverser om naturens egenskaper og v\aa r 
evne til \aa\ forst\aa\ den. Den kanskje mest kjente kritikeren av 
kvantemekanikk er vel Einstein, som var, sammen med Planck, en av teoriens
'jordm\"odre.

I tillegg, og her er det viktig \aa\ skille mellom kvantemekanikken som teori
og postulatene om naturen som resulterte i  etableringen av kvantemekanikken. 
Kvantemekanikk som teori baserer seg p\aa\ bla.~flere viktige 
p\aa stander om naturen. 
Det er fire viktige postulater som danner grunnlaget for kvantemekanikkens
beskrivelse av naturen og Schr\"odingers likning som bevegelseslov og som
dere kommer til \aa\ f\aa\ presentert i dette kurset.

Disse fire postulatene har ingen klassisk analog, og utgj\o r et sett
med  p\aa stander om naturen. 

\begin{itemize} 
\item Einsteins og Plancks postulat om energiens kvantisering 
     
      \[
          E=nh\nu,
      \]
       hvor $\nu$ er frekvensen og $h$ er Plancks konstant. Tallet 
       $n$ er et heltall og kalles for et kvantetall. 
\item De Broglie sitt postulat om materiens b\o lge og partikkel
      egenskaper. Det uttrykkes vha.~relasjonene
\[
     \lambda =\frac{h}{p}  \hspace{1cm} \nu=\frac{E}{h}    
\]
hvor $\lambda$ er b\o lgelengden og $p$ bevegelsesmengden.
Partikkelegenskapene uttrykkes via energien og bevegelsesmengden,
mens b\o lgelengden og frekvensen uttrykker b\o lgeegenskapene.
\item Heisenbergs uskarphetsrelasjon. 
\[
    \Delta {\bf p}\Delta {\bf x} \geq \frac{\hbar}{2},
\]
Klassisk er det slik at ${\bf x}$ og ${\bf p}$ er uavhengige st\o rrelser.
Kvantemekanisk  har vi en avhengighet gitt ved
Heisenbergs uskarphets relasjon. Dette impliserer igjen at vi ikke kan 
lokalisere en partikkel og samtidig bestemme dens bevegelsesmengde skarpt. 
En ytterligere konsekvens er at vi ikke kan i et bestemt eksperiment observere
b\aa de partikkel og b\o lgeegenskaper samtidig. 

\item Paulis eksklusjonsprinsipp: den totale b\o lgefunksjonen for et 
      system som best\aa r av identiske partikler med halvtallig
      spinn m\aa\ v\ae re antisymmetrisk.  Det har som f\o lge at i en
      sentralfelt modell som anvendes i f.eks.~atomfysikk, s\aa\ kan ikke
      to eller flere elektroner ha samme sett kvantetall. 
      Partikler med halvtallig spinn kalles for fermioner. Eksempler
      er elektroner, protoner, n\o ytroner, kvarker og n\o ytrinoer.
For partikler med heltallig spinn
      m\aa\ b\o lgefunksjonen v\ae re symmetrisk.
      Partikler med heltallig spinn kalles bosoner. Eksempler er
      fotoner, Helium atomer,  ulike mesoner og gluoner. 
\end{itemize}


\subsection{Hvorfor er kvantemekanikk spennende?}

Uten \aa\ ta munnen for full, dere som f\o lger dette kurset 
har en utrolig spennende tid \aa\ se fram til! Hvorfor? 


Kvantefysikken inneholder alts\aa\ en del postulater med konsekvenser
for v\aa r forst\aa else av naturen som er h\o yst ikke-trivielle.
Slik kvantemekanikken framst\aa r idag, utgj\o r den v\aa r beste
forst\aa else av naturen. Schr\"odingers likning, 
hvis tilh\o rende l\o sning forteller om egenskaper til et mikrosystem,  
er v\aa r naturlov. 

Fram til ca.~1970 kan vi si at mye av den eksperimentelle
informasjonen vi hadde om mikrosystemer i all hovedsak dreide seg om
systemer med mange partikler, f.eks.~mange enkeltatomer. 
Grovt sett kan en si at vi ikke hadde tilgang til informasjon om 
kvantemekaniske enkeltsystemer som f.eks.~et atom. Supraledning er et slikt 
eksempel. Her har vi en makroskopisk manifestasjon av en kvantemekanisk 
effekt, men vi kan ikke trekke ut eksperimentell informasjon om 
enkeltelektronene som bidrar.   
Siden 1970 har det blitt utviklet teknikker, f.eks.~det som g\aa r under
navnet ionefeller (Nobel pris i fysikk i 1989), 
hvor vi vha.~f.eks.~elektriske kvadrupolfelt 
kan fange inn enkeltatomer i sm\aa\
omr\aa der som er isolerte fra omgivelsene. Deretter kan vi studere 
ulike kvantemekaniske frihetsgrader til dette enkeltatomet ved f.eks.~\aa\
sende
laserlys med bestemte frekvenser. 
Den nye eksperimentelle og teknologiske situasjonen vi er ved kan v\ae re
med \aa\ legge grunnlaget for   
           \begin{itemize}
               \item Ny teknologi
               \item Nye fagfelt samt overlapp med flere eksisterende
                     felt, fysikk, matematikk, informatikk, kjemi m.m.
               \item Eksisterende teknologi gj\o r at vi kan studere
                     og kanskje utnytte sider av kvantemekanikken
                     som anses   som mindre
                     trivielle. Eksempler er Schr\"odingers katt 
                     paradokset og 'Entanglement' (mere om dette senere).  
               \item Kanskje vi utvikler ogs\aa\ en bedre forst\aa else av
                     naturen, {\bf en ny og bedre teori?}
               \item Det er ogs\aa\ en interessant parallell til 
                     begynnelsen av forrige \aa rhundre. Flere 
                     eksperiment (fotoelektrisk effekt, svart legeme str\aa ling m.m.) kunne ikke forklares vha.~klassisk fysikk. Det ledet igjen  
til utviklingen
av kvantemekanikken rundt 1925.  Med dagens teknologi kan vi f.eks.~fange inn enkeltatomer og elektroner i sm\aa\ omr\aa der (noen f\aa\ nanometre) 
og studere tilh\o rende kvantemekaniske egenskaper.                 
           \end{itemize}



V\aa r m\aa lsetting er \aa\ gi dere en introduksjon til 
kvantemekanikken, hvor vi vektlegger den historiske
gangen fram til Schr\"odingers likning, forst\aa\ enkle
kvantemekaniske systemer og det periodiske systemet. I tillegg,
tar vi med oss anvendelser fra moderne forskningsfelt som kvantedatamaskiner,
litt om molekyler, halvledere og til slutt litt kjerne og partikkelfysikk.

Matematikken i dette kurset er ikke vanskelig, selv om en del 
manipulering med matematiske uttrykk kan virke innfl\o kt 
innledningsvis. Den formelle 
matematiske formalismen som kjennetegner kvantemekanikk vektlegges ikke
i dette kurset. Videreg\aa ende emner som FYS 201 har dette som et viktig
tema.
De vanskeligste matematiske problem vi kan komme i kontakt med er integral
av typen
\[
   \int_a^be^{-\alpha x} x^n dx,
\]
og
\[
   \int_a^be^{-\alpha x^2} x^n dx,
\]
med $\alpha$ en reell positiv konstant og $n$ et positivt tall.
Bokstavene $a$ og $b$ representerer integrasjonsgrensene. 
I tillegg kommer kjennskap til regning med komplekse tall og variable.

For de av dere som er av den  ut\aa lmodige typen og t\o rster lengselsfullt  
etter en mer formell beskrivelse enn det som gis her, 
kan et alternativ v\ae re \aa\ f\o lge FYS 201 (et veldig bra kurs) 
parallellt med dette kurset.    

\section{Enheter i kvantefysikk}

I kvantefysikk er vi opptatt av \aa\ beskrive fysiske fenomen
p\aa\ det vi kan kalle mikroskala. Typiske lengdeskalaer av interesse
g\aa r fra $10^{-8}$ m ned til $=10^{-18}$ m. Enhetene som da benyttes er
{\bf nm}, leses nanometer, som er $10^{-9}$ m og {\bf fm}, leses femtometer,
og er gitt ved  1fm$=10^{-15}$ m.
Nanometer er lengdeskalaen i atomfysikk, 
             faste stoffers fysikk og molekylfysikk. Tilsvarende anvendes
femtometer i kjerne og
             partikkelfysikk. Senere i dette kurset skal vi se at
det er en sammenheng mellom de kreftene som virker (f.eks.~Coulomb 
vekselvirkningen i atomfysikk) og et systems energi og dermed dets lengdeskala. 

For \aa\ gi dere et enkelt eksempel p\aa\ de lengdeskalaer som vi skal
befatte oss med, la oss hente fram Avogadros tall
\[
   N_A=6.023\times 10^{23},
\]
som betyr at det er $N_A$ atomer i $A$ gram av et ethvert element, 
hvor $A$ er det atom\ae re massetall. Det vi si at et 1 g hydrogen,
12 g karbon ($^{12}$C) og 238 g uran ($^{238}$U) har like mange atomer.
La oss s\aa\ anta at vi har et gram av flytende hydrogen
og stiller oss selv sp\o rsm\aa let om hvor stor utbredelse
et hydrogenatom har, dvs. hvor stor er diameteren til hydrogenatomet
som best\aa r av et elektron og et proton. 
Vi har oppgitt at tettheten $\rho$ av flytende hydrogen
er $\rho=71$ kg/m$^3$.
Volumet opptatt av et gram er da
\begin{equation}
   V=\frac{10^{-3}\hspace{0.1cm}\mathrm{kg}}{\rho},
\end{equation}
og volumet opptatt av et atom er da
\begin{equation}
   V_{\hspace{0.1cm}\mathrm{atom}}=\frac{V}{N_A}=\frac{10^{-3}\hspace{0.1cm}\mathrm{kg}}{N_A\rho}=
                     \frac{10^{-3}\hspace{0.1cm}\mathrm{kg}}{71\hspace{0.1cm}\mathrm{kg/m}^2
                      6.023\times 10^{23}}=2.3\times 10^{-29} \hspace{0.1cm}\mathrm{m}^3.  
\end{equation}
Deretter antar vi at denne v\ae sken best\aa r av tettpakka kuler av hydrogen
atomer, slik at vi kan sette diameteren $d$ 
\begin{equation}
   d\sim V_{atom}^{1/3} = 3\times 10^{-10}\hspace{0.1cm}\mathrm{m}=0.3 \hspace{0.1cm}\mathrm{nm}.
\end{equation}
Vi skal senere i kurset ( i kapittel 7 i tektsboka) se at n\aa r vi regner
ut den gjennomsnittlige diameteren for hydrogenatomet vha.\
kvantemekanikk, vil vi finne en liknende st\o rrelsesorden for diameteren.
Dette enkle eksempel er ment som en illustrasjon p\aa\
de lengdeskalaer som er av betydning for det vi skal drive med her.
I forbifarten kan vi nevne at radius til et proton er ca.\ 1 fm,
mens radius til en atomkjerne (uten elektronene, kun protoner og n\o ytroner)
slik som bly er p\aa\ ca 7 fm. Dette forteller ogs\aa\ noe om 
at de sterke kjernekreftene som holder 
en atomkjerne sammen har kort rekkevidde (mere om dette
i kapittel 14 i l\ae reboka).

La oss n\aa\ introdusere den viktige energienheten v\aa r.
Fra FYS-ME1100 og FYS1120 har dere v\ae rt vant med Joule som 
energienhet, J=kgm$^2$/s$^2$. I FYS2140 vil vi operere med energiskalaer
av typen $10^{-19}$ J. Da er det hensiktsmessig \aa\ innf\o re en ny
energienhet, {\bf elektronvolt} med enhet eV.
Fra
 FYS2140 har vi at elektronets
ladning er gitt ved 
\begin{equation}
    e=1.602\times 10^{-19}\hspace{0.1cm}\mathrm{C},
\end{equation}
og at 1 V =1J/1C. 
I Fys2140 
 definerer vi
1 eV som den mengde kinetisk
energi som et elektron f\aa r n\aa r det akselereres gjennom en potensial
differanse p\aa\ 1 V.
Setter vi n\aa\
\begin{equation}
1\hspace{0.1cm}\mathrm{J}=1\hspace{0.1cm}\mathrm{V}1\hspace{0.1cm}\mathrm{C}=1\hspace{0.1cm}\mathrm{V}1\hspace{0.1cm}\mathrm{C}\frac{e}{e}=
            1\hspace{0.1cm}\mathrm{V}1\hspace{0.1cm}\mathrm{C}\frac{e}{1.602\times 10^{-19}\hspace{0.1cm}\mathrm{C}}=
\frac{1\hspace{0.1cm}\mathrm{eV}}{1.602\times 10^{-19}},
\end{equation}
har vi at 
\begin{equation}
   1 \hspace{0.1cm}\mathrm{eV}=1.602\times 10^{-19} \hspace{0.1cm}\mathrm{J}.
\end{equation}
Vi kan n\aa\ omregne hvileenergien til
elektronet $E_0^{\mathrm{elektron}}=m_ec^2$, hvor massen til elektronet er
\begin{equation}
    m_e=9.11 \times 10^{-31} \hspace{0.1cm}\mathrm{kg},
\end{equation}
i enheter eV ved \aa\ sette
             \begin{equation}
                m_ec^2= 9.11\times 10^{-31}(3\times 10^8)^2
                                       \hspace{0.1cm}\mathrm{J}=
                (9.11\times 10^{-31}(3\times 10^8)^2/1.602\times 10^{-19})
                                       \hspace{0.1cm}\mathrm{eV},
             \end{equation}
som gir
             \begin{equation}
                E_0^{\mathrm{elektron}}=m_ec^2=5.11\times 10^5 \hspace{0.1cm}\mathrm{eV}
             \end{equation}
             eller 0.511 MeV, med 1 MeV = 1000000 eV.
{\bf For massen brukes ofte}
             \begin{equation}
                m_e= E_0^{\mathrm{elektron}}/c^2
             \end{equation}
              dvs at vi skriver
$m_e=0.511$ MeV/$c^2$, som leses MeV-over-c-i-andre, men i bekvemmelighetens
\aa nd forkortes den oftest
til bare MeV.
             For protonet har vi  $m_p=938$ MeV/$c^2$.
       I atomfysikk, faste stoffers fysikk og molekylfysikk
             har vi energier p\aa\ st\o rrelsesorden {\bf med noen eV},
             i all hovedsak er det Coulomb vekselvirkningen som gir 
             vesentlige bidrag til disse systemenes fysikk.
       I kjerne og partikkelfysikk opererer vi med energier
             p\aa\ st\o rrelse med MeV, GeV (=1000 MeV) (massen til kvarker
             og tunge bosoner) eller TeV (massen til den mystiske Higgs
             partikkelen). Maks bindingsenergi til kjerner, n\aa r vi ser bort
             fra hvilenergien til protoner og n\o ytroner, er p\aa\ ca.
             8 MeV ($^{56}$Fe).
       Andre nyttige st\o rrelser er Plancks konstant $h$
             \begin{equation}
               h=6.626\times 10^{-34} \hspace{0.1cm}\mathrm{Js}
             \end{equation}
men oftest brukes $\hbar$ (leses h-strek) 
              \begin{equation}
               \hbar=\frac{h}{2\pi}=6.582\times 10^{-16} \hspace{0.1cm}\mathrm{eVs}.
             \end{equation}
I tillegg, forekommer $\hbar$ ofte sammen med lyshastighten $c$,
slik at en ny 'hendig' st\o rrelse er
              \begin{equation}
               \hbar c=197 \hspace{0.1cm}\mathrm{eVnm} (\hspace{0.1cm}\mathrm{MeVfm})
             \end{equation}
Til slutt kan faktoren i Coulombvekselvirkningen mellom
to f.eks.\ to elektroner
\begin{equation}
    \frac{e^2}{4\pi\epsilon_0 r}
\end{equation}
hvor $\epsilon_0$ er permittiviteten og $r$ er absolutt verdien av avstanden
mellom de to elektronene, settes lik
             \begin{equation}
               \frac{e^2}{4\pi\epsilon_0}=1.44 \hspace{0.1cm}\mathrm{eVnm}
             \end{equation}


Som et eksempel p\aa\ st\o rrelsesordener, la oss se p\aa\ forholdet
mellom gravitasjonskrefter og elektrostatiske krefter.
Gravitasjonskraften er gitt ved
\begin{equation}
    F_G=-\frac{Gm_1m_2}{r^2},
\end{equation}
hvor $G=6.67\times 10^{-11}$ m$^{3}$kg$^{-1}$s$^{-2}$ 
er gravitasjonskonstanten, 
$r$ er avstanden mellom legeme 1 og 2
og $m_1$ og $m_2$ deres respektive masser. La oss anta at $m_1$ er et elektron
og at $m_2$ er et proton. Coulombkraften er gitt ved
\begin{equation}
    F_C=-\frac{e^2}{4\pi\epsilon_0 r^2}.
\end{equation}
Setter vi inn for elektronets og protonets masse har vi
\begin{equation}
    F_G/F_C \approx 4 \times 10^{-40}.
\end{equation}
N\aa\ vet vi at bindingsenergien til elektronet i hydrogenatomet er
$-13.6$ eV og at den midlere avstanden mellom elektronet og protonet er
ca.~0.05 nm. Dersom det er gravitasjonskreftene som holder hydrogenatomet
sammen, hvor stor blir da den midlere avstanden?
Her kan du anta at bindingsenergien til elektronet er proporsjonalt
med 
\[
    -\frac{e^2}{4\pi\epsilon_0 r}.
\]
Sammenlikner radien du finner med den estimerte avstanden til den fjerneste
galakse, s\aa\ har vi ca.~$10^{25}$ m.

N\aa r vi f\o rst har tatt steget ut i universet,
la oss avslutte dette avsnittet med en ytterligere digresjon fra verdensrommet.
Solas masse er gitt ved 
\begin{equation}
M_{\odot}=1.989\times 10^{30} \hspace{0.1cm}\mathrm{kg}.
\end{equation}
og protonets masse er
\begin{equation}
  M_{p}=1.673\times 10^{-27} \hspace{0.1cm}\mathrm{kg}.
\end{equation}
Siden elektronet har en masse som er ca 2000 ganger mindre en protonet, kan vi anta at det er hovedsaklig protoner og n\o ytroner (som har nesten samme masse som
protonet) som bidrar til solas totale masse. 
Antallet protoner og n\o ytroner $N$ er da gitt ved
\begin{equation}
    N=  \frac{M_{\odot}}{m_p}\sim 10^{57}.
\end{equation}
Vi kan s\aa\ pr\o ve \aa\ gjenta eksemplet med hydrogenatomet. Vi antar
at sola best\aa r av en gass med tettpakkede protoner og n\o ytroner
og at volumet $V_p$ opptatt av et proton (n\o ytron) er
\begin{equation}
    V_p= \frac{4\pi r_p^3}{3},
\end{equation}
hvor $r_p$ er radien til protonet. Ovenfor oppga vi at $r_p\sim 1$ fm
eller $10^{-15}$ m.
Volumet til sola $V_S=4\pi R^3/3$, hvor $R$ er solas radius,  blir da
\begin{equation}
   V_S=NV_p = \frac{4\pi (10^{-15})^3 \hspace{0.1cm}\mathrm{m}^3}{3}\times 10^{57}.
\end{equation}
Dette gir oss en radius
\begin{equation}
   R\sim 10 \hspace{0.1cm}\mathrm{km}!!
\end{equation}
Solradien er gitt ved 
\begin{equation}
   R_{\odot}= 7\times 10^5 \hspace{0.1cm}\mathrm{km}.
\end{equation}
Dette forteller oss at den modellen vi antok for \aa\ beskrive den gjennomsnittlige tettheten i sola er feil. Gjennomsnittlig tetthet i sola er estimert til
ca.\ 1.4 g/cm$^3$. Derimot vil radien p\aa\ ca 10 km svare til radien
til ei n\o ytronstjerne, som er et mulig resultat av en supernova eksplosjon,
sluttstadiet for en stjerne som har brukt opp all brennstoffet sitt.
Massen til en n\o ytronstjerne er ca.\ 1.4 solmasser, s\aa\ dere kan tenke
dere sola konsentrert i et omr\aa de med Fysisk institutt som sentrum 
med radius 10 km. Det sier seg selv at tettheten m\aa\ v\ae re enorm.  
Tettheten  i ei n\o ytronstjerne varierer fra
$10^6$  g/cm$^3$ i de ytre lag til $10^{15}$  g/cm$^3$ i det indre av stjernen.

Til slutt, for \aa\ bringe oss over til neste tema, en st\o rrelse som er 
interesse er hvor mye energi sola str\aa ler ut per sekund.
Denne st\o rrelsen kalles luminositeten og er gitt
ved
\begin{equation}
   L_{\odot}=3.826\times 10^{26} \hspace{0.1cm}\mathrm{J/s},
\end{equation}
hvor J/s=W (watt). 
Ser vi p\aa\ utstr\aa lt energi per sekund, f\aa r vi radiansen,
som vi skal diskutere n\ae rmere i neste avsnitt.


\begin{table}[h]
\caption{Standard metrisk notasjon for tierpotenser}
\begin{tabular}{lll}\hline
{\bf Potens} & {\bf prefiks} & {\bf Symbol}\\ \hline
$10^{18}$ & exa & E\\ 
$10^{15}$ & peta & P\\ 
$10^{12}$ & tera & T\\ 
$10^{9}$ & giga & G\\ 
$10^{6}$ & mega & M\\ 
$10^{3}$ & kilo & k\\ 
$10^{-2}$ & centi & c\\ 
$10^{-3}$ & milli & m\\ 
$10^{-6}$ & micro & $\mu$\\ 
$10^{-9}$ & nano & n\\ 
$10^{-12}$ & pico & p\\ 
$10^{-15}$ & femto & f\\ 
$10^{-18}$ & atto & a\\ \hline
\end{tabular}
\end{table}

Vi kommer til \aa\ bruke det internasjonale enhets systemet, SI, 
hvor dynamiske variable uttrykkes i fem fundamentale enheter,
meter (m), kilogram (kg), sekund (s), ampere (A) og kelvin (K).
I kvantefysikk er det, som vist ovenfor, mer hensiktsmessig \aa\
bruke enheter som eV for energi.
Nyttige omregningsfaktorer er 1 eV=$1.60\times 10^{-19}$ J og
en atom\ae r masseenhet gitt ved 1 u$=1/12$ av massen til
$^{12}$C=$1.6604\times 10^{-27}$ kg $=931.48$ MeV/c$^2$.
Nedenfor finner dere flere konstanter som blir brukt i dette kurset.
\begin{table}[h]
\caption{Nyttige konstanter}
\begin{tabular}{lll}\hline
{\bf Konstant} & {\bf symbol} & {\bf verdi}\\ \hline 
 Lyshastighet&$c$ &$3.00\times 10^{8}$ m/s  \\ 
 Gravitasjonskonstant&$G$ &$6.67\times 10^{-11}$ Nm$^2$/kg$^2$ \\ 
 Coulombkonstant&$k$ &$8.99\times 10^{9}$ Nm$^2$/C$^2$ \\ 
 Boltzmannkonstant&$k_B$ &$1.38\times 10^{-23}$ J/K \\ 
 Element\ae rladning&$e$ &$1.60\times 10^{-19}$ C \\
 Plancks konstant&$h$ &$6.63\times 10^{-34}$ Js \\
                 &$hc$ &1240 eVnm \\
                 &$\hbar=h/2\pi$ &$1.055\times 10^{-34}$ Js \\
                 &$\hbar c$ &197 eVnm \\
  Bohrradius&$a_0=\hbar^2/m_eke^2$ &0.0529 nm  \\
Finstrukturkonstanten&$\alpha$ &1/137.036  \\
  Coulombfaktor&$ke^2$ &1.44 eVnm  \\
Elektronets gyromagnetisk faktor & $g_e$ & 2.002 \\ 
  Grunntilstand hydrogen&$E_0=-ke^2/2a_0$ &-13.606 eV  \\ 
Rydberg&Ry &13.606 eV  \\ \hline
\end{tabular}
\end{table}


\begin{table}[h]
\caption{Masser til viktige partikler}
\begin{tabular}{llll}\hline
{\bf Partikkel} & {\bf i kg } & {\bf i MeV/c$^2$} & {\bf i u}\\\hline 
elektron &$9.109\times 10^{-31}$ kg&0.511 MeV/c$^2$& 0.000549 u\\
proton &$1.672\times 10^{-27}$ kg&938.3 MeV/c$^2$& 1.007277 u\\
n\o ytron &$1.675\times 10^{-27}$ kg&939.6 MeV/c$^2$& 1.008665 u\\
hydrogen &$1.673\times 10^{-27}$ kg&938.8 MeV/c$^2$& 1.007825 u\\ \hline
\end{tabular}
\end{table}

Annen nyttig informasjon er b\o lgelengden til synlig lys
som g\aa r fra 700 nm (m\o rk r\o d ) til 400 nm (m\o rk fiolett).

I mange tekstb\o ker i fysikk brukes ogs\aa\ det Gaussiske enhetssystemet.
De viktigste enhetene der er gram, centimeter og sekund. Ladningsenheten,
som kalles statcoulomb, representerer en ladning, som i en avstand p\aa\
1 cm fra en identisk ladning f\o ler en kraft p\aa\ 1 dyne.
En statcoulomb er da gitt ved
\[
  1\hspace{0.1cm} \mathrm{statcoulomb}=1 \hspace{0.1cm} \mathrm{dyne}^{1/2}
\mathrm{cm}=1 \hspace{0.1cm}\mathrm{g}^{1/2}\mathrm{cm}^{3/2}\mathrm{s}^{-1}.
\]
Tabellen nedenfor gir faktoren som trengs n\aa r vi g\aa r fra SI systemet til
det Gaussiske systemet.
\begin{table}[h]
\caption{Fra SI systemet til Gaussiske enheter}
\begin{tabular}{ll}\hline
{\bf variabel} & {\bf transformasjon} \\ \hline
Elektrisk felt & ${\bf E}\rightarrow \frac{1}{\sqrt{4\pi\epsilon_0}}{\bf E}$ \\
Vektorpotensial & ${\bf A}\rightarrow\sqrt{\frac{\mu_0}{4\pi}}{\bf A}$ \\
B-felt & ${\bf B}\rightarrow\sqrt{\frac{\mu_0}{4\pi}}{\bf B} $\\
Magnetisk moment & ${\bf \mu}\rightarrow\sqrt{\frac{4\pi}{\mu_0}}{\bf \mu}$ \\
Skalart potensial & $\phi\rightarrow \frac{1}{\sqrt{4\pi\epsilon_0}}\phi$ \\\hline
\end{tabular}
\end{table}





\section{Plancks kvantiseringshypotese}


Et av problemene\footnote{Sidehenvisning i l\ae reboka er kap 2-1, sidene 74-80,
kap 2-2 er kun bakgrunnsmateriale, men vi kommer i kap 5 til \aa\ utlede
energien til et system som utviser st\aa ende b\o lger. Kap 2-3, side 86-93
undervises i FYS2160 og vil ikke bli vektlagt som pensum. Kap 2-4, side 93-99
gir den f\o rste virkelige anvendelse av kvantiseringshypotesen og det som
la grunnlaget for kvantemekanikken. Det viktige med disse avsnittene er at
dere har klart for dere hva som skapte bruddet med klassisk fysikk. 
En bedre forklaring vil dere f\aa\ i FYS2160 og eventuelt FYS3130.} en ikke var i stand til \aa\ forklare vha.~klassisk fysikk 
var frekvensfordelingen til elektromagnetisk (e.m.) str\aa ling fra et
legeme ved en gitt temperatur, f.eks.~sola eller ei kokeplate. N\aa r vi setter
p\aa\ ei kokeplate merker vi i begynnelsen ikke noen nevneverdig
fargeforandring, selv om vi registrerer  
at den blir litt varmere. Etter en stund
blir den r\o dgl\o dende og innbyr neppe til \aa\ bli tatt p\aa\ . Men 
f\o r vi definerer problemet noe n\ae rmere, la oss ta for oss noen 
definisjoner. 
    \begin{itemize}
       \item Termisk str\aa ling : den e.m.~str\aa ling som sendes ut fra
             et legeme som resultat av dets temperatur. 
             Alle legemer sender ut (emisjon) og mottar 
             (absorpsjon) e.m.~str\aa ling. 
       \item Ved gitt temperatur $T$ er vi interessert i \aa\ finne 
             fordelingen av emittert str\aa ling som funksjon 
             av den e.m.~str\aa lingen sin frekvens $\nu$ eller 
             b\o lgelengde $\lambda$. Vi har f\o lgende relasjon mellom
             frekvensen $\nu$ og b\o lgelengden $\lambda$ 
             \[
                \nu=\frac{c}{\lambda}
             \]
       \item Frekvensfordelingen 
             \[
                 M_{\nu}(T)d\nu
             \]
             kalles spektralfordelingen eller kanskje bedre fordelingsfunksjonen for frekvensspekteret, eller bare frekvensfordeling.
       Denne fordelingen leses ogs\aa\ som {\bf utstr\aa lt energi
             fra en gjenstand ved temperatur $T$ per areal per tid per frekvensenhet}. Figur \ref{fig:blackbody} viser eksempler p\aa\ frekvensfordelinger
for ulike temperaturer fra et s\aa kalt svart legeme. 
Denne figuren viser ogs\aa\ resultatet fra klassisk
teori. Vi ser at denne fordelingsfunksjonen viser en divergerende oppf\o rsel
(ultrafiolett katastrofe)
ved h\o ye frekvenser (eller sm\aa\ b\o lgelengder), i strid med eksperimentelle resultat.
       \item Integrerer vi over alle frekvenser 
             \[
                M(T)=\int_0^{\infty} M_{\nu}(T) d\nu
             \]
             f\aa r vi totalt utstr\aa lt energi per sekund per areal ved gitt
             temperatur $T$. Dimensjonen til $M(T)$ er da
             [$M(T)$]=J/(m$^2$s)=W/m$^2$. Denne st\o rrelsen kalles 
             radiansen.
       \item {\bf V\aa rt problem er \aa\ finne fram til en fysisk forklaring
             for den eksperimentelle formen til $M_{\nu}(T)$.       }
    \end{itemize}
\begin{figure}
% GNUPLOT: LaTeX picture with Postscript
\begingroup%
  \makeatletter%
  \newcommand{\GNUPLOTspecial}{%
    \@sanitize\catcode`\%=14\relax\special}%
  \setlength{\unitlength}{0.1bp}%
{\GNUPLOTspecial{!
%!PS-Adobe-2.0 EPSF-2.0
%%Title: planck.tex
%%Creator: gnuplot 3.7 patchlevel 1
%%CreationDate: Sun Jan 20 18:55:04 2002
%%DocumentFonts: 
%%BoundingBox: 0 0 360 216
%%Orientation: Landscape
%%EndComments
/gnudict 256 dict def
gnudict begin
/Color false def
/Solid false def
/gnulinewidth 5.000 def
/userlinewidth gnulinewidth def
/vshift -33 def
/dl {10 mul} def
/hpt_ 31.5 def
/vpt_ 31.5 def
/hpt hpt_ def
/vpt vpt_ def
/M {moveto} bind def
/L {lineto} bind def
/R {rmoveto} bind def
/V {rlineto} bind def
/vpt2 vpt 2 mul def
/hpt2 hpt 2 mul def
/Lshow { currentpoint stroke M
  0 vshift R show } def
/Rshow { currentpoint stroke M
  dup stringwidth pop neg vshift R show } def
/Cshow { currentpoint stroke M
  dup stringwidth pop -2 div vshift R show } def
/UP { dup vpt_ mul /vpt exch def hpt_ mul /hpt exch def
  /hpt2 hpt 2 mul def /vpt2 vpt 2 mul def } def
/DL { Color {setrgbcolor Solid {pop []} if 0 setdash }
 {pop pop pop Solid {pop []} if 0 setdash} ifelse } def
/BL { stroke userlinewidth 2 mul setlinewidth } def
/AL { stroke userlinewidth 2 div setlinewidth } def
/UL { dup gnulinewidth mul /userlinewidth exch def
      10 mul /udl exch def } def
/PL { stroke userlinewidth setlinewidth } def
/LTb { BL [] 0 0 0 DL } def
/LTa { AL [1 udl mul 2 udl mul] 0 setdash 0 0 0 setrgbcolor } def
/LT0 { PL [] 1 0 0 DL } def
/LT1 { PL [4 dl 2 dl] 0 1 0 DL } def
/LT2 { PL [2 dl 3 dl] 0 0 1 DL } def
/LT3 { PL [1 dl 1.5 dl] 1 0 1 DL } def
/LT4 { PL [5 dl 2 dl 1 dl 2 dl] 0 1 1 DL } def
/LT5 { PL [4 dl 3 dl 1 dl 3 dl] 1 1 0 DL } def
/LT6 { PL [2 dl 2 dl 2 dl 4 dl] 0 0 0 DL } def
/LT7 { PL [2 dl 2 dl 2 dl 2 dl 2 dl 4 dl] 1 0.3 0 DL } def
/LT8 { PL [2 dl 2 dl 2 dl 2 dl 2 dl 2 dl 2 dl 4 dl] 0.5 0.5 0.5 DL } def
/Pnt { stroke [] 0 setdash
   gsave 1 setlinecap M 0 0 V stroke grestore } def
/Dia { stroke [] 0 setdash 2 copy vpt add M
  hpt neg vpt neg V hpt vpt neg V
  hpt vpt V hpt neg vpt V closepath stroke
  Pnt } def
/Pls { stroke [] 0 setdash vpt sub M 0 vpt2 V
  currentpoint stroke M
  hpt neg vpt neg R hpt2 0 V stroke
  } def
/Box { stroke [] 0 setdash 2 copy exch hpt sub exch vpt add M
  0 vpt2 neg V hpt2 0 V 0 vpt2 V
  hpt2 neg 0 V closepath stroke
  Pnt } def
/Crs { stroke [] 0 setdash exch hpt sub exch vpt add M
  hpt2 vpt2 neg V currentpoint stroke M
  hpt2 neg 0 R hpt2 vpt2 V stroke } def
/TriU { stroke [] 0 setdash 2 copy vpt 1.12 mul add M
  hpt neg vpt -1.62 mul V
  hpt 2 mul 0 V
  hpt neg vpt 1.62 mul V closepath stroke
  Pnt  } def
/Star { 2 copy Pls Crs } def
/BoxF { stroke [] 0 setdash exch hpt sub exch vpt add M
  0 vpt2 neg V  hpt2 0 V  0 vpt2 V
  hpt2 neg 0 V  closepath fill } def
/TriUF { stroke [] 0 setdash vpt 1.12 mul add M
  hpt neg vpt -1.62 mul V
  hpt 2 mul 0 V
  hpt neg vpt 1.62 mul V closepath fill } def
/TriD { stroke [] 0 setdash 2 copy vpt 1.12 mul sub M
  hpt neg vpt 1.62 mul V
  hpt 2 mul 0 V
  hpt neg vpt -1.62 mul V closepath stroke
  Pnt  } def
/TriDF { stroke [] 0 setdash vpt 1.12 mul sub M
  hpt neg vpt 1.62 mul V
  hpt 2 mul 0 V
  hpt neg vpt -1.62 mul V closepath fill} def
/DiaF { stroke [] 0 setdash vpt add M
  hpt neg vpt neg V hpt vpt neg V
  hpt vpt V hpt neg vpt V closepath fill } def
/Pent { stroke [] 0 setdash 2 copy gsave
  translate 0 hpt M 4 {72 rotate 0 hpt L} repeat
  closepath stroke grestore Pnt } def
/PentF { stroke [] 0 setdash gsave
  translate 0 hpt M 4 {72 rotate 0 hpt L} repeat
  closepath fill grestore } def
/Circle { stroke [] 0 setdash 2 copy
  hpt 0 360 arc stroke Pnt } def
/CircleF { stroke [] 0 setdash hpt 0 360 arc fill } def
/C0 { BL [] 0 setdash 2 copy moveto vpt 90 450  arc } bind def
/C1 { BL [] 0 setdash 2 copy        moveto
       2 copy  vpt 0 90 arc closepath fill
               vpt 0 360 arc closepath } bind def
/C2 { BL [] 0 setdash 2 copy moveto
       2 copy  vpt 90 180 arc closepath fill
               vpt 0 360 arc closepath } bind def
/C3 { BL [] 0 setdash 2 copy moveto
       2 copy  vpt 0 180 arc closepath fill
               vpt 0 360 arc closepath } bind def
/C4 { BL [] 0 setdash 2 copy moveto
       2 copy  vpt 180 270 arc closepath fill
               vpt 0 360 arc closepath } bind def
/C5 { BL [] 0 setdash 2 copy moveto
       2 copy  vpt 0 90 arc
       2 copy moveto
       2 copy  vpt 180 270 arc closepath fill
               vpt 0 360 arc } bind def
/C6 { BL [] 0 setdash 2 copy moveto
      2 copy  vpt 90 270 arc closepath fill
              vpt 0 360 arc closepath } bind def
/C7 { BL [] 0 setdash 2 copy moveto
      2 copy  vpt 0 270 arc closepath fill
              vpt 0 360 arc closepath } bind def
/C8 { BL [] 0 setdash 2 copy moveto
      2 copy vpt 270 360 arc closepath fill
              vpt 0 360 arc closepath } bind def
/C9 { BL [] 0 setdash 2 copy moveto
      2 copy  vpt 270 450 arc closepath fill
              vpt 0 360 arc closepath } bind def
/C10 { BL [] 0 setdash 2 copy 2 copy moveto vpt 270 360 arc closepath fill
       2 copy moveto
       2 copy vpt 90 180 arc closepath fill
               vpt 0 360 arc closepath } bind def
/C11 { BL [] 0 setdash 2 copy moveto
       2 copy  vpt 0 180 arc closepath fill
       2 copy moveto
       2 copy  vpt 270 360 arc closepath fill
               vpt 0 360 arc closepath } bind def
/C12 { BL [] 0 setdash 2 copy moveto
       2 copy  vpt 180 360 arc closepath fill
               vpt 0 360 arc closepath } bind def
/C13 { BL [] 0 setdash  2 copy moveto
       2 copy  vpt 0 90 arc closepath fill
       2 copy moveto
       2 copy  vpt 180 360 arc closepath fill
               vpt 0 360 arc closepath } bind def
/C14 { BL [] 0 setdash 2 copy moveto
       2 copy  vpt 90 360 arc closepath fill
               vpt 0 360 arc } bind def
/C15 { BL [] 0 setdash 2 copy vpt 0 360 arc closepath fill
               vpt 0 360 arc closepath } bind def
/Rec   { newpath 4 2 roll moveto 1 index 0 rlineto 0 exch rlineto
       neg 0 rlineto closepath } bind def
/Square { dup Rec } bind def
/Bsquare { vpt sub exch vpt sub exch vpt2 Square } bind def
/S0 { BL [] 0 setdash 2 copy moveto 0 vpt rlineto BL Bsquare } bind def
/S1 { BL [] 0 setdash 2 copy vpt Square fill Bsquare } bind def
/S2 { BL [] 0 setdash 2 copy exch vpt sub exch vpt Square fill Bsquare } bind def
/S3 { BL [] 0 setdash 2 copy exch vpt sub exch vpt2 vpt Rec fill Bsquare } bind def
/S4 { BL [] 0 setdash 2 copy exch vpt sub exch vpt sub vpt Square fill Bsquare } bind def
/S5 { BL [] 0 setdash 2 copy 2 copy vpt Square fill
       exch vpt sub exch vpt sub vpt Square fill Bsquare } bind def
/S6 { BL [] 0 setdash 2 copy exch vpt sub exch vpt sub vpt vpt2 Rec fill Bsquare } bind def
/S7 { BL [] 0 setdash 2 copy exch vpt sub exch vpt sub vpt vpt2 Rec fill
       2 copy vpt Square fill
       Bsquare } bind def
/S8 { BL [] 0 setdash 2 copy vpt sub vpt Square fill Bsquare } bind def
/S9 { BL [] 0 setdash 2 copy vpt sub vpt vpt2 Rec fill Bsquare } bind def
/S10 { BL [] 0 setdash 2 copy vpt sub vpt Square fill 2 copy exch vpt sub exch vpt Square fill
       Bsquare } bind def
/S11 { BL [] 0 setdash 2 copy vpt sub vpt Square fill 2 copy exch vpt sub exch vpt2 vpt Rec fill
       Bsquare } bind def
/S12 { BL [] 0 setdash 2 copy exch vpt sub exch vpt sub vpt2 vpt Rec fill Bsquare } bind def
/S13 { BL [] 0 setdash 2 copy exch vpt sub exch vpt sub vpt2 vpt Rec fill
       2 copy vpt Square fill Bsquare } bind def
/S14 { BL [] 0 setdash 2 copy exch vpt sub exch vpt sub vpt2 vpt Rec fill
       2 copy exch vpt sub exch vpt Square fill Bsquare } bind def
/S15 { BL [] 0 setdash 2 copy Bsquare fill Bsquare } bind def
/D0 { gsave translate 45 rotate 0 0 S0 stroke grestore } bind def
/D1 { gsave translate 45 rotate 0 0 S1 stroke grestore } bind def
/D2 { gsave translate 45 rotate 0 0 S2 stroke grestore } bind def
/D3 { gsave translate 45 rotate 0 0 S3 stroke grestore } bind def
/D4 { gsave translate 45 rotate 0 0 S4 stroke grestore } bind def
/D5 { gsave translate 45 rotate 0 0 S5 stroke grestore } bind def
/D6 { gsave translate 45 rotate 0 0 S6 stroke grestore } bind def
/D7 { gsave translate 45 rotate 0 0 S7 stroke grestore } bind def
/D8 { gsave translate 45 rotate 0 0 S8 stroke grestore } bind def
/D9 { gsave translate 45 rotate 0 0 S9 stroke grestore } bind def
/D10 { gsave translate 45 rotate 0 0 S10 stroke grestore } bind def
/D11 { gsave translate 45 rotate 0 0 S11 stroke grestore } bind def
/D12 { gsave translate 45 rotate 0 0 S12 stroke grestore } bind def
/D13 { gsave translate 45 rotate 0 0 S13 stroke grestore } bind def
/D14 { gsave translate 45 rotate 0 0 S14 stroke grestore } bind def
/D15 { gsave translate 45 rotate 0 0 S15 stroke grestore } bind def
/DiaE { stroke [] 0 setdash vpt add M
  hpt neg vpt neg V hpt vpt neg V
  hpt vpt V hpt neg vpt V closepath stroke } def
/BoxE { stroke [] 0 setdash exch hpt sub exch vpt add M
  0 vpt2 neg V hpt2 0 V 0 vpt2 V
  hpt2 neg 0 V closepath stroke } def
/TriUE { stroke [] 0 setdash vpt 1.12 mul add M
  hpt neg vpt -1.62 mul V
  hpt 2 mul 0 V
  hpt neg vpt 1.62 mul V closepath stroke } def
/TriDE { stroke [] 0 setdash vpt 1.12 mul sub M
  hpt neg vpt 1.62 mul V
  hpt 2 mul 0 V
  hpt neg vpt -1.62 mul V closepath stroke } def
/PentE { stroke [] 0 setdash gsave
  translate 0 hpt M 4 {72 rotate 0 hpt L} repeat
  closepath stroke grestore } def
/CircE { stroke [] 0 setdash 
  hpt 0 360 arc stroke } def
/Opaque { gsave closepath 1 setgray fill grestore 0 setgray closepath } def
/DiaW { stroke [] 0 setdash vpt add M
  hpt neg vpt neg V hpt vpt neg V
  hpt vpt V hpt neg vpt V Opaque stroke } def
/BoxW { stroke [] 0 setdash exch hpt sub exch vpt add M
  0 vpt2 neg V hpt2 0 V 0 vpt2 V
  hpt2 neg 0 V Opaque stroke } def
/TriUW { stroke [] 0 setdash vpt 1.12 mul add M
  hpt neg vpt -1.62 mul V
  hpt 2 mul 0 V
  hpt neg vpt 1.62 mul V Opaque stroke } def
/TriDW { stroke [] 0 setdash vpt 1.12 mul sub M
  hpt neg vpt 1.62 mul V
  hpt 2 mul 0 V
  hpt neg vpt -1.62 mul V Opaque stroke } def
/PentW { stroke [] 0 setdash gsave
  translate 0 hpt M 4 {72 rotate 0 hpt L} repeat
  Opaque stroke grestore } def
/CircW { stroke [] 0 setdash 
  hpt 0 360 arc Opaque stroke } def
/BoxFill { gsave Rec 1 setgray fill grestore } def
end
%%EndProlog
}}%
\begin{picture}(3600,2160)(0,0)%
{\GNUPLOTspecial{"
gnudict begin
gsave
0 0 translate
0.100 0.100 scale
0 setgray
newpath
1.000 UL
LTb
600 300 M
63 0 V
2787 0 R
-63 0 V
600 740 M
63 0 V
2787 0 R
-63 0 V
600 1180 M
63 0 V
2787 0 R
-63 0 V
600 1620 M
63 0 V
2787 0 R
-63 0 V
600 2060 M
63 0 V
2787 0 R
-63 0 V
600 300 M
0 63 V
0 1697 R
0 -63 V
1170 300 M
0 63 V
0 1697 R
0 -63 V
1740 300 M
0 63 V
0 1697 R
0 -63 V
2310 300 M
0 63 V
0 1697 R
0 -63 V
2880 300 M
0 63 V
0 1697 R
0 -63 V
3450 300 M
0 63 V
0 1697 R
0 -63 V
1.000 UL
LTb
600 300 M
2850 0 V
0 1760 V
-2850 0 V
600 300 L
1.000 UL
LT0
3087 1947 M
263 0 V
629 305 M
29 15 V
28 24 V
29 30 V
29 37 V
29 41 V
29 46 V
28 49 V
29 51 V
29 53 V
29 54 V
28 54 V
29 53 V
29 53 V
29 52 V
29 50 V
28 48 V
29 46 V
29 43 V
29 41 V
29 38 V
28 35 V
29 32 V
29 29 V
29 26 V
28 23 V
29 19 V
29 17 V
29 14 V
29 10 V
28 8 V
29 5 V
29 3 V
29 0 V
29 -3 V
28 -4 V
29 -7 V
29 -8 V
29 -11 V
29 -12 V
28 -13 V
29 -16 V
29 -16 V
29 -17 V
28 -19 V
29 -19 V
29 -21 V
29 -21 V
29 -21 V
28 -22 V
29 -23 V
29 -22 V
29 -23 V
29 -23 V
28 -23 V
29 -24 V
29 -23 V
29 -23 V
28 -22 V
29 -23 V
29 -22 V
29 -22 V
29 -22 V
28 -21 V
29 -21 V
29 -20 V
29 -20 V
29 -20 V
28 -19 V
29 -18 V
29 -18 V
29 -18 V
29 -17 V
28 -16 V
29 -16 V
29 -16 V
29 -15 V
28 -14 V
29 -14 V
29 -13 V
29 -13 V
29 -13 V
28 -12 V
29 -12 V
29 -11 V
29 -10 V
29 -11 V
28 -9 V
29 -10 V
29 -9 V
29 -8 V
28 -9 V
29 -8 V
29 -7 V
29 -7 V
29 -7 V
28 -7 V
29 -6 V
29 -6 V
1.000 UL
LT1
3087 1847 M
263 0 V
629 304 M
29 13 V
28 18 V
29 25 V
29 28 V
29 32 V
29 35 V
28 36 V
29 37 V
29 37 V
29 38 V
28 36 V
29 35 V
29 34 V
29 32 V
29 30 V
28 28 V
29 26 V
29 22 V
29 21 V
29 17 V
28 16 V
29 12 V
29 10 V
29 8 V
28 5 V
29 3 V
29 1 V
29 -1 V
29 -3 V
28 -5 V
29 -6 V
29 -8 V
29 -9 V
29 -11 V
28 -11 V
29 -12 V
29 -14 V
29 -14 V
29 -14 V
28 -15 V
29 -16 V
29 -15 V
29 -16 V
28 -16 V
29 -16 V
29 -16 V
29 -16 V
29 -16 V
28 -16 V
29 -15 V
29 -15 V
29 -15 V
29 -15 V
28 -14 V
29 -14 V
29 -13 V
29 -13 V
28 -13 V
29 -12 V
29 -12 V
29 -11 V
29 -11 V
28 -11 V
29 -10 V
29 -9 V
29 -10 V
29 -8 V
28 -9 V
29 -8 V
29 -8 V
29 -7 V
29 -7 V
28 -6 V
29 -7 V
29 -6 V
29 -5 V
28 -6 V
29 -5 V
29 -5 V
29 -4 V
29 -4 V
28 -4 V
29 -4 V
29 -4 V
29 -3 V
29 -3 V
28 -3 V
29 -3 V
29 -3 V
29 -2 V
28 -3 V
29 -2 V
29 -2 V
29 -2 V
29 -2 V
28 -2 V
29 -1 V
29 -2 V
1.000 UL
LT2
3087 1747 M
263 0 V
629 303 M
29 10 V
28 14 V
29 18 V
29 21 V
29 22 V
29 24 V
28 24 V
29 24 V
29 23 V
29 22 V
28 20 V
29 20 V
29 17 V
29 15 V
29 13 V
28 11 V
29 9 V
29 7 V
29 5 V
29 4 V
28 1 V
29 0 V
29 -2 V
29 -3 V
28 -5 V
29 -5 V
29 -7 V
29 -7 V
29 -8 V
28 -9 V
29 -9 V
29 -10 V
29 -10 V
29 -10 V
28 -10 V
29 -10 V
29 -11 V
29 -10 V
29 -10 V
28 -10 V
29 -9 V
29 -10 V
29 -9 V
28 -9 V
29 -8 V
29 -8 V
29 -8 V
29 -8 V
28 -7 V
29 -7 V
29 -6 V
29 -6 V
29 -6 V
28 -6 V
29 -5 V
29 -5 V
29 -4 V
28 -5 V
29 -4 V
29 -3 V
29 -4 V
29 -3 V
28 -3 V
29 -3 V
29 -3 V
29 -2 V
29 -3 V
28 -2 V
29 -2 V
29 -2 V
29 -2 V
29 -1 V
28 -2 V
29 -1 V
29 -1 V
29 -1 V
28 -1 V
29 -1 V
29 -1 V
29 -1 V
29 -1 V
28 -1 V
29 0 V
29 -1 V
29 0 V
29 -1 V
28 0 V
29 -1 V
29 0 V
29 0 V
28 -1 V
29 0 V
29 0 V
29 -1 V
29 0 V
28 0 V
29 0 V
29 0 V
1.000 UL
LT3
3087 1647 M
263 0 V
600 300 M
29 8 V
29 25 V
28 41 V
29 58 V
29 74 V
29 91 V
29 108 V
28 123 V
29 141 V
29 157 V
29 173 V
28 190 V
29 206 V
29 223 V
17 142 V
stroke
grestore
end
showpage
}}%
\put(3037,1647){\makebox(0,0)[r]{Klassisk $k_BT=0.6$ eV}}%
\put(3037,1747){\makebox(0,0)[r]{$k_BT=0.4$ eV}}%
\put(3037,1847){\makebox(0,0)[r]{$k_BT=0.5$ eV}}%
\put(3037,1947){\makebox(0,0)[r]{$k_BT=0.6$ eV}}%
\put(2025,50){\makebox(0,0){Energi $h\nu$ [eV]}}%
\put(100,1180){%
\special{ps: gsave currentpoint currentpoint translate
270 rotate neg exch neg exch translate}%
\makebox(0,0)[b]{\shortstack{$M_{\nu}(h\nu)$ [eV/nm$^2$]}}%
\special{ps: currentpoint grestore moveto}%
}%
\put(3450,200){\makebox(0,0){5}}%
\put(2880,200){\makebox(0,0){4}}%
\put(2310,200){\makebox(0,0){3}}%
\put(1740,200){\makebox(0,0){2}}%
\put(1170,200){\makebox(0,0){1}}%
\put(600,200){\makebox(0,0){0}}%
\put(550,2060){\makebox(0,0)[r]{2e-06}}%
\put(550,1620){\makebox(0,0)[r]{1.5e-06}}%
\put(550,1180){\makebox(0,0)[r]{1e-06}}%
\put(550,740){\makebox(0,0)[r]{5e-07}}%
\put(550,300){\makebox(0,0)[r]{0}}%
\end{picture}%
\endgroup
\endinput

\caption{Figuren viser frekvensfordelingen fra Plancks kvantiseringspostulat
i likning (\ref{eq:plankcfirst}) og den klassiske fordelingsfunksjonen fra likning 
(\ref{eq:klassisk}). Legg merke til at energi er i enhet eV og frekvensfordelingen har enheten eV/nm$^2$. \label{fig:blackbody}. }
\end{figure}

Det klassiske eksempel p\aa\ en slik 
frekvensfordeling  $M_{\nu}(T)$ var gitt ved str\aa ling fra et 
s\aa kalt svart legeme. Et svart legeme er et idealisert objekt som 
ikke reflekterer noe av den innkommende  e.m.~str\aa ling. All innkommende
e.m.~str\aa ling blir absorbert. Grunnen til at det kalles svart legeme
var at ved lave temperaturer (tenk igjen p\aa\ ei kokeplate som nettopp
er satt p\aa\  ) s\aa\ forble legemet m\o rkt, selv om det sendte ut termisk
str\aa ling. Den var bare ikke synlig for oss.
Frekvensfordelingen til et svart legeme  er uavhengig av materiale, slik at
dets frekvensfordeling er en universell funksjon av frekvens $\nu$ og
temperatur $T$. P\aa\ slutten av 1800-tallet hadde en gjennomf\o rt
flere eksperiment ved \aa\ observere utstr\aa lt e.m.~energi fra modeller
som skulle representere et s\aa kalt svart legeme.
Modellen var et hulrom som ble varmet opp til en bestemt temperatur.
Atomene
i materialet til dette hulrommet ble da satt i svingninger og
sendte ut e.m.~str\aa ling (mere om dette i FYS2160). En kan tenke
seg atomene som harmoniske oscillatorer som vibrerer og sender ut str\aa ling.
Ved termisk likevekt var hulrommet fylt av e.m.~str\aa ling.
{\bf Teknisk sett vil dette svare til st\aa ende e.m.~b\o lger} og en kan da
regne ut energien til det e.m.~feltet i et slikt hulrom.  
Hulrommet hadde et hull, hvis st\o rrelse var mye mindre en hulrommets
overflate. E.m.~str\aa ling ble emittert fra dette hullet som ideelt sett
skal representere et svart legeme. 
Fordelen med dette oppsettet var at det lot seg b\aa de gjennomf\o re
eksperimentelt og at en kunne regne ut teoretisk frekvensfordelingen.

En kan da vise (se kap 2-1 og sidene 83-85 i l\ae reboka) at 
\begin{equation}
   M_{\nu}(T)=\frac{2\pi\nu^2}{c^2}\left\langle E\right\rangle ,
\end{equation}
hvor $\left\langle E\right\rangle $ er den gjennomsnittlige energien per svingemode til det
eletromagnetiske feltet i hulrommet. Dette feltet skal igjen gjenspeile 
svingingene til atomene i materialet til hulrommet.
Den tilsvarende radiansen var da gitt ved
\begin{equation}
   M(T)=\sigma T^4,
\end{equation}
hvor $\sigma$ er en konstant. Dette uttrykket kalles Stefan-Boltzmanns
lov. Det var Stefan som i 1879 foreslo basert p\aa\ data at
radiansen for et svart legeme skulle v\ae re proporsjonal med
$T^4$.  
Klassisk fysikk, se nedenfor, ga at 
\begin{equation}
   \left\langle E\right\rangle =\frac{3k_BT}{2},
\end{equation}
hvor $k_B$ er Boltzmanns konstant.
Dvs. 
\begin{equation}
   M_{\nu}(T)=\frac{\pi\nu^2}{c^2}3k_BT.
   \label{eq:klassisk}
\end{equation}
Dersom vi integrerer det siste uttrykket for \aa\ finne radiansen
\begin{equation}
   M(T)=\int_0^{\infty} \frac{\pi\nu^2}{c^2}3k_BT d\nu,
   \label{eq:divrad}
\end{equation}
ser vi at radiansen divergerer, i strid med den empiriske
oppf\o rselen i Stefan-Boltzmanns lov. 

Plancks hypotese (se nedenfor) ga
\begin{equation}
   \left\langle E\right\rangle =\frac{h\nu}{e^{h\nu/k_BT}-1},
\end{equation}
og dermed 
\begin{equation}
   M_{\nu}(T)=\frac{2\pi\nu^2}{c^2}\frac{h\nu}{e^{h\nu/k_BT}-1},
   \label{eq:plankcfirst}
\end{equation}
og samsvar med eksperiment. $h$ er Plancks konstant.
Vi ser at denne funksjonen har det riktige
forl\o p b\aa de ved sm\aa\  og store verdier av $\nu$, se
figur \ref{fig:blackbody}.
I denne figuren har vi valgt enheter eV og nm. Grunnen er
at dersom vi \o nsker \aa\ sette naturkonstantene $k_B$, $h$ og $c$
i enheter av henholdsvis J/K, Js og m/s$^2$, 
kan det lett lede til tap av presisjon i numeriske beregninger   
av frekvensfordelingen. For \aa\ konvertere til disse enhetene har vi 
multiplisert siste uttrykk med $h^2$ i teller og nevner
\begin{equation}
   M_{\nu}(T)=\frac{2\pi}{h^2c^2}\frac{(h\nu)^3}{e^{h\nu/k_BT}-1},
\end{equation}
og ved \aa\ sette $x=h\nu$ finner vi
\begin{equation}
   M_{x}(T)=\frac{2\pi}{h^2c^2}\frac{x^3}{e^{x/k_BT}-1},
\end{equation}
som gir med $hc=1240$ eVnm 
\begin{equation}
   M_{x}(T)=\frac{2\pi}{(1240)^2}\frac{x^3}{e^{x/k_BT}-1}.
\end{equation}
Temperaturen er ogs\aa\ uttrykt i eV. En temperatur p\aa\ 
1 eV (husk at $1 \mathrm{eV} = 1.6\times 10^{-19}$ J )svarer derfor til 
\[
   T=1.60\times 10^{-19}/(1.38\times 10^{-23})= 11594 \hspace{0.1cm}\mathrm{K}.
\]
Vi kan utlede 
Stefan-Boltzmanns lov 
vha.\ Plancks
kvantiseringshypotese. Vi trenger da
\begin{equation}
   M(T)=\int_0^{\infty}M_{\nu}(T)d\nu=\int_0^{\infty}\frac{2\pi\nu^2}{c^2}\frac{h\nu}{e^{h\nu/k_BT}-1} d\nu,
\end{equation}
og med variabel bytte $x=h\nu/k_BT$ f\aa r vi
\begin{equation}
M(T)=\frac{2\pi k_B^4}{c^2h^3}T^4\int_0^{\infty}\frac{x^3}{e^{x}-1} dx,
\end{equation}
og med 
\begin{equation}
   \int_0^{\infty}\frac{x^3}{e^{x}-1} dx=\frac{\pi^4}{15},
\end{equation}
f\aa r vi 
\begin{equation}
M(T)=\sigma T^4,
\label{eq:korrekt_sb}
\end{equation}
med
\begin{equation}
\sigma=\frac{2\pi^5k_B^4}{15c^2h^3}=5.676\times 10^{-8}\hspace{0.1cm}\mathrm{W/m^2K^4},
\end{equation}
i godt samsvar med verdier fra empiriske  data.



{\bf Det viktige budskapet er at Planck forlot den klassiske
m\aa ten \aa\ regne ut den midlere energien $\left\langle E\right\rangle$. 
Istedet for \aa\
tillatte alle mulige verdier av energien, krevde han at kun bestemte
diskrete verdier kunne tas med i utregningen 
av $\left\langle E\right\rangle$ .} 

Dette leder oss til en digresjon om fordelingsfunksjoner, litt
statistikk og hvorfor vi ofte betrakter kun midlere verdier av st\o rrelser
i fysikk. Mye av dette vil dere f\aa\ i st\o rre detalj i FYS2160.

Det er dog en ting som er viktig \aa\ ha klart for seg. 
Planck hadde dataene foran seg! og visste dermed hva svaret
skulle v\ae re og pr\o vde \aa\ tilpasse dataene med ulike
funksjoner. Det er p\aa\ dette viset vi ofte g\aa r fram i fysikk.
I mange tilfeller har vi data fra eksperiment som vi ikke kan forklare
med gjeldende teorier, andre ganger har vi teoretiske prediksjoner
p\aa\ fenomen som ikke er m\aa lt/observert. Fysikk representerer 
syntesen 
av eksperiment og teori med den m\aa lsetting \aa\ avdekke
bevegelseslovene til naturen. 

\subsection{Maxwells hastighets og energifordeling, klassisk}

Materialet her er bakgrunnsmateriale, og gjennomg\aa s i dybde i FYS2160.
Men vi trenger noen begrep for \aa\ forst\aa\ hvordan en kan regne
ut midlere energi $\left\langle E\right\rangle $. I tillegg, vil begrepet om sannsynlighetsfordeling
og normering v\ae re sentrale i v\aa r diskusjon av kvantemekanikken.
Det er en annen viktig grunn til at vi ofte bruker midlere verdier. Det skyldes
at fysiske systemer involver s\aa\ mange frihetsgrader, bare tenk p\aa\
Avogadros tall!, at vi ikke vil v\ae re i stand til \aa\ regne
med alle. En fullstendig mekanisk beskrivelse av et makroskopisk system som
en gass av hydrogenatomer er uoverkommelig, og heller ikke \o nskelig.
Nedskrevet p\aa\ papir m\aa tte det atskillige billass med begynnelsesbetingelser
til for \aa\ beskrive bare et gram av en slik hydrogen gass. 
Vi er egentlig bare interessert i visse midlere egenskaper ved atomenes
dynamikk. 
Derfor betrakter vi som regel et system ved likevekt og ser p\aa\
midlere st\o rrelser, slik som midlere hastighet, energi osv.
Makroskopiske systemer i likevekt beskrives f.eks.~med termodynamiske
st\o rrelser som trykk $P$ og temperatur $T$.  
For \aa\ regne ut slike st\o rrelser trenger vi en del begrep fra
sannsynlighetsl\ae re og statistikk.

 
Hvis vi tenker p\aa\ statistikk og sannsynlighetsregning,
s\aa\ innf\o rer vi begrepet sannsynlighet for at ei hending kan skje.
F.eks., sannsynligheten $P$ for \aa\ f\aa\ 1,2,3,4,5 eller 
6 ved et terningkast
er gitt ved $P=1/6$. Kaller vi et slikt utfall av terningkast for ei hending,
gir summen over alle hendinger
\begin{equation}
   \sum_{i=1}^{6}P_i=\sum_{i=1}^{6}\frac{1}{6}=1,
\end{equation}  
summen av alle sannsynligheter er lik 1. Vi sier da ogs\aa\ at
summen er normalisert. Hvis vi \o nsker \aa\ finne
gjennomsnittsverdien til en st\o rrelse $x$ er den definert ved
\begin{equation}
   \left\langle x\right\rangle =\frac{\sum_i xP_i}{\sum_i P_i},
\end{equation}
hvor leddet i nevneren s\o rger for at fordelingen over sannsynligheter
er normert. Sannsynlighetene for terningkastene tar kun 
diskrete verdier. Vi kaller derfor fordelingen ovenfor for 
ei diskret fordeling.
Men vi kan ogs\aa\ ha kontinuerlige sannsynlighetsfordelinger.
Et viktig eksempel i fysikk er Maxwells hastighetsfordeling\footnote{Denne
fordelingen utledes ikke her, og er ikke del av pensum}.
Den er gitt ved  
             \begin{equation}
                \frac{df}{dv}=4\pi\left(\frac{m}{2\pi k_BT}\right)^{3/2}
                              v^2e^{-mv^2/2k_BT}
                \label{eq:maxwell}
             \end{equation}
med 
             \begin{equation}
                \int_0^{\infty}\frac{df}{dv}dv=1
             \end{equation}
             dvs at den er normalisert. 
Tolkningen av $df/dvdv$ er at den gir br\o kdelen av partikler som
har en hastighet mellom $v$ og $v+dv$ ved en gitt temperatur $T$.

Dersom vi \o nsker \aa\ regne ut en midlere hastighet trenger vi \aa\ regne
ut 
\begin{equation}
  \left\langle v\right\rangle = \frac{\int_0^{\infty} v df/dv dv}{\int_0^{\infty}df/dv dv}.
\end{equation}
For Maxwells fordeling i likning (\ref{eq:maxwell}) er nevneren
allerede normalisert og lik 1. Midlere hastighet blir
\begin{equation}
\left\langle v\right\rangle =\int_0^{\infty}v4\pi\left(\frac{m}{2\pi k_BT}\right)^{3/2}
                              v^2e^{-mv^2/2k_BT}dv.
\end{equation}
Vi trenger alts\aa\ \aa\ regne ut integral av typen
\begin{equation}
   \int_0^{\infty}v^ne^{-\beta v^2}dv.
\end{equation}
Disse kan finnes slik: for $n$ ulike, velg ny variabel $t=\beta v^2$
og du f\aa r et integral av typen
\begin{equation}
   \int_0^{\infty}(t/\beta)^{(n-1)/2}e^{-t}dt,
\end{equation}
hvor $n-1$ er et like tall. For like $n$ trenger vi kun \aa\ 
foreta $n/2$ differensiasjoner m.h.p.\ $\beta$ av 
integralet $\int_0^{\infty}e^{-\beta v^2}dv=1/2\sqrt{\pi/\beta}$.
\O nsker vi midlere hastighet $\left\langle v\right\rangle $ finner vi da
\begin{equation}
\left\langle v\right\rangle =\sqrt{\frac{8k_BT}{\pi m}},
\end{equation}
eller dersom vi \o nsker $\left\langle v^2\right\rangle $ finner vi 
\begin{equation}
  \left\langle v^2\right\rangle =\frac{3k_BT}{m},
\end{equation}
noe som igjen gir en midlere energi p\aa\
\begin{equation}
   \left\langle E\right\rangle =\frac{1}{2}m\left\langle v^2\right\rangle =\frac{3k_BT}{2}.
\end{equation}
Det var dette resultatet som ga det klassiske resultat for 
frekvensfordelingen
av e.m.~str\aa ling.
Vi kan ogs\aa\ regne ut det siste uttrykk vha.\ energifordelingen
\begin{equation}
    \frac{df}{dE}.
\end{equation}
Bruker vi $E=1/2mv^2$ har vi $dE=mvdv$ noe som igjen gir
\begin{equation}
\frac{df}{dE}=\frac{dv}{dE}\frac{df}{dv}=\frac{4\pi}{mv}
\left(\frac{m}{2\pi k_BT}\right)^{3/2}
                              v^2e^{-mv^2/2k_BT},
\end{equation}
og setter vi inn at $v^2=2E/m$ f\aa r vi 
 energifordelingen
             \begin{equation}
                \frac{df}{dE}=2\pi\left(\frac{1}{\pi k_BT}\right)^{3/2}
                              \sqrt{E}e^{-E/k_BT},
             \end{equation}
             med
             \begin{equation}
                \int_0^{\infty}\frac{df}{dE}dE=1,
             \end{equation}
dvs.\ at den er normalisert.
\O nsker vi s\aa\ \aa\ finne midlere energi har vi
 \begin{equation}
       \left\langle E\right\rangle =\int_0^{\infty}E\frac{df}{dE}dE=\frac{3k_BT}{2},
 \end{equation}
som er det klassiske resultat.
Innsatt i uttrykket for frekvensfordelingen $M_{\nu}(T)$ i
likning (\ref{eq:klassisk}) f\aa r vi uttrykket til Rayleigh og Jeans.
Vi fant ogs\aa\ at radiansen divergerte, se igjen likning (\ref{eq:divrad}).

Vi kan oppsumere dette avsnittet med f\o lgende. 
Vi kan skrive energifordelingsfunksjonen som
\begin{equation}
                \frac{df}{dE}=g(E)e^{-E/k_BT},
\end{equation}
hvor $g(E)$ er en funksjon som kalles tettheten av tilstander med en gitt
energi $E$, men eksponensial faktoren $e^{-E/k_BT}$ kalles
Maxwell-Boltzmann faktoren og uttrykker sannsynligheten for \aa\ finne
systemetet i en tilstand med gitt energi $E$.

{\bf Og n\aa\ kommer det som er viktig:}\newline
Klassisk s\aa\ tillater vi alle verdier av $E$ i beregningen
av forventningsverdien $\left\langle E\right\rangle $, dvs vi har et integral
\begin{equation}
   \left\langle E\right\rangle =\frac{\int_0^{\infty}Eg(E)e^{-E/k_BT}dE}
{\int_0^{\infty}g(E)e^{-E/k_BT}dE},
\label{eq:eexpt}
\end{equation}
hvor alle $E$ er tillatt. Det resulterte i den s\aa kalte ultrafiolette
katastrofen.
      


\subsection{Plancks hypotese}
Plancks radikale hypotese var \aa\ foresl\aa\ at kun bestemte
energier i beregningene av $\left\langle E\right\rangle $ er tillatt, dvs. 
             \begin{equation}
               E_n(\nu)=n h \nu, \hspace{1cm} n=0,1,2,\dots
             \end{equation}
             hvor $\nu$ er frekvensen til den e.m.~str\aa lingen 
             og $h$ er en universell
             konstant var tillatt . 
Vi sier da at energien er kvantisert! De tillate energitilstandene
             kalles {\bf kvantetilstander} og heltallet $n$ kalles 
             et {\bf kvantetall}.
Det betyr at forventningsverdien i likning (\ref{eq:eexpt}) blir
\begin{equation}
   \left\langle E\right\rangle =\frac{\sum_{n=0}^{\infty}E_ng(E_n)e^{-E_n/k_BT}}
{\sum_{n=0}^{\infty}g(E_n)e^{-E_n/k_BT}}.
\end{equation}
Setter vi inn $E_n=nh\nu$ og tar bort leddet $g(E_n)$ f\aa r vi
\begin{equation}
   \left\langle E\right\rangle =\frac{\sum_{n=0}^{\infty}nh\nu e^{-nh\nu/k_BT}}
{\sum_{n=0}^{\infty}e^{-nh\nu/k_BT}}.
\end{equation}
Setter vi s\aa\ $x=h\nu/k_BT$ f\aa r vi
\begin{equation}
   \left\langle E\right\rangle =k_BT\frac{\sum_{n=0}^{\infty}nxe^{-nx}}
{\sum_{n=0}^{\infty}e^{-nx}}.
\end{equation}
Summen 
\begin{equation}
   \sum_{n=0}^{\infty}e^{-nx}=\frac{1}{1-e^{-x}},
\end{equation}
og bruker vi at
\begin{equation}
    \sum_{n=0}^{\infty}nxe^{-nx}=-x\frac{d}{dx}\sum_{n=0}^{\infty}e^{-nx},
\end{equation}
f\aa r vi at
\begin{equation}
  \left\langle E\right\rangle =k_BT x\frac{e^{-x}}{1-e^{-x}}=\frac{h\nu}{e^{h\nu/k_BT}-1},
\end{equation}
slik at frekvensfordelingen blir 
\[
   M_{\nu}(T)d\nu=\frac{2\pi\nu^2}{c^2}\frac{h\nu}{e^{h\nu/k_BT}-1}d\nu,
\]
eller dersom vi \o nsker uttrykket gitt ved b\o lgelengden $\lambda$
(vis dette)
har vi
\begin{equation} 
   M_{\lambda}(T)d\lambda=\frac{2\pi hc^2}{\lambda^5}\frac{1}{e^{hc/\lambda k_BT}-1}d\lambda.
\end{equation}
Som vist i likning (\ref{eq:korrekt_sb}), ga dette oss ogs\aa\ et korrekt
uttrykk for Stefan-Boltzmanns lov. 

Det spesielle med e.m.~str\aa ling i et hulrom er at det gir opphav
til st\aa ende e.m.~b\o lger som utviser enkle harmoniske
svingninger. Det s\ae regne  med denne type problem er at energien
antar diskrete verdier\footnote{I tilknyting Schr\"odingers likning,
skal vi vise dette for ei fj\ae r som svinger og er fastspent i begge
ender.}. 

\begin{center}
\shabox{\parbox{14cm}{Plancks hypotese (1900) kan dermed formuleres som f\o lger: Enhver 
             fysisk st\o rrelse som utviser enkle harmoniske
             svingninger har energier som tilfredsstiller 
             \[
               E_n(\nu)=n h \nu, \hspace{1cm} n=0,1,2,\dots
             \]
             hvor $\nu$ er frekvensen til svingningen og $h$ er en universell
             konstant. 
}}\end{center}
En idealisert pendel utviser ogs\aa\ enkle harmoniske svningninger,
og vi kan jo da stille sp\o rsm\aa let om hvorfor kan vi beskrive
en pendel vha.~klassisk fysikk og ikke e.m.~str\aa ling ?

Dette sp\o rsm\aa let belyser et viktig aspekt ved v\aa r forst\aa else av
fysikk og framgangsm\aa ter for \aa\ studere fysiske systemer.
{\em Det dreier seg om energiskalaer og st\o rrelsen p\aa\ systemet.}
Gang p\aa\ gang vil vi komme over eksempler p\aa\ det i kurset.
Dette dikterer igjen hvilken fysisk teori som er anvendbar.

La oss bruke pendelen til \aa\ se n\ae rmere p\aa\ dette.
Anta at vi har en idealisert pendel, vi ser bort fra luftmostand osv.
Vi gir pendelen en masse 
$m=0.01$ 
kg, en lengde p\aa\ $l=0.1$ m og vi tillater
at den kan svinge ut en vinkel p\aa\ maks $\theta=0.1$ rad.

Vi sp\o r deretter om hvor stor energiforskjellen er mellom
kvantetilstander
m\aa lt i forhold til maks potensiell energi pendelen kan ha n\aa r vi anvender
Plancks hypotese.
Maks potensiell energi $E$ er gitt ved 
\begin{equation}
  E=mgs=mgl(1-cos\theta)=5\times 10^{-5}\hspace{0.1cm} \mathrm{J},
\end{equation}
hvor $s$ er maksimal h\o yde som pendelen kan oppn\aa\ ved utsving
og $g$ er tyngdeakselerasjonen.
Svingefrekvensen $\nu$ finner vi ogs\aa\ ved \aa\ anvende en velkjent traver
\begin{equation}
   \nu=\frac{1}{2\pi}\sqrt{\frac{g}{l}}=1.6\hspace{0.1cm} \mathrm{s}^{-1}.
\end{equation}

N\aa\ skal vi anvende Plancks hypotese for \aa\ regne energiforskjellen
$\Delta E$
\begin{equation}
   \Delta E=(n+1)h\nu - nh\nu=h\nu\sim 10^{-33}\hspace{0.1cm} \mathrm{J}.
\end{equation}
Forholdet 
\begin{equation}
   \frac{\Delta E}{E}\sim 10^{-29}!
\end{equation}
viser at
vi kunne praktisk talt satt denne energiforskjellen lik null.
Det finnes ikke noe m\aa leinstrument som kan m\aa le en slik
energiforskjell.
I klassisk fysikk kan vi derfor sette $h=0$.

Problemet dukker opp n\aa r 
$\Delta E/E$ ikke er neglisjerbar. For h\o gfrekvent e.m.~str\aa ling
er dette tilfelle, og da kunnne ikke klassisk fysikk lenger forklare
fenomenene. Dersom $E$ er s\aa\ liten at 
$\Delta E=h\nu$ er p\aa\ samme st\o rrelse, da er vi p\aa\ energiskalaer
som ikke lenger kunne og kan forklares uten at ny teori anvendes.

Vi skal se p\aa\ mange flere slike eksempler i dette kurset. 


\section{Fotoelektrisk effekt}


Den fotoelektriske effekt\footnote{Svarer til kap 2-5 i boka, sidene 99-103. I diskusjonen her henviser jeg til figurer i l\ae reboka.} ble bla.~studert  av Hertz i 1886 og 1887, og ble da brukt
som en bekreftelse p\aa\ eksistensen av e.m.~b\o lger og av Maxwells e.m.~teori
for lysforplanting. Figur \ref{fig:fotoelektrisk} 
viser et oppsett for m\aa ling av fotoelektrisk
effekt.
\begin{figure}[h]
\begin{center}
{\centering
\mbox
{\psfig{figure=fotoelectric.ps,height=6cm,width=8cm}}
}
\end{center}
I tillegg til dette fant en ogs\aa\ at den kinetiske energien $K_{max}$
til de utsendte elektronene var uavhengig av intensiteten til str\aa lingen
og at det var en nedre frekvens den e.m.~str\aa lingen kunne ha 
for utsending av elektroner. I tillegg, utviste 
det p\aa satte 
potensialet $V_S$ ogs\aa\ en minste verdi for utsending av elektroner.
Figur \ref{fig:fotoelektrisk2} viser skjematisk  den resulterende 
fotostr\o mmen som funksjon av p\aa satt spenning og kinetisk energi
som funksjon av frekvensen til den innkommende e.m.~str\aa ling. 
\caption{Skjematisk oppsett for fotoelektrisk effekt. \label{fig:fotoelektrisk}}
\end{figure}
\begin{figure}[h]
\begin{center}
{\centering
\mbox
{\psfig{figure=exptfoto.ps,height=6cm,width=12cm}}
}
\end{center}
\caption{Fotostr\o m som funksjon av p\aa satt spenning og kinetisk energi
som funksjon av frekvensen til den innkommende e.m.~str\aa ling. \label{fig:fotoelektrisk2}}
\end{figure}

Oppsumert var   det
tre viktige egenskaper ved fotoelektrisk effekt som ikke
kunne forklares vha.~e.m.~b\o lgeteori for lys:
\begin{enumerate}
       \item  N\aa r intensiteten til lysstr\aa len \o kes, skal
ogs\aa\ amplituden til den oscillerende elektriske 
vektoren ${\bf E}$ \o ke. Siden kraften feltet ut\o ver p\aa\
et elektron er $e{\bf E}$, burde ogs\aa\ den kinetiske
energien til elektronene \o ke. Men eksperiment viste at
$K_{max}=eV_S$ var uavhengig av intensiteten. Dette har blitt utf\o rlig 
uttestet
for et  intesitetssprang p\aa\ $10^{7}$.  
\item I henhold til klassisk e.m.~teori, skal den fotoelektriske
effekt forekomme for enhver frekvens, gitt at lyset er intenst
nok til \aa\ gi den n\o dvendige energien til elektronene.
Dette var ikke tilfelle, jfr.~oppdagelsen av  
en nedre frekvens $\nu_0$. For lavere
frekvenser forekommer ikke fotoelektrisk effekt, et resultat som er uavhengig
av intensiteten til lyset.  
\item I henhold til klassisk teori, skulle det, n\aa r lys
      faller inn p\aa\ et materiale, ta litt tid fra n\aa r elektronene
      begynner \aa\ absorbere e.m.~str\aa ling til de slipper fri
      fra materialet. {\em Heller ikke noen slik tidsforskjell er detektert}. 
\end{enumerate}
La oss se p\aa\ det siste f\o rst. Anta at vi har ei plate av kalium
som er 1 m fra en lyskilde som sender ut e.m.~str\aa ling med
effekt 1 W$=$1 J/s. Vi antar deretter at et elektron i denne kaliumplaten
opptar et sirkelformet omr\aa de med radius $10^{-10}$ m.
Energien som trengs for \aa\ l\o srive det svakest bundne elektron
i kalium er 2.1 eV. Sp\o rsm\aa let vi stiller oss da er hvor lang tid
tar det f\o r dette ene elektronet oppn\aa r en energi p\aa\
2.1 eV n\aa r vi sender e.m.~str\aa ling fra lyskilden.

Tenker vi oss at lyskilden sender ut e.m.~str\aa ling som sf\ae riske
b\o lger, vet vi at overflaten til denne lyskjeglen ved 1 m fra lyskilden 
er gitt ved $A=4\pi 1^2$ m$^2$. Arealet elektronet opptar er 
$A_e=\pi r^2=\pi 10^{-20}$ m$^2$. Total energi $R$ som treffer $A_e$ per sekund
er da
\begin{equation}
   R= 1\hspace{0.1cm}\mathrm{W}\frac{\pi 10^{-20}}{4\pi}=0.015\hspace{0.1cm}\mathrm{eV/s},
\end{equation}
som igjen betyr at vi trenger
\begin{equation}
   t=\frac{2.1}{0.015}\hspace{0.1cm}\mathrm{s}\sim 135\hspace{0.1cm}\mathrm{s},
\end{equation}
for at dette elektronet skulle f\aa\ nok energi til \aa\ kunne l\o srives
fra metallplaten. En slik tidsforsinkelse er aldri observert,
tvertimot, elektronene blir emittert momentant.

For \aa\ l\o se de ovennevnte problemene med fotoelektrisk effekt, foreslo 
Einstein i 1905\footnote{Samme \aa r som han utviklet relativitetsteorien.} 
at den e.m.~energien er kvantisert
i konsentrerte deler (energibunter), som senere er blitt til partikler
med null masse som reiser med lysets fart, fotonene.
Han antok at energien til et slikt foton var bestemt
av dets frekvens
\begin{equation}
   E=h\nu.
\end{equation}
Han antok ogs\aa\ at i den fotoelektriske effekt
blir denne energibunten (fotonet) fullstendig
absorbert av elektronet. Energibalansen uttrykkes ved
\begin{equation}
K_{max}=h\nu-w
\end{equation}
hvor $w$ er arbeidet som kreves for \aa\ fjerne et elektron fra metallet.

Dersom vi ser p\aa\ de svakest bundne elektroner har vi
\begin{equation}
K_{max}=h\nu-w_0,
\end{equation}
hvor $w_0$ kalles arbeidsfunksjonen, den minste energi som trengs
for \aa\ fjerne det svakest bundne elektron for \aa\ unnslippe de
tiltrekkende kreftene (Coulomb) som binder et elektron til et metall.
Arbeidsfunksjonen $w_0$ er spesifikk for ethvert materiale og har typiske
verdier p\aa\ noen f\aa\ eV.

N\aa r det gjelder problemene med klassisk teori, 
s\aa\ kan vi se fra de to siste
likningene at hva ang\aa r punkt 1), s\aa\ er det n\aa\ samsvar med eksperiment
og teori. Det faktum at $K_{max}$ er proporsjonal med 
$E=h\nu$ viser at den kinetiske energien elektronene har 
er uavhengig av intensiteten til den
e.m.~str\aa lingen. Dersom vi dobler intensiteten, s\aa\ p\aa virker ikke
det energien til et foton, som er gitt ved kun $E=h\nu$.

Hva ang\aa r punkt 2), ser vi at vi kan bestemme den minste frekvensen
ved \aa\ sette den kinetiske energien til det frigjorte elektron lik null.
Da har vi 
\begin{equation}
h\nu_0=w_0,
\end{equation}
og kan dermed forklare observasjonen av en minste tillatt frekvens

Innvending nummer tre kan ogs\aa\ l\o ses dersom en tenker seg at det
er et foton som treffer elektronet, og overf\o rer det meste
av sin energi til elektronet. Da trenger en ikke \aa\ bombardere
med jevnt fordelt str\aa ling metallplaten av kalium. 
I tillegg er antall fotoner som sendes inn enormt. En enkel betraktning
kan hjelpe oss her.
Anta at vi sender inn monokromatisk (ensfarget) gult lys med
b\o lgelengde $\lambda=5890$ \AA\, hvor $1$ \AA\ $=10^{-10}$ m.
Effekt per arealenhet $M$ til flaten til lyskjeglen 1 m fra lyskilden er
da (dvs.~e.m.~energi per areal per sekund)
\begin{equation}
    M= \frac{1\hspace{0.1cm}\mathrm{J/s}}{4\pi\hspace{0.1cm}\mathrm{m}^2}=
        8\times 10^{-2} \hspace{0.1cm}\mathrm{J/m^2s}=5\times 10^{17} \hspace{0.1cm}\mathrm{eV/m^2s},
\end{equation}
og regner vi ut energien til hvert foton har vi
\begin{equation}
   E=h\nu =\frac{hc}{\lambda}=3.4\times 10^{-19} \hspace{0.1cm}\mathrm{J}=2.1 \hspace{0.1cm}\mathrm{eV}.
\end{equation}
Totalt antall fotoner $N$ per areal per sekund blir
\begin{equation}
   N=\frac{5\times 10^{17} \hspace{0.1cm}\mathrm{eV/m^2s}}{2.1 \hspace{0.1cm}\mathrm{eV}}= 2.4\times 10^{17}\hspace{0.1cm}\mathrm{foton/m^2s},
\end{equation}
som betyr at sannsynligheten for at et foton treffer et elektron og
overf\o rer det meste av sin energi er noks\aa\ stor.

Merk at fotonene blir absorbert i den fotolektriske prosess.
Det betyr at dersom vi skal bevare bevegelsesmengde og energi, s\aa\
m\aa\ elektronene v\ae re bundet til atomet/metallet. Vi skal diskutere
dette etter avsnittet om Compton effekten.





\section{R\"ontgen str\aa ling}

R\"ontgen str\aa ling\footnote{Henvisning til boka er kap 2-6, sidene 
103-107.} svarer til det motsatte av fotoelektrisk effekt. Her sendes
energirike elektroner som akselereres gjennom et potensialfall $V_R$ p\aa\ flere
tusen V mot et metall. 
Figur \ref{fig:xray1} viser en skisse over et 
eksperimentelt oppsett for produksjon av R\"ontgenstr\aa ling. 
\begin{figure}[h]
\begin{center}
{\centering
\mbox
{\psfig{figure=xray.ps,height=6cm,width=12cm}}
}
\end{center}
\caption{Eksperimentelt oppsett for produksjon av R\"ontgenstr\aa ling. \label{fig:xray1}}
\end{figure}


Elektronene vekselvirker med atomene i metallet via Coulomb
vekselvirkningen,
og overf\o rer bevegelsesmengde til atomene. Elektronene bremses dermed ned, og i denne
deakselerasjonen sendes det ut e.m.~str\aa ling i R\"ontgen omr\aa det.
Denne str\aa lingen som skyldes nedbremsingen av elektronet kalles
Bremsstraahlung fra tysk for bremsestr\aa ling. 
Dette svarer til e.m.~str\aa ling 
med frekvenser i st\o rrelsesorden
$10^{16}-10^{21}$ Hz, b\o lgelengder 
i st\o rrelsesorden
$10^{-7}-10^{-13}$ m og energier for fotoner i st\o rrelsesorden
$10^{1}-10^{6}$ eV. 

Idet elektronet bremses ned, kan vi anta at det vekselvirker mange ganger med atomene
i materialet og dermed f\aa r vi et kontinuerlig energispektrum for
utsendt  e.m.~str\aa ling i
R\"ontgen omr\aa det, slik som vist i figur \ref{fig:xray2}.
De to toppene $K_{\alpha}$ og $K_{\beta}$ skyldes bestemte eksiterte 
tilstander i metallet brukt under eksperimentet, se igjen  
figur \ref{fig:xray1}. Vi kommer tilbake til dette under v\aa r diskusjon
om det periodiske systemet.
\begin{figure}[h]
\begin{center}
{\centering
\mbox
{\psfig{figure=xray2.ps,height=6cm,width=10cm}}
}
\end{center}
\caption{Intensitets fordeling fra R\"ontgenstr\aa ling. \label{fig:xray2}}
\end{figure} 

Dersom vi antar at atomene er mye tyngre enn det enkelte elektron, kan vi 
idealisere prosessen til \aa\ v\ae re gitt ved 
\begin{equation}
   e^-\rightarrow e^-+\gamma,
\end{equation}
hvor vi heretter i dette kurset kommer til bruke indeks $\gamma$ for fotoner
og $e^-$ for elektroner. 
Kaller vi den kinetiske energien til elektronet f\o r st\o tet for $K_e$ og den etter
for $K_e'$ har vi f\o lgende energibalanse
\begin{equation}
    h\nu= \frac{hc}{\lambda}=K_e-K_e'.
\end{equation}
Det en observerte ved R\"ontgen str\aa ling var at det fantes en minste
b\o lgelengde for utsending av  R\"ontgen str\aa ling, noe som ikke kunne forklares
vha.~klassisk e.m.~teori. Dette svarer igjen til en maksimal energi som fotonene kan ha.
Med en gitt innkommende kinetisk energi for elektronene, observerte en  ulike 
frekvens(b\o lgelengde)fordelinger
for den e.m.~str\aa lingen. Men, felles for alle kinetiske energier var en minste b\o lgelengde.

Dersom vi antar at elektronet har null kinetisk energi etter st\o tet, dvs.~at det kommer til ro,
har vi
\begin{equation}
     h\nu= \frac{hc}{\lambda}=K_e,
\end{equation}
som igjen gir oss
\begin{equation}
     \frac{hc}{\lambda_{min}}=K_e,
\end{equation}
og setter vi inn at $K_e=eV_R$ har vi
\begin{equation}
     \lambda_{min}=\frac{hc}{eV_R}.
\end{equation}

Det finnes mange eksempler p\aa\ at n\aa r ladde partikler bremses ned
s\aa\ sendes det ut h\o genergetisk e.m.~str\aa ling. Kosmisk str\aa ling er et slikt
eksempel, og hvem har ikke sett Nordlys med det blotte \o ye?

Et mer eksotisk eksempel er s\aa kalt R\"ontgenstjerner, eller dersom vi bruker
det engelske faguttrykket X-ray pulsars og bursters!
Pulsarer er hurtigroterende n\o ytronstjerner i bin\ae re stjernesystemer (to stjerner n\ae r 
hverandre).
X-ray pulsarer og bursters
er antatt \aa\ v\ae re n\o ytronstjerner som mottar masse fra en annen stjerne
i et bin\ae rsystem. Massen til den andre stjerna kan v\ae re
 flere solmasser for pulsarer ($M > 
10M_\odot$) eller ha liten masse ($M <  1.2M_\odot$) for bursters.
Den utsendte e.m.~str\aa ling i R\"ontgenomr\aa det antas \aa\ skyldes
masse som samles enten ved polene eller over hele stjerna. N\aa r masse
slynges ned mot stjerna vil ulike kjernefysiske reaksjoner sende ut 
e.m.~str\aa ling i R\"ontgenomr\aa det.

\section{Compton spredning}
%\begin{figure}
%\begin{center}
%{\centering
%\mbox
%{\psfig{figure=comptonpr.ps,height=16cm,width=10cm}}
%}
%\caption{Utdrag fra Comptons artikkel i Physical Review i 1923. Compton fikk
%Nobelprisen for effekten oppkalt etter ham i 1927.}
%\end{center}
%\end{figure}

I et eksperiment fra 1923\footnote{Dekkes av kap 2-7, sidene 107-113.}, 
sendte Compton inn h\o genergetiske fotoner
(R\"ontgen str\aa ler) mot en grafittplate og observerte bla.\
en forskjell i b\o lgelengde mellom den innkommende e.m.~str\aa ling
og den utg\aa ende. Siden energien er knytta til b\o lgelengden via
\[
  E=\frac{hc}{\lambda}=h\nu,
\]
betyr ei forandring i b\o lgelengde en energiforandring.
Figur \ref{fig:comptoneksp1} viser en skisse av oppsettet for Comptons fors\o k.
\begin{figure}[h]
\begin{center}
{\centering
\mbox
{\psfig{figure=compton3.ps,height=8cm,width=12cm}}
}
\end{center}
\caption{Eksperimentelt oppsett for Comptons fors\o k.
\label{fig:comptoneksp1}}
\end{figure}
Figur \ref{fig:comptoneksp2} viser, skjematisk, intensitetsfordelingen for ulike
spredningsvinkler $\theta$ 
for den e.m.~str\aa lingen.
\begin{figure}[h]
\begin{center}
{\centering
\mbox
{\psfig{figure=compton2.ps,height=8cm,width=12cm}}
}
\end{center}
\caption{Intensitetsfordeling for e.m.~str\aa ling i Comptons fors\o k.\label{fig:comptoneksp2}}
\end{figure}
Compton antok
at denne spredningsprosessen kunne idealiser som et st\o t mellom
et tiln\ae rma fritt elektron i grafittplata og h\o genergetiske fotoner. 
Dvs.\ at reaksjonen er gitt ved
\begin{equation}
   \gamma +e^- \rightarrow  \gamma +e^-.
\end{equation}
Dersom fotonet skulle kunne overf\o re all sin energi til elektronet, vil vi
ha en prosess av typen 
\begin{equation}
   \gamma +e^- \rightarrow  e^-,
\end{equation} 
og som vi skal nedenfor s\aa\ strider en slik prosess med bevaring av
energi og bevegelsesmengde.
Vi kunne ogs\aa\ tenke oss at fotonet  kreerte et elektron, dvs.
\begin{equation}
   \gamma \rightarrow  e^-.
   \label{eq:gammaannihil}
\end{equation} 
En slik prosess er ogs\aa\ umulig, ikke bare fordi vi ikke kan tilfredsstille
bevaring av energi og bevegelsesmengde, men ogs\aa\ fordi ladning ikke
er bevart. Et foton har null ladning, mens et elektron har ladning $-e$.

I senere eksperiment observerte en ogs\aa\ det utg\aa ende elektronet.
I slik forstand kan vi betrakte denne prosessen som en analog til
fotoelektrisk effekt, hvor innkommende e.m.~str\aa ling river l\o s de svakest
bundne elektronene. En viktig forskjell er dog at i fotoelektrisk effekt
s\aa\ blir den e.m.~str\aa ling i all hovedsak absorbert. 

Grunnen til at Compton kunne idealisere sitt spredningseksperiment
som et st\o t mellom et tiln\ae rma fritt elektron og et foton
ligger i energien til fotonene, som n\aa\ er flere st\o rrelsesordener
st\o rre enn i fotoelektrisk effekt. 
Siden energien til de innkommende fotonene $E_{\gamma}$ er s\aa\ stor, 
betyr det at arbeidsfunksjonen $w_0$, som er p\aa\ noen f\aa\ eV,
kan neglisjeres i energibalansen og vi kan betrakte det hele som et st\o t
mellom et foton og et fritt elektron.  

Hvordan kan vi forst\aa\ dette? 
La oss f\o rst sette opp resultatet av Comptons regning, dvs.\
forandringen i b\o lgelengde ved Compton spredning,
gitt ved $\Delta \lambda=\lambda'-\lambda$.
Compton viste at den kunne skrives som (se utledning nedenfor)
\begin{equation}
     \Delta \lambda=\lambda_C(1-cos\theta),
\end{equation}
med 
\begin{equation}
\lambda_C=\frac{h}{m_ec}=0.0243\times 10^{-10}\hspace{0.1cm}\mathrm{m},
\end{equation}
og vi ser at $\Delta \lambda$ varierer fra $0$ til en maks verdi
$2\frac{h}{m_ec}$. $\lambda_C$ kalles ogs\aa\ Comptonb\o lgelengden.
{\em Legg merke til $\Delta \lambda$ er uavhengig av materiale
og b\o lgelengden til den innkommende e.m.~str\aa ling.} Det var dette,
som i motsetning til fotoelektrisk effekt med en arbeidsfunksjon $w_0$
bestemt av materialet,  som bla.~leda Compton til \aa\ anta at prosessen
kunne idealiserer vha.~et fritt elektron, og ikke et som er bundet
til et atom. 

Forklaringen ligger simpelthen i energist\o rrelsene.
I fotoelektrisk effekt har vi e.m.~str\aa ling i omr\aa det med synlig lys
til ultrafiolett lys. Det betyr at vi har frekvenser i st\o rrelsesorden
$10^{15}-10^{17}$ Hz, noe som tilsvarer b\o lgelengder
i st\o rrelsesorden
$10^{-7}-10^{-8}$ m og energier for fotoner i st\o rrelsesorden
$10^{0}-10^{2}$ eV. 
Dersom vi ser p\aa\ v\aa rt eksempel med monokromatisk gult lys,
s\aa\ er forholdet
\begin{equation}
   \frac{\lambda_C}{\lambda}\sim 10^{-6},
\end{equation}
dvs.~ at vi knapt kan observere en forandring i b\o lgelengde n\aa r vi har
med fotoelektrisk effekt \aa\ gj\o re. Legg ogs\aa\ merke til at arbeidsfunksjonen for f.eks.~kalium er p\aa\ 2.1 eV, p\aa\ st\o rrelse med energien
til fotonene. 

Med R\"ontgenstr\aa ler derimot, har vi  
e.m.~str\aa ling 
med frekvenser i st\o rrelsesorden
$10^{16}-10^{21}$ Hz, noe som tilsvarer b\o lgelengder
i st\o rrelsesorden
$10^{-7}-10^{-13}$ m og energier for fotoner i st\o rrelsesorden
$10^{1}-10^{6}$ eV. Dersom vi antar at fotonene har en frekvens
p\aa\  $10^{19}$ Hz, gir det en energi p\aa\ $41000$ eV og en 
b\o lgelengde
\begin{equation}
    \lambda=\frac{c}{\nu}=3\times 10^{-11}\hspace{0.1cm}\mathrm{m}.
\end{equation}

Ser vi f\o rst p\aa\  $\lambda_C/\lambda$ finner vi n\aa\ 
\begin{equation}
   \frac{\lambda_C}{\lambda}=\frac{0.0243\times 10^{-10}\hspace{0.1cm}\mathrm{m}}{3\times 10^{-11}\hspace{0.1cm}\mathrm{m}}=0.081
\end{equation}
som betyr at forskjellen i b\o lgelengde burde v\ae re observerbar.
\O ker vi frekvensen blir forholdet klart st\o rre, og dermed 
st\o rre sannsynlighet for \aa\ observere ei forandring i
b\o lgelengde. 
Ser vi deretter p\aa\ energien, ser vi at $41000$ eV er mye st\o rre enn 
typiske energier i fotoelektrisk effekt, og arbeidsfunksjonen $w_0$
som er p\aa\ noen f\aa\ eV. Derfor kunne Compton i dette tilfelle
idealisere spredningen av fotoner mot et materiale som spredning
av et foton mot et fritt elektron. Det vi har antatt er at den kinetiske
energien elektronet f\aa r er mye st\o rre enn $w_0$, og at vi kan 
neglisjere $w_0$. Det vil svare til en situasjon hvor elektronet 
ikke er bundet, dvs.~det er fritt.

Det vi f\o rst skal forklare er selve forandringen i
b\o lgelengde. Til slutt skal vi ogs\aa\ forklare tilfellet
med null b\o lgeforandring ogs\aa\, se igjen figur \ref{fig:comptoneksp2}.


Fotonet er en partikkel med masse null som reiser med lysets hastighet.
For \aa\ se at massen m\aa\ v\ae re null, kan vi bruke noen enkle
argument fra FY-ME100. Vi har at energien til en fri partikkel med hastighet
$v$ er gitt ved
\begin{equation}
    E=\frac{m_0c^2}{\sqrt{1-v^2/c^2}}.
\end{equation}
N\aa r vi kombinerer dette med det faktum at
fotonene reiser med lysets hastighet $c$ og at energien til et foton
er endelig og gitt ved $E=h\nu$, s\aa\ m\aa\ massen til fotonet v\ae re lik 
null, ellers vil energien divergere i det ovennevnte uttrykk. 
Bruker vi ogs\aa\ relasjonen 
\begin{equation}
   E=\sqrt{p^2c^2+m^2c^4}=pc=h\nu=\frac{hc}{\lambda},
\end{equation}
har vi at bevegelsesmengden $p$ er gitt ved
\begin{equation}
   p=\frac{h\nu}{c}.
\end{equation}


N\aa r vi skal utlede Comptons formel, trenger vi alts\aa\ \aa\ ta 
utganspunkt i to bevaringssatser, energi og bevegelsesmengde, f\o r
og etter st\o tet. 
Vi antar ogs\aa\ at f\o r st\o tet s\aa\ er elektronet i ro,
dvs.~at det har null kinetisk energi og bevegelsesmengde. Vi kan da,
se ogs\aa\ figur \ref{fig:compton1},  
\begin{figure}[h]
\begin{center}
{\centering
\mbox
{\psfig{figure=compton1.ps,height=8cm,width=12cm}}
}
\end{center}
\caption{Idealisering av kollisjonen mellom et foton og et elektron. Vi antar at det er et 
tiln\ae rmet fritt elektron i ro som kolliderer med et foton.\label{fig:compton1}}
\end{figure}
lage oss en tabell med definisjonene
av energi og bevegelsesmengde for elektroner og fotoner
\begin{table}[h]
\begin{center}
\caption{Definisjon av energi og begelsesmengde f\o r og etter st\o tet for
fotonet og elektronet.}
\begin{tabular} {lll} \\ \hline
                & F\o r & Etter \\ \hline
Bevegelsesmengde&  & \\
 $\gamma$          & $p_{\gamma}=\frac{h}{\lambda}$ & $p'_{\gamma}=\frac{h}{\lambda'}$ \\
 $e^-$          & $p_e=0$ & $p'_e$ \\
Energi&  & \\
 $\gamma$          & $E_{\gamma}=\frac{hc}{\lambda}$ & $E'_{\gamma}=\frac{hc}{\lambda'}$ \\
 $e^-$   & $E_e=m_ec^2$ & $E'_e=\sqrt{p'_ec)^2+m_e^2c^4}$    \\
\\ \hline
\end{tabular}
\end{center}
\end{table}
hvor $m_e$ er elektronets masse.

Energibevaring gir
\begin{equation}
  E_{\gamma}+E_e=E'_{\gamma}+E'_e,
\end{equation}
dvs.
\begin{equation}
  \frac{hc}{\lambda}+m_ec^2=\frac{hc}{\lambda'}+\sqrt{(p'_ec)^2+m_e^2c^4}.
   \label{eq:energycompt} 
\end{equation}
Tilsvarende har vi for bevaring av bevegelsesmengde
\be
     {\bf p}_{\gamma}   +{\bf 0 }  =   {\bf p}'_{\gamma} + {\bf p}'_e.
    \label{eq:momcompt}
\end{equation}
Dersom vi kvadrerer likning (\ref{eq:energycompt}) f\aa r vi
\begin{equation}
   (p'_ec)^2+m_e^2c^4=\left(\frac{hc}{\lambda}-\frac{hc}{\lambda'}\right)^2
   +2m_ec^2\left(\frac{hc}{\lambda}-\frac{hc}{\lambda'}\right)+m_e^2c^4,
\end{equation}
som gir
\begin{equation}
   (p'_ec)^2=\left(\frac{hc}{\lambda}\right)^2+\left(\frac{hc}{\lambda'}\right)^2-2\frac{h^2c^2}{\lambda\lambda'}+2m_ec^2\left(\frac{hc}{\lambda}-\frac{hc}{\lambda'}\right).
\label{eq:left}
\end{equation}
Deretter kvadrerer vi likning (\ref{eq:momcompt}) og f\aa r 
\begin{equation}
     \left({\bf p}_{\gamma}-{\bf p}'_{\gamma}\right)^2 =({\bf p}'_e)^2,
\end{equation}
som gir n\aa r vi multipliserer begge sider med $c^2$ 
\begin{equation}
   ({\bf p}'_ec)^2=\left(\frac{hc}{\lambda}\right)^2+\left(\frac{hc}{\lambda'}\right)^2-2\frac{h^2c^2}{\lambda\lambda'}cos\theta.
\label{eq:right}
\end{equation}
Setter vi venstresidene i likningene (\ref{eq:left}) og (\ref{eq:right}) like
og multipliserer med
\begin{equation}
    \frac{\lambda\lambda'}{2hc},
\end{equation}
f\aa r vi 
\begin{equation}
   m_ec^2\left(\lambda'-\lambda\right)-hc=-cos\theta hc,
\end{equation}
som igjen gir
\begin{equation}
   \left(\lambda'-\lambda\right)=\Delta \lambda = \frac{hc}{m_ec^2}\left(1-cos\theta\right)=\frac{h}{m_ec}\left(1-cos\theta\right),
\end{equation}
som er Comptons formel. Vi ser at siden b\o lgelengden til den e.m.~str\aa lingen
etter st\o tet er st\o rre enn f\o r, s\aa\ inneb\ae rer det at 
energien til fotonet etter st\o tet er mindre enn f\o r, dvs.~siden
\begin{equation}
   \lambda' > \lambda,
\end{equation}
s\aa\ har vi at
\begin{equation}
   E_{\lambda'} = \frac{hc}{\lambda'} < E_{\lambda} = \frac{hc}{\lambda}.
\end{equation}


Kan s\aa\ fotonet  overf\o re all sin energi til et fritt elektron? 
Det betyr at vi tar utgangspunkt i prosessen i likning (\ref{eq:gammaannihil}),
noe som impliserer at fotonet blir annihilert, og at dets energi og bevegelsesmengde etter
st\o tet er null. Tar vi i bruk bevaring av energi og bevegelsesmengde igjen 
finner vi at 
energibevaring gir
\begin{equation}
  E_{\gamma}+E_e=E'_e,
\end{equation}
dvs.
\begin{equation}
  \frac{hc}{\lambda}+m_ec^2=\sqrt{(p'_ec)^2+m_e^2c^4}.
   \label{eq:energycompt2} 
\end{equation}
Tilsvarende har vi for bevaring av bevegelsesmengde
\begin{equation}
     {\bf p}_{\gamma}   +{\bf 0 }  =  {\bf 0 }  + {\bf p}'_e,
    \label{eq:momcompt2}
\end{equation}
som gir n\aa r vi kvadrerer og multipliserer med $c^2$
\begin{equation}
   ({\bf p}'_ec)^2=\left(\frac{hc}{\lambda}\right)^2.
\end{equation}

Vi kvadrerer igjen likning (\ref{eq:energycompt2}) og f\aa r 
\begin{equation}
   (p'_ec)^2+m_e^2c^4=\left(\frac{hc}{\lambda}\right)^2
   +2m_ec^2\frac{hc}{\lambda}+m_e^2c^4,
\end{equation}
som gir, n\aa r vi bruker bevaring av bevegelsesmengde
\begin{equation}
   \left(\frac{hc}{\lambda}\right)^2=\left(\frac{hc}{\lambda}\right)^2
   +2m_ec^2\frac{hc}{\lambda},
\end{equation}
som impliserer at 
\begin{equation}
   2m_ec^2\frac{hc}{\lambda}=0,
\end{equation}
i strid med eksperiment, elektronet har en endelig masse $m_ec^2=0.511$ MeV.
Men vi ser at klassisk er denne prosessen fullt mulig, da setter vi nemlig
$h=0$, og vi har ikke noen inkonsistens med elektronets masse.


Det vi ikke har forklart er hvorfor vi har en topp p\aa\ Figur \ref{fig:comptoneksp2}
hvor vi tilsynelatende ikke har noen forandring
i b\o lgelengde. 
Dette kan forst\aa s p\aa\ f\o lgende vis. N\aa r vi har en forandring
i b\o lgelengde, vekselvirker egentlig fotonet med de svakest 
bundne elektronene, og vi idealiserer reaksjonen som et st\o t mellom
et foton og et elektron. Men vi kan jo tenke oss at elektronet er sterkt
bundet til atomet, eller at den innfallende e.m.~str\aa ling ikke
er sterk nok til \aa\ sparke ut et elektron. I dette tilfelle kan vi betrakte
kollisjonen mellom fotonet og materialet som en kollisjon mellom
et foton og et atom, og atomet f\aa r en rekyleffekt gjennom
kollisjonen. 
Dersom vi antar at materialet best\aa r av karbon, blir reaksjonen
v\aa r
\begin{equation}
   \gamma + C \rightarrow  \gamma + C,
\end{equation}
og 
i dette tilfelle m\aa\ vi erstatte massen til elektronet med   
den til karbonatomet, som er 22000 ganger tyngre enn elektronet.
Da blir
\begin{equation}
\lambda_C=\frac{h}{22000m_ec}\sim 10^{-16}\hspace{0.1cm}\mathrm{m},
\end{equation}
og med de aktuelle b\o lgelendge blir $\lambda_C/\lambda$
knapt observerbar!

Som en oppsumering p\aa\ fotoelektrisk effekt, R\"ontgen str\aa ling  og Compton spredning
kan vi si at e.m.~str\aa ling utviser b\aa de partikkel og 
b\o lgeegenskaper. Str\aa lingen, hva den enn m\aa tte v\ae re, utviser
i noen tilfeller partikkel egenskaper, og andre tilfelle reine
b\o lgeegenskaper slik som diffraksjon og interferens.

Comptons eksperiment utviser begge deler:\newline
1) Prinsippene bak m\aa lingen 
av den spredte e.m.~str\aa lingen ble gjort vha.\
standard b\o lgel\ae re.\newline
2) Spredningen p\aa virker b\o lgelengden p\aa\ et vis som kun
kan forst\aa s ved \aa\ behandle R\"ontgen str\aa lene som 
partikler som kolliderer  med elektronene i et atom.

Vi kan si at uttrykket
\[
   E=h\nu
\]
har i seg b\o lgebeskrivelsen ved
\be
    \lambda  \hspace{0.1cm} og  \hspace{0.1cm}  \nu
\end{equation}
og partikkelbeskrivelsen ved
\begin{equation}
    E \hspace{0.1cm} og  \hspace{0.1cm}  p=h/\lambda
\end{equation}

\begin{center}
\shabox{\parbox{14cm}{Vi skal legge merke til at b\aa de for den fotoelektriske effekt og for Compton spredning,
s\aa\ er det av avgj\o rende betydning at vi ser p\aa\ sv\ae rt sm\aa\ b\o lgelengder. I grensa
$\lambda\rightarrow \infty$ g\aa r de kvantemekaniske resultat mot de klassiske.
N\aa r vi beveger oss inn i R\"ontgenstr\aa lingen sitt domene, begynner energiene \aa\ bli
s\aa pass store at vi kan betrakte kollisjonen som et st\o t mellom et foton og et elektron
som er tiln\ae rminsgvis fritt. Forandringen i b\o lgelengde blir s\aa pass stor 
at det er mulig \aa\ observere den. Energiskalaen og det faktum at $h$ er liten 
dikterer hva slags fysisk teori og fysisk forst\aa else vi m\aa\ ta i bruk for \aa\ forklare
eksperiment. Det kan derfor i v\aa r analyse 
v\ae re hensiktsmessig \aa\ studere st\o rrelser som
\[
   \frac{\Delta E}{E},
\]
og 
\[
   \frac{\Delta \lambda}{\lambda}.
\]
}}\end{center}

\section{Oppgaver}
\subsection{Analytiske oppgaver}

\subsubsection*{Oppgave 1.1}
%
\begin{itemize}
\item[a)] Energienheten 1~eV {(\sl elektronvolt)} er definert som
{\o}kningen i kinetisk energi n{\aa}r et elek\-tron akselereres
gjennom et potensialsprang  p{\aa} 1~volt.
Vis at 1 eV $= 1,602 \times 10^{-19}$~J.\\
Hva blir energi{\o}kningen n{\aa}r elektronet akselereres i et
potensialsprang p{\aa} 10~V,\\
 50~kV $ = 5 \times 10^4$~V
og 1~MV $ = 10^6$~V ?\\
Beregn hastigheten elektronene f{\aa}r etter akselerasjon
i potensialene nevnt ovenfor n{\aa}r utgangshastigheten
$v_0 = 0$.
%
\item[b)] Beregn den potensielle energien for to partikler
med ladning $e = 1,602 \times 10^{-19}$~C som befinner
seg i en avstand 0,1~nm $ = 10^{-10}$~m. Vis at
enheten elektronvolt er en naturlig enhet i dette tilfelle.
%
\item[c)] La oss anta at de to partiklene ovenfor er et proton
og et elektron. Diskut\'{e}r forholdet mellom gravitasjons--
og elektrostatisk potensiell energi i dette tilfelle.\\
(Gravitasjons konstanten: $\gamma = 6,67 \times 10^{-11}$
 N m$^2$~kg$^{-2}$.)
%
\item[d)] Beregn hvileenergien for elektronet og protonet.
Under hvilke betingelser m{\aa} vi regne med relativistiske
effekter?
%
\end{itemize}

\subsubsection*{Oppgave 1.2}
%
Anta at sola med radius $6,96 \times 10^8$~m str{\aa}ler som et
svart legeme. Av denne str{\aa}lingen mottar vi 1370~Wm$^{-2}$
her p{\aa} jorda i en avstand av $1,5 \times 10^{11}$~m.
Jorda har en radius $6378$~km.
%
\begin{itemize}
%
\item[a)] Beregn temperaturen til sola.
%
\item[b)] Anta at atmosf{\ae}ren rundt jorda reflekterer
$30 \%$ av den innkommende str{\aa}ling. Hvor mye energi
fra solen absorberer den per sekund og per kvadratmeter?
%
\item[c)] For {\aa} v{\ae}re i termisk likevekt, m{\aa}
jorda emittere like mye energi som den absorberer via
atmosf{\ae}ren hvert sekund. Anta at den str{\aa}ler
som et sort legeme. Finn temperaturen.
Hva betyr drivhuseffekten ?
\end{itemize}

\subsubsection*{Oppgave 1.3}
Defin\'{e}r uttrykket ``svart legeme'' (``blackbody'').
Plancks str\aa lingslov sier at utstr\aa lt effekt pr. arealenhet og 
pr. b\o lgelengde, $dR/d\lambda$, fra et sort legeme er gitt ved 
\[
  \frac{dR}{d\lambda}=\frac{2\pi hc^{2}}{\lambda^{5}(e^{hc/\lambda kT}-1)},
\]
der $k$ er Boltzmanns konstant og $T$ er legemets temperatur.  
I utledningen av likningen over gj\o res en antakelse som bryter med 
klassisk fysikk.  Forklar kort hva denne antakelsen best\aa r i.   
Bruk Plancks lov til \aa \ vise Wiens forskyvningslov
\[
 \lambda_{{\rm m}}T={\rm konstant}, 
\]
der $\lambda_{{\rm m}}$ er b\o lgelengden hvor $dR/d\lambda$ er maksimal
(konstanten skal ikke bestemmes, men har verdien 0.00290 ${\rm K}\cdot
{\rm m}$).
En del av de astrofysiske objekter som kalles pulsarer sender 
ut str\aa ling i r{\o}ntgenomr\aa det.
Ved observasjon av en slik pulsar finner man at den  
utstr\aa lte effekt pr. arealenhet og pr. b{\o}lgelengde 
er st\o rst ved en b\o lgelengde p\aa \ 
5.58 nm.  Anta at pulsaren str\aa ler som et sort legeme og 
regn ut overflatetemperaturen. 

\subsubsection*{Oppgave 1.4, Eksamen V-1994}
%
%
\begin{itemize}
%
\item[a)] Gj{\o}r rede for den fotoelektriske effekten. Tegn en figur
som viser en eksperimentell oppstilling for {\aa} studere denne effekten.
%
\item[b)] Anta at vi bruker denne oppstillingen til \aa\ bestr\aa le et metall
med monokromatisk lys, $\lambda = 2.58 \cdot 10^{-7}$ m. Figur~1.1 viser observerte
verdier av str{\o}mmen $I$ n{\aa}r spenningen varieres.
Forklar forl{\o}pet av kurven.
Hva skjer hvis intensiteten av lyset {\o}kes til det dobbelte ?
%
\begin{figure}[hbtp]
%
\setlength{\unitlength}{1cm}
%
\begin{center}
%
\begin{picture}(10,6)
%
\thicklines
%
%\put(0,0){\framebox(14,10){}}
	\put(0,0){\epsfxsize= 10cm \epsfbox{volt-1.eps}}
%

\end{picture}
%
\caption{Str{\o}m $I$ som funksjon av spenningen $V$ ved
foto--elektrisk effekt. \label{fig:figfoto}}
%
\end{center}
%
\end{figure}
%
\item[c)] Bruk figur~\ref{fig:figfoto} til {\aa} bestemme arbeidsfunksjonen for metallet.
%
\item[d)] Vi bytter n{\aa} ut  metallet foran med et nytt.  Det
bestr{\aa}les med lys med forskjellig b{\o}lgelengde. Tabell~\ref{tabfoto}
viser stoppepotensialet for noen forskjellige verdier av b{\o}lgelengden.
Bestem ut fra dette arbeidsfunksjonen for det nye metallet, den lavest mulige
frekvensen for prosessen og Plancks konstant $h$.
%
\begin{table}[htbp]
%
\begin{center}
%
\begin{tabular}{||r|r|r|r|r||}\hline
$\lambda$ (i 10$^{-7}$ m)& 2,536 & 3,132 & 3,650 & 4,047\\ \hline
$V_S$ (i volt)            & 1,95 & 0,98 & 0,50 & 0,14\\ \hline
\end{tabular}
%
\label{tabfoto}
%
\caption{Stoppepotensial for forskjellige verdier av b{\o}lgelengden.}
%
\end{center}
%
\end{table}
%
\item[e)] Forklar hvorfor et foton ikke kan overf{\o}re all
sin energi og bevegelsesmengde til et fritt elektron.
%
\end{itemize}
%

\subsubsection*{Oppgave 1.5}
En str�le av ultrafiolett lys og med en intensitet p�  $1.6 \times 10^{-12}$W blitt pluselig satt p� 
og bestr�ler en metalflate. Ved fotoelektrisk effekt blir elektroner sendt ut fra metallflaten.
Den innkommende str�len har et tverrsnitt p� 1~cm$^2$ og b�lgelengden svarer til en foton energi p� 10~eV. 
Arbeidsfunksjonen for metallet er 5~eV.  Vi skal i det f�lgende analysere hvor lang tid det vil ta f�r 
elektroner emitteres.
%
\begin{itemize}
%
\item[a)] Gj�r et klassisk estimat basert p� den tid det vil ta f�r et elektron med 
radius $\approx 1$~� har absorbert nok energi til � emitteres.
%
\item[b)] Lord Rayleigh viste imidlertid (Phil. Mag. {\bf 32}, (1916), side 188) at dette
 estimatet er for pessimistisk. Absorbsjonsarealet for et elektron i et atom er av st�rrelsesorden 
$\lambda^2$ for lys av b�lgelengde $\lambda$. Beregn  den klassiske  forsinkelsestiden
for emisjon av et elektron.
%
\item[c)] Kvantemekanisk  er det mulig for et elektron � emitteres  umiddelbart -- s� snart 
et foton med tilstrekkelig energi  treffer metalflaten. For � f� et estimat som kan sammenlignes med 
den klassiske verdien beregn det midlere tidsintervall mellom to fotoner i str�len. 
%
\end{itemize}

\subsubsection*{Oppgave 1.6}
\begin{itemize}
\item[a)] Gj{\o}r kort rede for den fotoelektriske effekten,
og skiss\'{e}r en  eksperimentell oppstilling
som kan observere og m{\aa}le denne
effekten.
%
\item[b)] Den fotoelektriske arbeidsfunksjonen for kalium (K)
er 2,0~eV. Anta at lys med en b{\o}lgelengde p{\aa}
360~nm ( 1~nm = $10^{-9}$~m)
faller p{\aa} kaliumet. Finn stoppepotensialet for fotoelektronene,
den kinetiske energien og hastigheten for de hurtigste av de emitterte
elektronene.
%
\item[c)] En uniform monokromatisk lysstr{\aa}le med b{\o}lgelengde
400~nm faller p{\aa} et materiale med arbeidsfunksjon p{\aa} 2,0~eV, og
med en intensitet p{\aa} $3,0 \times 10^{-9}$ Wm$^{-2}$.
Anta at materialet reflekterer 50~\% av den
innfallende str{\aa}le, og at 10~\% av de absorberte fotoner
f{\o}rer til et emittert elektron.
Finn antall
elektroner emittert pr. m$^2$ og pr. sec, den absorberte
energi pr. m$^2$ og pr. sec, samt den kinetiske energi for
fotoelektronene.
\end{itemize}


\subsubsection*{Oppgave 1.7}
I et r{\o}ntgenr\o r lar vi en elektronstr\aa le gjennoml\o pe et 
spenningsfall $V$ f\o r den treffer en anode av wolfram (W).  
Den resulterende str\aa lingen fra r{\o}ntgenr\o ret observeres.  
Figur~\ref{fig:xray2} viser hvordan str\aa lingens intensitet varierer med 
b\o lgelengden $\lambda$.
%
\begin{itemize}
%
\item[a)]Tegn en skjematisk skisse av et r{\o}ntgenr\o r.  Gi en kort og 
kvalitativ 
beskrivelse av hvordan r{\o}ntgenstr\aa lingen dannes og typiske trekk 
ved den jevne kontinuerlige delen av spekteret i figur~\ref{fig:xray2}.
%
\item[b)]Skriv ned og begrunn sammenhengen mellom spenningen $V$ og 
den minste 
b\o lgelengden $\lambda_{{\rm min}}$ for r\o ntgenstr\aa lingen.  Gj\o r 
det samme for $V$ og den maksimale frekvensen $\nu_{{\rm maks}}$.  

\end{itemize}

\subsubsection*{Oppgave 1.8}

\begin{itemize}
%
\item[a)] Tegn en skjematisk skisse av et r\"{o}ntgenr{\o}r. Gi
en kort og kvalitativ beskrivelse av hvordan r\"{o}ntgenstr{\aa}lingen
dannes og typiske trekk ved fotonspektret.
%
\item[b)] Hvis et elektron har kinetisk energi $E_k = 10000$~eV,
beregn den minste b{\o}lgelengden $\lambda_{min}$ for de utsendte fotonene.
%
\item[c)] Vi antar  at vi har r\"{o}ntgenstr{\aa}ler med  b{\o}lgelengde
$\lambda = 1 \mbox{nm} = 1 \times 10^{-9}\mbox{m}$.
De faller inn mot en krystall  under en
vinkel p{\aa} 45$^{\circ}$ med innfallsloddet. Dette gir
en koherent refleksjon fra krystallen.
Beskriv prosessen og beregn ut fra dette
avstanden mellom de reflekterende plan i krystallen.
%
\item[d)] N{\aa}r et foton passerer en atomkjerne, kan det omdannes
til et elektron--positron par (pardannelse).
Finn den minste energien $E_{\nu}$  fotonet kan ha
for at denne prosessen kan skje (se bort fra mulige rekyleffekter).
%
\end{itemize}

\subsubsection*{Oppgave 1.9}
%
Et foton er spredt en vinkel $\theta$  i forhold til dets opprinnelige
retning etter {\aa} ha vekselvirket med et fritt elektron som var
i ro.
%
\begin{itemize}
%
\item[a)]  Gj{\o}r rede for hvilke prinsipper som anvendes til {\aa}
	  beregne det spredte fotonets b{\o}lgelengde og utled Comptons
	  formel.
%
\item[b)]  Begrunn at b{\o}lgelengden for det spredte
	  fotonet er st{\o}rre enn b{\o}lgelengden for fotonet f{\o}r
	  vekselvirkningen (spredningsvinkelen   forutsettes {\aa} v{\ae}re
	  st{\o}rre enn null).
%
\item[c)] Et foton er spredt en vinkel $\theta$   etter {\aa} ha vekselvirket med et
          fritt elektron som f{\o}r vekselvirkningen hadde en
          bevegelsesmengde som var like stor som fotonets bevegelsesmengde, men
          motsatt rettet. Hvor stor blir fotonets b{\o}lgelengdeforandring i dette
	  tilfellet? Vi antar at elektronet kan beskrives
	  ikke--relativistisk.

\item[d)]  Kan et foton overf{\o}re hele sin energi og bevegelsesmengde
	  til et fritt elektron? Begrunn svaret.
\end{itemize}




\subsubsection*{Oppgave 1.10}
Et foton med b\o lgelengde $\lambda = 1,00\cdot 10^{- 11}$ m treffer et fritt
elektron i ro. Fotonet blir spredt under en vinkel $\theta$, og f�r
en b\o lgelengdeforandring gitt ved Comptons ligning $\Delta \lambda =
\lambda^{'} - \lambda = \lambda_{c} (1 - \cos \theta )$, hvor Comptonb\o
lgelengden er $\lambda_{c} = 2,426\cdot 10^{-12}$ m.
%
\begin{figure}[htbp]
%
\begin{center}

\setlength{\unitlength}{1cm}
%
\begin{picture}(9,6)
\thicklines
              \put(0,0){\makebox(0,0)[bl]{
              \put(3,0){\vector(0,1){6}}
              \put(3,3){\vector(1,0){4}}
              \dottedline{0.1}(0,3)(2.3,3)
              \put(2.2,3){\vector(1,0){0.5}}
              \put(3,3){\circle*{0.2}}
              \put(3,3){\vector(3,-1){3.5}}
              \put(6,2){\circle*{0.2}}
              \put(6,2.2){\small elektron} 
               \dottedline{0.1}(3,3)(4.5,5.25)
               \put(4.5,5.25){\vector(2,3){0.3}}
              \put(0.5,3.2){\makebox(0,0)[bl]{\small $\lambda$: foton}}
              \put(5,4.5){\makebox(0,0)[b]{\small $\lambda^{'}$: foton}}
              \put(3.7,3.4){\makebox(0,0)[bl]{\small $\theta = 60^{\circ}$}} 
              \put(4.5,2.6){\makebox(0,0)[bl]{\small $\phi = \;?$}}
              \put(7.1,3){\makebox(0,0)[bl]{$x$}}
              \put(3.2,5.5){\makebox(0,0)[bl]{$y$}}
         }} 
%
\end{picture}
%
\caption{\label{figcomp}}

\end{center}

\end{figure}
%
I denne oppgaven skal vi bare se p\aa ~det som observeres under en vinkel
$\theta = 60^{\circ}$ (se figur~\ref{figcomp}). Energi og bevegelsesmengde beregningene
skal uttrykkes ved enheten eV\@.
%
\begin{itemize}
%
\item[a)] Beregn energien og bevegelsesmengden til det innkommende fotonet.

\item[b)] Finn b\o lgelengden, bevegelsesmengde og den kinetiske energien
til det spredte fotonet.

\item[c)] Finn den kinetiske energien, bevegelsesmengden og
spredningsvinkelen for elektronet.
%
\end{itemize}
%

\subsubsection*{Oppgave 1.11, Eksamen V-1992}
Vi skal i denne oppgaven gj{\o}re bruk av det relativistiske uttrykk
for energien av en partikkel
%
\[
E = \sqrt{E_0^2 + (pc)^2}.
\]
%
Som energi enhet bruk $eV$ og $eV/c$ som enhet for bevegelsesmengde.
%
\begin{itemize}
\item[a)] Gj{\o}r kort rede for de st{\o}rrelser som inng{\aa}r i relasjonen
ovenfor.
%
\item[b)] Anvend relasjonen ovenfor p{\aa} et foton og finn fotonets
energi og bevegelsesmengde n{\aa}r b{\o}lgelengden er
$500 \; nm = 5,00 \times 10^{-7}\; m$.
%
\end{itemize}
%
Et foton med energi $E = h \nu_0 = h c / \lambda_0$ spres en vinkel
$\theta$ ved Comptoneffekt
mot et elektron som antas {\aa} ligge i ro f{\o}r spredning.
%
\begin{itemize}
%
\item[c)] Tegn en prinsippskisse for et Compton eksperiment
og angi spredningsvinklene for fotonet og elektronet etter
spredningen.
%
\item[d)] Formul\'{e}r de prinsipper som anvendes til {\aa} beregne
det spredte fotonets b{\o}lgelengde $\lambda^{'}$ og bruk dem
til {\aa} utlede Comptons formel
%
\[
\lambda^{'} - \lambda_0 = \frac{h}{m_e c} (1 - cos \theta )
\]
%
\end{itemize}
%
R{\o}ntgenstr{\aa}ler med en b{\o}lgelengde
$\lambda_0 = 1,21 \times 10^{-1}\; nm$
treffer en m{\aa}lskive med carbon atomer.
De spredte r{\o}ntgenstr{\aa}ler blir observert i en vinkel
p{\aa} 90$^{\circ}$.
\begin{itemize}
\item[e)] Beregn b{\o}lgelengden $\lambda^{'}$ av det spredte
fotonet ved $90^{\circ}$.
%
\item[f)] Figur~\ref{fig:comptoneksp2} viser en topp ved b{\o}lgelengden $\lambda_0$.
Hvilken prosess skyldes denne toppen?
\end{itemize}

\clearemptydoublepage
\chapter{BOHRS ATOMMODELL}

\section{Introduksjon}
\begin{figure}[h]
\begin{center}
{\centering
\mbox
{\psfig{figure=stamp_bohr.ps,height=6cm,width=8cm}}
}
\end{center}
\caption{Dansk frimerke som markerer Bohrs atomteori.}
\end{figure}

Allerede\footnote{Lesehenvisning til l\ae reboka er 
Kap.\ 3-5, 3-6, 3-7 og 3-8, sidene 144-170.} 
mot  slutten av 1800-tallet fantes det store mengder med
eksperimentelle data om spektroskopi fra atomer. Det dreide seg i 
all hovedsak om utsendt (emittert) e.m.~str\aa ling fra atomer.
Den viktige observasjonen var  at e.m.~str\aa ling utsendt fra
atomer er konsentrert ved bestemte diskrete b\o lgelengder.
Et av de mest studerte atomer var hydrogenatomet, som best\aa r
av et proton og et elektron. 
\begin{figure}[h]
\begin{center}
{\centering
\mbox
{\psfig{figure=hydrogen1.ps,height=6cm,width=12cm}}
}
\end{center}
\caption{Fotonspekteret for hydrogen. B\o lgelengdene er i nm.}
\end{figure}

Basert p\aa\ slike diskrete verdier for b\o lgelengden 
(spektrallinjer)
til den utsendte
e.m.~str\aa ling, se Figur \ref{fig:spektrabalmer}, 
\begin{figure}[h]
\begin{center}
{\centering
\mbox
{\psfig{figure=hydrogenspekter.ps,height=8cm,width=12cm}}
}
\end{center}
\caption{Ulike mulige overganger i hydrogenatomet. Grunntilstanden
har kvantetall $n=1$ og energi $-13.6$ eV, som ogs\aa\ svarer til energien for 
\aa\ frigj\o re et elektron i hydrogenatomet. \label{fig:spektrabalmer}}
\end{figure}
laget Balmer en
parametrisering for de observerte spektrallinjene allerede i 1885,
med b\o lgelengden gitt som
\be
    \lambda =(3645.6\times 10^{-10}\hspace{0.1cm}\mathrm{m})\frac{n^2}{n^2-4}
             \hspace{0.1cm}n=3,4,\dots ,
\end{equation}
som skal svare til overganger fra eksiterte tilstander i hydrogenatomet
til den f\o rste eksisterte tilstanden i hydrogen, dvs.\ $n=2$.
Balmerserien av overganger representerer overganger til den f\o rste
eksiterte tilstanden fra h\o yereliggende eksiterte tilstander.
Tilsvarende overganger til grunntilstanden kalles for Lymanserien.
For \aa\ foregripe den p\aa f\o lgende diskusjonen, kan vi si at
tallene $n=1,2,3,\dots$ skal svare til bestemte kvantetilstander
(merk at verdiene er diskrete) for elektronet i hydrogenatomet.
Merk at vi heretter, n\aa r vi snakker om et atoms energi, s\aa\ tenker
vi p\aa\ energien til elektronet(ne). Da opererer vi med en energi skala
p\aa\ noen f\aa\ eV. 
Tallet $n=1$ skal svare til grunntilstanden, eller normaltilstanden.
Denne tilstanden har en energi, som vi skal utlede b\aa de vha.~Bohrs atommodell og kvantemekanikk,  p\aa\ $-13.6$ eV. 
For $n=2$, som svarer til en energi p\aa\ $-3.4$ eV, har vi den f\o rste
eksiterte tilstand for hydrogenatomet. For $n=3,4,\dots$ har vi h\o yere
eksiterte tilstander, se Figur \ref{fig:spektrabalmer}. 


Grunnen til at bla.~Balmer studerte overganger fra $n=3,4,\dots$ til 
den f\o rste eksiterte tilstand med $n=2$ skyldtes enkelt og greit
at den utsendte str\aa ling var i omr\aa det for synlig lys til det 
ultrafiolette. I tillegg var det slik, og er fremdeles slik, at den e.m.~str\aa ling som blir sendt utfra fotosf\ae ren p\aa\ sola, dvs.~solas overflate,
ligger i omr\aa det for synlig lys til det ultrafiolette. Og sola har fra
forhistorisk tid alltid v\ae rt gjenstand for v\aa r nysgjerrighet og
vitebegj\ae r. I omr\aa det for synlig lys utsendt i fotosf\ae ren, 
har det negative hydrogenionet avgj\o rende betydning. Dette ionet har i kort
tid to elektroner i bane rundt kjernen og er meget viktig for emisjon
og absorpsjon av e.m.~str\aa ling i fotosf\ae ren. 
Dersom vi sammenligner data for utsendt e.m.~str\aa ling
fra solas overflate og e.m.~str\aa ling utsendt fra et svart
legeme med temperatur 5800 K (som er ca.~temperatur ved solas overflate),
ville vi f\aa tt en mer hakkete kurve med ulike fordypninger. 
Disse fordypningene
eller topper, svarer til utsendt str\aa ling fra ulike atomer i solas
atmosf\ae re. En bestemt topp er f.eks.~gitt  
ved b\o lgelengden $\lambda=656.1$ nm,
som igjen svarer til overgangen fra $n=3$ til $n=2$ i
hydrogenatomet. Denne overgangen fikk benevning  $H_{\alpha}$.  
 

I 1890 kom Rydberg med en parametrisering for alle mulige overganger
i hydrogenatomet, nemlig
\begin{equation}
\frac{1}{\lambda}=R_H\left(\frac{1}{n_f^2}-\frac{1}{n_i^2}\right),
\label{eq:ryd}
\end{equation}  
med 
\begin{enumerate}
\item $n_f=1$ og $n_i=2,3,\dots$ svarende til den s\aa kalte Lymanserien,(ultrafiolett)
dvs.~overganger fra eksiterte tilstander til grunntilstanden,
\item  $n_f=2$ og $n_i=3,4,\dots$ svarende til den s\aa kalte Balmerserien (synlig lys),
\item  $n_f=3$ og $n_i=4,5,\dots$ svarende til den s\aa kalte Paschenserien (infrar\o dt), osv.
\end{enumerate}
Konstanten $R_H$ er Rydbergkonstanten for hydrogen og gitt ved
\begin{equation}
    R_H=1.0967757\times 10^7\hspace{0.1cm}\mathrm{m}^{-1}.
\end{equation}

selv om parametriseringene reproduserte dataene, s\aa\ manglet en mer
grunnleggende teori til beskrivelsen av atomer. F\o rst med kvantemekanikken
var en i stand til \aa\ gi en tilfredsstillende matematisk beskrivelse.
Et viktig bidrag mot kvantemekanikken var Bohrs atommodell fra 1913.
Bohr utviklet en teori basert p\aa\ kombinasjonen av klassisk mekanikk
og Plancks og Einsteins kvantiseringsteorier fra e.m.~str\aa ling.

Teorien Bohr presenterte forklarte hydrogenspekteret 
og dermed Rydbergs formel. La oss se n\ae rmere p\aa\ postulatene
Bohr kom med.

\section{Bohrs postulater og hydrogenatomet}
Vi skal se p\aa\ de ulike postulatene som Bohr lanserte og diskutere
hvert av dem. Postulatene er som f\o lger 
       \begin{enumerate}
          \item Et elektron i et atom beveger seg i sirkul\ae re
                baner om en kjerne. Kreftene er gitt ved tiltrekning via Coulomb
                vekselvirkningen mellom elektron og kjerne.

          \item Klassisk kan et elektron befinne seg i en uendelighet
                av orbitaler. {\bf Bohr antok at et elektron kan kun bevege 
                seg i baner hvis banespinn $L$ er et heltalls
                multiplum av $\hbar=h/2\pi$.}

          \item Et elektron i en slik bane vil klassisk ha en 
                sentripetalakselearasjon og skal dermed 
                sende ut e.m.~str\aa ling. Ved utsending av energi
                deakselereres elektronet, og til slutt vil det
                komme helt inn til kjernen. {\bf Bohr antok 
                derimot at elektronets energi i en gitt orbital 
                er konstant.} 

          \item E.m.~str\aa ling sendes ut dersom et elektron
                som beveger seg i en orbital med energi 
                $E_i$, forandrer p\aa\ et diskontinuerlig vis
                sin bevegelse slik at energien er gitt ved 
                den til en orbital med energi $E_f$.
                \[
                   \nu=\frac{(E_i-E_f)}{h} 
                \]
      \end{enumerate}

\subsection{F\o rste postulat} 
Dette postulatet baseres seg p\aa\ eksistensen av en atomkjerne, og utledningen
av energien som et elektron kan ha er basert p\aa\ betrakninger fra
klassisk mekanikk. Siden protonet er ca 2000 ganger tyngre enn elektronet,
kan vi i f\o rste omgang beskrive systemet som et 
elektron som kretser om protonet som tyngdepunkt. Dvs.~at vi kun tar
hensyn til elektronets posisjon og energi som frihetsgrader.
Radien $r$ svarer da til avstanden mellom protonet og elektronet.
Kreftene $F$ som virker p\aa\ elektronet er gitt via Coulombtiltrekningen 
\begin{equation}
    F = k\frac{(-e)(+e)}{r^2},
\label{eq:fcoul}
\end{equation}
hvor vi heretter skal bruke
\begin{equation}
    k=\frac{1}{4\pi\epsilon_0}=8.988\times 10^9\hspace{0.1cm}\mathrm{Nm}^2/\mathrm{C}^2.
\end{equation}
Bruker vi ogs\aa\ elektronets ladning kvadrert 
har vi at $ke^2=1.44$ eVnm. Dette er en st\o rrelse som er grei \aa\ huske, da den forekommer ofte i dette kurset.
Bohr antok sirkul\ae re baner. Da er sentripetakselarasjonen gitt ved
\begin{equation}
   a_r=-\frac{v^2}{r},
\end{equation}
og bruker vi Newtons lov med $m_ea_r$ og likning (\ref{eq:fcoul}) f\aa r vi at
\begin{equation}
   m_ea_r=-m_e\frac{v^2}{r}=-k\frac{e^2}{r^2}.
   \label{eq:vn}
\end{equation} 
N\aa\ kan vi ogs\aa\ finne elektronets energi $E$ ved \aa\ summere
kinetisk energi $E_{kin}$ og potensiell energi $E_{pot}$.
Sistnevnte f\o lger fra likning (\ref{eq:fcoul}) 
\begin{equation}
   F=-\frac{\partial E_{pot}}{\partial r},
\end{equation}
slik at vi f\aa r
\begin{equation}
    E=E_{kin}+E_{pot}=\frac{1}{2}mv^2-k\frac{e^2}{r},
\end{equation} 
og bruker vi likning (\ref{eq:vn}) 
finner vi
\begin{equation}
   E=\frac{1}{2}k\frac{e^2}{r}-k\frac{e^2}{r}=-k\frac{e^2}{2r}.
\end{equation}
\subsubsection{Andre postulat} 
Vi har tidligere sett Plancks og Einsteins kvantiseringspostulat
for e.m.~str\aa ling $E=nh\nu$. Bohr antok, basert p\aa\ at elektronet
kun kunne bevege seg i bestemte baner, at det var banespinnet til
elektronet (bane angul\ae rt moment) som var kvantisert, dvs.~at banespinnet 
kunne
anta bare diskrete verdier
\begin{equation}
   |{\bf L}|=m_evr=n\hbar.
\end{equation}
med $n=1, 2, 3,\dots$
Fra  likning (\ref{eq:vn}) har vi at 
\begin{equation}
   v=\sqrt{\frac{ke^2}{m_er}},
\end{equation}
som gir
\begin{equation}
   m_evr=\sqrt{ke^2m_er}=n\hbar.
\end{equation}
Kvadrerer vi siste likning har vi definisjonen p\aa\ Bohrradier
\begin{equation}
   r_n=\frac{\hbar^2n^2}{m_eke^2}=a_0n^2,
   \label{eq:bohrradius}
\end{equation}
hvor $a_0$ kalles {\em Bohrradien}, og danner en viktig lengdeskala
i atomfysikk, faste stoffers fysikk og molekylfysikk. Denne minste
Bohrradien har verdi
\begin{equation}
   a_0=\frac{\hbar^2}{m_eke^2}=0.529\times 10^{-10} \hspace{0.1cm}\mathrm{m}.
\end{equation}

Vi ser av uttrykket for Bohrradiene at de tillatte radiene antar
diskrete verdier diktert av verdien p\aa\ heltallet $n$, som {\em vi kaller
et kvantetall}.

Setter vi uttrykket for $r_n$ i det for energien til elektronet finner vi
\begin{equation}
   E=E_n=-k\frac{e^2}{2r_n}=-k\frac{e^2}{2a_0}\frac{1}{n^2}=-13.6\frac{1}{n^2} \hspace{0.1cm}\mathrm{eV}. \label{eq:energyradius}
\end{equation}

{\em Det viktige budskapet her er at 
energikvantisering f\o lger n\aa\ fra postulatet om 
kvantisering av banespinn.}
Vi kan generalisere uttrykket for energien, radien osv.~til \aa\ gjelde
tyngre atomer ved \aa\ erstatte ladningen til et enkelt proton
$+e$ med den til $Z$ protoner. Da f\aa r vi
\begin{equation}
   E_n=-k\frac{Z^2e^2}{2a_0}\frac{1}{n^2}=-13.6\frac{Z^2}{n^2} \hspace{0.1cm}\mathrm{eV},
\end{equation}
\begin{equation}
   r_n=\frac{\hbar^2n^2}{m_ekZe^2}=a_0\frac{n^2}{Z},
\end{equation}
og
\begin{equation}
   v_n=\frac{n\hbar}{m_er_n}=\frac{kZe^2}{n\hbar}.
\end{equation}
Dersom vi velger $n=1$ finner vi 
\begin{equation}
   v_1=\frac{kZe^2}{\hbar},
\end{equation}
og med innsetting av lyshastigheten 
har vi
\begin{equation}
   v_1=\frac{ckZe^2}{\hbar c}.
\end{equation}
N\aa\ er $ke^2=1.44$ eVnm, $\hbar c=197$ eVnm, slik at
\begin{equation}
   v_1\approx Z\times 2.2\times 10^6 \hspace{0.1cm}\mathrm{m/s},
\end{equation}
og for lette atomer er hastigheten kun noen f\aa\ prosent av 
lyshastigheten. 
Dette forteller oss at Bohrs atommodell, som er basert p\aa\ ikke
relativistisk teori, kun kan anvendes for lette atomer.
For $Z\sim 100$, begynner elektronets hastighet \aa\ n\ae rme seg
lysets.
Vi avslutter dette avsnittet med definisjonen av den s\aa kalt
finstrukturkonstanten
                \begin{equation}
                  \alpha=\frac{e^2}{4\pi\epsilon_0\hbar c}\approx 1/137,
                \end{equation}
som vi kan bruke til \aa\ skrive om uttrykkene til Bohrradien 
\begin{equation}
   a_0=\frac{\hbar}{\alpha m_ec},
\end{equation}
og energien 
\begin{equation}
E_n=-\frac{\alpha^2}{2n^2}m_ec^2,
\label{eq:ealpha}
\end{equation}
som likner p\aa\ det velkjente uttrykket $E=m_ec^2$!

Setter vi inn for ulike $n$-verdier finner vi 
$E_1=-13.6$ eV, $E_2=-3.4 $ eV, $E_3=-1.5 $ eV osv., i bra samsvar 
med de eksperimentelle verdiene.

\subsection{Tredje postulat} 

Hva tredje postulat ang\aa r, s\aa\ tar det utgangspunkt i det 
faktum at atomer tross alt er stabile. Vi skal se senere
i v\aa r diskusjon av kvantemekanikken at vi kan, om enn ikke
forklare hvorfor atomer er stabile, ihvertfall lage
en teori hvor formalismen gir som resultat at det er stor
sannsynlighet for at atomer forblir stabile. 



\subsection{Fjerde postulat} 
N\aa\ kan vi tenke oss et elektron som er i en orbital $n_i$ med
energi $E_i$ g\aa r spontant til en orbital $n_f$ hvor
$n_f < n_i$. 
I denne prosessen sendes det ut e.m.~str\aa ling og
siden energien til $E_i \ne E_f$, s\aa\ betyr det at energibevaring krever
\begin{equation}
   E_i=E_f+E_{\gamma}=E_f+h\nu.
\end{equation}   
Den utsendte e.m.~str\aa ling er gitt ved 
\begin{equation}
   \Delta=h\nu=E_i-E_f,
\end{equation}
dvs.\
\begin{equation}
  \nu=\frac{(E_i-E_f)}{h}=\left(\frac{1}{4\pi\epsilon_0}\right)^2\frac{m_eZ^2e^4}{4\pi\hbar^3}\left(\frac{1}{n_f^2}- \frac{1}{n_i^2}\right)
\label{eq:emisjon}
\end{equation}
Dersom vi n\aa\ 
bruker at $1/\lambda=\nu/c$ og 
setter inn tallverdier for de ulike konstantene og velger $Z=1$ (hydrogenatomet) finner vi
\begin{equation}
  \frac{1}{\lambda}=R_H\left(\frac{1}{n_f^2}- \frac{1}{n_i^2}\right),
   \label{eq:trans}
\end{equation}
med 
\begin{equation}
   R_H=\left(\frac{1}{4\pi\epsilon_0}\right)^2\frac{m_ee^4}{4\pi\hbar^3c}=
        \frac{ke^2}{2a_0hc},
   \label{eq:rh}
\end{equation}
som har samme form som Rydbergs formel i likning (\ref{eq:ryd}).
Setter vi deretter inn numeriske verdier finner vi
\begin{equation}
   R_H=1.0973732\times 10^7\hspace{0.1cm}\mathrm{m}^{-1},
\end{equation} 
mens den eksperimentelle verdien for hydrogenatomet er 
$R_H=1.0967757\times 10^7$ m$^{-1}$. Vi har alts\aa\ en teori
som reproduserer tilpasningen til de eksperimentelle spektrallinjene
gitt i likning (\ref{eq:ryd}). I neste avsnitt skal vi se at n\aa r vi
tar hensyn  til protonets masse vil verdien av $R_H$ n\ae rme
seg den eksperimentelle.

P\aa\ tilsvarende vis kan vi studere absorpsjon av e.m.~str\aa ling,
en prosess hvor innkommende e.m.~str\aa ling
eksiterer et elektron til en h\o yere liggende energitilstand.
Energibevaring gir 
\begin{equation}
   h\nu+E_i=E_f
\end{equation}
og den inverse b\o lgelengde blir da
\begin{equation}
  \frac{1}{\lambda}=R_H\left(\frac{1}{n_i^2}- \frac{1}{n_f^2}\right).
\end{equation}
 
Til slutt skal vi nevne at vi dersom $n\rightarrow \infty$ f\aa r vi
$E_{\infty}=0$. Dette svarer igjen til at avstanden mellom elektronet
og protonet er uendelig stor. Vi sier da at elektronet er fritt.
Den energien som trengs for \aa\ l\o srive et elektron fra
hydrogenkjernen kalles ionisasjonsenergien og er gitt ved
\begin{equation}
   E_{\infty}-E_1=0-(-13.6)=13.6 \hspace{0.1cm}\mathrm{eV}.
\end{equation}


\subsection{Tyngdepunktskorreksjon}

Vi har sett at massen til elektronet $m_e$ er  mye mindre enn massen
til protonet $m_p$, $m_p\sim 2000 m_e$. Dersom vi n\aa\ tar utgangspunkt
i bevegelse om et felles tyngdepunkt for protonet og elektronet 
s\aa\ trenger vi \aa\ introdusere en relativ masse
for hydrogenatomet $\mu_H$
\begin{equation}
    \mu_H=\frac{m_em_p}{m_e+m_p}=m_e\frac{1}{1+\frac{m_e}{m_p}}=0.99946m_e.
\end{equation}
Dette virker som en ubetydelig korreksjon. Men det 
er n\aa\ slik at 
spektroskopiske data er sv\ae rt n\o yaktige. Dersom vi f.eks.~
studerer n\ae rmere  uttrykket for $R_H$ i likning (\ref{eq:rh}) 
ser vi at det avhenger av massen til elektronet.
Tar vi forholdet mellom den utrekna verdien og den eksperimentelle
for $R_H$ finner vi
\begin{equation}
   1.0973732\times 10^7/1.0967757\times 10^7=1.000545 
\end{equation}
Verdien av $\mu_H=0.99946m_e$ gir oss dermed et hint om mulige
korreksjoner til den teoretiske verdien av $R_H$. 
La oss studere denne mulige korreksjonen i n\ae rmere detalj.

Hittil har vi alts\aa\ tenkt at siden elektronet er mye lettere enn protonet,
kan vi approksimere systemet til et elektron som kretser rundt et 
tyngdepunktsenter, som er protonet. Dette senteret definerer origo.

La oss heller betrakte systemet som et to-legeme
system, hvor vi ser p\aa\ tyngdepunktsenteret til dette to-legeme systemet
som det nye origo.
Vi definerer da en relativ avstand 
\begin{equation}
   {\bf r}={\bf r_e}-{\bf r_p},
\end{equation}
hvor   ${\bf r_e}$ er elektronets avstand til det nye origo og
${\bf r_p}$ den tilsvarende for protonet. 
Siden origo er lokalisert i massesenteret m\aa\ vi tilfredsstille f\o lgende
likning
\begin{equation}
    m_e {\bf r_e} + m_p{\bf r_p}=0.
\end{equation}

De to siste likningene hjelper oss \aa\ finne et uttrykk for 
\begin{equation}
    {\bf r_e}=\frac{m_p}{m_e+m_p}{\bf r},
\end{equation}
og 
\begin{equation}
    {\bf r_p}=-\frac{m_e}{m_e+m_p}{\bf r}.
\end{equation}
Bruker vi deretter definisjonen p\aa\ hastighet $v=dr/dt$ finner vi
\begin{equation}
    {\bf v_e}=\frac{m_p}{m_e+m_p}{\bf v},
\end{equation}
og 
\begin{equation}
    {\bf v_p}=-\frac{m_e}{m_e+m_p}{\bf v}.
\end{equation}
Vha.~de to uttrykkene for hastighetene kan vi uttrykke b\aa de kinetisk
energi og banespinnet for protonet og elektronet. Den kinetiske
energien til dette to-legeme systemet blir da
\begin{equation}
   E_{kin}=\frac{1}{2}m_ev_e^2+\frac{1}{2}m_pv_p^2=
            \frac{1}{2}\frac{m_em_p}{m_e+m_p}v^2,
\end{equation}
eller 
\begin{equation}
   E_{kin}=\frac{1}{2}\mu_Hv^2,
\end{equation}
hvor $v$ er den relative hastighet gitt ved 
$   {\bf v}={\bf v_e}-{\bf v_p}$ og $\mu_H$ er den reduserte
massen til et system som best\aa r av et proton og et elektron.

P\aa\ tilsvarende vis kan vi uttrykke det totale banespinnet som
\begin{equation}
   L= m_er_ev_e + m_pr_pv_p=m_e\left(\frac{m_p}{m_e+m_p}\right)^2vr+
      m_p\left(\frac{m_e}{m_e+m_p}\right)^2vr,
\end{equation}
eller 
\begin{equation}
   L=\mu_Hvr.
\end{equation}
Bruker vi atter en gang Bohrs kvantiseringspostulat p\aa\ banespinnet, finner
vi at Bohrradiene $r_n$ er gitt ved 
\begin{equation}
    r_n=\frac{4\pi\epsilon_0\hbar^2}{e^2\mu_H}n^2.
\end{equation}

Dette uttrykket skiller seg fra likning (\ref{eq:bohrradius}) 
ved at vi har den reduserte massen istedet for elektronets masse.
Multipliserer vi oppe og nede med massen til elektronet finner vi
\begin{equation}
    r_n=\frac{m_e4\pi\epsilon_0\hbar^2}{m_ee^2\mu_H}n^2=
    \frac{m_e}{\mu_H}a_0n^2,
\end{equation}
eller
\begin{equation}
   r_n=\frac{m_e}{\mu_H}a_0\frac{n^2}{Z},
\end{equation}
hvor Bohrradien $a_0$ er gitt ved
$a_0=\hbar^2/m_eke^2$. I tillegg 
har vi introdusert ladningen til kjernen ved $Z$. For hydrogenatomet har
vi $Z=1$.

I likning (\ref{eq:energyradius}) ga vi et uttrykk for energien 
vha.~Bohrradiene $r_n$. Setter vi inn det nye uttrykket for $r_n$ finner vi
at energien kan skrives som 
\begin{equation}
   E_n=-k\frac{\mu_H}{m_e}\frac{Ze^2}{2a_0}\frac{1}{n^2}=-13.6\frac{\mu_H}{m_e}\frac{Z}{n^2} \hspace{0.1cm}\mathrm{eV},
\end{equation}
som da gir oss den nye Rydbergkonstanten $R_H'$ for hydrogenatomet
gitt ved
\begin{equation}
   R_H'=\frac{\mu_H}{m_e}R_H=0.99946R_H=1.096781\times 10^7\hspace{0.1cm}\mathrm{m}^{-1},
\end{equation}
n\ae rmere den eksperimentelle verdien.

En interessant anvendelse av tyngdepunktskorreksjon var oppdagelsen
av deutronet. Deutronet best\aa r av et proton, et n\o ytron og et elektron.
N\o ytronet har omtrent samme masse som protonet. Setter vi massen
til deutronets kjerne lik $2m_p=2\times 938$ MeV/c$^2$, finner vi at
den reduserte massen til deutronet (med et elektron er)
\begin{equation}
   \mu_D=\frac{m_e2m_p}{m_e+2m_p}=m_e\frac{1}{1+\frac{m_e}{2m_p}}=
0.99973m_e.
\end{equation}
En observerte at for $n_f=2$ og $n_i=3$ overgangen i Balmerserien,
den s\aa kalt $H_{\alpha}$ linjen s\aa\ fantes det to b\o lgelengder,
en gitt ved $\lambda=656.1$ nm som vi allerede har diskutert og en
ukjent ved $\lambda=656.3$ nm.
Dersom vi bruker likning (\ref{eq:trans}) og korrigerer for tyngdepunktskorreksjoner for henholdsvis hydrogenatomet og deutronet finner vi
for hydrogenatomet
\begin{equation}
  \frac{1}{\lambda_H}=\frac{\mu_H}{m_e}R_H\left(\frac{1}{2^2}- \frac{1}{3^2}\right),
\end{equation}
og for deutronet 
\begin{equation}
  \frac{1}{\lambda_D}=\frac{\mu_D}{m_e}R_H\left(\frac{1}{2^2}- \frac{1}{3^2}\right).
\end{equation}
Ser vi p\aa\ forholdet 
\begin{equation}
    \frac{\lambda_D-\lambda_H}{\lambda_H}=
    \frac{\mu_H-\mu_D}{\mu_D}=\frac{\mu_H}{\mu_D}-1,
\end{equation}
som skal gi oss forskjellen i b\o lgelengde. Setter vi inn tallverdier finner
vi 
\begin{equation}
   \frac{\mu_H-\mu_D}{\mu_D}=\frac{m_e(0.99946-0.99973)}{m_e0.99973}=-0.00027,
\end{equation}
som svarer til ei forandring p\aa\ $0.027\%$. Dette gir oss ei forandring
i b\o lgelengde mellom hydrogenatomet og deutronet for denne spesielle
overgangen p\aa\  0.177 nm, i bra samsvar med observerte verdier.
Deutronet ble oppdaga i 1932 av Urey, 3 \aa r etter oppdagelsen
av n\o ytronet. 
\subsection{Korrespondanseprinsippet}

Bohr formulerte korrespondanse prinsippet slik:
       \begin{itemize}
          \item Kvanteteoriens forutsigelser om oppf\o rselen
                til et fysisk system skal svare til de for et klassisk system i grensa hvor kvantetallene som bestemmer en tilstand
blir veldig store. 
      \end{itemize}
Teknisk sett betyr det at den klassiske grense skal n\aa s n\aa r kvantetallene
blir 'store', f.eks.~ved \aa\ la kvantetallet $n$ i Bohrs modell
bli stort. (Senere i dette kurset skal vi se at kvantemekanikken som teori
inneholder automatisk klassisk fysikk som et grensetilfelle.)
Vi kan vise dette prinsippet med et enkelt eksempel. La oss se p\aa\
frekvensen for en overgang fra en tilstand med kvantetall $n+1$ til en
med kvantetall $n$.  Bruker vi likning (\ref{eq:emisjon}) 
\begin{equation}
  \nu=\frac{(E_{n+1}-E_n)}{h}=\frac{m_ec^2}{4\pi \hbar}(Z\alpha)^2\left(\frac{1}{n^2}- \frac{1}{(n+1)^2}\right),
\end{equation}
ser vi at n\aa r $n$ er stor kan vi approksimere frekvensen med
\begin{equation}
   \nu\approx \frac{m_ec^2}{2\pi \hbar}(Z\alpha)^2\frac{1}{n^3}.
    \label{eq:klassiskbohr}
\end{equation}
Det klassiske uttrykket for frekvensen er gitt fra FY-ME100 som
\begin{equation}
   \nu_{klassisk}=\frac{v}{2\pi r}.
\end{equation}
Setter vi s\aa\ inn uttrykket for Bohrradien har vi
\begin{equation}
   \nu_{klassisk}=\frac{Z\alpha m_ec}{n}\frac{Z\alpha c}{2\pi n^2\hbar}=
 \frac{m_ec^2}{2\pi \hbar}(Z\alpha)^2\frac{1}{n^3},
\end{equation}
som svarer til likning (\ref{eq:klassiskbohr}). Legg merke til at det er
bare overganger av typen $n+1\rightarrow n$ som gir det klassiske
resultat. Str\aa ling som svarer til en overgang av typen 
$n+2\rightarrow n$ har ikke noen klassisk analog. 

\subsection{Franck-Hertz eksperimentet}

I 1914 utf\o rte Franck og Hertz et eksperiment som var 
den f\o rste bekreftelse p\aa\ eksistensen av stasjon\ae re tilstander
i atomer, dvs.~at kun bestemet eksiterte tilstander var mulig,
i motsetning til det kontinuerlige spekteret av tilstander 
som klassik fysikk ga. 
For dette fikk Franck og Hertz Nobelprisen i fysikk i  1925. Et eksperimentelt
oppsett er vist i Figur \ref{fig:fh1}. Eksperimentet ble gjennomf\o rt
ved \aa\ akselerere elektroner gjennom en gass 
av kvikks\o lv vha.~en p\aa satt spenning mellom katoden og anoden. 
Etter som elektronene 
kolliderer med kvikks\o lvatomene taper de kinetisk energi. For \aa\ kunne 
n\aa\ anoden og for at en skulle kunne m\aa le en str\o m av elektroner
m\aa\ elektronene ha en bestemt kinetisk energi. \O kes spenningsforskjellen,
vil flere og flere elektroner n\aa\ anoden, og dermed skulle en registrere en
monotont \o kende str\o m av elektroner. 
\begin{figure}[h]
\begin{center}
{\centering
\mbox
{\psfig{figure=franckhertz.ps,height=6cm,width=12cm}}
}
\end{center}
\caption{Skjematisk oppsett for Franck og Hertz sitt eksperiment med
kvkks\o lv i 1914. Et r\o r med tre elektroder fylles med kvikks\o lvdamp. Elektroner aksellereres over et spenningsfall $V$ mellom katoden og et gitter. Et mindre spenningsfall med motsatt polaritet settes opp mellom gitteret og anoden. Str\o mmen av elektroner gjennom r\o ret avhenger av hvordan de vekselvirker med kvikks\o lv.\label{fig:fh1}}
\end{figure}
\begin{figure}[t]
\begin{center}
{\centering
\mbox
{\psfig{figure=franckhertz2.ps,height=6cm,width=12cm}}
}
\end{center}
\caption{Str\o mmen av elektroner gjennom r\o ret m\aa les som funksjon av det aksellererende 
potensialet $V$. Flere og flere elektroner n\aa r anoden etter som spenningen
\o kes. Ved $V\approx 4.9$ V, $2\times 4.9$ V, $3\times 4.9$ V osv.~avtar str\o mmen. 
Det svarer til at elektronene
taper 4.9 eV per kollisjon.\label{fig:fh2}}
\end{figure}
Dette var ikke tilfelle eksperimentelt. Et plott av elektrisk str\o m 
som funksjon av det akselererende potensialet viste topper ved heltalls
verdier av en spenning p\aa\ ca 4.9 V, se Figur \ref{fig:fh2}.  
Det som skjer er at ved bestemte
kinetiske energier (og dermed spenninger) for elektronene, har elektronene
en energi som svarer til den f\o rste eksiterte tilstanden i 
kvikks\o lvatomet, se Figur \ref{fig:fh3}. 
N\aa r elektronene dermed overf\o rer sin kinetiske energi til 
kvikks\o lvatomet, bremses det ned og f\ae rre elektroner n\aa r anoden. 
Derfor faller str\o mmen for spenningsfall over 4.9 V.
\O kes spenningen over 4.9 V, \o ker str\o mmen igjen til elektronene har
en kinetiske energi p\aa\
\[
   E_{kin}= 2\times 4.9 \hspace{0.1cm} \mathrm{eV}.
\]
Da har elektronet energi nok til \aa\ eksitere to kvikks\o lvatomer og en 
observerte en ny topp i elektronstr\o mmen. 

N\aa r kvikks\o lvatomet g\aa r tilbake til grunntilstanden, sendes
det ut elektromagnetisk str\aa ling med b\o lgelengde p\aa\ 
$\lambda=254$ nm.
Det svarer til en energiforksjell gitt ved 
\begin{equation}
  \Delta E=\frac{hc}{\lambda}=
   \frac{1240\hspace{0.1cm} \mathrm{eVnm}}{254\hspace{0.1cm} \mathrm{nm}}=
           4.88 \hspace{0.1cm} \mathrm{eV},
\end{equation}
i samsvar med den m\aa lte spenningen og dermed den kinetiske energi som
elektronene har.


\subsection{Problemer med Bohrs atommodell}
Bohrs atommodell var et viktig steg i retning mot kvantemekanikken.
Det faktum at en teoretisk kunne forklare spektrene til atomer var
en aldri s\aa\ liten revolusjon. Men modellen hadde klare svakheter
etterhvert som den ble konfrontert med nye eksperimentelle data.

Modellen var i stand til \aa\ forklare flere egenskaper ved alkalimetallene,
slik som litium, natrium, kalium, rubidium og cesium. Dette er alle
metaller hvis kjemiske egenskaper henger n\o ye sammen med det faktum at de
alle har et valenselektron. Vi kan da tenke oss at de svarer til modifiserte
hydrogenatomer, hvor frihetsgradene til kjernene og de andre elektronene
er absorbert i ladningen $Z$. Men som vi skal se i v\aa r gjennomgang av
det periodiske systemet, kunne ikke alle egenskapene til 
alkaliatomene 
forklarers vha.~Bohrs atommodell. For atomer med flere valenselektroner
var samsvaret mellom teori og eksperiment d\aa rlig. 

I tillegg, inkluderte Bohrs atommodell ad hoc postulater slik som
postulatet om kvantiseringen av banespinnet, 
som i det lange l\o p ikke var intellektuelt
tilfredsstillende. 
\begin{figure}[t]
\begin{center}
{\centering
\mbox
{\psfig{figure=franckhertz3.ps,height=4cm,width=10cm}}
}
\end{center}
\caption{Eksitasjonsenergien fra grunntilstanden til den f\o rste eksiterte tilstanden svarer til at et elektron i et kvikks\o lvatom har f\aa tt tilf\o rt en energi p\aa\ $4.9$ eV.
Ved henfall tilbake til grunntilstanden sendes det ut e.m.~str\aa ling med b\o lgelengde
  $\lambda=\frac{hc}{\Delta E}=254$ nm.
\label{fig:fh3}}
\end{figure}
\section{Oppgaver}
\subsection{Analytiske oppgaver}
\subsubsection*{Oppgave 2.1}
Synlig lys har b{\o}lgelengder i omr{\aa}det 4000--7000 {\AA}.
%
\begin{itemize}
%
\item[a)] Fra Bohrs formel for energiniv{\aa}ene i H--atomet, vis at
b{\o}lgelengden til
det emitterte lyset vil ligge utenfor det synlige omr{\aa}det ved alle
overganger til laveste niv{\aa}~(Lyman--serien).
%
\item[b)] Hva blir den korteste og den lengste b{\o}lgelengden
ved overganger til
det nest laveste niv{\aa}et (Balmer--serien)?

\item[c)] Hvor mange spektrallinjer ligger i det synlige omr{\aa}det i denne
serien?
%
\end{itemize}



\subsubsection*{Oppgave 2.2}
I den enkleste versjonen av Bohrs atommodell antas det at elektronet
beveger seg
i sirkul{\ae}re baner rundt en atomkjerne som ligger i ro,
d.v.s. at den har en
masse som regnes {\aa} v{\ae}re uendelig stor. Spesielt for
de letteste atomene er
dette ikke en s{\ae}rlig god antagelse. I H--atomet best{\aa}r
kjernen kun av et
enkelt proton med en masse $m_{p}$ som er 1836 ganger tyngre
enn elek\-tron\-massen $m_{e}$.
%
\begin{itemize}
%
\item[a)] Ved {\aa} ta med bevegelsen til protonet i H--atomet,
vis at den totale
banespinnet for atom\-et i dets massesentersystem er
$L = \mu \omega r^{2}$,
hvor r er avstanden mellom elektronet og protonet, $\omega$ er
vinkelfrekvensen
til deres sirkul{\ae}re bevegelse og $\mu$ er deres reduserte masse
\[
\mu = \frac{m_{e} m_{p}}{m_{e} + m_{p}}.
\]
%
\item[b)] Finn energiene til de stasjon{\ae}re
tilstandene til atomet ved bruk av
Bohrs kvantiseringsbetingelse $L = n \hbar$, og vis at dette mer
n{\o}yaktige
resultatet er identisk med Bohrs energiformel med det unntak at den reduserte
massen $\mu$ n{\aa} inng{\aa}r i stedet for elektronmassen $m_{e}$.
%
\item[c)] Hvor mye forandres den lengste b{\o}lgelengden $H_{\alpha}$
($ n= 3 \longrightarrow n = 2$) i Balmer--serien if{\o}lge
denne mer korrekte formelen?
%
\item[d)] Spektrallinjen $H_{\alpha}$ fra deuterium (tungt hydrogen) har
b{\o}lgelengden $\lambda = $ 656,029 {\AA}. Finn massen
til atomkjernen i deuterium.
%
\end{itemize}
%

\subsubsection*{Oppgave 2.3}
\begin{itemize}
%
\item[a)] Hvilke energier har lyskvant som faller i den
synlige delen av spektret,\\
4000 {\AA} $\leq \lambda \leq $ 7000 {\AA}?
%
\item[b)] Oppgi noen spektrallinjer av atom{\ae}rt hydrogen
og enkeltionisert helium som tilsvarer synlig lys.
%
\item[c)] Finn st{\o}rrelsesorden av effekten p{\aa} disse
dersom man tar hensyn til rekylen av atomet under emisjonen,
og vis at denne effekten er langt mindre enn
isotopeffekten (effekt p.g.a. redusert masse).
%
\end{itemize}

\subsubsection*{Oppgave 2.4}
Et atom med masse M, opprinnelig i ro, emitterer et foton ved en overgang fra et
energiniv\aa ~$E_1$ til et annet niv\aa ~$E_2$. Det sies vanligvis at det
emitterte fotonets energi er gitt ved $h \nu = E_1 - E_2$, der $\nu$ er
fotonets frekvens og h er Plancks konstant. Imidlertid vil en liten del av
energien overf\o res til atomet som kinetisk energi (rekylvirkning), og fotonet
f\aa r derfor en tilsvarende mindre energi: $h \nu = E_1 - E_2 - \Delta E$.
%
\begin{itemize}
%
\item[a)] Finn et uttrykk for denne korreksjonen $\Delta E$. G\aa ~ut fra at
fotonet har en bevegelsesmengde lik $ h \nu / c$.

\item[b)] Gjennomf\o r en tilsvarende diskusjon for det tilfellet at atomet
 absorberer et foton.

\item[c)] Regn ut den numeriske verdien av $\Delta E / h \nu $ for
en overgang $E_1 - E_2 = 4,86$ eV i et kvikks\o lvatomet
som har masse lik  200 $\times $ massen til et proton.
%
\end{itemize}

\subsubsection*{Oppgave 2.5}

Bruk Bohrs kvantiseringsbetingelse  $L = n \hbar$ til � kvantisere energien av en partikkel
i et sentral potensial $V(r) = V_0 (r/a)^k$. Beregn den kvantiserte energien som 
funksjon av kvantetallet $n$. Kontroll\'{e}r $n$-avhenigheten for $k = -1$ som svarer til 
Bohrs atom modell for hydrog\'{e}n. Lag en figur som viser $V(r)$ for store verdier av $k$.
Forklar ut fra figuren verdien av elektronets bane-radius for $k \rightarrow \infty$. 
%


\clearemptydoublepage
\chapter{MATERIEB\O LGER}
\begin{figure}[h]
\begin{center}
{\centering
\mbox
{\psfig{figure=ride.ps,height=6cm,width=8cm}}
}
\end{center}
\caption{Dobbeltsspalt eksperiment. Nylig, se 
Phys.~Rev.~Lett.~{\bf 82} (1999) 2868, klarte fysikere \aa\ observere
interferens fenomen mellom b\o lgepakker som besto av kun to fotoner.}
\end{figure}
Hensikten med dette kapitelet, i tillegg til \aa\
diskutere de Broglies hypotese, er \aa\ gi en introduksjon
til viktige deler av b\o lgel\ae ren som vil f\o lge oss i
store av deler av kurset. B\o lgel\ae ren er i seg selv
et omfattende fagfelt, men vi skal i all hovedsak fokusere
p\aa\ de deler som er av interesse for kurset.
Dette dreier seg om interferens, diffraksjon, og gruppe
og fasehastighet. Dette er begrep som var sentrale 
i eksperiment som avdekket b\o lgeegenskapen til materien. 
For \aa\ forst\aa\ disse resultatene og \aa\ kunne relatere
dem til de Broglies hypotese, Heisenbergs uskarphetsrelasjon,
materiens b\o lge og partikkel egenskaper og Schr\"odingers
likning, trenger vi noen resultater fra b\o lgel\ae ren.
I tillegg, og kanskje minst like viktig, vil v\aa r diskusjon
av kvantemekanikken basere seg p\aa\ b\o lgeteori.
Et viktig aspekt i denne vektleggingen av b\o lgel\ae re
ligger i det rent pedagogiske, i den forstand at dere
vil se en fortsettelse og utvidelse 
av begrep fra e.m.~teori som dere har sett i kurs som FY101.
Sett utifra et slikt perspektiv kan en ogs\aa\ si at
bildet basert p\aa\ b\o lgeteori som vi kommer til
\aa\ vektlegge er kanskje mer fysisk intuitivt ved et 
introduksjonskurs som FYS2140. 
I videreg\aa ende kurs som FYS 201, FYS 303, FYS 305 og FYS 403,
 vil dere f\aa\ en introduksjon til
en mer moderne matematisk presentasjon av kvantemekanikken. 

Etter avsnittet med b\o lgel\ae re, f\o lger en diskusjon
av Heisenbergs uskarphetsrelasjon. Men f\o rst til et eksperimentelt
faktum som ikke har noen klassisk analogi. Her ligger ogs\aa\
det eneste 'mysteriet' med kvantemekanikken. 

\section{Materiens b\o lge og partikkelnatur}

Mer eller mindre vet vi hvordan dagligdagse gjenstander oppf\o rer seg
under p\aa virkninger fra ulike krefter. P\aa\ mikroniv\aa\ er dog v\aa rt 
erfaringsgrunnlag begrenset. Det vi skal ta for oss n\aa\ er et eksperimentelt
faktum uten sidestykke i den makroskopiske virkelighet, slik vi  kjenner den
fra v\aa rt erfaringsgrunnlag.
Materien, dvs.~elektroner, n\o ytroner, atomer m.m.~utviser ikke bare partikkel
egenskaper men ogs\aa\ b\o lgeegenskaper. Elektronene oppf\o rer seg akkurat
som lys. Det h\o rer ogs\aa\ med til historien at Newton
trodde lys besto av partikler, men da b\o lgeegenskapene til lys  ble avdekka
i mange eksperiment, ble denne tr\aa den forlatt, inntil Einsteins
radikale bruk av Plancks postulat for \aa\ forklare den fotoelektriske
effekt. Historisk ble elektronet tiltenkt kun partikkel egenskaper. 
Men ulike eksperiment med elektroner avdekket ogs\aa\ b\o lgeegenskaper. 
 
Det vi skal studere i dette kapitlet er alts\aa\ materiens b\o lgeegenskaper.
Vi kan ikke forklare dette vha.~klassisk teori. Det danner ogs\aa\ grunnlaget
for kvantemekanikken som teori og to av de viktige postulatene om naturen vi
nevnte innledningsvis, Heisenbergs uskarphetsrelasjon og de Broglies postulat
om materiens b\o lgeegenskaper. Kvantemekanikken som teori baserer seg p\aa\ disse postulatene ved \aa\ ta utgangspunkt i en b\o lgebeskrivelse av naturen\footnote{
Schr\"odingers likning som er v\aa r naturlov, er ikke noe annet enn en
b\o lgelikning, som, dersom vi tillater oss abstraksjonen med imagin\ae r tid,
kan skrives som en diffusjonlikning. Varmeledning er et klassisk 
eksempel p\aa\ et fysisk problem som kan modelleres vha.~en diffusjonslikning}. 
Vi er ikke i stand til \aa\ forklare {\em hvorfor} materien utviser slike
egenskaper. V\aa r innsikt begrenser seg {\em kun} til \aa\ forklare  
hva som skjer. Sagt litt annerledes, ved \aa\ fortelle deg hva som skjer,
vil vi ogs\aa\ avdekke kvantemekanikkens grunnleggende merkverdigheter.
Med siste ord mener vi egentlig sider ved v\aa r forst\aa else av naturen
som savner sidestykker i v\aa re erfaringsgrunnlag fra verden p\aa\
makroniv\aa\ . Du vil selvsagt sitte igjen med mange  store sp\o rsm\aa l
om hvorvidt v\aa r beskrivelse av den fysiske virkelighet p\aa\ mikroniv\aa\
virkelig f\o lger f.eks.~uskarphetsrelasjonen. I skrivende stund har vi ikke
en eneste eksperimentell observasjon p\aa\ brudd med de grunnleggende 
postulatene. 
Kanskje er uskarphetsrelasjonen en fundamental egenskap ved naturen?

Faktisk er det slik at idag tenker vi oss at vi kan bruke
sider av kvantemekanikken som er h\o yst ikke-trivielle 
i f.eks.~utviklingen av kvantedatamaskiner eller unders\o kelser av
systemer p\aa\ mikroniv\aa\ . Idag er vi f.eks.~istand til \aa\
fange inn  enkeltelektroner (kunstig hydrogenatom) i omr\aa der
p\aa\ noen f\aa\ nanometre. Eksperiment har blitt gjort
hvor en til og med kan lage s\aa kalte Schr\"odinger 
katt tilstander ved \aa\ manipulere enkeltioner i ionefeller vha.~laserlys. 
Bruk av 
entanglement\footnote{En mulig norsk oversettelse er sammenfiltrede tilstander.}
av kvantemekaniske systemer for sikker 
kryptering, rask s\o king i store databaser og teleportasjon er begrep
som bare for f\aa\ \aa r siden hadde et klart science fiction preg over seg.
Idag snakker vi om slike sider, om ikke med den st\o rste selvf\o lge, s\aa\
ihvertfall som spennende muligheter for \aa\ forst\aa\ naturen bedre. Men, 
mere om dette i neste kapittel. 

Tenk deg n\aa\ et eksperiment hvor vi sender inn f.eks.~en jevn str\o m
av tennisballer mot to spalte\aa pninger, slik som vist i Figuren nedenfor. 
Anta ogs\aa\ at m\aa ten disse tennisballene treffer spaltene er vilk\aa rlig.
P\aa\ baksiden av de to spalte\aa pningene har vi en detektor som m\aa ler hvor mange baller som kommer i et gitt omr\aa de. 
Anta ogs\aa\ at vi holder p\aa\ en stund slik at 
intensitetsfordelingen $I_{12}$ 
for  tennisballene (som uttrykker hvor mange
baller som kommer inn per tid per arealenhet) ser ut som vist p\aa\ 
figuren.  Siden vi ikke kan si helt n\o yaktig hvor ballene vil havne, 
uttrykker intensitetsfordelingen en sannsynlighet for hvor det er st\o rst 
eller minst sannsynlighet for at ballene treffer. 
\begin{figure}[htbp]
%
\begin{center}

\setlength{\unitlength}{1mm}
\begin{picture}(140,70)

\thicklines
                 
   \put(0,0.0){\makebox(0,0)[bl]{
              \put(0,0){\vector(1,0){50}}
              \put(-5,0){\circle*{50}}
              \put(0,0){\vector(4,1){50}}
              \put(0,0){\vector(4,-1){50}}
              \put(10,10){\makebox(0,0){$v$}}
              \put(85,-25){\makebox(0,0){skjerm}}
              \put(85,10){\makebox(0,0){$I_{12}$}}
              \put(115,25){\makebox(0,0){$I_{1}$}}
              \put(115,-25){\makebox(0,0){$I_{2}$}}
              \put(60,-20){\line(0,-1){10}}
              \put(60,-10){\line(0,1){20}}
              \put(60,20){\line(0,1){10}}
              \put(100,-30){\line(0,1){60}}
              \put(120,-30){\line(0,1){60}}
              \qbezier(100,-25)(80, 0)(100, 25)
              \qbezier(120,-25)(110, 0)(120, 10)
              \qbezier(120,-10)(110, 0)(120, 25)
         }}
\end{picture}
\end{center}
\caption{Tennisballer med hastighet $v$ sendes inn mot to spalt\aa pningen. P\aa\ baksiden har vi en detektor som registrerer hvor tennisballene treffer.
Den totale intensitetsfordelingen er gitt ved $I_{12}$.} 
\end{figure}

Dersom vi lukker en av spaltene, f.eks.~den andre, vil vi m\aa le en
intesitetsfordeling som vist i kurven $I_1$. Lukker vi spalt en, finner vi
en tilsvarende fordeling $I_2$.
Det vi skal merke oss her er at disse intensitetsfordelingene adderes opp, dvs.
\[
   I_{12}=I_1+I_2.
\]

La oss gjenta dette eksperimentet, men denne gang med lys. 
Anta at vi har en lyskilde som sender ensfarget lys med gitt b\o lgelendge
$\lambda$ mot samme type skjerm, som vist i Figur 
\ref{fig:fotonspalt}. Istedet for \aa\ m\aa le
antall baller som treffer bak spaltene har vi n\aa\ f.eks.~en fotografisk
plate som m\aa ler intensiteten til lyset som passerer de to spaltene.   
Vi kaller avstanden mellom spalte\aa pningene for $a$. Dersom forholdet mellom
$a$ og lysets b\o lgelendge 
(vi kommer til dette i de neste to avsnittene)  er gitt ved
\[
   a/\lambda\sim 1,
\]
kan vi observere et diffraksjonsm\o nster ved den fotografiske plata som vist
i Figur \ref{fig:fotonspalt}. 
Et diffraksjonsm\o nster utviser intensitetstopper og bunner.
Intensiteten $I^{\gamma}_{12}$ er n\aa\ forskjellig 
fra eksperimentet
v\aa rt med tennisballene.  Lukker vi n\aa\  den andre spalten, kan vi,
avhengig av forholdet mellom spaltens st\o rrelse og lysets b\o lgelengde
observere et diffraksjonsm\o nster eller en mer glatt kurve som vist 
i figuren med intensiteten $I^{\gamma}_1$. Lukker vi spalt en, ser vi en tilsvarende
intensitetsfordeling $I^{\gamma}_2$. 
\begin{figure}[htbp]
%
\begin{center}

\setlength{\unitlength}{1mm}
\begin{picture}(140,70)

\thicklines
                 
   \put(0,0.0){\makebox(0,0)[bl]{
              \put(0,0){\vector(1,0){50}}
              \put(-5,0){\circle*{50}}
              \put(0,0){\vector(4,1){50}}
              \put(0,0){\vector(4,-1){50}}
              \put(10,10){\makebox(0,0){$\lambda$}}
              \put(85,-25){\makebox(0,0){skjerm}}
              \put(85,10){\makebox(0,0){$I^{\gamma}_{12}$}}
              \put(115,25){\makebox(0,0){$I^{\gamma}_{1}$}}
              \put(115,-25){\makebox(0,0){$I^{\gamma}_{2}$}}
              \put(60,-20){\line(0,-1){10}}
              \put(60,-10){\line(0,1){20}}
              \put(60,20){\line(0,1){10}}
              \put(100,-30){\line(0,1){60}}
              \put(120,-30){\line(0,1){60}}
              \qbezier(100,-25)(95, -20)(100, -15)
              \qbezier(100,-15)(85, -10)(100, -5)
              \qbezier(100,-5)(70, 0)(100, 5)
              \qbezier(100,5)(85, 10)(100, 15)
              \qbezier(100,15)(95, 20)(100, 25)

              \qbezier(120,-25)(110, 0)(120, 10)
              \qbezier(120,-10)(110, 0)(120, 25)
         }}
\end{picture}
\end{center}
\caption{Lys med b\o lgelengde  $\lambda$ sendes inn mot to spalt\aa pningen. P\aa\ baksiden har vi en detektor som registrerer intensiteten.
Den totale intensitetsfordelingen er gitt ved $I^{\gamma}_{12}$.\label{fig:fotonspalt}} 
\end{figure}
Pr\o ver vi \aa\ addere de enkelte intensitetsfordelingene finner vi at
\[
   I^{\gamma}_{12}\ne I^{\gamma}_1+I^{\gamma}_2!
\]
Vi skal utlede det matematiske uttrykket i avsnittet om b\o lgel\ae ren.
Vi kan alst\aa\ ikke addere opp intensiteter for lys. Det at lys 
utviser et diffraksjonsm\o nster p\aa\ lik linje med hva f.eks.~vannb\o lger
gj\o r, f\o rte til forst\aa elsen av lys som b\o lger. 

Hva om vi erstatter lys med elektroner i Figur\ref{fig:fotonspalt}?   
Anta at du har en eller
annen kilde som sender elektroner med en gitt b\o lgelengde 
mot spaltene. I tillegg har vi en detektor ved skjermen som teller 
antall elektroner som passerer de to spaltene. Med begge spalter
\aa pne kan vi igjen observere en tilsvarende intensitets fordeling
som for lys. Elektronene utviser alts\aa\ b\o lgeegenskaper, i klar
strid med v\aa r intuitive oppfatning av partikler. 
Pr\o ver vi \aa\ lukke en av spaltene, 
ser vi en tilsvarende intensitetsfordeling som for lys. Legger vi sammen
intensitetsfordelingene 
kommer vi fram til samme
konklusjon, nemlig
\[
   I^{e^-}_{12}\ne I^{e^-}_1+I^{e^-}_2!
\]

Materien tilskrives  alts\aa\ b\o lgegenskaper, p\aa\ lik linje med 
fotonene.

La oss utvide eksperimentet med elektronene. 
Vi sender et elektron per sekund mot de to spaltene. Samtidig er
vi n\aa\ litt mer ambisi\o se. I tillegg til diffraksjonsm\o nsteret
\o nsker vi \aa\ finne ut fra hvilken spalt et enkelt elektron
gikk gjennom. Vi \o nsker \aa\ b\aa de studere b\o lgeengenskapene 
til elektronene (vi vil ha et diffraksjonsm\o nster) samtidig
som vi \o nsker \aa\ si noe om hvilken spalt det gikk gjennom,
dvs.~vi \o nsker \aa\ studere dets partikkelegenskaper.

Tenk deg da at vi plasserer en detektor ved spaltene.
Denne detektoren  sender ut fotoner som kolliderer 
med de innkommende elektronene, jfr.~Compton spredning. 
Detektoren er bygd slik 
at vi ser et skarpt lysglimt ved enten den ene eller andre spalten, 
avhengig
av hvor elektronet gikk gjennom. 
Ellers har eksperimentet v\aa rt samme oppsett som i 
Figur \ref{fig:fotonspalt}. 

Hva skjer n\aa\ ? Jo, vi observerer et lysglimt hver gang et elektron
passerer en av spaltene. Men, intensitetsfordelingen v\aa r utviser ikke
lenger et diffraksjonsm\o nster som vist i Figur \ref{fig:fotonspalt}! 
V\aa rt lille \o nske
om \aa\ studere b\aa de partikkel og b\o lgegenskaper {\em samtidig} i ett
og samme eksperiment er ikke mulig. 
Det var basert p\aa\ slike observasjoner at Heisenberg formulerte sin 
uskarphetsrelasjon. 
Per dags dato finnes det ikke noe eksperiment som bryter  med
denne relasjonen. 

Hensikten med dette kapittelet blir dermed \aa\ introdusere for dere
to viktige postulat om naturen, de Broglies hypotese om materiens
b\o lge og partikkelegenskaper og Heisenbergs uskarphetsrelasjon.


\section{De Broglies hypotese}    

I forrige
kapittel\footnote{Lesehenvisning her er kap 4-1 og 4-2.} 
diskuterte vi egenskapene til lys og
fant fra Comptons effekt, fotoelektrisk effekt og 
R\"ontgenstr\aa ling at e.m.~str\aa ling utviser b\aa de
partikkel og b\o lgeegenskaper. I noen tilfeller
kan vi avdekke b\o lgeegenskaper, i andre tilfeller ser vi
partikkelegenskapene.

Det er viktig \aa\ ha klart for seg skillet mellom partikler
og b\o lger siden de representer de eneste moder for transport
av energi. Klassisk har en partikkel
egenskaper som 
posisjon, bevegelsesmengde, kinetisk energi, masse og elektrisk
ladning. En klassisk b\o lge derimot beskrives ved st\o rrelser som
b\o lgelengde, frekvens, hastighet, amplitude, intensitet,
energi og bevegelsesmengde. Den viktigste forskjellen mellom
en klassisk partikkel og b\o lge er at partikkelen kan {\bf lokaliseres}, dvs.~har en skarpt definert posisjon, mens en b\o lge er spredt
utover rommet. Vi skal se mer av dette b\aa de n\aa r vi diskuterer
b\o lgel\ae re og i tilknytting Heisenbergs uskarphetsrelasjon,
og ikke minst i v\aa r diskusjon av kvantemekanikken.   

de Broglie tok utgangspunkt i resultatetene fra e.m.~str\aa ling og foreslo, utifra ideen om symmetrier i naturen, at ogs\aa\
materie, dvs.~partikler med endelig masse som f.eks.~elektroner,
protoner, atomer m.m.,utviser partikkel og b\o lge
egenskaper. For fotoner har vi relasjonen mellom b\o lge
og partikkel egenskaper gitt ved $E=h\nu$ og
$\lambda=h/p$, med Plancks konstant som knytter sammen disse to
relasjonene. de Broglie foreslo at liknende relasjoner
ogs\aa\ skulle gjelde for materie og postulerte at 
f.eks.~elektroner kunne tilordnes en b\o lgelengde
\begin{equation}
    \lambda=\frac{h}{p}=\frac{h}{mv}.
\end{equation}
Dette er de Broglies hypotese.

Det er en viktig ting \aa\ merke seg, som vi ogs\aa\ kommer
tilbake i neste avsnitt. For et foton har vi
en enkel relasjon som gir oss b\aa de b\o lgelengde og frekvens,
nemlig $\lambda\nu=c$, med gitt energi og bevegelsemengde.
For massive partikler derimot, trenger vi separate relasjoner
for henholdsvis b\o lgelengde $\lambda=h/p$ og 
frekvens $\nu=E/h$. 

Basert p\aa\ de Broglies hypotese foreslo Elsasser i 1926 at
b\o lgeegenskapene til materie burde kunne testes
p\aa\ lik linje med f.eks.~slik en observerte R\"ontgenstr\aa ler,
dvs.~ved \aa\ sende h\o genergetiske elektroner mot et metall
og detektere p\aa\ ei fotografisk plate et eventuelt diffraksjonsm\o nster for fotonene. Her skulle en alts\aa\ pr\o ve \aa\ detektere 
et diffraksjonsm\o nster for elektroner. Eksperimentet som ble
gjennomf\o rt av Davisson og Germer besto i \aa\ sende
elektroner mot en krystall av  nikkel for deretter \aa\ pr\o ve \aa\
detektere de utg\aa ende spredte elektroner som funksjon av
innkommende energi og spredningsvinkel $\theta$. 
Atomene i nikkel krystallen, med interatom\ae r avstand
$d$ skulle dermed fungere som diffraksjonssentra og i henhold
til b\o lgel\ae re skulle en dermed kunne observere et 
diffraksjonsm\o nster ved detektoren. Fra diffraksjonsteori,
som vi skal utlede i neste avsnitt, kan en vise at det er
en relasjon mellom b\o lgelengden til elektronene, avstanden
mellom atomene som fungerer som diffraksjonssentra og spredningsvinkelen
$\theta$ gitt ved
\begin{equation}
   n\lambda = 2dsin\theta,
\end{equation}
hvor $n$ er et heltall\footnote{Vi utleder denne formlen i neste
avsnitt. Her tillater vi oss \aa\ bare sette opp resultatet.} 
Figur \ref{fig:xray3} viser en skisse for dette diffraksjonseksperimentet.
\begin{figure}[h]
\begin{center}
{\centering
\mbox
{\psfig{figure=xray3.ps,height=8cm,width=12cm}}
}
\end{center}
\caption{Avstanden mellom atomene som fungerer som 
diffraksjonssentra med spredningsvinkelen
$\theta$ for innkommende og utg\aa ende elektroner.\label{fig:xray3}}
\end{figure}

Det som er viktig for den videre forst\aa else, er at 
diffraksjon skyldes en forstyrrelse av en innkommende b\o lge
hvor det forstyrrende element, om det er en spalte\aa pning
eller avstanden mellom atomer i en krystall, er p\aa\ st\o rrelse
med b\o lgelengden til den innkommende b\o lge, dvs.
\begin{equation}
    \lambda/d \sim 1.
\end{equation}
Dersom $\lambda/d\rightarrow 0$ kan vi ikke observere noe
diffraksjonsm\o nster. 

For \aa\ ta et eksempel, anta at vi sender en kontinuerlig
str\o m av baller som veier et 1 kg mot en spalte\aa pning
$d$. Anta at hastigheten er gitt ved $v=10$ ms$^{-1}$.
Bruker vi de Broglies hypotese  
finner vi at b\o lgelengden til denne innkommende str\o m av
baller er gitt ved
\begin{equation}
   \lambda_{ball} =\frac{h}{p}=\frac{h}{mv}=6.6\times 10^{-25}
    \times 10^{-10} \hspace{0.1cm}\mathrm{m} ,
\end{equation}
og dersom vi overhodet skal v\ae re i stand til \aa\ observere
et diffraksjonsm\o nster for ballene som passerer \aa pningen,
b\o r vi ihvertfall ha en spalte\aa pning $d\sim 10^{-35}$ m! 
\AA\ lage en ball p\aa\ et kg som passer til en slik \aa pning
overlates hermed til ekspertene!
I praksis vil det si at $\lambda_{ball}/d \sim 0$ og vi vil 
aldri v\ae re i stand til \aa\ observere et diffraksjonsm\o nster.


Derimot, dersom vi sender elektroner med kinetisk energi
p\aa\ $E_{kin}=100$ eV mot en nikkel krystall, hvor $d=0.91$ \AA\,
har vi 
\begin{equation}
   \lambda_{e^-}=\frac{h}{p}=\frac{h}{\sqrt{2m_eE_{kin}}}=1.2\times 10^{-10} \hspace{0.1cm}\mathrm{m},
\end{equation}
og vil v\ae re i stand til \aa\ observere et diffraksjonsm\o nster.

Det var nettopp det Davisson og Germer s\aa\ .
For en spredningsvinkel p\aa\ $50$ grader 
observerte de en interferens topp som kun kunne forklares dersom elektroner utviser ogs\aa\ 
b\o lgeegenskaper. Se avsnittene 4-1 og 4-2
i l\ae reboka for en mer utf\o rlig diskusjon av eksperimentene
til Davisson og Germer og ogs\aa\ Thompson.

F\o r vi g\aa r inn p\aa\ detaljer om interferens og diffraksjon,
kan vi kort oppsumere denne delen med f\o lgende.
For \aa\ observere et diffraksjonsm\o nster, som igjen
er en b\o lgeegenskap, b\o r vi ha et forhold
$\lambda/d\sim 1$. Har vi derimot $\lambda/d\sim 0$, kan ikke
noe slikt m\o nster observeres. 
Tenker vi deretter p\aa\ $\lambda=h/p$ har vi at
det er forholdet 
\begin{equation}
   \frac{h}{pd}=\frac{h}{mvd},
\end{equation}
som er viktig. Velger vi $h=0$ skal ikke
en partikkel utvise b\o lgeegenskaper. For massive
partikler som ballen i eksemplet ovenfor, blir $p$ s\aa\
stor at vi knapt kan detektere b\o lgelengden.
For elektronet derimot, er massen s\aa\ liten at ogs\aa\
bevegelsesmengden $p=mv$ blir liten, til tross for at hastigheten
$v$ kan bli sv\ae rt s\aa\ stor. Er i tillegg $d$ p\aa\ st\o rrelsesorden med $\lambda$ er det store muligheter for \aa\ observere
et diffraksjonsm\o nster. mer om dette i neste avsnitt. 


  
\section{Diffraksjon, fasehastighet og gruppehastighet}
\label{sec:boelgelaere}
I dette avsnittet skal\footnote{Lesehenvisning er kap 4-6, sidene 205-215.}
vi presentere en del emner fra b\o lgel\ae re som vi vil f\aa\ bruk for b\aa de
i tilknytting den p\aa f\o lgende diskusjonen om Heisenbergs uskarphetsrelasjon
og senere n\aa r vi g\aa r l\o s p\aa\ kvantemekanikken. 
\subsection{B\o lgepakker, gruppe og fasehastighet}
Her skal vi rekapitulere en del begrep fra FYS2140 samt innf\o re begrepet
gruppehastighet. Fra sistnevnte kurs har dere sett at Maxwells likninger gir 
en b\o lgelikning p\aa\ formen
\begin{equation}
  \frac{\partial^2 {\cal E}({\bf x},t)}{\partial {\bf x}^2}  =
  \epsilon_0\mu_0\frac{\partial^2 {\cal E}({\bf x},t)}{\partial t^2},
\label{eq:ewave}
\end{equation} 
hvor ${\cal E}$ er det elektriske feltet. Tilsvarende likning 
f\aa r vi for det magnetiske feltet ved \aa\ erstatte ${\cal E}$ med
${\cal B}$. 
Vi kan generalisere denne likningen ved \aa\ skrive den som
\begin{equation}
  v^2 \frac{\partial^2 \psi({\bf x},t)}{\partial {\bf x}^2}  =
  \frac{\partial^2 \psi({\bf x},t)}{\partial t^2},
\label{eq:generalwave}
\end{equation} 
hvor $\psi$ kan representere det elektriske ${\cal E}$ 
eller magnetiske ${\cal B}$ felt
og hvor
\begin{equation}
   v^2=c^2=\frac{1}{\epsilon_0\mu_0}
\end{equation}
hastigheten til lyset for det e.m.~tilfellet. 
Heretter i dette kurset kommer  vi konsekvent til \aa\ bruke 
$\psi$ som symbol for b\o lgefunksjonen. 

Slik likning (\ref{eq:generalwave})  st\aa r kan den representere
en b\o lgelikning for flere systemer, fra e.m.~felt til trykkb\o lger i en gass eller en
svingende streng. N\aa r vi kommer til Schr\"odingers likning skal vi se at
den tidsavhengige delen av 
denne likning blir forskjellig. Istedet for den andrederiverte av 
b\o lgefunksjonen mhp.~tiden $t$ har vi en f\o rste deriverte mhp.~$t$.

L\o sningen til likning (\ref{eq:generalwave}) kan skrives p\aa\  generell form som
\begin{equation}
\psi(x,t)=f(x\mp v_ft),
\end{equation}
som skal representere en b\o lge som propagerer som funksjon av tid og posisjon
med en hastighet $v_f$. Sistenevnte kalles
fasehastigheten og er hastigheten b\o lgen reiser med. Minustegnet svarer til en forover (i $x$-retning) reisende b\o lge,
mens plusstegnet svarer til ei b\o lge som brer seg bakover.
Figur \ref{fig:fasev} illustrerer
noe av dette. 
\begin{figure}[h]
   \setlength{\unitlength}{1mm}
   \begin{picture}(100,50)
   \put(25,0){\epsfxsize=12cm \epsfbox{fig6.eps}}
   \end{picture}
\label{fig:waveex}
\caption{B\o lge som propagerer forover med fasehastighet $v_f$ og
         b\o lgelengde $\lambda$.\label{fig:fasev}} 
\end{figure}
Ei spesiell l\o sning til b\o lgelikningen er gitt ved sinus eller cosinus
funksjoner, f.eks.~ved 
\begin{equation}
\psi(x,t)=Asin[k(x- v_ft)]=Asin[k(x- v_ft)+2\pi],
\end{equation}
som indikerer at n\aa r 
$x\rightarrow x+2\pi/k$ s\aa\ har $\psi$ samme verdi. St\o rrelsen $A$ er amplituden.
Alternativt kunne vi ha brukt cosinus-funksjonen som l\o sning
\begin{equation}
\psi(x,t)=Acos[k(x- v_ft)],
\end{equation}
eller uttrykke l\o sningen som
\begin{equation}
   \psi_i=Ae^{i(kx-kv_f t)}.
\end{equation}
Fra FY101 har vi sett at b\o lgelengden kan skrives som  
\begin{equation}
    \lambda=\frac{2\pi}{k},
\end{equation}
hvor $k$ 
kalles b\o lgetallet.
Vi kan da omskrive b\o lgefunksjonen til
\begin{equation}
\psi(x,t)=Asin[\frac{2\pi}{\lambda}(x- v_ft)],
\end{equation}
og med vinkelfrekvensen
\begin{equation}
   \omega=2\pi\nu=kv=\frac{2\pi v_f}{\lambda},
\end{equation}
har vi 
\begin{equation}
\psi(x,t)=Asin(kx- \omega t)
\end{equation}
Nullpunktene svarer til noder
\begin{equation}
   \frac{2\pi}{\lambda}(x_n-vt)=n\pi \hspace{0.3cm} n=0,\pm 1,\pm 2, \dots
\end{equation}
som gir
\begin{equation}
   x_n=\frac{n\lambda}{2}+vt.
\end{equation}
Fasehastigheten $v_f$ kan ogs\aa\ uttrykkes vha.~vinkelfrekvensen og b\o lgetallet ved
\begin{equation}
   v_f=\frac{\omega}{k}
   \label{eq:vf}.
\end{equation}


La oss anta at den formen vi har funnet p\aa\ l\o sningen til b\o lgefunksjonen
skal svare til b\o lgefunksjonen for et foton eller elektron. 
Dersom vi \o nsker \aa\ assosiere fasehastigheten $v_f$ 
med hastigheten $v$ til en partikkel f\aa r vi et problem. Dette ser vi fra
\begin{equation}
   v_f=\lambda\nu=\frac{hE}{ph}=\frac{E}{p},
\end{equation}
hvor vi har brukt de Broglie sin hypotese. Vi har sneket inn en ansats 
for materieb\o lger om
at $E=h\nu=\hbar\omega$, noe som vi skal komme tilbake til under 
Schr\"odingers likning. Antar vi at dette dreier seg om
ikke relativistiske partikler med energi $E=mv^2/2$ og bevegelsesmengde
 $p=mv$ finner vi at
\begin{equation}
   v_f=\frac{E}{p}=\frac{mv^2/2}{mv}=\frac{v}{2}.
\end{equation}
Dersom vi \o nsker at b\o lgen i Figur \ref{fig:fasev} 
skal representere en partikkel, b\o r b\o lgehastigheten (fasehastigheten) 
ogs\aa\ svare til partikkelens hastighet. Dette stemmer ikke helt i siste
likning.
Merk ogs\aa\ at b\o lgen i Figur \ref{fig:fasev} 
svarer til en kontinuerlig harmonisk b\o lge 
av uendelig utstrekning, med en gitt 
b\o lgelengde
og frekvens. 
En slik b\o lge 
kan ikke representere en partikkel som befinner seg p\aa\ et bestemt sted
i  rommet.
Ei heller er den egnet til \aa\ sende et signal,
siden et signal, tenk bare p\aa\ n\aa r dere roper etter noen, er noe som starter et sted
og slutter et annet sted etter et gitt tidspunkt. 
En slik b\o lge har en form som likner mer p\aa\ b\o lgen i Figur \ref{fig:fasev2}. Denne figuren viser  en s\aa kalt superposisjon (addisjon) av flere
b\o lger. Vi skal se at en slik superposisjon av b\o lger, som kalles b\o lgepakke, 
leder til begrepet 
gruppehastighet, som igjen vil gi oss en hastighet for b\o lgepakken som svarer til en
partikkels hastighet. 
\begin{figure}[h]
   \setlength{\unitlength}{1mm}
   \begin{picture}(100,80)
   \put(25,0){\epsfxsize=12cm \epsfbox{fig7.eps}}
   \end{picture}
\caption{Eksempel p\aa\ superponering av to b\o lger ved tiden $t=0$ og
$dk=k_1-k_2$.\label{fig:fasev2} }
\end{figure}
Som et eksempel for \aa\ belyse dette velger vi 
\aa\ se p\aa\ superponering av to b\o lger med ulik vinkelfrekvens
$\omega_1$ og $\omega_2$ 
og b\o lgetall $k_1$ og $k_2$. Vi antar at forskjellen mellom disse verdiene
er slik at
\begin{equation}
    \omega_1-\omega_2 << 1 \hspace{0.2cm} og \hspace{0.2cm} k_1-k_2 << 1.
\end{equation}
Legger vi sammen disse to b\o lgene har vi
\begin{equation}
   \psi=\psi_1+\psi_2=A_1sin(k_1x- \omega_1 t)+A_2sin(k_2x- \omega_2 t),
\end{equation}
og setter vi amplitudene like $A_1=A_2=A$ har vi
\begin{equation}
   \psi(x,t)=A(sin(k_1x- \omega_1 t)+sin(k_2x- \omega_2 t) ),
\end{equation}
som vha.~den trigonometriske relasjonen
\begin{equation}
   sin X+sinY=2cos\left(\frac{X-Y}{2}\right)sin\left(\frac{X+Y}{2}\right),
\end{equation}
gir
\begin{equation}
   \psi(x,t)=2Acos\left(\frac{(k_1-k_2)x-(w_1-w_2)t}{2}\right)sin\left(\frac{(k_1+k_2)x-(w_1+w_2)t}{2}\right).
\end{equation}
Setter vi s\aa\
\begin{equation}
    \omega_1+\omega_2 \approx 2\omega_1=2\omega \hspace{0.2cm} og \hspace{0.2cm}k_1+k_2 \approx2k_1=2k,
\end{equation}
har vi
\begin{equation}
   \psi(x,t)=2Acos(\frac{(k_1-k_2)x-(w_1-w_2)t}{2})sin(kx-\omega t).
\end{equation}
Leddet med cosinus funksjonen kalles for den modulerte del og vi sier at
amplituden $A$ er modulert gitt ved faktoren 
\begin{equation}
   2Acos(\frac{(k_1-k_2)x-(w_1-w_2)t}{2}).
\end{equation}
Den modulerte amplituden svarer til en hastighet 
\begin{equation}
   \frac{\omega_1-\omega_2}{k_1-k_2}\rightarrow \frac{d\omega}{dk}=v_g,
\end{equation}
som er definisjonen p\aa\ gruppehastigheten. Se Figur 2.4 for et eksempel ved 
$t=0$. 
Vi kan si at b\o lgepakkens amplitude reiser med en hastighet som svarer 
til gruppehastigheten.
Det er denne hastigheten som skal svare til partikkelens hastighet.
Det kan vi se dersom vi bruker at 
\[
   \omega=2\pi\nu=2\pi\frac{E}{h},
\]
og
\[
   k=2\pi\frac{p}{h},
\] som gir at
\begin{equation}
   v_g=\frac{d\omega}{dk}=\frac{d(2\pi E/h)}{d(2\pi p/h)}=\frac{dE}{dp},
\end{equation}
og antar vi at v\aa r partikkel kan beskrives vha.~ikke-relativistisk
mekanikk har vi
\begin{equation}
   v_g=\frac{dE}{dp}=\frac{d(mv^2/2)}{d(mv)}=v,
\end{equation}
dvs.~gruppehastigheten svarer til partikkelens hastighet. 
Gruppehastigheten til en materieb\o lge er lik hastigheten til partikkelen
hvis bevegelse den skal representere. de Broglie sitt postulat er dermed
konsistent. Vi f\aa r samme resultat dersom vi hadde antatt en relativistisk
beskrivelse.

selv om vi bare har addert to b\o lger med ulike frekvenser og b\o lgetall,
s\aa\ holder dette resultatet for mange b\o lger som adderes og har frekvenser
og b\o lgetall n\ae r hverandre.


Dersom $dv_f/dk \ne 0$ kan vi bruke $\omega =kv_f$ til \aa\ vise at
\begin{equation}
   v_g=v_f+k\frac{dv_f}{dk}.
\end{equation}
I et dispersivt medium vil et signal reise med gruppehastigheten.

For en ikke
relativistisk partikkel er $v_g=v=p/m=\hbar k/m$ slik at vi f\aa r
\begin{equation}
   \frac{d\omega(k)}{dk}=v_g=v=\frac{\hbar k}{m},
\end{equation}
som gir n\aa r vi integrer opp (og setter konstanten fra 
integrasjonen lik null) 
\begin{equation}
   \omega(k)=\frac{\hbar k^2}{2m}=\frac{\hbar^2 k^2}{2m\hbar}.
\end{equation}
En slik relasjon kalles for en dispersjonsrelasjon.
Med $\omega=2\pi\nu$ har vi
\begin{equation}
   \nu(k)=\frac{\hbar^2 k^2}{2m \hbar 2\pi}=
   \frac{\hbar^2 k^2}{2m h}=\frac{p^2}{2m}\frac{1}{h}=\frac{E}{h},
\end{equation}
som ventet! 
Utifra likning  (\ref{eq:vf}) finner vi s\aa\ fasehastigheten $v_f$.

\subsection{Interferens}

Interferens forekommer n\aa r to eller flere b\o lger sammenfaller i tid og rom
og gir opphav til en maksimal amplitude eller en minimal amplitude.
Intensiteten til f.eks.~e.m.~str\aa ling som treffer ei fotografisk plate
er proporsjonal med kvadratet av b\o lgefunksjonen, som igjen er proporsjonal
med amplituden $A$.
{\bf Intensitet er definert som energi som kommer inn per sekund per areal.} 
Dersom amplituden, som er resultatet av summen av alle amplituder, har et 
maksimum i et gitt punkt $P$, vil en kunne observere et kraftig m\o nster
p\aa\ f.eks.~ei fotografisk plate. Vi snakker da om konstruktiv interferens.
Dersom vi ikke observerer noe har vi destruktiv interferens.

La oss n\aa\ regne ut intensiteten for lys som blir sendt fra $N$ 
lyskilder mot en skjerm ved avstand $D$. Avstanden mellom hver lyskilde er
$a$. Vi antar at vi har med koherente lyskilder \aa\ gj\o re, at
amplituden til alle b\o lgefunksjonene $A_i$ er like,
\begin{equation}
   A_1=A_2=A_3\dots = A_N=A,
\end{equation}
og at b\aa de vinkelfrekvens og b\o lgetall er like.
\begin{figure}[h]
   \setlength{\unitlength}{1mm}
   \begin{picture}(100,120)
   \put(25,0){\epsfxsize=6cm \epsfbox{fig1.eps}}
   \end{picture}
\label{fig:interferensoppsett}
\caption{Skjematisk oppsett med $N$ lyskilder som sender koherent lys mot en skjerm.
         Avstanden mellom hver lyskilde er $a$, og $r_2-r_1$ kalles den optiske
         veilengde. Skjermen antas \aa\ v\ae re langt borte fra lyskildene, slik
         at linjene som skal representere b\o lgene kan antas \aa\ 
         v\ae re parallelle.}
\end{figure}
En slik b\o lgefunksjon vil ved skjermen v\ae re gitt ved
\begin{equation}
   \psi_i=Ae^{i(kr_i-\omega t)},
\end{equation}
hvor vi n\aa\ velger \aa\ bruke den komplekse representasjonen for b\o lgefunsjonen. St\o rrelsen $r_i$ er {\bf avstanden fra lyskilde $i$ til skjermen}. 
Vi antar ogs\aa\ at skjermen er s\aa\ langt borte at vi kan betrakte 
linjene fra lyskilden til skjermen som tiln\ae rma parallelle, 
se Figur \ref{fig:interferensoppsett} 
for et mulig oppsett med $N$ lyskilder.

Den totale b\o lgefunksjonen er n\aa\ gitt ved
\begin{equation}
   \psi=A\left( e^{i(kr_1-\omega t)}+e^{i(kr_2-\omega t)}+\dots +e^{i(kr_N-\omega t)}\right),
\end{equation}
eller
\begin{equation}
   \psi=Ae^{i(kr_1-\omega t)}\left(1+e^{i(k(r_2-r_1))}+\dots +e^{i(k(r_N-r_1)}\right).
\end{equation}
Definerer vi s\aa\
\begin{equation}
  \delta=k(r_2-r_1),\hspace{0.2cm}  2\delta=k(r_3-r_1),\hspace{0.2cm} \dots, (N-1)\delta=k(r_N-r_1),
\end{equation}
har vi
\begin{equation}
   \psi=Ae^{i(kr_1-\omega t)}\left(1+e^{i\delta}+\dots +e^{i\delta(N-1)}\right),
\end{equation}
som kan summeres opp ved formelen til ei geometrisk rekke for \aa\ gi
\begin{equation}
   \psi=Ae^{i(kr_1-\omega t)}\frac{e^{iN\delta}-1}{e^{i\delta}-1},
\end{equation}
eller
\begin{equation}
      \psi=Ae^{i(kr_1-\omega t)}\frac{e^{iN\delta/2}}{e^{i\delta/2}}\frac{e^{iN\delta/2}-e^{-iN\delta/2}}{e^{i\delta/2}-e^{-i\delta/2}},
\end{equation}
som vi kan skrive om vha.~definisjonen for sinus-funksjonen som
\begin{equation}
      \psi = Ae^{i(kr_1-\omega t)}e^{i(N-1)\delta/2}\frac{sin(N\delta/2)}{sin(\delta/2)}.
\end{equation}
Siden vi har antatt at linjene som treffer skjermen er parallelle, har vi at
den {\bf optiske veilengden} $r_2-r_1$ er gitt ved 
\begin{equation}
   r_2-r_1=asin\theta,
\end{equation}
som sammen med 
\begin{equation}
    k(r_2-r_1)=\frac{2\pi}{\lambda}(r_2-r_1)=\frac{2\pi}{\lambda}asin\theta =\delta,
\end{equation}
og definisjonen 
\begin{equation}
   R=\frac{1}{2}(N-1)asin\theta+r_1,
\end{equation}
som svarer til en linje med lengde $R$ trukket fra midtpunktet til alle lyskildene,
kan vi skrive om uttrykket for $\psi $ som
\begin{equation}
  \psi = Ae^{i(kR-\omega t)}\frac{sin(N\delta/2)}{sin(\delta/2)},
\end{equation}
eller
\begin{equation}
  \psi = Ae^{i(kR-\omega t)}\frac{sin(N\frac{\pi}{\lambda}asin\theta)}{sin(\frac{\pi}{\lambda}asin\theta)}.
\end{equation}
Intensiteten $I$ er proporsjonal med kvadratet av b\o lgefunskjonen, og siden
$|e^{i(kR-\omega t)}|^2=1$ har vi
\begin{equation}
    I=I_0\frac{sin^2(N\frac{\pi}{\lambda}asin\theta)}
               {sin^2(\frac{\pi}{\lambda}asin\theta)}.
\end{equation}
Konstanten $I_0$ inneholder bla.~$|A|^2$ samt konstanter som gir oss de riktige
enhetene. 
Intensiteten har en maksimal verdi ved
\begin{equation}
   asin\theta_m=m\lambda, \hspace{0.1cm} m=0,\pm 1, \pm 2, \dots ,
\end{equation}
som svarer til konstruktiv interferens.

For $N=2$ har vi det mer velkjente uttrykket
\begin{equation}
    I=4I_0cos^2(\frac{\pi}{\lambda}asin\theta),
\end{equation}
der vi har brukt at $\sin(2x)=2\cos(x)\sin(x)$
Eksempler p\aa\ mulige interferensm\o nstre er vist i Figur 
\ref{fig:interferensmoenster} for $N=2$ og $N=8$. Vi merker oss at etterhvert som vi 
\o ker $N$ blir interferensmaksima skarpere og skarpere. 
\begin{figure}
% GNUPLOT: LaTeX picture with Postscript
\begingroup%
  \makeatletter%
  \newcommand{\GNUPLOTspecial}{%
    \@sanitize\catcode`\%=14\relax\special}%
  \setlength{\unitlength}{0.1bp}%
{\GNUPLOTspecial{!
%!PS-Adobe-2.0 EPSF-2.0
%%Title: inter2.tex
%%Creator: gnuplot 3.7 patchlevel 0.2
%%CreationDate: Fri Feb 11 10:17:40 2000
%%DocumentFonts: 
%%BoundingBox: 0 0 360 216
%%Orientation: Landscape
%%EndComments
/gnudict 256 dict def
gnudict begin
/Color false def
/Solid false def
/gnulinewidth 5.000 def
/userlinewidth gnulinewidth def
/vshift -33 def
/dl {10 mul} def
/hpt_ 31.5 def
/vpt_ 31.5 def
/hpt hpt_ def
/vpt vpt_ def
/M {moveto} bind def
/L {lineto} bind def
/R {rmoveto} bind def
/V {rlineto} bind def
/vpt2 vpt 2 mul def
/hpt2 hpt 2 mul def
/Lshow { currentpoint stroke M
  0 vshift R show } def
/Rshow { currentpoint stroke M
  dup stringwidth pop neg vshift R show } def
/Cshow { currentpoint stroke M
  dup stringwidth pop -2 div vshift R show } def
/UP { dup vpt_ mul /vpt exch def hpt_ mul /hpt exch def
  /hpt2 hpt 2 mul def /vpt2 vpt 2 mul def } def
/DL { Color {setrgbcolor Solid {pop []} if 0 setdash }
 {pop pop pop Solid {pop []} if 0 setdash} ifelse } def
/BL { stroke userlinewidth 2 mul setlinewidth } def
/AL { stroke userlinewidth 2 div setlinewidth } def
/UL { dup gnulinewidth mul /userlinewidth exch def
      10 mul /udl exch def } def
/PL { stroke userlinewidth setlinewidth } def
/LTb { BL [] 0 0 0 DL } def
/LTa { AL [1 udl mul 2 udl mul] 0 setdash 0 0 0 setrgbcolor } def
/LT0 { PL [] 1 0 0 DL } def
/LT1 { PL [4 dl 2 dl] 0 1 0 DL } def
/LT2 { PL [2 dl 3 dl] 0 0 1 DL } def
/LT3 { PL [1 dl 1.5 dl] 1 0 1 DL } def
/LT4 { PL [5 dl 2 dl 1 dl 2 dl] 0 1 1 DL } def
/LT5 { PL [4 dl 3 dl 1 dl 3 dl] 1 1 0 DL } def
/LT6 { PL [2 dl 2 dl 2 dl 4 dl] 0 0 0 DL } def
/LT7 { PL [2 dl 2 dl 2 dl 2 dl 2 dl 4 dl] 1 0.3 0 DL } def
/LT8 { PL [2 dl 2 dl 2 dl 2 dl 2 dl 2 dl 2 dl 4 dl] 0.5 0.5 0.5 DL } def
/Pnt { stroke [] 0 setdash
   gsave 1 setlinecap M 0 0 V stroke grestore } def
/Dia { stroke [] 0 setdash 2 copy vpt add M
  hpt neg vpt neg V hpt vpt neg V
  hpt vpt V hpt neg vpt V closepath stroke
  Pnt } def
/Pls { stroke [] 0 setdash vpt sub M 0 vpt2 V
  currentpoint stroke M
  hpt neg vpt neg R hpt2 0 V stroke
  } def
/Box { stroke [] 0 setdash 2 copy exch hpt sub exch vpt add M
  0 vpt2 neg V hpt2 0 V 0 vpt2 V
  hpt2 neg 0 V closepath stroke
  Pnt } def
/Crs { stroke [] 0 setdash exch hpt sub exch vpt add M
  hpt2 vpt2 neg V currentpoint stroke M
  hpt2 neg 0 R hpt2 vpt2 V stroke } def
/TriU { stroke [] 0 setdash 2 copy vpt 1.12 mul add M
  hpt neg vpt -1.62 mul V
  hpt 2 mul 0 V
  hpt neg vpt 1.62 mul V closepath stroke
  Pnt  } def
/Star { 2 copy Pls Crs } def
/BoxF { stroke [] 0 setdash exch hpt sub exch vpt add M
  0 vpt2 neg V  hpt2 0 V  0 vpt2 V
  hpt2 neg 0 V  closepath fill } def
/TriUF { stroke [] 0 setdash vpt 1.12 mul add M
  hpt neg vpt -1.62 mul V
  hpt 2 mul 0 V
  hpt neg vpt 1.62 mul V closepath fill } def
/TriD { stroke [] 0 setdash 2 copy vpt 1.12 mul sub M
  hpt neg vpt 1.62 mul V
  hpt 2 mul 0 V
  hpt neg vpt -1.62 mul V closepath stroke
  Pnt  } def
/TriDF { stroke [] 0 setdash vpt 1.12 mul sub M
  hpt neg vpt 1.62 mul V
  hpt 2 mul 0 V
  hpt neg vpt -1.62 mul V closepath fill} def
/DiaF { stroke [] 0 setdash vpt add M
  hpt neg vpt neg V hpt vpt neg V
  hpt vpt V hpt neg vpt V closepath fill } def
/Pent { stroke [] 0 setdash 2 copy gsave
  translate 0 hpt M 4 {72 rotate 0 hpt L} repeat
  closepath stroke grestore Pnt } def
/PentF { stroke [] 0 setdash gsave
  translate 0 hpt M 4 {72 rotate 0 hpt L} repeat
  closepath fill grestore } def
/Circle { stroke [] 0 setdash 2 copy
  hpt 0 360 arc stroke Pnt } def
/CircleF { stroke [] 0 setdash hpt 0 360 arc fill } def
/C0 { BL [] 0 setdash 2 copy moveto vpt 90 450  arc } bind def
/C1 { BL [] 0 setdash 2 copy        moveto
       2 copy  vpt 0 90 arc closepath fill
               vpt 0 360 arc closepath } bind def
/C2 { BL [] 0 setdash 2 copy moveto
       2 copy  vpt 90 180 arc closepath fill
               vpt 0 360 arc closepath } bind def
/C3 { BL [] 0 setdash 2 copy moveto
       2 copy  vpt 0 180 arc closepath fill
               vpt 0 360 arc closepath } bind def
/C4 { BL [] 0 setdash 2 copy moveto
       2 copy  vpt 180 270 arc closepath fill
               vpt 0 360 arc closepath } bind def
/C5 { BL [] 0 setdash 2 copy moveto
       2 copy  vpt 0 90 arc
       2 copy moveto
       2 copy  vpt 180 270 arc closepath fill
               vpt 0 360 arc } bind def
/C6 { BL [] 0 setdash 2 copy moveto
      2 copy  vpt 90 270 arc closepath fill
              vpt 0 360 arc closepath } bind def
/C7 { BL [] 0 setdash 2 copy moveto
      2 copy  vpt 0 270 arc closepath fill
              vpt 0 360 arc closepath } bind def
/C8 { BL [] 0 setdash 2 copy moveto
      2 copy vpt 270 360 arc closepath fill
              vpt 0 360 arc closepath } bind def
/C9 { BL [] 0 setdash 2 copy moveto
      2 copy  vpt 270 450 arc closepath fill
              vpt 0 360 arc closepath } bind def
/C10 { BL [] 0 setdash 2 copy 2 copy moveto vpt 270 360 arc closepath fill
       2 copy moveto
       2 copy vpt 90 180 arc closepath fill
               vpt 0 360 arc closepath } bind def
/C11 { BL [] 0 setdash 2 copy moveto
       2 copy  vpt 0 180 arc closepath fill
       2 copy moveto
       2 copy  vpt 270 360 arc closepath fill
               vpt 0 360 arc closepath } bind def
/C12 { BL [] 0 setdash 2 copy moveto
       2 copy  vpt 180 360 arc closepath fill
               vpt 0 360 arc closepath } bind def
/C13 { BL [] 0 setdash  2 copy moveto
       2 copy  vpt 0 90 arc closepath fill
       2 copy moveto
       2 copy  vpt 180 360 arc closepath fill
               vpt 0 360 arc closepath } bind def
/C14 { BL [] 0 setdash 2 copy moveto
       2 copy  vpt 90 360 arc closepath fill
               vpt 0 360 arc } bind def
/C15 { BL [] 0 setdash 2 copy vpt 0 360 arc closepath fill
               vpt 0 360 arc closepath } bind def
/Rec   { newpath 4 2 roll moveto 1 index 0 rlineto 0 exch rlineto
       neg 0 rlineto closepath } bind def
/Square { dup Rec } bind def
/Bsquare { vpt sub exch vpt sub exch vpt2 Square } bind def
/S0 { BL [] 0 setdash 2 copy moveto 0 vpt rlineto BL Bsquare } bind def
/S1 { BL [] 0 setdash 2 copy vpt Square fill Bsquare } bind def
/S2 { BL [] 0 setdash 2 copy exch vpt sub exch vpt Square fill Bsquare } bind def
/S3 { BL [] 0 setdash 2 copy exch vpt sub exch vpt2 vpt Rec fill Bsquare } bind def
/S4 { BL [] 0 setdash 2 copy exch vpt sub exch vpt sub vpt Square fill Bsquare } bind def
/S5 { BL [] 0 setdash 2 copy 2 copy vpt Square fill
       exch vpt sub exch vpt sub vpt Square fill Bsquare } bind def
/S6 { BL [] 0 setdash 2 copy exch vpt sub exch vpt sub vpt vpt2 Rec fill Bsquare } bind def
/S7 { BL [] 0 setdash 2 copy exch vpt sub exch vpt sub vpt vpt2 Rec fill
       2 copy vpt Square fill
       Bsquare } bind def
/S8 { BL [] 0 setdash 2 copy vpt sub vpt Square fill Bsquare } bind def
/S9 { BL [] 0 setdash 2 copy vpt sub vpt vpt2 Rec fill Bsquare } bind def
/S10 { BL [] 0 setdash 2 copy vpt sub vpt Square fill 2 copy exch vpt sub exch vpt Square fill
       Bsquare } bind def
/S11 { BL [] 0 setdash 2 copy vpt sub vpt Square fill 2 copy exch vpt sub exch vpt2 vpt Rec fill
       Bsquare } bind def
/S12 { BL [] 0 setdash 2 copy exch vpt sub exch vpt sub vpt2 vpt Rec fill Bsquare } bind def
/S13 { BL [] 0 setdash 2 copy exch vpt sub exch vpt sub vpt2 vpt Rec fill
       2 copy vpt Square fill Bsquare } bind def
/S14 { BL [] 0 setdash 2 copy exch vpt sub exch vpt sub vpt2 vpt Rec fill
       2 copy exch vpt sub exch vpt Square fill Bsquare } bind def
/S15 { BL [] 0 setdash 2 copy Bsquare fill Bsquare } bind def
/D0 { gsave translate 45 rotate 0 0 S0 stroke grestore } bind def
/D1 { gsave translate 45 rotate 0 0 S1 stroke grestore } bind def
/D2 { gsave translate 45 rotate 0 0 S2 stroke grestore } bind def
/D3 { gsave translate 45 rotate 0 0 S3 stroke grestore } bind def
/D4 { gsave translate 45 rotate 0 0 S4 stroke grestore } bind def
/D5 { gsave translate 45 rotate 0 0 S5 stroke grestore } bind def
/D6 { gsave translate 45 rotate 0 0 S6 stroke grestore } bind def
/D7 { gsave translate 45 rotate 0 0 S7 stroke grestore } bind def
/D8 { gsave translate 45 rotate 0 0 S8 stroke grestore } bind def
/D9 { gsave translate 45 rotate 0 0 S9 stroke grestore } bind def
/D10 { gsave translate 45 rotate 0 0 S10 stroke grestore } bind def
/D11 { gsave translate 45 rotate 0 0 S11 stroke grestore } bind def
/D12 { gsave translate 45 rotate 0 0 S12 stroke grestore } bind def
/D13 { gsave translate 45 rotate 0 0 S13 stroke grestore } bind def
/D14 { gsave translate 45 rotate 0 0 S14 stroke grestore } bind def
/D15 { gsave translate 45 rotate 0 0 S15 stroke grestore } bind def
/DiaE { stroke [] 0 setdash vpt add M
  hpt neg vpt neg V hpt vpt neg V
  hpt vpt V hpt neg vpt V closepath stroke } def
/BoxE { stroke [] 0 setdash exch hpt sub exch vpt add M
  0 vpt2 neg V hpt2 0 V 0 vpt2 V
  hpt2 neg 0 V closepath stroke } def
/TriUE { stroke [] 0 setdash vpt 1.12 mul add M
  hpt neg vpt -1.62 mul V
  hpt 2 mul 0 V
  hpt neg vpt 1.62 mul V closepath stroke } def
/TriDE { stroke [] 0 setdash vpt 1.12 mul sub M
  hpt neg vpt 1.62 mul V
  hpt 2 mul 0 V
  hpt neg vpt -1.62 mul V closepath stroke } def
/PentE { stroke [] 0 setdash gsave
  translate 0 hpt M 4 {72 rotate 0 hpt L} repeat
  closepath stroke grestore } def
/CircE { stroke [] 0 setdash 
  hpt 0 360 arc stroke } def
/Opaque { gsave closepath 1 setgray fill grestore 0 setgray closepath } def
/DiaW { stroke [] 0 setdash vpt add M
  hpt neg vpt neg V hpt vpt neg V
  hpt vpt V hpt neg vpt V Opaque stroke } def
/BoxW { stroke [] 0 setdash exch hpt sub exch vpt add M
  0 vpt2 neg V hpt2 0 V 0 vpt2 V
  hpt2 neg 0 V Opaque stroke } def
/TriUW { stroke [] 0 setdash vpt 1.12 mul add M
  hpt neg vpt -1.62 mul V
  hpt 2 mul 0 V
  hpt neg vpt 1.62 mul V Opaque stroke } def
/TriDW { stroke [] 0 setdash vpt 1.12 mul sub M
  hpt neg vpt 1.62 mul V
  hpt 2 mul 0 V
  hpt neg vpt -1.62 mul V Opaque stroke } def
/PentW { stroke [] 0 setdash gsave
  translate 0 hpt M 4 {72 rotate 0 hpt L} repeat
  Opaque stroke grestore } def
/CircW { stroke [] 0 setdash 
  hpt 0 360 arc Opaque stroke } def
/BoxFill { gsave Rec 1 setgray fill grestore } def
end
%%EndProlog
}}%
\begin{picture}(3600,2160)(0,0)%
{\GNUPLOTspecial{"
gnudict begin
gsave
0 0 translate
0.100 0.100 scale
0 setgray
newpath
1.000 UL
LTb
400 300 M
63 0 V
2987 0 R
-63 0 V
400 520 M
63 0 V
2987 0 R
-63 0 V
400 740 M
63 0 V
2987 0 R
-63 0 V
400 960 M
63 0 V
2987 0 R
-63 0 V
400 1180 M
63 0 V
2987 0 R
-63 0 V
400 1400 M
63 0 V
2987 0 R
-63 0 V
400 1620 M
63 0 V
2987 0 R
-63 0 V
400 1840 M
63 0 V
2987 0 R
-63 0 V
400 2060 M
63 0 V
2987 0 R
-63 0 V
705 300 M
0 63 V
0 1697 R
0 -63 V
1315 300 M
0 63 V
0 1697 R
0 -63 V
1925 300 M
0 63 V
0 1697 R
0 -63 V
2535 300 M
0 63 V
0 1697 R
0 -63 V
3145 300 M
0 63 V
0 1697 R
0 -63 V
1.000 UL
LTb
400 300 M
3050 0 V
0 1760 V
-3050 0 V
400 300 L
1.000 UL
LT0
3087 1947 M
263 0 V
400 300 M
31 44 V
31 127 V
30 199 V
31 249 V
31 275 V
31 274 V
31 245 V
30 192 V
31 119 V
31 36 V
31 -53 V
31 -135 V
31 -205 V
30 -253 V
31 -276 V
893 866 L
924 625 L
955 440 L
985 328 L
31 -26 V
31 61 V
31 143 V
31 210 V
30 257 V
31 277 V
31 270 V
31 236 V
31 179 V
30 103 V
31 18 V
31 -70 V
31 -150 V
31 -216 V
30 -260 V
31 -278 V
31 -268 V
31 -231 V
31 -172 V
31 -95 V
30 -9 V
31 78 V
31 158 V
31 221 V
31 263 V
30 278 V
31 266 V
31 226 V
31 165 V
31 87 V
30 0 V
31 -87 V
31 -165 V
31 -226 V
31 -266 V
30 -278 V
31 -263 V
31 -221 V
31 -158 V
31 -78 V
30 9 V
31 95 V
31 172 V
31 231 V
31 268 V
31 278 V
30 260 V
31 216 V
31 150 V
31 70 V
31 -18 V
30 -103 V
31 -179 V
31 -236 V
31 -270 V
31 -277 V
30 -257 V
31 -210 V
31 -143 V
31 -61 V
31 26 V
30 112 V
31 185 V
31 241 V
31 272 V
31 276 V
30 253 V
31 205 V
31 135 V
31 53 V
31 -36 V
31 -119 V
30 -192 V
31 -245 V
31 -274 V
31 -275 V
31 -249 V
30 -199 V
31 -127 V
31 -44 V
stroke
grestore
end
showpage
}}%
\put(3037,1947){\makebox(0,0)[r]{N=2}}%
\put(100,1180){%
\special{ps: gsave currentpoint currentpoint translate
270 rotate neg exch neg exch translate}%
\makebox(0,0)[b]{\shortstack{I}}%
\special{ps: currentpoint grestore moveto}%
}%
\put(3145,200){\makebox(0,0){2}}%
\put(2535,200){\makebox(0,0){1}}%
\put(1925,200){\makebox(0,0){0}}%
\put(1315,200){\makebox(0,0){-1}}%
\put(705,200){\makebox(0,0){-2}}%
\end{picture}%
\endgroup
\endinput

% GNUPLOT: LaTeX picture with Postscript
\begingroup%
  \makeatletter%
  \newcommand{\GNUPLOTspecial}{%
    \@sanitize\catcode`\%=14\relax\special}%
  \setlength{\unitlength}{0.1bp}%
{\GNUPLOTspecial{!
%!PS-Adobe-2.0 EPSF-2.0
%%Title: inter8.tex
%%Creator: gnuplot 3.7 patchlevel 0.2
%%CreationDate: Fri Feb 11 10:17:44 2000
%%DocumentFonts: 
%%BoundingBox: 0 0 360 216
%%Orientation: Landscape
%%EndComments
/gnudict 256 dict def
gnudict begin
/Color false def
/Solid false def
/gnulinewidth 5.000 def
/userlinewidth gnulinewidth def
/vshift -33 def
/dl {10 mul} def
/hpt_ 31.5 def
/vpt_ 31.5 def
/hpt hpt_ def
/vpt vpt_ def
/M {moveto} bind def
/L {lineto} bind def
/R {rmoveto} bind def
/V {rlineto} bind def
/vpt2 vpt 2 mul def
/hpt2 hpt 2 mul def
/Lshow { currentpoint stroke M
  0 vshift R show } def
/Rshow { currentpoint stroke M
  dup stringwidth pop neg vshift R show } def
/Cshow { currentpoint stroke M
  dup stringwidth pop -2 div vshift R show } def
/UP { dup vpt_ mul /vpt exch def hpt_ mul /hpt exch def
  /hpt2 hpt 2 mul def /vpt2 vpt 2 mul def } def
/DL { Color {setrgbcolor Solid {pop []} if 0 setdash }
 {pop pop pop Solid {pop []} if 0 setdash} ifelse } def
/BL { stroke userlinewidth 2 mul setlinewidth } def
/AL { stroke userlinewidth 2 div setlinewidth } def
/UL { dup gnulinewidth mul /userlinewidth exch def
      10 mul /udl exch def } def
/PL { stroke userlinewidth setlinewidth } def
/LTb { BL [] 0 0 0 DL } def
/LTa { AL [1 udl mul 2 udl mul] 0 setdash 0 0 0 setrgbcolor } def
/LT0 { PL [] 1 0 0 DL } def
/LT1 { PL [4 dl 2 dl] 0 1 0 DL } def
/LT2 { PL [2 dl 3 dl] 0 0 1 DL } def
/LT3 { PL [1 dl 1.5 dl] 1 0 1 DL } def
/LT4 { PL [5 dl 2 dl 1 dl 2 dl] 0 1 1 DL } def
/LT5 { PL [4 dl 3 dl 1 dl 3 dl] 1 1 0 DL } def
/LT6 { PL [2 dl 2 dl 2 dl 4 dl] 0 0 0 DL } def
/LT7 { PL [2 dl 2 dl 2 dl 2 dl 2 dl 4 dl] 1 0.3 0 DL } def
/LT8 { PL [2 dl 2 dl 2 dl 2 dl 2 dl 2 dl 2 dl 4 dl] 0.5 0.5 0.5 DL } def
/Pnt { stroke [] 0 setdash
   gsave 1 setlinecap M 0 0 V stroke grestore } def
/Dia { stroke [] 0 setdash 2 copy vpt add M
  hpt neg vpt neg V hpt vpt neg V
  hpt vpt V hpt neg vpt V closepath stroke
  Pnt } def
/Pls { stroke [] 0 setdash vpt sub M 0 vpt2 V
  currentpoint stroke M
  hpt neg vpt neg R hpt2 0 V stroke
  } def
/Box { stroke [] 0 setdash 2 copy exch hpt sub exch vpt add M
  0 vpt2 neg V hpt2 0 V 0 vpt2 V
  hpt2 neg 0 V closepath stroke
  Pnt } def
/Crs { stroke [] 0 setdash exch hpt sub exch vpt add M
  hpt2 vpt2 neg V currentpoint stroke M
  hpt2 neg 0 R hpt2 vpt2 V stroke } def
/TriU { stroke [] 0 setdash 2 copy vpt 1.12 mul add M
  hpt neg vpt -1.62 mul V
  hpt 2 mul 0 V
  hpt neg vpt 1.62 mul V closepath stroke
  Pnt  } def
/Star { 2 copy Pls Crs } def
/BoxF { stroke [] 0 setdash exch hpt sub exch vpt add M
  0 vpt2 neg V  hpt2 0 V  0 vpt2 V
  hpt2 neg 0 V  closepath fill } def
/TriUF { stroke [] 0 setdash vpt 1.12 mul add M
  hpt neg vpt -1.62 mul V
  hpt 2 mul 0 V
  hpt neg vpt 1.62 mul V closepath fill } def
/TriD { stroke [] 0 setdash 2 copy vpt 1.12 mul sub M
  hpt neg vpt 1.62 mul V
  hpt 2 mul 0 V
  hpt neg vpt -1.62 mul V closepath stroke
  Pnt  } def
/TriDF { stroke [] 0 setdash vpt 1.12 mul sub M
  hpt neg vpt 1.62 mul V
  hpt 2 mul 0 V
  hpt neg vpt -1.62 mul V closepath fill} def
/DiaF { stroke [] 0 setdash vpt add M
  hpt neg vpt neg V hpt vpt neg V
  hpt vpt V hpt neg vpt V closepath fill } def
/Pent { stroke [] 0 setdash 2 copy gsave
  translate 0 hpt M 4 {72 rotate 0 hpt L} repeat
  closepath stroke grestore Pnt } def
/PentF { stroke [] 0 setdash gsave
  translate 0 hpt M 4 {72 rotate 0 hpt L} repeat
  closepath fill grestore } def
/Circle { stroke [] 0 setdash 2 copy
  hpt 0 360 arc stroke Pnt } def
/CircleF { stroke [] 0 setdash hpt 0 360 arc fill } def
/C0 { BL [] 0 setdash 2 copy moveto vpt 90 450  arc } bind def
/C1 { BL [] 0 setdash 2 copy        moveto
       2 copy  vpt 0 90 arc closepath fill
               vpt 0 360 arc closepath } bind def
/C2 { BL [] 0 setdash 2 copy moveto
       2 copy  vpt 90 180 arc closepath fill
               vpt 0 360 arc closepath } bind def
/C3 { BL [] 0 setdash 2 copy moveto
       2 copy  vpt 0 180 arc closepath fill
               vpt 0 360 arc closepath } bind def
/C4 { BL [] 0 setdash 2 copy moveto
       2 copy  vpt 180 270 arc closepath fill
               vpt 0 360 arc closepath } bind def
/C5 { BL [] 0 setdash 2 copy moveto
       2 copy  vpt 0 90 arc
       2 copy moveto
       2 copy  vpt 180 270 arc closepath fill
               vpt 0 360 arc } bind def
/C6 { BL [] 0 setdash 2 copy moveto
      2 copy  vpt 90 270 arc closepath fill
              vpt 0 360 arc closepath } bind def
/C7 { BL [] 0 setdash 2 copy moveto
      2 copy  vpt 0 270 arc closepath fill
              vpt 0 360 arc closepath } bind def
/C8 { BL [] 0 setdash 2 copy moveto
      2 copy vpt 270 360 arc closepath fill
              vpt 0 360 arc closepath } bind def
/C9 { BL [] 0 setdash 2 copy moveto
      2 copy  vpt 270 450 arc closepath fill
              vpt 0 360 arc closepath } bind def
/C10 { BL [] 0 setdash 2 copy 2 copy moveto vpt 270 360 arc closepath fill
       2 copy moveto
       2 copy vpt 90 180 arc closepath fill
               vpt 0 360 arc closepath } bind def
/C11 { BL [] 0 setdash 2 copy moveto
       2 copy  vpt 0 180 arc closepath fill
       2 copy moveto
       2 copy  vpt 270 360 arc closepath fill
               vpt 0 360 arc closepath } bind def
/C12 { BL [] 0 setdash 2 copy moveto
       2 copy  vpt 180 360 arc closepath fill
               vpt 0 360 arc closepath } bind def
/C13 { BL [] 0 setdash  2 copy moveto
       2 copy  vpt 0 90 arc closepath fill
       2 copy moveto
       2 copy  vpt 180 360 arc closepath fill
               vpt 0 360 arc closepath } bind def
/C14 { BL [] 0 setdash 2 copy moveto
       2 copy  vpt 90 360 arc closepath fill
               vpt 0 360 arc } bind def
/C15 { BL [] 0 setdash 2 copy vpt 0 360 arc closepath fill
               vpt 0 360 arc closepath } bind def
/Rec   { newpath 4 2 roll moveto 1 index 0 rlineto 0 exch rlineto
       neg 0 rlineto closepath } bind def
/Square { dup Rec } bind def
/Bsquare { vpt sub exch vpt sub exch vpt2 Square } bind def
/S0 { BL [] 0 setdash 2 copy moveto 0 vpt rlineto BL Bsquare } bind def
/S1 { BL [] 0 setdash 2 copy vpt Square fill Bsquare } bind def
/S2 { BL [] 0 setdash 2 copy exch vpt sub exch vpt Square fill Bsquare } bind def
/S3 { BL [] 0 setdash 2 copy exch vpt sub exch vpt2 vpt Rec fill Bsquare } bind def
/S4 { BL [] 0 setdash 2 copy exch vpt sub exch vpt sub vpt Square fill Bsquare } bind def
/S5 { BL [] 0 setdash 2 copy 2 copy vpt Square fill
       exch vpt sub exch vpt sub vpt Square fill Bsquare } bind def
/S6 { BL [] 0 setdash 2 copy exch vpt sub exch vpt sub vpt vpt2 Rec fill Bsquare } bind def
/S7 { BL [] 0 setdash 2 copy exch vpt sub exch vpt sub vpt vpt2 Rec fill
       2 copy vpt Square fill
       Bsquare } bind def
/S8 { BL [] 0 setdash 2 copy vpt sub vpt Square fill Bsquare } bind def
/S9 { BL [] 0 setdash 2 copy vpt sub vpt vpt2 Rec fill Bsquare } bind def
/S10 { BL [] 0 setdash 2 copy vpt sub vpt Square fill 2 copy exch vpt sub exch vpt Square fill
       Bsquare } bind def
/S11 { BL [] 0 setdash 2 copy vpt sub vpt Square fill 2 copy exch vpt sub exch vpt2 vpt Rec fill
       Bsquare } bind def
/S12 { BL [] 0 setdash 2 copy exch vpt sub exch vpt sub vpt2 vpt Rec fill Bsquare } bind def
/S13 { BL [] 0 setdash 2 copy exch vpt sub exch vpt sub vpt2 vpt Rec fill
       2 copy vpt Square fill Bsquare } bind def
/S14 { BL [] 0 setdash 2 copy exch vpt sub exch vpt sub vpt2 vpt Rec fill
       2 copy exch vpt sub exch vpt Square fill Bsquare } bind def
/S15 { BL [] 0 setdash 2 copy Bsquare fill Bsquare } bind def
/D0 { gsave translate 45 rotate 0 0 S0 stroke grestore } bind def
/D1 { gsave translate 45 rotate 0 0 S1 stroke grestore } bind def
/D2 { gsave translate 45 rotate 0 0 S2 stroke grestore } bind def
/D3 { gsave translate 45 rotate 0 0 S3 stroke grestore } bind def
/D4 { gsave translate 45 rotate 0 0 S4 stroke grestore } bind def
/D5 { gsave translate 45 rotate 0 0 S5 stroke grestore } bind def
/D6 { gsave translate 45 rotate 0 0 S6 stroke grestore } bind def
/D7 { gsave translate 45 rotate 0 0 S7 stroke grestore } bind def
/D8 { gsave translate 45 rotate 0 0 S8 stroke grestore } bind def
/D9 { gsave translate 45 rotate 0 0 S9 stroke grestore } bind def
/D10 { gsave translate 45 rotate 0 0 S10 stroke grestore } bind def
/D11 { gsave translate 45 rotate 0 0 S11 stroke grestore } bind def
/D12 { gsave translate 45 rotate 0 0 S12 stroke grestore } bind def
/D13 { gsave translate 45 rotate 0 0 S13 stroke grestore } bind def
/D14 { gsave translate 45 rotate 0 0 S14 stroke grestore } bind def
/D15 { gsave translate 45 rotate 0 0 S15 stroke grestore } bind def
/DiaE { stroke [] 0 setdash vpt add M
  hpt neg vpt neg V hpt vpt neg V
  hpt vpt V hpt neg vpt V closepath stroke } def
/BoxE { stroke [] 0 setdash exch hpt sub exch vpt add M
  0 vpt2 neg V hpt2 0 V 0 vpt2 V
  hpt2 neg 0 V closepath stroke } def
/TriUE { stroke [] 0 setdash vpt 1.12 mul add M
  hpt neg vpt -1.62 mul V
  hpt 2 mul 0 V
  hpt neg vpt 1.62 mul V closepath stroke } def
/TriDE { stroke [] 0 setdash vpt 1.12 mul sub M
  hpt neg vpt 1.62 mul V
  hpt 2 mul 0 V
  hpt neg vpt -1.62 mul V closepath stroke } def
/PentE { stroke [] 0 setdash gsave
  translate 0 hpt M 4 {72 rotate 0 hpt L} repeat
  closepath stroke grestore } def
/CircE { stroke [] 0 setdash 
  hpt 0 360 arc stroke } def
/Opaque { gsave closepath 1 setgray fill grestore 0 setgray closepath } def
/DiaW { stroke [] 0 setdash vpt add M
  hpt neg vpt neg V hpt vpt neg V
  hpt vpt V hpt neg vpt V Opaque stroke } def
/BoxW { stroke [] 0 setdash exch hpt sub exch vpt add M
  0 vpt2 neg V hpt2 0 V 0 vpt2 V
  hpt2 neg 0 V Opaque stroke } def
/TriUW { stroke [] 0 setdash vpt 1.12 mul add M
  hpt neg vpt -1.62 mul V
  hpt 2 mul 0 V
  hpt neg vpt 1.62 mul V Opaque stroke } def
/TriDW { stroke [] 0 setdash vpt 1.12 mul sub M
  hpt neg vpt 1.62 mul V
  hpt 2 mul 0 V
  hpt neg vpt -1.62 mul V Opaque stroke } def
/PentW { stroke [] 0 setdash gsave
  translate 0 hpt M 4 {72 rotate 0 hpt L} repeat
  Opaque stroke grestore } def
/CircW { stroke [] 0 setdash 
  hpt 0 360 arc Opaque stroke } def
/BoxFill { gsave Rec 1 setgray fill grestore } def
end
%%EndProlog
}}%
\begin{picture}(3600,2160)(0,0)%
{\GNUPLOTspecial{"
gnudict begin
gsave
0 0 translate
0.100 0.100 scale
0 setgray
newpath
1.000 UL
LTb
350 300 M
63 0 V
3037 0 R
-63 0 V
350 551 M
63 0 V
3037 0 R
-63 0 V
350 803 M
63 0 V
3037 0 R
-63 0 V
350 1054 M
63 0 V
3037 0 R
-63 0 V
350 1306 M
63 0 V
3037 0 R
-63 0 V
350 1557 M
63 0 V
3037 0 R
-63 0 V
350 1809 M
63 0 V
3037 0 R
-63 0 V
350 2060 M
63 0 V
3037 0 R
-63 0 V
660 300 M
0 63 V
0 1697 R
0 -63 V
1280 300 M
0 63 V
0 1697 R
0 -63 V
1900 300 M
0 63 V
0 1697 R
0 -63 V
2520 300 M
0 63 V
0 1697 R
0 -63 V
3140 300 M
0 63 V
0 1697 R
0 -63 V
1.000 UL
LTb
350 300 M
3100 0 V
0 1760 V
-3100 0 V
350 300 L
1.000 UL
LT0
3087 1947 M
263 0 V
350 300 M
31 24 V
32 -15 V
31 3 V
31 22 V
32 -34 V
31 71 V
31 -37 V
32 93 V
31 900 V
31 574 V
31 -792 V
726 349 L
31 8 V
31 -4 V
32 -51 V
31 35 V
31 -32 V
32 11 V
31 2 V
31 -16 V
32 24 V
31 -23 V
31 18 V
32 6 V
31 -22 V
31 77 V
31 -70 V
32 233 V
31 988 V
31 301 V
32 -938 V
31 -586 V
31 65 V
32 -41 V
31 -26 V
31 27 V
32 -34 V
31 21 V
31 -10 V
32 -6 V
31 19 V
31 -25 V
31 29 V
32 -11 V
31 -1 V
31 67 V
32 -84 V
31 404 V
31 1003 V
32 0 V
1947 704 L
31 -404 V
32 84 V
31 -67 V
31 1 V
32 11 V
31 -29 V
31 25 V
31 -19 V
32 6 V
31 10 V
31 -21 V
32 34 V
31 -27 V
31 26 V
32 41 V
31 -65 V
31 586 V
32 938 V
31 -301 V
31 -988 V
32 -233 V
31 70 V
31 -77 V
31 22 V
32 -6 V
31 -18 V
31 23 V
32 -24 V
31 16 V
31 -2 V
32 -11 V
31 32 V
31 -35 V
32 51 V
31 4 V
31 -8 V
32 760 V
31 792 V
31 -574 V
31 -900 V
32 -93 V
31 37 V
31 -71 V
32 34 V
31 -22 V
31 -3 V
32 15 V
31 -24 V
stroke
grestore
end
showpage
}}%
\put(3037,1947){\makebox(0,0)[r]{N=8}}%
\put(1900,50){\makebox(0,0){$asin\theta/\lambda$}}%
\put(100,1180){%
\special{ps: gsave currentpoint currentpoint translate
270 rotate neg exch neg exch translate}%
\makebox(0,0)[b]{\shortstack{I}}%
\special{ps: currentpoint grestore moveto}%
}%
\put(3140,200){\makebox(0,0){2}}%
\put(2520,200){\makebox(0,0){1}}%
\put(1900,200){\makebox(0,0){0}}%
\put(1280,200){\makebox(0,0){-1}}%
\put(660,200){\makebox(0,0){-2}}%
\end{picture}%
\endgroup
\endinput

\caption{Eksempel p\aa\ interferensm\o nster med $N=2$ og $N=8$ lyskilder.\label{fig:interferensmoenster}} 
\end{figure}



\subsection{Diffraksjon}

Tenk dere at en sender inn en b\o lge mot en spalte\aa pning
som vist p\aa\ Figur \ref{fig:diffmonster}. 
Dersom spalte\aa pningen $d$ er p\aa\ st\o rrelse
med b\o lgelengden $\lambda$\footnote{Vi nevnte  dette ogs\aa\ i anledning 
de Broglies hypotese.}, kan b\o lgen som brer seg fra spalte\aa pningen
danne et interferens m\o nster. Vi sier at, i motsetning til interferens
hvor flere b\o lger summeres opp i et omr\aa de $P$, s\aa\ svarer diffraksjon til 
at det er  selve b\o lgen som vekselvirker med seg selv og danner et interferensm\o nster
ved skjermen. 
\begin{figure}[h]
   \setlength{\unitlength}{1mm}
   \begin{picture}(100,60)
   \put(25,0){\epsfxsize=10cm \epsfbox{fig2.eps}}
   \end{picture}
\caption{Skjematisk oppsett for lysstr\aa le som sendes mot en spalte\aa pning. Vinkelen
$\theta$ indikerer hvor f\o rste diffraksjonsminimum opptrer.\label{fig:diffmonster}} 
\end{figure}

Her skal vi all hovedsak kun anskueliggj\o re utledningen av
uttrykket for intensiteten
ved en skjerm langt borte fra spalte\aa pningen. 
Vi skal betrakte superposisjon
av b\o lger som reiser ut fra spalten og treffer skjermen ved 
et omr\aa de $P$ 
som har en avstand $R$ fra sentrum av spalten. Huygens prinsipp forteller
at hvert element $dy$ i spalte\aa pningen virker som en lyskilde, jfr.~diskusjonen
fra foreg\aa ende avsnitt om interferens. 
Igjen antar vi at avstanden er s\aa\ stor at vi kan anta tiln\ae rma parallelle
linjer for alle b\o lger som treffer omr\aa det $P$. 

Vi definerer
\begin{equation}
   R=\frac{1}{2}(N-1)dsin\theta+dy,
\end{equation}
dvs.~vi har delt opp spalten i $N$ deler $dy$. 
En b\o lge som blir utsendt i et punkt $y$ reiser derfor en 
avstand $R-ysin\theta$ for \aa\ komme til omr\aa det $P$. 
Bruker vi s\aa\ definisjonen av b\o lgefunksjonen
fra et punkt $dy$ (som svarer til $r_i$ i foreg\aa ende avsnitt)
\begin{equation}
  d\psi = Ae^{i(k(R-ysin\theta)-\omega t)}\frac{dy}{d},
\end{equation}
og integrerer vi opp bidragene fra hele spalten f\aa r vi
\begin{equation}
\psi=\int_{-ad/2}^{d/2}d\psi=\int_{-d/2}^{d/2}Ae^{i(k(R-ysin\theta)-\omega t)}\frac{dy}{d},
\end{equation}
som gir
\begin{equation}
    \psi=Ae^{i(kR-\omega t)}\frac{sin(\frac{\pi}{\lambda}dsin\theta)}{\frac{\pi}{\lambda}dsin\theta},
\end{equation}
som resulterer i en intensitet gitt ved 
\begin{equation}
I=I_0\frac{sin^2(\frac{\pi}{\lambda}dsin\theta)}{(\frac{\pi}{\lambda}dsin\theta)^2},
\end{equation}
eller
\begin{equation}
I=I_0\frac{sin^2(u)}{u^2},
\end{equation}
med
\begin{equation}
    u=\frac{\pi}{\lambda}dsin\theta.
\end{equation}
Dette forteller oss 
vi har maks intensitet n\aa r $u=0$ og min intensitet, dvs $I=0$  n\aa r
\begin{equation}
    u=n\pi \hspace{0.3cm} n=\pm 1, \pm 2, \dots ,
\end{equation}
eller
\begin{equation}
    dsin\theta_n=n\lambda.
\end{equation}  
\begin{figure}[h]
   \setlength{\unitlength}{1mm}
   \begin{picture}(100,60)
   \put(25,0){\epsfxsize=10cm \epsfbox{fig4.eps}}
   \end{picture}
\caption{Skjematisk oppsett for lysstr\aa le som sendes mot to spalte\aa pninger
med \aa pning $d$ og avstand mellom \aa pningene gitt ved $a$.} 
\end{figure}
Dersom vi har to spalter med avstand $a$ og spalte\aa pning $d$, se Figur 2.6,
finner vi at intensiteten er gitt ved
\be
I=I_0cos^2(\frac{\pi}{\lambda}asin\theta)\frac{sin^2(\frac{\pi}{\lambda}dsin\theta)}{(\frac{\pi}{\lambda}dsin\theta)^2},
\label{eq:diffint}
\end{equation}
som er relevant n\aa r en studerer Bragg spredning.
I dette tilfelle har vi konstruktiv interferens b\aa de n\aa r 
\[ 
    dsin\theta_n=\frac{(2n+1)\lambda}{2}.
\]
og n\aa r
\[ 
    asin\theta_n=n\lambda.
\]
Eksempler for diffraksjonsm\o nster med henholdsvis en og to spalte\aa pninger
er vist i Figur \ref{fig:diff}. 
Legg merke til at for tilfellet med
to spalter har vi satt avstanden mellom spalte\aa pningene $a$ lik st\o rrelsen
p\aa\ spalte\aa pnigen $d$.  
\begin{figure}
% GNUPLOT: LaTeX picture with Postscript
\begingroup%
  \makeatletter%
  \newcommand{\GNUPLOTspecial}{%
    \@sanitize\catcode`\%=14\relax\special}%
  \setlength{\unitlength}{0.1bp}%
{\GNUPLOTspecial{!
%!PS-Adobe-2.0 EPSF-2.0
%%Title: diff1.tex
%%Creator: gnuplot 3.7 patchlevel 0.2
%%CreationDate: Fri Feb 11 10:29:02 2000
%%DocumentFonts: 
%%BoundingBox: 0 0 360 216
%%Orientation: Landscape
%%EndComments
/gnudict 256 dict def
gnudict begin
/Color false def
/Solid false def
/gnulinewidth 5.000 def
/userlinewidth gnulinewidth def
/vshift -33 def
/dl {10 mul} def
/hpt_ 31.5 def
/vpt_ 31.5 def
/hpt hpt_ def
/vpt vpt_ def
/M {moveto} bind def
/L {lineto} bind def
/R {rmoveto} bind def
/V {rlineto} bind def
/vpt2 vpt 2 mul def
/hpt2 hpt 2 mul def
/Lshow { currentpoint stroke M
  0 vshift R show } def
/Rshow { currentpoint stroke M
  dup stringwidth pop neg vshift R show } def
/Cshow { currentpoint stroke M
  dup stringwidth pop -2 div vshift R show } def
/UP { dup vpt_ mul /vpt exch def hpt_ mul /hpt exch def
  /hpt2 hpt 2 mul def /vpt2 vpt 2 mul def } def
/DL { Color {setrgbcolor Solid {pop []} if 0 setdash }
 {pop pop pop Solid {pop []} if 0 setdash} ifelse } def
/BL { stroke userlinewidth 2 mul setlinewidth } def
/AL { stroke userlinewidth 2 div setlinewidth } def
/UL { dup gnulinewidth mul /userlinewidth exch def
      10 mul /udl exch def } def
/PL { stroke userlinewidth setlinewidth } def
/LTb { BL [] 0 0 0 DL } def
/LTa { AL [1 udl mul 2 udl mul] 0 setdash 0 0 0 setrgbcolor } def
/LT0 { PL [] 1 0 0 DL } def
/LT1 { PL [4 dl 2 dl] 0 1 0 DL } def
/LT2 { PL [2 dl 3 dl] 0 0 1 DL } def
/LT3 { PL [1 dl 1.5 dl] 1 0 1 DL } def
/LT4 { PL [5 dl 2 dl 1 dl 2 dl] 0 1 1 DL } def
/LT5 { PL [4 dl 3 dl 1 dl 3 dl] 1 1 0 DL } def
/LT6 { PL [2 dl 2 dl 2 dl 4 dl] 0 0 0 DL } def
/LT7 { PL [2 dl 2 dl 2 dl 2 dl 2 dl 4 dl] 1 0.3 0 DL } def
/LT8 { PL [2 dl 2 dl 2 dl 2 dl 2 dl 2 dl 2 dl 4 dl] 0.5 0.5 0.5 DL } def
/Pnt { stroke [] 0 setdash
   gsave 1 setlinecap M 0 0 V stroke grestore } def
/Dia { stroke [] 0 setdash 2 copy vpt add M
  hpt neg vpt neg V hpt vpt neg V
  hpt vpt V hpt neg vpt V closepath stroke
  Pnt } def
/Pls { stroke [] 0 setdash vpt sub M 0 vpt2 V
  currentpoint stroke M
  hpt neg vpt neg R hpt2 0 V stroke
  } def
/Box { stroke [] 0 setdash 2 copy exch hpt sub exch vpt add M
  0 vpt2 neg V hpt2 0 V 0 vpt2 V
  hpt2 neg 0 V closepath stroke
  Pnt } def
/Crs { stroke [] 0 setdash exch hpt sub exch vpt add M
  hpt2 vpt2 neg V currentpoint stroke M
  hpt2 neg 0 R hpt2 vpt2 V stroke } def
/TriU { stroke [] 0 setdash 2 copy vpt 1.12 mul add M
  hpt neg vpt -1.62 mul V
  hpt 2 mul 0 V
  hpt neg vpt 1.62 mul V closepath stroke
  Pnt  } def
/Star { 2 copy Pls Crs } def
/BoxF { stroke [] 0 setdash exch hpt sub exch vpt add M
  0 vpt2 neg V  hpt2 0 V  0 vpt2 V
  hpt2 neg 0 V  closepath fill } def
/TriUF { stroke [] 0 setdash vpt 1.12 mul add M
  hpt neg vpt -1.62 mul V
  hpt 2 mul 0 V
  hpt neg vpt 1.62 mul V closepath fill } def
/TriD { stroke [] 0 setdash 2 copy vpt 1.12 mul sub M
  hpt neg vpt 1.62 mul V
  hpt 2 mul 0 V
  hpt neg vpt -1.62 mul V closepath stroke
  Pnt  } def
/TriDF { stroke [] 0 setdash vpt 1.12 mul sub M
  hpt neg vpt 1.62 mul V
  hpt 2 mul 0 V
  hpt neg vpt -1.62 mul V closepath fill} def
/DiaF { stroke [] 0 setdash vpt add M
  hpt neg vpt neg V hpt vpt neg V
  hpt vpt V hpt neg vpt V closepath fill } def
/Pent { stroke [] 0 setdash 2 copy gsave
  translate 0 hpt M 4 {72 rotate 0 hpt L} repeat
  closepath stroke grestore Pnt } def
/PentF { stroke [] 0 setdash gsave
  translate 0 hpt M 4 {72 rotate 0 hpt L} repeat
  closepath fill grestore } def
/Circle { stroke [] 0 setdash 2 copy
  hpt 0 360 arc stroke Pnt } def
/CircleF { stroke [] 0 setdash hpt 0 360 arc fill } def
/C0 { BL [] 0 setdash 2 copy moveto vpt 90 450  arc } bind def
/C1 { BL [] 0 setdash 2 copy        moveto
       2 copy  vpt 0 90 arc closepath fill
               vpt 0 360 arc closepath } bind def
/C2 { BL [] 0 setdash 2 copy moveto
       2 copy  vpt 90 180 arc closepath fill
               vpt 0 360 arc closepath } bind def
/C3 { BL [] 0 setdash 2 copy moveto
       2 copy  vpt 0 180 arc closepath fill
               vpt 0 360 arc closepath } bind def
/C4 { BL [] 0 setdash 2 copy moveto
       2 copy  vpt 180 270 arc closepath fill
               vpt 0 360 arc closepath } bind def
/C5 { BL [] 0 setdash 2 copy moveto
       2 copy  vpt 0 90 arc
       2 copy moveto
       2 copy  vpt 180 270 arc closepath fill
               vpt 0 360 arc } bind def
/C6 { BL [] 0 setdash 2 copy moveto
      2 copy  vpt 90 270 arc closepath fill
              vpt 0 360 arc closepath } bind def
/C7 { BL [] 0 setdash 2 copy moveto
      2 copy  vpt 0 270 arc closepath fill
              vpt 0 360 arc closepath } bind def
/C8 { BL [] 0 setdash 2 copy moveto
      2 copy vpt 270 360 arc closepath fill
              vpt 0 360 arc closepath } bind def
/C9 { BL [] 0 setdash 2 copy moveto
      2 copy  vpt 270 450 arc closepath fill
              vpt 0 360 arc closepath } bind def
/C10 { BL [] 0 setdash 2 copy 2 copy moveto vpt 270 360 arc closepath fill
       2 copy moveto
       2 copy vpt 90 180 arc closepath fill
               vpt 0 360 arc closepath } bind def
/C11 { BL [] 0 setdash 2 copy moveto
       2 copy  vpt 0 180 arc closepath fill
       2 copy moveto
       2 copy  vpt 270 360 arc closepath fill
               vpt 0 360 arc closepath } bind def
/C12 { BL [] 0 setdash 2 copy moveto
       2 copy  vpt 180 360 arc closepath fill
               vpt 0 360 arc closepath } bind def
/C13 { BL [] 0 setdash  2 copy moveto
       2 copy  vpt 0 90 arc closepath fill
       2 copy moveto
       2 copy  vpt 180 360 arc closepath fill
               vpt 0 360 arc closepath } bind def
/C14 { BL [] 0 setdash 2 copy moveto
       2 copy  vpt 90 360 arc closepath fill
               vpt 0 360 arc } bind def
/C15 { BL [] 0 setdash 2 copy vpt 0 360 arc closepath fill
               vpt 0 360 arc closepath } bind def
/Rec   { newpath 4 2 roll moveto 1 index 0 rlineto 0 exch rlineto
       neg 0 rlineto closepath } bind def
/Square { dup Rec } bind def
/Bsquare { vpt sub exch vpt sub exch vpt2 Square } bind def
/S0 { BL [] 0 setdash 2 copy moveto 0 vpt rlineto BL Bsquare } bind def
/S1 { BL [] 0 setdash 2 copy vpt Square fill Bsquare } bind def
/S2 { BL [] 0 setdash 2 copy exch vpt sub exch vpt Square fill Bsquare } bind def
/S3 { BL [] 0 setdash 2 copy exch vpt sub exch vpt2 vpt Rec fill Bsquare } bind def
/S4 { BL [] 0 setdash 2 copy exch vpt sub exch vpt sub vpt Square fill Bsquare } bind def
/S5 { BL [] 0 setdash 2 copy 2 copy vpt Square fill
       exch vpt sub exch vpt sub vpt Square fill Bsquare } bind def
/S6 { BL [] 0 setdash 2 copy exch vpt sub exch vpt sub vpt vpt2 Rec fill Bsquare } bind def
/S7 { BL [] 0 setdash 2 copy exch vpt sub exch vpt sub vpt vpt2 Rec fill
       2 copy vpt Square fill
       Bsquare } bind def
/S8 { BL [] 0 setdash 2 copy vpt sub vpt Square fill Bsquare } bind def
/S9 { BL [] 0 setdash 2 copy vpt sub vpt vpt2 Rec fill Bsquare } bind def
/S10 { BL [] 0 setdash 2 copy vpt sub vpt Square fill 2 copy exch vpt sub exch vpt Square fill
       Bsquare } bind def
/S11 { BL [] 0 setdash 2 copy vpt sub vpt Square fill 2 copy exch vpt sub exch vpt2 vpt Rec fill
       Bsquare } bind def
/S12 { BL [] 0 setdash 2 copy exch vpt sub exch vpt sub vpt2 vpt Rec fill Bsquare } bind def
/S13 { BL [] 0 setdash 2 copy exch vpt sub exch vpt sub vpt2 vpt Rec fill
       2 copy vpt Square fill Bsquare } bind def
/S14 { BL [] 0 setdash 2 copy exch vpt sub exch vpt sub vpt2 vpt Rec fill
       2 copy exch vpt sub exch vpt Square fill Bsquare } bind def
/S15 { BL [] 0 setdash 2 copy Bsquare fill Bsquare } bind def
/D0 { gsave translate 45 rotate 0 0 S0 stroke grestore } bind def
/D1 { gsave translate 45 rotate 0 0 S1 stroke grestore } bind def
/D2 { gsave translate 45 rotate 0 0 S2 stroke grestore } bind def
/D3 { gsave translate 45 rotate 0 0 S3 stroke grestore } bind def
/D4 { gsave translate 45 rotate 0 0 S4 stroke grestore } bind def
/D5 { gsave translate 45 rotate 0 0 S5 stroke grestore } bind def
/D6 { gsave translate 45 rotate 0 0 S6 stroke grestore } bind def
/D7 { gsave translate 45 rotate 0 0 S7 stroke grestore } bind def
/D8 { gsave translate 45 rotate 0 0 S8 stroke grestore } bind def
/D9 { gsave translate 45 rotate 0 0 S9 stroke grestore } bind def
/D10 { gsave translate 45 rotate 0 0 S10 stroke grestore } bind def
/D11 { gsave translate 45 rotate 0 0 S11 stroke grestore } bind def
/D12 { gsave translate 45 rotate 0 0 S12 stroke grestore } bind def
/D13 { gsave translate 45 rotate 0 0 S13 stroke grestore } bind def
/D14 { gsave translate 45 rotate 0 0 S14 stroke grestore } bind def
/D15 { gsave translate 45 rotate 0 0 S15 stroke grestore } bind def
/DiaE { stroke [] 0 setdash vpt add M
  hpt neg vpt neg V hpt vpt neg V
  hpt vpt V hpt neg vpt V closepath stroke } def
/BoxE { stroke [] 0 setdash exch hpt sub exch vpt add M
  0 vpt2 neg V hpt2 0 V 0 vpt2 V
  hpt2 neg 0 V closepath stroke } def
/TriUE { stroke [] 0 setdash vpt 1.12 mul add M
  hpt neg vpt -1.62 mul V
  hpt 2 mul 0 V
  hpt neg vpt 1.62 mul V closepath stroke } def
/TriDE { stroke [] 0 setdash vpt 1.12 mul sub M
  hpt neg vpt 1.62 mul V
  hpt 2 mul 0 V
  hpt neg vpt -1.62 mul V closepath stroke } def
/PentE { stroke [] 0 setdash gsave
  translate 0 hpt M 4 {72 rotate 0 hpt L} repeat
  closepath stroke grestore } def
/CircE { stroke [] 0 setdash 
  hpt 0 360 arc stroke } def
/Opaque { gsave closepath 1 setgray fill grestore 0 setgray closepath } def
/DiaW { stroke [] 0 setdash vpt add M
  hpt neg vpt neg V hpt vpt neg V
  hpt vpt V hpt neg vpt V Opaque stroke } def
/BoxW { stroke [] 0 setdash exch hpt sub exch vpt add M
  0 vpt2 neg V hpt2 0 V 0 vpt2 V
  hpt2 neg 0 V Opaque stroke } def
/TriUW { stroke [] 0 setdash vpt 1.12 mul add M
  hpt neg vpt -1.62 mul V
  hpt 2 mul 0 V
  hpt neg vpt 1.62 mul V Opaque stroke } def
/TriDW { stroke [] 0 setdash vpt 1.12 mul sub M
  hpt neg vpt 1.62 mul V
  hpt 2 mul 0 V
  hpt neg vpt -1.62 mul V Opaque stroke } def
/PentW { stroke [] 0 setdash gsave
  translate 0 hpt M 4 {72 rotate 0 hpt L} repeat
  Opaque stroke grestore } def
/CircW { stroke [] 0 setdash 
  hpt 0 360 arc Opaque stroke } def
/BoxFill { gsave Rec 1 setgray fill grestore } def
end
%%EndProlog
}}%
\begin{picture}(3600,2160)(0,0)%
{\GNUPLOTspecial{"
gnudict begin
gsave
0 0 translate
0.100 0.100 scale
0 setgray
newpath
1.000 UL
LTb
400 300 M
63 0 V
2987 0 R
-63 0 V
400 476 M
63 0 V
2987 0 R
-63 0 V
400 652 M
63 0 V
2987 0 R
-63 0 V
400 828 M
63 0 V
2987 0 R
-63 0 V
400 1004 M
63 0 V
2987 0 R
-63 0 V
400 1180 M
63 0 V
2987 0 R
-63 0 V
400 1356 M
63 0 V
2987 0 R
-63 0 V
400 1532 M
63 0 V
2987 0 R
-63 0 V
400 1708 M
63 0 V
2987 0 R
-63 0 V
400 1884 M
63 0 V
2987 0 R
-63 0 V
400 2060 M
63 0 V
2987 0 R
-63 0 V
705 300 M
0 63 V
0 1697 R
0 -63 V
1315 300 M
0 63 V
0 1697 R
0 -63 V
1925 300 M
0 63 V
0 1697 R
0 -63 V
2535 300 M
0 63 V
0 1697 R
0 -63 V
3145 300 M
0 63 V
0 1697 R
0 -63 V
1.000 UL
LTb
400 300 M
3050 0 V
0 1760 V
-3050 0 V
400 300 L
1.000 UL
LT0
3087 1947 M
263 0 V
400 328 M
31 0 V
31 -1 V
30 -2 V
31 -4 V
31 -4 V
31 -5 V
31 -5 V
30 -4 V
31 -2 V
31 -1 V
31 1 V
31 4 V
31 6 V
30 9 V
31 10 V
31 11 V
31 12 V
31 10 V
30 9 V
31 6 V
31 3 V
31 -1 V
31 -6 V
30 -9 V
31 -13 V
31 -14 V
31 -15 V
31 -13 V
30 -8 V
31 -2 V
31 8 V
31 20 V
31 35 V
30 50 V
31 67 V
31 84 V
31 102 V
31 116 V
31 130 V
30 140 V
31 145 V
31 146 V
31 144 V
31 134 V
30 121 V
31 103 V
31 81 V
31 56 V
31 29 V
30 0 V
31 -29 V
31 -56 V
31 -81 V
31 -103 V
30 -121 V
31 -134 V
31 -144 V
31 -146 V
31 -145 V
30 -140 V
31 -130 V
31 -116 V
31 -102 V
31 -84 V
31 -67 V
30 -50 V
31 -35 V
31 -20 V
31 -8 V
31 2 V
30 8 V
31 13 V
31 15 V
31 14 V
31 13 V
30 9 V
31 6 V
31 1 V
31 -3 V
31 -6 V
30 -9 V
31 -10 V
31 -12 V
31 -11 V
31 -10 V
30 -9 V
31 -6 V
31 -4 V
31 -1 V
31 1 V
31 2 V
30 4 V
31 5 V
31 5 V
31 4 V
31 4 V
30 2 V
31 1 V
31 0 V
stroke
grestore
end
showpage
}}%
\put(3037,1947){\makebox(0,0)[r]{I for 1 spalt}}%
\put(100,1180){%
\special{ps: gsave currentpoint currentpoint translate
270 rotate neg exch neg exch translate}%
\makebox(0,0)[b]{\shortstack{I}}%
\special{ps: currentpoint grestore moveto}%
}%
\put(3145,200){\makebox(0,0){2}}%
\put(2535,200){\makebox(0,0){1}}%
\put(1925,200){\makebox(0,0){0}}%
\put(1315,200){\makebox(0,0){-1}}%
\put(705,200){\makebox(0,0){-2}}%
\end{picture}%
\endgroup
\endinput

% GNUPLOT: LaTeX picture with Postscript
\begingroup%
  \makeatletter%
  \newcommand{\GNUPLOTspecial}{%
    \@sanitize\catcode`\%=14\relax\special}%
  \setlength{\unitlength}{0.1bp}%
{\GNUPLOTspecial{!
%!PS-Adobe-2.0 EPSF-2.0
%%Title: diff2.tex
%%Creator: gnuplot 3.7 patchlevel 0.2
%%CreationDate: Fri Feb 11 10:31:02 2000
%%DocumentFonts: 
%%BoundingBox: 0 0 360 216
%%Orientation: Landscape
%%EndComments
/gnudict 256 dict def
gnudict begin
/Color false def
/Solid false def
/gnulinewidth 5.000 def
/userlinewidth gnulinewidth def
/vshift -33 def
/dl {10 mul} def
/hpt_ 31.5 def
/vpt_ 31.5 def
/hpt hpt_ def
/vpt vpt_ def
/M {moveto} bind def
/L {lineto} bind def
/R {rmoveto} bind def
/V {rlineto} bind def
/vpt2 vpt 2 mul def
/hpt2 hpt 2 mul def
/Lshow { currentpoint stroke M
  0 vshift R show } def
/Rshow { currentpoint stroke M
  dup stringwidth pop neg vshift R show } def
/Cshow { currentpoint stroke M
  dup stringwidth pop -2 div vshift R show } def
/UP { dup vpt_ mul /vpt exch def hpt_ mul /hpt exch def
  /hpt2 hpt 2 mul def /vpt2 vpt 2 mul def } def
/DL { Color {setrgbcolor Solid {pop []} if 0 setdash }
 {pop pop pop Solid {pop []} if 0 setdash} ifelse } def
/BL { stroke userlinewidth 2 mul setlinewidth } def
/AL { stroke userlinewidth 2 div setlinewidth } def
/UL { dup gnulinewidth mul /userlinewidth exch def
      10 mul /udl exch def } def
/PL { stroke userlinewidth setlinewidth } def
/LTb { BL [] 0 0 0 DL } def
/LTa { AL [1 udl mul 2 udl mul] 0 setdash 0 0 0 setrgbcolor } def
/LT0 { PL [] 1 0 0 DL } def
/LT1 { PL [4 dl 2 dl] 0 1 0 DL } def
/LT2 { PL [2 dl 3 dl] 0 0 1 DL } def
/LT3 { PL [1 dl 1.5 dl] 1 0 1 DL } def
/LT4 { PL [5 dl 2 dl 1 dl 2 dl] 0 1 1 DL } def
/LT5 { PL [4 dl 3 dl 1 dl 3 dl] 1 1 0 DL } def
/LT6 { PL [2 dl 2 dl 2 dl 4 dl] 0 0 0 DL } def
/LT7 { PL [2 dl 2 dl 2 dl 2 dl 2 dl 4 dl] 1 0.3 0 DL } def
/LT8 { PL [2 dl 2 dl 2 dl 2 dl 2 dl 2 dl 2 dl 4 dl] 0.5 0.5 0.5 DL } def
/Pnt { stroke [] 0 setdash
   gsave 1 setlinecap M 0 0 V stroke grestore } def
/Dia { stroke [] 0 setdash 2 copy vpt add M
  hpt neg vpt neg V hpt vpt neg V
  hpt vpt V hpt neg vpt V closepath stroke
  Pnt } def
/Pls { stroke [] 0 setdash vpt sub M 0 vpt2 V
  currentpoint stroke M
  hpt neg vpt neg R hpt2 0 V stroke
  } def
/Box { stroke [] 0 setdash 2 copy exch hpt sub exch vpt add M
  0 vpt2 neg V hpt2 0 V 0 vpt2 V
  hpt2 neg 0 V closepath stroke
  Pnt } def
/Crs { stroke [] 0 setdash exch hpt sub exch vpt add M
  hpt2 vpt2 neg V currentpoint stroke M
  hpt2 neg 0 R hpt2 vpt2 V stroke } def
/TriU { stroke [] 0 setdash 2 copy vpt 1.12 mul add M
  hpt neg vpt -1.62 mul V
  hpt 2 mul 0 V
  hpt neg vpt 1.62 mul V closepath stroke
  Pnt  } def
/Star { 2 copy Pls Crs } def
/BoxF { stroke [] 0 setdash exch hpt sub exch vpt add M
  0 vpt2 neg V  hpt2 0 V  0 vpt2 V
  hpt2 neg 0 V  closepath fill } def
/TriUF { stroke [] 0 setdash vpt 1.12 mul add M
  hpt neg vpt -1.62 mul V
  hpt 2 mul 0 V
  hpt neg vpt 1.62 mul V closepath fill } def
/TriD { stroke [] 0 setdash 2 copy vpt 1.12 mul sub M
  hpt neg vpt 1.62 mul V
  hpt 2 mul 0 V
  hpt neg vpt -1.62 mul V closepath stroke
  Pnt  } def
/TriDF { stroke [] 0 setdash vpt 1.12 mul sub M
  hpt neg vpt 1.62 mul V
  hpt 2 mul 0 V
  hpt neg vpt -1.62 mul V closepath fill} def
/DiaF { stroke [] 0 setdash vpt add M
  hpt neg vpt neg V hpt vpt neg V
  hpt vpt V hpt neg vpt V closepath fill } def
/Pent { stroke [] 0 setdash 2 copy gsave
  translate 0 hpt M 4 {72 rotate 0 hpt L} repeat
  closepath stroke grestore Pnt } def
/PentF { stroke [] 0 setdash gsave
  translate 0 hpt M 4 {72 rotate 0 hpt L} repeat
  closepath fill grestore } def
/Circle { stroke [] 0 setdash 2 copy
  hpt 0 360 arc stroke Pnt } def
/CircleF { stroke [] 0 setdash hpt 0 360 arc fill } def
/C0 { BL [] 0 setdash 2 copy moveto vpt 90 450  arc } bind def
/C1 { BL [] 0 setdash 2 copy        moveto
       2 copy  vpt 0 90 arc closepath fill
               vpt 0 360 arc closepath } bind def
/C2 { BL [] 0 setdash 2 copy moveto
       2 copy  vpt 90 180 arc closepath fill
               vpt 0 360 arc closepath } bind def
/C3 { BL [] 0 setdash 2 copy moveto
       2 copy  vpt 0 180 arc closepath fill
               vpt 0 360 arc closepath } bind def
/C4 { BL [] 0 setdash 2 copy moveto
       2 copy  vpt 180 270 arc closepath fill
               vpt 0 360 arc closepath } bind def
/C5 { BL [] 0 setdash 2 copy moveto
       2 copy  vpt 0 90 arc
       2 copy moveto
       2 copy  vpt 180 270 arc closepath fill
               vpt 0 360 arc } bind def
/C6 { BL [] 0 setdash 2 copy moveto
      2 copy  vpt 90 270 arc closepath fill
              vpt 0 360 arc closepath } bind def
/C7 { BL [] 0 setdash 2 copy moveto
      2 copy  vpt 0 270 arc closepath fill
              vpt 0 360 arc closepath } bind def
/C8 { BL [] 0 setdash 2 copy moveto
      2 copy vpt 270 360 arc closepath fill
              vpt 0 360 arc closepath } bind def
/C9 { BL [] 0 setdash 2 copy moveto
      2 copy  vpt 270 450 arc closepath fill
              vpt 0 360 arc closepath } bind def
/C10 { BL [] 0 setdash 2 copy 2 copy moveto vpt 270 360 arc closepath fill
       2 copy moveto
       2 copy vpt 90 180 arc closepath fill
               vpt 0 360 arc closepath } bind def
/C11 { BL [] 0 setdash 2 copy moveto
       2 copy  vpt 0 180 arc closepath fill
       2 copy moveto
       2 copy  vpt 270 360 arc closepath fill
               vpt 0 360 arc closepath } bind def
/C12 { BL [] 0 setdash 2 copy moveto
       2 copy  vpt 180 360 arc closepath fill
               vpt 0 360 arc closepath } bind def
/C13 { BL [] 0 setdash  2 copy moveto
       2 copy  vpt 0 90 arc closepath fill
       2 copy moveto
       2 copy  vpt 180 360 arc closepath fill
               vpt 0 360 arc closepath } bind def
/C14 { BL [] 0 setdash 2 copy moveto
       2 copy  vpt 90 360 arc closepath fill
               vpt 0 360 arc } bind def
/C15 { BL [] 0 setdash 2 copy vpt 0 360 arc closepath fill
               vpt 0 360 arc closepath } bind def
/Rec   { newpath 4 2 roll moveto 1 index 0 rlineto 0 exch rlineto
       neg 0 rlineto closepath } bind def
/Square { dup Rec } bind def
/Bsquare { vpt sub exch vpt sub exch vpt2 Square } bind def
/S0 { BL [] 0 setdash 2 copy moveto 0 vpt rlineto BL Bsquare } bind def
/S1 { BL [] 0 setdash 2 copy vpt Square fill Bsquare } bind def
/S2 { BL [] 0 setdash 2 copy exch vpt sub exch vpt Square fill Bsquare } bind def
/S3 { BL [] 0 setdash 2 copy exch vpt sub exch vpt2 vpt Rec fill Bsquare } bind def
/S4 { BL [] 0 setdash 2 copy exch vpt sub exch vpt sub vpt Square fill Bsquare } bind def
/S5 { BL [] 0 setdash 2 copy 2 copy vpt Square fill
       exch vpt sub exch vpt sub vpt Square fill Bsquare } bind def
/S6 { BL [] 0 setdash 2 copy exch vpt sub exch vpt sub vpt vpt2 Rec fill Bsquare } bind def
/S7 { BL [] 0 setdash 2 copy exch vpt sub exch vpt sub vpt vpt2 Rec fill
       2 copy vpt Square fill
       Bsquare } bind def
/S8 { BL [] 0 setdash 2 copy vpt sub vpt Square fill Bsquare } bind def
/S9 { BL [] 0 setdash 2 copy vpt sub vpt vpt2 Rec fill Bsquare } bind def
/S10 { BL [] 0 setdash 2 copy vpt sub vpt Square fill 2 copy exch vpt sub exch vpt Square fill
       Bsquare } bind def
/S11 { BL [] 0 setdash 2 copy vpt sub vpt Square fill 2 copy exch vpt sub exch vpt2 vpt Rec fill
       Bsquare } bind def
/S12 { BL [] 0 setdash 2 copy exch vpt sub exch vpt sub vpt2 vpt Rec fill Bsquare } bind def
/S13 { BL [] 0 setdash 2 copy exch vpt sub exch vpt sub vpt2 vpt Rec fill
       2 copy vpt Square fill Bsquare } bind def
/S14 { BL [] 0 setdash 2 copy exch vpt sub exch vpt sub vpt2 vpt Rec fill
       2 copy exch vpt sub exch vpt Square fill Bsquare } bind def
/S15 { BL [] 0 setdash 2 copy Bsquare fill Bsquare } bind def
/D0 { gsave translate 45 rotate 0 0 S0 stroke grestore } bind def
/D1 { gsave translate 45 rotate 0 0 S1 stroke grestore } bind def
/D2 { gsave translate 45 rotate 0 0 S2 stroke grestore } bind def
/D3 { gsave translate 45 rotate 0 0 S3 stroke grestore } bind def
/D4 { gsave translate 45 rotate 0 0 S4 stroke grestore } bind def
/D5 { gsave translate 45 rotate 0 0 S5 stroke grestore } bind def
/D6 { gsave translate 45 rotate 0 0 S6 stroke grestore } bind def
/D7 { gsave translate 45 rotate 0 0 S7 stroke grestore } bind def
/D8 { gsave translate 45 rotate 0 0 S8 stroke grestore } bind def
/D9 { gsave translate 45 rotate 0 0 S9 stroke grestore } bind def
/D10 { gsave translate 45 rotate 0 0 S10 stroke grestore } bind def
/D11 { gsave translate 45 rotate 0 0 S11 stroke grestore } bind def
/D12 { gsave translate 45 rotate 0 0 S12 stroke grestore } bind def
/D13 { gsave translate 45 rotate 0 0 S13 stroke grestore } bind def
/D14 { gsave translate 45 rotate 0 0 S14 stroke grestore } bind def
/D15 { gsave translate 45 rotate 0 0 S15 stroke grestore } bind def
/DiaE { stroke [] 0 setdash vpt add M
  hpt neg vpt neg V hpt vpt neg V
  hpt vpt V hpt neg vpt V closepath stroke } def
/BoxE { stroke [] 0 setdash exch hpt sub exch vpt add M
  0 vpt2 neg V hpt2 0 V 0 vpt2 V
  hpt2 neg 0 V closepath stroke } def
/TriUE { stroke [] 0 setdash vpt 1.12 mul add M
  hpt neg vpt -1.62 mul V
  hpt 2 mul 0 V
  hpt neg vpt 1.62 mul V closepath stroke } def
/TriDE { stroke [] 0 setdash vpt 1.12 mul sub M
  hpt neg vpt 1.62 mul V
  hpt 2 mul 0 V
  hpt neg vpt -1.62 mul V closepath stroke } def
/PentE { stroke [] 0 setdash gsave
  translate 0 hpt M 4 {72 rotate 0 hpt L} repeat
  closepath stroke grestore } def
/CircE { stroke [] 0 setdash 
  hpt 0 360 arc stroke } def
/Opaque { gsave closepath 1 setgray fill grestore 0 setgray closepath } def
/DiaW { stroke [] 0 setdash vpt add M
  hpt neg vpt neg V hpt vpt neg V
  hpt vpt V hpt neg vpt V Opaque stroke } def
/BoxW { stroke [] 0 setdash exch hpt sub exch vpt add M
  0 vpt2 neg V hpt2 0 V 0 vpt2 V
  hpt2 neg 0 V Opaque stroke } def
/TriUW { stroke [] 0 setdash vpt 1.12 mul add M
  hpt neg vpt -1.62 mul V
  hpt 2 mul 0 V
  hpt neg vpt 1.62 mul V Opaque stroke } def
/TriDW { stroke [] 0 setdash vpt 1.12 mul sub M
  hpt neg vpt 1.62 mul V
  hpt 2 mul 0 V
  hpt neg vpt -1.62 mul V Opaque stroke } def
/PentW { stroke [] 0 setdash gsave
  translate 0 hpt M 4 {72 rotate 0 hpt L} repeat
  Opaque stroke grestore } def
/CircW { stroke [] 0 setdash 
  hpt 0 360 arc Opaque stroke } def
/BoxFill { gsave Rec 1 setgray fill grestore } def
end
%%EndProlog
}}%
\begin{picture}(3600,2160)(0,0)%
{\GNUPLOTspecial{"
gnudict begin
gsave
0 0 translate
0.100 0.100 scale
0 setgray
newpath
1.000 UL
LTb
400 300 M
63 0 V
2987 0 R
-63 0 V
400 476 M
63 0 V
2987 0 R
-63 0 V
400 652 M
63 0 V
2987 0 R
-63 0 V
400 828 M
63 0 V
2987 0 R
-63 0 V
400 1004 M
63 0 V
2987 0 R
-63 0 V
400 1180 M
63 0 V
2987 0 R
-63 0 V
400 1356 M
63 0 V
2987 0 R
-63 0 V
400 1532 M
63 0 V
2987 0 R
-63 0 V
400 1708 M
63 0 V
2987 0 R
-63 0 V
400 1884 M
63 0 V
2987 0 R
-63 0 V
400 2060 M
63 0 V
2987 0 R
-63 0 V
705 300 M
0 63 V
0 1697 R
0 -63 V
1315 300 M
0 63 V
0 1697 R
0 -63 V
1925 300 M
0 63 V
0 1697 R
0 -63 V
2535 300 M
0 63 V
0 1697 R
0 -63 V
3145 300 M
0 63 V
0 1697 R
0 -63 V
1.000 UL
LTb
400 300 M
3050 0 V
0 1760 V
-3050 0 V
400 300 L
1.000 UL
LT0
3087 1947 M
263 0 V
400 300 M
31 1 V
31 2 V
30 2 V
31 2 V
31 2 V
31 -1 V
31 -2 V
30 -3 V
31 -2 V
31 -1 V
31 1 V
31 4 V
31 4 V
30 4 V
31 1 V
31 -1 V
31 -3 V
31 -5 V
30 -4 V
31 -1 V
31 3 V
31 6 V
31 9 V
30 7 V
31 3 V
31 -2 V
31 -7 V
31 -9 V
30 -8 V
31 -2 V
31 8 V
31 17 V
31 22 V
30 21 V
31 12 V
31 -3 V
31 -18 V
31 -29 V
31 -24 V
30 -3 V
31 41 V
31 100 V
31 169 V
31 233 V
30 278 V
31 292 V
31 268 V
31 205 V
31 111 V
30 0 V
31 -111 V
31 -205 V
31 -268 V
31 -292 V
30 -278 V
31 -233 V
31 -169 V
31 -100 V
31 -41 V
30 3 V
31 24 V
31 29 V
31 18 V
31 3 V
31 -12 V
30 -21 V
31 -22 V
31 -17 V
31 -8 V
31 2 V
30 8 V
31 9 V
31 7 V
31 2 V
31 -3 V
30 -7 V
31 -9 V
31 -6 V
31 -3 V
31 1 V
30 4 V
31 5 V
31 3 V
31 1 V
31 -1 V
30 -4 V
31 -4 V
31 -4 V
31 -1 V
31 1 V
31 2 V
30 3 V
31 2 V
31 1 V
31 -2 V
31 -2 V
30 -2 V
31 -2 V
31 -1 V
stroke
grestore
end
showpage
}}%
\put(3037,1947){\makebox(0,0)[r]{I for 2 spalter}}%
\put(1925,50){\makebox(0,0){$u=asin\theta/\lambda$}}%
\put(100,1180){%
\special{ps: gsave currentpoint currentpoint translate
270 rotate neg exch neg exch translate}%
\makebox(0,0)[b]{\shortstack{I}}%
\special{ps: currentpoint grestore moveto}%
}%
\put(3145,200){\makebox(0,0){2}}%
\put(2535,200){\makebox(0,0){1}}%
\put(1925,200){\makebox(0,0){0}}%
\put(1315,200){\makebox(0,0){-1}}%
\put(705,200){\makebox(0,0){-2}}%
\end{picture}%
\endgroup
\endinput

\caption{Eksempel p\aa\ interferensm\o nster med henhodsvis 1 og 2 spalte\aa pninger. Her har vi for enkelthetsskyld satt $a=d$ for tilfellet med to spalter.
\label{fig:diff}} 
\end{figure}

\subsection{Fourieranalyse og uskarphetsrelasjonen}
Her skal vi pr\o ve \aa\ bygge en bro mellom begrep som b\o lgepakker
og gruppehastighet og neste avsnitt om Heisenbergs uskarphetsrelasjon.

La oss repetere litt av diskusjonen rundt Figurene \ref{fig:fasev} og 
\ref{fig:waveex}. 
De fleste av dere er kanskje vant med at en b\o lge er noe som kan 
likne p\aa\ den harmoniske svingningen vist i Figur \ref{fig:fasev}. Problemet her er
at b\o lgen v\aa r har en utstrekning i rom som er uendelig. Det blir dermed
vanskelig \aa\ tilordne en partikkel eller f.eks.~en h\o yttaler puls, eller
en radiob\o lge. Skal vi kople b\o lgebeskrivelsen til en beskrivelse av
materie, m\aa\ b\o lgen
ha ei begrenset utstrekning i rom. Vi vil da bruke b\o lgen for \aa\ uttrykke
en sannsynlighet for \aa\ finne partikkelen et bestemt sted. Er utbredelsen
uendelig, kan vi ikke si noe som helst om hvor partikkelen er. 

Det vi ogs\aa\ la merke til var at dersom vi satte at hastigheten til 
partikkelen skulle svare til fasehastigheten, fikk vi problemer med
de Broglie sitt postulat. 
Vi \o nsker derfor at v\aa r b\o lge skal ha en form som likner mer p\aa\ 
hva vi ser i Figur \ref{fig:waveex}. Rekningen v\aa r viste at dersom vi introduserer
begrepet gruppehastighet $v_g$, dvs.~hastigheten som b\o lgepakken (som er satt
sammen av flere b\o lger) reiser med, fant vi at partikkelens hastighet svarte
til gruppehastigheten, 

For \aa\ f\aa\ til dette finnes det et teorem
fra Fourier som sier at dersom vi \o nsker at v\aa r b\o lge skal
v\ae re forskjellig fra null kun i et bestemt omr\aa de i rommet, 
m\aa\ vi integrere
over alle frekvenser $\nu$ eller b\o lgetall $k$. 

F\o lgende eksempel illustrer dette. Eksemplet tillatter 
oss ogs\aa\ \aa\ lage ei kopling til   
Heisenbergs uskarphetsrelasjon.

Anta at du er i stand til \aa\ lage en firkantpuls som vist i neste figur.
Denne pulsen er sentrert rundt et bestemt b\o lgetall $k_0$ (eller frekvens
om du foretrekker det). Den er null
n\aa r $k < k_0-\Delta $ og $k > k_0+\Delta $. 

\begin{figure}[htbp]
%
\begin{center}

\setlength{\unitlength}{1cm}
\begin{picture}(13,6)

\thicklines

   \put(0,0.5){\makebox(0,0)[bl]{
              \put(0,1){\vector(1,0){12}}
              \put(12.3,1){\makebox(0,0){k}}
              \put(5.2,0.8){\makebox(0,0){$k_0-\Delta$}}
              \put(8.2,0.8){\makebox(0,0){$k_0+\Delta$}}
              \put(6.7,0.8){\makebox(0,0){$k_0$}}
              \put(5,1){\line(0,1){3}}
              \put(5,4){\line(1,0){3}}
              \put(8,1){\line(0,1){3}}
         }}
\end{picture}
\end{center}
\caption{Eksempel p\aa\ firkantpuls sentrert rundt et b\o lgetall $k_0$.} 
\end{figure}

Vi kan n\aa\ tenke oss at vi har en b\o lgefunksjon $\psi(x)$ 
ved tida $t=0$ gitt som et 
integral over alle mulige b\o lgetall $k$.  Vi antar ogs\aa\ at pulsen v\aa r
har en amplitude lik $A(k)=1$ for hvert b\o lgetall som faller innenfor
det tillatte intervallet, slik at b\o lgefunksjonen blir da
\be
   \psi(x)=\int_{-\infty}^{\infty}A(k)cos(2\pi kx)dk=
           \int_{k_0-\Delta}^{k_0+\Delta}cos(2\pi kx)dk.
\end{equation}
Her har vi valgt en cosinus funksjon for hver $k$-verdi. I tillegg s\aa\ er 
pulsen v\aa r forskjellig fra null for kun bestemte verdier av $k$, 
slik at vi kan redusere integralet til et lite omr\aa de. 

Dette integralet kan lett l\o ses, og vi finner
\begin{equation}
   \psi(x)=\frac{1}{2\pi x}\int_{2\pi(k_0-\Delta)x}^{2\pi(k_0+\Delta)x}
           cos(2\pi kx)d(2\pi kx),
\end{equation}
som gir
\begin{equation}
  \psi(x)=\frac{1}{2\pi x}\left[sin(2\pi (k_0+\Delta)x)-
                                sin(2\pi (k_0-\Delta)x)\right],
\end{equation}
som gir
\begin{equation}
    \psi(x)=\frac{1}{\pi x}\left[cos(2\pi k_0x)sin(2\pi \Delta x)\right],
\end{equation}
eller
\begin{equation}
    \psi(x)=2\Delta cos(2\pi k_0x)\frac{sin(2\pi \Delta x)}{2\pi \Delta x}.
\end{equation}

La oss plotte  $\psi$ som funksjon av $x$ for to ulike verdier av $\Delta$.
Velg f\o rst $\Delta =0.001$, dvs.~at v\aa r puls er skarpt
bestemt rundt verdien $k_0$. Deretter velger vi $\Delta=100$, som betyr at
vi integrerer over mange flere frekvenser. 
\begin{figure}
% GNUPLOT: LaTeX picture with Postscript
\begingroup%
  \makeatletter%
  \newcommand{\GNUPLOTspecial}{%
    \@sanitize\catcode`\%=14\relax\special}%
  \setlength{\unitlength}{0.1bp}%
{\GNUPLOTspecial{!
%!PS-Adobe-2.0 EPSF-2.0
%%Title: fourier1.tex
%%Creator: gnuplot 3.7 patchlevel 1
%%CreationDate: Thu Feb 15 20:26:47 2001
%%DocumentFonts: 
%%BoundingBox: 0 0 360 216
%%Orientation: Landscape
%%EndComments
/gnudict 256 dict def
gnudict begin
/Color false def
/Solid false def
/gnulinewidth 5.000 def
/userlinewidth gnulinewidth def
/vshift -33 def
/dl {10 mul} def
/hpt_ 31.5 def
/vpt_ 31.5 def
/hpt hpt_ def
/vpt vpt_ def
/M {moveto} bind def
/L {lineto} bind def
/R {rmoveto} bind def
/V {rlineto} bind def
/vpt2 vpt 2 mul def
/hpt2 hpt 2 mul def
/Lshow { currentpoint stroke M
  0 vshift R show } def
/Rshow { currentpoint stroke M
  dup stringwidth pop neg vshift R show } def
/Cshow { currentpoint stroke M
  dup stringwidth pop -2 div vshift R show } def
/UP { dup vpt_ mul /vpt exch def hpt_ mul /hpt exch def
  /hpt2 hpt 2 mul def /vpt2 vpt 2 mul def } def
/DL { Color {setrgbcolor Solid {pop []} if 0 setdash }
 {pop pop pop Solid {pop []} if 0 setdash} ifelse } def
/BL { stroke userlinewidth 2 mul setlinewidth } def
/AL { stroke userlinewidth 2 div setlinewidth } def
/UL { dup gnulinewidth mul /userlinewidth exch def
      10 mul /udl exch def } def
/PL { stroke userlinewidth setlinewidth } def
/LTb { BL [] 0 0 0 DL } def
/LTa { AL [1 udl mul 2 udl mul] 0 setdash 0 0 0 setrgbcolor } def
/LT0 { PL [] 1 0 0 DL } def
/LT1 { PL [4 dl 2 dl] 0 1 0 DL } def
/LT2 { PL [2 dl 3 dl] 0 0 1 DL } def
/LT3 { PL [1 dl 1.5 dl] 1 0 1 DL } def
/LT4 { PL [5 dl 2 dl 1 dl 2 dl] 0 1 1 DL } def
/LT5 { PL [4 dl 3 dl 1 dl 3 dl] 1 1 0 DL } def
/LT6 { PL [2 dl 2 dl 2 dl 4 dl] 0 0 0 DL } def
/LT7 { PL [2 dl 2 dl 2 dl 2 dl 2 dl 4 dl] 1 0.3 0 DL } def
/LT8 { PL [2 dl 2 dl 2 dl 2 dl 2 dl 2 dl 2 dl 4 dl] 0.5 0.5 0.5 DL } def
/Pnt { stroke [] 0 setdash
   gsave 1 setlinecap M 0 0 V stroke grestore } def
/Dia { stroke [] 0 setdash 2 copy vpt add M
  hpt neg vpt neg V hpt vpt neg V
  hpt vpt V hpt neg vpt V closepath stroke
  Pnt } def
/Pls { stroke [] 0 setdash vpt sub M 0 vpt2 V
  currentpoint stroke M
  hpt neg vpt neg R hpt2 0 V stroke
  } def
/Box { stroke [] 0 setdash 2 copy exch hpt sub exch vpt add M
  0 vpt2 neg V hpt2 0 V 0 vpt2 V
  hpt2 neg 0 V closepath stroke
  Pnt } def
/Crs { stroke [] 0 setdash exch hpt sub exch vpt add M
  hpt2 vpt2 neg V currentpoint stroke M
  hpt2 neg 0 R hpt2 vpt2 V stroke } def
/TriU { stroke [] 0 setdash 2 copy vpt 1.12 mul add M
  hpt neg vpt -1.62 mul V
  hpt 2 mul 0 V
  hpt neg vpt 1.62 mul V closepath stroke
  Pnt  } def
/Star { 2 copy Pls Crs } def
/BoxF { stroke [] 0 setdash exch hpt sub exch vpt add M
  0 vpt2 neg V  hpt2 0 V  0 vpt2 V
  hpt2 neg 0 V  closepath fill } def
/TriUF { stroke [] 0 setdash vpt 1.12 mul add M
  hpt neg vpt -1.62 mul V
  hpt 2 mul 0 V
  hpt neg vpt 1.62 mul V closepath fill } def
/TriD { stroke [] 0 setdash 2 copy vpt 1.12 mul sub M
  hpt neg vpt 1.62 mul V
  hpt 2 mul 0 V
  hpt neg vpt -1.62 mul V closepath stroke
  Pnt  } def
/TriDF { stroke [] 0 setdash vpt 1.12 mul sub M
  hpt neg vpt 1.62 mul V
  hpt 2 mul 0 V
  hpt neg vpt -1.62 mul V closepath fill} def
/DiaF { stroke [] 0 setdash vpt add M
  hpt neg vpt neg V hpt vpt neg V
  hpt vpt V hpt neg vpt V closepath fill } def
/Pent { stroke [] 0 setdash 2 copy gsave
  translate 0 hpt M 4 {72 rotate 0 hpt L} repeat
  closepath stroke grestore Pnt } def
/PentF { stroke [] 0 setdash gsave
  translate 0 hpt M 4 {72 rotate 0 hpt L} repeat
  closepath fill grestore } def
/Circle { stroke [] 0 setdash 2 copy
  hpt 0 360 arc stroke Pnt } def
/CircleF { stroke [] 0 setdash hpt 0 360 arc fill } def
/C0 { BL [] 0 setdash 2 copy moveto vpt 90 450  arc } bind def
/C1 { BL [] 0 setdash 2 copy        moveto
       2 copy  vpt 0 90 arc closepath fill
               vpt 0 360 arc closepath } bind def
/C2 { BL [] 0 setdash 2 copy moveto
       2 copy  vpt 90 180 arc closepath fill
               vpt 0 360 arc closepath } bind def
/C3 { BL [] 0 setdash 2 copy moveto
       2 copy  vpt 0 180 arc closepath fill
               vpt 0 360 arc closepath } bind def
/C4 { BL [] 0 setdash 2 copy moveto
       2 copy  vpt 180 270 arc closepath fill
               vpt 0 360 arc closepath } bind def
/C5 { BL [] 0 setdash 2 copy moveto
       2 copy  vpt 0 90 arc
       2 copy moveto
       2 copy  vpt 180 270 arc closepath fill
               vpt 0 360 arc } bind def
/C6 { BL [] 0 setdash 2 copy moveto
      2 copy  vpt 90 270 arc closepath fill
              vpt 0 360 arc closepath } bind def
/C7 { BL [] 0 setdash 2 copy moveto
      2 copy  vpt 0 270 arc closepath fill
              vpt 0 360 arc closepath } bind def
/C8 { BL [] 0 setdash 2 copy moveto
      2 copy vpt 270 360 arc closepath fill
              vpt 0 360 arc closepath } bind def
/C9 { BL [] 0 setdash 2 copy moveto
      2 copy  vpt 270 450 arc closepath fill
              vpt 0 360 arc closepath } bind def
/C10 { BL [] 0 setdash 2 copy 2 copy moveto vpt 270 360 arc closepath fill
       2 copy moveto
       2 copy vpt 90 180 arc closepath fill
               vpt 0 360 arc closepath } bind def
/C11 { BL [] 0 setdash 2 copy moveto
       2 copy  vpt 0 180 arc closepath fill
       2 copy moveto
       2 copy  vpt 270 360 arc closepath fill
               vpt 0 360 arc closepath } bind def
/C12 { BL [] 0 setdash 2 copy moveto
       2 copy  vpt 180 360 arc closepath fill
               vpt 0 360 arc closepath } bind def
/C13 { BL [] 0 setdash  2 copy moveto
       2 copy  vpt 0 90 arc closepath fill
       2 copy moveto
       2 copy  vpt 180 360 arc closepath fill
               vpt 0 360 arc closepath } bind def
/C14 { BL [] 0 setdash 2 copy moveto
       2 copy  vpt 90 360 arc closepath fill
               vpt 0 360 arc } bind def
/C15 { BL [] 0 setdash 2 copy vpt 0 360 arc closepath fill
               vpt 0 360 arc closepath } bind def
/Rec   { newpath 4 2 roll moveto 1 index 0 rlineto 0 exch rlineto
       neg 0 rlineto closepath } bind def
/Square { dup Rec } bind def
/Bsquare { vpt sub exch vpt sub exch vpt2 Square } bind def
/S0 { BL [] 0 setdash 2 copy moveto 0 vpt rlineto BL Bsquare } bind def
/S1 { BL [] 0 setdash 2 copy vpt Square fill Bsquare } bind def
/S2 { BL [] 0 setdash 2 copy exch vpt sub exch vpt Square fill Bsquare } bind def
/S3 { BL [] 0 setdash 2 copy exch vpt sub exch vpt2 vpt Rec fill Bsquare } bind def
/S4 { BL [] 0 setdash 2 copy exch vpt sub exch vpt sub vpt Square fill Bsquare } bind def
/S5 { BL [] 0 setdash 2 copy 2 copy vpt Square fill
       exch vpt sub exch vpt sub vpt Square fill Bsquare } bind def
/S6 { BL [] 0 setdash 2 copy exch vpt sub exch vpt sub vpt vpt2 Rec fill Bsquare } bind def
/S7 { BL [] 0 setdash 2 copy exch vpt sub exch vpt sub vpt vpt2 Rec fill
       2 copy vpt Square fill
       Bsquare } bind def
/S8 { BL [] 0 setdash 2 copy vpt sub vpt Square fill Bsquare } bind def
/S9 { BL [] 0 setdash 2 copy vpt sub vpt vpt2 Rec fill Bsquare } bind def
/S10 { BL [] 0 setdash 2 copy vpt sub vpt Square fill 2 copy exch vpt sub exch vpt Square fill
       Bsquare } bind def
/S11 { BL [] 0 setdash 2 copy vpt sub vpt Square fill 2 copy exch vpt sub exch vpt2 vpt Rec fill
       Bsquare } bind def
/S12 { BL [] 0 setdash 2 copy exch vpt sub exch vpt sub vpt2 vpt Rec fill Bsquare } bind def
/S13 { BL [] 0 setdash 2 copy exch vpt sub exch vpt sub vpt2 vpt Rec fill
       2 copy vpt Square fill Bsquare } bind def
/S14 { BL [] 0 setdash 2 copy exch vpt sub exch vpt sub vpt2 vpt Rec fill
       2 copy exch vpt sub exch vpt Square fill Bsquare } bind def
/S15 { BL [] 0 setdash 2 copy Bsquare fill Bsquare } bind def
/D0 { gsave translate 45 rotate 0 0 S0 stroke grestore } bind def
/D1 { gsave translate 45 rotate 0 0 S1 stroke grestore } bind def
/D2 { gsave translate 45 rotate 0 0 S2 stroke grestore } bind def
/D3 { gsave translate 45 rotate 0 0 S3 stroke grestore } bind def
/D4 { gsave translate 45 rotate 0 0 S4 stroke grestore } bind def
/D5 { gsave translate 45 rotate 0 0 S5 stroke grestore } bind def
/D6 { gsave translate 45 rotate 0 0 S6 stroke grestore } bind def
/D7 { gsave translate 45 rotate 0 0 S7 stroke grestore } bind def
/D8 { gsave translate 45 rotate 0 0 S8 stroke grestore } bind def
/D9 { gsave translate 45 rotate 0 0 S9 stroke grestore } bind def
/D10 { gsave translate 45 rotate 0 0 S10 stroke grestore } bind def
/D11 { gsave translate 45 rotate 0 0 S11 stroke grestore } bind def
/D12 { gsave translate 45 rotate 0 0 S12 stroke grestore } bind def
/D13 { gsave translate 45 rotate 0 0 S13 stroke grestore } bind def
/D14 { gsave translate 45 rotate 0 0 S14 stroke grestore } bind def
/D15 { gsave translate 45 rotate 0 0 S15 stroke grestore } bind def
/DiaE { stroke [] 0 setdash vpt add M
  hpt neg vpt neg V hpt vpt neg V
  hpt vpt V hpt neg vpt V closepath stroke } def
/BoxE { stroke [] 0 setdash exch hpt sub exch vpt add M
  0 vpt2 neg V hpt2 0 V 0 vpt2 V
  hpt2 neg 0 V closepath stroke } def
/TriUE { stroke [] 0 setdash vpt 1.12 mul add M
  hpt neg vpt -1.62 mul V
  hpt 2 mul 0 V
  hpt neg vpt 1.62 mul V closepath stroke } def
/TriDE { stroke [] 0 setdash vpt 1.12 mul sub M
  hpt neg vpt 1.62 mul V
  hpt 2 mul 0 V
  hpt neg vpt -1.62 mul V closepath stroke } def
/PentE { stroke [] 0 setdash gsave
  translate 0 hpt M 4 {72 rotate 0 hpt L} repeat
  closepath stroke grestore } def
/CircE { stroke [] 0 setdash 
  hpt 0 360 arc stroke } def
/Opaque { gsave closepath 1 setgray fill grestore 0 setgray closepath } def
/DiaW { stroke [] 0 setdash vpt add M
  hpt neg vpt neg V hpt vpt neg V
  hpt vpt V hpt neg vpt V Opaque stroke } def
/BoxW { stroke [] 0 setdash exch hpt sub exch vpt add M
  0 vpt2 neg V hpt2 0 V 0 vpt2 V
  hpt2 neg 0 V Opaque stroke } def
/TriUW { stroke [] 0 setdash vpt 1.12 mul add M
  hpt neg vpt -1.62 mul V
  hpt 2 mul 0 V
  hpt neg vpt 1.62 mul V Opaque stroke } def
/TriDW { stroke [] 0 setdash vpt 1.12 mul sub M
  hpt neg vpt 1.62 mul V
  hpt 2 mul 0 V
  hpt neg vpt -1.62 mul V Opaque stroke } def
/PentW { stroke [] 0 setdash gsave
  translate 0 hpt M 4 {72 rotate 0 hpt L} repeat
  Opaque stroke grestore } def
/CircW { stroke [] 0 setdash 
  hpt 0 360 arc Opaque stroke } def
/BoxFill { gsave Rec 1 setgray fill grestore } def
end
%%EndProlog
}}%
\begin{picture}(3600,2160)(0,0)%
{\GNUPLOTspecial{"
gnudict begin
gsave
0 0 translate
0.100 0.100 scale
0 setgray
newpath
1.000 UL
LTb
550 300 M
63 0 V
2837 0 R
-63 0 V
550 520 M
63 0 V
2837 0 R
-63 0 V
550 740 M
63 0 V
2837 0 R
-63 0 V
550 960 M
63 0 V
2837 0 R
-63 0 V
550 1180 M
63 0 V
2837 0 R
-63 0 V
550 1400 M
63 0 V
2837 0 R
-63 0 V
550 1620 M
63 0 V
2837 0 R
-63 0 V
550 1840 M
63 0 V
2837 0 R
-63 0 V
550 2060 M
63 0 V
2837 0 R
-63 0 V
550 300 M
0 63 V
0 1697 R
0 -63 V
1275 300 M
0 63 V
0 1697 R
0 -63 V
2000 300 M
0 63 V
0 1697 R
0 -63 V
2725 300 M
0 63 V
0 1697 R
0 -63 V
3450 300 M
0 63 V
0 1697 R
0 -63 V
1.000 UL
LTb
550 300 M
2900 0 V
0 1760 V
-2900 0 V
550 300 L
1.000 UL
LT0
3087 1947 M
263 0 V
550 1620 M
29 -309 V
609 818 L
29 16 V
29 503 V
29 282 V
30 -335 V
755 803 L
29 49 V
30 511 V
29 253 V
29 -360 V
902 789 L
29 82 V
29 517 V
29 224 V
30 -383 V
29 -452 V
29 115 V
30 520 V
29 194 V
29 -405 V
29 -434 V
30 147 V
29 521 V
29 163 V
30 -425 V
29 -415 V
29 178 V
29 521 V
30 131 V
29 -443 V
29 -394 V
30 209 V
29 519 V
29 98 V
30 -460 V
29 -371 V
29 239 V
29 513 V
30 66 V
29 -474 V
29 -348 V
30 268 V
29 507 V
29 33 V
29 -487 V
30 -323 V
29 296 V
29 498 V
30 0 V
29 -498 V
29 -296 V
30 323 V
29 487 V
29 -33 V
29 -507 V
30 -268 V
29 348 V
29 474 V
30 -66 V
29 -513 V
29 -239 V
29 371 V
30 460 V
29 -98 V
29 -519 V
30 -209 V
29 394 V
29 443 V
30 -131 V
29 -521 V
29 -178 V
29 415 V
30 425 V
29 -163 V
29 -521 V
30 -147 V
29 434 V
29 405 V
29 -194 V
30 -520 V
29 -115 V
29 452 V
30 383 V
29 -224 V
29 -517 V
29 -82 V
30 467 V
29 360 V
29 -253 V
30 -511 V
29 -49 V
29 481 V
30 335 V
29 -282 V
29 -503 V
29 -16 V
30 493 V
29 309 V
stroke
grestore
end
showpage
}}%
\put(3037,1947){\makebox(0,0)[r]{$\psi$ for $\Delta=0.001$}}%
\put(2000,50){\makebox(0,0){$x$}}%
\put(100,1180){%
\special{ps: gsave currentpoint currentpoint translate
270 rotate neg exch neg exch translate}%
\makebox(0,0)[b]{\shortstack{$\psi(x)$}}%
\special{ps: currentpoint grestore moveto}%
}%
\put(3450,200){\makebox(0,0){10}}%
\put(2725,200){\makebox(0,0){5}}%
\put(2000,200){\makebox(0,0){0}}%
\put(1275,200){\makebox(0,0){-5}}%
\put(550,200){\makebox(0,0){-10}}%
\put(500,2060){\makebox(0,0)[r]{0.004}}%
\put(500,1840){\makebox(0,0)[r]{0.003}}%
\put(500,1620){\makebox(0,0)[r]{0.002}}%
\put(500,1400){\makebox(0,0)[r]{0.001}}%
\put(500,1180){\makebox(0,0)[r]{0}}%
\put(500,960){\makebox(0,0)[r]{-0.001}}%
\put(500,740){\makebox(0,0)[r]{-0.002}}%
\put(500,520){\makebox(0,0)[r]{-0.003}}%
\put(500,300){\makebox(0,0)[r]{-0.004}}%
\end{picture}%
\endgroup
\endinput

% GNUPLOT: LaTeX picture with Postscript
\begingroup%
  \makeatletter%
  \newcommand{\GNUPLOTspecial}{%
    \@sanitize\catcode`\%=14\relax\special}%
  \setlength{\unitlength}{0.1bp}%
{\GNUPLOTspecial{!
%!PS-Adobe-2.0 EPSF-2.0
%%Title: fourier2.tex
%%Creator: gnuplot 3.7 patchlevel 1
%%CreationDate: Thu Feb 15 20:23:41 2001
%%DocumentFonts: 
%%BoundingBox: 0 0 360 216
%%Orientation: Landscape
%%EndComments
/gnudict 256 dict def
gnudict begin
/Color false def
/Solid false def
/gnulinewidth 5.000 def
/userlinewidth gnulinewidth def
/vshift -33 def
/dl {10 mul} def
/hpt_ 31.5 def
/vpt_ 31.5 def
/hpt hpt_ def
/vpt vpt_ def
/M {moveto} bind def
/L {lineto} bind def
/R {rmoveto} bind def
/V {rlineto} bind def
/vpt2 vpt 2 mul def
/hpt2 hpt 2 mul def
/Lshow { currentpoint stroke M
  0 vshift R show } def
/Rshow { currentpoint stroke M
  dup stringwidth pop neg vshift R show } def
/Cshow { currentpoint stroke M
  dup stringwidth pop -2 div vshift R show } def
/UP { dup vpt_ mul /vpt exch def hpt_ mul /hpt exch def
  /hpt2 hpt 2 mul def /vpt2 vpt 2 mul def } def
/DL { Color {setrgbcolor Solid {pop []} if 0 setdash }
 {pop pop pop Solid {pop []} if 0 setdash} ifelse } def
/BL { stroke userlinewidth 2 mul setlinewidth } def
/AL { stroke userlinewidth 2 div setlinewidth } def
/UL { dup gnulinewidth mul /userlinewidth exch def
      10 mul /udl exch def } def
/PL { stroke userlinewidth setlinewidth } def
/LTb { BL [] 0 0 0 DL } def
/LTa { AL [1 udl mul 2 udl mul] 0 setdash 0 0 0 setrgbcolor } def
/LT0 { PL [] 1 0 0 DL } def
/LT1 { PL [4 dl 2 dl] 0 1 0 DL } def
/LT2 { PL [2 dl 3 dl] 0 0 1 DL } def
/LT3 { PL [1 dl 1.5 dl] 1 0 1 DL } def
/LT4 { PL [5 dl 2 dl 1 dl 2 dl] 0 1 1 DL } def
/LT5 { PL [4 dl 3 dl 1 dl 3 dl] 1 1 0 DL } def
/LT6 { PL [2 dl 2 dl 2 dl 4 dl] 0 0 0 DL } def
/LT7 { PL [2 dl 2 dl 2 dl 2 dl 2 dl 4 dl] 1 0.3 0 DL } def
/LT8 { PL [2 dl 2 dl 2 dl 2 dl 2 dl 2 dl 2 dl 4 dl] 0.5 0.5 0.5 DL } def
/Pnt { stroke [] 0 setdash
   gsave 1 setlinecap M 0 0 V stroke grestore } def
/Dia { stroke [] 0 setdash 2 copy vpt add M
  hpt neg vpt neg V hpt vpt neg V
  hpt vpt V hpt neg vpt V closepath stroke
  Pnt } def
/Pls { stroke [] 0 setdash vpt sub M 0 vpt2 V
  currentpoint stroke M
  hpt neg vpt neg R hpt2 0 V stroke
  } def
/Box { stroke [] 0 setdash 2 copy exch hpt sub exch vpt add M
  0 vpt2 neg V hpt2 0 V 0 vpt2 V
  hpt2 neg 0 V closepath stroke
  Pnt } def
/Crs { stroke [] 0 setdash exch hpt sub exch vpt add M
  hpt2 vpt2 neg V currentpoint stroke M
  hpt2 neg 0 R hpt2 vpt2 V stroke } def
/TriU { stroke [] 0 setdash 2 copy vpt 1.12 mul add M
  hpt neg vpt -1.62 mul V
  hpt 2 mul 0 V
  hpt neg vpt 1.62 mul V closepath stroke
  Pnt  } def
/Star { 2 copy Pls Crs } def
/BoxF { stroke [] 0 setdash exch hpt sub exch vpt add M
  0 vpt2 neg V  hpt2 0 V  0 vpt2 V
  hpt2 neg 0 V  closepath fill } def
/TriUF { stroke [] 0 setdash vpt 1.12 mul add M
  hpt neg vpt -1.62 mul V
  hpt 2 mul 0 V
  hpt neg vpt 1.62 mul V closepath fill } def
/TriD { stroke [] 0 setdash 2 copy vpt 1.12 mul sub M
  hpt neg vpt 1.62 mul V
  hpt 2 mul 0 V
  hpt neg vpt -1.62 mul V closepath stroke
  Pnt  } def
/TriDF { stroke [] 0 setdash vpt 1.12 mul sub M
  hpt neg vpt 1.62 mul V
  hpt 2 mul 0 V
  hpt neg vpt -1.62 mul V closepath fill} def
/DiaF { stroke [] 0 setdash vpt add M
  hpt neg vpt neg V hpt vpt neg V
  hpt vpt V hpt neg vpt V closepath fill } def
/Pent { stroke [] 0 setdash 2 copy gsave
  translate 0 hpt M 4 {72 rotate 0 hpt L} repeat
  closepath stroke grestore Pnt } def
/PentF { stroke [] 0 setdash gsave
  translate 0 hpt M 4 {72 rotate 0 hpt L} repeat
  closepath fill grestore } def
/Circle { stroke [] 0 setdash 2 copy
  hpt 0 360 arc stroke Pnt } def
/CircleF { stroke [] 0 setdash hpt 0 360 arc fill } def
/C0 { BL [] 0 setdash 2 copy moveto vpt 90 450  arc } bind def
/C1 { BL [] 0 setdash 2 copy        moveto
       2 copy  vpt 0 90 arc closepath fill
               vpt 0 360 arc closepath } bind def
/C2 { BL [] 0 setdash 2 copy moveto
       2 copy  vpt 90 180 arc closepath fill
               vpt 0 360 arc closepath } bind def
/C3 { BL [] 0 setdash 2 copy moveto
       2 copy  vpt 0 180 arc closepath fill
               vpt 0 360 arc closepath } bind def
/C4 { BL [] 0 setdash 2 copy moveto
       2 copy  vpt 180 270 arc closepath fill
               vpt 0 360 arc closepath } bind def
/C5 { BL [] 0 setdash 2 copy moveto
       2 copy  vpt 0 90 arc
       2 copy moveto
       2 copy  vpt 180 270 arc closepath fill
               vpt 0 360 arc } bind def
/C6 { BL [] 0 setdash 2 copy moveto
      2 copy  vpt 90 270 arc closepath fill
              vpt 0 360 arc closepath } bind def
/C7 { BL [] 0 setdash 2 copy moveto
      2 copy  vpt 0 270 arc closepath fill
              vpt 0 360 arc closepath } bind def
/C8 { BL [] 0 setdash 2 copy moveto
      2 copy vpt 270 360 arc closepath fill
              vpt 0 360 arc closepath } bind def
/C9 { BL [] 0 setdash 2 copy moveto
      2 copy  vpt 270 450 arc closepath fill
              vpt 0 360 arc closepath } bind def
/C10 { BL [] 0 setdash 2 copy 2 copy moveto vpt 270 360 arc closepath fill
       2 copy moveto
       2 copy vpt 90 180 arc closepath fill
               vpt 0 360 arc closepath } bind def
/C11 { BL [] 0 setdash 2 copy moveto
       2 copy  vpt 0 180 arc closepath fill
       2 copy moveto
       2 copy  vpt 270 360 arc closepath fill
               vpt 0 360 arc closepath } bind def
/C12 { BL [] 0 setdash 2 copy moveto
       2 copy  vpt 180 360 arc closepath fill
               vpt 0 360 arc closepath } bind def
/C13 { BL [] 0 setdash  2 copy moveto
       2 copy  vpt 0 90 arc closepath fill
       2 copy moveto
       2 copy  vpt 180 360 arc closepath fill
               vpt 0 360 arc closepath } bind def
/C14 { BL [] 0 setdash 2 copy moveto
       2 copy  vpt 90 360 arc closepath fill
               vpt 0 360 arc } bind def
/C15 { BL [] 0 setdash 2 copy vpt 0 360 arc closepath fill
               vpt 0 360 arc closepath } bind def
/Rec   { newpath 4 2 roll moveto 1 index 0 rlineto 0 exch rlineto
       neg 0 rlineto closepath } bind def
/Square { dup Rec } bind def
/Bsquare { vpt sub exch vpt sub exch vpt2 Square } bind def
/S0 { BL [] 0 setdash 2 copy moveto 0 vpt rlineto BL Bsquare } bind def
/S1 { BL [] 0 setdash 2 copy vpt Square fill Bsquare } bind def
/S2 { BL [] 0 setdash 2 copy exch vpt sub exch vpt Square fill Bsquare } bind def
/S3 { BL [] 0 setdash 2 copy exch vpt sub exch vpt2 vpt Rec fill Bsquare } bind def
/S4 { BL [] 0 setdash 2 copy exch vpt sub exch vpt sub vpt Square fill Bsquare } bind def
/S5 { BL [] 0 setdash 2 copy 2 copy vpt Square fill
       exch vpt sub exch vpt sub vpt Square fill Bsquare } bind def
/S6 { BL [] 0 setdash 2 copy exch vpt sub exch vpt sub vpt vpt2 Rec fill Bsquare } bind def
/S7 { BL [] 0 setdash 2 copy exch vpt sub exch vpt sub vpt vpt2 Rec fill
       2 copy vpt Square fill
       Bsquare } bind def
/S8 { BL [] 0 setdash 2 copy vpt sub vpt Square fill Bsquare } bind def
/S9 { BL [] 0 setdash 2 copy vpt sub vpt vpt2 Rec fill Bsquare } bind def
/S10 { BL [] 0 setdash 2 copy vpt sub vpt Square fill 2 copy exch vpt sub exch vpt Square fill
       Bsquare } bind def
/S11 { BL [] 0 setdash 2 copy vpt sub vpt Square fill 2 copy exch vpt sub exch vpt2 vpt Rec fill
       Bsquare } bind def
/S12 { BL [] 0 setdash 2 copy exch vpt sub exch vpt sub vpt2 vpt Rec fill Bsquare } bind def
/S13 { BL [] 0 setdash 2 copy exch vpt sub exch vpt sub vpt2 vpt Rec fill
       2 copy vpt Square fill Bsquare } bind def
/S14 { BL [] 0 setdash 2 copy exch vpt sub exch vpt sub vpt2 vpt Rec fill
       2 copy exch vpt sub exch vpt Square fill Bsquare } bind def
/S15 { BL [] 0 setdash 2 copy Bsquare fill Bsquare } bind def
/D0 { gsave translate 45 rotate 0 0 S0 stroke grestore } bind def
/D1 { gsave translate 45 rotate 0 0 S1 stroke grestore } bind def
/D2 { gsave translate 45 rotate 0 0 S2 stroke grestore } bind def
/D3 { gsave translate 45 rotate 0 0 S3 stroke grestore } bind def
/D4 { gsave translate 45 rotate 0 0 S4 stroke grestore } bind def
/D5 { gsave translate 45 rotate 0 0 S5 stroke grestore } bind def
/D6 { gsave translate 45 rotate 0 0 S6 stroke grestore } bind def
/D7 { gsave translate 45 rotate 0 0 S7 stroke grestore } bind def
/D8 { gsave translate 45 rotate 0 0 S8 stroke grestore } bind def
/D9 { gsave translate 45 rotate 0 0 S9 stroke grestore } bind def
/D10 { gsave translate 45 rotate 0 0 S10 stroke grestore } bind def
/D11 { gsave translate 45 rotate 0 0 S11 stroke grestore } bind def
/D12 { gsave translate 45 rotate 0 0 S12 stroke grestore } bind def
/D13 { gsave translate 45 rotate 0 0 S13 stroke grestore } bind def
/D14 { gsave translate 45 rotate 0 0 S14 stroke grestore } bind def
/D15 { gsave translate 45 rotate 0 0 S15 stroke grestore } bind def
/DiaE { stroke [] 0 setdash vpt add M
  hpt neg vpt neg V hpt vpt neg V
  hpt vpt V hpt neg vpt V closepath stroke } def
/BoxE { stroke [] 0 setdash exch hpt sub exch vpt add M
  0 vpt2 neg V hpt2 0 V 0 vpt2 V
  hpt2 neg 0 V closepath stroke } def
/TriUE { stroke [] 0 setdash vpt 1.12 mul add M
  hpt neg vpt -1.62 mul V
  hpt 2 mul 0 V
  hpt neg vpt 1.62 mul V closepath stroke } def
/TriDE { stroke [] 0 setdash vpt 1.12 mul sub M
  hpt neg vpt 1.62 mul V
  hpt 2 mul 0 V
  hpt neg vpt -1.62 mul V closepath stroke } def
/PentE { stroke [] 0 setdash gsave
  translate 0 hpt M 4 {72 rotate 0 hpt L} repeat
  closepath stroke grestore } def
/CircE { stroke [] 0 setdash 
  hpt 0 360 arc stroke } def
/Opaque { gsave closepath 1 setgray fill grestore 0 setgray closepath } def
/DiaW { stroke [] 0 setdash vpt add M
  hpt neg vpt neg V hpt vpt neg V
  hpt vpt V hpt neg vpt V Opaque stroke } def
/BoxW { stroke [] 0 setdash exch hpt sub exch vpt add M
  0 vpt2 neg V hpt2 0 V 0 vpt2 V
  hpt2 neg 0 V Opaque stroke } def
/TriUW { stroke [] 0 setdash vpt 1.12 mul add M
  hpt neg vpt -1.62 mul V
  hpt 2 mul 0 V
  hpt neg vpt 1.62 mul V Opaque stroke } def
/TriDW { stroke [] 0 setdash vpt 1.12 mul sub M
  hpt neg vpt 1.62 mul V
  hpt 2 mul 0 V
  hpt neg vpt -1.62 mul V Opaque stroke } def
/PentW { stroke [] 0 setdash gsave
  translate 0 hpt M 4 {72 rotate 0 hpt L} repeat
  Opaque stroke grestore } def
/CircW { stroke [] 0 setdash 
  hpt 0 360 arc Opaque stroke } def
/BoxFill { gsave Rec 1 setgray fill grestore } def
end
%%EndProlog
}}%
\begin{picture}(3600,2160)(0,0)%
{\GNUPLOTspecial{"
gnudict begin
gsave
0 0 translate
0.100 0.100 scale
0 setgray
newpath
1.000 UL
LTb
450 300 M
63 0 V
2937 0 R
-63 0 V
450 476 M
63 0 V
2937 0 R
-63 0 V
450 652 M
63 0 V
2937 0 R
-63 0 V
450 828 M
63 0 V
2937 0 R
-63 0 V
450 1004 M
63 0 V
2937 0 R
-63 0 V
450 1180 M
63 0 V
2937 0 R
-63 0 V
450 1356 M
63 0 V
2937 0 R
-63 0 V
450 1532 M
63 0 V
2937 0 R
-63 0 V
450 1708 M
63 0 V
2937 0 R
-63 0 V
450 1884 M
63 0 V
2937 0 R
-63 0 V
450 2060 M
63 0 V
2937 0 R
-63 0 V
450 300 M
0 63 V
0 1697 R
0 -63 V
1200 300 M
0 63 V
0 1697 R
0 -63 V
1950 300 M
0 63 V
0 1697 R
0 -63 V
2700 300 M
0 63 V
0 1697 R
0 -63 V
3450 300 M
0 63 V
0 1697 R
0 -63 V
1.000 UL
LTb
450 300 M
3000 0 V
0 1760 V
-3000 0 V
450 300 L
1.000 UL
LT0
3087 1947 M
263 0 V
450 652 M
30 -8 V
31 22 V
30 -28 V
30 24 V
31 -12 V
30 -5 V
30 21 V
30 -31 V
31 30 V
30 -17 V
30 -2 V
31 21 V
30 -34 V
30 35 V
31 -23 V
30 2 V
30 21 V
30 -38 V
31 42 V
30 -31 V
30 9 V
31 18 V
30 -41 V
30 51 V
31 -43 V
30 18 V
30 16 V
30 -47 V
31 63 V
30 -58 V
30 31 V
31 11 V
30 -53 V
30 81 V
31 -82 V
30 51 V
30 3 V
31 -65 V
30 114 V
30 -128 V
30 95 V
31 -16 V
30 -93 V
30 200 V
31 -264 V
30 242 V
30 -84 V
31 -304 V
30 1610 V
30 0 V
1995 366 L
31 304 V
30 84 V
30 -242 V
31 264 V
30 -200 V
30 93 V
31 16 V
30 -95 V
30 128 V
30 -114 V
31 65 V
30 -3 V
30 -51 V
31 82 V
30 -81 V
30 53 V
31 -11 V
30 -31 V
30 58 V
31 -63 V
30 47 V
30 -16 V
30 -18 V
31 43 V
30 -51 V
30 41 V
31 -18 V
30 -9 V
30 31 V
31 -42 V
30 38 V
30 -21 V
30 -2 V
31 23 V
30 -35 V
30 34 V
31 -21 V
30 2 V
30 17 V
31 -30 V
30 31 V
30 -21 V
30 5 V
31 12 V
30 -24 V
30 28 V
31 -22 V
30 8 V
stroke
grestore
end
showpage
}}%
\put(3037,1947){\makebox(0,0)[r]{$\psi$ for $\Delta=100$}}%
\put(1950,50){\makebox(0,0){$x$}}%
\put(100,1180){%
\special{ps: gsave currentpoint currentpoint translate
270 rotate neg exch neg exch translate}%
\makebox(0,0)[b]{\shortstack{$\psi(x)$}}%
\special{ps: currentpoint grestore moveto}%
}%
\put(3450,200){\makebox(0,0){10}}%
\put(2700,200){\makebox(0,0){5}}%
\put(1950,200){\makebox(0,0){0}}%
\put(1200,200){\makebox(0,0){-5}}%
\put(450,200){\makebox(0,0){-10}}%
\put(400,2060){\makebox(0,0)[r]{1.6}}%
\put(400,1884){\makebox(0,0)[r]{1.4}}%
\put(400,1708){\makebox(0,0)[r]{1.2}}%
\put(400,1532){\makebox(0,0)[r]{1}}%
\put(400,1356){\makebox(0,0)[r]{0.8}}%
\put(400,1180){\makebox(0,0)[r]{0.6}}%
\put(400,1004){\makebox(0,0)[r]{0.4}}%
\put(400,828){\makebox(0,0)[r]{0.2}}%
\put(400,652){\makebox(0,0)[r]{0}}%
\put(400,476){\makebox(0,0)[r]{-0.2}}%
\put(400,300){\makebox(0,0)[r]{-0.4}}%
\end{picture}%
\endgroup
\endinput

\caption{Eksempel p\aa\ b\o lgefunksjon $\psi$ for ulike verdier av
$\Delta$.} 
\end{figure}

Vi legger merke til f\o lgende:
\begin{itemize}
\item Dersom vi gj\o r $\Delta$ liten, som betyr at vi bestemmer 
$k$ skarpt n\ae r $k_0$, resulterer dette i en b\o lgefunksjon 
$\psi(x) \propto cos(2\pi k_0x)$. Dette vil svare til v\aa r harmoniske
svingning i Figur \ref{fig:fasev} og en b\o lge av uendelig utstrekning. Integrerer 
vi over f\aa\ b\o lgetall, f\aa r vi alts\aa\ en b\o lge sentrert rundt
kun et b\o lgetall. Vi kan si at sinus-delen svarer til den  modulerte
b\o lgefunskjonen, mens cosinus delen representer de enkelte b\o lgene som 
settes sammen.  
\item \O nsker vi derimot en b\o lge som er sterkt lokalisert i rom og null
eller liten for de fleste verdier av $x$, m\aa\ vi \o ke intervallet over
$k$-verdier i integrasjonen v\aa r. 
\item Det sistnevnte gir oss den n\o dvendige koplingen til v\aa r diskusjon
om gruppehastighet og fasehastighet og en b\o lgepakkes utbredelse.
\item Dersom vi n\aa\ \o nsker \aa\ kople dette resultatet med fysikk
      har vi at siden
      \[ 
         k=\frac{2\pi}{\lambda}=\frac{h2\pi}{h\lambda}=\frac{p2\pi}{h}
      \]
eller 
       \[
          \hbar k=p
       \]
s\aa\ betyr det at dersom vi \o nsker \aa\ fiksere bevegelsesmengden skarpt,
dvs.~$\Delta \rightarrow 0$, resulterer det i en b\o lgefunksjon av uendelig
utbredelse. \O nsker vi at denne b\o lgefunksjonen skal representere en
partikkel, betyr det igjen at vi ikke kan fiksere posisjon og bevegelsesmengde
skarpt samtidig. Tilsvarende, \o nsker vi \aa\ fiksere posisjonen skarpt,
finner vi at uskarpheten til partikkelen \o ker.
\end{itemize}
 
Det siste leder oss til neste avsnitt om Heisenbergs 
      uskarphetsrelasjon.

\section{Heisenbergs uskarphetsrelasjon}

La oss f\o rst anvende\footnote{Lesehenvisning her er kapitlene 
4-3, 4-4 og 4-5, sidene 188-205. To sm\aa\ oppfordringer: 
den f\o rste med et snev av moralisme i seg. Jeg tar meg nemlig den frihet 
\aa\ anbefale sterkt at dere
leser disse avsnittene i boka, da de danner et viktig grunnlag for bruddet
med klassisk fysikk og introduksjonen av kvantemekanikken som teori. 
Vi kommer ikke til \aa\ g\aa\ noe i dybden p\aa\ de filosofiske aspektene
i disse notatene. Derfor den andre oppfordringen. Jeg setter pris p\aa\
dersom dere tar disse emnene opp til diskusjon enten under forelesningene
eller under oppgavel\o sning.} 
resultatet fra avsnittet om diffraksjon
p\aa\ en innkommende b\o lge av elektroner med b\o lgelengde $\lambda$
mot en spalte\aa pning med st\o rrelse $d=\Delta y$.
Et eksempel p\aa\ et slikt oppsett er vist i Figur \ref{fig:diffex}.

N\aa r monokromatiske b\o lger med b\o lgelengde
$\lambda$ passerer \aa pningen vil det dannes et diffraksjonsm\o nster p\aa\
skjermen. 
Det innfallende elektron har bevegelsesmengde kun i $x$-retningen f\o r det
treffer spalte\aa pningen. 
F\o rste punkt hvor vi har destruktiv interferens er gitt ved  
\begin{equation}
   sin\theta=\frac{\lambda}{d}=\frac{\lambda}{\Delta y}=\frac{h}{p\Delta y},
\end{equation}
siden $\lambda=h/p$.  
Etter at elektronet har passert \aa pningen vil det ogs\aa\ ha en bevegelsemengde i 
$y$-retningen som vi ikke kjenner.
Vi veit ikke med n\o yaktighet hvor elektronet  vil treffe skjermen, det eneste
vi kan si er at det er en stor sannsynlighet for at det treffer i et omr\aa de 
hvor intensiteten er maksimal. { \bf Her har vi innf\o rt et sannsynlighetsbegrep
som vi skal diskutere n\ae rmere etter dette eksemplet.}

Vi kan anta at $y$-komponenten til bevegelsesmengden har en verdi mellom
$0$ og $psin\theta$, dvs. 
\begin{equation}
   \Delta p_y=psin\theta=\frac{h\lambda}{\lambda \Delta y}=\frac{h}{\Delta y},
\end{equation}
som gir en uskarphet 
\begin{equation}
    \Delta p_y\Delta y=h,
\end{equation}
som p\aa\ en konstant n\ae r ($h=2\pi\hbar$) er gitt ved Heisenbergs uskaperhetsrelasjon
\begin{equation}
    \Delta p_y\Delta y \geq \frac{\hbar}{2},
    \label{eq:heisenberg}
\end{equation}
eller mer generelt
\begin{equation}
    \Delta {\bf p}\Delta {\bf x} \geq \frac{\hbar}{2},
    \label{eq:heisenberg}
\end{equation}
Dette uttrykket skal vi utlede senere, under kapitlet om
kvantemekanikkens formelle grunnlag.

Det vi skal merke oss her er at dette resultatet som kun er basert p\aa\ 
b\o lgel\ae re forteller oss at dersom vi \o nsker \aa\ lokalisere
skarpt p\aa\ skjermen hvor elektronet treffer, dvs.~at dersom vi setter 
$\Delta y=0$ s\aa\ vil
$\Delta p_y$ divergere. Vi kan alts\aa\ ikke fiksere skarpt b\aa de posisjon
og bevegelsesmengde. Vi merker oss dog at dersom $h=0$, ville vi ikke hatt noe slikt
problem.   
\begin{figure}[h]
   \setlength{\unitlength}{1mm}
   \begin{picture}(100,60)
   \put(25,0){\epsfxsize=10cm \epsfbox{fig3.eps}}
   \end{picture}
\caption{Diffraksjon for materieb\o lge med b\o lgelengde $\lambda$  som sendes mot en spalte\aa pning. \label{fig:diffex}} 
\end{figure}
Klassisk er det slik at ${\bf x}$ og ${\bf p}$ er uavhengige st\o rrelser.
Kvantemekanisk derimot, siden $h\ne 0$, s\aa\ er avhengighets forholdet gitt ved
Heisenbergs uskarphets relasjon. Dette impliserer igjen at vi ikke kan 
lokalisere en partikkel og samtidig bestemme dens bevegelsesmengde skarpt. 
En ytterligere konsekvens er at vi ikke kan i et bestemt eksperiment observere
b\aa de partikkel og b\o lgeegenskaper. Vi skal se et eksempel p\aa\ dette
i diskusjonen lenger nede, men f\o rst noen ord om partikkel-b\o lge dualiteten.

\subsection{Partikkel-B\o lge dualitet}
\label{subsec:pbdualitet}
I det siste eksemplet innf\o rte vi at diffraksjonsm\o nsteret som vi ser
p\aa\ skjermen skal representer en sannsynlighet for at 
elektronet er \aa\ finne et bestemt sted. 
Vi skal n\aa\ anskueliggj\o re dette ved \aa\ se p\aa\ uttrykket
for intensiteten for det elektromagnetiske feltet.

Klassisk, dvs.~fra e.m.~teori og dermed FY101, har vi at
b\o lgefunksjonen til f.eks.~det elektriske feltet er gitt ved
\begin{equation}
   {\cal E}(x,t)={\cal E}_0sin(kx-\omega t),
\end{equation}
og at  
intensiteten til det e.m.~feltet, dvs.~energi som kommer inn per areal
per sekund, er gitt ved 
\be
    I=c\epsilon_0{\cal E}_0^2,
\end{equation}
hvor ${\cal E}_0$ er amplituden til feltet. 
Et eksempel p\aa\ uttrykk for $I$ er gitt ved likning (\ref{eq:diffint}).
Vi kan si at intensiteten er proporsjonal med b\o lgefunskjonen
kvadrert, dvs.
\begin{equation}
   I\propto  {\cal E}^2.
\end{equation}

Dette er et resultat som baserer seg kun p\aa\ klassisk b\o lgel\ae re,
dvs.~et reint b\o lgebilde av e.m.~str\aa ling. 

Dersom vi tar utgangspunkt i Einsteins og Plancks kvantseringshypoteser for
det e.m.~felt, s\aa\ tilordner vi partikkel egenskaper til intensiteten.
I dette tilfellet har vi at 
\begin{equation}
   I=h\nu N_{\gamma},
\end{equation}
hvor $h\nu$ er det velkjente uttrykket for energien til et foton 
mens $N_{\gamma}$ er midlere antall fotoner per areal per sekund som kommer inn
i et omr\aa de. 
Denne st\o rrelsen er igjen et uttrykk for 
en sannsynlighet for \aa\ finne et gitt antall 
fotoner i et bestemt omr\aa de. {\bf Det er dette begrepet vi n\aa\ skal 
kople til det klassiske resultatet for $I$.}    

Dersom vi n\aa\ g\aa r tilbake til diffraksjonsm\o nsteret i likning
(\ref{eq:diffint}) kan vi kople til  
interferensmaksima et begrep om at det er der flest fotoner treffer plata. 

Interferensmaksima er kopla til b\o lgeegenskapen til fotonene, men n\aa r vi
setter
\begin{equation}
   I=h\nu N_{\gamma}=c\epsilon_0{\cal E}_0^2,
\end{equation}
s\aa\ relaterer vi partikkelegenskaper til b\o lgeegenskaper og vi sier at
relasjonen
\begin{equation}
   N_{\gamma}\propto {\cal E}_0^2
\end{equation}
skal uttrykke et forhold mellom en intensitestfordeling fra b\o lgeoppf\o rsel
og en sannsynlighet for \aa\ detektere fotoner.

Mere generelt, n\aa r vi erstatter det elektriske felt med 
b\o lgefunskjonen $\psi$, s\aa\ skal st\o rrelsen 
\begin{equation}
    |\psi(x,t)|^2dx,
\end{equation}
uttrykke sannsynligheten for \aa\ finne en partikkel 
(hvis bevegelseslikninger er
beskrevet av en b\o lge $\psi$) i et omr\aa de mellom
$x$ og $x+dx$. 
Det er denne tolkningen som danner basis for kvantemekanikken. 



\subsubsection{To spalte\aa pninger}


Anta at vi har to spalte\aa pninger som vi sender elektroner mot.
Avstanden mellom spalte\aa pningene kaller vi igjen for $a$.
Intensiteten ved skjermen var da gitt ved likning (\ref{eq:diffint}), som
igjen fortalte oss  
at vi kan ha  konstruktiv interferens n\aa r betingelsen
\begin{equation}
    sin\theta_n=n\frac{\lambda}{a},
\end{equation}
er oppfylt med gitt
b\o lgelengden $\lambda$. Dersom vi kaller
avstanden fra spalte\aa pningene til skjermen hvor vi observerer et
diffraksjonsm\o nster for $b$, s\aa\ har vi at avstanden mellom
to interferensmaksima er gitt ved
\begin{equation}
    b(sin\theta_n-sin\theta_{n-1})=\frac{b\lambda}{a}.
    \label{eq:intensmaks}
\end{equation} 
La oss anta at vi krever at vi er istand til \aa\ m\aa le elektronets
posisjon med en presisjon som er mindre en halvparten av avstanden mellom
spaltene, dvs.\
\begin{equation}
    \Delta y < \frac{a}{2}.
\end{equation}
Vi kan f.eks.~anta at vi har en eller annen monitor rett ved skjermen 
som forteller oss hvilken
spalte\aa pning elektronet passerte. Ved bruk av en slik monitor tilf\o res
elektronet en bevegelsesmengde i $y$-retningen (parallellt med skjermen) 
hvis st\o rrelse er uskarp og gitt ved
\begin{equation}
    \Delta p_y > \frac{2h}{a}.
\end{equation}
En slik uskarphet i bevegelsesmengde resulterer i en relativ uskarphet
\begin{equation}
    \frac{\Delta p_y}{p}>\frac{2h}{ap}=\frac{2\lambda}{a}.
\end{equation}
En slik uskarphet medf\o rer en uskarphet i elektronets posisjon p\aa\
skjermen som ihvertfall er gitt ved 
\begin{equation}
   \frac{2\lambda b}{a},
\end{equation} 
som er st\o rre enn avstanden mellom interferensmaksima i likning
(\ref{eq:intensmaks}).
Det betyr igjen at dersom vi \o nsker \aa\ m\aa le elektronets posisjon presist,
vil vi komme til \aa\ \o delegge interferensm\o nsteret. 
Vi kan ikke 
observere  fra et enkeltst\aa ende eksperiment b\aa de b\o lge og partikkelegenskapene til materie,
eller e.m.~str\aa ling for den saks skyld.
B\o lge og partikkelegenskapene er alts\aa\ 
to komplent\ae re egenskaper ved materien. 

Dersom vi \o nsker \aa\ formulere matematisk konsekvensene av det siste
eksemplet s\aa\ m\aa\ vi ha 
\begin{equation}
   \Delta p_y\Delta y > h.
\end{equation}



\subsection{St\o rrelsesorden estimater, leik med uskarphetsrelasjonen}

Hensikten med dette avsnittet er \aa\ vise
at n\aa r vi kjenner til formen for energien et system har, s\aa\ kan vi bruke
uskarphetsrelasjonen til \aa\ estimere st\o rrelser slik som
bindingsenergien til et elektron i et hydrogenatom, eller hvor stor
massen til en n\o ytronstjerne er!

 
\subsubsection{``Realiteten'' til Bohrs orbitaler}

Vi s\aa\ i forrige kapittel at Bohrs atommodell ga oss
en definisjon p\aa\ Bohrradiene gitt ved
\begin{equation}
    r_n=\frac{\hbar n^2}{\alpha m_ec},
\end{equation}
hvor $\alpha$ er finstrukturkonstanten. 
Anta at vi er interessert i \aa\ m\aa le hvor elektronet
befinner seg i et atom. Vi krever at n\o yaktigheten er slik at 
\begin{equation}
   \Delta x << r_n-r_{n-1}=\frac{\hbar( 2n-1)}{\alpha m_ec}, 
\end{equation}
for at vi skal kunne fastsl\aa\ med rimelig n\o yaktighet
at et elektron er i en  bestemt orbital med kvantetall $n$.  
Men dette kravet leder til en uskarphet i bevegelsesmengde\footnote{Her bruker vi det eksakte uttrykket for 
Heisenbergs uskarphetsrelasjon gitt i likning
(\ref{eq:heisenberg}).} 
 gitt ved
\begin{equation}
   \Delta p >> \frac{\alpha m_ec}{2(2n-1)}.
\end{equation}
Dersom vi bruker resultatet fra Bohrs atommodell for hastigheten til et 
elektron i en gitt bane, $v_n$, har vi en bevegelsesmengde
\begin{equation}
    p=m_ev_n=\frac{\alpha cm_e}{n}
\end{equation}
som er p\aa\ st\o rrelse med $\Delta p$!
Vi kan s\aa\ pr\o ve \aa\ regne
ut uskarpheten i energien til elektronet. Da trenger vi det ikke-relativistiske
uttrykket for $E=p^2/2m_e$, som gir at
\begin{equation}
     \Delta E=\frac{p\Delta p}{m_e},
\end{equation}
som gir 
\begin{equation}
    \Delta E =\frac{p\Delta p}{m_e} >> \frac{1}{2}\frac{\alpha^2 m_ec^2}{n^2(2n-1)}
    =13.6\frac{1}{2n^2-1},
\end{equation}
som betyr at uskarpheten i energien er mye st\o rre enn bindingsenergien
til elektronet.  En slik m\aa ling av elektronet vil sparke 
elektronet ut av sin orbital, dvs.~vi er ikke i stand til \aa\ m\aa le
hvor det befinner seg i atomet.

\subsubsection{Bindingsenergien til elektronet i hydrogenatomet}

La oss anta at elektronet er i sin grunntilstand i hydrogenatomet.
Vi antar ogs\aa\, basert p\aa\ Heisenbergs uskarphetsrelasjon at vi kan
sette 
\begin{equation}
    p\sim \frac{\hbar}{r}.
    \label{eq:happrox}
\end{equation}
Energien til dette elektronet vil v\ae re gitt av summen av kinetisk
og potensiell energi, dvs.
\begin{equation}
   E=\frac{p^2}{2m_e}-\frac{e^2}{4\pi\epsilon_0r}.
\end{equation}
Setter vi inn uttrykket for $p$ finner vi
\begin{equation}
   E\sim
\frac{\hbar^2}{2r^2m_e}-\frac{e^2}{4\pi\epsilon_0r}.
\end{equation}
N\aa\ kan vi betrakte $r$ som en variabel som kan varieres. Dersom vi
\o nsker \aa\ finne energiens minimum, trenger vi kun \aa\ finne radien
n\aa  r
\begin{equation}
   \frac{dE}{dr}=0.
\end{equation}
Dette gir 
\begin{equation}
    r=\frac{\hbar^24\pi\epsilon_0}{m_ee^2},
\end{equation}
og setter vi $r$ i uttrykket for energien har vi
\begin{equation}
    E\sim -\frac{1}{2}\frac{m_ee^4}{(\hbar4\pi\epsilon_0)^2}=-13.6 
    \hspace{0.1cm}\mathrm{eV},
\end{equation}
akkurat den eksperimentelle verdien!

Det kan da ikke stemme at en s\aa\  enkel antagelse som den gitt i
likning (\ref{eq:happrox}) skal kunne gi oss den eksperimentelle
energien til hydrogenatomet. 
Selv om vi var heldige ved valget, ser vi at dersom vi hadde valgt en annen
verdi for forholdet mellom $p$, $h$ og $r$, s\aa\ ville energien avveket bare
p\aa\ en konstant n\ae r. Dvs.~at vi f\aa r ut en energi som har riktig 
st\o rrelsesorden. N\aa r vi kommer til Kap 7 i boka, skal vi faktisk
regne ut $\Delta p$ og $\Delta r$ for hydrogenatomet og se at den
antagelsen vi gjorde faktisk er fornuftig. 
Men, det viktige her er at uskarphetsrelasjonen gir oss en energi som har riktig
st\o rrelsesorden.

Vi skal ikke slippe dette eksemplet helt, fordi det er noen sider ved
uskarphetsrelasjonen og bindingsenergien til et system som er av betydning
for den videre forst\aa else. 
Det f\o rste dere skal bite dere merke i er at uskarphetsrelasjonen begrenser
nedad bindingsenergien, den kan ikke bli uendelig stor i absoluttverdi. 
Det andre er at dersom vi tenker tilbake p\aa\ Bohrs tredje
postulat og ser for oss muligheten for at elektronet skulle deaksellereres og miste
all sin energi, s\aa\ vil det bety at det til slutt ville klappe sammen p\aa\
atomkjernen. N\aa r avstanden til kjernen minsker, kan vi ogs\aa\ anta at
$\Delta x$ minsker. 
Men det betyr igjen at $\Delta p$ \o ker slik at \o kningen i potensiell
energi kompenseres ved en \o kning i kinetisk energi. 
 
\subsubsection{Massen til en n\o ytronstjerne}
N\aa\ har vi f\aa tt blod p\aa\ tann. Vi
kaster  oss derfor 
friskt og freidig 
ut i verdensrommet og p\aa st\aa r at vi kan estimere massen til en
n\o ytronstjerne. Ei n\o ytronstjerne best\aa r i all hovedsak av n\o ytroner,
samt en liten innmiksing av protoner, elektroner, og andre baryoner, og kanskje
muligens en fase av kvarkmaterie i sitt indre, hvor tettheten er ekstremt h\o y.
Her skal vi anta at den best\aa r av bare n\o ytroner.
Radiusen til ei slik stjerne kaller vi
for $R$.  Vi bruker den gjennomsnittlige massen til protoner og n\o ytroner,
og setter massen til n\o ytronet $m_n=938$ MeV/c$^2$.
Vi antar at vi har $N$ n\o ytroner i stjerna, dermed er
antallstettheten $n$ gitt ved
\begin{equation}
   n\sim \frac{N}{R^3},
\end{equation}
noe som gir et volum per n\o ytron ved $1/n$. Radien til det volumelementet et enkelt
n\o ytron da opptar er tiln\ae rmingsvis gitt 
ved $r\sim 1/n^{1/3}$. Med en slik radius kan vi n\aa\ bruke
Heisenbergs uskarphets relasjon p\aa\ nytt
\begin{equation}
    p\sim \frac{\hbar}{r}\sim \hbar n^{1/3}.
\end{equation}
La oss s\aa\ anta at n\o ytronene  v\aa re er ekstremt relativistiske
og at de ikke vekselvirker. Da er den kinetiske energien
til et n\o ytron gitt ved 
\begin{equation}
    E_{n}
=\sqrt{p^2c^2+m_n^2c^4}\approx pc\sim \hbar n^{1/3}c\sim 
\hbar c\frac{N^{1/3}}{R}.
\end{equation}
Den gravitasjonelle energien per n\o ytron, 
n\aa r den totale massen til stjerna er gitt ved
\begin{equation}
   M=Nm_n,
\end{equation}
er gitt ved 
\begin{equation}
    E_G\sim -\frac{GMm_n}{R},
\end{equation}
hvor $G$ er gravitasjonskonstanten.
Energien m\aa\ da v\ae re gitt ved energien til n\o ytronene og den 
gravitasjonelle energien. Vi har da
\begin{equation}
    E=E_n+E_G=\hbar c\frac{N^{1/3}}{R}-\frac{GNm_n^2}{R}.
\end{equation}
Likevekt oppn\aa s n\aa r energi er et minimum, noe som forteller
om balansen mellom gravitasjonskreftene og den kinetiske energien til
n\o ytronene. Deriverer vi $E$ mhp.~$R$ finner vi 
det totale antall partikler
ved likevekt er gitt ved
\begin{equation}
    N_{likevekt}\sim \left(\frac{\hbar c}{Gm_n^2}\right)^{3/2}\sim
    2\times 10^{57}.
\end{equation}
Massen ved likevekt er da gitt ved
\begin{equation}
   M_{likevekt}\sim N_{likevekt}m_n =2\times 10^{57}\dot 1.67 \times 10^{-27}
  \hspace{0.1cm}\mathrm{kg}=3.3 \times 10^{30}\hspace{0.1cm}\mathrm{kg}.
\end{equation}
Massen til sola er $M_{\odot}=1.98 \times 10^{30}$ kg.
Dvs.~at vi finner at massen til v\aa r n\o ytronstjerne er
ca.~$1.66 M_{\odot}$. Massen til n\o ytronstjerner m\aa lt i bin\ae re systemer\footnote{Nobelpris i Fysikk 1993 til Hulse og Taylor.} ligger p\aa\
ca.~$1.4 M_{\odot}$. Ikke verst hva?

\subsubsection{Estimat for kjernekreftenes st\o rrelse}

For elektroner, og dermed for systemer som behandles i atomfysikk, faste  stoffers fysikk
og molekylfysikk, opererer vi med en energi skala p\aa\ noen f\aa\ til kanskje
noen tusen eV. Dette gir igjen lengdeskalaer p\aa\ noen nanometer.

For kjernekreftene,
dvs de kreftene som uttrykker vekselvirkningene mellom f.eks~protoner
og n\o ytroner,  veit vi at rekkevidden til kreftene er bare
noen f\aa\ femtometer, 1 fm = $10^{-15}$ m. Dette impliserer en ny energiskala
som er forskjellig fra den vi kjenner til i Bohrs atommodell. 
For \aa\ se dette kan vi igjen bruke uskarphetrelasjonen p\aa\ forma
\[
   p\sim \frac{\hbar}{r}.
\]
Setter vi $r\sim 1$ fm, f\aa r vi at den kinetiske energien som et proton eller
et n\o ytron kan ha i en atomkjerne blir
\begin{equation}
   \frac{p^2}{2m_p}\sim\frac{\hbar^2}{2m_pr^2}.
\end{equation}
Setter vi inn $m_p=938$ MeV/c$^2$ og at $\hbar c=197$ MeVfm, finner vi
\begin{equation}
   \frac{\hbar^2}{2m_pr^2}=\frac{197^2}{2\times 938}\hspace{0.1cm}\mathrm{MeV}=
                          21\hspace{0.1cm}\mathrm{MeV}.
\end{equation}
Det f\o rste vi merker oss er at energien m\aa les n\aa\ i MeV, en million eV.
Av denne grunn kalles ogs\aa\ kjernekreftene for de sterke krefter. 
N\aa\ er det slik, akkurat som for elektroner i atomer, at protoner og
n\o ytroner er bundet sammen i en atomkjerne. Dvs.~at vi ogs\aa\ m\aa\
ta med en potensiell energi, som igjen m\aa\ v\ae re st\o rre (i absoluttverdi)
enn den kinetiske energien til protoner og n\o ytroner for at disse
partiklene skal holde sammen. Dvs.~at den potensielle energien b\o r v\ae re
st\o rre en 21 MeV.   

\subsection{Energi-tid uskarphetsrelasjonen}
Gitt 
\[
       \Delta p_x\Delta x \geq \frac{\hbar}{2},
\]
har vi n\aa r vi bruker  
\[
   \Delta E=\frac{p\Delta p}{m},
\]
og 
\[
   v=\frac{\Delta x}{\Delta t},
\]
sammen med $p=mv$ at
\be
    \frac{p\Delta p}{m}\frac{\Delta xm}{p}=\Delta E\frac{\Delta x}{v}
\end{equation}
som gir oss
\be
           \Delta E\Delta t \geq \frac{\hbar}{2},
\end{equation}
som er den tilsvarende 
energi-tid uskarphetsrelasjonen.


\section{Oppgaver}
\subsection{Analytiske oppgaver}
\subsubsection*{Oppgave 3.1}
I et elektronmikroskop f�r elektronene stor kinetisk energi ved
at de akselereres i et elektrostatisk potensial V.
%
\begin{itemize}
%
\item[a)] Vis at de Broglie b\o lgelengden for et slikt
ikke--relativistisk elektron er gitt ved formelen
\[
\lambda = \frac{1,23}{\sqrt{V}} \; \mbox{nm}
\]
n\aa r V m\aa les i volt.
%
\item[b)] Hva blir den tilsvarende formelen hvis elektronet er
relativistisk?
%
\item[c)] For hvilken spenning V gir den ikke--relativistiske
formelen et svar som er 5 \% feil?
%
\end{itemize}

\subsubsection*{Oppgave 3.2}
En partikkel med ladning $e$ og masse $m_0$
akselereres av en elektrisk potensial $V$
til en relativistisk hastighet.
%
\begin{itemize}
%
\item[a)]Vis at de Broglie b{\o}lgelengden for
partikkelen er gitt ved
\[
\lambda = \frac{h}{\sqrt{2 m_0 e V}}
\left ( 1 + \frac{e V}{2 m_0 c^2}\right )^{-1/2}
\]
%
\item[b)] Vis at dette gir $\lambda = h / p$ i den
ikke--relativiske grense.
%
\item[c)] Vis at for en relativistisk partikkel med hvileenergi
$E_0$ er de Broglie b{\o}lgelengden gitt ved
\[
\lambda = \frac{1,24 \times 10^{-2}}{E_0 (MeV)}
\cdot \frac{\sqrt{(1-\beta^2)}}{\beta}\;\;\mbox{�},
\]
hvor $\beta = v / c$.
\end{itemize}


\subsubsection*{Oppgave 3.3}
Fra optikken vet vi at for \aa ~observere sm\aa ~objekter med
lys, m\aa ~b\o lgelengden maksimalt v\ae re av samme
st\o rrelsesorden som objektets utstrekning.
%
\begin{itemize}
%
\item[a)] Hva er den laveste frekvensen p\aa ~lys som kan
benyttes til \aa~unders\o ke et objekt med radius
0.3 nm i et mikroskop?
\item[b)] Hva er den tilsvarende energien?
%
\item[c)] Hvis man i stedet ville bruke et elektronmikroskop, hva
er da den laveste energien elek\-tronene kan ha for at
partikkelen skal kunne studeres i detalj?
%
\item[d)] Er lys-- eller elektronmikroskop \aa ~foretrekke for
denne typen arbeid?
%
\end{itemize}

\subsubsection*{Oppgave 3.4}
Lyd med frekvens 440 Hz og hastighet 340 $\mbox{ms}^{-1}$ sendes
normalt mot en smal spalte i en vegg. Spalten har en slik bredde at
lydintensiteten har avtatt til det halve i en retning
p\aa ~$45^{\circ}$ fra innfallsvinkelen bak veggen. Hvor
bred er spalten?

\subsubsection*{Oppgave 3.5}
Et fysisk system beskrevet ved hjelp av b\o lgeligninger som
tillater $y = A cos( kx - \omega t)$ som l\o sninger,
der sirkelfrekvensen $\omega$ er en reell
funksjon av b\o lgetallet k, kalles for et line\ae rt, dispersivt system.
Funksjonen $\omega (k)$ kalles  for {\em dispersjonsrelasjonen} til
systemet.
%
\begin{itemize}
%
\item[a)] Vis at dispersjonsrelasjonen for frie, relativistiske
elektronb\o lger er gitt ved

\begin{eqnarray}
\omega (k) = c \cdot \sqrt{ k^{2} + \left( \frac{mc}{\hbar }\right)^{2}},
\nonumber
\end{eqnarray}
der m er elektronets hvilemasse.

\item[b)] Finn et uttrykk for fasehastigheten $v_{f} (k)$ og gruppehastigheten
$v_{g} (k)$ til disse b\o lgene, og vis at produktet $v_{f} (k) \cdot v_{g} (k)$
er en konstant (uavhengig av k).

\item[c)] Fra uttrykket for $v_{f}$ ser vi at $v_{f} > c$ ! Kommenter dette
fenomenet og hva det har \aa ~si for tolkningene av $v_{f}$ og $v_{g}$ ut fra
den spesielle relativitetsteorien.
%
\end{itemize}


\subsubsection*{Oppgave 3.6}
\begin{itemize}
%
\item[a)] En partikkel med masse 1 g har en uskarphet i hastigheten p\aa
~$\Delta v = 10^{-6}\; ms^{-1}$. Hva blir den kvantemekaniske uskarpheten i
posisjonen?

\item[b)] Et elektron med energi 10 keV er lokalisert i et omr\aa de med
utstrekning 0,1 nm. Hva blir uskarpheten i bevegelsesmengden
til elektronet?

\item[c)] Hva blir uskarpheten i elektronets energi?

\item[d)] Et proton i en atomkjerne kan bevege seg i et omr\aa de med en
utstrekning av st{\o}r\-rel\-ses\-or\-den $10^{-15}$ m.  Hvis vi antar at protonet er
"fanget" i et uendelig potensial, hva er den minste kinetiske energien det kan
ha? Ansl\aa ~en st\o rrelsesorden for styrken av potensialet hvis dette likevel kan
antas \aa ~v\ae re endelig.
%
\end{itemize}

\subsubsection*{Oppgave 3.7. Eksamen H-1995}
%
Den relativistiske sammenhengen mellom en partikkels bevegelsesmengde $p$
og energi $E$ er gitt ved 
%
\[
E = \sqrt{E_0^2 + (pc)^2},
\]
%
hvor c er lyshastigheten.
%
\begin{itemize}
%
\item[a)] Forklar hva $E_k = E - E_0$ st{\aa}r for. Hva er $E_0$ og $E_k$ for 
et foton? Finn sammenhengen mellom  b{\o}lgelengden $\lambda$ og 
bevegelsesmengden $p$ for et foton.
%
\item[b)] Gj{\o}r \underline{kort} rede for de Broglie's ideer om materieb{\o}lger
og sett opp uttrykket for de Broglie b{\o}lgelengden for en materiell partikkel 
med bevegelsesmengde  $p$.
%
\item[c)] Finn de Broglie b{\o}lgelengden for et elektron som funksjon av elektronets 
kinetiske energi. Bruk i dette tilfelle det ikke--relativistiske uttrykk for den 
kinetiske energien.
%
\end{itemize}
%
En mono--energetisk elektronstr{\aa}le blir sendt skr{\aa}tt inn mot overflaten av en 
\'{e}n--krystall, og spres i visse retninger fra krystallen. Under behandlingen av et 
slikt spredningsproblem gj{\o}r en bruk av den s{\aa}kalte Braggbetingelsen
%
\[
2 d \sin \theta = n \lambda
\]
%
\begin{itemize}
%
\item[d)] Forklar de st{\o}rrelsene som inng{\aa}r  og lag en 
enkel skisse av det eksperimentelle oppsettet.
%
\end{itemize}
%
\subsubsection*{Kort fasit}
\begin{itemize}
%
\item[a)] $E_k = E - E_0 = \sqrt{E_0^2 + (p c)^2} = $kinetisk energi.
For et foton er $E_0 = \mbox{hvileenergi} = 0,\; E_k = p c, 
          \; \lambda = h / p.$
%
\item[b)] Se l{\ae}reboka {\sl Brehm and Mullin: Introduction to the
structure of matter}, avsnitt 4.1,  $\lambda = h/p$.
%
\item[c)] $E_k = p^2 / 2m \longrightarrow p = \sqrt{2 m E_k}, 
\mbox{innsat i} \lambda = h / p$.
%
\item[d)] Se l{\ae}reboka {\sl Brehm and Mullin: Introduction to the
structure of matter}, avsnitt 2.6
\end{itemize}
